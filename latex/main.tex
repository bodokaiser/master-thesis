\documentclass[a4paper,appendixprefix]{scrreprt}

\usepackage[utf8]{inputenc}
\usepackage{amsthm,amsmath,amssymb}
\usepackage{authblk}
\usepackage[english]{babel}
\usepackage[backend=biber]{biblatex}
\usepackage{booktabs}
%\usepackage{csquotes}
\usepackage[acronym,nonumberlist,toc]{glossaries}
\usepackage{hyperref}
\usepackage{cleveref}
\usepackage{physics}
\usepackage[separate-uncertainty=true]{siunitx}
%\usepackage[subpreambles=true]{standalone}
\usepackage{thmtools,thm-restate}
%\usepackage{subfig}
\usepackage{xcolor}

\addbibresource{references/articles.bib}
\addbibresource{references/books.bib}
\addbibresource{references/thesis.bib}

% add bibliography as section (not chapter)
% https://tex.stackexchange.com/questions/568580/make-the-bibliography-as-a-section-in-each-included-chapter
\defbibheading{bibliography}[\bibname]{\section*{#1}}

% approximately proportional to symbol
% https://tex.stackexchange.com/questions/33538/how-to-get-an-approximately-proportional-to-symbol
\def\app#1#2{%
    \mathrel{%
        \setbox0=\hbox{$#1\sim$}%
        \setbox2=\hbox{%
            \rlap{\hbox{$#1\propto$}}%
            \lower1.1\ht0\box0%
        }%
        \raise0.25\ht2\box2%
    }%
}
\def\approxprop{\mathpalette\app\relax}

% overwrite real and imaginary part operators
\let\Re\undefined
\let\Im\undefined
\DeclareMathOperator{\Re}{\operatorname{Re}}
\DeclareMathOperator{\Im}{\operatorname{Im}}
% other functions
\DeclareMathOperator{\sinc}{\operatorname{sinc}}

\newcommand{\floor}[1]{\left\lfloor#1\right\rfloor}
\newcommand{\ceil}[1]{\left\lceil#1\right\rceil}

% transpose
% https://tex.stackexchange.com/questions/403104/small-caps-mathsf-font-for-writing-transpose-of-a-matrix
\newcommand{\trans}{{\scriptscriptstyle\mathsf{T}}}

% theorems
\newtheorem{theorem}{Theorem}[section]
\newtheorem{lemma}[theorem]{Lemma}
\newtheorem{corollary}[theorem]{Corollary}
\theoremstyle{definition}
\newtheorem{definition}{Definition}[section]
\newtheorem{conjecture}{Conjecture}[section]
\newtheorem{example}{Example}[section]
\theoremstyle{remark}
\newtheorem*{remark}{Remark}

% prefix equation numbers with section number
\numberwithin{equation}{section}

% optics
\newacronym{ar}{AR}{anti-reflective}
\newacronym{mzm}{MZM}{Mach-Zehnder modulator}
\newacronym{bs}{BS}{beam splitter}
\newacronym{fc}{FC}{fiber coupler}
\newacronym{qe}{QE}{quantum efficiency}

% physics
\newacronym{dv}{DV}{discrete-variable}
\newacronym{cv}{CV}{continuous-variable}
\newacronym{dof}{DOF}{degrees of freedom}
\newacronym{eom}{EOM}{equation(s) of motion}
\newacronym{pbc}{PBC}{periodic boundary conditions}
\newacronym{bch}{BCH}{Baker-Campbell-Hausdorff}
\newacronym{ccr}{CCR}{canonical commutation relation}

% signal processing
\newacronym{dsp}{DSP}{digital signal processing}
\newacronym{lo}{LO}{local oscillator}
\newacronym{if}{IF}{intermediate frequency}
\newacronym{lp}{LP}{low-pass}
\newacronym{adc}{ADC}{analog-to-digital converter}
\newacronym{dac}{DAC}{digital-to-analog converter}
\newacronym{qam}{QAM}{quadrature amplitude modulation}
\newacronym{qpsk}{QPSK}{quadrature phase-shift keying}

% quantum-key distribution
\newacronym{qkd}{QKD}{quantum-key distribution}
\newacronym{dvqkd}{DV-QKD}{discrete-variable quantum-key distribution}
\newacronym{cvqkd}{CV-QKD}{continuous-variable quantum-key distribution}

\title{A theoretical framework for quantum optical communication - towards CV-QKD}
\author{Bodo Kaiser}
\affil{\textit{bodo.kaiser@huawei.com}}

\begin{document}

	\maketitle
	\tableofcontents

	\chapter{Introduction}
	\begin{refsection}
		\section{Continuous-variable quantum key-distribution}
		\section{Problem statement}
        \section{Overview and structure}
        \section{Notation and conventions}
	
		\addcontentsline{toc}{section}{References}
		\printbibliography[title=References]
	\end{refsection}

	\chapter{Quantum field theory of light}
	\begin{refsection}
		\section{Quantization of the Maxwell field}
		\section{Quantum states, operators and expectation values}

\subsection{Vacuum state}

\subsection{Positive and negative frequency operators}

% TODO: cite operator-valued distributions, smeared fields

The positive and negative frequency operators of the Maxwell field
\begin{align}
	\hat{\vb{A}}^{(-)}
	&=
	\sum_{\lambda=1,2}
	\int_{\mathbb{R}^3}
	\frac{\dd[3]{p}}{(2\pi)^3\sqrt{2\omega(\vb{p})}}
	\hat{a}_\lambda(\vb{p})
	\vu{e}_\lambda(\vb{p})
	\eval{e^{-ip_\mu x^\mu}}_{p_0=\omega(\vb{p})}
	\\
	\hat{\vb{A}}^{(+)}
	&=
	\sum_{\lambda=1,2}
	\int_{\mathbb{R}^3}
	\frac{\dd[3]{p}}{(2\pi)^3\sqrt{2\omega(\vb{p})}}
	\hat{a}_\lambda(\vb{p})^\dagger
	\vu{e}_\lambda(\vb{p})^*
	\eval{e^{+ip_\mu x^\mu}}_{p_0=\omega(\vb{p})}
\end{align}
are operator-valued distributions.

\subsection{Number state}

\subsection{Displacement operator}

\begin{equation}
	\hat{D}[\alpha]
	=
	\exp\left\{
		\int\frac{\dd[3]{p}}{(2\pi)^3\sqrt{2\omega(\vb{p})}}
		\left\{
			\alpha(\vb{p})
			\hat{a}^\dagger(\vb{p})
			-
			\alpha(\vb{p})^*
			\hat{a}(\vb{p})
		\right\}
	\right\}
\end{equation}

\subsection{Coherent state}

\begin{equation}
	\begin{split}
		\ket{\alpha}
		&=
		\exp\left\{
			-
			\frac{1}{2}
			\int\frac{\dd[3]{p}}{(2\pi)^32\omega(\vb{p})}
			\abs{\alpha(\vb{p})}^2
		\right\}
		\\
		&\times
		\exp\left\{
			\int\frac{\dd[3]{p}}{(2\pi)^3\sqrt{2\omega(\vb{p})}}
			\alpha(\vb{p})
			\hat{a}(\vb{p})^\dagger
		\right\}
		\ket{0}
	\end{split}
\end{equation}

\subsection{Quadrature operator}

We define the generalized quadrature operator by 
\begin{equation}
	\hat{X}(\theta)
	=
	\int_{\mathbb{R}^3}
	\frac{\dd[3]{p}}{(2\pi)^3}
	\frac{1}{\sqrt{2}}
	\left\{
		\hat{a}(\vb{p})
		e^{-i\theta}
		+
		\hat{a}^\dagger(\vb{p})
		e^{+i\theta}
	\right\}
\end{equation}
where the prefactor ensures that the commutator takes the standard form
\begin{equation}
	\comm{\hat{X}(\theta)}{\hat{X}(\theta+\Delta\theta)}
	=
	\frac{i}{2}
	\sin(\Delta\theta)
	V_p
\end{equation}
and $V_p$ is the momentum space volume
\begin{equation}
	V_p
	=
	\int_{\mathbb{R}^3}\frac{\dd[3]{p}}{(2\pi)^3}
	=
	\frac{4\pi}{(2\pi)^3}
	\int_0^\Lambda\dd{p}p^2
	=
	\frac{\Lambda^3}{6\pi^2}
\end{equation}
where we introduced the cut-off momentum $\Lambda$ for regularization.

The Robertson uncertainty relation yields a lower bound for the product of the variances
\begin{equation}
	\expval{\left(\Delta\hat{X}(\theta)\right)^2}
	\expval{\left(\Delta\hat{X}(\theta+\Delta\theta)\right)^2}
	\geq
	\frac{1}{4}
	\sin(\Delta\theta)^2
	V_p^2
	.
\end{equation}
The uncertainty is maximal for $\Delta\theta=\pi/2$.
The coherent state is a minimal uncertainty state in the sense that
\begin{align}
	\expval{\hat{X}(\theta)}{\alpha}
	=
	\sqrt{2}
	\int_{\mathbb{R}^3}
	\frac{\dd[3]{p}}{(2\pi)^3\sqrt{2\omega(\vb{p})}}
	\Re\left\{
		\alpha(\vb{p})
		e^{-i\theta}
	\right\}
	&&
	\expval{\left(\Delta\hat{X}(\theta)\right)^2}{\alpha}
	=
	\frac{1}{2}
	V_p
\end{align}

\subsection{Electromagnetic field operator}
		\section{Time-dependent interactions}

\subsection{Time-evolution operator}

Let $\ket{\psi(t_0)}$ be a state at time $t_0$, then the time-evolution relates the state $\ket{\psi(t)}$ at some later time $t>t_0$ to $\ket{\psi(t_0)}$ via
\begin{equation}
	\ket{\psi(t)}
	=
	\hat{U}(t,t_0)
	\ket{\psi(t_0)}
	.
\end{equation}
Inserting $\ket{\psi(t)}$ into the Schrödinger equation leads to
\begin{equation}
	i\dv{t}
	\hat{U}(t,t_0)
	=
	\hat{H}(t)
	\hat{U}(t,t_0)
\end{equation}
which formal solution is the time-ordered exponential, see Ref.~\cite[p.~380]{Bartelmann2018},
\begin{equation}
	\hat{U}(t,t_0)
	=
	T\exp\left\{
		-i
		\int_{t_0}^t\dd{t^\prime}
		\hat{H}(t^\prime)
	\right\}
\end{equation}
where $T$ denotes the time-ordering symbol.
Only for simple time-dependent systems an exact time-evolution operator exists.
In contrast to the Dyson expansion, the Magnus expansion yields a unitary time-evolution operator even for finite order, in particular,
\begin{equation}
	\hat{U}(t,t_0)
	=
	\exp\left\{
		\sum_{n=1}
		\hat{\Omega}^{(n)}(t,t_0)
	\right\}
\end{equation}
where the first two expansion terms are given by
\begin{align}
	\hat{\Omega}^{(1)}(t,t_0)
	&=
	\frac{(-i)}{1!}
	\int_{t_0}^t\dd{t^\prime}
	\hat{H}(t^\prime)
	\\
	\hat{\Omega}^{(2)}(t,t_0)
	&=
	\frac{(-i)^2}{2!}
	\int_{t_0}^t\dd{t^\prime}
	\int_{t_0}^{t^\prime}\dd{t^{\prime\prime}}
	\comm{\hat{H}(t^\prime)}{\hat{H}(t^\prime)}
\end{align}
and represent time-ordering corrections, see Ref.~\cite{QuesadaMejia2015}.

\subsection{Interaction with classical current}

The Schrödinger-picture Hamiltonian describing the interaction of the Maxwell field $\hat{\vb{A}}$ with a classical current $\vb{j}$ is
\begin{equation}
	\hat{H}_\text{int}(t)
	=
	-
	\vb{j}(t,\vb{x})
	\vdot
	\hat{\vb{A}}(t,\vb{x})
	.
\end{equation}
Inserting the mode expansion
		\section{Approximations}

\begin{itemize}
	\item Polarization
	\item One-dimension
	\item Relativistic filter
\end{itemize}

		\addcontentsline{toc}{section}{References}
		\printbibliography[title=References]
	\end{refsection}
	
	\chapter{Optical components}
	\begin{refsection}
		\section{Laser}

The Hamiltonian describing a laser is given by
\begin{equation}
	\hat{H}
	=
	\hat{H}_a
	+
	\hat{H}_m
	+
	\hat{H}_{ae}
	+
	\hat{H}_{am}
	+
	\hat{H}_{me}
\end{equation}
where we have the free Hamiltonian of the atoms inside the cavity
\begin{equation}
	\hat{H}_a
	=
	\sum_{n=1}^N
	\frac{1}{2}
	\omega
	\hat\sigma_{z,n}
\end{equation}
which are approximated as independent two-level spin-like system and the free photon field
\begin{equation}
	\hat{H}_m
	=
	\int\dd{\omega}
	\omega
	\hat{a}^\dagger(\omega)
	\hat{a}(\omega)
\end{equation}
The environment is coupled to the photons
\begin{equation}
	\hat{H}_{me}
	=
	\Gamma_p
	\left(
		\hat{a}(\omega)
		+
		\hat{a}^\dagger(\omega)
	\right)
\end{equation}
and the atoms
\begin{equation}
	\hat{H}_{ae}
	=
	\sum_{n=1}^N
	\Gamma_{a,n}
	\left(
		\hat\sigma_{+,n}
		+
		\hat\sigma_{-,n}
	\right)
\end{equation}
The atoms are coupled with the photon field with
\begin{equation}
	\hat{H}_{am}
	=
	ig
	\sum_{n=1}^N
	\left(
		\hat{a}^\dagger(\omega)
		\hat\sigma_{-,n}
		-
		\hat{a}(\omega)
		\hat\sigma_{-,n}^\dagger
	\right)
\end{equation}
		\section{Mode coupler}

\subsection{Modulator}

\subsubsection{Sampling}

		\documentclass[tikz]{standalone}

\usepackage{circuitikz}
\usepackage{physics}

\usetikzlibrary{calc,positioning}

\begin{document}
	\begin{tikzpicture}
		\coordinate (in) at (0,0);
		\node (splt) [twoportsplitshape, right=of in]{};
		\node (phst) [vphaseshiftershape, above right=7em of splt] {};
		\node (osc) [vcoshape, below=2em of phst, label={180:$x(t)$}] {};
		\coordinate [right=14em of splt] (out1);
		\coordinate (out2) at (out1|-phst.west);

		\draw (in) node[left]{$\ket{\alpha(t)}$} -- (splt.west) node[inputarrow]{};
		\draw (splt.east) -- (out1) node[inputarrow]{} node[right]{$\ket{\frac{1}{\sqrt{2}}\alpha(t)e^{i\phi_1}}$};
		\draw (splt.north) -- (splt.north|-phst.west) -- (phst.west) node[inputarrow]{};
		\draw (phst.east) -- (out2) node[inputarrow]{} node[right]{$\ket{\frac{1}{\sqrt{2}}\alpha(t)e^{i\phi_2+i\varphi_2(t)}}$};		
		\draw (osc.north) -- (phst.south) node[inputarrow, rotate=90]{};
	\end{tikzpicture}
\end{document}

		\section{Detector}

\subsection{Photodiode}

\subsection{Homo- and heterodyne detectors}

\Cref{fig:balanced_detector_optics} illustrates the essential optical setup for balanced detection:
a 50:50 beam splitter (depicted by the square with the diagonal line) mixes two optical input fields $E_l,E_s$.
\begin{figure}[htb]
    \centering
    \includestandalone{figures/tikz/balanced-detector}
    \caption{Optical part of the balanced detector setup comprising a beam splitter and two photodetectors.}\label{fig:balanced_detector_optics}
\end{figure}
The intensities of the two optical output fields leaving the beam splitter, represented by $E_\pm$, are separately measured by two photodiodes (depicted by the filled half-circles).
The photodiodes output a current signal $i_\pm$ proportional to the registered light field's intensity.

$E_l$ is called the \gls{lo} and is assumed monochromatic, i.e.,
\begin{equation}
    E_l(t)
    =
    E_l(0)\cos(\omega_l t+\theta)
    \label{eq:efield_lo}.
\end{equation}
$E_s$ is a real passband signal with optical carrier frequency $\omega_c$
\begin{equation}
    E_s(t)
    =
    E_I(t)\cos(\omega_c t)-E_Q(t)\sin(\omega_c t)
    \label{eq:efield_signal}
\end{equation}
with $E_I(t),E_Q(t)$ being the quadrature components of the baseband signal of $E_s(t)$.
We have factored the relative phase $\theta$ between \gls{lo} and signal to $E_l(t)$.

From our discussions of the beam splitter in \cref{sec:beam_splitter}, we know that
\begin{equation}
    E_\pm(t)
    =
    \frac{E_s(t)\pm E_l(t)}{\sqrt{2}}
\end{equation}
up to some phase factors.

\Cref{fig:balanced_detector_electronics} depicts the electronic circuit of the balanced detector.
The balanced detector is characterized by two photodiodes PD+ and PD- connected in series and allowing current flow in one direction.
According to Kirchhoff's current law, the balanced current signal is equal to the difference of the individual photocurrents, i.e., $i_+-i_-$
In the present embodiment, we include an operational amplifier into the balanced detector configuration.
\begin{figure}[htb]
    \centering
    \includestandalone{figures/circuitikz/balanced-detector}
    \caption{Schematic of the balanced detector circuit.}\label{fig:balanced_detector_electronics}
\end{figure}
The operational amplifier is configured as transimpedance amplifier where the non-inverting input is grounded, the inverting input is feed with the input current signal, and the voltage output is coupled through a feedback impedance $Z_f$ with the inverting voltage input.
For an ideal transimpedance amplifier, the output voltage relates to the photocurrents via
\begin{equation}
    V_\text{out}
    =-Z_f(i_+-i_-).
\end{equation}
We end our analysis of the balanced detector to conclude that the photocurrent difference is our relevant quantity we need to discuss.

The photocurrents of the photodectors are proportional to the incident beam power which in turn is proportional to the squared amplitude of the electric field component
\begin{equation}
    i_\pm(t)
    \propto
    E_\pm(t)^2
    =
    \frac{1}{2}
    \bigl\{
        E_l(t)^2+E_s(t)^2\pm 2E_l(t)E_s(t)
    \bigr\}
    \label{eq:photocurrent_individual}.
\end{equation}
From \Cref{fig:balanced_detector_electronics} we know that the current signal of the balanced detector is equal to the difference of the individual photocurrents
\begin{equation}
    i(t)
    =
    i_+(t)-i_-(t)
    \propto
    2E_l(t)E_s(t)
    \label{eq:photocurrent_difference1}
\end{equation}
Inserting \cref{eq:efield_lo} and \cref{eq:efield_signal} into \cref{eq:photocurrent_difference1} yields
\begin{equation}
    i(t)
    \propto
    2E_l(0)\cos(\omega_lt+\theta)\bigl\{E_I(t)\cos(\omega_ct)-E_Q(t)\sin(\omega_ct)\bigr\}
    \label{eq:photocurrent_difference3}.
\end{equation}
Finally, by using the trigonometric identities
\begin{align}
    \cos(x)\cos(y)
    &=
    \frac{1}{2}\bigl(\cos(x+y)+\cos(x-y)\bigr)\\
    \cos(x)\sin(y)
    &=
    \frac{1}{2}\bigl(\sin(x+y)-\sin(x-y)\bigr)
\end{align}
we can rewrite \cref{eq:photocurrent_difference3} as
\begin{equation}
    i(t)
    \propto
    E_l(0)\biggl\{
        E_I(t)\bigl[\cos(\omega_+t+\theta)+\cos(\omega_-t+\theta)\bigr]
        +
        E_Q(t)\bigl[\sin(\omega_+t+\theta)-\sin(\omega_-t+\theta)\bigr]
    \biggr\}
    \label{eq:photocurrent_difference_final}
\end{equation}
where we have defined $\omega_\pm=\omega_l\pm\omega_c$.

So far we neglected the finite response-time of the photodiodes.
We assume that the photocurrent is attenuated by a low-pass with transfer function $h(\omega)$ and cut-off frequency $\omega_0$.
The bandwidth-limited balanced detector current signal $i^\prime(t)$ is obtained by convolution of the low-pass transfer function with the infinite bandwidth signal $i(t)$
\begin{equation}
    i^\prime(t)
    =
    (h*i)(t)
    =
    \int_\mathbb{R}\dd{\omega} h(\omega)i(\omega)e^{-i\omega t}
    \label{eq:photocurrent_difference_bandwidth_limited}
\end{equation}
where we invoked the convolution theorem for the last equal.
The Fourier transform of the unlimited bandwidth balanced detector signal is
\begin{align}
    i(\omega)
    \propto
    E_l(0)\int_\mathbb{R}\dd{t}e^{i\omega t}
    \biggl\{
        &E_I(t)\bigl[\cos(\omega_+t+\theta)+\cos(\omega_-t+\theta)\bigr]\\
       -&E_Q(t)\bigl[\sin(\omega_+t+\theta)-\sin(\omega_-t+\theta)\bigr]
    \biggr\}
    \label{eq:photocurrent_difference_spectrum1}.
\end{align}
We can express \cref{eq:photocurrent_difference_spectrum1} in terms of the spectral representation of the in-phase and quadrature components.
We show this exemplary for the in-phase component
\begin{align*}
    \int_\mathbb{R}\dd{t}E_I(t)e^{i\omega t}\cos(\omega_\pm t+\theta)
    &=
    \frac{e^{+i\theta}}{2}\underbrace{\int_\mathbb{R}\dd{t}E_I(t)e^{i(\omega+\omega_\pm)t}}_{E_I(\omega+\omega_\pm)}
    +
    \frac{e^{-i\theta}}{2}\underbrace{\int_\mathbb{R}\dd{t}E_I(t)e^{i(\omega-\omega_\pm)t}}_{E_I(\omega-\omega_\pm)}\\
    \int_\mathbb{R}\dd{t}E_Q(t)e^{i\omega t}\sin(\omega_\pm t+\theta)
    &=
    \frac{e^{+i\theta}}{2i}\underbrace{\int_\mathbb{R}\dd{t}E_Q(t)e^{i(\omega+\omega_\pm)t}}_{E_Q(\omega+\omega_\pm)}
    -
    \frac{e^{-i\theta}}{2i}\underbrace{\int_\mathbb{R}\dd{t}E_Q(t)e^{i(\omega-\omega_\pm)t}}_{E_Q(\omega-\omega_\pm)}
    .
\end{align*}
with this in mind \cref{eq:photocurrent_difference_spectrum1} becomes
\begin{align*}
    i(\omega)
    \propto
    E_l(0)
    \biggl\{
        &\frac{e^{+i\theta}}{2}E_I(\omega+\omega_+)+\frac{e^{-i\theta}}{2}E_I(\omega-\omega_+)\\
        +&\frac{e^{+i\theta}}{2}E_I(\omega+\omega_-)+\frac{e^{-i\theta}}{2}E_I(\omega-\omega_-)\\
        -&\frac{e^{+i\theta}}{2i}E_I(\omega+\omega_+)+\frac{e^{-i\theta}}{2i}E_I(\omega-\omega_+)\\
        +&\frac{e^{+i\theta}}{2i}E_I(\omega+\omega_-)-\frac{e^{-i\theta}}{2i}E_I(\omega-\omega_-)
    \biggr\}
\end{align*}
we factor out the phases
\begin{align*}
    i(\omega)
    \propto
    \frac{1}{2}
    E_l(0)
    \biggl\{
        &e^{+i\theta}
        \bigl[
            E_I(\omega+\omega_+)+E_I(\omega+\omega_-)+iE_Q(\omega+\omega_+)-iE_Q(\omega+\omega_-)
        \bigr]\\
        +
        &e^{-i\theta}
        \bigl[
            E_I(\omega-\omega_+)+E_I(\omega-\omega_-)-iE_Q(\omega-\omega_+)+iE_Q(\omega-\omega_-)
        \bigr]
    \biggr\},
\end{align*}
after defining the complex spectrum
\begin{equation}
    A_\pm(\omega)
    =
    E_I(\omega)\pm iE_Q(\omega)
    \label{eq:aux_spectrum},
\end{equation}
we find the compact expression
\begin{align}
    i(\omega)
    \propto
    \frac{1}{2}E_l(0)
    \biggl\{
        &e^{+i\theta}
        \bigl[
            A_+(\omega+\omega_+)+A_-(\omega+\omega_-)
        \bigr]\\
        +
        &e^{-i\theta}
        \bigl[
            A_-(\omega-\omega_+)+A_+(\omega-\omega_-)
        \bigr]
    \biggr\}
\end{align}
Inserting \cref{eq:photocurrent_difference_spectrum2} into \cref{eq:photocurrent_difference_bandwidth_limited}, we find
\begin{align}
    i^\prime(t)
    \propto
    \frac{1}{2}E_l(0)
    \int_\mathbb{R}\dd{\omega}
    h(\omega)
    \biggl\{
        &e^{+i\theta}
        \bigl[
            A_+(\omega+\omega_+)+A_-(\omega+\omega_-)
        \bigr]\\
        +
        &e^{-i\theta}
        \bigl[
            A_-(\omega-\omega_+)+A_+(\omega-\omega_-)
        \bigr]
    \biggr\}
    e^{-i\omega t}
    \label{eq:photocurrent_difference_spectrum3}.
\end{align}
Usually, the transimpedance amplifier has a smaller bandwidth than the photodiode, and we can approximate the photodiode's transfer function by
\begin{equation}
    h(\omega)
    \approx
    \begin{cases}
        1 & \text{if } -B<\omega<+B\\
        0 & \text{otherwise}
    \end{cases}
\end{equation}
and the balanced detector's bandwidth-limited current signal take the form
\begin{align}
    i^\prime(t)
    \approxprop
    \frac{1}{2}
    E_l(0)
    \int_{-B}^{+B}\dd{\omega}
    \biggl\{
        &e^{+i\theta}
        \bigl[
            A_+(\omega+\omega_+)+A_-(\omega+\omega_-)
        \bigr]\\
        +
        &e^{-i\theta}
        \bigl[
            A_-(\omega-\omega_+)+A_+(\omega-\omega_-)
        \bigr]
    \biggr\}
    e^{-i\omega t}.
\end{align}
For real photodiodes, we have $B\ll\omega_+$ which allows us to neglect the $A_+$ terms as they represent complex passband signals, i.e., their signal power is concentrated at $\pm\omega_+$,
\begin{align}
    i^\prime(t)
    &\approxprop
    \frac{1}{2}
    E_l(0)
    \int_{-B}^{+B}\dd{\omega}
    \biggl\{
        A_-(\omega+\omega_\text{IF})e^{+i\theta}
        +
        A_+(\omega-\omega_\text{IF})e^{-i\theta}
    \biggr\}
    e^{-i\omega t}
\end{align}
where we now refer to $\omega_-$ as the \gls{if} $\omega_\text{IF}$.
Using the variable substitution $\omega^\prime=\omega\pm\omega_\text{IF}$, we can write the previous equation as
\begin{align}
    i^\prime(t)
    \approxprop
    \frac{1}{2}
    E_l(0)
    \biggl\{
        &e^{+i(\omega_\text{IF}t+\theta)}
        \int_{-B+\omega_\text{IF}}^{+B-\omega_\text{IF}}\dd{\omega^\prime}
            A_-(\omega^\prime)e^{-i\omega^\prime t}\\
        +
        &e^{-i(\omega_\text{IF}t+\theta)}
        \int_{-B-\omega_\text{IF}}^{+B+\omega_\text{IF}}\dd{\omega^\prime}
            A_+(\omega^\prime)e^{-i\omega^\prime t}
    \biggr\}.
\end{align}
$A_\pm(\omega)$ inherits a symmetry from the conjugate symmetry of the quadrature baseband signals
\begin{equation}
    A_\pm(-\omega)
    =
    E_I(-\omega)\pm iE_Q(-\omega)
    =
    \bigl(
    E_I(\omega)\mp iE_Q(\omega)
    \bigr)^*
    =
    A_\mp^*(\omega)
    .
\end{equation}
With this symmetry in mind, we can rewrite the spectrum
\begin{align}
    \int_{-(B\mp\omega_\text{IF})}^{+(B\mp\omega_\text{IF})}\dd{\omega^\prime}
    A_\mp(\omega^\prime)e^{-i\omega^\prime t}
    =
    \int_0^{B\mp\omega_\text{IF}}\dd{\omega^\prime}
    \bigg\{
        A_\mp(\omega^\prime)e^{-i\omega^\prime t}
        +
        A_\pm^*(\omega^\prime)e^{+i\omega^\prime t}
    \bigg\},
\end{align}
and our balanced detector signal becomes
\begin{align}
    i^\prime(t)
    \approxprop
    \frac{1}{2}E_l(0)
    \biggl\{
        &e^{+i(\omega_\text{IF}t+\theta)}
        \int_{0}^{B-\omega_\text{IF}}\dd{\omega^\prime}
            \bigl[
                A_-(\omega^\prime)e^{-i\omega^\prime t}
                +
                A_+^*(\omega^\prime)e^{+i\omega^\prime t}
            \bigr]\\
            +
        &e^{-i(\omega_\text{IF}t+\theta)}
        \int_{0}^{B+\omega_\text{IF}}\dd{\omega^\prime}
            \bigl[
                A_+(\omega^\prime)e^{-i\omega^\prime t}
                +
                A_-^*(\omega^\prime)e^{+i\omega^\prime t}
            \bigr]
    \biggr\}.
\end{align}
If we ignore the \gls{if} in the integration domain
\begin{equation}
    \int_{-B}^{+B}\dd{\omega}A_\mp(\omega)e^{-i\omega t}
    =
    \underbrace{\int_{-B}^{+B}\dd{\omega}E_I(\omega)e^{-i\omega t}}_{\overline{E}_I(t)}
    \mp
    i\underbrace{\int_{-B}^{+B}\dd{\omega}E_Q(\omega)e^{-i\omega t}}_{\overline{E}_Q(t)}
\end{equation}
we recover the (bandwidth-limited) quadrature signals.
In summary, the bandwidth-limited balanced detection current signal turns out to be
\begin{align}
    i^\prime(t)
    &\approxprop
    \frac{1}{2}E_l(0)
    \biggl\{
        \bigl[
            \overline{E}_I(t)-i\overline{E}_Q(t)
        \bigr]
        e^{+i(\omega_\text{IF}t+\theta)}
        +
        \bigl[
            \overline{E}_I(t)+i\overline{E}_Q(t)
        \bigr]
        e^{-i(\omega_\text{IF}t+\theta)}
    \biggr\}\\
    &=
    E_l(0)
    \bigl\{
        \overline{E}_I(t)\cos(\omega_\text{IF}t+\theta)
        +
        \overline{E}_Q(t)\sin(\omega_\text{IF}t+\theta)
    \bigr\}
    \label{eq:balanced_detection_current_final}.
\end{align}

In the case of homodyne detection, $\omega_\text{IF}=0$, \cref{eq:balanced_detection_current_final} reduces to
\begin{equation}
    i^\prime(t)
    \propto
    E_l(0)
    \bigl\{
        \overline{E}_I(t)\cos(\theta)
        +
        \overline{E}_Q(t)\sin(\theta)
    \bigr\}.
\end{equation}
If the \gls{lo}'s phase $\theta$ runs in sync with the carrier phase $\theta=0$, the current signal is proportional to the bandwidth-limited in-phase baseband $\overline{E}_I(t)$.
If the \gls{lo}'s phase $\theta$ is shifted by $\theta=\pi$, the current signal is proportional to the bandwidth-limited quadrature baseband $\overline{E}_Q(t)$.

		
		\addcontentsline{toc}{section}{References}
		\printbibliography[title=References]
	\end{refsection}

	\chapter{Coherent transmitter and receiver}
	\begin{refsection}
		\section{Overview and architecture}
		\section{Transmitter}
		\section{Receiver}
		
		\addcontentsline{toc}{section}{References}
		\printbibliography[title=References]
	\end{refsection}

	\chapter{Conclusion and outlook}
	\begin{refsection}	
		\addcontentsline{toc}{section}{References}
		\printbibliography[title=References]
	\end{refsection}

	\appendix

\end{document}

\documentclass[a4paper,appendixprefix]{scrreprt}

\usepackage[utf8]{inputenc}
\usepackage{amsthm,amsmath,amssymb}
\usepackage{authblk}
\usepackage[english]{babel}
\usepackage[backend=biber]{biblatex}
\usepackage{booktabs}
%\usepackage{csquotes}
\usepackage[acronym,nonumberlist,toc]{glossaries}
\usepackage{hyperref}
\usepackage{cleveref}
\usepackage{physics}
\usepackage[separate-uncertainty=true]{siunitx}
\usepackage{standalone}
%\usepackage[subpreambles=true]{standalone}
\usepackage{thmtools,thm-restate}
%\usepackage{subfig}
\usepackage{wrapfig}
\usepackage{xcolor}

\addbibresource{references/articles.bib}
\addbibresource{references/books.bib}
\addbibresource{references/thesis.bib}

% add bibliography as section (not chapter)
% https://tex.stackexchange.com/questions/568580/make-the-bibliography-as-a-section-in-each-included-chapter
\defbibheading{bibliography}[\bibname]{\section*{#1}}

% approximately proportional to symbol
% https://tex.stackexchange.com/questions/33538/how-to-get-an-approximately-proportional-to-symbol
\def\app#1#2{%
    \mathrel{%
        \setbox0=\hbox{$#1\sim$}%
        \setbox2=\hbox{%
            \rlap{\hbox{$#1\propto$}}%
            \lower1.1\ht0\box0%
        }%
        \raise0.25\ht2\box2%
    }%
}
\def\approxprop{\mathpalette\app\relax}

% overwrite real and imaginary part operators
\let\Re\undefined
\let\Im\undefined
\DeclareMathOperator{\Re}{\operatorname{Re}}
\DeclareMathOperator{\Im}{\operatorname{Im}}
% other functions
\DeclareMathOperator{\sinc}{\operatorname{sinc}}

\newcommand{\floor}[1]{\left\lfloor#1\right\rfloor}
\newcommand{\ceil}[1]{\left\lceil#1\right\rceil}

% transpose
% https://tex.stackexchange.com/questions/403104/small-caps-mathsf-font-for-writing-transpose-of-a-matrix
\newcommand{\trans}{{\scriptscriptstyle\mathsf{T}}}

% theorems
\newtheorem{theorem}{Theorem}[section]
\newtheorem{lemma}[theorem]{Lemma}
\newtheorem{corollary}[theorem]{Corollary}
\theoremstyle{definition}
\newtheorem{definition}{Definition}[section]
\newtheorem{conjecture}{Conjecture}[section]
\newtheorem{example}{Example}[section]
\theoremstyle{remark}
\newtheorem*{remark}{Remark}

% prefix equation numbers with section number
\numberwithin{equation}{section}

% optics
\newacronym{ar}{AR}{anti-reflective}
\newacronym{mzm}{MZM}{Mach-Zehnder modulator}
\newacronym{bs}{BS}{beam splitter}
\newacronym{fc}{FC}{fiber coupler}
\newacronym{qe}{QE}{quantum efficiency}

% physics
\newacronym{dv}{DV}{discrete-variable}
\newacronym{cv}{CV}{continuous-variable}
\newacronym{dof}{DOF}{degrees of freedom}
\newacronym{eom}{EOM}{equation(s) of motion}
\newacronym{pbc}{PBC}{periodic boundary conditions}
\newacronym{bch}{BCH}{Baker-Campbell-Hausdorff}
\newacronym{ccr}{CCR}{canonical commutation relation}

% signal processing
\newacronym{dsp}{DSP}{digital signal processing}
\newacronym{lo}{LO}{local oscillator}
\newacronym{if}{IF}{intermediate frequency}
\newacronym{lp}{LP}{low-pass}
\newacronym{adc}{ADC}{analog-to-digital converter}
\newacronym{dac}{DAC}{digital-to-analog converter}
\newacronym{qam}{QAM}{quadrature amplitude modulation}
\newacronym{qpsk}{QPSK}{quadrature phase-shift keying}

% quantum-key distribution
\newacronym{qkd}{QKD}{quantum-key distribution}
\newacronym{dvqkd}{DV-QKD}{discrete-variable quantum-key distribution}
\newacronym{cvqkd}{CV-QKD}{continuous-variable quantum-key distribution}

\title{A theoretical framework for quantum optical communication - towards CV-QKD}
\author{Bodo Kaiser}
\affil{\textit{bodo.kaiser@huawei.com}}

\begin{document}

	\maketitle
	\tableofcontents

	\chapter{Introduction}
	\begin{refsection}
		\section{Quantum science and technology}
		\section{Motivation and problem statement}
        \section{Thesis overview and structure}
        \section{Notation and conventions}
	
		\addcontentsline{toc}{section}{References}
		\printbibliography[title=References]
	\end{refsection}

	\chapter{Quantum field theory of light}
	\begin{refsection}
		The present chapter identifies common characteristics of \gls{qkd} protocols and attempts to formalize the notion of a \gls{qkd} protocol.
We test the proposed formula with qubit-based \gls{qkd} protocols like BB84 and the six-state protocol as well as Gaussian \gls{cvqkd} protocols.
The chapter ends with a summary and literature review regarding practical \gls{qkd} protocols' post-processing and security analysis.

\Cref{fig:qkd_protocol} illustrates our proposed notion of a \gls{qkd} protocol, with the feature being a logical quantum system from which the random bits are encoded and decoded.
The logical quantum system is a subspace of the physical quantum system.
The physical quantum system depends strongly on the physical implementation and quantum encoding.
\begin{figure}[htb]
	\centering
	\includestandalone{figures/tikz/qkd-protocol}
	\caption{A \gls{qkd} protocol comprises a binary encoder, a logical quantum system, and a binary decoder. The binary encoder maps bits $\vb{b}\in\{0,1\}^n$ onto a quantum state of the logical quantum system $\ket{\psi}$. The binary decoder extracts the bits $\vb{b}$ back from the quantum state $\ket{\psi^\prime}$. The logical quantum system is a subspace of a larger physical quantum system. The state encoder and decoder map between the logical and physical quantum states.}\label{fig:qkd_protocol}
\end{figure}
The distinction between logical and quantum systems is vital to separate the implementation and security concerns.
Many security proofs show equivalence between the physical implementation and the logical system to use an established security proof.
One should keep in mind that such a separation implicitly assumes no loopholes from the particular implementation.

\Cref{fig:qkd_classification} illustrates common features among the \gls{qkd} protocols.
For an overview of \gls{qkd} protocols, see Ref.~\cite{Duvsek2006}.
\begin{figure}[htb]
	\centering
	\includestandalone{figures/tikz/qkd-classification}
	\caption{Common features among \gls{qkd} protocols: Detection, physical encoding, logical state space, measurement basis selection and schema.}\label{fig:qkd_classification}
\end{figure}
Every \gls{qkd} system requires a detector, e.g., a coherent detector or a single-photon (click) detector.
The detection does not necessarily imply the dimension of the logical state space as BB84 has been implemented with coherent detection~\cite{Qi2021}.
Concerning measurement basis selection, Bob can either actively choose a random measurement basis for every transmission or passively measure all (orthogonal) bases for measurement basis selection.
We will cover both active and passive measurement basis selection in the discussion of the polarization-encoding BB84 protocol.
Finally, the \gls{qkd} schema determines if either Alice prepares a state and sends it to Bob for measurement (prepare-and-measure) or if Alice and Bob share an entangled state (entanglement-based).
Most practical \gls{qkd} implementations use prepare-and-measure.
On a theoretical level, both schemas are equivalent, and security proofs are often more convenient in an entanglement-based setting.
		\section{Quantization of the Maxwell field in the Coulomb gauge}

\subsection{Relativistic field theory}

The Lagrangian of the Maxwell field $A^\mu(t,\vb{x})$ reads~\cite[p.~339]{Srednicki2007}
\begin{equation}
	\mathcal{L}
	=
	\frac{1}{2}
	(\partial_\mu A_\nu)
	\left(
		\partial^\nu A^\mu
		-
		\partial^\mu A^\nu
	\right)
\end{equation}
and the covariant generalization of the Euler-Lagrange equations
\begin{equation}
	0
	=
	\partial_\mu
	\pdv{\mathcal{L}}{(\partial_\mu A_\nu)}
	-
	\pdv{\mathcal{L}}{A_\nu}
	=
	\partial_\mu\partial^\mu A^\nu
	-
	\partial^\nu\partial_\mu A^\mu
\end{equation}
leads to the free equations of motion.
We ignore static charges $A_0=0$ and employ the Coulomb gauge $\partial_iA^i=0$ in which the Maxwell field is transverse.
The equations of motion simplify to relativistic wave equation
\begin{equation}
	0
	=
	\partial_\mu\partial^\mu\vb{A}
	=
	\partial_t^2\vb{A}
	-
	\laplacian\vb{A}
	.
\end{equation}

\subsection{Mode decomposition}

In momentum space the transverse field $\vb{A}$ reads
\begin{equation}
	\vb{A}(t,\vb{x})
	=
	\int_{\mathbb{R}^4}\frac{\dd[4]{p}}{(2\pi)^4}
	\vb{A}(p_0,\vb{p})
	e^{ip_0t-i\vb{p}\vdot\vb{x}}
	.
\end{equation}
In momentum space, we can construct a polarization basis $\vu{e}_1(\vb{p}),\vu{e}_2(\vb{p})$ being transverse
\begin{equation}
	\vb{p}\vdot\vu{e}_\lambda(\vb{p})
	=
	0
	,
\end{equation}
orthonormal
\begin{equation}
	\vu{e}_\lambda(\vb{p})
	\vdot
	\vu{e}_{\lambda^\prime}(\vb{p})
	=
	\delta_{\lambda,\lambda^\prime}
\end{equation}
and complete
\begin{equation}
	\sum_{\lambda=1,2}
	\vu{e}_\lambda^i(\vb{p})
	\vu{e}_{\lambda^\prime}^j(\vb{p})
	=
	\delta^{ij}
	-
	\frac{p^ip^j}{\vb{p}^2}
	=
	P_\perp^{ij}(\vb{p})
\end{equation}
with $P_\perp(\vb{p})$ being the transverse projector.
Expressing $\vb{A}(p_0,\vb{p})$ in the polarization basis, we find
\begin{equation}
	\vb{A}(p_0,\vb{p})
	=
	\sum_{\lambda=1,2}
	A_\lambda(p_0,\vb{p})
	\vu{e}_\lambda(\vb{p})
\end{equation}
and the mode decomposition reads
\begin{equation}
	\vb{A}(t,\vb{x})
	=
	\sum_{\lambda=1,2}
	\int_{\mathbb{R}^4}\frac{\dd[4]{p}}{(2\pi)^4}
	A_\lambda(p_0,\vb{p})
	e^{ip_0t-i\vb{p}\vdot\vb{x}}
	\vu{e}_\lambda(\vb{p})
	.
\end{equation}
Inserting the mode decomposition into the relativistic wave equation, we recover the relativistic energy-momentum relation for massless particles
\begin{equation}
	E(\vb{p})
	=
	\omega(\vb{p})
	=
	\vb{p}
	.
\end{equation}
Hence, if the Fourier modes $a_\lambda(p_0,\vb{p})$ satisfy the relativistic energy-momentum relation, $\vb{A}(t,\vb{x})$ satisfies the relativistic wave equation.
We enforce the mode decomposition to satisfy the energy-momentum relation by constraining the integration domain to
\begin{equation}
	V_p
	=
	\left\{
		(p_0,\vb{p})\in\mathbb{R}^4
		\colon
		p_0^2
		=
		\omega(\vb{p})^2
	\right\}
\end{equation}
or equivalent, adding a factor
\begin{equation}
	(2\pi)
	\delta^{(1)}\left(p_0^2-\omega(\vb{p})^2\right)
	=
	(2\pi)
	\frac{
		\delta^{(1)}\left(p_0-\omega(\vb{p})\right)
		-
		\delta^{(1)}\left(p_0+\omega(\vb{p})\right)
	}{\sqrt{2\omega(\vb{p})}}
\end{equation}
to the integrand.
Finally, we arrive at
\begin{equation}
	\vb{A}(t,\vb{x})
	=
	\sum_{\lambda=1,2}
	\int_{\mathbb{R}^3}\frac{\dd[3]{p}}{(2\pi)^3\sqrt{2\omega(\vb{p})}}
	\left\{
		a_\lambda(\vb{p})
		e^{i\omega(\vb{p})t-i\vb{p}\vdot\vb{x}}
		\vu{e}_\lambda(\vb{p})
		+
		\text{c.c.}
	\right\}
\end{equation}
where we defined
\begin{equation}
	a_\lambda(\vb{p})
	=
	A_\lambda\left(\omega(\vb{p}),\vb{p}\right)
\end{equation}
and used the conjugate symmetry of the Fourier amplitudes $a_\lambda(-\vb{p})=a_\lambda(\vb{p})^*$.

\subsection{Canonical quantization}
		\section{Quantum states, operators and expectation values}

\subsection{Vacuum state}

\subsection{Positive and negative frequency operators}

% TODO: cite operator-valued distributions, smeared fields

The positive and negative frequency operators of the Maxwell field
\begin{align}
	\hat{\vb{A}}^{(-)}
	&=
	\sum_{\lambda=1,2}
	\int_{\mathbb{R}^3}
	\frac{\dd[3]{p}}{(2\pi)^3\sqrt{2\omega(\vb{p})}}
	\hat{a}_\lambda(\vb{p})
	\vu{e}_\lambda(\vb{p})
	\eval{e^{-ip_\mu x^\mu}}_{p_0=\omega(\vb{p})}
	\\
	\hat{\vb{A}}^{(+)}
	&=
	\sum_{\lambda=1,2}
	\int_{\mathbb{R}^3}
	\frac{\dd[3]{p}}{(2\pi)^3\sqrt{2\omega(\vb{p})}}
	\hat{a}_\lambda(\vb{p})^\dagger
	\vu{e}_\lambda(\vb{p})^*
	\eval{e^{+ip_\mu x^\mu}}_{p_0=\omega(\vb{p})}
\end{align}
are operator-valued distributions.

\subsection{Number state}

\subsection{Displacement operator}

\begin{equation}
	\hat{D}[\alpha]
	=
	\exp\left\{
		\int\frac{\dd[3]{p}}{(2\pi)^3\sqrt{2\omega(\vb{p})}}
		\left\{
			\alpha(\vb{p})
			\hat{a}^\dagger(\vb{p})
			-
			\alpha(\vb{p})^*
			\hat{a}(\vb{p})
		\right\}
	\right\}
\end{equation}

\subsection{Coherent state}

\begin{equation}
	\begin{split}
		\ket{\alpha}
		&=
		\exp\left\{
			-
			\frac{1}{2}
			\int\frac{\dd[3]{p}}{(2\pi)^32\omega(\vb{p})}
			\abs{\alpha(\vb{p})}^2
		\right\}
		\\
		&\times
		\exp\left\{
			\int\frac{\dd[3]{p}}{(2\pi)^3\sqrt{2\omega(\vb{p})}}
			\alpha(\vb{p})
			\hat{a}(\vb{p})^\dagger
		\right\}
		\ket{0}
	\end{split}
\end{equation}

\subsection{Quadrature operator}

We define the generalized quadrature operator by 
\begin{equation}
	\hat{X}(\theta)
	=
	\int_{\mathbb{R}^3}
	\frac{\dd[3]{p}}{(2\pi)^3}
	\frac{1}{\sqrt{2}}
	\left\{
		\hat{a}(\vb{p})
		e^{-i\theta}
		+
		\hat{a}^\dagger(\vb{p})
		e^{+i\theta}
	\right\}
\end{equation}
where the prefactor ensures that the commutator takes the standard form
\begin{equation}
	\comm{\hat{X}(\theta)}{\hat{X}(\theta+\Delta\theta)}
	=
	\frac{i}{2}
	\sin(\Delta\theta)
	V_p
\end{equation}
and $V_p$ is the momentum space volume
\begin{equation}
	V_p
	=
	\int_{\mathbb{R}^3}\frac{\dd[3]{p}}{(2\pi)^3}
	=
	\frac{4\pi}{(2\pi)^3}
	\int_0^\Lambda\dd{p}p^2
	=
	\frac{\Lambda^3}{6\pi^2}
\end{equation}
where we introduced the cut-off momentum $\Lambda$ for regularization.

The Robertson uncertainty relation yields a lower bound for the product of the variances
\begin{equation}
	\expval{\left(\Delta\hat{X}(\theta)\right)^2}
	\expval{\left(\Delta\hat{X}(\theta+\Delta\theta)\right)^2}
	\geq
	\frac{1}{4}
	\sin(\Delta\theta)^2
	V_p^2
	.
\end{equation}
The uncertainty is maximal for $\Delta\theta=\pi/2$.
The coherent state is a minimal uncertainty state in the sense that
\begin{align}
	\expval{\hat{X}(\theta)}{\alpha}
	=
	\sqrt{2}
	\int_{\mathbb{R}^3}
	\frac{\dd[3]{p}}{(2\pi)^3\sqrt{2\omega(\vb{p})}}
	\Re\left\{
		\alpha(\vb{p})
		e^{-i\theta}
	\right\}
	&&
	\expval{\left(\Delta\hat{X}(\theta)\right)^2}{\alpha}
	=
	\frac{1}{2}
	V_p
\end{align}

\subsection{Electromagnetic field operator}
		\section{Time-dependent interactions}

\subsection{Time-evolution operator}

Let $\ket{\psi(t_0)}$ be a state at time $t_0$, then the time-evolution relates the state $\ket{\psi(t)}$ at some later time $t>t_0$ to $\ket{\psi(t_0)}$ via
\begin{equation}
	\ket{\psi(t)}
	=
	\hat{U}(t,t_0)
	\ket{\psi(t_0)}
	.
\end{equation}
Inserting $\ket{\psi(t)}$ into the Schrödinger equation leads to
\begin{equation}
	i\dv{t}
	\hat{U}(t,t_0)
	=
	\hat{H}(t)
	\hat{U}(t,t_0)
\end{equation}
which formal solution is the time-ordered exponential, see Ref.~\cite[p.~380]{Bartelmann2018},
\begin{equation}
	\hat{U}(t,t_0)
	=
	T\exp\left\{
		-i
		\int_{t_0}^t\dd{t^\prime}
		\hat{H}(t^\prime)
	\right\}
\end{equation}
where $T$ denotes the time-ordering symbol.
Only for simple time-dependent systems an exact time-evolution operator exists.
In contrast to the Dyson expansion, the Magnus expansion yields a unitary time-evolution operator even for finite order, in particular,
\begin{equation}
	\hat{U}(t,t_0)
	=
	\exp\left\{
		\sum_{n=1}
		\hat{\Omega}^{(n)}(t,t_0)
	\right\}
\end{equation}
where the first two expansion terms are given by
\begin{align}
	\hat{\Omega}^{(1)}(t,t_0)
	&=
	\frac{(-i)}{1!}
	\int_{t_0}^t\dd{t^\prime}
	\hat{H}(t^\prime)
	\\
	\hat{\Omega}^{(2)}(t,t_0)
	&=
	\frac{(-i)^2}{2!}
	\int_{t_0}^t\dd{t^\prime}
	\int_{t_0}^{t^\prime}\dd{t^{\prime\prime}}
	\comm{\hat{H}(t^\prime)}{\hat{H}(t^\prime)}
\end{align}
and represent time-ordering corrections, see Ref.~\cite{QuesadaMejia2015}.

\subsection{Interaction with classical current}

The Schrödinger-picture Hamiltonian describing the interaction of the Maxwell field $\hat{\vb{A}}$ with a classical current $\vb{j}$ is
\begin{equation}
	\hat{H}_\text{int}(t)
	=
	-
	\vb{j}(t,\vb{x})
	\vdot
	\hat{\vb{A}}(t,\vb{x})
	.
\end{equation}
Inserting the mode expansion
		\section{Connection to quantum optics}

Here we reduce the results of the previous sections to a continuous-mode description of (linear polarized) light, i.e., frequency range limited to the optical domain.
Assumptions and limitations:
\begin{enumerate}
	\item Fixed reference frame (at rest), i.e., no Lorentz boosts allowed.
	\item Light is linearly polarized.
	\item \textcolor{red}{No distribution of the momentum}
\end{enumerate}

\begin{equation}
	\hat{N}
	=
	\int\dd{\omega}
	\hat{a}^\dagger(\omega)
	\hat{a}(\omega)
\end{equation}

\begin{equation}
	\hat{E}
	=
	i
	\int\dd{\omega}
	\omega
	\hat{a}^\dagger(\omega)
	\hat{a}(\omega)
\end{equation}

\begin{equation}
	\ket{f}
	=
	\int\dd{\omega}
	f(\omega)
	\hat{a}(\omega)
\end{equation}

\begin{equation}
	\ket{\alpha}
	=
	\exp\left\{
		-
		\frac{1}{2}
		\int\dd{\omega}
		\abs{\alpha(\omega)}^2
	\right\}
	\exp\left\{
		-
		\int\dd{\omega}
		\alpha(\omega)
		\hat{a}^\dagger(\omega)
	\right\}
	\ket{0}
\end{equation}

		\addcontentsline{toc}{section}{References}
		\printbibliography[title=References]
	\end{refsection}
	
	\chapter{Interaction theory of optical components}
	\begin{refsection}
		\section{Laser}

The Hamiltonian describing a laser is given by
\begin{equation}
	\hat{H}
	=
	\hat{H}_a
	+
	\hat{H}_m
	+
	\hat{H}_{ae}
	+
	\hat{H}_{am}
	+
	\hat{H}_{me}
\end{equation}
where we have the free Hamiltonian of the atoms inside the cavity
\begin{equation}
	\hat{H}_a
	=
	\sum_{n=1}^N
	\frac{1}{2}
	\omega
	\hat\sigma_{z,n}
\end{equation}
which are approximated as independent two-level spin-like system and the free photon field
\begin{equation}
	\hat{H}_m
	=
	\int\dd{\omega}
	\omega
	\hat{a}^\dagger(\omega)
	\hat{a}(\omega)
\end{equation}
The environment is coupled to the photons
\begin{equation}
	\hat{H}_{me}
	=
	\Gamma_p
	\left(
		\hat{a}(\omega)
		+
		\hat{a}^\dagger(\omega)
	\right)
\end{equation}
and the atoms
\begin{equation}
	\hat{H}_{ae}
	=
	\sum_{n=1}^N
	\Gamma_{a,n}
	\left(
		\hat\sigma_{+,n}
		+
		\hat\sigma_{-,n}
	\right)
\end{equation}
The atoms are coupled with the photon field with
\begin{equation}
	\hat{H}_{am}
	=
	ig
	\sum_{n=1}^N
	\left(
		\hat{a}^\dagger(\omega)
		\hat\sigma_{-,n}
		-
		\hat{a}(\omega)
		\hat\sigma_{-,n}^\dagger
	\right)
\end{equation}
		\section{Coupler}

Optical couplers, including optical splitters, redistribute two optical inputs among two outputs and are an essential passive component for almost every setup.\footnote{The optical splitter is a special case of the optical coupler where one of the two optical inputs is zero, or, more precisely, the vacuum state.}

A plethora of approaches towards the (quantum) beam splitter exists~\cite{Leonhardt2010,Gerry2005,Loudon2000} but are vague on assumptions, which, if not discussed, lead to misconception and confusion, for instance, regarding the phase and energy conservation.
We would therefore approach the optical coupler, or beam splitter, from two directions:
First, we approach the free-ray beam splitter from an experimentalist's perspective, considering the reflection and transmission properties of a beam splitter.
Second, we discuss the fiber coupler from a theoretical mode-coupling perspective leading to an interaction Hamiltonian.
Of course, both paths lead to the same unitary transformations, which we present in matrix and operator form.
Finally, we discuss the input-output relations for coherent states and the interpretation of a frequency-dependent beam splitter as an optical filter.

\subsection{Beam splitter}

The most commonly employed (free-ray) designs of the beam splitter are the cubic, plate, and pellicle beam splitters, see \Cref{fig:beam_splitter_types}.
\begin{figure}[htb]
    \centering
    \includegraphics{figures/tikz/beam-splitter-types}
    \caption{Different types of free-ray beam splitters: (a) Cubic beam splitter made of two triangular prisms glued at their base (grey). (b) Plate beam splitter made of a dielectric plate. (c) Pellicle beam splitter made from a thin membrane.}\label{fig:beam_splitter_types}
\end{figure}
The cubic beam splitter is made of two triangular prisms.
The interface between the two prisms is finished with a dielectric coating.
The outward-facing surface of the prisms is grafted with an \gls{ar} coating.\footnote{The incident angle of the electric field is perpendicular to the surface of the cubic beam splitter. As the reflection angle is equal to the incidence angle, we have back-reflection of the input fields.}
he pellicle beam splitter consists of a few micrometer thin membrane, optionally with a one-sided coating.
The plate beam splitter is like a thick pellicle beam splitter made of glass.

To deduce the relation between the in- and output fields, we sequentially couple a laser pulse into each input while monitoring both outputs with a spectrum analyzer, see \Cref{fig:beam_splitter_inputs_outputs}.
Assuming the beam splitter to be an \gls{lti} system, knowing the spectral shape of the laser pulse lets us infer the frequency responses of the beam splitter.
\begin{figure}[htb]
    \centering
    \includegraphics{figures/tikz/beam-splitter-cubic-plate}
    \caption{Cubic (left) and plate beam splitter (right) with the two input fields, $\hat{E}_1(\omega)$ and $\hat{E}_2(\omega)$, and two output fields, $\hat{E}_1^\prime(\omega)$ and $\hat{E}_2^\prime(\omega)$, labelled by the momentum representation of the electric field operators.}\label{fig:beam_splitter_inputs_outputs}
\end{figure}
Invoking the superposition principle for electromagnetic waves, we find the frequency responses of the beam splitter to relate the electric fields by
\begin{equation}
    \begin{pmatrix}
        \expval{\hat{E}_1^\prime(\omega)} \\
        \expval{\hat{E}_2^\prime(\omega)}
    \end{pmatrix}
    =
    \begin{pmatrix}
        t(\omega) & r^\prime(\omega)
        \\
        r(\omega) & t^\prime(\omega)
    \end{pmatrix}
    \begin{pmatrix}
		\expval{\hat{E}_1(\omega)} \\
        \expval{\hat{E}_2(\omega)}
    \end{pmatrix}
    \label{eq:beam_splitter_expval}
\end{equation}
wherein $r(\omega),r^\prime(\omega)$ and $t(\omega),t^\prime(\omega)$ are the complex reflection respective transmission coefficients.
The absolute values of the transmission, $\abs{t(\omega)}$ and $\abs{t^\prime(\omega)}$, and reflection coefficients, $\abs{r(\omega)}$ and $\abs{r^\prime(\omega)}$, determine the splitting ratio of the input power among the outputs.
The complex phase factor of the reflection and transmission coefficients characterizes the phase shifts the output fields concerning the input fields.
The beam splitter is a passive device implying the output energy to be bound by the input energy
\begin{equation}
    \abs{\expval{\hat{E}_1^\prime(\omega)}}^2
    +
    \abs{\expval{\hat{E}_2^\prime(\omega)}}^2
    \leq
    \abs{\expval{\hat{E}_1(\omega)}}^2
    +
    \abs{\expval{\hat{E}_2(\omega)}}^2
    \label{eq:beam_splitter_passive}
    ,
\end{equation}
or equivalently, constraining the reflection and transmission coefficients by
\begin{align}
    \abs{r(\omega)}^2+\abs{t(\omega)}^2
    &\leq
    1,
    &
    \abs{r^\prime(\omega)}^2+\abs{t^\prime(\omega)}^2
    &\leq
    1
    \label{eq:beam_splitter_coefficients_constraint}
    .
\end{align}
The equality of these inequalities is only true for lossless devices for which there is no back-scattering.\footnote{Using an optical circulator it is in principle possible to measure all \num{16} scattering parameters.}
Sometimes, one finds the claim~\cite[p.~129]{Haroche2006} that the matrix transformation in \cref{eq:beam_splitter_expval} is required to be symmetric (or reciprocal) due to Maxwell's equations.
However, only optical systems with a single dielectric layer are reciprocal~\cite{Potton2004}, but most physical beam splitters comprise multiple dielectric layers.\footnote{For example, cubic beam splitters typically have a coating followed by optical cement between the prisms breaking reciprocal symmetry of the system.}
It is possible to derive exact expressions of the complex reflection, $r(\omega),r^\prime(\omega)$, and transmission coefficients, $t(\omega),t^\prime(\omega)$ using classical wave optics and perfect knowledge of the dimensions and material properties.
For example, Hénault~\cite{Henault2015} derived an exact expression for the reflected and transmitted amplitudes of a plate beam splitter with one input and a single dielectric layer.
Likewise, Hamilton~\cite{Hamilton2000} discusses the cubic beam splitter with two inputs and different coatings.
In general, the complex reflection and transmission coefficients need to account for multiple reflections at different dielectric layers inside the beam splitter.

Inserting the mode expansion of the electric field operators, \cref{eq:electric_operator,eq:electric_negative_operator,eq:electric_positive_operator}, and using the linearity of the device and the expectation value, we recover the transformation for the annihilation operators, sometimes referred to as quantum modes,
\begin{equation}
    \begin{pmatrix}
        \hat{a}_1^\prime(\omega) \\
        \hat{a}_2^\prime(\omega)
    \end{pmatrix}
    =
    \begin{pmatrix}
        t(\omega) & r^\prime(\omega)
        \\
        r(\omega) & t^\prime(\omega)
    \end{pmatrix}
    \begin{pmatrix}
        \hat{a}_1(\omega) \\
        \hat{a}_2(\omega)
    \end{pmatrix}
    \label{eq:beam_splitter_annihilation}
\end{equation}
in agreement with Refs.~\cite{Leonhardt2010,Gerry2005}.

\subsection{Mode coupler}

Contrary to the direct coupling in free-ray beam splitters, a fiber or waveguide coupler uses indirect coupling through the evanescent field.
The evanescent field of an electromagnetic field does not propagate but decays exponentially.
We often observe evanescent fields at the boundary of waveguiding structures.
One must bring the waveguides in proximity for the evanescent fields of two waveguided modes to couple efficiently.
The range where the waveguides are close is the interaction length $l$, see \Cref{fig:waveguide_coupler}.
Over the interaction length, the two energy of the field modes oscillates back and forth between the two waveguides.
\begin{figure}[htb]
    \centering
    \includegraphics{figures/tikz/waveguide-coupler}
    \caption{Waveguide coupler with input quantum modes $\hat{a}_1(\omega)$ and $\hat{a}_2(\omega)$ coupled evanescent over an interaction length $l$ yielding the output quantum modes $\hat{a}_1^\prime(\omega)$ and $\hat{a}_2^\prime(\omega)$.}\label{fig:waveguide_coupler}
\end{figure}
The weak coupling through evanescent fields is conceptionally similar to weakly coupled harmonic oscillators.
Haroché~\cite[p.~131]{Haroche2006} successfully exploits the analogy to derive the quantum beam splitter transform from interaction theory.
We generalize his approach for the mode continuum derived in the previous chapter.

Let $\hat{a}_1(\omega)$ and $\hat{a}_2(\omega)$ be the annihilation operators of the first and second waveguide modes.
The interaction Hamiltonian
\begin{equation}
	\hat{H}_\text{int}
	=
	-
	\int\frac{\dd{\omega}}{2\pi}
	\left\{
		g(\omega)
		\hat{a}_1(\omega)
		\hat{a}_2^\dagger(\omega)
		+
		g^*(\omega)
		\hat{a}_1^\dagger(\omega)
		\hat{a}_2(\omega)
	\right\}
	,
\end{equation}
wherein $g(\omega)$ is a complex-valued coupling parameter encoding the material and geometry of the coupler, describes the transitions of excitations between the first and the second mode.
As the interaction Hamiltonian is time-independent, all but the first term in the Magnus expansion vanish, and the time evolution operator is
\begin{equation}
	\hat{U}_\text{int}
	=
	\exp\left\{
		i
		\int\dd{t^\prime}
		\int\frac{\dd{\omega}}{2\pi}
		\left\{
			g(\omega)
			\hat{a}_1(\omega)
			\hat{a}_2^\dagger(\omega)
			+
			g^*(\omega)
			\hat{a}_1^\dagger(\omega)
			\hat{a}_2(\omega)
		\right\}
	\right\}
\end{equation}
wherein the time integration is over the duration of the interaction.
Assuming the interaction to be limited to the interaction length $l$, the interaction duration $T$ is approximately equal to the interaction length $l$ divided by the group velocity $v_g(\omega)$.
The group velocity depends on the materials of the coupler, suggesting redefining the coupling parameter to include the different interaction durations, i.e.,
\begin{equation}
	\hat{U}_\text{int}
	=
	\exp\left\{
		i
		\int\frac{\dd{\omega}}{2\pi}
		\theta(\omega)
		\left\{
			\hat{a}_1(\omega)
			\hat{a}_2^\dagger(\omega)
			e^{-i\varphi(\omega)}
			+
			\hat{a}_1^\dagger(\omega)
			\hat{a}_2(\omega)
			e^{+i\varphi(\omega)}
		\right\}
	\right\}
\end{equation}
where the real-valued couplings $\theta(\omega)$ and $\varphi(\omega)$ implicitly depend on the materials and geometry of the waveguide coupler and the interaction length $l$.
We define the generator
\begin{equation}
	\hat{G}
	=
	-i
	\int\frac{\dd{\omega}}{2\pi}
	\theta(\omega)
	\left\{
		\hat{a}_1(\omega)
		\hat{a}_2^\dagger(\omega)
		e^{-i\varphi(\omega)}
		+
		\hat{a}_1^\dagger(\omega)
		\hat{a}_2(\omega)
		e^{+i\varphi(\omega)}
	\right\}
\end{equation}
and calculate the commutator of the generator with the annihilation operators
\begin{align}
	\comm{\hat{G}}{\hat{a}_1(\omega)}
	&=
	i
	\theta(\omega)
	\hat{a}_2(\omega)
	e^{+i\varphi(\omega)}
	&
	\comm{\hat{G}}{\hat{a}_2(\omega)}
	&=
	i
	\theta(\omega)
	\hat{a}_1(\omega)
	e^{-i\varphi(\omega)}
	.
\end{align}
The transformed annihilation operators turn out to be\footnote{Strictly speaking, the annihilation operators in the interaction picture have an additional factor $e^{-i\omega t}$.},
\begin{equation}
	\begin{split}
		\hat{a}_1^\prime(\omega)
		&=
		\hat{U}_\text{int}^\dagger
		\hat{a}_1(\omega)
		\hat{U}_\text{int}
		=
		e^{+\hat{G}}
		\hat{a}_1(\omega)
		e^{-\hat{G}}
		\\
		&=
		\hat{a}_1
		+
		\comm{\hat{G}}{\hat{a}_1}
		+
		\frac{1}{2!}
		\comm{\hat{G}}{\comm{\hat{G}}{\hat{a}_1}}
		+
		\frac{1}{3!}
		\comm{\hat{G}}{\comm{\hat{G}}{\comm{\hat{G}}{\hat{a}_1}}}
		+
		\dots
		\\
		&=
		\hat{a}_1(\omega)
		+
		i\theta(\omega)
		\hat{a}_2(\omega)
		e^{+i\varphi(\omega)}
		+
		\frac{1}{2!}
		\left(i\theta(\omega)\right)^2
		\hat{a}_1(\omega)
		+
		\frac{1}{3!}
		\left(i\theta(\omega)\right)^3
		\hat{a}_2(\omega)
		e^{+i\varphi(\omega)}
		+
		\dots
		\\
		&=
		\cos\theta(\omega)
		\hat{a}_1(\omega)
		+
		i\sin\theta(\omega)
		\hat{a}_2(\omega)
		e^{+i\varphi(\omega)}
	\end{split}
\end{equation}
and
\begin{equation}
	\begin{split}
		\hat{a}_2^\prime(\omega)
		&=
		\hat{U}_\text{int}^\dagger
		\hat{a}_2(\omega)
		\hat{U}_\text{int}
		=
		e^{+\hat{G}}
		\hat{a}_2(\omega)
		e^{-\hat{G}}
		\\
		&=
		\hat{a}_2
		+
		\comm{\hat{G}}{\hat{a}_2}
		+
		\frac{1}{2!}
		\comm{\hat{G}}{\comm{\hat{G}}{\hat{a}_2}}
		+
		\frac{1}{3!}
		\comm{\hat{G}}{\comm{\hat{G}}{\comm{\hat{G}}{\hat{a}_2}}}
		+
		\dots
		\\
		&=
		\hat{a}_2(\omega)
		+
		i\theta(\omega)
		\hat{a}_1(\omega)
		e^{-i\varphi(\omega)}
		+
		\frac{1}{2!}
		\left(i\theta(\omega)\right)^2
		\hat{a}_2(\omega)
		+
		\frac{1}{3!}
		\left(i\theta(\omega)\right)^3
		\hat{a}_1(\omega)
		e^{-i\varphi(\omega)}
		+
		\dots
		\\
		&=
		\cos\theta(\omega)
		\hat{a}_2(\omega)
		+
		i\sin\theta(\omega)
		\hat{a}_1(\omega)
		e^{-i\varphi(\omega)}
		,
	\end{split}
\end{equation}
where we used a kind of \gls{bch} formula, in agreement with Ref.~\cite[p.~131]{Haroche2006}.
In matrix notation, the transformation of the annihilation operators reads
\begin{equation}
	\begin{pmatrix}
        \hat{a}_1^\prime(\omega) \\
        \hat{a}_2^\prime(\omega)
    \end{pmatrix}
    =
    U(\omega)
    \begin{pmatrix}
        \hat{a}_1(\omega) \\
        \hat{a}_2(\omega)
    \end{pmatrix}
    =
    \begin{pmatrix}
        \cos\theta(\omega) & i\sin\theta(\omega)e^{+i\varphi} 
        \\
        i\sin\theta(\omega)e^{-i\varphi} & \cos\theta(\omega)
    \end{pmatrix}
    \begin{pmatrix}
        \hat{a}_1(\omega) \\
        \hat{a}_2(\omega)
    \end{pmatrix}
    \label{eq:waveguide_coupler_transformation}
    .
\end{equation}
Comparison of the annihilation operator transformation for the waveguide coupler, \cref{eq:waveguide_coupler_transformation}, and the beam splitter, \cref{eq:beam_splitter_annihilation}, our waveguide result implies lossless coupling.
Lossless coupling is essential for the transformed annihilation operators to satisfy the \gls{ccr}~\cite[p.~38]{Gerry2005}.
Modeling an absorbing coupler requires four quantum modes, two annihilation operators for the field, and two for a bosonic reservoir, see Ref.~\cite[p.~210]{Vogel2006} for details.

\subsection{Unitary operator transform}

The derived transforms of the free-ray beam splitter and the fiber or waveguide coupler, \cref{eq:waveguide_coupler_transformation,eq:beam_splitter_annihilation}, have in common that they are two-dimensional unitary matrices, which is not surprising since a unitary matrix transform conserves energy.
The optical coupler transform being linear and unitary is not surprising since the coupler is a linear passive device, which we further assumed to be lossless.
It presents itself to take the unitary matrix transform as the defining property of an ideal optical coupler.

A general decomposition of a two-dimensional unitary matrix is the product~\cite[p.~95]{Leonhardt2010}
\begin{equation}
	U(\omega)
	=
	e^{i\Lambda/2}
	\begin{pmatrix}
		e^{+i\Phi/2} & 0 \\
		0 & e^{-i\Phi/2}
	\end{pmatrix}
	\begin{pmatrix}
		\cos(\Theta/2) & \sin(\Theta/2) \\
		-\sin(\Theta/2) & \cos(\Theta/2)
	\end{pmatrix}
	\begin{pmatrix}
		e^{+i\Psi/2} & 0 \\
		0 & e^{-i\Psi/2}
	\end{pmatrix}
	\label{eq:unitary_matrix}
\end{equation}
wherein we suppress the frequency-dependence of the real parameters, $\Lambda(\omega),\Theta(\omega),\Psi(\omega),\Phi(\omega)$, for clarity.
We can read the matrix decomposition, \cref{eq:unitary_matrix}, as first adding a global and relative phase shift, $\Lambda/2,\pm\Psi/2$, to the incident fields, rotating (mixing) the field amplitudes by the angle $\Theta/2$, and adding another relative phase shift of $\pm\Psi/2$ to the outgoing fields.

While the unitary matrix transforms the annihilation operators and the field amplitudes, it cannot transform a generic quantum state.
In the previous subsection, we found a time evolution operator $\hat{U}$ from linear mode coupling theory, which related to the unitary matrix transform $U$ via
\begin{equation}
	U(\omega)
	\begin{pmatrix}
		\hat{a}_1(\omega) \\
		\hat{a}_2(\omega)
	\end{pmatrix}
	=
	\begin{pmatrix}
		\hat{a}_1^\prime(\omega) \\
		\hat{a}_2^\prime(\omega)
	\end{pmatrix}
	=
	\begin{pmatrix}
		\hat{U}^\dagger\hat{a}_1(\omega)\hat{U} \\
		\hat{U}^\dagger\hat{a}_2(\omega)\hat{U}
	\end{pmatrix}
	=
	\hat{U}^\dagger
	\begin{pmatrix}
		\hat{a}_1(\omega) \\
		\hat{a}_2(\omega)
	\end{pmatrix}
	\hat{U}
	\label{eq:unitary_matrix_operator}
	.
\end{equation}
The unitary operators corresponding to the unitary matrix decomposition in \cref{eq:unitary_matrix} are the Jordan-Schwinger operators\footnote{Generalized to a frequency continuum from Ref.~\cite[p.~97]{Leonhardt2010}.}
\begin{align}
	\hat{L}_t
	&=
	\frac{1}{2}
	\int\frac{\dd{\omega}}{2\pi}
	\begin{pmatrix}
		\hat{a}_1(\omega) \\
		\hat{a}_2(\omega)
	\end{pmatrix}^\dagger
	\mathbb{1}_2
	\begin{pmatrix}
		\hat{a}_1(\omega) \\
		\hat{a}_2(\omega)
	\end{pmatrix}
	=
	\frac{1}{2}
	\int\frac{\dd{\omega}}{2\pi}
	\left(
		\hat{a}_1^\dagger(\omega)
		\hat{a}_1(\omega)
		+
		\hat{a}_2^\dagger(\omega)
		\hat{a}_2(\omega)
	\right)
	\\
	\hat{L}_x
	&=
	\frac{1}{2}
	\int\frac{\dd{\omega}}{2\pi}
	\begin{pmatrix}
		\hat{a}_1(\omega) \\
		\hat{a}_2(\omega)
	\end{pmatrix}^\dagger
	\sigma_x
	\begin{pmatrix}
		\hat{a}_1(\omega) \\
		\hat{a}_2(\omega)
	\end{pmatrix}
	=
	\frac{1}{2}
	\int\frac{\dd{\omega}}{2\pi}
	\left(
		\hat{a}_1^\dagger(\omega)
		\hat{a}_2(\omega)
		+
		\hat{a}_2^\dagger(\omega)
		\hat{a}_1(\omega)
	\right)
	\\
	\hat{L}_y
	&=
	\frac{1}{2}
	\int\frac{\dd{\omega}}{2\pi}
	\begin{pmatrix}
		\hat{a}_1(\omega) \\
		\hat{a}_2(\omega)
	\end{pmatrix}^\dagger
	\sigma_y
	\begin{pmatrix}
		\hat{a}_1(\omega) \\
		\hat{a}_2(\omega)
	\end{pmatrix}
	=
	\frac{i}{2}
	\int\frac{\dd{\omega}}{2\pi}
	\left(
		\hat{a}_2^\dagger(\omega)
		\hat{a}_1(\omega)
		-
		\hat{a}_1^\dagger(\omega)
		\hat{a}_2(\omega)
	\right)
	\\
	\hat{L}_z
	&=
	\frac{1}{2}
	\int\frac{\dd{\omega}}{2\pi}
	\begin{pmatrix}
		\hat{a}_1(\omega) \\
		\hat{a}_2(\omega)
	\end{pmatrix}^\dagger
	\sigma_z
	\begin{pmatrix}
		\hat{a}_1(\omega) \\
		\hat{a}_2(\omega)
	\end{pmatrix}
	=
	\frac{1}{2}
	\int\frac{\dd{\omega}}{2\pi}
	\left(
		\hat{a}_1^\dagger(\omega)
		\hat{a}_1(\omega)
		-
		\hat{a}_2^\dagger(\omega)
		\hat{a}_2(\omega)
	\right)
\end{align}
where $\sigma_1,\sigma_2,\sigma_3$ denote the two-dimensional Pauli matrices.
The Jordan-Schwinger operators satisfy the angular-momentum commutation algebra~\cite[p.~97]{Leonhardt2010}
\begin{align}
	\comm{\hat{L}_i}{\hat{L}_j}
	&=
	i\varepsilon_{ijk}\hat{L}^k
	&
	\comm{\hat{L}_t}{\hat{L}_i}
	&=
	0
\end{align}
and act as generator for the individual components of the matrix decomposition in \cref{eq:unitary_matrix}.
The generator of the unitary matrix, \cref{eq:unitary_matrix}, is
\begin{equation}
	\hat{U}
	=
	e^{i\Lambda\hat{L}_t}
	e^{i\Phi\hat{L}_z}
	e^{i\Theta\hat{L}_y}
	e^{i\Psi\hat{L}_z}
	\label{eq:unitary_operator}
	.
\end{equation}
The inverse of the unitary operator, \cref{eq:unitary_operator}, can be written
\begin{equation}
	\begin{split}
		\hat{U}(\Lambda,\Phi,\Theta,\Psi)^\dagger
		&=
		e^{-i\Psi\hat{L}_z}
		e^{-i\Theta\hat{L}_y}
		e^{-i\Phi\hat{L}_z}
		e^{-i\Lambda\hat{L}_t}
		\\
		&=
		e^{-i\Lambda\hat{L}_t}
		e^{-i\Psi\hat{L}_z}
		e^{-i\Theta\hat{L}_y}
		e^{-i\Phi\hat{L}_z}
		\\
		&=
		\hat{U}(-\Lambda,-\Psi,-\Theta,-\Phi)
		,
	\end{split}
	\label{eq:unitary_operator_inverse}
\end{equation}
where we used that $\hat{L}_t$ commutes with the other Jordan-Schwinger operators, and can be used to find the reversed transform,
\begin{equation}
	\hat{U}
	\begin{pmatrix}
		\hat{a}_1(\omega) \\
		\hat{a}_2(\omega)
	\end{pmatrix}
	\hat{U}^\dagger
	=
	U(\omega)^\dagger
	\begin{pmatrix}
		\hat{a}_1(\omega) \\
		\hat{a}_2(\omega)
	\end{pmatrix}
	\label{eq:unitary_matrix_operator_reverse}
	,
\end{equation}
 of the annihilation operators.

\subsection{Coherent state transform}

Let us now consider the action of the ideal coupler on the tensor product of coherent input states\footnote{Other quantum states typically produce entangled output states, see, for instance, Ref.~\cite{Windhager2011}, which is not of interest here.}
\begin{equation}
	\ket*{\vb{\alpha}(t)}
	=
	\ket*{\alpha_1(t),\alpha_2(t)}
	.
\end{equation}
The output states are given by applying the unitary evolution operator $\hat{U}$, e.g., \cref{eq:unitary_operator}, onto the input state
\begin{equation}
	\hat{U}
	\ket*{\vb{\alpha}(t)}
	=
	\hat{U}
	\hat{D}\left[\vb{\alpha}(t)\right]
	\hat{U}^\dagger
	\hat{U}
	\ket*{0,0}
	=
	\hat{U}
	\hat{D}\left[\vb{\alpha}(t)\right]
	\hat{U}^\dagger
	\vacuum
	,
\end{equation}
wherein we inserted $\mathbb{1}=\hat{U}^\dagger\hat{U}$ in the second step and we used the invariance of the vacuum state in the third step.
The transformed displacement operator reads\footnote{We adopt matrix notation as in Ref.~\cite[p.~206]{Vogel2006} to have our result independent of a particular choice of the unitary matrix.}
\begin{equation}
	\begin{split}
		\hat{U}
		\hat{D}\left[\vb{\alpha}(t)\right]
		\hat{U}^\dagger
		&=
		\hat{U}
		\exp\left\{
			\int\frac{\dd{\omega}}{2\pi}
			\left\{
				\vb{\alpha}(\omega)^\trans
				e^{-i\omega t}
				\begin{pmatrix}
					\hat{a}_1^\dagger(\omega) \\
					\hat{a}_2^\dagger(\omega)
				\end{pmatrix}
				-
				\vb{\alpha}(\omega)^\dagger
				e^{+i\omega t}
				\begin{pmatrix}
					\hat{a}_1(\omega) \\
					\hat{a}_2(\omega)
				\end{pmatrix}
			\right\}
		\right\}
		\hat{U}^\dagger
		\\
		&=
		\exp\left\{
			\int\frac{\dd{\omega}}{2\pi}
			\left\{
				\vb{\alpha}^\trans
				e^{-i\omega t}
				\hat{U}
				\begin{pmatrix}
					\hat{a}_1^\dagger(\omega) \\
					\hat{a}_2^\dagger(\omega)
				\end{pmatrix}
				\hat{U}^\dagger
				-
				\vb{\alpha}^\dagger
				e^{+i\omega t}
				\hat{U}
				\begin{pmatrix}
					\hat{a}_1(\omega) \\
					\hat{a}_2(\omega)
				\end{pmatrix}
				\hat{U}^\dagger
			\right\}
		\right\}
	\end{split}
	\label{eq:displacement_operator_transformed}
	,
\end{equation}
wherein we used the operator identity
\begin{equation}
	\hat{U}
	e^{\hat{A}}
	\hat{U}^\dagger
	=
	\sum_{n=0}^\infty
	\frac{1}{n!}
	\hat{U}
	\hat{A}^n
	\hat{U}^\dagger
	=
	\sum_{n=0}^\infty
	\frac{1}{n!}
	\hat{U}
	\hat{A}
	\hat{U}^\dagger
	\cdots
	\hat{U}
	\hat{A}
	\hat{U}^\dagger
	=
	\sum_{n=0}^\infty
	\frac{1}{n!}
	\left(
		\hat{U}
		\hat{A}
		\hat{U}^\dagger
	\right)^n
	=
	e^{\hat{U}\hat{A}\hat{U}^\dagger}
\end{equation}
in the second step to move the unitary operators into the argument of the exponential.
We already expressed the transformed annihilation operators in the second term of the exponential using the unitary matrix in \cref{eq:unitary_matrix_operator_reverse}.
The transformed creation operators in the first term can be brought into a similar form, i.e.,
\begin{equation}
	\begin{split}
		\hat{U}
		\begin{pmatrix}
			\hat{a}_1^\dagger(\omega) \\
			\hat{a}_2^\dagger(\omega)
		\end{pmatrix}
		\hat{U}^\dagger
		&=
		\begin{pmatrix}
			\left[
				\hat{U}
				\hat{a}_1(\omega)
				\hat{U}^\dagger
			\right]^\dagger \\
			\left[
				\hat{U}
				\hat{a}_2(\omega)
				\hat{U}^\dagger
			\right]^\dagger
		\end{pmatrix}
		=
		\left[
			\begin{pmatrix}
				\hat{U}
				\hat{a}_1(\omega)
				\hat{U}^\dagger
				\\
				\hat{U}
				\hat{a}_2(\omega)
				\hat{U}^\dagger
			\end{pmatrix}^\dagger
		\right]^\trans
		=
		\left[
			\left(
				U(\omega)
				\begin{pmatrix}
					\hat{a}_1(\omega) \\
					\hat{a}_2(\omega)
				\end{pmatrix}
			\right)^\dagger
		\right]^\trans		
		\\
		&=
		\left[
			\begin{pmatrix}
				\hat{a}_1(\omega)
				\\
				\hat{a}_2(\omega)
			\end{pmatrix}^\dagger
			U(\omega)
		\right]^\trans
		=
		U(\omega)^\trans
		\begin{pmatrix}
			\hat{a}_1^\dagger(\omega) \\
			\hat{a}_2^\dagger(\omega)
		\end{pmatrix}
		.
	\end{split}
\end{equation}
Inserting the previous result back into the transformed displacement operator, \cref{eq:displacement_operator_transformed}, we factor the unitary matrix to the Fourier amplitudes
\begin{equation}
	\begin{split}
		\hat{D}^\prime\left[\vb{\alpha}(t)\right]
		&=
		\exp\left\{
			\int\frac{\dd{\omega}}{2\pi}
			\left\{
				\vb{\alpha}(\omega)^\trans
				e^{-i\omega t}
				U(\omega)^\trans
				\begin{pmatrix}
					\hat{a}_1^\dagger(\omega) \\
					\hat{a}_2^\dagger(\omega)
				\end{pmatrix}
				-
				\vb{\alpha}(\omega)^\dagger
				e^{+i\omega t}
				U(\omega)^\dagger
				\begin{pmatrix}
					\hat{a}_1(\omega) \\
					\hat{a}_2(\omega)
				\end{pmatrix}
			\right\}
		\right\}
		\\
		&=
		\exp\left\{
			\int\frac{\dd{\omega}}{2\pi}
			\left\{
				\left(
					U(\omega)
					\vb{\alpha}(\omega)
				\right)^\trans
				e^{-i\omega t}
				\begin{pmatrix}
					\hat{a}_1^\dagger(\omega) \\
					\hat{a}_2^\dagger(\omega)
				\end{pmatrix}
				-
				\left(
					U(\omega)
					\vb{\alpha}(\omega)
				\right)^\dagger
				e^{+i\omega t}
				\begin{pmatrix}
					\hat{a}_1(\omega) \\
					\hat{a}_2(\omega)
				\end{pmatrix}
			\right\}
		\right\}
	\end{split}
\end{equation}
in agreement with Ref.~\cite[p.~210]{Vogel2006}.
The transformed Fourier amplitudes are given by the matrix product
\begin{equation}
	\vb{\alpha}^\prime(\omega)
	=
	U(\omega)
	\vb{\alpha}(\omega)
	\label{eq:coupler_coherent_amplitudes_frequency}
	.
\end{equation}
The product in frequency space implies a convolution in the time domain, i.e.,
\begin{equation}
	\vb{\alpha}^\prime(t)
	=
	\left(U\conv\vb{\alpha}\right)(t)
	=
	\int\frac{\dd{\omega}}{2\pi}
	U(\omega)
	\vb{\alpha}(\omega)
	e^{+i\omega t}
	\label{eq:coupler_coherent_amplitudes_time}
	,
\end{equation}
and we conclude that a tensor product of coherent states transforms under an ideal optical coupler according to
\begin{align}
	\hat{U}
	\ket*{\alpha(t)}
	&=
	\ket*{\left(U\conv\vb{\alpha}\right)(t)}	
	&
	U(t)
	&=
	\int\frac{\dd{\omega}}{2\pi}
	U(\omega)
	e^{+i\omega t}
	\label{eq:coupler_coherent_state}
\end{align}
wherein $U(\omega)$ is a two-dimensional unitary matrix characterizing the coupler.

The fact that the ideal optical coupler only transforms the amplitudes of the coherent input states is specific to coherent states.
The coherent states owe this special property due to having a Poisson number distribution and the Poisson distribution being memoryless.
As a consequence, the coherent output states are independent and consider a subsystem by performing a partial trace, e.g.,
\begin{equation}
	\trace_2\left\{
		\ketbra{\alpha,\beta}
	\right\}
	=
	\trace_2\left\{
		\ketbra{\alpha}
		\otimes
		\ketbra{\beta}
	\right\}
	=
	\ketbra{\alpha}
	\otimes
	\trace_2\left\{
		\ketbra{\beta}
	\right\}
	=
	\ketbra{\alpha}
	,
\end{equation}
where we used
\begin{equation}
	\trace_2\left\{
		\ketbra{\beta}
	\right\}
	=
	\sum_{n=0}^\infty
	\braket{n}{\beta}
	\braket{\beta}{n}
	=
	\sum_{n=0}^\infty
	\abs{\braket{n}{\beta}}^2
	=
	1
	,
\end{equation}
is equivalent to the projection of the subsystem
\begin{equation}
	\trace_2\left\{
		\ketbra{\alpha,\beta}
	\right\}
	=
	\ketbra{\alpha}
	=
	\hat{P}_1
	\ketbra{\alpha,\beta}
	\hat{P}_1
	.
\end{equation}
For non-coherent quantum states, it is not correct to project out a subsystem as the partial trace does not generally yield a mixed but a pure state.

\subsection{Spectral filter}

Our considerations have so far been quite general except for restricting ourselves to coherent input states.
We will now discuss two applications of our results:
First, we consider the special case of an optical being used as a splitter. 
Second, we consider a coupler as an optical filter relevant to signal processing and quantum information theory.

An ideal optical splitter redistributes the power of one input among two outputs and is a special case of the optical coupler with one input state being the vacuum state.
Using \cref{eq:coupler_coherent_amplitudes_frequency}, we find the transformed Fourier amplitudes to be
\begin{equation}
	\begin{split}
		\begin{pmatrix}
			\alpha_1^\prime(\omega) \\
			\alpha_2^\prime(\omega)
		\end{pmatrix}
		&=
		e^{i\Lambda/2}
		\begin{pmatrix}
			\cos(\Theta/2)
			e^{i\left(+\Phi+\Psi\right)/2}
			&
			\sin(\Theta/2)
			e^{i\left(+\Phi-\Psi\right)/2}
			\\
			-
			\sin(\Theta/2)
			e^{i\left(-\Phi+\Psi\right)/2}
			&
			\cos(\Theta/2)
			e^{i\left(-\Phi-\Psi\right)/2}
		\end{pmatrix}
		\begin{pmatrix}
			\alpha(\omega) \\
			0
		\end{pmatrix}
		\\
		&=
		\alpha(\omega)
		\begin{pmatrix}
			+
			\cos(\Theta/2)
			e^{+i\Phi/2}
			\\
			-
			\sin(\Theta/2)
			e^{-i\Phi/2}
		\end{pmatrix}
		e^{i(\Lambda+\Psi)/2}
	\end{split}
\end{equation}
wherein we again suppressed the frequency dependency of the splitting parameters.
Instead of choosing a parametrization for the splitting coefficients, which directly ensures energy conservation, we can more generally write
\begin{align}
	\begin{pmatrix}
		\alpha_1^\prime(\omega) \\
		\alpha_2^\prime(\omega)
	\end{pmatrix}
	&=
	\alpha(\omega)
	\begin{pmatrix}
		c_1(\omega) \\
		c_2(\omega)
	\end{pmatrix}
	&
	\abs{c_1(\omega)}^2
	+
	\abs{c_2(\omega)}^2
	&=
	1
	.
\end{align}
Assuming the Fourier transform of $c_1(\omega),c_2(\omega)$ to be well-defined, we find the coherent output states to be
\begin{equation}
	\hat{U}
	\ket*{\alpha(t),0}
	=
	\ket*{\left(c_1\conv\alpha\right)(t),\left(c_2\conv\alpha\right)(t)}
	\label{eq:splitter_state_convolution}
\end{equation}
according to \cref{eq:coupler_coherent_state}.
If we further assume the signal $\alpha(t)$ to be bandwidth-limited to $B$ and the splitting coefficients to be approximately constant over the bandwidth, i.e.,
\begin{align}
	c_1(\omega)
	&\approx
	c_1(\omega_0)
	&
	c_2(\omega)
	&\approx
	c_2(\omega_0)
	&
	\forall
	\omega
	&\in
	B
	,
\end{align}
the output state takes the simple form
\begin{equation}
	\hat{U}
	\ket*{\alpha(t),0}
	=
	\ket*{c_1\alpha(t),c_2\alpha(t)}
\end{equation}
and the splitter only redistributes the signal power among the outputs while leaving the signal itself unaltered.

\Cref{eq:splitter_state_convolution} already suggests the similarity of the ideal optical splitter with an optical filter, the difference being that only one output matters for the optical filter.
To remove the second output, we perform a partial trace over the second subsystem, equivalent to applying the projection operator to \cref{eq:splitter_state_convolution}, i.e.,
\begin{equation}
	\hat{P}_1
	\hat{U}
	\ket*{\alpha(t),0}
	=
	\ket*{\left(h\conv\alpha\right)(t)}
	\label{eq:filter_state}
\end{equation}
where we introduced the optical filter function $h=c_1$.
Ref.~\cite[p.~199]{Vogel2006} discusses a spectral filter made of a dielectric slab of thickness $l$ and refractive index $n$, see \Cref{fig:dielectric_filter}, with incoming and outgoing quantum modes to each side.
\begin{figure}[htb]
    \centering
    \includegraphics{figures/tikz/dielectric-filter}
    \caption{Dielectric slab of thickness $l$ and refractive index $n$ used as a spectral filter with incident quantum modes, denoted by the annihilation operators, $\hat{a}_1(\omega)$ from the left, and $\hat{a}_2(\omega)$ from the right, and outgoing quantum modes, $\hat{a}_1^\prime(\omega)$ to the left, and $\hat{a}_2(\omega)$ to the right.}\label{fig:dielectric_filter}
\end{figure}
Let $\hat{a}_1(\omega)$ be the signal mode approaching the dielectric slab from the left.
Let us assume that the mode $\hat{a}_2(\omega)$ is in vacuum and that we are only interested in the mode $\hat{a}_2^\prime(\omega)$, outgoing to the right.
Then the optical filter function in \cref{eq:filter_state} is equal to the transmission coefficient of the dielectric slab which is equal to~\cite[p.~199]{Vogel2006}
\begin{align}
	h(\omega)
	&=
	\frac{1-r^2}{1-r^2\exp(2i\omega nl)}
	\exp\left[-i(n-1)l\omega\right]
	&
	r^2
	&=
	\left(\frac{n-1}{n+1}\right)^2
	.
\end{align}
By carefully selecting the geometry and dielectric (layers), it should be possible to tailor the transmission coefficient in a specific bandwidth to implement a custom optical filter function $h$.
		\include{chapters/interaction-theory-of-optical-components/splitters-couplers}		
		\section{Mach-Zehnder modulator}
		\documentclass[tikz]{standalone}

\usepackage{amsmath}
\usepackage{unicode-math}
\usepackage{mathtools}
\usepackage{derivative}

\setmainfont{Stix Two Text}
\setmathfont{Stix Two Math}

\usetikzlibrary{arrows.meta,fit,positioning}

\renewcommand{\familydefault}{\sfdefault}

% prefix equation numbers with section number
\numberwithin{equation}{section}

\DeclarePairedDelimiter{\ceil}{\lceil}{\rceil}
\DeclarePairedDelimiter{\floor}{\lfloor}{\rfloor}
\DeclarePairedDelimiter{\abs}{\lvert}{\rvert}
\DeclarePairedDelimiter{\norm}{\lVert}{\rVert}
\DeclarePairedDelimiter{\bra}{\langle}{\rvert}
\DeclarePairedDelimiter{\ket}{\lvert}{\rangle}
\DeclarePairedDelimiter{\expval}{\langle}{\rangle}
\DeclarePairedDelimiter{\norder}{\mathcolon}{\mathcolon}
\DeclarePairedDelimiter{\anorder}{\typecolon}{\typecolon}
	
\newcommand{\laplace}{\mbfnabla^2}
\newcommand{\trans}{{\scriptscriptstyle\mathsf{T}}}

\newcommand{\vdot}{\cdot}
\newcommand{\vcross}{\vectimes}
\newcommand{\vb}[1]{\symbfup{#1}}
\newcommand{\vu}[1]{\hat{\vb{#1}}}
\newcommand*\dd[2][\relax]{\mathop{\ifx\relax#1\odif{#2}\else \odif[order={#1}]{#2}\fi\,}}

\newcommand{\vacuum}{\ket*{\vb{0}}}

\DeclareMathOperator{\trace}{Tr}
\DeclareMathOperator{\sinc}{sinc}

\AtBeginDocument{
	\let\Re\relax
	\let\Im\relax
	\DeclareMathOperator{\Re}{Re}
	\DeclareMathOperator{\Im}{Im}

	\renewcommand{\div}{\mathop{\mbfnabla\vdot}}
	\newcommand{\curl}{\mathop{\mbfnabla\vectimes}}
}

\DeclarePairedDelimiterX{\comm}[2]{[}{]}{#1,#2}

\DeclarePairedDelimiterX{\braket}[2]{\langle}{\rangle}{#1\delimsize\vert#2}
\DeclarePairedDelimiterX{\ketbra}[1]{\lvert}{\rvert}{#1\rangle\delimsize\langle#1}



\usetikzlibrary{arrows.meta}
\usetikzlibrary{decorations.pathmorphing}

\begin{document}
	\begin{tikzpicture}[
		photon/.style={-Latex, decorate, decoration={snake, post length=1mm}},
		electron/.style={Circle-Latex},
	]
		\draw[very thick] (0,0) rectangle ++(4,2);
		\draw[very thick, fill] (0,0) rectangle ++(4,-0.3);
		\draw[very thick, fill] (2.5,2) rectangle ++(0.5,0.2);
	
		\draw (4,-0.15) node[right] {Cathode};
		\draw (3.6,2) node[above] {Anode};
		\draw[thick, -{Circle[open]}] (2.75,2) -- ++(0,1);
		
		\draw[thick, rounded corners=1pt] (0.8,2) rectangle ++(2.4,-.3);
		\draw[thick, rounded corners=1pt] (0.4,2) rectangle ++(3.2,-.6);
		
		\draw[-Latex] (-1,1) node[left] {Depletion layer} -- (0.7,1.6);
		\draw[-Latex] (-1,1.5) node[left] {P layer} -- (1.1,1.9);
		\draw[-Latex] (-1,0.5) node[left] {N layer} -- (0.6,.9);

		\draw[photon] (1.4,4) -- ++(0,-2) node[midway, left, xshift=-0.1cm] {$\gamma$};
		\draw[photon] (1.7,4) -- ++(0,-2);
		\draw[photon] (2,4) -- ++(0,-2);
		\draw[electron] (2.1,1.92) -- ++(0.3,-1.7);
		\draw[electron] (1.5,1.92) -- ++(-0.13,-1.2);
		
		\draw[thick, -Latex] (2,0) -- ++(0,-1) node[right, yshift=0.15cm] {$I$};
		\draw[thick, -{Circle[open]}] (2,-0.6) -- ++(0,-1);
	\end{tikzpicture}
\end{document}

		\section{Homo- and heterodyne detector}
		
		\addcontentsline{toc}{section}{References}
		\printbibliography[title=References]
	\end{refsection}

	\chapter{Coherent state communication system}
	\begin{refsection}
		The present chapter identifies common characteristics of \gls{qkd} protocols and attempts to formalize the notion of a \gls{qkd} protocol.
We test the proposed formula with qubit-based \gls{qkd} protocols like BB84 and the six-state protocol as well as Gaussian \gls{cvqkd} protocols.
The chapter ends with a summary and literature review regarding practical \gls{qkd} protocols' post-processing and security analysis.

\Cref{fig:qkd_protocol} illustrates our proposed notion of a \gls{qkd} protocol, with the feature being a logical quantum system from which the random bits are encoded and decoded.
The logical quantum system is a subspace of the physical quantum system.
The physical quantum system depends strongly on the physical implementation and quantum encoding.
\begin{figure}[htb]
	\centering
	\includestandalone{figures/tikz/qkd-protocol}
	\caption{A \gls{qkd} protocol comprises a binary encoder, a logical quantum system, and a binary decoder. The binary encoder maps bits $\vb{b}\in\{0,1\}^n$ onto a quantum state of the logical quantum system $\ket{\psi}$. The binary decoder extracts the bits $\vb{b}$ back from the quantum state $\ket{\psi^\prime}$. The logical quantum system is a subspace of a larger physical quantum system. The state encoder and decoder map between the logical and physical quantum states.}\label{fig:qkd_protocol}
\end{figure}
The distinction between logical and quantum systems is vital to separate the implementation and security concerns.
Many security proofs show equivalence between the physical implementation and the logical system to use an established security proof.
One should keep in mind that such a separation implicitly assumes no loopholes from the particular implementation.

\Cref{fig:qkd_classification} illustrates common features among the \gls{qkd} protocols.
For an overview of \gls{qkd} protocols, see Ref.~\cite{Duvsek2006}.
\begin{figure}[htb]
	\centering
	\includestandalone{figures/tikz/qkd-classification}
	\caption{Common features among \gls{qkd} protocols: Detection, physical encoding, logical state space, measurement basis selection and schema.}\label{fig:qkd_classification}
\end{figure}
Every \gls{qkd} system requires a detector, e.g., a coherent detector or a single-photon (click) detector.
The detection does not necessarily imply the dimension of the logical state space as BB84 has been implemented with coherent detection~\cite{Qi2021}.
Concerning measurement basis selection, Bob can either actively choose a random measurement basis for every transmission or passively measure all (orthogonal) bases for measurement basis selection.
We will cover both active and passive measurement basis selection in the discussion of the polarization-encoding BB84 protocol.
Finally, the \gls{qkd} schema determines if either Alice prepares a state and sends it to Bob for measurement (prepare-and-measure) or if Alice and Bob share an entangled state (entanglement-based).
Most practical \gls{qkd} implementations use prepare-and-measure.
On a theoretical level, both schemas are equivalent, and security proofs are often more convenient in an entanglement-based setting.
		\section{Transmitter}                                                                                                                                                                                                                                                                                                                                                                                                                                                                                                                                                                                                                                                                                                                     

We introduced the transmitter as a component encoding a sequence of complex symbols, $\left\{\alpha_n\in\mathbb{C}\colon n\in I\right\}$, onto a coherent state $\ket{\alpha(t)}$.
Efficient transmission through the channel and effective receiver detection impose additional constraints on the space of useful coherent states.
Together with practical considerations, these constraints lead to the particular design embodiment of the transmitter we will discuss.

First and foremost, the channel and receiver limit the spectrum of useful coherent states.
For instance, the receiver has limited bandwidth to detect the signal with signal power outside that bandwidth being lost.
Apart from that, the physical channel only shows favorable transmission properties over a certain frequency range, outside the signal is strongly suppressed and distorted.\footnote{For instance, the C-band, spanning wavelengths from \SI{1530}{\nano\meter} to \SI{1565}{\nano\meter}, is widely deployed for optical telecommunication.}
Additionally, different users may jointly use the same physical channel, and using the available bandwidth efficiently while reducing interference between users, requires the signal bandwidth to be minimal.
\begin{figure}[htb]
	\centering
	\includegraphics{figures/tikz/baseband-passband}
	\caption{Power spectrum showing the relationship between a real-valued baseband and passband signal. Both base- and passband signals have bandwidth $B$. The baseband signal is centered at $\omega=0$. The passband is shifted by $\omega_0$ and has a conjugate symmetric counterpart at $-\omega_0$.}\label{fig:baseband_passband}
\end{figure}
To illustrate the signal and channel bandwidth, we introduce the concept of baseband and passband signals\cite[p.~26]{Madhow2008}.
\Cref{fig:baseband_passband} depicts the power spectrum of a baseband and passband signal each with bandwidth $B$.
The spectrum of the baseband signal is centered at zero frequency, $\omega=0$, while the spectrum of the passband signal is located at $\pm\omega_0$.
For efficient use of the limited channel and receiver bandwidth, we want to minimize $B$ while keeping the \gls{snr} high.
In addition, we need to shift the baseband spectrum to an optical frequency $\omega_0$ for which the channel shows desirable transmission characteristics.
So from a signal processing point of view, we want the transmitter to
\begin{enumerate}
	\item first create a baseband signal with minimum bandwidth $B$ and
	\item then transfer it to a passband signal in the optical domain.
\end{enumerate}
In the following, we term the first step signal synthesis and the second step up-conversion.

Nowadays, the signal is almost exclusively constructed in the digital domain, and the analog part is limited to the digital-to-analog conversion.
Constructing the signal digitally allows for greater flexibility in the development process as the synthesis is mostly software-defined.
For up-conversion of the synthesized signal to the optical domain, we modulate the electric signal onto an optical carrier.
\begin{figure}[htb]
	\centering
	\includegraphics{figures/tikz/software-defined-transmitter}
	\caption{Block diagram of the transmitter's signal processing domains. The \gls{dsp} transforms a complex symbol sequence $\{\alpha_n\}_{n\in I}$ into two digital signals, $x^\prime[m]$ and $p^\prime[m]$, corresponding to the real and imaginary part. The \gls{dac} converts the digital signals to analog signals, $x(t)$ and $p(t)$ we modulate onto a coherent state $\ket{\alpha(t)}$.}\label{fig:software_defined_transmitter}
\end{figure}
\Cref{fig:software_defined_transmitter} illustrates how such a software-defined transmitter architecture applies to our coherent state transmission system.
The software-defined \gls{dsp} constructs the bandwidth-optimized digital signals $x[m]$ and $p[m]$, encoding the real and imaginary parts of the complex symbols.
The \gls{dac} stage converts the digital signals, $x[m]$ and $p[m]$, to bandwidth-limited analog signals $x(t)$ and $p(t)$.
Finally, the analog signals are modulated onto an optical carrier yielding a coherent state $\ket{\alpha(t)}$ which meets the bandwidth requirements of the channel.

\subsection{Symbol encoding in baseband}

To construct a bandwidth-optimized baseband signal, encoding the complex symbol sequence $\{\alpha_n\in\mathbb{C}\colon n\in I\}$, we first remark that the symbol sequence itself has no notion of time.
In contrast, a digital (time-discrete) signal, consisting of discrete samples, includes a time reference, the sample period $T_s$, denoting the temporal distance between two consecutive samples.
By defining the digital signal with samples equal to the symbols, $\alpha[n]=\alpha_n$, and introducing the symbol period $T_s$ as sample period, we find ourselves with the complete \gls{dsp} toolbox at our disposal.\footnote{Even if the \gls{dsp} itself does not work explicitly with the time reference $T_s$, we need time to give a meaningful interpretation of the signal between the steps.}
\begin{figure}[htb]
	\centering
	\includegraphics{figures/circuitikz/baseband-construction}
	\caption{Block diagram of the signal-processing for baseband construction. The digital signal $x[km]$ is upsampled by a factor $k$ to $x^\prime[m]$ and pulse-shaped by a \gls{rrc} filter to yield $x^{\prime\prime}[m]$. A \gls{dac} converts the pulse-shaped signal $x^{\prime\prime}[m]$ to the analog signal $x^\prime(t)$. Finally, the analog signal $x^\prime(t)$ is \gls{lp} filtered to yield the analog anti-aliased signal $x(t)$.}\label{fig:baseband_construction}
\end{figure}
\Cref{fig:baseband_construction} summarizes the essential \gls{dsp} steps including analog conversion of a real-valued digital signal $x[km]$ to construct a bandwidth-optimized analog baseband signal $x(t)$.\footnote{The baseband construction generalizes to a complex digital signal by applying the real-valued baseband construction separately to the real and imaginary part.}
The digital signal $x[km]$, containing the symbols, is first upsampled by an upsampling factor of $k$, adding $k$ zero-valued samples in between the original samples.
The pulse-shaping of the \gls{rrc} filter interpolates between the non-zero samples, the symbols, to reduce the effective signal bandwidth.
Finally, a \gls{dac} converts the digital signal $x^{\prime\prime}[m]$ to the analog signal $x^\prime(t)$ through infinite upsampling.
The analog signal $x^\prime(t)$ contains infinite aliases through the upsampling, which we remove by filtering $x^\prime(t)$ with a \gls{lp}, yielding the anti-aliased analog signal $x(t)$.
\begin{figure}[htb]
	\centering
	\includegraphics{figures/pgfplots/tx-unit-time}
	\caption{Baseband construction for a single unit symbol in the time domain. The symbol sequence $\{x_n\in\mathbb{R}\colon n\in I\}$ is represented by the digital signal $x[km]$ with sample period $T_s$ (first row). The digital signal $x[km]$ is upsampled to $x[m]$ by an upsampling factor of $k=2$ (second row). The upsampled signal $x[m]$ is pulse-shaped with a \gls{rrc} filter to yield $x^\prime[m]$ (third row). The pulse-shaped digital signal $x^\prime[m]$ is converted to the anti-aliased analog signal $x(t)$.}\label{fig:baseband_construction_unit_time}
\end{figure}
\Cref{fig:baseband_construction_unit_time} illustrates the time domain signals for each signal-processing step for a symbol sequence which contains only a single non-zero symbol with unit value.
We see very well how the upsampling increases the resolution of the digital signal in the time domain and how the pulse-shaping filter interpolates between the samples.
We also see that the pulse-shaping filter corresponds to a $\sinc$-like impulse response.
The similarity of the analog signal with a $\sinc$ pulse is not surprising since its \gls{rrc} square equals a rectangular filter with roll-off, i.e.,
\begin{equation}
	\abs{h_\text{rc}\left(f/f_s\right)}
	=
	\begin{cases}
		1 & \abs*{f/f_s}\leq(1-\alpha) \\
		\cos\left[\frac{\pi}{4\alpha}\left(\abs*{f/f_s}-1+\alpha\right)\right] & 1-\alpha\leq\abs*{f/f_s}\leq1+\alpha \\
		0 & \text{otherwise}
	\end{cases}
	,
\end{equation}
wherein $f_s=1/T_s$ is the symbol rate and $\alpha$ determines the roll-off.
\begin{figure}[htb]
	\centering
	\includegraphics{figures/pgfplots/tx-rand-time}
	\caption{Baseband construction for a random \gls{qpsk} symbol sequence in the time domain. The real and imaginary part are colored orange respectively green. The complex symbol sequence $\{\alpha_n\in\mathbb{C}\colon n\in I\}$ is represented by the digital signal $\alpha[km]$ with sample period $T_s$ (first row). The digital signal $\alpha[km]$ is upsampled to $\alpha^\prime[m]$ by an upsampling factor of $k=2$ (second row). The upsampled signal $\alpha^\prime[m]$ is pulse-shaped with a \gls{rrc} filter to yield $\alpha^{\prime\prime}[m]$ (third row). The pulse-shaped digital signal $\alpha^{\prime\prime}[m]$ is converted to the anti-aliased analog signal $\alpha(t)$.}\label{fig:baseband_construction_rand_time}
\end{figure}
\Cref{fig:baseband_construction_rand_time} illustrates the time domain signals for each signal-processing step for a random \gls{qpsk} symbol sequence.
\begin{figure}[htb]
	\centering
	\includegraphics{figures/pgfplots/tx-frequency}
	\caption{Baseband construction for a random \gls{qpsk} symbol sequence in the frequency domain. The power spectrum corresponding to the unit symbol sequence is colored orange, the \gls{qpsk} symbol sequence is colored green. Initially, the spectrum spans from $-1/2$ to $+1/2$ the normalized sampling frequency $f/f_s$ (first row). Upsampling by $k=2$ adds aliases left and right to the initial spectrum (second row). Pulse-shaping suppresses the left and ride flanks of the spectrum (third row). Analog conversion corresponds to infinite upsampling, adding infinite aliases left and right of the spectrum. (fourth row). Applying a \gls{lp} filter strongly suppresses the aliases (last row).}\label{fig:baseband_construction_freq}
\end{figure}
\Cref{fig:baseband_construction_freq} provides further inside into the signal-processing steps by presenting the power spectrum of the unit and \gls{qpsk} symbol sequences.
In the frequency domain, it is very clear to see how upsampling widens the spectrum without adding additional information.
We also see how the pulse-shaping filter shapes the upsampled spectrum, and the \gls{lp} filter suppresses aliases.

\FloatBarrier
\subsection{Up-conversion of baseband to passband}

We previously constructed the bandwidth-optimized baseband signals $x(t)$ and $p(t)$ encoding the real respective imaginary part of a complex symbol sequence $\{\alpha_n\in\mathbb{C}\colon n\in I\}$.
We now want to shift the spectrum of the baseband signals into the optical domain, i.e., transform the baseband to a passband signal centered around some optical carrier frequency $\omega_c$.
One way to shift a real signal in the frequency domain is to multiply it by a sine or cosine, effectively corresponding to amplitude modulation.
With the orthogonality of sine and cosine, it is possible to write two real baseband signals as one real passband signal, i.e.,
\begin{equation}
	x(t)
	\cos(\omega_ct)
	+
	p(t)
	\sin(\omega_ct)
	.
\end{equation}
Rewriting the sine as the real part of a complex exponential
\begin{equation}
	\sin(\omega_ct)
	=
	\cos(\omega_c t-\pi/2)
	=
	\Re\left\{e^{-i\omega_c t+i\pi/2}\right\}
	,
\end{equation}
we can rewrite the real passband signal as the real part of a complex signal,
\begin{equation}
	\begin{split}
		x(t)
		\cos(\omega_ct)
		+
		p(t)
		\sin(\omega_ct)
		&=
		\Re\left\{
			x(t)
			e^{-i\omega_ct}
			+
			p(t)
			e^{-i\omega_ct+i\pi/2}
		\right\}
		\\
		&=
		\Re\left\{
			\left[
				x(t)
				+
				ip(t)
			\right]
			e^{-i\omega_ct}
		\right\}
		,
	\end{split}
	\label{eq:real_complex_passband}
\end{equation}
where we can identify the complex baseband
\begin{equation}
	\alpha(t)
	=
	x(t)
	+
	ip(t)
	.
\end{equation}
The complex signal is often used when dealing with only a real signal since the up-conversion here corresponds to a simple multiplication by a complex exponent.
\begin{figure}[htb]
	\centering
	\includegraphics{figures/circuitikz/up-conversion}
	\caption{Block diagram illustrating up-conversion of two real-valued baseband signals, $x(t)$ and $p(t)$, to a real-valued passband signal, $\Re\left\{\alpha(t)e^{-i\omega_ct}\right\}$. A \gls{lo} with up-conversion frequency $\omega_c$ is split into two branches with a relative phase shift between the branches of $\pi/2$. One branch is mixed with the baseband signal $p(t)$ and the other branch is mixed with $x(t)$. The product of the mixing is added giving the passband signal.}\label{fig:up_conversion}
\end{figure}
Our derivation of the complex passband, \cref{eq:real_complex_passband}, already hints implementing efficient up-conversion of two real baseband signals by splitting a \gls{lo} with a relative phase shift of $\pi/2$.
If we replace the mixer in \Cref{fig:up_conversion} with electro-optical amplitude modulators, e.g.,\gls{mzm}, we find ourselves with an electro-optical \gls{iqm}.
Indeed, if we assume a single-mode laser as an input state, we find that the \gls{iqm} produces the output state
\begin{equation}
	\ket{e^{-i\omega_c t}}
	\to
	\ket{\alpha(t)e^{-i\omega_c t}}
\end{equation}
exactly as we found for the up-conversion.
		\section{Channel}

% TODO: physical effects of a channel
% TODO: original signal, signal with noise, signal with noise and attenuation
% TODO: signal-to-noise ratios (?)
% TODO: quantum analog of noise (?)

\begin{figure}[htb]
	\centering
	\includestandalone{figures/circuitikz/channel}
	\caption{Flow diagram of a communication system.}
\end{figure}
\begin{equation}
	y(t)
	=
	h(t)x(t)
	+
	n(t)
\end{equation}

		\section{Receiver}
\FloatBarrier

We introduced the receiver as a component decoding a sequence of complex symbols,
\begin{equation}
	\left\{
		\beta_n
		\in
		\mathbb{C}
		\colon
		n\in I
	\right\}
	,
\end{equation}
from a coherent state $\ket{\beta(t)}$.
Just like the transmitter, we want to keep the receiver software-defined.
\begin{figure}[htb]
	\centering
	\includegraphics{figures/tikz/software-defined-receiver}
	\caption{Block diagram of the receiver's signal processing domains. The analog electrical signals $u(t)$, and optional $v(t)$, are demodulated from the quadratures of the optical coherent state $\ket{\beta(t)}$, and then converted to the digital signals $u[m]$, and optional $v[m]$, from which the \gls{dsp} decodes the symbol sequence $\{\beta_n\in\mathbb{C}\colon n\in I\}$.}\label{fig:software_defined_receiver}
\end{figure}
\Cref{fig:software_defined_receiver} shows the signal processing of a possible software-defined receiver.
The coherent state is transferred from the optical via the analog to the digital.

\FloatBarrier
\subsection{Downconversion}

At the transmitter, we upconverted two real baseband signals to a real passband signal.
For the receiver, we discuss options involving one and two real baseband signals.
We first consider the simpler case of direct downconversion as depicted in \Cref{fig:downconversion_single}.
\begin{figure}[htb]
	\centering
	\includegraphics{figures/circuitikz/downconversion-single}
	\caption{Block diagram of single-quadrature downconversion. The signal $z(t)$ is mixed with the \gls{lo} signal $\cos(\omega_lt+\vartheta)$. The downconverted signal $u(t)$ is obtained by \gls{lp} filtering the output signal of the mixing.}\label{fig:downconversion_single}
\end{figure}
In direct downconversion, we mix a real-valued signal,
\begin{equation}
	\begin{split}
		z(t)
		=
		\int_{-\infty}^{+\infty}\frac{\dd{\omega}}{2\pi}
		z(\omega)
		e^{+i\omega t}
		&=
		\int_0^{+\infty}\frac{\dd{\omega}}{2\pi}
		z(\omega)
		e^{+i\omega t}
		+
		\int_{-\infty}^0\frac{\dd{\omega}}{2\pi}
		z(\omega)
		e^{+i\omega t}
		\\
		&=
		\int_0^{+\infty}\frac{\dd{\omega}}{2\pi}
		z(\omega)
		e^{+i\omega t}
		-
		\int_{+\infty}^0\frac{\dd{\omega}}{2\pi}
		z(-\omega)
		e^{-i\omega t}
		\\
		&=
		\int_0^{+\infty}\frac{\dd{\omega}}{2\pi}
		\left[
			z(\omega)
			e^{+i\omega t}
			+
			z(\omega)^*
			e^{-i\omega t}
		\right]
		\\
		&=
		\int_0^{+\infty}\frac{\dd{\omega}}{2\pi}
		2\Re\left[
			z(\omega)
			e^{+i\omega t}
		\right]
		,
	\end{split}
\end{equation}
where we used the conjugate symmetry, $z(-\omega)=z(\omega)^*$, of the Fourier transform of a real-valued function $z(t)$.
Multiplication with the \gls{lo} signal $\cos(\omega_lt+\vartheta)$, the mixing produces a high- and low-frequency band
\begin{equation}
	\begin{split}
		z(t)
		\cos(\omega_lt+\vartheta)
		&=
		2\Re
		\int_0^\infty\frac{\dd{\omega}}{2\pi}
		z(\omega)
		e^{+i\omega t}
		\cos(\omega_lt+\vartheta)
		\\
		&=
		\Re
		\int_0^\infty\frac{\dd{\omega}}{2\pi}
		z(\omega)
		e^{+i\omega t}
		\left[
			e^{+i(\omega_lt+\vartheta)}
			+
			e^{-i(\omega_lt+\vartheta)}
		\right]
		\\
		&=
		\Re
		\int_0^\infty\frac{\dd{\omega}}{2\pi}
		z(\omega)
		\left[
			e^{+i(\omega+\omega_l)t+i\vartheta}
			+
			e^{+i(\omega-\omega_l)t-i\vartheta}
		\right]
		\\
		&=
		\Re
		\int_{+\omega_l}^\infty\frac{\dd{\omega}}{2\pi}
		z(\omega-\omega_l)
		e^{+i\omega t+i\vartheta}
		\\
		&\qquad+
		\Re
		\int_{-\omega_l}^\infty\frac{\dd{\omega}}{2\pi}
		z(\omega+\omega_l)
		e^{+i\omega t-i\vartheta}
		.
	\end{split}
\end{equation}
However, we suppress the high-frequency band using an ideal \gls{lp} filter with bandwidth $B$,
\begin{equation}
	\begin{split}
		u(t)
		&=
		\Re
		\int_{-B/2}^{+B/2}\frac{\dd{\omega}}{2\pi}
		z(\omega+\omega_l)
		e^{+i\omega t-i\vartheta}
		\\
		&=
		\Re
		\int_{0}^{+B/2}\frac{\dd{\omega}}{2\pi}
		z(\omega+\omega_l)
		e^{+i\omega t-i\vartheta}
		+
		\Re
		\int_{-B/2}^{0}\frac{\dd{\omega}}{2\pi}
		z(\omega+\omega_l)
		e^{+i\omega t-i\vartheta}
		\\
		&=
		\Re
		\int_{0}^{+B/2}\frac{\dd{\omega}}{2\pi}
		z(\omega+\omega_l)
		e^{+i\omega t-i\vartheta}
		-
		\Re
		\int_{B/2}^{0}\frac{\dd{\omega}}{2\pi}
		z(\omega-\omega_l)^*
		e^{-i\omega t-i\vartheta}
		\\
		&=
		\Re
		\int_{0}^{+B/2}\frac{\dd{\omega}}{2\pi}
		\left[
			z(\omega+\omega_l)
			e^{+i\omega t}
			+
			z(\omega-\omega_l)^*
			e^{-i\omega t}
		\right]
		e^{-i\vartheta}
		,
	\end{split}
	\label{eq:downconversion_real}
\end{equation}
where we assumed $\omega_l\gg B/2$.
For $\vartheta=0$, the downconverted signal $v(t)$ is equal to projecting the real part of the complex input spectrum $z(\omega)$, losing the imaginary part's information.
Furthermore, when rewriting $u(t)$ as an integral over positive frequencies, i.e., frequencies we can measure, we find a second term mirroring the first term.
\begin{figure}[htb]
	\centering
	\includegraphics{figures/tikz/spectrum-downconversion-single}
	\caption{Power spectrum illustrating downconversion of a passband signal (solid spectrum) mixed with a \gls{lo} signal $\omega_l$ to the intermediate frequency $\omega_i$ (dashed spectrum) and measurement with bandwidth $B_d$ (dotted spectrum).}\label{fig:spectrum_downconversion_single}
\end{figure}
\Cref{fig:spectrum_downconversion_single} shows the downconversion of the signal $z(t)$ around the \gls{lo} at $\omega_l$ to the intermediate frequency $\omega_i$.
The actual measurement involved only positive frequencies up to the detector bandwidth $B/2$ causing the actual signal to be imposed with the mirrored spectrum.

Single-quadrature downconversion only reveals a real projection of the complex spectrum.
To conserve both quadratures, we need to split the input signal into two branches and perform single-quadrature downconversion with two orthogonal phase references of the \gls{lo}, see \Cref{fig:downconversion_dual}.
\begin{figure}[htb]
	\centering
	\includegraphics{figures/circuitikz/downconversion-dual}
	\caption{Block diagram of dual-quadrature downconversion. The signal $z(t)$ is divided equally into an upper and a lower branch. The upper branch is mixed with the phase shifted \gls{lo} signal $\cos(\omega_ct+\vartheta)$. The lower branch is mixed with \gls{lo} signal $\sin(\omega_ct+\vartheta)$. The mixer outputs are filtered separately by a \gls{lp} yielding the downconverted signals $u(t)$ and $v(t)$.}\label{fig:downconversion_dual}
\end{figure}
The signal of the upper branch $u(t)$ is equal to our result for the single-quadrature downconversion, \cref{eq:downconversion_real}.
The signal of the lower branch,
\begin{equation}
	\begin{split}
		v(t)
		&=
		\Im
		\int_{-B/2}^{+B/2}\frac{\dd{\omega}}{2\pi}
		z(\omega-\omega_l)
		e^{+i(\omega t+\vartheta)}
		\\
		&=
		\Im
		\int_{0}^{+B/2}\frac{\dd{\omega}}{2\pi}
		\left[
			z(\omega-\omega_l)
			e^{+i(\omega t+\vartheta)}
			+
			z(\omega+\omega_l)^*
			e^{-i(\omega t+\vartheta)}
		\right]
		,
	\end{split}
	\label{eq:downconversion_imag}	
\end{equation}
is simply obtained from \cref{eq:downconversion_real} by shifting the \gls{lo} phase reference by \SI{90}{\degree}, i.e., $\vartheta\to\vartheta+\pi/2$.
Regardless of the particular value of the \gls{lo} phase reference $\vartheta$, dual-quadrature downconversion recovers the complete information, the real and imaginary part, of the input signal spectrum $z(\omega)$.

We presented the electro-optical receiver setups implementing single- and dual-quadrature downconversion in \Cref{fig:coherent_receiver_active} and \Cref{fig:coherent_receiver_passive} in \Cref{ch:qkd}.
Essentially, the electro-optical setups combine the optical signal and \gls{lo} in an optical coupler and perform balanced detection on the coupler outputs for which derived the quantum theory in \Cref{sec:photodetectors}.
From a quantum viewpoint, balanced detection corresponds to a quadrature measurement at a particular frequency represented by the generalized quadrature operator, \cref{eq:quadrature_operator_generalized}.

\subsection{Homo- and heterodyning}

So far, we have not assumed any particular signal for the downconversion but treated the receiver as a spectrum analyzer.
If we now assume the input signal to be from the coherent-state transmitter $\ket{\beta(t)}$, \cref{eq:upconversion_dual}, the downconverted signals read
\begin{align}
	u(t)
	&=
	\Re
	\int_{-B/2}^{+B/2}\frac{\dd{\omega}}{2\pi}
	\beta(\omega-\omega_c+\omega_l)
	e^{+i(\omega t+\theta)}
	\label{eq:receiver_demod_real}
	\\
	v(t)
	&=
	\Im
	\int_{-B/2}^{+B/2}\frac{\dd{\omega}}{2\pi}
	\beta(\omega-\omega_c+\omega_l)
	e^{+i(\omega t+\theta)}
	\label{eq:receiver_demod_imag}
	,
\end{align}
wherein $\theta$ accounts for the phases of the up- and downconversion \glspl{lo}.
We define the the intermediate frequency as the difference between the transmitter and receiver \glspl{lo}, i.e.,
\begin{equation}
	\omega_i
	=
	\abs{\omega_c-\omega_l}
	<
	B_d/2
	,
\end{equation}
and distinguish between homodyning for zero intermediate frequency $\omega_i=0$, and otherwise, heterodyning $\omega_i\neq 0$.
\begin{figure}[htb]
	\centering
	\includegraphics{figures/tikz/spectrum-heterodyning}
	\caption{Power spectrum illustrating heterodyne detection. The passband signal with carrier frequency $\omega_c$ and bandwidth $B_s$ (dashed) is downconverted with the \gls{lo} frequency $\omega_l>\omega_c$. The downconverted spectrum is measured with bandwidth $B_d$ (solid), which contains the image band (dotted) from the right side of the \gls{lo}.}\label{fig:spectrum_heterodyning}
\end{figure}
In heterodyning, we downconvert the passband signal at $\omega_c$ to an intermediate frequency $\omega_i$.


The negative frequencies collapse onto the positive frequencies by being mirrored around zero frequency $\omega=0$.
In \Cref{fig:spectrum_heterodyning}, the left spectrum at $\omega_l$ (dashed) is mirrored 

\Cref{fig:spectrum_heterodyning} shows how the frequencies left of the $\omega_l$ fold onto the positive frequencies.


Furthermore, we distinguish between single- and dual-quadrature homodyning, depending if we perform single- or dual- quadrature downconversion.
If the detector bandwidth is large enough to cover the baseband signal at the intermediate frequency, we can resolve both quadratures of the incoming signal with heterodyning and single-quadrature downconversion.
\begin{table}[htb]
  \centering
  \begin{tabular}{lccccc}
    \toprule
    Scheme & Homodyne (single) & Homodyne (dual) & Heterodyne \\
    \midrule
    Balanced detectors & \num{1} & \num{2} & \num{1} \\
    Quadratures & \num{1} & \num{2} & \num{2} \\
    Intermediate frequency & \multicolumn{2}{c}{$\omega_i=0$} & $\omega_i\neq 0$ \\
    Optical complexity & Low & High & Low \\
    Signal bandwidth & High & High & Low \\
    \gls{lo} synchronization & Frequency and phase & Frequency & Bandwidth \\
    \bottomrule
  \end{tabular}
  \caption{Comparison of receiver schemes according to Ref.~\cite{Brunner2017}. The single-quadrature homodyne detection offers low optical complexity and high bandwidth but only resolves one of two quadratures and required frequency and phase synchronization of the \gls{lo}. The dual-quadrature homodyne detection resolves both quadratures with high bandwidth but requires two balanced detectors increasing the optical complexity and phase synchronization of the \gls{lo}. The heterodyne detection schemes resolves both quadratures with low complexity and no requirements on \gls{lo} synchronization at the cost of signal bandwidth.}\label{tab:receivers}
\end{table}
\Cref{tab:receivers} summarizes the characteristics between the single and dual homodyne and the heterodyne receiver schemes.
A strong advantage of the heterodyne receiver design is that both quadratures can be resolved with a single balanced detector, keeping the optical complexity low.

\begin{figure}[htb]
	\centering
	\includegraphics{figures/tikz/spectrum-homodyning}
	\caption{Power spectrum illustrating homodyne detection. The passband signal with carrier frequency $\omega_c$ and bandwidth $B_s$ (dashed) is downconverted with the \gls{lo} frequency, equal to the carrier frequency, $\omega_l=\omega_c$. The downconverted spectrum is measured with bandwidth $B_d$ (solid), which contains the mirror (dotted) from the right side of the \gls{lo}.}\label{fig:spectrum_homodyning}
\end{figure}

\FloatBarrier
\subsection{Symbol decoding}

We continue our receiver description, starting from the single-quadrature downconversion and assuming the more general heterodyning, which for $\omega_i=0$ reduces to single-quadrature homodyning.
\begin{figure}[htb]
	\centering
	\includegraphics{figures/circuitikz/symbol-decoding}
	\caption{Block diagram of the signal processing for the symbol decoding. The analog signal $u(t)$ is converted to the digital signal $u[m/(kl)]$. The real digital signal $u[m/(kl)]$ is multiplied with the complex exponential $\exp(i\omega_it)$, yielding the complex digital signal $\sigma[m/(kl)]$. $\sigma[m/(kl)]$ is downsampled by $l$ to yield the complex digital signal $\mu[m/k]$. $\mu[m/k]$ is pulse-shaped with the matched \gls{rrc} filter to yield the complex digital signal $\kappa[m/k]$. $\kappa[m/k]$ is downsampled to the complex digital signal $\beta[m]$ corresponding to the decoded symbol sequence.}\label{fig:symbol_decoding_blocks}
\end{figure}
\Cref{fig:symbol_decoding_blocks} summarizes the relevant signal processing for the symbol decoding.
The downconverted signal $u(t)$ corresponding to the real part of the received coherent-state spectrum $\beta(\omega)$, \cref{eq:receiver_demod_real}, is sampled by an \gls{adc}, yielding the digital signal $u[m/(kl)]$.
We remove the intermediate frequency in $u[m/(kl)]$ by multiplication with a complex exponential, i.e,
\begin{equation}
	\sigma\left[\frac{m}{kl}\right]
	=
	u\left[\frac{m}{kl}\right]
	e^{+2\pi i (m/kl) T_s}
	,
\end{equation}
making the signal complex-valued.
It follows a downsampling by $l$ of the signal such that we can apply the same \gls{rrc} filter, the matched filter, which we used in the symbol encoding to maximize \gls{snr}.
Finally, we downsample by $k$ to restore a digital signal corresponding to the symbol sequence.
\begin{figure}[htb]
	\centering
	\includegraphics{figures/pgfplots/symbol-decoding-frequency}
	\caption{Power spectrum of the symbol-decoding steps for a random \gls{qpsk} symbol-sequence. The demodulated signal is a real-valued passband signal centered at the intermediate frequency (first row). After digital downconversion we have a complex-valued baseband signal, centered at zero frequency (second row). Applying the matched \gls{rrc} filter completes the pulse-shaping (third row). Downsampling recovers the initial symbol band (last row).}\label{fig:symbol_decoding_frequency}
\end{figure}
\Cref{fig:symbol_decoding_frequency} illustrates how the symbol decoding is carried out in the frequency domain.
The demodulated signal spectrum is a passband signal at the intermediate frequency and downconversion reduces the passband to a baseband signal.
Completing the pulse-shaping with the matched filter increases the steepness of the flanks which are collapsed with aliasing by the final downsampling step.
\begin{figure}[htb]
	\centering
	\includegraphics{figures/pgfplots/symbol-decoding-time-qpsk}
	\caption{Signal amplitude of the symbol decoding steps for a random \gls{qpsk} symbol-sequence. The real-valued demodulated signal oscillates at the intermediate frequency (first row). Digital downconversion removed the intermediate frequency, yielding a complex signal (second row). Completing the pulse-shaping and downsampling by applying a matched \gls{rrc} filter (third row). Downsampling recovers the complex symbol sequence equal to the transmitted sequence (fourth and last row).}\label{fig:symbol_decoding_time}
\end{figure}
\Cref{fig:symbol_decoding_time}) shows the symbol decoding in the time domain.

		\addcontentsline{toc}{section}{References}
		\printbibliography[title=References]
	\end{refsection}
	
	\chapter{Continuous-variable quantum-key distribution}
	\begin{refsection}

		\addcontentsline{toc}{section}{References}
		\printbibliography[title=References]
	\end{refsection}

	\chapter{Conclusion and outlook}
	\begin{refsection}	
		\addcontentsline{toc}{section}{References}
		\printbibliography[title=References]
	\end{refsection}

	\appendix

\end{document}

\documentclass[
	a4paper,
	parskip,
	appendixprefix,
	chapterprefix,
	headings=big,
]{scrreprt}

\usepackage{amsthm}
\usepackage{amsmath}
\usepackage{authblk}
\usepackage[english]{babel}
\usepackage{biblatex}
\usepackage{booktabs}
\usepackage{csquotes}
\usepackage[acronym,nonumberlist,toc]{glossaries}
\usepackage{hyperref}
\usepackage{graphicx}
\usepackage{cleveref}
\usepackage[section]{placeins}
\usepackage[separate-uncertainty=true]{siunitx}
\usepackage[mode=buildnew]{standalone}
%\usepackage{thmtools,thm-restate}
\usepackage{tensor}
\usepackage{multirow}
\usepackage{xcolor}
\usepackage{unicode-math}
\usepackage{mathtools}
\usepackage{derivative}

% most complete math font
\setmainfont{Stix Two Text}
\setmathfont{Stix Two Math}

\DeclarePairedDelimiter{\ceil}{\lceil}{\rceil}
\DeclarePairedDelimiter{\floor}{\lfloor}{\rfloor}
\DeclarePairedDelimiter{\abs}{\lvert}{\rvert}
\DeclarePairedDelimiter{\norm}{\lVert}{\rVert}
\DeclarePairedDelimiter{\bra}{\langle}{\rvert}
\DeclarePairedDelimiter{\ket}{\lvert}{\rangle}
\DeclarePairedDelimiter{\expval}{\langle}{\rangle}
	
\newcommand{\trans}{{\scriptscriptstyle\mathsf{T}}}

\DeclareMathOperator{\trace}{\operatorname{Tr}}
\DeclareMathOperator{\sinc}{\operatorname{sinc}}

\AtBeginDocument{%
	\let\Re\undefined
	\let\Im\undefined
	\DeclareMathOperator{\Re}{\operatorname{Re}}
	\DeclareMathOperator{\Im}{\operatorname{Im}}
}

\let\div\undefined
\let\curl\undefined
\DeclareMathOperator{\div}{\nabla\vdot}
\DeclareMathOperator{\curl}{\nabla\vectimes}

\DeclarePairedDelimiterX{\comm}[2]{[}{]}{#1,#2}

\DeclarePairedDelimiterX{\braket}[2]{\langle}{\rangle}{#1\delimsize\vert#2}
\DeclarePairedDelimiterX{\ketbra}[1]{\lvert}{\rvert}{#1\rangle\delimsize\langle#1}

\newcommand{\vdot}{\cdot}
\newcommand{\vcross}{\vectimes}
\newcommand{\vb}[1]{\symbfup{#1}}
\newcommand{\vu}[1]{\hat{\vb{#1}}}

\addbibresource{literature.bib}

% add bibliography as section (not chapter)
% https://tex.stackexchange.com/questions/568580/make-the-bibliography-as-a-section-in-each-includ-chapter
\defbibheading{bibliography}[\bibname]{\section*{#1}}

% chapter appearance
% https://tex.stackexchange.com/questions/159502/koma-script-scrreprt-how-to-change-chapter-appearance-and-produce-a-chapter-bas
\addtokomafont{chapterprefix}{\raggedleft}
\renewcommand*{\chapterformat}{%
\mbox{\chapappifchapterprefix{\nobreakspace}%
\scalebox{3}{\color{gray}\thechapter\autodot}\enskip}}

% caption label linebreak
\setkomafont{captionlabel}{\sffamily}

% theorems
\newtheorem{theorem}{Theorem}[section]
\newtheorem{lemma}[theorem]{Lemma}
\newtheorem{corollary}[theorem]{Corollary}
\theoremstyle{definition}
\newtheorem{definition}{Definition}[section]
\newtheorem{conjecture}{Conjecture}[section]
\newtheorem{example}{Example}[section]
\theoremstyle{remark}
\newtheorem*{remark}{Remark}

% prefix equation numbers with section number
\numberwithin{equation}{section}

% optics
\newacronym{ar}{AR}{anti-reflective}
\newacronym{mzm}{MZM}{Mach-Zehnder modulator}
\newacronym{iqm}{IQM}{In-phase and quadrature modulator}
\newacronym{mzi}{MZI}{Mach-Zehnder interferometer}
\newacronym{bs}{BS}{beam splitter}
\newacronym{fc}{FC}{fiber coupler}
\newacronym{qe}{QE}{quantum efficiency}
\newacronym{voa}{VOA}{variable optical attenuator}

% physics
\newacronym{dv}{DV}{discrete-variable}
\newacronym{cv}{CV}{continuous-variable}
\newacronym{dof}{DOF}{degrees of freedom}
\newacronym{com}{COM}{center of mass}
\newacronym{eom}{EOM}{equation(s) of motion}
\newacronym{pbc}{PBC}{periodic boundary conditions}
\newacronym{povm}{POVM}{positive operator-valued measure}
\newacronym{bch}{BCH}{Baker-Campbell-Hausdorff}
\newacronym{ccr}{CCR}{canonical commutation relation}

% electrical engineering
\newacronym{asp}{ASP}{analog signal processing}
\newacronym{dsp}{DSP}{digital signal processing}
\newacronym{osp}{OSP}{optical signal processing}
\newacronym{lo}{LO}{local oscillator}
\newacronym{lti}{LTI}{linear time-invariant}
\newacronym{rf}{RF}{radio frequency}
\newacronym{if}{IF}{intermediate frequency}
\newacronym{lp}{LP}{low-pass}
\newacronym{tia}{TIA}{transimpedance amplifier}
\newacronym{iq}{I/Q}{in-phase/quadrature}
\newacronym{psd}{PSD}{power spectral density}
\newacronym{snr}{SNR}{signal-to-noise ratio}
\newacronym{rc}{RC}{raised-cosine}
\newacronym{rrc}{RRC}{root-raised-cosine}
\newacronym{adc}{ADC}{analog-to-digital converter}
\newacronym{dac}{DAC}{digital-to-analog converter}
\newacronym{am}{AM}{amplitude modulation}
\newacronym{fm}{FM}{frequency modulation}
\newacronym{pm}{PM}{phase modulation}
\newacronym{qam}{QAM}{quadrature amplitude modulation}
\newacronym{qpsk}{QPSK}{quadrature phase-shift keying}

% cryptography
\newacronym{iid}{i.i.d.}{independent and identically distributed}
\newacronym{aes}{AES}{advanced encryption standard}
\newacronym{cvqkd}{CV-QKD}{continuous-variable quantum-key distribution}
\newacronym{dvqkd}{DV-QKD}{discrete-variable quantum-key distribution}
\newacronym{dpsqkd}{DPS-QKD}{differential phase-shift quantum-key distribution}
\newacronym{gnfs}{GNFS}{generalized number field sieve}
\newacronym{mac}{MAC}{message authentication code}
\newacronym{ldpc}{LDPC}{low-density parity-check}
\newacronym{otp}{OTP}{one-time pad}
\newacronym{pkd}{PKD}{public-key distribution}
\newacronym{qkd}{QKD}{quantum-key distribution}
\newacronym{rsa}{RSA}{Rivest–Shamir–Adleman}
\newacronym{ecdh}{ECDH}{elliptic-curve Diffie-Hellman}
\newacronym{qber}{QBER}{quantum-bit error-rate}
\newacronym{dps}{DPS}{differential-phase-shift}

\begin{document}
	\begin{titlepage}
		\begin{center}
			\large
   		    \textbf{\textsf{Experimental quantum optics}}\\
		    \vspace{0.8em}
			\huge
		    \textbf{\textsf{A theoretical framework for quantum optical communication - towards CV-QKD}}\\
			
			\vspace{1.2em}
			\begin{figure}[htb]
				\centering
			    \includestandalone[scale=1.2]{figures/pgfplots/phase-space-coherent}
			\end{figure}
			
			\vspace{.6em}
		    \large
		    \textbf{Master thesis by}\\
			\vspace{.8em}
		    \large
			Bodo Kaiser\\
		    \vspace{.2em}
			\textit{bodo.kaiser@physik.uni-muenchen.de}

		    %\usekomafont{date}
		    \large
		    \today

		    \vspace{1.9em}
			\normalsize
			\begin{tabular}{ll}
			Internal supervisor: & Prof. Dr. Monika Aidelsburger \\
			External supervisor: & Dr. Hans H. Brunner \\
			\end{tabular}
		\end{center}
	\end{titlepage}
	\tableofcontents

	\chapter{Introduction}
	\begin{refsection}
		\textit{The following chapter presents the fundamental idea of quantum-key distribution and its many facets, usually overlooked in the introductory material.}
        \documentclass[tikz]{standalone}

\usetikzlibrary{positioning}

\begin{document}
	\begin{tikzpicture}[
		node distance=1em,
		block/.style={draw, very thick, fill=white, minimum height=18ex, minimum width=8em, text width=8em, align=center},
		super block/.style={draw, very thick, fill=white, minimum height=10ex, minimum width=32em, align=center},
	]
		\coordinate (in) at (0,0);
		\node [block, right=of in] (comm) {\textsf{Communication\\and\\security}};
		\node [block, right=of comm] (comp) {\textsf{Computation\\and\\simulation}};
		\node [block, right=of comp] (sens) {\textsf{Sensing\\and\\metrology}};
		\coordinate[right=of sens] (out);

		\node [super block, below=of comp] {\textsf{Basic science}};
		\node [super block, above=of comp] {\textsf{Application}};
	\end{tikzpicture}
\end{document}

        \documentclass[tikz]{standalone}

\usetikzlibrary{arrows.meta,positioning}

\begin{document}
	\begin{tikzpicture}[
		arrow/.style={-Latex},
		block/.style={draw, very thick, fill=white, minimum height=8ex, minimum width=3.5em},
	]
		\coordinate (in) at (0,0);
		\node (pla) [right=of in] {\textsf{Plain text}};
		\node (enc) [block, right=of pla] {\textsf{Encoder}};
		\node (dec) [block, right=8em of enc] {\textsf{Decoder}};
		\node (plb) [right=of dec] {\textsf{Plain text}};
		\coordinate[right=of plb] (out);
		
		\node (keya) [below=of enc] {\textsf{Secret key}};
		\node (keyb) [below=of dec] {\textsf{Secret key}};
		
		\draw[arrow] (pla) -- (enc);
		\draw[arrow] (enc) -- (dec) node[midway, fill=white] {\textsf{Cipher text}};
		\draw[arrow] (dec) -- (plb);
		\draw[arrow] (keya) -- (enc);
		\draw[arrow] (keyb) -- (dec);
	\end{tikzpicture}
\end{document}

        \section{Problem statement}

Shortcomings of the state of the art:
 Most \gls{qkd} literature is extremely specific to the authors background, e.g., quantum optics, communication engineering, quantum information theory, while neglecting the view points of other fields.
However, one needs to the integrate all the view points and agree on a common language to get the whole picture.

To the best of our knowledge, our work is the first complete description of a \gls{cvqkd} implementation incorporating a time-dependent continuous-mode description of quantum states.

On an intellectual level, the complete description of a \gls{cvqkd} device heavily challenges us our understanding of physics.
In particular, we found that there are many implicit assumptions made to carry classical results over to a quantum model.
        \documentclass[tikz]{standalone}

\usetikzlibrary{fit,positioning}
\pgfdeclarelayer{z1}
\pgfdeclarelayer{z2}
\pgfdeclarelayer{z3}
\pgfdeclarelayer{z4}
\pgfsetlayers{main,z1,z2,z3,z4}

\begin{document}
	\begin{tikzpicture}[
		box/.style={draw, very thick, rectangle, rounded corners, fill=white, align=center, inner xsep=12pt, inner ysep=4pt},
	]
		\begin{pgfonlayer}{z4}
			\node[box, minimum height=24pt] (A) {\textsf{Quantum field theory of light}};
		\end{pgfonlayer}
		\begin{pgfonlayer}{z3}
			\node[box, fit=(A), yshift=10pt, text depth=40pt, text width=180] (B) {\textsf{Interaction theory of optical components}};
		\end{pgfonlayer}
		\begin{pgfonlayer}{z2}
			\node[box, fit=(A), yshift=22pt, text depth=80pt, text width=220pt] {\textsf{Coherent state transmission system}};
		\end{pgfonlayer}
		\begin{pgfonlayer}{z1}
			\node[box, fit=(A), yshift=34pt, text width=280pt, text depth=120pt] {\textsf{Classical post-processing}};
		\end{pgfonlayer}
	\end{tikzpicture}
\end{document}

        \section{Conventions and notation}

% Minkowski space, four vectors
% why we use p instead of omega for modes -> to distinguish between frequency and momentum

\begin{align}
	f(t)
	=
	\int_{\mathbb{R}}\frac{\dd{\omega}}{2\pi}
	f(\omega)
	e^{+i\omega t}
	&&
	f(\omega)
	=
	\int_{\mathbb{R}}\dd{t}
	f(t)
	e^{-i\omega t}
\end{align}
\begin{align}
	f(\vb{x})
	=
	\int_{\mathbb{R}^3}\frac{\dd[3]{p}}{(2\pi)^3}
	f(\vb{p})
	e^{-i\vb{p}\vdot\vb{x}}
	&&
	f(\vb{p})
	=
	\int_{\mathbb{R}^3}\dd[3]{x}
	f(\vb{x})
	e^{+i\vb{p}\vdot\vb{x}}
\end{align}
	
		\addcontentsline{toc}{section}{References}
		\printbibliography[title=References]
	\end{refsection}
	
	\chapter{Quantum-key distribution}
	\begin{refsection}
		\chapter*{Introduction}
\addcontentsline{toc}{chapter}{Introduction}

Optical communication enables humanity worldwide to share information in a split second, with companies like Huawei undergoing tremendous efforts to advance the frontiers.
In addition to incremental innovation increasing the performance and decreasing the cost of optical communication technology, we observe intensified activities towards disruptive innovations that challenge our present understanding of communication.
One such branch of activity is quantum optical communication, incorporating quantum aspects of light into classical communication and leading to novel communication technology like \gls{qkd}, which enables practical and secure key generation.
As a still young discipline, which emerged from two highly advanced fields, communication engineering and quantum physics, quantum communication lacks a unified description to which both communication engineers and quantum physicists agree.
The present thesis aims to resolve the seeming discrepancies between communication engineering and quantum physics by reviewing a practical implementation of a quantum communication system implementing a \gls{qkd} protocol.
In the process, we hope to develop a theoretical framework for quantum optical communication, incorporating quantum effects into classical communication, which has applicability beyond \gls{qkd}.

\subsection*{Problem statement}

To raise awareness of the challenges ahead, we review the best-known quantum theory of light, single-mode quantum optics, along with central ideas from classical communication and outline where these pictures conflict.

In single-mode quantum optics, we model monochromatic light with frequency $\omega_0$ as a quantum harmonic oscillator with unit mass, $m=1$, and Hamiltonian~\cite{Gerry2005,Fox2006}
\begin{equation}
	\hat{H}
	=
	\omega_0
	\hat{a}^\dagger
	\hat{a}
	,
\end{equation}
wherein $\hat{a}$ and $\hat{a}^\dagger$ are the quantum annihilation and creation operators, destroying or creating an excitation or "mode" of frequency $\omega_0$.
The electric field operator,
\begin{equation}
	\hat{E}(t,x)
	=
	\mathcal{E}_0
	\left(
		\hat{a}
		+
		\hat{a}^\dagger
	\right)
	\sin(\omega_0x)
	,
\end{equation}
wherein $\mathcal{E}_0$ has the interpretation of an electric field density, establishes the connection between the quantum harmonic oscillator and electromagnetic radiation, including light~\cite[p.~12]{Gerry2005}.
Two of the most important quantum states are the number and the coherent state,
\begin{align}
	\ket{n}
	&=
	\frac{1}{\sqrt{n!}}
	\left(\hat{a}^\dagger\right)^n
	\ket{0}
	&
	\ket{\alpha}
	&=
	\exp\left(-\frac{1}{2}\abs{\alpha}^2\right)
	\sum_{n=0}^\infty
	\frac{\alpha^n}{\sqrt{n!}}
	\ket{n}
	.
\end{align}
The number state is parametrized by a natural number $n\in\mathbb{N}_0$ counting the number excitations.
The coherent state is parametrized by a complex number $\alpha\in\mathbb{C}$ encoding amplitude and phase.
The expectation value of the electric field operator with respect to a coherent state,
\begin{equation}
	\bra{\alpha}
	\hat{E}(t)
	\ket{\alpha}
	=
	\sqrt{2}
	\abs{\alpha}
	\mathcal{E}_0
	\sin\left(\omega_0t-\theta\right)
	,
\end{equation}
equals a classical monochromatic wave with amplitude proportional to $\abs{\alpha}$ and phase $\theta$~\cite[p.~45]{Gerry2005}.

In communication engineering, light is primarily a means of transmitting signals that appear as frequency bands centered around an optical carrier frequency $\omega_c$, as illustrated in \Cref{fig:signal_spectrum}.
\begin{figure}[ht]
	\centering
	\includegraphics{figures/pgfplots/signal-spectrum}
	\caption{Receiver spectrum comprising multiple signal bands relative to a carrier frequency at $\omega=0$. At \SI{+100}{\mega\hertz}, the spectrum has a pilot tone broadened by phase noise. Centered at \SI{-25}{\mega\hertz}, the spectrum contains a first signal band with \SI{12.5}{\mega\hertz} bandwidth. Centered at \SI{-168.75}{\mega\hertz}, the spectrum contains a second signal band with \SI{12.5}{\mega\hertz} bandwidth. The remaining segments of the spectrum include mirror bands or disturbances.}\label{fig:signal_spectrum}
\end{figure}
The concept of frequency bands is extremely powerful as it allows the transmission of multiple independent signals around one carrier frequency.
More formally, we need to distinguish between base- and passband signals.
For a baseband signal $x_b(t)$, the signal power outside of the signal's bandwidth $B$ is negligible~\cite[p.~15]{Madhow2008}, i.e.,
\begin{align}
	\abs{x_b(\omega)}^2
	&\approx
	0
	&
	\abs{\omega}
	&>
	B/2
	.
\end{align}
For a passband signal $x_p(t)$, the signal power outside the signal's band around a carrier frequency $\omega_c$ is negligible~\cite[p.~16]{Madhow2008}, i.e.,
\begin{align}
	\abs{x_p(\omega)}^2
	&\approx
	0
	&
	\abs{\omega\pm\omega_c}
	&>
	B/2
	.	
\end{align}
\begin{figure}[ht]
	\centering
	\includegraphics{figures/tikz/up-conversion}
	\caption{Power spectrum illustrating up-conversion of a real-valued passband signal with bandwidth $B$ centered at $\omega_0$. Up-conversion by $\omega_c$ shifts the passband to $\omega_c+\omega_0$ and creates a mirror band at $\omega_c-\omega_0$.}\label{fig:up_conversion}
\end{figure}
A baseband signal can be up-converted to a passband signal at carrier frequency $\omega_c$ by shifting the spectrum by $\omega_c$ is known as up-conversion, see \Cref{fig:up_conversion}, and implemented by modulation.
Similar, a passband signal at carrier frequency $\omega_c$ is down-converted to a baseband signal by demodulation~\cite[p.~26]{Madhow2008}.

To sum up, single-mode quantum optics provides precise physical meaning to light, including quantum effects, although limited to monochromatic light.
On the other side, communication engineering provides a framework for efficiently constructing and transmitting signals.
For quantum optical communication, it is inevitable to welcome and incorporate both views.
For instance, people with a background in quantum optics but foreign to communication engineering often advocate the concept of "one state, one universe", where each quantum transmission is completely independent.
However, if we include practical considerations, like assuming a single transmission line, the picture of "one state, one universe" is plagued by several ambiguities.
For example, a single-mode quantum state has a single well-defined frequency $\omega_0$, which by Fourier uncertainty implies infinite temporal duration but makes information transmission absurd.
The typical counter-argument is that single-mode quantum optics implicitly assumes pulses with $\omega_0$ being the center frequency of the pulse.
While the counter-argument is technically valid, we must admit that it only raises new questions, such as bandwidth-limitations on the pulse parameters, all properly addressed in communication engineering.

The multi-mode quantum optics mentioned in popular quantum optics books~\cite{Gerry2005,Fox2006} are insufficient to represent continuous-time signals, and performing a continuum limit might not be correct if we consider the huge differences between linear algebra and functional analysis.
The advanced quantum optics literature~\cite{Vogel2006,Mandel1995} does sometimes use a continuous-mode formalism but does not explicitly investigate its properties.
We are only aware of two books~\cite{Loudon2000,Barnett2002} that explicitly present a continuous-mode theory of light but again open up new questions regarding the fundamental assumptions and justification thereof.
If we are willing to go one step deeper, we find answers in the quantum field theory literature~\cite{Peskin1995,Srednicki2007,Greiner2013,Itzykson2012}, but it is up to us to transfer these insights from particle physics to quantum optics applications.
We even have to go a bit deeper and look into mathematical quantum field theory~\cite{Streater2016,Bogoliubov1982,Bogolubov1989} to answer some questions.
Finally, we want to understand and upgrade quantum models of (electro-)optical components in the literature~\cite{Vogel2006,Leonhardt2003,Haroche2006,Mandel1995} to a mode continuum for comparison with the results from the optical communication community~\cite{Shapiro2009,Kikuchi2016}.

\subsection*{Thesis outline}

Our work is divided into four chapters.
In \Cref{ch:qkd}, we present an introduction to \gls{qkd}, emphasizing the similarities between the plethora of seemingly different protocols and attempting to argue why practical \gls{qkd} based on weak coherent states is effectively a coherent state communication system.
In the following three chapters, we construct our theoretical framework for quantum optical communication towards practical \gls{qkd}, starting from a general quantum theory of light, \Cref{ch:light}, over applying the quantum theory to describe the building blocks of coherent communication systems, \Cref{ch:components}, to an abstract description of a coherent state transmission system's signal-processing, \Cref{ch:system}.
While the thesis chapter structure supports a bottom-up approach, it is equally possible to read the thesis from the back to the front, revealing more and more details.
Likewise, it is possible to skip certain chapters and pare down to the chapter summary at the end of each chapter.
		\section{Taxonomy of protocols}

\subsection{Common features among the protocol zoo}

\Cref{fig:qkd_classification} illustrates common features among the \gls{qkd} protocols.
For an overview of \gls{qkd} protocols, see Ref.~\cite{Duvsek2006}.
\begin{figure}[htb]
	\centering
	\includestandalone{figures/tikz/qkd-classification}
	\caption{Common features among \gls{qkd} protocols: Detection, physical encoding, logical state space, measurement basis selection and schema.}\label{fig:qkd_classification}
\end{figure}
Every \gls{qkd} system requires a detector, e.g., a coherent detector or a single-photon (click) detector.
The detection does not necessarily imply the dimension of the logical state space.
For instance, BB84 having a two-dimensional logical state space has been implemented with coherent detection~\cite{Qi2021} having an uncountable detection space.
It remains to discuss where to place differential-phase-shift \gls{dps} \gls{qkd} into the proposed categories.
Concerning measurement basis selection, Bob can either actively choose a random measurement basis for every transmission or passively measure all (orthogonal) bases for measurement basis selection.
We will cover both active and passive measurement basis selection in the discussion of the polarization-encoding BB84 protocol.
Finally, the \gls{qkd} schema determines if either Alice prepares a state and sends it to Bob for measurement (prepare-and-measure) or if Alice and Bob share an entangled state (entanglement-based).
Most practical \gls{qkd} implementations use prepare-and-measure.
On a theoretical level, both schemas are equivalent, and security proofs are often more convenient in an entanglement-based setting.

\subsection{Abstract structure of a protocol}

\Cref{fig:qkd_protocol} illustrates our proposed notion of a \gls{qkd} protocol, with the feature being a logical quantum system from which the random bits are encoded and decoded.
The logical quantum system is a subspace of the physical quantum system.
The physical quantum system depends strongly on the physical implementation and quantum encoding.
\begin{figure}[htb]
	\centering
	\includestandalone{figures/tikz/qkd-protocol}
	\caption{A \gls{qkd} protocol comprises a binary encoder, a logical quantum system, and a binary decoder. The binary encoder maps bits $\vb{b}\in\{0,1\}^n$ onto a quantum state of the logical quantum system $\ket{\psi}$. The binary decoder extracts the bits $\vb{b}$ back from the quantum state $\ket{\psi^\prime}$. The logical quantum system is a subspace of a larger physical quantum system. The state encoder and decoder map between the logical and physical quantum states.}\label{fig:qkd_protocol}
\end{figure}
The distinction between logical and quantum systems is vital to separate the implementation and security concerns.
Many security proofs show equivalence between the physical implementation and the logical system to use an established security proof.
One should keep in mind that such a separation implicitly assumes no loopholes from the particular implementation.

\subsection{Comparison of qubit- and boson-based protocols}

% TODO: critic about CV-, DV-QKD, DPS-QKD

\begin{table}[htb]
	\centering	
	\begin{tabular}{lcc}
		\toprule
			& Qubit (spin) & Boson (quadrature) \\
		\midrule
			Visualization & Bloch sphere & Phase space \\
			Hilbert space (dim) & Finite (two) & Uncountable (infinite) \\
			Description & Density matrix & Wigner distribution \\
			Observable & $\vb{\hat{S}}(\vb{n})=\hat{S}_in^i$ & $\hat{X}(\vartheta)=\frac{1}{\sqrt{2}}\left(\hat{a}e^{-i\vartheta}+\hat{a}^\dagger e^{+i\vartheta}\right)$ \\
			Standard basis & $\left\{\ket{0},\ket{1}\right\}$ & $\left\{\ket{x}\right\}_{x\in\mathbb{R}}$ or $\left\{\ket{p}\right\}_{p\in\mathbb{R}}$ \\
			Basis transformation & Rotation & Unitary \\
			Other relevant states & $\ket{x_\pm},\ket{y_\pm},\ket{z_\pm}$ & $\ket{n},\ket{\alpha},\ket{\alpha,\xi}$ \\
		\bottomrule
	\end{tabular}
	\caption{Possible physical systems to encode a qubit. Many of the physical systems have a much higher dimension than the qubit space but can pre reduced to a qubit space by a proper mapping.}
\end{table}
		\section{Qubit- or spin-based protocols}

Many \gls{dvqkd} protocols, e.g., the BB84~\cite{Bennett1984} or six-state protocol~\cite{Bechmann1999}, are qubit-based in that the logical quantum system underlying the key generation is a two-state quantum system.
A qubit state $\ket{\psi}$ is an element of a two-dimensional complex Hilbert space with norm one, i.e., $\abs{\braket{\psi}}^2=1$.
In the qubit basis $\{\ket{0},\ket{1}\}$, a generic qubit state takes the form
\begin{align}
	\ket{\psi}
	=
	c_1\ket{0}
	+
	c_2\ket{1}
	&&
	\text{with}\
	\abs{c_1}^2
	+
	\abs{c_2}^2
	=
	1
	.
\end{align}
The state space equals the surface of a three-dimensional unit sphere.
A useful visualization of qubit states is the Bloch sphere, see \Cref{fig:bloch_sphere}.
\begin{figure}[htb]
	\centering
	\includestandalone{figures/tikz/bloch-sphere}
	\caption{Two-state quantum system in the Bloch sphere representation: A (pure) quantum state  $\ket{\psi}=c_1\ket{0}+c_2\ket{1}$ with $\abs{c_1}^2+\abs{c_2}^2=1$ takes a point on the surface on a unit radius sphere. By convention, the standard basis $\left\{\ket{0},\ket{1}\right\}$ is set to equal the $Z$ Pauli eigenbasis $\left\{\ket{z_+},\ket{z_-}\right\}$. The $Y,Z$ Pauli eigenbasis are both orthogonal to the $Z$ basis.}\label{fig:bloch_sphere}
\end{figure}
The Bloch sphere is embedded in three-dimensional space, and there are three independent bases.
Two antiparallel vectors on the Bloch sphere denote two orthonormal basis elements.
The Pauli algebra representing two-dimensional rotations describes state transitions on the Bloch sphere.
It is convenient to select the three Pauli eigenbasis as an orthogonal basis triplet.
By convention one identifies the $Z$ Pauli eigenbasis with the qubit basis $\left\{\ket{0},\ket{1}\right\}$.
\Cref{tab:qubit_encodings} lists different physical quantum systems  which allow encoding of a qubit.
\begin{table}[htb]
	\centering	
	\begin{tabular}{lcc}
		\toprule
		& \multicolumn{2}{c}{Standard basis} \\
		\cmidrule{2-3}
		Encoding variable & $\ket{0}$ & $\ket{1}$ \\
		\midrule
		Polarization & Horizontal & Vertical \\
		Photon number & Vacuum & Single-photon \\
		Time-bin & Early & Late \\
		Phase-bin & \SI{0}{\deg} & \SI{180}{\deg} \\
		\bottomrule
	\end{tabular}
	\caption{Possible physical systems to encode a qubit. Many of the physical systems have a much higher dimension than the qubit space but can pre reduced to a qubit space by a proper mapping.}\label{tab:qubit_encodings}
\end{table}

\begin{figure}[htb]
	\centering
	\includestandalone{figures/pgfplots/state-space-qubit}
	\caption{Two-dimensional state space spanned by the $X,Z$ Pauli eigenbases: Projecting the $\ket{x_-}$ state onto the $Z$ eigenbasis yields a constant probability amplitude of $1/\sqrt{2}$.}
\end{figure}

For qubit-based \gls{qkd}, Alice and Bob must agree on two (or three) orthogonal bases and a mapping between the basis states and some bit sequence, then
\begin{enumerate}
	\item Alice encodes her bits into the state $\ket{\psi}$ and sends it to Bob.
	\item Bob receives the state $\ket{\psi}$ from Alice and performs a measurement decoding some bits.
\end{enumerate}
If Alice and Bob select the same basis, Bob can accurately decode Alice's key bit from the measurement.
Alice and Bob's probability of choosing the same basis for one transmission is one divided by the number of orthogonal bases Alice and Bob have agreed on, e.g., \SI{50}{\percent} if Alice and Bob agreed to use the $X$ and $Z$ Pauli eigenbasis, also called the \gls{qber}.
In the asymptotic limit of many transmissions, the \gls{qber} should approach the theoretical limit.
Otherwise, an opposing third party, Eve, might have tempered with the transmission.
\Cref{tab:qubit_transmission_sequence} displays a possible transmission sequence between Alice and Bob.
\begin{table}[htb]
	\centering
	\begin{tabular}{llccccc}
		\toprule
		& & \multicolumn{5}{c}{Transmission} \\
		\cmidrule{3-7}
		Party & Step & 1 & 2 & 3 & 4 & 5 \\ 
		\midrule
		\multirow{3}{*}{Alice} & Initial key bit & \num{0} & \num{1} & \num{1} & \num{0} & \num{0} \\
		& State basis & $Z$ & $X$ & $X$ & $Z$ & $X$ \\
		& Prepared state & $\ket{z_+}$ & $\ket{x_-}$ & $\ket{x_-}$ & $\ket{z_+}$ & $\ket{x_+}$ \\
		\cmidrule{1-1}
		\multirow{3}{*}{Bob} & Measurement basis & $X$ & $Z$ & $X$ & $Z$ & $Z$ \\
		& Possible outcomes & \num{0},\num{1} & \num{0},\num{1} & \num{1} & \num{0} & \num{0},\num{1} \\
		& Sifted outcomes & - & - & 1 & 0 & - \\
		\bottomrule
	\end{tabular}
	\caption{Possible transmission sequence for qubit-based \gls{qkd}: Alice randomly selects an initial key bit \num{0} or \num{1} and a state basis $X$ or $Z$ where $X$ respective $Z$ denote the eigenbasis of the Pauli $\sigma_x$ respective $\sigma_z$ matrix. Alice's initial key bit and selected basis determine the quantum state she prepares and sends to Bob. Bob randomly chooses a measurement basis. Only if Alice's and Bob's basis agree, the key bit is not discarded.}\label{tab:qubit_transmission_sequence}
\end{table}
After the transmission sequence, Alice and Bob hold a partially correlated and partially secret bit string from which they can distill a shared secret bit string using classical post-processing.

While we formulated the measurement in terms of basis projections, we can also describe the measurement in terms of observable operators.
For example, for a set of three-dimensional unit vectors $\left\{\vu{n}\right\}$, the generalized spin operator
\begin{equation}
	\vb{\hat{S}}(\vu{n})
	=
	\hat{S}_jn^j
\end{equation}
can be used to construct a \gls{povm} for BB84-like protocols using non-orthogonal bases like BB92.

\FloatBarrier
\subsection{Polarization-encoding}

In polarization-encoding qubit-based \gls{qkd} protocols, the polarization of light is used as physical quantum system to encode the logical qubit system.
Let $\ket{\leftrightarrow}$ and $\ket{\updownarrow}$ denote the horizontal respective vertical polarization states forming the rectilinear basis.
Let $\ket{\nwsearrow}$ and $\ket{\neswarrow}$ denote the left- and right-diagonal polarization states forming the diagonal basis.
Let $\ket{\circlearrowleft}$ and $\ket{\circlearrowright}$ denote the left- and right-circular polarization states forming the circular basis.
We can express the diagonal and circular basis elements in terms of the rectilinear basis elements:
\begin{align}
	\ket{\nwsearrow}
	&=
	\frac{1}{\sqrt{2}}
	\left(
		\ket{\leftrightarrow}
		+
		\ket{\updownarrow}
	\right)
	&
	\ket{\neswarrow}
	&=
	\frac{1}{\sqrt{2}}
	\left(
		\ket{\leftrightarrow}
		-
		\ket{\updownarrow}
	\right)
	\\
	\ket{\circlearrowleft}
	&=
	\frac{1}{\sqrt{2}}
	\left(
		\ket{\leftrightarrow}
		+
		i\ket{\updownarrow}
	\right)
	&
	\ket{\circlearrowright}
	&=
	\frac{1}{\sqrt{2}}
	\left(
		\ket{\leftrightarrow}
		-
		i\ket{\updownarrow}
	\right)
\end{align}
For clarity, we restrict the following discussion to qubit-based \gls{qkd} protocols where two orthogonal bases are used, e.g., rectilinear and diagonal.
Other protocols exist that use three orthogonal bases (six-state protocol) or even non-orthogonal bases.

A possible optical setup to implement such polarization-encoding is depicted in \Cref{fig:polarization_encoding_active}.
\begin{figure}[htb]
	\centering
	\includestandalone{figures/pstricks/qubit-polarization-active}
	\caption{Optical setup to implement polarization-encoding BB84 with active basis selection at the receiver: Alice configures her linear polarizer to select a basis element of the rectilinear or diagonal polarization. Bob rotates a rectilinear polarized beam splitter by either \SI{0}{\degree} or \SI{45}{\degree} to detect either rectilinear or diagonal polarized light with his two single-photon detectors placed at the beam splitter output.}\label{fig:polarization_encoding_active}
\end{figure}
Alice selects one of four polarization states $\ket{\leftrightarrow},\ket{\updownarrow},\ket{\nwsearrow},\ket{\neswarrow}$ by adjusting her linear polarizer to one of four angles $\theta=0,\pi,\pi/2,3\pi/2$.
We can write Alice's state as
\begin{equation}
	\ket{\theta}
	=
	\frac{1}{\sqrt{2}}
	\left(
		\ket{\circlearrowleft}
		+
		e^{i\theta}
		\ket{\circlearrowright}
	\right)
	.
\end{equation}
Unrotated, Bob's rectilinear polarized beam splitter monitored by two single-photon detectors is equivalent to the \gls{povm} for detecting rectilinear-polarized light
\begin{equation}
	\biggl\{
		\hat{P}_{\leftrightarrow}
		=
		\ketbra{\leftrightarrow},
		\hat{P}_{\updownarrow}
		=
		\ketbra{\updownarrow}
	\biggr\}
	.
\end{equation}
Rotated by \SI{45}{\degree}, Bob's rectilinear polarized beam splitter monitored by two single-photon detectors is equivalent to the \gls{povm} for detecting diagonal-polarized light
\begin{equation}
	\biggl\{
		\hat{P}_{\nwsearrow}
		=
		\ketbra{\nwsearrow},
		\hat{P}_{\neswarrow}
		=
		\ketbra{\neswarrow}
	\biggr\}
	.
\end{equation}
Instead of Bob actively selecting the measurement basis, he can passively let the quantum randomness decide by splitting the photon with an unpolarized beam splitter towards a rectilinear and diagonal polarization detector.
\Cref{fig:polarization_encoding_passive} shows an optical setup implementing polarization-encoding BB84 with passive measurement basis selection.
While Alice's transmitter setup is unchanged, Bob has two polarization detectors.
One polarization detector comprises a rectilinear-polarized beam splitter and two single-photon detectors.
Another polarization detector comprises a diagonal-polarized beam splitter.
\begin{figure}[htb]
	\centering
	\includestandalone{figures/pstricks/qubit-polarization-passive}
	\caption{Optical setup to implement polarization-encoding BB84 with passive basis selection at the receiver: Alice configures her linear polarizer to select a basis element of the rectilinear or diagonal polarization. Bob first splits the incoming light into two branches. For the first branch a polarization measurement in the rectilinear basis is performed. For the second branch a polarization measurement in the diagonal basis is performed.}\label{fig:polarization_encoding_passive}
\end{figure}
The \gls{povm} describing Bob's measurement with passive basis selection is
\begin{equation}
	\biggl\{
		\frac{1}{2}\hat{P}_{\leftrightarrow},
		\frac{1}{2}\hat{P}_{\updownarrow},
		\frac{1}{2}\hat{P}_{\nwsearrow},
		\frac{1}{2}\hat{P}_{\neswarrow}
	\biggr\}
	.
\end{equation}
Bob may still have inconclusive measurements.
For instance, if he receives a horizontal polarization state $\ket{\leftrightarrow}$ and the photon chooses the path towards the diagonal polarization detector, the clicks among the two single-photon detectors are equally distributed.

The polarization of light is a qubit and we can simply relabel the polarization states with the Pauli eigenstates, i.e.,
\begin{align}
	\ket{\nwsearrow}
	=
	\ket{x_+},
	&
	\ket{\neswarrow}
	=
	\ket{x_-},
	&
	\ket{\circlearrowleft}
	=
	\ket{y_+},
	&
	\ket{\circlearrowright}
	=
	\ket{y_-},
	&
	\ket{\leftrightarrow}
	=
	\ket{z_+},
	&
	\ket{\updownarrow}
	=
	\ket{z_-}
\end{align}
to show equivalence to the general qubit description.

\FloatBarrier
\subsection{Time-phase-encoding}

In the following, we discuss the practical time-phase-encoding BB84 protocol and show its equivalence to the polarization-encoding BB84.
The idea of using phase-encoding was first proposed as part of the BB92 protocol~\cite{Bennett1992}.
The basic setup is illustrated in \Cref{fig:qubit_time_phase_active} and comprises a single-photon source and a first \gls{mzi} on Alice's side as well as a second \gls{mzi} and two single-photon detectors on Bob's side.
\begin{figure}[htb]
	\centering
	\includestandalone{figures/pstricks/qubit-time-phase-active}
	\caption{Fiber-optical setup of the phase-encoding BB84 DV-QKD protocol: Alice creates an entangled photon state using a first \gls{mzi} with phase $\theta=0,\pi/2,\pi,3\pi/2$ and sends it to Bob. Bob detects the photon state using a second \gls{mzi} with phase $\phi=0,\pi/2$ and two single-photon detectors monitoring the outputs.}\label{fig:qubit_time_phase_active}
\end{figure}

To understand the time-phase encoding, we analyze the action of the (asymmetric) \gls{mzi} with variable phase $\varphi$ on a photon pulse $\ket{t_0}$ arriving at time $t_0$, see \Cref{fig:mzi_asymmetric}.
\begin{figure}[htb]
    \centering
    \includegraphics{figures/pstricks/mzi-asymmetric}
     \caption{Asymmetric \gls{mzi} adding a constant time delay and variable phase difference between the upper and lower path: A pulsed state enters the first beam splitter BS1 to the left and is split among a longer upper path and a shorter lower path. A first mirror M1 directs the pulse from the upper path to a phase shifter which adds a relative phase of $\varphi$ between the upper and lower path. A second mirror M2 directs the pulse from the phase shifter to a second beam splitter BS2 while the lower path is between BS1 and BS2.}\label{fig:mzi_asymmetric}
\end{figure}
An ideal (lossless) and symmetric beam splitter transforms the single-photon input states into a superposition according to\footnote{See Ref.~\cite[p.~137]{Haroche2006} and Ref.~\cite[p.~143]{Gerry2005}}
\begin{align}
	\hat{U}_\text{BS}
	\ket{1,0}
	&=
	\frac{1}{\sqrt{2}}
	\left(\ket{1,0}+i\ket{0,1}\right)
	\\
	\hat{U}_\text{BS}
	\ket{0,1}
	&=
	\frac{1}{\sqrt{2}}
	\left(i\ket{1,0}+\ket{0,1}\right)
	.
\end{align}
Then, the first beam splitter BS1 in \Cref{fig:mzi_asymmetric} (instantly) splits a photon pulse $\ket{t_0}$ arriving at $t_0$ into the superposition
\begin{equation}
	\hat{U}_\text{BS}
	\ket{t_0,0}
	=
	\frac{1}{\sqrt{2}}
	\left(\ket{t_0,0}+i\ket{0,t_0}\right)
\end{equation}
where the first mode corresponds to the upper and the second mode to the lower optical path in \Cref{fig:mzi_asymmetric}.
The phase shifter adds a relative phase of $\varphi$ between the upper and lower path and the input state to the second beam splitter BS2 is
\begin{equation}
	\hat{U}_\text{PS}
	\hat{U}_\text{BS}
	\ket{t_0,0}
	=
	\frac{1}{\sqrt{2}}
	\left(
		\ket{t_0+\tau,0}
		+
		ie^{i\varphi}
		\ket{0,t_0+\tau+\Delta\tau}
	\right)
\end{equation}
wherein $\tau$ is the time delay the pulse accumulates over the short upper path and $\Delta\tau$ is the difference in time delay between the shorter, upper and longer, lower path.
The output state of BS2 is equal to the action of the \gls{mzi}
\begin{equation}
	\begin{split}
		\hat{U}_\text{MZM}
		\ket{t_0,0}
		&=
		\hat{U}_\text{BS}
		\hat{U}_\text{PS}
		\hat{U}_\text{BS}
		\ket{t_0,0}
		\\
		&=
		\frac{1}{2}
		\biggl[
			\left(
				\ket{t_0+\tau,0}
				+
				i\ket{0,t_0+\tau}
			\right)
			+
			ie^{i\varphi}
			\left(
				i\ket{t_0+\tau+\Delta\tau,0}
				+
				\ket{0,t_0+\tau+\Delta\tau}
			\right)
		\biggr]
		\\
		&=
		\frac{1}{2}
		\biggl[
			\ket{t_0+\tau,0}
			-
			e^{i\varphi}
			\ket{t_0+\tau+\Delta\tau,0}
			+
			i
			\left(
				\ket{0,t_0+\tau}
				+
				e^{i\varphi}
				\ket{0,t_0+\tau+\Delta\tau}
			\right)
		\biggr]
		.
	\end{split}
	\label{eq:mzi_asymmetric}
\end{equation}

Back to the time-phase-encoding BB84 setup depicted in \Cref{fig:qubit_time_phase_active}, we note that Alice's transmitter consists of a single-photon source and an asymmetric \gls{mzi} where on output is dumped.
Therefore, Alice's states are parametrized by the relative phase $\theta$,
\begin{equation}
	\ket{t_0,\theta}
	=
	\frac{1}{\sqrt{2}}
	\left(
		\ket{t_0}
		-
		e^{i\theta}
		\ket{t_0+\Delta\tau}
	\right)
	,
\end{equation}
which is obtained from \cref{eq:mzi_asymmetric} by absorbing the time delay of the shorter path $\tau$ into the time reference $t_0$ and projecting the first output mode of the \gls{mzi}.
By adding a time delay to the states of $\Delta\tau$, we receive the states for the time-delayed signal
\begin{align}
	\ket{t_1,\phi}_1
	&=
	\frac{1}{\sqrt{2}}
	\left(
		\ket{t_1}
		-
		e^{i\phi}
		\ket{t_1+\Delta\tau}
	\right)
	\\
	\ket{t_1,\phi}_2
	&=
	\frac{i}{\sqrt{2}}
	\left(
		\ket{t_1}
		+
		e^{i\phi}
		\ket{t_1+\Delta\tau}
	\right)
	\label{eq:qubit_time_phase_bob_nodelay}
\end{align}
where we again choose the time $t_1$ such that it cancels the time delay of the short path $\tau$.
If Bob receives a pulse with time delay $\Delta\tau$ at some time $t_1$, i.e., $\ket{t_1+\Delta\tau}$, then his \gls{mzi} provides the two detectors with the states
\begin{align}
	\ket{t_1+\Delta\tau,\phi}_1
	&=
	\frac{1}{\sqrt{2}}
	\left(
		\ket{t_1+\Delta\tau}
		-
		e^{i\phi}
		\ket{t_1+2\Delta\tau}
	\right)
	\\
	\ket{t_1+\Delta\tau,\phi}_2
	&=
	\frac{i}{\sqrt{2}}
	\left(
		\ket{t_1+\Delta\tau}
		+
		e^{i\phi}
		\ket{t_1+2\Delta\tau}
	\right)
	.
	\label{eq:qubit_time_phase_bob}
\end{align}
We note that these are superpositions of states at three different time instances $0,\Delta\tau,2\Delta\tau$.
We drop the pulse time and introduce the state
\begin{equation}
	\ket{\Delta\tau=m}
	=
	\ket{t_1+m\Delta\tau}
\end{equation}
corresponding to the $m$th detection time slot.
In the new notation, Bob's detectors receive a superposition of Alice's states, \cref{eq:qubit_time_phase_bob} and \cref{eq:qubit_time_phase_bob_delayed},
\begin{equation}
	\ket{\theta,\phi}_\pm
	=
	\frac{c_\pm(\theta-\phi)}{\sqrt{2}}
	\biggl[
		\ket{\Delta\tau=0}
		\mp
		\left(
			e^{i\phi}
			\pm
			e^{i\theta}
		\right)
		\ket{\Delta\tau=1}
		\pm
		e^{i(\phi+\theta)}
		\ket{\Delta\tau=2}
	\biggr]
	\label{eq:qubit_time_phase_bob_delayed}
\end{equation}
with phase-dependent normalization constant
\begin{equation}
	c_\pm(\theta-\phi)
	=
	\frac{1}{\sqrt{2\pm\cos(\theta-\phi)}}
	.
\end{equation}
Using the \gls{povm} for detecting a click at time slot $m$,
\begin{equation}
	\left\{
		\hat{P}_m
		=
		\ketbra{\Delta\tau=m}
	\right\}_{m=0,1,2}
	,
\end{equation}
we find the click probabilities for the plus and minus detectors to equal
\begin{equation}
	\begin{split}
		p_{\pm,m}
		=
		\trace{\hat\rho_\pm\hat{P}_m}
		&=
		\expval{\hat{P}_m}{\theta,\phi}_{\pm}
		\\
		&=
		\begin{cases}
			\abs{c_\pm(\theta-\phi)}^2\left(1\pm\cos(\theta-\phi)\right) & m=1 \\
			\abs{c_\pm(\theta-\phi)}^2\frac{1}{2} & m=0,2
		\end{cases}
		.
	\end{split}
\end{equation}
If we configure the detectors to trigger only on the $m=1$ time slot, we find the probability for a click of the plus and minus detectors to be
\begin{equation}
	p_\pm(\theta-\phi)
	=
	\frac{1}{2}
	\left(1\pm\cos(\theta-\phi)\right)
	.
\end{equation}
\Cref{tab:qubit_time_phase_clicks} summarizes the click probability of the plus and minus detectors triggered on the $m=1$ time slot for a restricted choice of phases.
\begin{table}[htb]
	\centering
	\begin{tabular}{cccc}
		\toprule
		\multicolumn{2}{c}{Phase} & \multicolumn{2}{c}{Detector click probability} \\
		\cmidrule{1-2}
		\cmidrule{3-4}
		$\theta$ & $\phi$ & $p_1(\theta-\phi)$ & $p_2(\theta-\phi)$ \\
		\midrule
		\multirow{2}{*}{$0$} & $0$ & \SI{100}{\percent} & \SI{0}{\percent} \\
		& $\pi/2$ & \SI{50}{\percent} & \SI{50}{\percent} \\
		\cmidrule{1-2}
		\multirow{2}{*}{$\pi$} & $0$ & \SI{0}{\percent} & \SI{100}{\percent} \\
		& $\pi/2$ & \SI{50}{\percent} & \SI{50}{\percent} \\
		\cmidrule{1-2}
		\multirow{2}{*}{$\pi/2$} & $0$ & \SI{50}{\percent} & \SI{50}{\percent} \\
		& $\pi/2$ & \SI{100}{\percent} & \SI{0}{\percent} \\
		\cmidrule{1-2}
		\multirow{2}{*}{$3\pi/2$} & $0$ & \SI{50}{\percent} & \SI{50}{\percent} \\
		& $\pi/2$ & \SI{0}{\percent} & \SI{100}{\percent} \\
		\bottomrule
	\end{tabular}
	\caption{Click probabilities for the time-phase-encoding BB84 protocol: Alice choosing $\theta=0,\pi$ corresponds to choosing the $Z$ eigenbasis while $\theta=\pi/2,3\pi/2$ corresponds to her choosing the $X$ eigenbasis. Bob using no phase shift $\phi=0$ corresponds to a selection of the $X$ basis while Bob adding a phase shift of $\phi=\pi/2$ corresponds to selection of $X$ as measurement basis. Only if Alice and Bob choose the same basis, Bob's click is perfectly correlated with Alice's choice for a basis element. Otherwise, it is completely random.}\label{tab:qubit_time_phase_clicks}
\end{table}
Comparing the click probabilities of \Cref{tab:qubit_time_phase_clicks} with the probabilities of the qubit-based BB84, suggests equivalence of the time-phase-encoding BB84 with the more general qubit-based description of BB84.
While we can identify Alice's state in the time basis in terms with the $Y$ qubit basis,
\begin{equation}
	\ket{t_0,\theta}
	=
	\frac{1}{\sqrt{2}}
	\left(
		\ket{t_0}
		-
		e^{i\theta}
		\ket{t_0+\Delta\tau}
	\right)
	=
	\frac{1}{\sqrt{2}}
	\left(
		\ket{y_+}
		-
		e^{i\theta}
		\ket{y_-}
	\right)
	=
	\ket{\theta}
	,
\end{equation}
the receiver side cannot simply be relabeled into the qubit-based description:
Bob's Hilbert space, spanned by the three time slot states, $\ket{\Delta\tau=m}_{m=0,1,2}$, has one additional dimension compared to the qubit Hilbert space.
Such complication can be addressed using the "squashing"~\cite{Beaudry2008,Gittsovich2014}:
We first find a unitary transformation for the input mode of Bob's receiver.
Second, we show that the \gls{povm} yields the same probability distribution as the qubit-based description for all possible quantum states.
The number state basis $\left\{\ket{n}\right\}_{n\in\mathbb{N}_0}$ is complete and countable allowing a proof by induction.
It is important to show equivalence for all number states as Eve's not limited to the single-photon state.
		\section{Boson- or quadrature-based protocols}

The quantum system of interest in boson-based protocols is a single bosonic mode, i.e., a quantum harmonic oscillator.

The central observable is the generalized quadrature operator~\cite[p.~36]{Barnett2002}
\begin{equation}
	\hat{X}(\vartheta)
	=
	\frac{1}{\sqrt{2}}
	\left(
		\hat{a}
		e^{-i\vartheta}
		+
		\hat{a}^\dagger
		e^{+i\vartheta}
	\right)
\end{equation}
wherein $\hat{a}^\dagger,\hat{a}$ are the bosonic creation and annihilation operators satisfying the canonical commutation relation
\begin{align}
	\comm{\hat{a}}{\hat{a}^\dagger}
	=
	1
	&&
	\comm{\hat{a}}{\hat{a}}
	=
	0
	=
	\comm{\hat{a}^\dagger}{\hat{a}^\dagger}
	.
\end{align}
It follows that the generalized quadrature operator satisfies the commutator
\begin{equation}
	\comm{\hat{X}(\vartheta)}{\hat{X}(\vartheta+\Delta\vartheta)}
	=
	i\sin\Delta\vartheta
	.
\end{equation}
The Robertson uncertainty relation provides a lower bound for the product of the standard deviation of two operators in terms of their commutator.
The Robertson uncertainty relation for the generalized quadrature operator,
\begin{equation}
	\expval{\Delta\hat{X}(\vartheta)}
	\expval{\Delta\hat{X}(\vartheta+\Delta\vartheta)}
	\geq
	\frac{1}{2}
	\abs{\expval{\comm{\hat{X}(\vartheta)}{\hat{X}(\vartheta+\Delta\vartheta)}}}
	=
	\frac{1}{2}
	\sin\Delta\vartheta
	,
\end{equation}
generalizes Heisenberg's uncertainty relation and implies maximal uncertainty for orthogonal quadratures $\Delta\vartheta=\pi/2$.
Let us assume the existence of an eigenstate $\ket{x,\vartheta}$ of the generalized quadrature operator $\Delta\hat{X}(\vartheta)$ with eigenvalue $x\in\mathbb{R}$\footnote{Actually, the quadrature eigenstates only exist on the extended Hilbert space as they itself are not square-integrable.}, i.e.,
\begin{equation}
	\hat{X}(\vartheta)
	\ket{x,\vartheta}
	=
	x(\vartheta)
	\ket{x,\vartheta}
	,
\end{equation}
then the uncertainty relation implies that $\ket{x,\vartheta}$ and $\ket{x(\vartheta+\Delta\vartheta)}$, corresponding to position and momentum, are conjugate variables, i.e., increasing the precision of one variable decreases the precision of the other.
Unsurprisingly, we can show that these eigenstates are non-orthogonal~\cite[p.~29]{Mukhanov2007}
\begin{equation}
	\braket{x,\vartheta}{x,\vartheta+\Delta\vartheta}
	=
	\frac{e^{ipx}}{\sqrt{2\pi}}
	\label{eq:position_momentum_product}
\end{equation}
and related by a Fourier transform.\footnote{This can be shown combining the completeness relation and \cref{eq:position_momentum_product}.}

The non-orthogonality of the quadrature eigenstates, makes the bosonic system a candidate for \gls{qkd}.
For instance, we can envision boson-based BB84 protocol:
\begin{enumerate}
	\item Alice prepares the state $\ket{x(\vartheta+\Delta\vartheta)}$ where she randomly picks $x=\pm x_0$ and $\Delta\vartheta=0,\pi/2$.
	\item Bob performs a homodyne measurement represented by the \gls{povm} $\left\{\ketbra{x,\theta+\Delta\theta}\right\}_{x\in\mathbb{R}}$.
\end{enumerate}
If Bob measures in the correct basis, he (theoretically) is able to resolve perfectly $x_i,p_j\in\mathbb{R}$.
Otherwise, Bob measures an outcome completely uncorrelated with what Alice has prepared, see \Cref{tab:boson_transmission_sequence}.
\begin{table}[htb]
	\centering
	\begin{tabular}{llccccc}
		\toprule
		& & \multicolumn{5}{c}{Transmission} \\
		\cmidrule{3-7}
		Party & Step & 1 & 2 & 3 & 4 & 5 \\ 
		\midrule
		\multirow{3}{*}{Alice} & State value & $p_1$ & $x_1$ & $x_2$ & $p_2$ & $x_3$ \\
		& State basis & $P$ & $X$ & $X$ & $P$ & $X$ \\
		& Prepared state & $\ket{p_1}$ & $\ket{x_1}$ & $\ket{x_2}$ & $\ket{p_2}$ & $\ket{x_3}$ \\
		\cmidrule{1-1}
		\multirow{2}{*}{Bob} & Measurement basis & $X$ & $P$ & $X$ & $P$ & $P$ \\
		& Sifted outcome & - & - & $x_2$ & $p_2$ & - \\
		\bottomrule
	\end{tabular}
	\caption{Possible transmission sequence for boson-based BB84: Alice randomly selects a value and a basis, encodes this information into a quantum state and sends it to Bob. Bob randomly selects a measurement outcome. Only if Alice's and Bob's basis match, is Bob's outcome correlated with Alice's value.}\label{tab:boson_transmission_sequence}
\end{table}
To convert the sifted outcome to bits, we can simply assign the bit value according to the sign.
While the suggested boson-based BB84 highlights the differences of boson- with qubit-based \gls{qkd} it cannot be implemented as there are no physical position or momentum eigenstates as there is always an uncertainty involved.

However, we can use squeezed states as an approximation to the position and momentum eigenstates.
We are going to discuss such a squeeze-encoding in the subsequent section.

\FloatBarrier
\subsection{Squeezed-coherent-encoding}

A squeezed coherent state, denoted $\ket{\alpha,\xi}$, has expected quadrature~\cite[p.~91,94]{Vogel2006}
\begin{equation}
	\expval{\hat{X}(\vartheta)}{\alpha,\xi}
	=
	\frac{1}{\sqrt{2}}
	\left[
		\left(\mu\alpha-\nu\alpha^*\right)
		e^{+i\vartheta}
		+
		\text{c.c.}
	\right]
\end{equation}
with standard deviation~\cite[p.~95]{Vogel2006}
\begin{equation}
	\expval{\Delta\hat{X}(\vartheta)}{\alpha,\xi}
	=
	\abs{\mu e^{+i\vartheta}-\nu^* e^{-i\vartheta}}
	\label{eq:squeezed_quadrature_std}
\end{equation}
wherein parameters $\nu,\mu$ relate to the complex squeezing parameter $\xi=\abs{\xi}e^{i\varphi_\xi}$ via~\cite[p.~90]{Vogel2006}
\begin{align}
	\mu
	&=
	\cosh\abs{\xi}
	=
	1+\abs{\nu}^2
	&
	\nu
	&=
	e^{i\varphi_\xi}
	\sinh\abs{\xi}
	=
	\abs{\nu}
	e^{i\varphi_\xi}
	\label{eq:squeezing_parameters}
	.
\end{align}
Inserting the polar representation of the squeezing parameters \cref{eq:squeezing_parameters} into the qaudrature standard deviation \cref{eq:squeezed_quadrature_std}, we find that there exists two phase relation between $\vartheta$ and $\varphi_\xi$ which yield minimum and maximum quadrature uncertainties~\cite[p.~96]{Vogel2006}
\begin{equation}
	\expval{\Delta\hat{X}(\vartheta)}{\alpha,\xi}_{\vartheta=\vartheta_{\text{max}/\text{min}}}
	=
	e^{\pm\abs{\xi}}
	.
\end{equation}
In the limit of infinite squeezing magnitude $\abs{\xi}\to\infty$, we obtain position and momentum eigenstates for $\hat{X}(\vartheta_\text{min})$ and $\hat{X}(\vartheta_\text{max})=\hat{X}(\vartheta_\text{min}+\pi/2)$.
\Cref{fig:phase_space_squeezed} depicts the optical phase space of two highly squeezed states.
The green ellipsis indicates the variances of a $\hat{P}$-quadrature squeezed state and the orange ellipsis indicated the variance of a $\hat{X}$-quadrature squeezed state.
If Alice prepares such states and Bob attempts to measure them, then his outcome will have almost no correlation if he selects the unsqueezed quadrature as measurement basis.
\begin{figure}[htb]
	\centering
	\includestandalone{figures/pgfplots/phase-space-squeezed}
	\caption{Phase space representation of squeezed-coherent states where the minimum uncertainty is in the $\hat{X}$ quadrature (orange) and the $\hat{P}$ quadrature (green).}\label{fig:phase_space_squeezed}
\end{figure}
To implement the quadrature measurement, we can employ a homodyne detection.
A fiber-optical setup for homodyne detection is depicted in \Cref{fig:coherent_receiver_active}.
At its heart, the homodyne detector consists of a \gls{lo}, a balanced coupler (or beam splitter) and two photodiodes in balanced configuration.
The \gls{lo} is superimposed with the signal through the coupler and the two coupler outputs are monitored by one photodiode.
In balanced configuration, the photocurrent of the photodiodes is subtracted removing the constant power of the signal and \gls{lo}.
\begin{figure}[htb]
	\centering
	\includestandalone{figures/pstricks/coherent-receiver-active}
	\caption{Fiber-optical setup of a single homodyne detector implementing active measurement basis section: A \gls{lo} is synced to the optical carrier and phase-shifted by either $0,\pi/2$ to select one of two orthogonal quadratures. The phase-shifted \gls{lo} is superimposed with the received signal in a balanced coupler. The two coupler outputs are monitored by photodiodes in balanced configuration.}\label{fig:coherent_receiver_active}
\end{figure}
Assuming a perfect detector and strong \gls{lo} with coherent state $\ket{\alpha_l}$ and $\abs{\alpha_l}\gg1$, the mean balanced photodiode current is proportional to~\cite[p.~217]{Vogel2006}
\begin{equation}
	\expval{\Delta\hat{N}^\prime}
	=
	\expval{\hat{N}_1^\prime}
	-
	\expval{\hat{N}_2^\prime}
	=
	\abs{\alpha_l}
	\expval{\hat{X}(\vartheta)}
\end{equation}
wherein $\vartheta$ is the phase difference between the signal and the \gls{lo}.
Moreover, it can be shown that the \gls{povm} of an ideal homodyne detector is~\cite[p.~220]{Vogel2006}
\begin{equation}
	\left\{\hat{P}_{\Delta n}=\frac{1}{\abs{\alpha_l}}\ketbra{x,\vartheta}\right\}_{\Delta n\in\mathbb{Z}}
\end{equation}
wherein $\ket{x,\vartheta}$ has quadrature eigenvalue $x=\Delta n/\abs{\alpha_l}$.
The single homodyne detector corresponds to an active measurement basis selection of Bob.
As in the case of the polarization-encoding qubit-based \gls{qkd}, Bob can also use a second homodyne detector to implement passive measurement basis selection.
Such a setup is illustrated in \Cref{fig:coherent_receiver_passive}.
\begin{figure}[htb]
	\centering
	\includestandalone{figures/pstricks/coherent-receiver-passive}
	\caption{Fiber-optical setup of a dual homodyne detector implementing passive measurement basis section: A \gls{lo} is synced to the optical carrier and split into two branches. The first branch is used for a first homodyne detector while the second branch is phase-shifted by $\pi/2$ and used for a second homodyne detector. As the two homodyne detectors have orthogonal \gls{lo}, the two orthogonal quadratures can be resolved at the same time.}\label{fig:coherent_receiver_passive}
\end{figure}
The squeezed-coherent-encoding is a more realistic implementation of a boson-based \gls{qkd} protocol as squeezed-coherent states are physically contrary to position and momentum states.
The production of squeezed-coherent states requires nonlinear interactions, which are challenging to control.
Furthermore, squeezed-coherent states quickly lose their squeezing by attenuation.

\FloatBarrier
\subsection{Coherent-encoding}

Coherent-encoding is the most practical encoding for boson-based \gls{qkd}:
It is easier to create and manipulate coherent states using standard telecommunication hardware, and channel loss only deteriorates the amplitude but does not change the kind of quantum state.
The expected quadrature of a coherent state $\ket{\alpha}$ is
\begin{equation}
	\expval{\hat{X}(\vartheta)}{\alpha}
	=
	\frac{1}{\sqrt{2}}
	\left(
		\alpha
		e^{-i\vartheta}
		+
		\alpha^*
		e^{+i\vartheta}
	\right)
	=
	\sqrt{2}
	\Re\left(
		\alpha
		e^{-i\vartheta}
	\right)
	=
	\sqrt{2}
	\abs{\alpha}
	\cos(\phi-\vartheta)
\end{equation}
where we used the complex polar representation $\alpha=\abs{\alpha}e^{i\phi}$ and variance~\cite[p.~59]{Barnett2002}
\begin{equation}
	\expval{\Delta\hat{X}(\vartheta)}{\alpha}
	=
	\frac{1}{\sqrt{2}}
	.
\end{equation}
Consequently, coherent states have equal uncertainty in both quadratures and Bob's measurements will always be off Alice's encoded symbol requiring sophisticated error correction techniques.
Let us use visualize the quantum transmission of coherent-encoding boson-based \gls{qkd} in phase space.
\Cref{fig:phase_space_coherent} depicts the variance of Alice's prepared coherent state (blue) and how it looses amplitude through attenuation of the channel (green).
The variance of Bob's measurement is highlighted by the orange circle which includes electronic noise from the detector.
\begin{figure}[htb]
	\centering
	\includestandalone{figures/pgfplots/phase-space-coherent}
	\caption{Phase space representation of quantum transmission in coherent-encoding boson-based \gls{qkd}: Alice prepares a coherent state with mean and shot noise variance (blue circle). Bob receives the attenuated coherent state with mean and shot noise variance (green circle). Bob's measurement (orange circle) has an increased variance due to other noise sources, e.g., electronic noise.}\label{fig:phase_space_coherent}
\end{figure}
If Eve attempts an intercept-resend attack (\Cref{fig:phase_space_intercept_resend}), she measures Alice's coherent state with some outcome most likely in the blue circle but unlikely to be the exact value Alice used for encoding.
Eve's best guess is to prepare a new state onto which she encodes her measurement outcome (red circle).
The channel attenuation reduces the power of Eve's state.
Including electronic noise, Bob measures an outcome most likely inside the orange circle.
When Alice and Bob perform error correction they will notice a higher than usual error from Eve's attempt to copy Alice's state.
\begin{figure}[htb]
	\centering
	\includestandalone{figures/pgfplots/phase-space-intercept-resend}
	\caption{Phase space representation of quantum transmission in coherent-encoding boson-based \gls{qkd} including intercept-resend attack from Eve: Alice prepares a coherent state with mean and shot noise variance (blue circle). Eve intercepts the state and resends a new state to Bob (red circle). Bob receives the Eve's modified coherent state and performs a measurement. When Alice and Bob perform error correction the imperfect copy Eve sent increases the error noticeable.}\label{fig:phase_space_intercept_resend}
\end{figure}
\Cref{fig:phase_space_intercept_resend_squeezed} illustrates Eve's intercept-resent attack if she uses a squeezed state in an attempt to "hide" her error.
\begin{figure}[htb]
	\centering
	\includestandalone{figures/pgfplots/phase-space-intercept-resend-squeezed}
	\caption{Phase space representation of quantum transmission in coherent-encoding boson-based \gls{qkd} including intercept-resend attack from Eve using squeezing: Eve intercepts Alice's coherent state and prepares a squeezed state to reduce the error she introduced. Bobs measurement variance is now comparable to an untampered transmission for one quadrature but the variance of the other quadrature is dramatically increased. If Bob uses passive measurement basis selection, he will instantly notice the high error in the other quadrature. If Bob uses active measurement basis selection, he will too notice a large error introduced by Eve in the asymptotic limit of many measurements.}\label{fig:phase_space_intercept_resend_squeezed}
\end{figure}
If Bob uses a dual homodyne receiver, he will directly notice the increase of noise in one of the quadratures.
If Bob uses a single homodyne receiver, he will only notice every second measurement on average, which is still sufficient to detect Eve's tempering.

To implement coherent-encoding, we can use the same receiver setup as described for squeezed-coherent-encoding.
A possible transmitter setup is presented in \label{fig:coherent_transmitter}.
\begin{figure}[htb]
	\centering
	\includestandalone{figures/pstricks/coherent-transmitter}
	\caption{Fiber-optical setup of a coherent transmitter for coherent-encoding boson-based \gls{qkd}: \gls{iq}-modulation is performed on a transmit laser to encode the complex symbol onto the optical carrier. A \gls{voa} reduces the power of the transmit signal such that the signal strength is comparable to the quantum noise.}\label{fig:coherent_transmitter}
\end{figure}
		\section{Post-processing}

\begin{enumerate}
	\item bit mapping (encoding,decoding)
	\item error correction and estimation
	\item privacy amplification
\end{enumerate}


% aim of classical post-processing (correlated variables -> shared secret, estimate error -> protocol abortion)}

% what about (base) sifting?

\begin{figure}[htb]
	\centering
	\includestandalone{figures/tikz/post-processing}
	\caption{\Gls{qkd} transmission system from a signal-processing perspective.}
\end{figure}

% citations
\cite{Silberhorn2002} % post-selection mechanism to mitigate beam splitter attack
\cite{Fung2010} % security analysis and overview of post-processing

%\subsection{Reconciliation}
% reconciliation
\cite{Leverrier2008} % multidimensional (sphere) 
\cite{Elkouss2011} % simpler reconciliation scheme

\subsection{Information reconciliation}

Information reconciliation summarizes methods required for Alice and Bob to agree on shared data.
It includes error correction, and discarding of data failed to correct.

Let us first consider procedures for error correction.
Error correction is a subdiscipline of coding theory, or more precisely, channel coding, which studies the arrangement of data for efficient and reliable transmission, see \Cref{fig:error_correction_codes}.
\begin{figure}[htb]
	\centering
	\includestandalone{figures/tikz/error-correction-codes}
	\caption{Taxonomy of codes in coding theory with emphasis on linear block codes for error correction.}\label{fig:error_correction_codes}
\end{figure}

\subsection{Privacy amplification}

% XOR-ing using Toeplitz matrices

\cite{Bennett1995} % Generalized privacy amplfication
		\section{Security analysis}

For completeness, we provide a brief introduction to the security analysis of QKD in which we show under which assumptions a \gls{qkd} protocol is secure.
An overview of security proofs, including background information, can be found in Ref.~\cite{Scarani2009}.
For a security analysis of \gls{cvqkd}, see Ref.~\cite{Diamanti2015} and Ref.~\cite{Laudenbach2018}. 
A mathematical treatment using recent information-theoretical tools, see Ref.~\cite{Wolf2021}.

Every \gls{qkd} security proof assumes fundamentally~\cite[p.~10]{Scarani2009}:
\begin{enumerate}
	\item Quantum theory to be complete and correct.
	\item Authenticated communication to be possible.
\end{enumerate}
The first assumption provides us with the framework of quantum (information) theory to formulate our proof.
Furthermore, it states that an adversary is only limited by physical - not technological - means.
The second assumption is vital to exclude man-in-the-middle attacks from an adversary.
It can be practical implemented using \gls{mac}s, for a security proof of Wegman-Carter-Shoup-type authenticators, see Ref.~\cite{Bernstein2005}.
Most security proofs further assume ideal implementation~\cite[p.~124]{Wolf2021}:
\begin{enumerate}
	\item Isolation of the transmitter and receiver from the adversary.
	\item Perfect quantum state preparation and measurement.
	\item True randomness in the state and bases selection.
	\item Perfect timing and synchronization of the transmitter and receiver.
	\item Post-processing protocols are secure and work as intended.
\end{enumerate}
After establishing the security proof of the ideal protocol implementation, we can discuss side-channel attacks originating from imperfect implementations separately.
For example, Ref.~\cite[p.~8]{Lo2014} discusses attacks due to hardware imperfections, Ref.~\cite{Fung2010} analysis the security of a practical post-processing pipeline for BB84, and Ref.~\cite{Renner2005} gives a security proof of privacy amplification in the context of \gls{qkd}.

So far, we have been rather vague about the notion of security.
In particular, we need to parametrize the security of a key as there is no strict security.
For example, consider the security of a binary key of length $n$.
The probability for an adversary to guess the correct key is $\varepsilon=2^{-n}$.
Such a brute-force attack marks the absolute floor of a key's security which we refer to as $\varepsilon$-secure.

More formally, we define a $\epsilon$-secure key obtained by a \gls{qkd} protocol to satisfy~\cite[p.~10]{Scarani2009}
\begin{equation}
	\frac{1}{2}
	\norm{\rho_{AE}-\rho_U\otimes\rho_E}_{\trace{}}
	\leq
	\varepsilon
	\label{eq:qkd_security}
\end{equation}
where $\rho_{AE}$ is the quantum state encoding the correlations between Alice's final key\footnote{It is sufficient to only consider Alice's state as Bob's shares the exact same state after post-processing.} and Eve, $\rho_U$ is the mixed state of possible key configurations, and $\rho_E$ is a generic state of Eve.
Intuitively \Cref{eq:qkd_security} encodes the distance between an ideal key state $\rho_U\otimes\rho_E$ and a real key state $\rho_{AE}$.
The real key state $\rho_{AE}$ may be entangled with Eve's system while for the ideal key state, Eve's state $\rho_E$ factorizes as a tensor product with the key state $\rho_U$, i.e., $\rho_E$ and $\rho_U$ describe independent systems.
Further definitions with respect to security, for instance, $\epsilon$-correctness, -robustness, and composability, are formalized in Ref.~\cite[p.~119]{Wolf2021}.

The ultimate objective of a security proof is to prove an inequality of the form~\cite[p.~11]{Scarani2009}
\begin{equation}
	\mathbb{P}\left[
		\frac{1}{2}
		\norm{\rho_{AE}-\rho_U\otimes\rho_E}_{\trace{}}
		\leq
		\varepsilon
	\right]
	\lesssim
	e^{l-F(\rho_{AE},\varepsilon)}
\end{equation}
where $l$ is the secret key length and $F$ encodes the information leakage to Eve.
Alternatively, one can derive a lower bound for the secret key rate~\cite{Brunner2017}
\begin{equation}
	r_\text{sec}
	\propto
	r_\text{raw}
	(1-\nu)
	(\beta I_{AB}-\chi_{BE})
\end{equation}
wherein $r_\text{raw}$ is the raw transmission rate, $\nu$ is the fraction of data revealed for parameter estimation, $\beta$ is the error correction efficiency, $I_{AB}$ is the mutual information between Alice and Bob, and $\chi_{BE}$ is the Holevo information encoding Eve's information on Bob's measurements.
% TODO: plots showing the secret key rate with different parameters

\begin{table}[htb]
	\centering
	\begin{tabular}{c}
		\toprule
			Attack \\
		\midrule
			Individual \\
			Collective \\
			Coherent \\
		\bottomrule
	\end{tabular}
	\caption{Summary of Eve's attacks:~\cite[p.~128]{Wolf2021}.}
\end{table}
A systematic approach to security proofs first converts a prepare-and-measure to an entanglement-based protocol.
See, for example, Ref.~\cite[p.~106]{Wolf2021} to show equivalence between prepare-and-measure and entanglement-based BB84.
In the entanglement-based picture, quantum post-processing steps distill key qubit states from the quantum system of Alice and Eve.
The qubit states attributed to Eve's state after quantum post-processing directly estimate Eve's information about the key.
Shor and Peskill first suggested the entanglement distillation-based security proofs in providing a simple proof for BB84~\cite{Shor2000}.
Finally, one shows the equivalence of the quantum post-processing to the classical post-processing.

		\addcontentsline{toc}{section}{Summary}
		\section*{Summary}
\addcontentsline{toc}{section}{Summary}

In the present chapter, we introduced \gls{qkd} as an example for quantum optical communication and as a mechanism for practical and secure key distribution, which, together with classical symmetric ciphers, enables secure communication with means to estimate information leakage.
We then analyzed the quantum transmission phase of qubit- and boson-based \gls{qkd} protocols generating correlated information between the receiver and transmitter.
\begin{table}[htb]
	\centering	
	\begin{tabular}{lcc}
		\toprule
			& Qubit-based & Boson-based \\
		\midrule
			Visualization & Bloch sphere & Phase space \\
			Hilbert space (dim) & Finite (two) & Countable (infinite) \\
			Measurement operator & $\vb{\hat{S}}(\vb{n})=\hat{S}_in^i$ & $\hat{X}(\vartheta)=\frac{1}{\sqrt{2}}\left(\hat{a}e^{-i\vartheta}+\hat{a}^\dagger e^{+i\vartheta}\right)$ \\
			Standard basis & $\left\{\ket{0},\ket{1}\right\}$ & $\left\{\ket{x},\ket{p}\colon x,p\in\mathbb{R}\right\}$ \\
		\bottomrule
	\end{tabular}
	\caption{Comparison of qubit- and boson-based \gls{qkd} protocols.}\label{tab:qkd_comparison}
\end{table}
By introducing the concept of qubit- and boson-based \gls{qkd} protocols, with their respective key properties summarized in \Cref{tab:qkd_comparison}, we formalized the concept of DV- and CV-QKD.
Additionally, we formulated the concept of a logical and an encoding quantum system allowing us to encode qubits onto number or coherent states, which might share some similarity with the concept of symbols and pulse-shaping in classical signal-processing.
Because the technology to prepare and measure coherent states is highly mature, most practical QKD implementations, regardless of qubit- or boson-based, transmit weak coherent states.
\begin{figure}[htb]
	\centering
	\includegraphics{figures/tikz/qkd-protocol}
	\caption{An abstract \gls{qkd} protocol comprises a binary encoder, a logical quantum system, a binary decoder, and some post-processing. The binary encoder maps bits onto a quantum state of the logical quantum system. The binary decoder extracts the bits from the logical quantum system. The logical quantum system is a subspace of a larger physical quantum system. The state encoder and decoder map between the logical and physical quantum states.}\label{fig:qkd_protocol}
\end{figure}
Given the correlated data from the quantum transmission, we compiled classical methods to distill a shared secret key between the transmitter and receiver, known as classical post-processing. 
The classical post-processing maps the discrete or continuous data from the transmission sequence to binary symbols, corrects errors, discards failed data blocks, and removes information from the partially secret key using privacy amplification.
Finally, we roughly outlined some ideas for the security analysis of QKD.

The concept of a logical and en encoding quantum layer in QKD needs further investigation but might open up new more general protocols and simplify security proofs.
Concerning our thesis, our investigations suggest developing our theoretical framework for quantum optical communication towards a coherent state transmission system.
		\addcontentsline{toc}{section}{References}
		\printbibliography[title=References]
	\end{refsection}

	\chapter{Quantum theory of light}
	\begin{refsection}
		\chapter*{Introduction}
\addcontentsline{toc}{chapter}{Introduction}

Optical communication enables humanity worldwide to share information in a split second, with companies like Huawei undergoing tremendous efforts to advance the frontiers.
In addition to incremental innovation increasing the performance and decreasing the cost of optical communication technology, we observe intensified activities towards disruptive innovations that challenge our present understanding of communication.
One such branch of activity is quantum optical communication, incorporating quantum aspects of light into classical communication and leading to novel communication technology like \gls{qkd}, which enables practical and secure key generation.
As a still young discipline, which emerged from two highly advanced fields, communication engineering and quantum physics, quantum communication lacks a unified description to which both communication engineers and quantum physicists agree.
The present thesis aims to resolve the seeming discrepancies between communication engineering and quantum physics by reviewing a practical implementation of a quantum communication system implementing a \gls{qkd} protocol.
In the process, we hope to develop a theoretical framework for quantum optical communication, incorporating quantum effects into classical communication, which has applicability beyond \gls{qkd}.

\subsection*{Problem statement}

To raise awareness of the challenges ahead, we review the best-known quantum theory of light, single-mode quantum optics, along with central ideas from classical communication and outline where these pictures conflict.

In single-mode quantum optics, we model monochromatic light with frequency $\omega_0$ as a quantum harmonic oscillator with unit mass, $m=1$, and Hamiltonian~\cite{Gerry2005,Fox2006}
\begin{equation}
	\hat{H}
	=
	\omega_0
	\hat{a}^\dagger
	\hat{a}
	,
\end{equation}
wherein $\hat{a}$ and $\hat{a}^\dagger$ are the quantum annihilation and creation operators, destroying or creating an excitation or "mode" of frequency $\omega_0$.
The electric field operator,
\begin{equation}
	\hat{E}(t,x)
	=
	\mathcal{E}_0
	\left(
		\hat{a}
		+
		\hat{a}^\dagger
	\right)
	\sin(\omega_0x)
	,
\end{equation}
wherein $\mathcal{E}_0$ has the interpretation of an electric field density, establishes the connection between the quantum harmonic oscillator and electromagnetic radiation, including light~\cite[p.~12]{Gerry2005}.
Two of the most important quantum states are the number and the coherent state,
\begin{align}
	\ket{n}
	&=
	\frac{1}{\sqrt{n!}}
	\left(\hat{a}^\dagger\right)^n
	\ket{0}
	&
	\ket{\alpha}
	&=
	\exp\left(-\frac{1}{2}\abs{\alpha}^2\right)
	\sum_{n=0}^\infty
	\frac{\alpha^n}{\sqrt{n!}}
	\ket{n}
	.
\end{align}
The number state is parametrized by a natural number $n\in\mathbb{N}_0$ counting the number excitations.
The coherent state is parametrized by a complex number $\alpha\in\mathbb{C}$ encoding amplitude and phase.
The expectation value of the electric field operator with respect to a coherent state,
\begin{equation}
	\bra{\alpha}
	\hat{E}(t)
	\ket{\alpha}
	=
	\sqrt{2}
	\abs{\alpha}
	\mathcal{E}_0
	\sin\left(\omega_0t-\theta\right)
	,
\end{equation}
equals a classical monochromatic wave with amplitude proportional to $\abs{\alpha}$ and phase $\theta$~\cite[p.~45]{Gerry2005}.

In communication engineering, light is primarily a means of transmitting signals that appear as frequency bands centered around an optical carrier frequency $\omega_c$, as illustrated in \Cref{fig:signal_spectrum}.
\begin{figure}[ht]
	\centering
	\includegraphics{figures/pgfplots/signal-spectrum}
	\caption{Receiver spectrum comprising multiple signal bands relative to a carrier frequency at $\omega=0$. At \SI{+100}{\mega\hertz}, the spectrum has a pilot tone broadened by phase noise. Centered at \SI{-25}{\mega\hertz}, the spectrum contains a first signal band with \SI{12.5}{\mega\hertz} bandwidth. Centered at \SI{-168.75}{\mega\hertz}, the spectrum contains a second signal band with \SI{12.5}{\mega\hertz} bandwidth. The remaining segments of the spectrum include mirror bands or disturbances.}\label{fig:signal_spectrum}
\end{figure}
The concept of frequency bands is extremely powerful as it allows the transmission of multiple independent signals around one carrier frequency.
More formally, we need to distinguish between base- and passband signals.
For a baseband signal $x_b(t)$, the signal power outside of the signal's bandwidth $B$ is negligible~\cite[p.~15]{Madhow2008}, i.e.,
\begin{align}
	\abs{x_b(\omega)}^2
	&\approx
	0
	&
	\abs{\omega}
	&>
	B/2
	.
\end{align}
For a passband signal $x_p(t)$, the signal power outside the signal's band around a carrier frequency $\omega_c$ is negligible~\cite[p.~16]{Madhow2008}, i.e.,
\begin{align}
	\abs{x_p(\omega)}^2
	&\approx
	0
	&
	\abs{\omega\pm\omega_c}
	&>
	B/2
	.	
\end{align}
\begin{figure}[ht]
	\centering
	\includegraphics{figures/tikz/up-conversion}
	\caption{Power spectrum illustrating up-conversion of a real-valued passband signal with bandwidth $B$ centered at $\omega_0$. Up-conversion by $\omega_c$ shifts the passband to $\omega_c+\omega_0$ and creates a mirror band at $\omega_c-\omega_0$.}\label{fig:up_conversion}
\end{figure}
A baseband signal can be up-converted to a passband signal at carrier frequency $\omega_c$ by shifting the spectrum by $\omega_c$ is known as up-conversion, see \Cref{fig:up_conversion}, and implemented by modulation.
Similar, a passband signal at carrier frequency $\omega_c$ is down-converted to a baseband signal by demodulation~\cite[p.~26]{Madhow2008}.

To sum up, single-mode quantum optics provides precise physical meaning to light, including quantum effects, although limited to monochromatic light.
On the other side, communication engineering provides a framework for efficiently constructing and transmitting signals.
For quantum optical communication, it is inevitable to welcome and incorporate both views.
For instance, people with a background in quantum optics but foreign to communication engineering often advocate the concept of "one state, one universe", where each quantum transmission is completely independent.
However, if we include practical considerations, like assuming a single transmission line, the picture of "one state, one universe" is plagued by several ambiguities.
For example, a single-mode quantum state has a single well-defined frequency $\omega_0$, which by Fourier uncertainty implies infinite temporal duration but makes information transmission absurd.
The typical counter-argument is that single-mode quantum optics implicitly assumes pulses with $\omega_0$ being the center frequency of the pulse.
While the counter-argument is technically valid, we must admit that it only raises new questions, such as bandwidth-limitations on the pulse parameters, all properly addressed in communication engineering.

The multi-mode quantum optics mentioned in popular quantum optics books~\cite{Gerry2005,Fox2006} are insufficient to represent continuous-time signals, and performing a continuum limit might not be correct if we consider the huge differences between linear algebra and functional analysis.
The advanced quantum optics literature~\cite{Vogel2006,Mandel1995} does sometimes use a continuous-mode formalism but does not explicitly investigate its properties.
We are only aware of two books~\cite{Loudon2000,Barnett2002} that explicitly present a continuous-mode theory of light but again open up new questions regarding the fundamental assumptions and justification thereof.
If we are willing to go one step deeper, we find answers in the quantum field theory literature~\cite{Peskin1995,Srednicki2007,Greiner2013,Itzykson2012}, but it is up to us to transfer these insights from particle physics to quantum optics applications.
We even have to go a bit deeper and look into mathematical quantum field theory~\cite{Streater2016,Bogoliubov1982,Bogolubov1989} to answer some questions.
Finally, we want to understand and upgrade quantum models of (electro-)optical components in the literature~\cite{Vogel2006,Leonhardt2003,Haroche2006,Mandel1995} to a mode continuum for comparison with the results from the optical communication community~\cite{Shapiro2009,Kikuchi2016}.

\subsection*{Thesis outline}

Our work is divided into four chapters.
In \Cref{ch:qkd}, we present an introduction to \gls{qkd}, emphasizing the similarities between the plethora of seemingly different protocols and attempting to argue why practical \gls{qkd} based on weak coherent states is effectively a coherent state communication system.
In the following three chapters, we construct our theoretical framework for quantum optical communication towards practical \gls{qkd}, starting from a general quantum theory of light, \Cref{ch:light}, over applying the quantum theory to describe the building blocks of coherent communication systems, \Cref{ch:components}, to an abstract description of a coherent state transmission system's signal-processing, \Cref{ch:system}.
While the thesis chapter structure supports a bottom-up approach, it is equally possible to read the thesis from the back to the front, revealing more and more details.
Likewise, it is possible to skip certain chapters and pare down to the chapter summary at the end of each chapter.
		\section{Canonical quantization of Maxwell field}

What do we do here?
- Section structure
- Expected results

Hamiltonian vs Lagrangian mechanics?
Why we start with Lagrangian mechanics?

From the Lagrangian we can infer everything else.

\subsection{Maxwell Lagrangian from first principles}

From a theoretical physics viewpoint, Maxwell theory is a relativistic vector field theory with local gauge symmetry.

As a relativistic vector field, we expect the Maxwell field $A^\mu$ to have four components, one time and three spatial components, and to transform as a Lorentz vector~\cite[p.~37]{Peskin1995}
\begin{equation}
	A^\mu(x)
	\to
	A^{\prime\mu}(x^\prime)
	=
	\Lambda^\mu_\nu
	A^\nu(\Lambda^{-1}x)
	.
\end{equation}
Contracting Lorentz vectors, or in general Lorentz tensors, to a scalar yields a Lorentz-invariant quantity.
Exclusively using Lorentz invariant quantities when constructing our theory ensures the compatibility of our theory with special relativity.

As a starting point for our theory's the Lagrangian (density), we propose
\begin{equation}
	\mathcal{L}
	=
	c_1
	\left(
		\partial_\mu
		A^\nu
	\right)
	\left(
		\partial^\mu
		A_\nu
	\right)
	+
	c_2
	\left(
		\partial_\mu
		A^\mu
	\right)
	\left(
		\partial_\nu
		A^\nu
	\right)
	+
	c_3
	\left(
		\partial_\mu
		A^\nu
	\right)
	\left(
		\partial_\nu
		A^\mu
	\right)
	+
	c_4
	m^2
	A_\mu A^\mu
	\label{eq:proposed_maxwell_lagrangian_proposal},
\end{equation}
wherein we restricted ourselves to terms which allow for dimensionless coefficients $c1,c2,c3,c4\in\mathbb{R}$, a mass term, and no (self-)interactions.\footnote{The restriction to dimensionless coefficients can be further motivated by renormalization arguments.}
For the action integral
\begin{equation}
	S
	=
	\int\dd{t}
	L
	=
	\int\dd[4]{x}
	\mathcal{L}
\end{equation}
to exist, we need $A^\mu$ to be square-integrable, i.e., vanish at the integration boundaries.
Using partial integration, we can show the redundancy of two of the first three terms in \cref{eq:proposed_maxwell_lagrangian}, see Ref.~\cite{deRham2014}.
We remove the redundancy by setting the third coefficient to zero, $c_3=0$.
Demanding invariance under local gauge transformations requires removing the mass term and determining the Lagrangian up to an overall constant.
The overall constant does not affect the dynamics, and its absolute value is equal to $1/2$ by convention.
We finally arrive at the well-known Lagrangian density of the Maxwell field,
\begin{equation}
	\mathcal{L}
	=
	-
	\frac{1}{2}
	\left(
		\partial_\mu
		A^\nu
	\right)
	\left(
		\partial^\mu
		A_\nu
	\right)
	+
	\frac{1}{2}
	\left(
		\partial_\mu
		A^\mu
	\right)
	\left(
		\partial_\nu
		A^\nu
	\right)
	.
\end{equation}
While we can experimentally verify the vector nature of the Maxwell field by polarization experiments, the requirement of the Maxwell field having local gauge symmetry appears artificial.
If we accept the Dirac Lagrangian, describing charged fermions, then Noether's theorem links local gauge symmetry to charge conservation.
However, the Dirac Lagrangian itself cannot be made gauge-invariant without coupling to a gauge-invariant vector field, which turns out to be the Maxwell field.
The complete gauge-invariant Lagrangian describing charged fermions coupled to the Maxwell field lays down the foundation of \gls{qed}.

\subsection{Lorentz force, field-strength tensor, electromagnetic components and Maxwell equations}

% How do we get from the vector field to the field-strength tensor?
We consider the action of a particle with mass $m$ and charge $q$ coupled to the vector field~\cite[p.~244]{Zee2013}
\begin{equation}
	S
	=
	-
	m\int\dd{\tau}
	\sqrt{-g_{\mu\nu}\odv{x^\mu}{\tau}\odv{x^\nu}{\tau}}
	+
	q\int\dd{\tau}
	A_\mu\left(x(\tau)\right)
	\odv{x^\mu}{\tau}
\end{equation}
for which the action principle yields the equation of motion
\begin{equation}
	m
	\odv[order={2}]{x^\mu}{\tau}
	=
	qF^\mu_\nu(x)\odv{x^\nu}{\tau}
	\label{eq:cov_lorentz_force}
\end{equation}
corresponding to a covariant formulation of Lorentz force law, wherein the field-strength tensor
\begin{equation}
	F_{\mu\nu}
	=
	\partial_\mu
	A_\nu
	-
	\partial_\nu
	A_\mu
\end{equation}
is the covariant generalization of the electromagnetic field.

Comparison of the covariant Lorentz force law, \cref{eq:cov_lorentz_force}, with the vector Lorentz force law
\begin{equation}
	m
	\odv[order={2}]{\vb{x}}{t}
	=
	q\left(
		\vb{E}
		+
		\odv{\vb{x}}{t}
		\vcross
		\vb{B}
	\right)
\end{equation}
we can identify the components of the field-strength tensor with the electromagnetic field~\cite[p.~245]{Zee2013}
\begin{align}
	E^1
	&=
	F^{01}
	=
	-
	F_{01}
	&
	E^2
	&=
	F^{02}
	=
	-
	F_{02}
	&
	E^3
	&=
	F^{03}
	=
	-
	F_{03}
	\\
	B^1
	&=
	F^{23}
	=
	-
	F_{23}
	&
	B^2
	&=
	F^{31}
	=
	-
	F_{31}
	&
	B^3
	&=
	F^{12}
	=
	-
	F_{12}
\end{align}
or in index notation~\cite[p.~336]{Srednicki2007}
\begin{align}
	F^{0i}
	&=
	E^i
	&
	F^{ij}
	&=
	\varepsilon^{ijk}
	B_k
	.
\end{align}
\textcolor{red}{identify charged moving particle with current}

\begin{equation}
	\mathcal{L}
	=
	-
	\frac{1}{4}
	F_{\mu\nu}
	F^{\mu\nu}
	-
	j_\mu A^\mu
\end{equation}

\begin{align}
	\div\vb{E}
	&=
	j_0
	&
	\div\vb{B}
	&=
	0
	\\
	\curl\vb{E}
	&=
	-\partial_t\vb{B}
	&
	\curl\vb{B}
	&=
	\vb{j}
	+
	\partial_t\vb{E}
\end{align}

\subsection{Coulomb gauge and plane-wave expansion}

The physical observable of the Maxwell field is the field-strength tensor $F_{\mu\nu}$.
The field-strength tensor $F_{\mu\nu}$ is invariant under local gauge transformations.
We perform a local gauge transformation 
We choose a gauge field $\chi$ satisfying
\begin{align}
	\laplace\chi
	&=
	\partial_i A^i
	=
	-
	\div\vb{A}
	&
	\partial_t
	\chi
	&=
	-
	A_0
\end{align}
and perform a gauge transformation.
After the gauge transformation, the Maxwell field satisfies
\begin{align}
	\div\vb{A}
	&=
	0
	&
	A_0
	&=0
\end{align}
corresponding to the Coulomb and temporal gauge condition~\cite[p.~144]{Greiner2013},
The temporal gauge condition $A_0=0$ implicitly assumes that there are no free charges.

% Relation between Coulomb and temporal gauge and transverse and longitudinal components
% Coulomb interaction energy and A_0 component

% Lorentz-invariant measure

% plane-wave expansion
% energy-momentum relation

\subsection{Canonical quantization and quantum operators}

The canonical energy-momentum tensor~\cite[p.~174]{Greiner2013}
\begin{equation}
	\begin{split}
		T^{\mu\nu}
		&=
		\pdv{\mathcal{L}}{(\partial_\mu A_\sigma)}
		\partial^\nu A_\sigma
		-
		g^{\mu\nu}\mathcal{L}
		\\
		&=
		-
		\left(
			\partial^\mu
			A^\sigma
		\right)
		\left(
			\partial^\nu
			A_\sigma
		\right)
		+
		\frac{1}{2}
		g^{\mu\nu}
		\left(
			\partial^\rho
			A^\sigma
		\right)
		\left(
			\partial_\rho
			A_\sigma
		\right)
	\end{split}
\end{equation}

The energy density is the time-time component of the energy-momentum tensor
\begin{equation}
	\begin{split}
		T^{00}
		&=
		-
		\left(
			\partial^0
			A^\sigma
		\right)
		\left(
			\partial^0
			A_\sigma
		\right)
		+
		\frac{1}{2}
		g^{00}
		\left(
			\partial^\rho
			A^\sigma
		\right)
		\left(
			\partial_\rho
			A_\sigma
		\right)
		\\
		&=
		-
		\left(
			\partial^0
			A^j
		\right)
		\left(
			\partial^0
			A_j
		\right)
		+
		\frac{1}{2}
		\left(
			\partial^0
			A^j
		\right)
		\left(
			\partial_0
			A_j
		\right)
		+
		\frac{1}{2}
		\left(
			\partial^i
			A^j
		\right)
		\left(
			\partial_i
			A_j
		\right)
		\\
		&=
		-
		\frac{1}{2}
		\left(
			\partial_t
			A^j
		\right)
		\left(
			\partial_t
			A_j
		\right)
		+
	\end{split}
\end{equation}

% normal order to remove (infinite) vacuum energy
\begin{equation}
	H
	=
\end{equation}

\begin{equation}
	P_j
	=
\end{equation}

We define the transverse delta distribution as the transverse projection of the three-dimensional delta distribution
\begin{equation}
	\delta_{\perp,ij}^{(3)}
	\left(\vb{x}\right)
	=
	P_{\perp,ij}
	\delta^{(3)}
	\left(\vb{x}\right)
	=
	\int\frac{\dd[3]{p}}{(2\pi)^3}
	\left(
		\delta_{ij}
		-
		\frac{p_ip_j}{\vb{p}^2}
	\right)
	e^{i\vb{p}\vdot\vb{x}}
\end{equation}

\begin{align}
	\comm{\hat{A}_i(t,\vb{x})}{\hat{E}_j(t,\vb{y})}
	&=
	-i
	\delta_{\perp,ij}^{(3)}
	\left(\vb{x}-\vb{y}\right)
	\\
	\comm{\hat{A}_i(t,\vb{x})}{\hat{A}_j(t,\vb{y})}
	&=
	0
	=
	\comm{\hat{E}_i(t,\vb{x})}{\hat{E}_j(t,\vb{y})}
\end{align}
where the negative sign for the non-vanishing commutator arises from the relation between the canonical momentum density and the electric field, $\hat\pi_i=-\hat{E}_i$ (?).

In the Coulomb gauge, the Maxwell field operator is
\begin{equation}
	\vu{A}(t,\vb{x})
	=
	\vu{A}^{(+)}(t,\vb{x})
	+
	\vu{A}^{(-)}(t,\vb{x})
\end{equation}
\begin{align}
	\vu{A}^{(-)}(t,\vb{x})
	&=
	\sum_{\lambda=1,2}
	\int\dd{\mu(\vb{p})}
	\hat{a}_\lambda(\vb{p})
	\vu{e}_\lambda(\vb{p})
	e^{-i\omega(\vb{p})t+i\vb{p}\vdot\vb{x}}
	\\
	\vu{A}^{(+)}(t,\vb{x})
	&=
	\sum_{\lambda=1,2}
	\int\dd{\mu(\vb{p})}
	\hat{a}_\lambda^\dagger(\vb{p})
	\vu{e}_\lambda(\vb{p})^*
	e^{+i\omega(\vb{p})t-i\vb{p}\vdot\vb{x}}
\end{align}
Disagreement between ~\cite[p.~341]{Srednicki2007}, ~\cite[p.~198]{Greiner2013}, and \cite[p.~123]{Peskin1995}

\subsection{Quantum observable operators}

\begin{equation}
	\hat{H}
	=
\end{equation}
\begin{equation}
	\hat{P}
	=
\end{equation}

The electric field operator is given by
\begin{equation}
	\vu{E}(t,\vb{x})
	=
	\vu{E}^{(+)}(t,\vb{x})
	+
	\vu{E}^{(-)}(t,\vb{x})
	.
\end{equation}
\begin{align}
	\vu{E}^{(-)}(t,\vb{x})
	&=
	-i
	\sum_{\lambda=1,2}
	\int\dd{\mu(\vb{p})}
	\omega(\vb{p})
	\hat{a}_\lambda(\vb{p})
	\vu{e}_\lambda(\vb{p})
	e^{-i\omega(\vb{p})t+i\vb{p}\vdot\vb{x}}
	\\
	\vu{E}^{(+)}(t,\vb{x})
	&=
	+i
	\sum_{\lambda=1,2}
	\int\dd{\mu(\vb{p})}
	\omega(\vb{p})
	\hat{a}_\lambda^\dagger(\vb{p})
	\vu{e}_\lambda(\vb{p})^*
	e^{+i\omega(\vb{p})t-i\vb{p}\vdot\vb{x}}
\end{align}
		\section{Axiomatic Fock space}

\subsection{Vacuum space}

\subsection{Annihilation and creation operators}

\subsection{Number states}
		\section{Coherent states}

\begin{definition}[Displacement operator]
	The displacement operator with spectrum $\alpha(\vb{p})$ is
	\begin{equation}
		\hat{D}[\alpha]
		=
		\exp\left\{
			\int\frac{\dd[3]{p}}{(2\pi)^3\sqrt{2\omega(\vb{p})}}
			\left\{
				\alpha(\vb{p})
				\hat{a}^\dagger(\vb{p})
				-
				\alpha(\vb{p})^*
				\hat{a}(\vb{p})
			\right\}
		\right\}
		\label{eq:qkg_displacement_operator}
		.
	\end{equation}
\end{definition}
\begin{corollary}\label{thm:qkg_displacement_smeared}
	The displacement operators can be expressed in terms of the smeared positive and negative frequency Klein-Gordon operators
	\begin{equation}
		\hat{D}[\alpha]
		=
		\exp\left\{
			\hat\phi^-[\alpha]
			-
			\hat\phi^+[\alpha]
		\right\}
		.
	\end{equation}
\end{corollary}
\begin{restatable}{lemma}{qkgdisplacementordered}\label{thm:qkg_displacement_ordered}
	The displacement operator can be put in normal-
	\begin{equation}\label{eq:qkg_displacement_normal}
		\hat{D}[\alpha]
		=
		\exp\left\{
			+\hat\phi^-[\alpha]
		\right\}
		\exp\left\{
			-\hat\phi^+[\alpha]
		\right\}
		\exp\left\{
			-
			\frac{1}{2}
			\comm{\hat\phi^+[\alpha]}{\hat\phi^-[\alpha]}
		\right\}
	\end{equation}
	and antinormal-order
	\begin{equation}\label{eq:qkg_displacement_antinormal}
		\hat{D}[\alpha]
		=
		\exp\left\{
			-\hat\phi^+[\alpha]
		\right\}
		\exp\left\{
			+\hat\phi^-[\alpha]
		\right\}
		\exp\left\{
			+
			\frac{1}{2}
			\comm{\hat\phi^+[\alpha]}{\hat\phi^-[\alpha]}
		\right\}
	\end{equation}
	wherein the commutator evaluates to
	\begin{equation}
		\comm{\hat\phi^+[\alpha]}{\hat\phi^-[\alpha]}
		=
		\int\frac{\dd[3]{p}}{(2\pi)^32\omega(\vb{p})}
		\abs{\alpha(\vb{p})}^2
	\end{equation}
	in momentum space.
\end{restatable}
\begin{restatable}{lemma}{qkgdisplacementproduct}\label{thm:qkg_displacement_product}
	Let $\hat{D}[\alpha],\hat{D}[\beta]$ be two displacement operators with spectrum $\alpha(\vb{p}),\beta(\vb{p})$, then their product equals
	\begin{equation}\label{eq:qkg_displacement_product}
		\begin{split}
			\hat{D}[\alpha]
			\hat{D}[\beta]
			&=
			\hat{D}[\alpha+\beta]
			\exp\left\{
				-
				\frac{1}{2}
				\comm{\hat\phi^+[\alpha]}{\hat\phi^-[\beta]}
				+
				\frac{1}{2}
				\comm{\hat\phi^+[\beta]}{\hat\phi^-[\alpha]}
			\right\}
			\\
			&=
			\hat{D}[\alpha+\beta]
			\exp\left\{
				-
				\frac{1}{2}
				\iint\frac{\dd[3]{p}}{(2\pi)^32\omega(\vb{p})}
				\left\{
					\alpha(\vb{p})^*
					\beta(\vb{p})
					-
					\alpha(\vb{p})
					\beta(\vb{p})^*
				\right\}
			\right\}
			.
		\end{split}
	\end{equation}
\end{restatable}
\begin{restatable}{lemma}{qkgdisplacementunitary}\label{thm:qkg_displacement_unitary}
	The displacement operator is unitary
	\begin{equation}
		\hat{D}[\alpha]^{-1}
		=
		\hat{D}[\alpha]^\dagger
		=
		\hat{D}[-\alpha]
		.
	\end{equation}
\end{restatable}
\begin{definition}[Coherent state]
	A coherent state $\ket{\alpha}$ with spectrum $\alpha(\vb{p})$
	\begin{equation}
		\begin{split}
			\ket{\alpha}
			&=
			\exp\left\{
				-
				\frac{1}{2}
				\comm{\hat\phi^+[\alpha]}{\hat\phi^-[\alpha]}
			\right\}
			\exp\left\{
				+\hat\phi^-[\alpha]
			\right\}
			\ket{0}
			\\
			&=
			\exp\left\{
				-
				\frac{1}{2}
				\int\frac{\dd[3]{p}}{(2\pi)^32\omega(\vb{p})}
				\abs{\alpha(\vb{p})}^2
			\right\}
			\exp\left\{
				\int\frac{\dd[3]{p}}{(2\pi)^3\sqrt{2\omega(\vb{p})}}
				\alpha(\vb{p})
				\hat{a}^\dagger(\vb{p})
			\right\}
			\ket{0}
		\end{split}
	\end{equation}
	is a coherent superposition of number states without constrained spectrum.
\end{definition}
\begin{restatable}{lemma}{qkgdisplacementvacuum}\label{thm:qkg_displacement_vacuum}
	The displacement operator creates a coherent state from the vacuum
	\begin{equation}\label{eq:qkg_displacement_vacuum}
		\hat{D}[\alpha]
		\ket{0}
		=
		\ket{\alpha}
		.
	\end{equation}
\end{restatable}
\begin{corollary}
	The coherent state is normalized\footnote{In contrast to the number state, there is no constraint on the spectrum of the coherent state $\alpha(\vb{p})$.}
	\begin{equation}
		\braket{\alpha}
		=
		1
		.
	\end{equation}
\end{corollary}
\begin{restatable}{theorem}{qkgcoherenteigenstate}\label{thm:qkg_coherent_state_eigenstate}
	The coherent state is an eigenstate of the annihilation operator to eigenvalue $\alpha(\vb{p})/\sqrt{2\omega(\vb{p})}$, i.e.,
	\begin{equation}\label{eq:qkg_coherent_state_eigenstate}
		\hat{a}(\vb{p})
		\ket{\alpha}
		=
		\frac{\alpha(\vb{p})}{\sqrt{2\omega(\vb{p})}}
		\ket{\alpha}
		.
	\end{equation}
\end{restatable}
\begin{restatable}{lemma}{qkgcoherentenergy}\label{thm:qkg_coherent_state_energy}
	The mean energy of the coherent state is
	\begin{equation}
		\expval{\hat{H}}{\alpha}
		=
		\int\frac{\dd[3]{p}}{(2\pi)^32\omega(\vb{p})}
		\omega(\vb{p})
		\abs{\alpha(\vb{p})}^2
	\end{equation}
	and the variance is
	\begin{equation}
		\expval{\left(\Delta\hat{H}\right)^2}{\alpha}
		=
		\int\frac{\dd[3]{p}}{(2\pi)^32\omega(\vb{p})}
		\omega(\vb{p})^2
		\abs{\alpha(\vb{p})}^2
		.
	\end{equation}
\end{restatable}
Although, the mean energy of the coherent state is the same as the mean energy of the single-particle number state, the spectrum $\alpha(\vb{p})$ of the coherent state is not bound.
\begin{restatable}{lemma}{qkgcoherentnumber}\label{thm:qkg_coherent_state_number}
	The mean number of particles is
	\begin{equation}
		\overline{n}
		=
		\expval{\hat{N}}{\alpha}
		=
		\int\frac{\dd[3]{p}}{(2\pi)^32\omega(\vb{p})}
		\abs{\alpha(\vb{p})}^2
	\end{equation}
	and the variance is
	\begin{equation}
		\expval{\left(\Delta\hat{N}\right)^2}{\alpha}
		=
		\int\frac{\dd[3]{p}}{(2\pi)^32\omega(\vb{p})}
		\abs{\alpha(\vb{p})}^2
		=
		\overline{n}
		,
	\end{equation}
	i.e., the particle number is Poisson distributed.
\end{restatable}
\begin{restatable}{lemma}{qkgcoherentnumberinnerproduct}\label{thm:qkg_coherent_state_number_state_inner_product}
	The inner product between an $n$-particle number state with spectrum $f(\vb{p})$ and a coherent state with spectrum $\alpha(\vb{p})$ is
	\begin{equation}
		\braket{n_f}{\alpha}
		=
		\frac{1}{\sqrt{n!}}
		\left(
			\int\frac{\dd[3]{p}}{(2\pi)^32\omega(\vb{p})}
			f(\vb{p})^*
			\alpha(\vb{p})
		\right)^n
		e^{-\overline{n}/2}
	\end{equation}
	where $\overline{n}$ is the mean particle number of the coherent state.
\end{restatable}
\begin{corollary}
	If $\alpha(\vb{p})=f(\vb{p})$ the former inner product reduces to
	\begin{equation}
		\braket{n_f}{\alpha}
		=
		\frac{1}{\sqrt{n!}}
		e^{-\overline{n}/2}
	\end{equation}
	which absolute squared is a Poisson distribution with unit variance.
\end{corollary}
\begin{corollary}
	If the product of the spectrums is equal to
	\begin{equation}
		f(\vb{p})
		\alpha(\vb{p})
		=
		2\omega(\vb{p})
		(2\pi)^3
		\delta(\vb{p}-\vb{k})
	\end{equation}
	the inner product reduces to
	\begin{equation}
		\braket{n_f}{\alpha}
		=
		\frac{\alpha^n}{\sqrt{n!}}
		e^{-\overline{n}/2}
	\end{equation}
	and we recover the Poisson distribution known from single-mode quantum optics
	\begin{equation}
		p_n
		=
		\abs{\braket{n_f}{\alpha}}^2
		=
		\frac{\overline{n}^n}{\sqrt{n!}}
		e^{-\overline{n}}
	\end{equation}
	where $\overline{n}=\abs{\alpha}^2$.
\end{corollary}
\begin{restatable}{lemma}{qkgcoherentinnerproduct}\label{thm:coherent_state_inner_product}
	Let $\ket{\alpha},\ket{\beta}$ be two coherent states, then their inner product is
	\begin{equation}
		\begin{split}
			\braket{\alpha}{\beta}
			&=
			\exp\left\{
				-
				\frac{1}{2}
				\comm{\hat\phi^+[\alpha]}{\hat\phi^-[\alpha]}
				+
				\comm{\hat\phi^+[\alpha]}{\hat\phi^-[\beta]}
				-
				\frac{1}{2}
				\comm{\hat\phi^+[\beta]}{\hat\phi^-[\beta]}
			\right\}
			\\
			&=
			\exp\left\{
				-
				\frac{1}{2}
				\int\frac{\dd[3]{p}}{(2\pi)^32\omega(\vb{p})}
				\left\{
					\abs{\alpha(\vb{p})}^2
					+
					\abs{\beta(\vb{p})}^2
					-
					2\alpha(\vb{p})\beta(\vb{p})^*
				\right\}
			\right\}
		\end{split}
	\end{equation}
\end{restatable}

		\addcontentsline{toc}{section}{References}
		\printbibliography[title=References]
	\end{refsection}
	
	\chapter{Classical transmission system}
	\begin{refsection}
		\chapter*{Introduction}
\addcontentsline{toc}{chapter}{Introduction}

Optical communication enables humanity worldwide to share information in a split second, with companies like Huawei undergoing tremendous efforts to advance the frontiers.
In addition to incremental innovation increasing the performance and decreasing the cost of optical communication technology, we observe intensified activities towards disruptive innovations that challenge our present understanding of communication.
One such branch of activity is quantum optical communication, incorporating quantum aspects of light into classical communication and leading to novel communication technology like \gls{qkd}, which enables practical and secure key generation.
As a still young discipline, which emerged from two highly advanced fields, communication engineering and quantum physics, quantum communication lacks a unified description to which both communication engineers and quantum physicists agree.
The present thesis aims to resolve the seeming discrepancies between communication engineering and quantum physics by reviewing a practical implementation of a quantum communication system implementing a \gls{qkd} protocol.
In the process, we hope to develop a theoretical framework for quantum optical communication, incorporating quantum effects into classical communication, which has applicability beyond \gls{qkd}.

\subsection*{Problem statement}

To raise awareness of the challenges ahead, we review the best-known quantum theory of light, single-mode quantum optics, along with central ideas from classical communication and outline where these pictures conflict.

In single-mode quantum optics, we model monochromatic light with frequency $\omega_0$ as a quantum harmonic oscillator with unit mass, $m=1$, and Hamiltonian~\cite{Gerry2005,Fox2006}
\begin{equation}
	\hat{H}
	=
	\omega_0
	\hat{a}^\dagger
	\hat{a}
	,
\end{equation}
wherein $\hat{a}$ and $\hat{a}^\dagger$ are the quantum annihilation and creation operators, destroying or creating an excitation or "mode" of frequency $\omega_0$.
The electric field operator,
\begin{equation}
	\hat{E}(t,x)
	=
	\mathcal{E}_0
	\left(
		\hat{a}
		+
		\hat{a}^\dagger
	\right)
	\sin(\omega_0x)
	,
\end{equation}
wherein $\mathcal{E}_0$ has the interpretation of an electric field density, establishes the connection between the quantum harmonic oscillator and electromagnetic radiation, including light~\cite[p.~12]{Gerry2005}.
Two of the most important quantum states are the number and the coherent state,
\begin{align}
	\ket{n}
	&=
	\frac{1}{\sqrt{n!}}
	\left(\hat{a}^\dagger\right)^n
	\ket{0}
	&
	\ket{\alpha}
	&=
	\exp\left(-\frac{1}{2}\abs{\alpha}^2\right)
	\sum_{n=0}^\infty
	\frac{\alpha^n}{\sqrt{n!}}
	\ket{n}
	.
\end{align}
The number state is parametrized by a natural number $n\in\mathbb{N}_0$ counting the number excitations.
The coherent state is parametrized by a complex number $\alpha\in\mathbb{C}$ encoding amplitude and phase.
The expectation value of the electric field operator with respect to a coherent state,
\begin{equation}
	\bra{\alpha}
	\hat{E}(t)
	\ket{\alpha}
	=
	\sqrt{2}
	\abs{\alpha}
	\mathcal{E}_0
	\sin\left(\omega_0t-\theta\right)
	,
\end{equation}
equals a classical monochromatic wave with amplitude proportional to $\abs{\alpha}$ and phase $\theta$~\cite[p.~45]{Gerry2005}.

In communication engineering, light is primarily a means of transmitting signals that appear as frequency bands centered around an optical carrier frequency $\omega_c$, as illustrated in \Cref{fig:signal_spectrum}.
\begin{figure}[ht]
	\centering
	\includegraphics{figures/pgfplots/signal-spectrum}
	\caption{Receiver spectrum comprising multiple signal bands relative to a carrier frequency at $\omega=0$. At \SI{+100}{\mega\hertz}, the spectrum has a pilot tone broadened by phase noise. Centered at \SI{-25}{\mega\hertz}, the spectrum contains a first signal band with \SI{12.5}{\mega\hertz} bandwidth. Centered at \SI{-168.75}{\mega\hertz}, the spectrum contains a second signal band with \SI{12.5}{\mega\hertz} bandwidth. The remaining segments of the spectrum include mirror bands or disturbances.}\label{fig:signal_spectrum}
\end{figure}
The concept of frequency bands is extremely powerful as it allows the transmission of multiple independent signals around one carrier frequency.
More formally, we need to distinguish between base- and passband signals.
For a baseband signal $x_b(t)$, the signal power outside of the signal's bandwidth $B$ is negligible~\cite[p.~15]{Madhow2008}, i.e.,
\begin{align}
	\abs{x_b(\omega)}^2
	&\approx
	0
	&
	\abs{\omega}
	&>
	B/2
	.
\end{align}
For a passband signal $x_p(t)$, the signal power outside the signal's band around a carrier frequency $\omega_c$ is negligible~\cite[p.~16]{Madhow2008}, i.e.,
\begin{align}
	\abs{x_p(\omega)}^2
	&\approx
	0
	&
	\abs{\omega\pm\omega_c}
	&>
	B/2
	.	
\end{align}
\begin{figure}[ht]
	\centering
	\includegraphics{figures/tikz/up-conversion}
	\caption{Power spectrum illustrating up-conversion of a real-valued passband signal with bandwidth $B$ centered at $\omega_0$. Up-conversion by $\omega_c$ shifts the passband to $\omega_c+\omega_0$ and creates a mirror band at $\omega_c-\omega_0$.}\label{fig:up_conversion}
\end{figure}
A baseband signal can be up-converted to a passband signal at carrier frequency $\omega_c$ by shifting the spectrum by $\omega_c$ is known as up-conversion, see \Cref{fig:up_conversion}, and implemented by modulation.
Similar, a passband signal at carrier frequency $\omega_c$ is down-converted to a baseband signal by demodulation~\cite[p.~26]{Madhow2008}.

To sum up, single-mode quantum optics provides precise physical meaning to light, including quantum effects, although limited to monochromatic light.
On the other side, communication engineering provides a framework for efficiently constructing and transmitting signals.
For quantum optical communication, it is inevitable to welcome and incorporate both views.
For instance, people with a background in quantum optics but foreign to communication engineering often advocate the concept of "one state, one universe", where each quantum transmission is completely independent.
However, if we include practical considerations, like assuming a single transmission line, the picture of "one state, one universe" is plagued by several ambiguities.
For example, a single-mode quantum state has a single well-defined frequency $\omega_0$, which by Fourier uncertainty implies infinite temporal duration but makes information transmission absurd.
The typical counter-argument is that single-mode quantum optics implicitly assumes pulses with $\omega_0$ being the center frequency of the pulse.
While the counter-argument is technically valid, we must admit that it only raises new questions, such as bandwidth-limitations on the pulse parameters, all properly addressed in communication engineering.

The multi-mode quantum optics mentioned in popular quantum optics books~\cite{Gerry2005,Fox2006} are insufficient to represent continuous-time signals, and performing a continuum limit might not be correct if we consider the huge differences between linear algebra and functional analysis.
The advanced quantum optics literature~\cite{Vogel2006,Mandel1995} does sometimes use a continuous-mode formalism but does not explicitly investigate its properties.
We are only aware of two books~\cite{Loudon2000,Barnett2002} that explicitly present a continuous-mode theory of light but again open up new questions regarding the fundamental assumptions and justification thereof.
If we are willing to go one step deeper, we find answers in the quantum field theory literature~\cite{Peskin1995,Srednicki2007,Greiner2013,Itzykson2012}, but it is up to us to transfer these insights from particle physics to quantum optics applications.
We even have to go a bit deeper and look into mathematical quantum field theory~\cite{Streater2016,Bogoliubov1982,Bogolubov1989} to answer some questions.
Finally, we want to understand and upgrade quantum models of (electro-)optical components in the literature~\cite{Vogel2006,Leonhardt2003,Haroche2006,Mandel1995} to a mode continuum for comparison with the results from the optical communication community~\cite{Shapiro2009,Kikuchi2016}.

\subsection*{Thesis outline}

Our work is divided into four chapters.
In \Cref{ch:qkd}, we present an introduction to \gls{qkd}, emphasizing the similarities between the plethora of seemingly different protocols and attempting to argue why practical \gls{qkd} based on weak coherent states is effectively a coherent state communication system.
In the following three chapters, we construct our theoretical framework for quantum optical communication towards practical \gls{qkd}, starting from a general quantum theory of light, \Cref{ch:light}, over applying the quantum theory to describe the building blocks of coherent communication systems, \Cref{ch:components}, to an abstract description of a coherent state transmission system's signal-processing, \Cref{ch:system}.
While the thesis chapter structure supports a bottom-up approach, it is equally possible to read the thesis from the back to the front, revealing more and more details.
Likewise, it is possible to skip certain chapters and pare down to the chapter summary at the end of each chapter.
		\section{Transmitter}

\subsection{Mach-Zehnder modulator}

\begin{figure}[htb]
	\centering
	\includestandalone{figures/pstricks/mzi-symmetric}
	\caption{Free-space setup of a symmetric \gls{mzi}: The input light field enters a first beam splitter BS1 from the left. The light field exits BS1 to the right and the bottom. Right of BS1, a first phase shifter adds a relative phase of $\varphi_1$. Right of the first phase shifter, a first mirror M1 reflects the light to the bottom, hitting a second beam splitter BS2 from the top. Below BS1, a second mirror M2 directs the light to the right, where a second phase shifter adds a relative phase of $\varphi_2$, and the light hits BS2 from the left.}
\end{figure}
\begin{equation}
	\begin{split}
		\begin{pmatrix}
			\hat{a}_1^\prime \\
			\hat{a}_2^\prime
		\end{pmatrix}
		&=
		\frac{1}{\sqrt{2}}
		\begin{pmatrix}
			i & 1 \\
			1 & i
		\end{pmatrix}
		\begin{pmatrix}
			ie^{i\varphi_1} & 0 \\
			0 & ie^{i\varphi_2}
		\end{pmatrix}
		\frac{1}{\sqrt{2}}
		\begin{pmatrix}
			1 & i \\
			i & 1
		\end{pmatrix}
		\begin{pmatrix}
			\hat{a}_1 \\
			\hat{a}_2
		\end{pmatrix}
		\\
		&=
		e^{i\frac{\varphi_1+\varphi_2}{2}+i\pi}
		\begin{pmatrix}
			\cos(\frac{\varphi_2-\varphi_1}{2}) & \sin(\frac{\varphi_2-\varphi_1}{2}) \\
			-\sin(\frac{\varphi_2-\varphi_1}{2}) & \cos(\frac{\varphi_2-\varphi_1}{2})
		\end{pmatrix}
		\begin{pmatrix}
			\hat{a}_1 \\
			\hat{a}_2
		\end{pmatrix}
	\end{split}
\end{equation}

\begin{table}[htb]
	\centering
	\begin{tabular}{lcc}
		\toprule
		Configuration & Phase relation & Modulation \\
		\midrule
		Push-pull & $\varphi_1=-\varphi_2$ & Amplitude \\
		Push-push & $\varphi_1=+\varphi_2$ & Phase \\
		\bottomrule
	\end{tabular}
	\caption{Configurations of a symmetric \gls{mzm}.}
\end{table}

\begin{equation}
	\begin{pmatrix}
		\hat{a}_1^\prime \\
		\hat{a}_2^\prime
	\end{pmatrix}
		\begin{pmatrix}
			\cos(\theta) & \sin(\theta) \\
			-\sin(\theta) & \cos(\theta)
		\end{pmatrix}
		\begin{pmatrix}
			\hat{a}_1 \\
			\hat{a}_2
		\end{pmatrix}
\end{equation}

\begin{equation}
	\hat{U}(\theta)
	=
	e^{i\theta\left(
		\hat{a}_1
		\hat{a}_2^\dagger
		-
		\hat{a}_1^\dagger
		\hat{a}_2
	\right)}
\end{equation}

\begin{align}
	\hat{a}_1^\prime
	&=
	\hat{U}^\dagger(\theta)
	\hat{a}_1
	\hat{U}(\theta)
	=
	\cos(\theta)
	\hat{a}_1
	+
	\sin(\theta)
	\hat{a}_2
	\\
	\hat{a}_2^\prime
	&=
	\hat{U}^\dagger(\theta)
	\hat{a}_2
	\hat{U}(\theta)
	=
	\cos(\theta)
	\hat{a}_2
	-
	\sin(\theta)
	\hat{a}_1
\end{align}

\subsection{I/Q modulator}

\begin{figure}[htb]
	\centering
	\includestandalone{figures/pstricks/iqm}
	\caption{\gls{iq}-modulator comprising three \gls{mzm} and connected by an input and output fiber. MZM 1 and MZM 2, are used in push-pull configuration for \gls{am}. MZM 3 is used in push-push configuration for setting the relative phase between the upper and lower branch. Vacuum in- and outputs are indicated by the red dashed fiber.}
\end{figure}

\begin{equation}
	\begin{pmatrix}
		\hat{a}_1^\prime \\
		\hat{a}_2^\prime
	\end{pmatrix}
	=
	\frac{1}{\sqrt{2}}
	\begin{pmatrix}
		 1 & ie^{+i\varphi_1} \\
		 ie^{-i\varphi_1} & 1
	\end{pmatrix}
	\begin{pmatrix}
		\hat{a}_1 \\
		\hat{a}_2
	\end{pmatrix}
\end{equation}

\begin{equation}
	\begin{pmatrix}
		\hat{a}_3^{\prime\prime} \\
		\hat{a}_1^{\prime\prime}
	\end{pmatrix}
	=
	\begin{pmatrix}
		 \cos\theta_1 & \sin\theta_1 \\
		 -\sin\theta_1 & \cos\theta_1
	\end{pmatrix}
	\begin{pmatrix}
		\hat{a}_3^\prime \\
		\hat{a}_1^\prime
	\end{pmatrix}
\end{equation}

\begin{equation}
	\begin{pmatrix}
		\hat{a}_1^{\prime\prime} \\
		\hat{a}_2^{\prime\prime} \\
		\hat{a}_3^{\prime\prime} \\
		\hat{a}_4^{\prime\prime}
	\end{pmatrix}
	=
	\begin{pmatrix}
		 \cos\theta_1 & 0 & -\sin\theta_1 & 0 \\
		 0 & \cos\theta_2 & 0 & \sin\theta_2 \\
		 \sin\theta_1 & 0 & \cos\theta_1 & 0 \\
		 0 & -\sin\theta_2 & 0 & \cos\theta_2 \\
	\end{pmatrix}
	\begin{pmatrix}
		\hat{a}_1^{\prime} \\
		\hat{a}_2^{\prime} \\
		\hat{a}_3^{\prime} \\
		\hat{a}_4^{\prime}
	\end{pmatrix}
\end{equation}

\begin{equation}
	\begin{pmatrix}
		\hat{a}_2^{\prime\prime} \\
		\hat{a}_4^{\prime\prime}
	\end{pmatrix}
	=
	\begin{pmatrix}
		 \cos\theta_2 & \sin\theta_2 \\
		 -\sin\theta_2 & \cos\theta_2
	\end{pmatrix}
	\begin{pmatrix}
		\hat{a}_2^\prime \\
		\hat{a}_4^\prime
	\end{pmatrix}
\end{equation}


\begin{equation}
	\begin{split}
		\begin{pmatrix}
			\hat{a}_1^{\prime\prime\prime\prime} \\
			\hat{a}_2^{\prime\prime\prime\prime}
		\end{pmatrix}
		&=
		\frac{1}{\sqrt{2}}
		\begin{pmatrix}
			1 & ie^{+i\varphi_2} \\
			ie^{-i\varphi_2} & 1
		\end{pmatrix}
		\begin{pmatrix}
			\hat{a}_1^{\prime\prime\prime} \\
			\hat{a}_2^{\prime\prime\prime}
		\end{pmatrix}
		\\
		&=
		\frac{1}{\sqrt{2}}
		\begin{pmatrix}
			1 & ie^{+i\varphi_2} \\
			ie^{-i\varphi_2} & 1
		\end{pmatrix}
		\begin{pmatrix}
			e^{+i\phi} & 0 \\
			0 & e^{-i\phi}
		\end{pmatrix}
		\begin{pmatrix}
			\hat{a}_1^{\prime\prime} \\
			\hat{a}_2^{\prime\prime}
		\end{pmatrix}
		\\
		&=
		\frac{1}{\sqrt{2}}
		\begin{pmatrix}
			e^{+i\phi} & ie^{+i(\varphi_2-\phi)} \\
			ie^{-i(\varphi_2-\phi)} & e^{-i\phi}
		\end{pmatrix}
		\begin{pmatrix}
			\hat{a}_1^{\prime\prime} \\
			\hat{a}_2^{\prime\prime}
		\end{pmatrix}
	\end{split}
\end{equation}
		\section{Receiver}
\FloatBarrier

We introduced the receiver as a component decoding a sequence of complex symbols, $\left\{\beta_n\in\mathbb{C}\colon n\in I\right\}$, from a coherent state $\ket{\beta(t)}$.
As with the transmitter, we want our receiver to be software-defined, meaning that the predominant signal processing is in the digital domain and we need to down-convert the optical signal.
\begin{figure}[htb]
	\centering
	\includegraphics{figures/tikz/software-defined-receiver}
	\caption{Block diagram of the receiver's signal processing domains. The analog electrical signals $v(t)$ and $w(t)$ are demodulated from the quadratures of the optical coherent state $\ket{\beta(t)}$, and then converted to the digital signals $v[m]$ and $w[m]$ from which the \gls{dsp} decodes the symbol sequence $\{\beta_n\in\mathbb{C}\colon n\in I\}$.}\label{fig:software_defined_receiver}
\end{figure}
\Cref{fig:software_defined_receiver} illustrates the signal flow across the different domains inside the receiver.
Unlike the transmitter up-conversion, the receiver's down-conversion leaves us some design freedom.
In particular, it is possible to reduce the hardware complexity significantly by introducing an intermediate frequency, which carries over to the digital domain.

\FloatBarrier
\subsection{Down-conversion}

% outlook down-conversion freedoms

\begin{figure}[htb]
	\centering
	\includegraphics{figures/circuitikz/down-conversion-single}
	\caption{Block diagram of single quadrature down-conversion. The signal $z(t)$ is mixed with the \gls{lo} $\sin(\omega_lt+\vartheta)$. The output of the mixing is filtered by a \gls{lp} to produce the down-converted signal $u(t)$.}\label{fig:down_conversion_single}
\end{figure}
The mixing of the real-valued passband signal,
\begin{equation}
	\begin{split}
		z(t)
		=
		\int_{-\infty}^{+\infty}\frac{\dd{\omega}}{2\pi}
		z(\omega)
		e^{+i\omega t}
		&=
		\int_0^{+\infty}\frac{\dd{\omega}}{2\pi}
		z(\omega)
		e^{+i\omega t}
		+
		\int_{-\infty}^0\frac{\dd{\omega}}{2\pi}
		z(\omega)
		e^{+i\omega t}
		\\
		&=
		\int_0^{+\infty}\frac{\dd{\omega}}{2\pi}
		z(\omega)
		e^{+i\omega t}
		-
		\int_{\infty}^0\frac{\dd{\omega}}{2\pi}
		z(-\omega)
		e^{-i\omega t}
		\\
		&=
		\int_0^{+\infty}\frac{\dd{\omega}}{2\pi}
		\left[
			z(\omega)
			e^{+i\omega t}
			+
			z(\omega)^*
			e^{-i\omega t}
		\right]
		,
	\end{split}
\end{equation}
with the \gls{lo} signal
\begin{equation}
	\sin(\omega_lt+\vartheta)
	=
	\frac{1}{2i}
	\left[
		e^{+i(\omega_lt+\vartheta)}
		-
		e^{-i(\omega_lt+\vartheta)}
	\right]
\end{equation}
produces a high- and low-frequency band
\begin{equation}
	\begin{split}
		z(t)
		\sin(\omega_lt+\vartheta)
		&=
		\Im
		\int_0^\infty\frac{\dd{\omega}}{2\pi}
		\left[
			e^{+i(\omega-\omega_l)t}
			+
			e^{+i(\omega+\omega_l)t}
		\right]
		z(\omega)
		e^{+i\vartheta}
		\\
		&=
		\Im
		\int_0^\infty\frac{\dd{\omega}}{2\pi}
		\left[
			z(\omega+\omega_l)
			+
			z(\omega-\omega_l)
		\right]
		e^{+i\omega t+i\vartheta}
		.
	\end{split}
\end{equation}
The high-frequency band is highly suppressed by the ideal \gls{lp},
\begin{equation}
	\begin{split}
		v(t)
		&=
		\Im
		\int_0^{B/2}\frac{\dd{\omega}}{2\pi}
		\left[
			z(\omega+\omega_l)
			+
			z(\omega-\omega_l)
		\right]
		e^{+i(\omega t+\vartheta)}
		\\
		&\approx
		\Im
		\int_0^{B/2}\frac{\dd{\omega}}{2\pi}
		z(\omega-\omega_l)
		e^{+i(\omega t+\vartheta)}
		,
	\end{split}
	\label{eq:down_conversion_imag}
\end{equation}
wherein $B$ denotes the effective detector bandwidth.

% compare result with derived photocurrent for balanced detection
% problem of only one quadrature ->
% reference to active and passive coherent receiver from QKD chapter

\begin{figure}[htb]
	\centering
	\includegraphics{figures/circuitikz/down-conversion-dual}
	\caption{Block diagram of dual quadrature down-conversion. The signal $z(t)$ is divided with equal power among an upper and a lower branch. The upper branch is mixed with the phase shifted $\gls{lo}$ signal $\cos(\omega_ct+\vartheta)$. The lower branch is mixed with \gls{lo} signal $\sin(\omega_ct+\vartheta)$. The mixer outputs are filtered separately by a \gls{lp} yielding the down-converted signals $v(t)$ and $w(t)$.}\label{fig:down_conversion_dual}
\end{figure}
\Cref{fig:down_conversion_dual} shows the block diagram corresponding to down-conversion of both quadratures.
The lower branch is equal to the result of the single quadrature downconversion after accounting for the power loss due to the signal split, i.e.,
\begin{equation}
	u(t)
	=
	\Re
	\int_0^B\frac{\dd{\omega}}{2\pi}
	z(\omega+\omega_l)
	e^{+i(\omega t+\vartheta)}
	\label{eq:down_conversion_real}	
\end{equation}

% alternative: heterodyne receiver with IF and digital down-conversion

\begin{table}[htb]
  \centering
  \begin{tabular}{lccccc}
    \toprule
    Receiver design & Homodyne (single) & Homodyne (dual) & Heterodyne \\
    \midrule
    Balanced detectors & \num{1} & \num{2} & \num{1} \\
    Quadratures & \num{1} & \num{2} & \num{2} \\
    Optical complexity & Low & High & Low \\
    Signal bandwidth & High & High & Low \\
    \gls{lo} synchronization & Frequency and phase & Frequency & Bandwidth \\
    \bottomrule
  \end{tabular}
  \caption{Comparison of receiver implementations according to Ref.~\cite{Brunner2017}: The single quadrature homodyne detection offers low optical complexity and high bandwidth but only resolves one of two quadratures and required frequency and phase synchronization of the \gls{lo}. The dual quadrature homodyne detection resolves both quadratures with high bandwidth but requires two balanced detectors increasing the optical complexity and phase synchronization of the \gls{lo}. The heterodyne detection schemes resolves both quadratures with low complexity and no requirements on \gls{lo} synchronization at the cost of signal bandwidth.}\label{tab:receivers}
\end{table}

% heterodyne receiver generalizes (single) homodyne receiver

\FloatBarrier
\subsection{Symbol decoding}

\begin{figure}[htb]
	\centering
	\includegraphics{figures/circuitikz/symbol-decoding}
	\caption{foo}
\end{figure}

\begin{figure}[htb]
	\centering
	\includegraphics{figures/pgfplots/rx-frequency}
	\caption{foo}
\end{figure}

\begin{figure}[htb]
	\centering
	\includegraphics{figures/pgfplots/rx-rand-time}
	\caption{foo}
\end{figure}

		\addcontentsline{toc}{section}{References}
		\printbibliography[title=References]
	\end{refsection}
	
	\chapter{Quantum transmission system}
	\begin{refsection}
		\section{Transmitter}

\subsection{Mach-Zehnder modulator}

\begin{figure}[htb]
	\centering
	\includestandalone{figures/pstricks/mzi-symmetric}
	\caption{Free-space setup of a symmetric \gls{mzi}: The input light field enters a first beam splitter BS1 from the left. The light field exits BS1 to the right and the bottom. Right of BS1, a first phase shifter adds a relative phase of $\varphi_1$. Right of the first phase shifter, a first mirror M1 reflects the light to the bottom, hitting a second beam splitter BS2 from the top. Below BS1, a second mirror M2 directs the light to the right, where a second phase shifter adds a relative phase of $\varphi_2$, and the light hits BS2 from the left.}
\end{figure}
\begin{equation}
	\begin{split}
		\begin{pmatrix}
			\hat{a}_1^\prime \\
			\hat{a}_2^\prime
		\end{pmatrix}
		&=
		\frac{1}{\sqrt{2}}
		\begin{pmatrix}
			i & 1 \\
			1 & i
		\end{pmatrix}
		\begin{pmatrix}
			ie^{i\varphi_1} & 0 \\
			0 & ie^{i\varphi_2}
		\end{pmatrix}
		\frac{1}{\sqrt{2}}
		\begin{pmatrix}
			1 & i \\
			i & 1
		\end{pmatrix}
		\begin{pmatrix}
			\hat{a}_1 \\
			\hat{a}_2
		\end{pmatrix}
		\\
		&=
		e^{i\frac{\varphi_1+\varphi_2}{2}+i\pi}
		\begin{pmatrix}
			\cos(\frac{\varphi_2-\varphi_1}{2}) & \sin(\frac{\varphi_2-\varphi_1}{2}) \\
			-\sin(\frac{\varphi_2-\varphi_1}{2}) & \cos(\frac{\varphi_2-\varphi_1}{2})
		\end{pmatrix}
		\begin{pmatrix}
			\hat{a}_1 \\
			\hat{a}_2
		\end{pmatrix}
	\end{split}
\end{equation}

\begin{table}[htb]
	\centering
	\begin{tabular}{lcc}
		\toprule
		Configuration & Phase relation & Modulation \\
		\midrule
		Push-pull & $\varphi_1=-\varphi_2$ & Amplitude \\
		Push-push & $\varphi_1=+\varphi_2$ & Phase \\
		\bottomrule
	\end{tabular}
	\caption{Configurations of a symmetric \gls{mzm}.}
\end{table}

\begin{equation}
	\begin{pmatrix}
		\hat{a}_1^\prime \\
		\hat{a}_2^\prime
	\end{pmatrix}
		\begin{pmatrix}
			\cos(\theta) & \sin(\theta) \\
			-\sin(\theta) & \cos(\theta)
		\end{pmatrix}
		\begin{pmatrix}
			\hat{a}_1 \\
			\hat{a}_2
		\end{pmatrix}
\end{equation}

\begin{equation}
	\hat{U}(\theta)
	=
	e^{i\theta\left(
		\hat{a}_1
		\hat{a}_2^\dagger
		-
		\hat{a}_1^\dagger
		\hat{a}_2
	\right)}
\end{equation}

\begin{align}
	\hat{a}_1^\prime
	&=
	\hat{U}^\dagger(\theta)
	\hat{a}_1
	\hat{U}(\theta)
	=
	\cos(\theta)
	\hat{a}_1
	+
	\sin(\theta)
	\hat{a}_2
	\\
	\hat{a}_2^\prime
	&=
	\hat{U}^\dagger(\theta)
	\hat{a}_2
	\hat{U}(\theta)
	=
	\cos(\theta)
	\hat{a}_2
	-
	\sin(\theta)
	\hat{a}_1
\end{align}

\subsection{I/Q modulator}

\begin{figure}[htb]
	\centering
	\includestandalone{figures/pstricks/iqm}
	\caption{\gls{iq}-modulator comprising three \gls{mzm} and connected by an input and output fiber. MZM 1 and MZM 2, are used in push-pull configuration for \gls{am}. MZM 3 is used in push-push configuration for setting the relative phase between the upper and lower branch. Vacuum in- and outputs are indicated by the red dashed fiber.}
\end{figure}

\begin{equation}
	\begin{pmatrix}
		\hat{a}_1^\prime \\
		\hat{a}_2^\prime
	\end{pmatrix}
	=
	\frac{1}{\sqrt{2}}
	\begin{pmatrix}
		 1 & ie^{+i\varphi_1} \\
		 ie^{-i\varphi_1} & 1
	\end{pmatrix}
	\begin{pmatrix}
		\hat{a}_1 \\
		\hat{a}_2
	\end{pmatrix}
\end{equation}

\begin{equation}
	\begin{pmatrix}
		\hat{a}_3^{\prime\prime} \\
		\hat{a}_1^{\prime\prime}
	\end{pmatrix}
	=
	\begin{pmatrix}
		 \cos\theta_1 & \sin\theta_1 \\
		 -\sin\theta_1 & \cos\theta_1
	\end{pmatrix}
	\begin{pmatrix}
		\hat{a}_3^\prime \\
		\hat{a}_1^\prime
	\end{pmatrix}
\end{equation}

\begin{equation}
	\begin{pmatrix}
		\hat{a}_1^{\prime\prime} \\
		\hat{a}_2^{\prime\prime} \\
		\hat{a}_3^{\prime\prime} \\
		\hat{a}_4^{\prime\prime}
	\end{pmatrix}
	=
	\begin{pmatrix}
		 \cos\theta_1 & 0 & -\sin\theta_1 & 0 \\
		 0 & \cos\theta_2 & 0 & \sin\theta_2 \\
		 \sin\theta_1 & 0 & \cos\theta_1 & 0 \\
		 0 & -\sin\theta_2 & 0 & \cos\theta_2 \\
	\end{pmatrix}
	\begin{pmatrix}
		\hat{a}_1^{\prime} \\
		\hat{a}_2^{\prime} \\
		\hat{a}_3^{\prime} \\
		\hat{a}_4^{\prime}
	\end{pmatrix}
\end{equation}

\begin{equation}
	\begin{pmatrix}
		\hat{a}_2^{\prime\prime} \\
		\hat{a}_4^{\prime\prime}
	\end{pmatrix}
	=
	\begin{pmatrix}
		 \cos\theta_2 & \sin\theta_2 \\
		 -\sin\theta_2 & \cos\theta_2
	\end{pmatrix}
	\begin{pmatrix}
		\hat{a}_2^\prime \\
		\hat{a}_4^\prime
	\end{pmatrix}
\end{equation}


\begin{equation}
	\begin{split}
		\begin{pmatrix}
			\hat{a}_1^{\prime\prime\prime\prime} \\
			\hat{a}_2^{\prime\prime\prime\prime}
		\end{pmatrix}
		&=
		\frac{1}{\sqrt{2}}
		\begin{pmatrix}
			1 & ie^{+i\varphi_2} \\
			ie^{-i\varphi_2} & 1
		\end{pmatrix}
		\begin{pmatrix}
			\hat{a}_1^{\prime\prime\prime} \\
			\hat{a}_2^{\prime\prime\prime}
		\end{pmatrix}
		\\
		&=
		\frac{1}{\sqrt{2}}
		\begin{pmatrix}
			1 & ie^{+i\varphi_2} \\
			ie^{-i\varphi_2} & 1
		\end{pmatrix}
		\begin{pmatrix}
			e^{+i\phi} & 0 \\
			0 & e^{-i\phi}
		\end{pmatrix}
		\begin{pmatrix}
			\hat{a}_1^{\prime\prime} \\
			\hat{a}_2^{\prime\prime}
		\end{pmatrix}
		\\
		&=
		\frac{1}{\sqrt{2}}
		\begin{pmatrix}
			e^{+i\phi} & ie^{+i(\varphi_2-\phi)} \\
			ie^{-i(\varphi_2-\phi)} & e^{-i\phi}
		\end{pmatrix}
		\begin{pmatrix}
			\hat{a}_1^{\prime\prime} \\
			\hat{a}_2^{\prime\prime}
		\end{pmatrix}
	\end{split}
\end{equation}
		\section{Receiver}
\FloatBarrier

We introduced the receiver as a component decoding a sequence of complex symbols, $\left\{\beta_n\in\mathbb{C}\colon n\in I\right\}$, from a coherent state $\ket{\beta(t)}$.
As with the transmitter, we want our receiver to be software-defined, meaning that the predominant signal processing is in the digital domain and we need to down-convert the optical signal.
\begin{figure}[htb]
	\centering
	\includegraphics{figures/tikz/software-defined-receiver}
	\caption{Block diagram of the receiver's signal processing domains. The analog electrical signals $v(t)$ and $w(t)$ are demodulated from the quadratures of the optical coherent state $\ket{\beta(t)}$, and then converted to the digital signals $v[m]$ and $w[m]$ from which the \gls{dsp} decodes the symbol sequence $\{\beta_n\in\mathbb{C}\colon n\in I\}$.}\label{fig:software_defined_receiver}
\end{figure}
\Cref{fig:software_defined_receiver} illustrates the signal flow across the different domains inside the receiver.
Unlike the transmitter up-conversion, the receiver's down-conversion leaves us some design freedom.
In particular, it is possible to reduce the hardware complexity significantly by introducing an intermediate frequency, which carries over to the digital domain.

\FloatBarrier
\subsection{Down-conversion}

% outlook down-conversion freedoms

\begin{figure}[htb]
	\centering
	\includegraphics{figures/circuitikz/down-conversion-single}
	\caption{Block diagram of single quadrature down-conversion. The signal $z(t)$ is mixed with the \gls{lo} $\sin(\omega_lt+\vartheta)$. The output of the mixing is filtered by a \gls{lp} to produce the down-converted signal $u(t)$.}\label{fig:down_conversion_single}
\end{figure}
The mixing of the real-valued passband signal,
\begin{equation}
	\begin{split}
		z(t)
		=
		\int_{-\infty}^{+\infty}\frac{\dd{\omega}}{2\pi}
		z(\omega)
		e^{+i\omega t}
		&=
		\int_0^{+\infty}\frac{\dd{\omega}}{2\pi}
		z(\omega)
		e^{+i\omega t}
		+
		\int_{-\infty}^0\frac{\dd{\omega}}{2\pi}
		z(\omega)
		e^{+i\omega t}
		\\
		&=
		\int_0^{+\infty}\frac{\dd{\omega}}{2\pi}
		z(\omega)
		e^{+i\omega t}
		-
		\int_{\infty}^0\frac{\dd{\omega}}{2\pi}
		z(-\omega)
		e^{-i\omega t}
		\\
		&=
		\int_0^{+\infty}\frac{\dd{\omega}}{2\pi}
		\left[
			z(\omega)
			e^{+i\omega t}
			+
			z(\omega)^*
			e^{-i\omega t}
		\right]
		,
	\end{split}
\end{equation}
with the \gls{lo} signal
\begin{equation}
	\sin(\omega_lt+\vartheta)
	=
	\frac{1}{2i}
	\left[
		e^{+i(\omega_lt+\vartheta)}
		-
		e^{-i(\omega_lt+\vartheta)}
	\right]
\end{equation}
produces a high- and low-frequency band
\begin{equation}
	\begin{split}
		z(t)
		\sin(\omega_lt+\vartheta)
		&=
		\Im
		\int_0^\infty\frac{\dd{\omega}}{2\pi}
		\left[
			e^{+i(\omega-\omega_l)t}
			+
			e^{+i(\omega+\omega_l)t}
		\right]
		z(\omega)
		e^{+i\vartheta}
		\\
		&=
		\Im
		\int_0^\infty\frac{\dd{\omega}}{2\pi}
		\left[
			z(\omega+\omega_l)
			+
			z(\omega-\omega_l)
		\right]
		e^{+i\omega t+i\vartheta}
		.
	\end{split}
\end{equation}
The high-frequency band is highly suppressed by the ideal \gls{lp},
\begin{equation}
	\begin{split}
		v(t)
		&=
		\Im
		\int_0^{B/2}\frac{\dd{\omega}}{2\pi}
		\left[
			z(\omega+\omega_l)
			+
			z(\omega-\omega_l)
		\right]
		e^{+i(\omega t+\vartheta)}
		\\
		&\approx
		\Im
		\int_0^{B/2}\frac{\dd{\omega}}{2\pi}
		z(\omega-\omega_l)
		e^{+i(\omega t+\vartheta)}
		,
	\end{split}
	\label{eq:down_conversion_imag}
\end{equation}
wherein $B$ denotes the effective detector bandwidth.

% compare result with derived photocurrent for balanced detection
% problem of only one quadrature ->
% reference to active and passive coherent receiver from QKD chapter

\begin{figure}[htb]
	\centering
	\includegraphics{figures/circuitikz/down-conversion-dual}
	\caption{Block diagram of dual quadrature down-conversion. The signal $z(t)$ is divided with equal power among an upper and a lower branch. The upper branch is mixed with the phase shifted $\gls{lo}$ signal $\cos(\omega_ct+\vartheta)$. The lower branch is mixed with \gls{lo} signal $\sin(\omega_ct+\vartheta)$. The mixer outputs are filtered separately by a \gls{lp} yielding the down-converted signals $v(t)$ and $w(t)$.}\label{fig:down_conversion_dual}
\end{figure}
\Cref{fig:down_conversion_dual} shows the block diagram corresponding to down-conversion of both quadratures.
The lower branch is equal to the result of the single quadrature downconversion after accounting for the power loss due to the signal split, i.e.,
\begin{equation}
	u(t)
	=
	\Re
	\int_0^B\frac{\dd{\omega}}{2\pi}
	z(\omega+\omega_l)
	e^{+i(\omega t+\vartheta)}
	\label{eq:down_conversion_real}	
\end{equation}

% alternative: heterodyne receiver with IF and digital down-conversion

\begin{table}[htb]
  \centering
  \begin{tabular}{lccccc}
    \toprule
    Receiver design & Homodyne (single) & Homodyne (dual) & Heterodyne \\
    \midrule
    Balanced detectors & \num{1} & \num{2} & \num{1} \\
    Quadratures & \num{1} & \num{2} & \num{2} \\
    Optical complexity & Low & High & Low \\
    Signal bandwidth & High & High & Low \\
    \gls{lo} synchronization & Frequency and phase & Frequency & Bandwidth \\
    \bottomrule
  \end{tabular}
  \caption{Comparison of receiver implementations according to Ref.~\cite{Brunner2017}: The single quadrature homodyne detection offers low optical complexity and high bandwidth but only resolves one of two quadratures and required frequency and phase synchronization of the \gls{lo}. The dual quadrature homodyne detection resolves both quadratures with high bandwidth but requires two balanced detectors increasing the optical complexity and phase synchronization of the \gls{lo}. The heterodyne detection schemes resolves both quadratures with low complexity and no requirements on \gls{lo} synchronization at the cost of signal bandwidth.}\label{tab:receivers}
\end{table}

% heterodyne receiver generalizes (single) homodyne receiver

\FloatBarrier
\subsection{Symbol decoding}

\begin{figure}[htb]
	\centering
	\includegraphics{figures/circuitikz/symbol-decoding}
	\caption{foo}
\end{figure}

\begin{figure}[htb]
	\centering
	\includegraphics{figures/pgfplots/rx-frequency}
	\caption{foo}
\end{figure}

\begin{figure}[htb]
	\centering
	\includegraphics{figures/pgfplots/rx-rand-time}
	\caption{foo}
\end{figure}
		
		\addcontentsline{toc}{section}{References}
		\printbibliography[title=References]
	\end{refsection}

	\chapter{Conclusion and outlook}
	\begin{refsection}
		\chapter*{Introduction}
\addcontentsline{toc}{chapter}{Introduction}

Optical communication enables humanity worldwide to share information in a split second, with companies like Huawei undergoing tremendous efforts to advance the frontiers.
In addition to incremental innovation increasing the performance and decreasing the cost of optical communication technology, we observe intensified activities towards disruptive innovations that challenge our present understanding of communication.
One such branch of activity is quantum optical communication, incorporating quantum aspects of light into classical communication and leading to novel communication technology like \gls{qkd}, which enables practical and secure key generation.
As a still young discipline, which emerged from two highly advanced fields, communication engineering and quantum physics, quantum communication lacks a unified description to which both communication engineers and quantum physicists agree.
The present thesis aims to resolve the seeming discrepancies between communication engineering and quantum physics by reviewing a practical implementation of a quantum communication system implementing a \gls{qkd} protocol.
In the process, we hope to develop a theoretical framework for quantum optical communication, incorporating quantum effects into classical communication, which has applicability beyond \gls{qkd}.

\subsection*{Problem statement}

To raise awareness of the challenges ahead, we review the best-known quantum theory of light, single-mode quantum optics, along with central ideas from classical communication and outline where these pictures conflict.

In single-mode quantum optics, we model monochromatic light with frequency $\omega_0$ as a quantum harmonic oscillator with unit mass, $m=1$, and Hamiltonian~\cite{Gerry2005,Fox2006}
\begin{equation}
	\hat{H}
	=
	\omega_0
	\hat{a}^\dagger
	\hat{a}
	,
\end{equation}
wherein $\hat{a}$ and $\hat{a}^\dagger$ are the quantum annihilation and creation operators, destroying or creating an excitation or "mode" of frequency $\omega_0$.
The electric field operator,
\begin{equation}
	\hat{E}(t,x)
	=
	\mathcal{E}_0
	\left(
		\hat{a}
		+
		\hat{a}^\dagger
	\right)
	\sin(\omega_0x)
	,
\end{equation}
wherein $\mathcal{E}_0$ has the interpretation of an electric field density, establishes the connection between the quantum harmonic oscillator and electromagnetic radiation, including light~\cite[p.~12]{Gerry2005}.
Two of the most important quantum states are the number and the coherent state,
\begin{align}
	\ket{n}
	&=
	\frac{1}{\sqrt{n!}}
	\left(\hat{a}^\dagger\right)^n
	\ket{0}
	&
	\ket{\alpha}
	&=
	\exp\left(-\frac{1}{2}\abs{\alpha}^2\right)
	\sum_{n=0}^\infty
	\frac{\alpha^n}{\sqrt{n!}}
	\ket{n}
	.
\end{align}
The number state is parametrized by a natural number $n\in\mathbb{N}_0$ counting the number excitations.
The coherent state is parametrized by a complex number $\alpha\in\mathbb{C}$ encoding amplitude and phase.
The expectation value of the electric field operator with respect to a coherent state,
\begin{equation}
	\bra{\alpha}
	\hat{E}(t)
	\ket{\alpha}
	=
	\sqrt{2}
	\abs{\alpha}
	\mathcal{E}_0
	\sin\left(\omega_0t-\theta\right)
	,
\end{equation}
equals a classical monochromatic wave with amplitude proportional to $\abs{\alpha}$ and phase $\theta$~\cite[p.~45]{Gerry2005}.

In communication engineering, light is primarily a means of transmitting signals that appear as frequency bands centered around an optical carrier frequency $\omega_c$, as illustrated in \Cref{fig:signal_spectrum}.
\begin{figure}[ht]
	\centering
	\includegraphics{figures/pgfplots/signal-spectrum}
	\caption{Receiver spectrum comprising multiple signal bands relative to a carrier frequency at $\omega=0$. At \SI{+100}{\mega\hertz}, the spectrum has a pilot tone broadened by phase noise. Centered at \SI{-25}{\mega\hertz}, the spectrum contains a first signal band with \SI{12.5}{\mega\hertz} bandwidth. Centered at \SI{-168.75}{\mega\hertz}, the spectrum contains a second signal band with \SI{12.5}{\mega\hertz} bandwidth. The remaining segments of the spectrum include mirror bands or disturbances.}\label{fig:signal_spectrum}
\end{figure}
The concept of frequency bands is extremely powerful as it allows the transmission of multiple independent signals around one carrier frequency.
More formally, we need to distinguish between base- and passband signals.
For a baseband signal $x_b(t)$, the signal power outside of the signal's bandwidth $B$ is negligible~\cite[p.~15]{Madhow2008}, i.e.,
\begin{align}
	\abs{x_b(\omega)}^2
	&\approx
	0
	&
	\abs{\omega}
	&>
	B/2
	.
\end{align}
For a passband signal $x_p(t)$, the signal power outside the signal's band around a carrier frequency $\omega_c$ is negligible~\cite[p.~16]{Madhow2008}, i.e.,
\begin{align}
	\abs{x_p(\omega)}^2
	&\approx
	0
	&
	\abs{\omega\pm\omega_c}
	&>
	B/2
	.	
\end{align}
\begin{figure}[ht]
	\centering
	\includegraphics{figures/tikz/up-conversion}
	\caption{Power spectrum illustrating up-conversion of a real-valued passband signal with bandwidth $B$ centered at $\omega_0$. Up-conversion by $\omega_c$ shifts the passband to $\omega_c+\omega_0$ and creates a mirror band at $\omega_c-\omega_0$.}\label{fig:up_conversion}
\end{figure}
A baseband signal can be up-converted to a passband signal at carrier frequency $\omega_c$ by shifting the spectrum by $\omega_c$ is known as up-conversion, see \Cref{fig:up_conversion}, and implemented by modulation.
Similar, a passband signal at carrier frequency $\omega_c$ is down-converted to a baseband signal by demodulation~\cite[p.~26]{Madhow2008}.

To sum up, single-mode quantum optics provides precise physical meaning to light, including quantum effects, although limited to monochromatic light.
On the other side, communication engineering provides a framework for efficiently constructing and transmitting signals.
For quantum optical communication, it is inevitable to welcome and incorporate both views.
For instance, people with a background in quantum optics but foreign to communication engineering often advocate the concept of "one state, one universe", where each quantum transmission is completely independent.
However, if we include practical considerations, like assuming a single transmission line, the picture of "one state, one universe" is plagued by several ambiguities.
For example, a single-mode quantum state has a single well-defined frequency $\omega_0$, which by Fourier uncertainty implies infinite temporal duration but makes information transmission absurd.
The typical counter-argument is that single-mode quantum optics implicitly assumes pulses with $\omega_0$ being the center frequency of the pulse.
While the counter-argument is technically valid, we must admit that it only raises new questions, such as bandwidth-limitations on the pulse parameters, all properly addressed in communication engineering.

The multi-mode quantum optics mentioned in popular quantum optics books~\cite{Gerry2005,Fox2006} are insufficient to represent continuous-time signals, and performing a continuum limit might not be correct if we consider the huge differences between linear algebra and functional analysis.
The advanced quantum optics literature~\cite{Vogel2006,Mandel1995} does sometimes use a continuous-mode formalism but does not explicitly investigate its properties.
We are only aware of two books~\cite{Loudon2000,Barnett2002} that explicitly present a continuous-mode theory of light but again open up new questions regarding the fundamental assumptions and justification thereof.
If we are willing to go one step deeper, we find answers in the quantum field theory literature~\cite{Peskin1995,Srednicki2007,Greiner2013,Itzykson2012}, but it is up to us to transfer these insights from particle physics to quantum optics applications.
We even have to go a bit deeper and look into mathematical quantum field theory~\cite{Streater2016,Bogoliubov1982,Bogolubov1989} to answer some questions.
Finally, we want to understand and upgrade quantum models of (electro-)optical components in the literature~\cite{Vogel2006,Leonhardt2003,Haroche2006,Mandel1995} to a mode continuum for comparison with the results from the optical communication community~\cite{Shapiro2009,Kikuchi2016}.

\subsection*{Thesis outline}

Our work is divided into four chapters.
In \Cref{ch:qkd}, we present an introduction to \gls{qkd}, emphasizing the similarities between the plethora of seemingly different protocols and attempting to argue why practical \gls{qkd} based on weak coherent states is effectively a coherent state communication system.
In the following three chapters, we construct our theoretical framework for quantum optical communication towards practical \gls{qkd}, starting from a general quantum theory of light, \Cref{ch:light}, over applying the quantum theory to describe the building blocks of coherent communication systems, \Cref{ch:components}, to an abstract description of a coherent state transmission system's signal-processing, \Cref{ch:system}.
While the thesis chapter structure supports a bottom-up approach, it is equally possible to read the thesis from the back to the front, revealing more and more details.
Likewise, it is possible to skip certain chapters and pare down to the chapter summary at the end of each chapter.
		\documentclass[tikz]{standalone}

\usepackage{amsmath}
\usepackage{physics}

\usetikzlibrary{arrows.meta,fit,positioning}

\begin{document}
	\begin{tikzpicture}[
		node distance=6em,
		block/.style={draw, very thick, minimum height=10ex, minimum width=6em, align=center},
	]
		\node (enc) [block] {Quantum\\encoder};
		\node (pch) [block, right=of enc] {Physical\\quantum\\channel};
		\node (dec) [block, right=of pch] {Quantum\\decoder};
		\node[block, fit=(enc) (dec), label={Logical quantum channel}] (lqc) {};
%		\node[block, below=0.8cm of lqc, minimum height=8ex, minimum width=12em] {Post-processing};
		
		\coordinate[left=6em of lqc] (in);
		\coordinate[right=6em of lqc] (out);
		
		\draw[-Latex] (in) -- node[above]{$\ket{\alpha_1,\dots,\alpha_n}$} (lqc);
		\draw[-Latex] (enc) -- node[above]{$\ket{\alpha(t)}$} (pch);
		\draw[-Latex] (pch) -- node[above]{$\ket{\beta(t)}$} (dec);
		\draw[-Latex] (lqc) -- node[above]{$\ket{\beta_1,\dots,\beta_n}$} (out);
	\end{tikzpicture}
\end{document}


		\addcontentsline{toc}{section}{References}
		\printbibliography[title=References]
	\end{refsection}

	\appendix
	
%	\chapter{Theorems}
%	\begin{refsection}
%		\section{Relativistic field theory}

\begin{definition}[Klein-Gordon Lagrangian]
	The Klein-Gordon Lagrangian density
	\begin{equation}
		\mathcal{L}
		=
		\frac{1}{2}
		\left(\partial_\mu\phi\right)
		\left(\partial^\mu\phi\right)
		-
		\frac{1}{2}
		m^2\phi^2
		\label{eq:kg_lagrangian}
	\end{equation}
	describes a real-valued massive scalar field $\phi(t,\vb{x})$.
\end{definition}
\begin{corollary}
	The Klein-Gordon Lagrangian is manifest Lorentz-covariant, i.e., as a Lorentz scalar, the Klein-Gordon Lagrangian is invariant under Lorentz transformations and thereby valid in any reference frame.
\end{corollary}
\begin{theorem}[Relativistic energy-momentum relation]\label{th:relativistic_energy_momentum}
	Excitations of the Klein-Gordon field satisfy the relativistic energy-momentum relation
	\begin{equation}
		\omega(\vb{p})
		=
		\sqrt{\vb{p}^2+m^2}
		=
		E(\vb{p})
		\label{eq:energy_momentum_relation}
		.
	\end{equation}
\end{theorem}
\begin{theorem}[Mode expansion of the Klein-Gordon field]\label{thm:kg_fourier_expansion}
	The mode expansion of the Klein-Gordon field
	\begin{equation}
		\phi(x^\mu)
		=
		\int_{\mathbb{R}^3}\frac{\dd[3]{p}}{(2\pi)^3\sqrt{2\omega(\vb{p})}}
		\biggl\{
			a(\vb{p})
			e^{-ip_\mu x^\mu}
			+
			a(\vb{p})^*
			e^{+ip_\mu x^\mu}
		\biggr\}_{p_0=\omega(\vb{p})}
	\end{equation}
	satisfies the equations of motion for any choice of $a(\vb{p})=\phi(\omega(\vb{p}),\vb{p})$ where $\phi(p_0,\vb{p})=\phi(p^\mu)$ is the Fourier amplitude of the Klein-Gordon field $\phi(x^\mu)=\phi(t,\vb{x})$.
\end{theorem}
\begin{definition}[Energy-momentum tensor]
	The energy-momentum tensor of a scalar field is
	\begin{equation}
		T^{\mu\nu}
		=
		\pdv{\mathcal{L}}{(\partial_\mu\phi)}\partial^\nu\phi
		-
		g^{\mu\nu}\mathcal{L}
		\label{eq:energy_momentum_tensor}
		.
	\end{equation}
	The energy-momentum tensor's components encode the energy density $T^{00}$, the momentum density $T^{0i}$, and the stress densities $T^{ij}$.
\end{definition}
\begin{lemma}
	The energy density of the Klein-Gordon's energy-momentum tensor
	\begin{equation*}
		T^{00}
		=
		\frac{1}{2}
		\left(\partial_t\phi\right)^2
		+
		\frac{1}{2}
		\left(\grad\phi\right)^2
		+
		\frac{1}{2}
		\left(m\phi\right)^2
		=
		\mathcal{H}
	\end{equation*}
	is equal to its Hamiltonian density $\mathcal{H}$.
\end{lemma}
\begin{lemma}\label{thm:kg_total_energy_momentum}
	The total energy and the total momentum of the Klein-Gordon field are
	\begin{align}
		H
		=
		\int\frac{\dd[3]{p}}{(2\pi)^3}
		\omega(\vb{p})\abs{a(\vb{p})}^2
		&&
		\vb{P}
		=
		\int\frac{\dd[3]{p}}{(2\pi)^3}
		\vb{p}\abs{a(\vb{p})}^2
		\label{eq:kg_energy_momentum}
		.
	\end{align}
\end{lemma}
\begin{definition}
	The Klein-Gordon Hamiltonian density is
	\begin{equation}
		\mathcal{H}
		=
		\frac{1}{2}
		\pi^2
		+
		\frac{1}{2}
		\left(\grad\phi\right)^2
		+
		\frac{1}{2}
		\left(m\phi\right)^2
		.
	\end{equation}
	$\phi(x^\mu)$ is the position and $\pi(x^\mu)$ the momentum density.
\end{definition}
\begin{lemma}
	The mode expansions of the Klein-Gordon's position and momentum densities read
	\begin{align}
		\phi(x^\mu)
		&=
		\int_{\mathbb{R}^3}\frac{\dd[3]{p}}{(2\pi)^3}
		\frac{1}{\sqrt{2\omega(\vb{p})}}
		\biggl\{
			a(\vb{p})
			e^{-ip_\mu x^\mu}
			+
			a(\vb{p})^*
			e^{+ip_\mu x^\mu}
		\biggr\}_{p_0=\omega(\vb{p})}
		,
		\\
		\pi(x^\mu)
		&=
		\int_{\mathbb{R}^3}\frac{\dd[3]{p}}{(2\pi)^3}
		\left(-i\sqrt{\frac{\omega(\vb{p})}{2}}\right)
		\biggl\{
			a(\vb{p})
			e^{-ip_\mu x^\mu}
			-
			a(\vb{p})^*
			e^{+ip_\mu x^\mu}
		\biggr\}_{p_0=\omega(\vb{p})}
		.
	\end{align}
\end{lemma}
\begin{definition}
	The current
\end{definition}
%		\section{Relativistic wave packets}

\begin{definition}[Coordinate wave function]
	The Klein-Gordon states $\ket{\psi}$ coordinate wave function is
	\begin{equation}
		\psi(x^\mu)
		=
		\bra{0}
		\hat\phi(x^\mu)
		\ket{\psi}
		.
	\end{equation}
\end{definition}
\begin{definition}[Relativistic probability current]
	The relativistic probability current is
	\begin{equation}
		j_\mu(x^\mu)
		=
		2
		\Im\left\{
			\psi(x^\mu)^*
			\partial_\mu
			\psi(x^\mu)
		\right\}
		\label{eq:qkg_probability_current}
	\end{equation}
	where $\psi(t,\vb{x})$ is a coordinate wave function.
	The time component $j^0(t,\vb{x})$ is equal to the probability density $\rho(t,\vb{x})$ and the spatial components $j^i(t,\vb{x})$ are equal to the probability current.
\end{definition}
\begin{definition}[Localization]
	The center-of-mass position of the probability density
	\begin{equation}
		\expval{\vb{x}(t)}
		=
		\int\dd[3]{x}
		\vb{x}
		\rho(t,\vb{x})
	\end{equation}
	lets us localize a Klein-Gordon state.\footnote{There exists no position operator free of contradictions in quantum field theory!}
\end{definition}
\begin{definition}[Group velocity]
	The total probability current
	\begin{equation}
		\expval{\vb{v}(t)}
		=
		\int\dd[3]{x}
		\vb{j}(t,\vb{x})
		\label{eq:group_velocity}
	\end{equation}
	equals the group velocity of a field excitation.
\end{definition}
\begin{definition}[Spatial dispersion]
	The spatial dispersion is equal to the variance of the position weighted by the probability density
	\begin{equation}
		\sigma_x(t)^2
		=
		\expval{\vb{x}(t)^2}
		-
		\expval{\vb{x}(t)}^2
	\end{equation}
	and quantifies the spatial spread of a wave packet.
\end{definition}

\begin{definition}[Covariant Gaussian spectrum]
	The spectrum of a Gaussian number state is
	\begin{equation}
		f(\vb{p})
		\propto
		\exp\left\{
			\frac{(p_\mu-k_\mu)(p^\mu-k^\mu)}{4\sigma^2}
		\right\}_{\substack{p_0=\omega(\vb{p})\\k_0=\omega(\vb{k})}}
		\label{eq:covariant_gaussian_spectrum}
	\end{equation}
	where the spectrum has mean $k^\mu=(k_0,\vb{k})$ and variance $\sigma^2$.\footnote{See Ref.~\cite{Naumov2013,Naumov2009} for a in-depth discussion.}
\end{definition}
\begin{lemma}\label{thm:non_relativistic_gaussian_momentum}
%{nonrelativisticgaussianmom}
	For massless particles and $\omega(\vb{p})\ll\sigma$, the covariant Gaussian spectrum can be approximated
	\begin{equation}
		f(\vb{p})
		\propto
		\exp\left\{
			-
			\frac{
				\vb{p}_\perp^2
			}{4\sigma^2}
		\right\}
	\end{equation}
	wherein $\vb{p}_\perp$ is the momentum transverse to $\vb{k}$, for instance, if $\vb{k}=k_0\vu{e}_z$ then $\vb{p}_\perp^2=p_x^2+p_y^2$.
\end{lemma}
\begin{proof}
	For massless particles, we have $p_\mu p^\mu=0=k_\mu k^\mu$ and \cref{eq:covariant_gaussian_spectrum} reduces to
	\begin{equation*}
		f(\vb{p})
		\propto
		\exp\left\{
			-
			\frac{\omega(\vb{p})\omega(\vb{k})-\vb{p}\vdot\vb{k}}{2\sigma^2}
		\right\}
		.
	\end{equation*}
	For $\omega(\vb{p})\ll\sigma$, we perform a Taylor expansion around $\vb{k}$
	\begin{equation*}
		\omega(\vb{p})
		=
		\omega(\vb{k})
		+
		\omega^\prime(\vb{k})
		\vdot
		(\vb{p}-\vb{k})
		+
		\frac{1}{2}
		(\vb{p}-\vb{k})^\trans
		\omega^{\prime\prime}(\vb{k})
		(\vb{p}-\vb{k})
		+
		\order{\vb{p}^3}
	\end{equation*}
	where $\omega^\prime(\vb{k})=\vb{k}/\omega(\vb{k})$ denotes the gradient and $\omega^{\prime\prime}(\vb{k})$ the Hessian of $\omega(\vb{p})$ evaluated at $\vb{k}$ and further reducing the spectrum to
	\begin{equation*}
		f(\vb{p})
		\propto
		\exp\left\{
			-
			\frac{
				(\vb{p}-\vb{k})^\trans
				\omega(\vb{k})
				\omega^{\prime\prime}(\vb{k})
				(\vb{p}-\vb{k})
			}{4\sigma^2}
		\right\}
		.
	\end{equation*}
	The expressions
	\begin{equation*}
		\omega(\vb{k})
		\omega^{\prime\prime}(\vb{k})_{ij}
		=
		\omega(\vb{k})
		\pdv{\omega(\vb{k})}{k_i}{k_j}
		=
		\delta_{ij}
		-
		\frac{k_ik_j}{\omega(\vb{k})^2}
		=
		\delta_{ij}
		-
		\frac{k_ik_j}{\vb{k}^2}
		=
		P_\perp
	\end{equation*}
	turns out to be equal to the transverse projector, see \Cref{def:projector_long_trans}.
	Inserting the transverse project, we find the non-relativistic Gaussian spectrum to be approximately equal to
	\begin{equation*}
		f(\vb{p})
		\propto
		\exp\left\{
			-
			\frac{
				(\vb{p}-\vb{k})^\trans
				P_\perp(\vb{k})
				(\vb{p}-\vb{k})
			}{4\sigma^2}
		\right\}
		=
		\exp\left\{
			-
			\frac{
				\vb{p}_\perp^2
			}{4\sigma^2}
		\right\}
		.
	\end{equation*}
	The final result is confirmed by Ref.~\cite[eq.~(25)]{Naumov2013} when removing the longitudinal momentum component.
\end{proof}

\begin{example}
	The smearing function
	\begin{equation*}
		f(p_0,\vb{p})
		=
		\sqrt{2\omega(\vb{p}_0)}
		\left(\frac{2\pi}{\sigma^2}\right)^{\frac{3}{4}}
		\exp\left\{
			-
			\frac{(\vb{p}-\vb{p}_0)^2}{4\sigma^2}
		\right\}
	\end{equation*}
	is normalized
	\begin{equation*}
		\begin{split}
			\int\frac{\dd[3]{p}}{(2\pi)^32\omega(\vb{p})}
			\abs{f\left(\omega(\vb{p}),\vb{p}\right)}^2
			&=
			2\omega(\vb{p}_0)
			\left(\frac{2\pi}{\sigma^2}\right)^{\frac{3}{2}}
			\int\frac{\dd[3]{p}}{(2\pi)^32\omega(\vb{p})}
			\exp\left\{
				-
				\frac{(\vb{p}-\vb{p}_0)^2}{2\sigma^2}
			\right\}
			\\
			&=
			\left(\frac{2\pi}{\sigma^2}\right)^{\frac{3}{2}}
			\left(
				\int\frac{\dd[3]{p}}{2\pi}
				\exp\left\{
					-
					\frac{(p-p_0)^2}{2\sigma^2}
				\right\}
			\right)^3
			\\
			&=
			\left(\frac{2\pi}{\sigma^2}\right)^{\frac{3}{2}}
			\left(
				\frac{1}{2\pi}
				\sqrt{2\pi\sigma^2}
			\right)^3
			=
			1
		\end{split}
	\end{equation*}
	where we used the mean value theorem in the second line.
	The coordinate representation of the smearing function is
	\begin{equation*}
		\begin{split}
			f_1(t,\vb{x})
			&=
			\int\frac{\dd[4]{p}}{(2\pi)^4}
			f(p_0,\vb{p})
			e^{+ip_\mu x^\mu}
			\\
			&=
			\sqrt{2\omega(\vb{p}_0)}
			\left(\frac{2\pi}{\sigma^2}\right)^{\frac{3}{4}}
			\int\frac{\dd[4]{p}}{(2\pi)^4}
			\exp\left\{
				-
				\frac{(\vb{p}-\vb{p}_0)^2}{4\sigma^2}
			\right\}
			e^{+i(p_0t-\vb{p}\vdot\vb{x})}
			\\
			&=
			\sqrt{2\omega(\vb{p}_0)}
			\left(\frac{2\pi}{\sigma^2}\right)^{\frac{3}{4}}
			\int\frac{\dd{p_0}}{2\pi}
			e^{-ip_0t}
			\int\frac{\dd[3]{p}}{(2\pi)^3}
			\exp\left\{
				-
				\frac{(\vb{p}-\vb{p}_0)^2}{4\sigma^2}
				-
				i\vb{p}\vdot\vb{x}
			\right\}
			\\
			&=
			\sqrt{2\omega(\vb{p}_0)}
			\left(\frac{2\pi}{\sigma^2}\right)^{\frac{3}{4}}
			\delta(t)
			e^{-i\vb{p}_0\vdot\vb{x}}
			\left(
				\int\frac{\dd{q}}{2\pi}
				\exp\left\{
					-
					\frac{q^2+2q(2i\sigma^2x)+(2i\sigma^2x)^2-(2i\sigma^2x)^2}{4\sigma^2}
				\right\}
			\right)^3
			\\
			&=
			\sqrt{2\omega(\vb{p}_0)}
			\left(\frac{2\pi}{\sigma^2}\right)^{\frac{3}{4}}
			\delta(t)
			e^{-i\vb{p}_0\vdot\vb{x}}
			e^{-(\sigma\vb{x})^2}
			\left(
				\int\frac{\dd{q}}{2\pi}
				\exp\left\{
					-
					\frac{(q+i\sigma^2x)^2}{4\sigma^2}
				\right\}
			\right)^3
			\\
			&=
			\sqrt{2\omega(\vb{p}_0)}
			\left(\frac{2\pi}{\sigma^2}\right)^{\frac{3}{4}}
			\delta(t)
			e^{-i\vb{p}_0\vdot\vb{x}}
			e^{-(\sigma\vb{x})^2}
			\left(
				\frac{1}{2\pi}
				\sqrt{4\pi\sigma^2}
			\right)^3
			\\
			&=
			\sqrt{2\omega(\vb{p}_0)}
			\left(\frac{2\pi}{\sigma^2}\right)^{\frac{3}{4}}
			\delta(t)
			e^{-i\vb{p}_0\vdot\vb{x}}
			e^{-(\sigma\vb{x})^2}
			\left(
				\frac{\sigma^2}{\pi}
			\right)^\frac{3}{2}
			\\
			&=
			\sqrt{2\omega(\vb{p}_0)}
			\left(\frac{2\sigma^2}{\pi}\right)^{\frac{3}{4}}
			\delta(t)
			e^{-i\vb{p}_0\vdot\vb{x}}
			e^{-(\sigma\vb{x})^2}
			,
		\end{split}
	\end{equation*}
	hence, at time $t=0$, the wave packet is localized at $\vb{x}=0$ with spatial spread $\sigma$.

	We conclude that the smearing function represents the state at some initial time.
	The complete time-dependent coordinate representation is encoded in the wave function
	\begin{equation*}
		\begin{split}
			\psi_1(t,\vb{x})
			&=
			\int\frac{\dd[3]{p}}{(2\pi)^32p_0}
			\eval{
				f(p_0,\vb{p})^*
				e^{+ip_\mu x^\mu}
			}_{p_0=\omega(\vb{p})}
			\\
			&=
			2\omega(\vb{p}_0)
			\left(\frac{2\pi}{\sigma^2}\right)^{\frac{3}{2}}
			\int\frac{\dd[3]{p}}{(2\pi)^32\omega(\vb{p})}
			\exp\left\{
				-
				\frac{(\vb{p}-\vb{p}_0)^2}{4\sigma^2}
			\right\}
			e^{+i\omega(\vb{p})t-i\vb{p}\vdot\vb{x}}
			\\
			&=
			\left(\frac{2\pi}{\sigma^2}\right)^{\frac{3}{2}}
			\int\frac{\dd[3]{p}}{(2\pi)^3}
			\exp\left\{
				-
				\left(
					\frac{\vb{p}-\vb{p}_0}{2\sigma}
				\right)^2
				-
				i\vb{p}\vdot\vb{x}
				+
				i\omega(\vb{p})t
			\right\}
			\\
			&=
			\left(\frac{2}{\pi}\right)^{\frac{3}{2}}
			\int\dd[3]{q}
			\exp\left\{
				-
				\vb{q}^2
				-
				i\left(\vb{p}_0+2\sigma\vb{q}\right)\vdot\vb{x}
				+
				i\omega(\vb{p}_0+2\sigma\vb{q})t
			\right\}
			\\
			&=
			\left(\frac{2}{\pi}\right)^{\frac{3}{2}}
			e^{-i\vb{p}_0\vdot\vb{x}}
			\int\dd[3]{q}
			\exp\left\{
				-
				\vb{q}^2
				-
				2i\sigma
				\vb{q}\vdot\vb{x}
				+
				i\omega(\vb{p}_0+2\sigma\vb{q})t
			\right\}
		\end{split}
	\end{equation*}
	where we again used the mean-value theorem in the second line.
	To make any further progress, we need to expand the dispersion relation
	\begin{equation*}
		\begin{split}
			\omega(\vb{p}_0+2\sigma\vb{q})
			&=
			\omega(\vb{p}_0)
			+
			2\sigma\omega^\prime(\vb{p}_0)\vdot\vb{q}
			+
			4\sigma^2
			\vb{q}^\trans
			\omega^{\prime\prime}(\vb{p}_0)
			\vb{q}
			+
			\order{\vb{q}^3}
			\\
			&=
			\omega(\vb{p}_0)
			+
			2\sigma\vu{p}_0\vdot\vb{q}
			+
			\frac{4\sigma^2}{\omega(\vb{p}_0)}
			q^i
			\left(\delta_{ij}-\frac{p_{0i}p_{0j}}{\vb{p}_0^2}\right)
			q^j
			+
			\order{\vb{q}^3}
			\\
			&=
			\omega(\vb{p}_0)
			+
			2\sigma\vu{p}_0\vdot\vb{q}
			+
			\frac{4\sigma^2}{\omega(\vb{p}_0)}
			\left(
				\vb{q}^2
				-
				(\vu{p}_0\vdot\vb{q})^2
			\right)
			+
			\order{\vb{q}^3}
			\\
			&=
			\omega(\vb{p}_0)
			+
			\omega(2\sigma\vb{q}_\parallel)
			+
			\frac{\omega(2\sigma\vb{q}_\perp)^2}{\omega(\vb{p}_0)}
			+
			\order{\vb{q}^3}
		\end{split}
	\end{equation*}
	where we have used the longitudinal momentum $\vb{q}_\parallel=(\vb{q}\vdot\vu{p}_0)\vu{p}_0$ and the transverse momentum $\vb{q}_\perp^2=\vb{q}^2-\vb{q}_\parallel^2$
	\begin{equation*}
		\psi(x^\mu)
		\approx
		\left(\frac{2}{\pi}\right)^\frac{3}{2}
		e^{i\omega(\vb{p}_0)t-i\vb{p}_0\vdot\vb{x}}
		\int\dd[3]{q}
		\exp\left\{
			-
			\vb{q}^2
			-
			2i\sigma
			\vb{q}
			\vdot
			\vb{x}
			+
			i\omega(2\sigma\vb{q}_\parallel)t
			+
			i\frac{\omega(2\sigma\vb{q}_\perp)^2}{\omega(\vb{p}_0)}t
		\right\}
	\end{equation*}
	Switching to cylindrical coordinates and further evaluating the integral
	\begin{equation*}
		\begin{split}
			&\
			\int\dd{q_\perp}\dd{q_\parallel}\dd{\theta} q_\perp
			\exp\left\{
				-
				q_\perp^2
				-
				q_\parallel^2
				-
				2i\sigma
				\left(
					q_\perp
					\cos\theta
					x^1
					+
					q_\perp
					\sin\theta
					x^2
					+
					q_\parallel
					x^3
				\right)
				+
				i2\sigma q_\parallel t
				+
				i\frac{4\sigma^2 q_\perp^2}{\omega(\vb{p}_0)}t
			\right\}
			\\
			=&\
			\int\dd{q_\perp}\dd{\theta} q_\perp
			\exp\left\{
				-
				q_\perp^2
				-
				2i\sigma
				q_\perp
				\left(
					\cos\theta
					x^1
					+
					\sin\theta
					x^2
				\right)
				+
				i\frac{4\sigma^2 q_\perp^2}{\omega(\vb{p}_0)}t
			\right\}
			\int\dd{q_\parallel}
			\exp\left\{
				-
				q_\parallel^2
				-
				2i\sigma
				q_\parallel x^3
				+
				2i\sigma q_\parallel t
			\right\}
		\end{split}		
	\end{equation*}
	The second integral is
	\begin{equation*}
		\begin{split}
			&\
			\int\dd{q_\parallel}
			\exp\left\{
				-
				q_\parallel^2
				-
				2q_\parallel i\sigma (x^3-t)
				-
				\left(i\sigma (x^3-t)\right)^2
				+
				\left(i\sigma (x^3-t)\right)^2
			\right\}
			\\
			=&\
			e^{-\sigma(x^3-t)^2}
			\int\dd{q_\parallel}
			\exp\left\{
				-
				\left(
					q_\parallel^2
					+
					i\sigma (x^3-t)
				\right)^2
			\right\}
			\\
			=&\
			\sqrt{\pi}
			e^{-\sigma^2(x^3-t)^2}
		\end{split}
	\end{equation*}
	and the first integral becomes
	\begin{equation*}
		\begin{split}
			&\
			\int\dd{q_\perp}\dd{\theta} q_\perp
			\exp\left\{
				-
				q_\perp^2
				-
				2i\sigma
				q_\perp
				\left(
					\cos\theta
					x^1
					+
					\sin\theta
					x^2
				\right)
				+
				i\frac{4\sigma^2 q_\perp^2}{\omega(\vb{p}_0)}t
			\right\}
			\\
			=&\
			\int\dd{q_\perp}q_\perp
			\exp\left\{
				-
				\left(
					1
					-
					4i\frac{\sigma^2t}{\omega(\vb{p}_0)}
				\right)
				q_\perp^2
			\right\}
			\int\dd{\theta}
			\exp\left\{
				-
				2i\sigma
				q_\perp
				\left(
					\cos\theta
					x^1
					+
					\sin\theta
					x^2
				\right)
			\right\}
			\\
			=&\
			\int\dd{q_\perp}q_\perp
			\exp\left\{
				-
				\left(
					1
					-
					4i\frac{\sigma^2t}{\omega(\vb{p}_0)}
				\right)
				q_\perp^2
			\right\}
			2\pi
			I_0\left(2i\sigma q_\perp\rho\right)
			\\
			\approx &\
			\frac{2\pi}{2\left(1-4i\sigma^2 t/\omega(\vb{p}_0)\right)}
		\end{split}
	\end{equation*}
	Putting these things together, we have for $\rho\ll1$
	\begin{equation*}
		\begin{split}
			\psi(t,\rho,z)
			&\approx
			\left(\frac{2}{\pi}\right)^\frac{3}{2}
			e^{i\omega(\vb{p}_0)t-i\vb{p}_0\vdot\vb{x}}
			\frac{\pi}{\left(1-4i\sigma^2 t/\omega(\vb{p}_0)\right)}
			\sqrt{\pi}
			e^{-\sigma^2(z-t)^2}
			\\
			&=
			\frac{2^\frac{3}{2}e^{-\sigma^2(z-t)}}{1-4i\sigma^2 t/\omega(\vb{p}_0)}
			e^{i\omega(\vb{p}_0)t-i\vb{p}_0\vdot\vb{x}}
		\end{split}
	\end{equation*}
	comparing this to the smearing function, we find
	\begin{align*}
		f(t,\vb{x})
		&\approx
		\sqrt{2\omega(\vb{p}_0)}
		\left(\frac{2\sigma^2}{\pi}\right)^{\frac{3}{4}}
		\delta(t)
		e^{-i\vb{p}_0\vdot\vb{x}}
		e^{-(\sigma\vb{x})^2}
		\\
		\psi(t,\rho,z)
		&\approx
		\frac{2^\frac{3}{2}e^{-\sigma^2(z-t)}}{1-4i\sigma^2 t/\omega(\vb{p}_0)}
		e^{i\omega(\vb{p}_0)t-i\vb{p}_0\vdot\vb{x}}
	\end{align*}
	which for $t=0$ are equal up to a proportionality constant.
\end{example}
\begin{example}
	The smearing function
	\begin{equation*}
		f(\omega,\vb{p})
		=
		\sqrt{2\omega_0}
		\left(\frac{1}{2\pi\Delta\omega^2}\right)^\frac{1}{4}
		\exp\left\{
			-
			\frac{(\omega-\omega_0)^2}{4\Delta\omega^2}
		\right\}
		e^{+i\omega t}
		(2\pi)\delta^{(1)}(p^1)
		(2\pi)\delta^{(1)}(p^2)
	\end{equation*}
	is normalized and the coordinate representation turns out to be
	\begin{equation*}
		\begin{split}
			f(t,\vb{x})
			&=
			\int\frac{\dd[4]{p}}{(2\pi)^4}
			f(p_0,\vb{p})
			e^{+ip_\mu x^\mu}
			\\
			&=
			\sqrt{2\omega_0}
			\left(\frac{1}{2\pi\Delta\omega^2}\right)^\frac{1}{4}
			\int\dd{\omega}
			\exp\left\{
				-
				\left(\frac{\omega-\omega_0}{2\Delta\omega}\right)^2
				+
				i\omega t
			\right\}
			\int\frac{\dd{p^3}}{2\pi}
			e^{-ip^3z}
			\\
			&=
			\sqrt{2\omega_0}
			\left(\frac{1}{2\pi\Delta\omega^2}\right)^\frac{1}{4}
			e^{+i\omega_0t}
			\delta^{(1)}(z)
			2\Delta\omega
			\int\dd{u}
			\exp\left\{
				-
				u^2
				+
				2u i\Delta\omega t
			\right\}
			\\
			&=
			\sqrt{2\omega_0}
			\left(\frac{1}{2\pi\Delta\omega^2}\right)^\frac{1}{4}
			e^{+i\omega_0t}
			e^{-\Delta\omega^2 t}
			\delta^{(1)}(z)
			2\Delta\omega
			\sqrt{\pi}
			\\
			&=
			\sqrt{2\omega_0}
			\left(8\pi\Delta\omega^2\right)^\frac{1}{4}
			e^{+i\omega_0t}
			e^{-\Delta\omega^2 t}
			\delta^{(1)}(z)
			.
		\end{split}
	\end{equation*}
	The coordinate wave function is
	\begin{equation*}
		\begin{split}
			\psi(t,\vb{x})
			&=
			\int\frac{\dd[3]{p}}{(2\pi)^32\omega(\vb{p})}
			\eval{
				f(p_0,\vb{p})^*
				e^{+ip_\mu x^\mu}
			}_{p_0=\omega(\vb{p})}
			\\
			&=
			\left(\frac{1}{2\pi\Delta\omega^2}\right)^\frac{1}{4}
			\int\dd{\omega}
			\exp\left\{
				-
				\left(\frac{\omega-\omega_0}{2\Delta\omega}\right)^2
				+
				i\omega(t-z)
			\right\}
			\\
			&=
			\left(\frac{1}{2\pi\Delta\omega^2}\right)^\frac{1}{4}
			2\Delta\omega
			\int\dd{u}
			\exp\left\{
				-
				u^2
				+
				i(\omega_0+2\Delta\omega u)(t-z)
			\right\}
			\\
			&=
			\left(\frac{8\Delta\omega^2}{\pi}\right)^\frac{1}{4}
			e^{-i\omega_0(t-z)}
			\int\dd{u}
			\exp\left\{
				-
				u^2
				+
				2u i\Delta\omega(t-z)
			\right\}
			\\
			&=
			\left(\frac{8\Delta\omega^2}{\pi}\right)^\frac{1}{4}
			e^{-i\omega_0(t-z)}
			\sqrt{\pi}
			e^{-\Delta\omega^2(t-z)}
			\\
			&=
			\left(8\pi\Delta\omega^2\right)^\frac{1}{4}
			e^{-i\omega_0(t-z)}
			e^{-\Delta\omega^2(t-z)}
		\end{split}
	\end{equation*}
\end{example}
\begin{lemma}
	More generally, we find
	\begin{equation}
		\psi(x^\mu)
		=
		\int\dd[4]{y}
		D(x^\mu-y^\mu)
		f(y^\mu)
		,
	\end{equation}
	i.e., the smearing function is an initial wave function evolved to another point in time.
\end{lemma}
\begin{proof}
	\begin{equation*}
		\begin{split}
			\int\dd[4]{y}
			D(x^\mu-y^\mu)
			f(y^\mu)
			&=
			\int\dd[4]{y}
			\int\frac{\dd[3]{p}}{(2\pi)^32\omega(\vb{p})}
			\eval{e^{-ip_\mu(x^\mu-y^\mu)}}_{p_0=\omega(\vb{p})}
			\int\frac{\dd[4]{q}}{(2\pi)^4}
			f(q^\mu)
			e^{+iq_\mu y^\mu}
			\\
			&=
			\int\frac{\dd[3]{p}}{(2\pi)^32\omega(\vb{p})}
			\eval{e^{-ip_\mu x^\mu}}_{p_0=\omega(\vb{p})}
			\int\frac{\dd[4]{q}}{(2\pi)^4}
			f(q^\mu)
			\\
			&\times
			\int\dd{y^0}
			e^{+i(\omega(\vb{p})+q^0)y^0}
			\int\dd[3]{y}
			e^{-i(\vb{q}+\vb{p})\vdot\vb{y}}
			\\
			&=
			\int\frac{\dd[3]{p}}{(2\pi)^32\omega(\vb{p})}
			f\left(\omega(\vb{p}),\vb{p}\right)^*
			\eval{e^{-ip_\mu x^\mu}}_{p_0=\omega(\vb{p})}
			=
			\psi(x^\mu)
		\end{split}
	\end{equation*}
\end{proof}

%		\section{Canonical quantization}

\begin{definition}[Canonical quantization]
	In the canonical quantization procedure, the dynamical variables are promoted to operators satisfying the equal-time canonical commutation relations
	\begin{align}
		\comm{\hat\phi(t,\vb{x})}{\hat\pi(t,\vb{y})}
		&=
		i\delta^{(3)}(\vb{x}-\vb{y})
		\\
		\comm{\hat\phi(t,\vb{x})}{\hat\phi(t,\vb{y})}
		&=
		\comm{\hat\pi(t,\vb{x})}{\hat\pi(t,\vb{y})}
		=
		0
		\label{eq:qkg_comm_pm}
		.
	\end{align}
\end{definition}
\begin{corollary}[Klein-Gordon field operators]
	The Klein-Gordon field operators are
	\begin{align}
		\hat\phi(x^\mu)
		&=
		\int\frac{\dd[3]{p}}{(2\pi)^3}
		\frac{1}{\sqrt{2\omega(\vb{p})}}
		\left\{
			\hat{a}(\vb{p})
			e^{-ip_\mu x^\mu}
			+
			\hat{a}^\dagger(\vb{p})
			e^{+ip_\mu x^\mu}
		\right\}_{p_0=\omega(\vb{p})}
		\label{eq:qkg_pos}
		\\
		\hat\pi(x^\mu)
		&=
		\int\frac{\dd[3]{p}}{(2\pi)^3}
		\left(-i\sqrt{\frac{\omega(\vb{p})}{2}}\right)
		\left\{
			\hat{a}(\vb{p})
			e^{-ip_\mu x^\mu}
			-
			\hat{a}^\dagger(\vb{p})
			e^{+ip_\mu x^\mu}
		\right\}_{p_0=\omega(\vb{p})}
		\label{eq:qkg_mom}
	\end{align}
	where $\hat\phi(x^\mu)$ and $\hat\pi(x^\mu)$ satisfy the equal-time canonical commutation relations.
\end{corollary}
\begin{restatable}{theorem}{qkgcommac}\label{thm:qkg_comm_ac}
	The operators $\hat{a}(\vb{p}),\hat{a}^\dagger(\vb{p})$ obey the commutation relations
	\begin{align}
		\comm{\hat{a}(\vb{p})}{\hat{a}^\dagger(\vb{q})}
		&=
		(2\pi)^3
		\delta^{(3)}(\vb{p}-\vb{q})
		\\
		\comm{\hat{a}^\dagger(\vb{p})}{\hat{a}^\dagger(\vb{q})}
		&=
		\comm{\hat{a}(\vb{p})}{\hat{a}(\vb{q})}
		=
		0
		\label{eq:kg_comm_ac}
		.
	\end{align}	
\end{restatable}
\begin{definition}[Energy and momentum operator]\label{def:qkg_energy_momentum}
	The Klein-Gordon's total energy and total momentum operators are
	\begin{align}
		\hat{H}
		=
		\int\frac{\dd[3]{p}}{(2\pi)^3}
		\omega(\vb{p})\hat{a}^\dagger(\vb{p})\hat{a}(\vb{p})
		&&
		\hat{\vb{P}}
		=
		\int\frac{\dd[3]{p}}{(2\pi)^3}
		\vb{p}\hat{a}^\dagger(\vb{p})\hat{a}(\vb{p})
		\label{eq:qkg_energy_momentum}
		.
	\end{align}
\end{definition}
\begin{definition}\label{def:vacuum_state}
	The vacuum state $\ket{0}$ is the unique eigenstate of the total energy and total momentum operators with eigenvalue zero, i.e.,
	\begin{align}
		\hat{H}
		\ket{0}
		=
		0
		&&
		\hat{\vb{P}}
		\ket{0}
		=
		0
		.
	\end{align}
\end{definition}
\begin{corollary}\label{thm:vacuum_state_ac}
	The definition of the vacuum state and the energy operator implies
	\begin{align}
		\hat{a}(\vb{p})
		\ket{0}
		=
		0
		&&
		\bra{0}
		\hat{a}^\dagger(\vb{p})
		=
		0
		.
	\end{align}
\end{corollary}
\begin{corollary}
	The commutators of the operators $\hat{a}(\vb{p}),\hat{a}^\dagger(\vb{p})$ with the Hamiltonian $\hat{H}$ yield
	\begin{align}
		\comm{\hat{H}}{\hat{a}^\dagger(\vb{p})}
		=
		\omega(\vb{p})
		\hat{a}^\dagger(\vb{p})
		&&
		\comm{\hat{H}}{\hat{a}(\vb{p})}
		=
		-
		\omega(\vb{p})
		\hat{a}(\vb{p})
		.
	\end{align}
\end{corollary}
\begin{definition}
	The number density and total number operators are
	\begin{align}
		\hat{n}(\vb{p})
		=
		\hat{a}^\dagger(\vb{p})
		\hat{a}(\vb{p})
		&&
		\hat{N}
		=
		\int\frac{\dd[3]{p}}{(2\pi)^3}
		\hat{n}(\vb{p})
		=
		\int\frac{\dd[3]{p}}{(2\pi)^3}
		\hat{a}^\dagger(\vb{p})
		\hat{a}(\vb{p})
		.
	\end{align}
\end{definition}
\begin{corollary}
	The total momentum and number observables are conserved
	\begin{equation}
		\comm{\hat{H}}{\hat{\vb{P}}}
		=
		0
		=
		\comm{\hat{H}}{\hat{N}}
		.
	\end{equation}
\end{corollary}

\begin{definition}[Positive and negative frequency Klein-Gordon operator]
	The Klein-Gordon operators can be decomposed into the sum
	\begin{equation}
		\hat\phi(x^\mu)
		=
		\hat\phi^+(x^\mu)
		+
		\hat\phi^-(x^\mu)
	\end{equation}
	wherein the positive and negative frequency Klein-Gordon operators are
	\begin{equation}
		\begin{split}
			\hat\phi^+(x^\mu)
			&=
			\int\frac{\dd[3]{p}}{(2\pi)^3\sqrt{2\omega(\vb{p})}}
			\eval{
				e^{-ip_\mu x^\mu}
				\hat{a}(\vb{p})
			}_{p_0=\omega(\vb{p})}
			\\
			\hat\phi^-(x^\mu)
			&=
			\int\frac{\dd[3]{p}}{(2\pi)^3\sqrt{2\omega(\vb{p})}}
			\eval{
				e^{+ip_\mu x^\mu}
				\hat{a}^\dagger(\vb{p})
			}_{p_0=\omega(\vb{p})}
		\end{split}
		\label{eq:qkg_positive_negative_frequency}
		.
	\end{equation}
	The positive and negative frequency Klein-Gordon operators are related by the hermitian conjugate
	\begin{equation}
		\hat\phi^-(x^\mu)
		=
		\hat\phi^+(x^\mu)^\dagger
		.
	\end{equation}
\end{definition}
\begin{corollary}\label{thm:vacuum_state_pn}
	The positive and negative frequency Klein-Gordon operators satisfy
	\begin{align}
		\hat\phi^+(x^\mu)
		\ket{0}
		=
		0
		&&
		\bra{0}
		\hat\phi^-(x^\mu)
		=
		0
		\label{eq:vacuum_state_pn}
		.
	\end{align}
\end{corollary}
\begin{definition}[Klein-Gordon Propagator]
	The Klein-Gordon propagator is
	\begin{equation}
		D(x^\mu-y^\mu)
		=
		\int\frac{\dd[3]{p}}{(2\pi)^32\omega(\vb{p})}
		\eval{
			e^{-ip_\mu (x^\mu-y^\mu)}
		}_{p_0=\omega(\vb{p})}
		.
	\end{equation}
\end{definition}
\begin{restatable}{lemma}{qkgcommpn}\label{thm:qkg_comm_pn}
	The positive and negative frequency Klein-Gordon operators satisfy the commutation relations
	\begin{align}
		\comm{\hat\phi^+(x^\mu)}{\hat\phi^-(y^\mu)}
		&=
		D(x^\mu-y^\mu)
		\\
		\comm{\hat\phi^+(x^\mu)}{\hat\phi^+(y^\mu)}
		&=
		\comm{\hat\phi^-(x^\mu)}{\hat\phi^-(y^\mu)}
		=
		0
	\end{align}
\end{restatable}
\begin{restatable}{lemma}{qkgpropcorr}\label{thm:qkg_prop_corr}
	The propagator is equal to the expectation value
	\begin{equation}
		D(x^\mu-y^\mu)
		=
		\expval{\hat\phi(x^\mu)\hat\phi(y^\mu)}{0}
		.
	\end{equation}
\end{restatable}
%		\section{Connection to Klein-Gordon states}

\begin{corollary}\label{thm:qmw_qkg}
	Let $\vb{f}$ be a smearing vector function, then the smeared positive frequency transverse Maxwell operator
	\begin{equation}
		\hat{\vb{A}}_\perp^+[\vb{f}]
		=
		\sum_{\lambda=1,2}
		\int\frac{\dd[3]{p}}{(2\pi)^3\sqrt{2\omega(\vb{p})}}
		f_\lambda\left(\omega(\vb{p}),\vb{p}\right)^*
		\hat{a}_\lambda^\dagger(\vb{p})
		=
		\hat\phi_1^+[f_1]
		+
		\hat\phi_2^+[f_2]
	\end{equation}
	 reduces to the sum of two independent smeared positive frequency Klein-Gordon operator.
\end{corollary}

\begin{lemma}{qmwqkgnumberstate}\label{thm:qmw_qkg_number_state}
	The transverse Maxwell fields' $n$-particle number state with smearing vector function $\vb{f}$ is
	\begin{equation}
		\label{eq:qmw_number_state}
		\ket{n_{\vb{f}}}
		=
		\frac{1}{\sqrt{n!}}
		\hat{\vb{A}}_\perp^+[\vb{f}]^n
		\ket{0}
	\end{equation}
	where the smearing vector function is required to satisfy~\cite[p.~175]{Itzykson2012}
	\begin{equation}
		\int\frac{\dd[3]{p}}{(2\pi)^32\omega(\vb{p})}
		\norm{\vb{f}\left(\omega(\vb{p}),\vb{p}\right)}^2
		=
		\sum_{\lambda=1,2}
		\int\frac{\dd[3]{p}}{(2\pi)^32\omega(\vb{p})}
		\abs{f_\lambda\left(\omega(\vb{p}),\vb{p}\right)}^2
		=
		1
		.
	\end{equation}
\end{lemma}
\begin{proof}
	The smeared positive Klein-Gordon operators $\hat\phi^+_1,\hat\phi^+_2$ commute and we can use the binomial theorem to write
	\begin{equation*}
		\begin{split}
			\ket{n_{\vb{f}}}
			&=
			\frac{1}{\sqrt{n!}}
			\hat{\vb{A}}_\perp^+[\vb{f}]^n
			\ket{0}
			\\
			&=
			\frac{1}{\sqrt{n!}}
			\left(
				\hat\phi_1^+[f_1]
				+
				\hat\phi_2^+[f_2]
			\right)^n
			\ket{0}
			\\
			&=
			\frac{1}{\sqrt{n!}}
			\sum^n_{m=0}
			\binom{n}{m}
			\hat\phi_1^+[f_1]^m
			\hat\phi_2^+[f_2]^{n-m}
			\ket{0}
			\\
			&=
			\sum^n_{m=0}
			\binom{n}{m}^{1/2}
			\ket{m_{f_1},n-m_{f_2}}
			.
		\end{split}
	\end{equation*}
\end{proof}

\begin{corollary}
	Using \Cref{thm:qmw_qkg}, we can write the transverse Maxwell $n$-particle state as the tensor product of $n$-particle Klein-Gordon states
	\begin{equation}
		\ket{n_{\vb{f}}}
		=
		\sum^n_{m=0}
		\binom{n}{m}^{1/2}
		\ket{m_{f_1},n-m_{f_2}}
		.
	\end{equation}
\end{corollary}

\begin{lemma}\label{thm:qmw_qkg_number_state_inner_product}
	Let $\ket{n_{\vb{f}}}$ be an $n$-particle number state with spectrum $\vb{f}$, and $\ket{m_{\vb{g}}}$ an $m$-particle number state with spectrum $\vb{g}$, then their inner product is
	\begin{equation}
		\braket{n_{\vb{f}}}{m_{\vb{g}}}
		=
		\delta_{n,m}
		\left[
			\braket{1_{f_1}}{1_{g_1}}
			+
			\braket{1_{f_2}}{1_{g_2}}
		\right]^n
		.
	\end{equation}
\end{lemma}
\begin{proof}
	Inserting the transverse Maxwell number state in terms of two independent Klein-Gordon number states \Cref{thm:qmw_qkg_number_state}
	\begin{equation*}
		\begin{split}
			\braket{n_{\vb{f}}}{m_{\vb{g}}}
			&=
			\sum^n_{k=0}
			\sum^m_{l=0}
			\binom{n}{k}^{1/2}
			\binom{m}{l}^{1/2}
			\braket{k_{f_1},n-k_{f_2}}{l_{g_1},m-l_{g_2}}
			\\
			&=
			\sum^n_{k=0}
			\sum^m_{l=0}
			\binom{n}{k}^{1/2}
			\binom{m}{l}^{1/2}
			\braket{k_{f_1}}{l_{g_1}}
			\braket{n-k_{f_2}}{m-l_{g_2}}
			\\
			&=
			\sum^n_{k=0}
			\sum^m_{l=0}
			\binom{n}{k}^{1/2}
			\binom{m}{l}^{1/2}
			\delta_{k,l}
			\braket{1_{f_1}}{1_{g_1}}^k
			\delta_{n-k,m-l}
			\braket{1_{f_2}}{1_{g_2}}^{n-k}
			\\
			&=
			\sum^n_{k=0}
			\binom{n}{k}^{1/2}
			\binom{m}{k}^{1/2}
			\braket{1_{f_1}}{1_{g_1}}^k
			\delta_{n-k,m-k}
			\braket{1_{f_2}}{1_{g_2}}^{n-k}
			\\
			&=
			\delta_{n,m}
			\sum^n_{k=0}
			\binom{n}{k}
			\braket{1_{f_1}}{1_{g_1}}^k
			\braket{1_{f_2}}{1_{g_2}}^{n-k}
			\\
			&=
			\delta_{n,m}
			\left(
				\braket{1_{f_1}}{1_{g_1}}
				+
				\braket{1_{f_2}}{1_{g_2}}
			\right)^n
			.
		\end{split}
	\end{equation*}
\end{proof}

\begin{definition}[Coherent state]
	The coherent state with spectrum $\vb{\alpha}(p_0,\vb{p})$ is
	\begin{equation}
		\label{eq:qmw_coherent_state}
		\begin{split}
			\ket{\vb{\alpha}}
			=
			\exp\left\{
				-
				\frac{1}{2}
				\comm{\hat{\vb{A}}_\perp^-[\vb{\alpha}]}{\hat{\vb{A}}_\perp^+[\vb{\alpha}]}
			\right\}
			\exp\left\{
				\hat{\vb{A}}_\perp^+[\vb{\alpha}]
			\right\}
			\ket{0}
			.
		\end{split}
	\end{equation}
\end{definition}
\begin{lemma}\label{thm:qmw_qkg_coherent_state}
	The transverse Maxwell' coherent state can be written as a tensor product of Klein-Gordon' coherent state
	\begin{equation}
		\ket{\vb{\alpha}}
		=
		\hat{D}_1[\alpha_1]
		\hat{D}_2[\alpha_2]
		\ket{0}
		=
		\ket{\alpha_1,\alpha_2}
		.
	\end{equation}
\end{lemma}
\begin{proof}
	\begin{equation*}
		\begin{split}
			\ket{\vb{\alpha}}
			&=
			\exp\left\{
				-
				\frac{1}{2}
				\comm{\hat{\vb{A}}_\perp^-[\vb{\alpha}]}{\hat{\vb{A}}_\perp^+[\vb{\alpha}]}
			\right\}
			\exp\left\{
				\hat{\vb{A}}_\perp^+[\vb{\alpha}]
			\right\}
			\ket{0}
			\\
			&=
			\exp\left\{
				-
				\frac{1}{2}
				\comm{
					\hat\phi_1^-[\alpha_1]
					+
					\hat\phi_2^-[\alpha_2]
				}{
					\hat\phi_1^+[\alpha_1]
					+
					\hat\phi_2^+[\alpha_2]
				}
			\right\}
			\exp\left\{
				\hat\phi_1^+[\alpha_1]
				+
				\hat\phi_2^+[\alpha_2]
			\right\}
			\ket{0}
			\\
			&=
			\exp\left\{
				-
				\frac{1}{2}
				\comm{\hat\phi_1^-[\alpha_1]}{\hat\phi_1^+[\alpha_1]}
				-
				\frac{1}{2}
				\comm{\hat\phi_2^-[\alpha_2]}{\hat\phi_2^+[\alpha_2]}
			\right\}
			\exp\left\{
				\hat\phi_1^+[\alpha_1]
				+
				\hat\phi_2^+[\alpha_2]
			\right\}
			\ket{0}
			\\
			&=
			\exp\left\{
				\frac{1}{2}
				\comm{\hat\phi_2^-[\alpha_2]}{\hat\phi_2^+[\alpha_2]}
			\right\}
			\exp\left\{
				\hat\phi_2^+[\alpha_2]
			\right\}
			\exp\left\{
				-
				\frac{1}{2}
				\comm{\hat\phi_1^-[\alpha_1]}{\hat\phi_1^+[\alpha_1]}
			\right\}
			\exp\left\{
				\hat\phi_1^+[\alpha_1]
			\right\}
			\ket{0}
			\\
			&=
			\hat{D}_1[\alpha_1]
			\hat{D}_2[\alpha_2]
			\ket{0}
			\\
			&=
			\ket{\alpha_1,\alpha_2}
			.
		\end{split}
	\end{equation*}
\end{proof}

%		\section{Quantum states}

\begin{corollary}\label{thm:qmw_qkg}
	Let $\vb{f}$ be a smearing vector function, then the smeared positive frequency transverse Maxwell operator
	\begin{equation}
		\hat{\vb{A}}_\perp^+[\vb{f}]
		=
		\sum_{\lambda=1,2}
		\int\frac{\dd[3]{p}}{(2\pi)^3\sqrt{2\omega(\vb{p})}}
		f_\lambda\left(\omega(\vb{p}),\vb{p}\right)^*
		\hat{a}_\lambda^\dagger(\vb{p})
		=
		\hat\phi_1^+[f_1]
		+
		\hat\phi_2^+[f_2]
	\end{equation}
	 reduces to the sum of two independent smeared positive frequency Klein-Gordon operator.
\end{corollary}
\begin{restatable}{lemma}{qmwqkgnumberstate}\label{thm:qmw_qkg_number_state}
	The transverse Maxwell fields' $n$-particle number state with smearing vector function $\vb{f}$ is
	\begin{equation}
		\label{eq:qmw_number_state}
		\ket{n_{\vb{f}}}
		=
		\frac{1}{\sqrt{n!}}
		\hat{\vb{A}}_\perp^+[\vb{f}]^n
		\ket{0}
	\end{equation}
	where the smearing vector function is required to satisfy~\cite[p.~175]{Itzykson2012}
	\begin{equation}
		\int\frac{\dd[3]{p}}{(2\pi)^32\omega(\vb{p})}
		\norm{\vb{f}\left(\omega(\vb{p}),\vb{p}\right)}^2
		=
		\sum_{\lambda=1,2}
		\int\frac{\dd[3]{p}}{(2\pi)^32\omega(\vb{p})}
		\abs{f_\lambda\left(\omega(\vb{p}),\vb{p}\right)}^2
		=
		1
		.
	\end{equation}
\end{restatable}
\begin{corollary}
	Using \Cref{thm:qmw_qkg}, we can write the transverse Maxwell $n$-particle state as the tensor product of $n$-particle Klein-Gordon states
	\begin{equation}
		\ket{n_{\vb{f}}}
		=
		\sum^n_{m=0}
		\binom{n}{m}^{1/2}
		\ket{m_{f_1},n-m_{f_2}}
		.
	\end{equation}
\end{corollary}
\begin{restatable}{lemma}{qmwqkgnumberstateinnerproduct}\label{thm:qmw_qkg_number_state_inner_product}
	Let $\ket{n_{\vb{f}}}$ be an $n$-particle number state with spectrum $\vb{f}$, and $\ket{m_{\vb{g}}}$ an $m$-particle number state with spectrum $\vb{g}$, then their inner product is
	\begin{equation}
		\braket{n_{\vb{f}}}{m_{\vb{g}}}
		=
		\delta_{n,m}
		\left[
			\braket{1_{f_1}}{1_{g_1}}
			+
			\braket{1_{f_2}}{1_{g_2}}
		\right]^n
		.
	\end{equation}
\end{restatable}
\begin{example}
	Let $\vb{f}$ be a smearing vector function with transverse components
	\begin{align*}
		f_1(p_0,\vb{p})
		=
		\sqrt{2\omega(\vb{p}_0)}
		\left(\frac{2\pi}{\sigma^2}\right)^{\frac{3}{4}}
		\exp\left\{
			-
			\frac{(\vb{p}-\vb{p}_0)^2}{4\sigma^2}
		\right\}
		&&
		f_2(p_0,\vb{p})
		=
		0
		.
	\end{align*}
	The smearing vector function is normalized
	\begin{equation*}
		\begin{split}
			\int\frac{\dd[3]{p}}{(2\pi)^32\omega(\vb{p})}
			\norm{\vb{f}\left(\omega(\vb{p}),\vb{p}\right)}^2
			&=
			2\omega(\vb{p}_0)
			\left(\frac{2\pi}{\sigma^2}\right)^{\frac{3}{2}}
			\int\frac{\dd[3]{p}}{(2\pi)^32\omega(\vb{p})}
			\exp\left\{
				-
				\frac{(\vb{p}-\vb{p}_0)^2}{2\sigma^2}
			\right\}
			\\
			&=
			\left(\frac{2\pi}{\sigma^2}\right)^{\frac{3}{2}}
			\left(
				\int\frac{\dd[3]{p}}{2\pi}
				\exp\left\{
					-
					\frac{(p-p_0)^2}{2\sigma^2}
				\right\}
			\right)^3
			\\
			&=
			\left(\frac{2\pi}{\sigma^2}\right)^{\frac{3}{2}}
			\left(
				\frac{1}{2\pi}
				\sqrt{2\pi\sigma^2}
			\right)^3
			\\
			&=
			\left(\frac{2\pi}{\sigma^2}\right)^{\frac{3}{2}}
			\left(
				\frac{\sigma^2}{2\pi}
			\right)^{\frac{3}{2}}
			=
			1
		\end{split}
	\end{equation*}
	where we used the mean value theorem in the second line.
	The coordinate representation of the smearing function is
	\begin{equation*}
		\begin{split}
			f_1(t,\vb{x})
			&=
			\int\frac{\dd[4]{p}}{(2\pi)^4}
			f(p_0,\vb{p})
			e^{+ip_\mu x^\mu}
			\\
			&=
			\sqrt{2\omega(\vb{p}_0)}
			\left(\frac{2\pi}{\sigma^2}\right)^{\frac{3}{4}}
			\int\frac{\dd[4]{p}}{(2\pi)^4}
			\exp\left\{
				-
				\frac{(\vb{p}-\vb{p}_0)^2}{4\sigma^2}
			\right\}
			e^{+i(p_0t-\vb{p}\vdot\vb{x})}
			\\
			&=
			\sqrt{2\omega(\vb{p}_0)}
			\left(\frac{2\pi}{\sigma^2}\right)^{\frac{3}{4}}
			\int\frac{\dd{p_0}}{2\pi}
			e^{-ip_0t}
			\int\frac{\dd[3]{p}}{(2\pi)^3}
			\exp\left\{
				-
				\frac{(\vb{p}-\vb{p}_0)^2}{4\sigma^2}
				-
				i\vb{p}\vdot\vb{x}
			\right\}
			\\
			&=
			\sqrt{2\omega(\vb{p}_0)}
			\left(\frac{2\pi}{\sigma^2}\right)^{\frac{3}{4}}
			\delta(t)
			e^{-i\vb{p}_0\vdot\vb{x}}
			\left(
				\int\frac{\dd{q}}{2\pi}
				\exp\left\{
					-
					\frac{q^2+2q(2i\sigma^2x)+(2i\sigma^2x)^2-(2i\sigma^2x)^2}{4\sigma^2}
				\right\}
			\right)^3
			\\
			&=
			\sqrt{2\omega(\vb{p}_0)}
			\left(\frac{2\pi}{\sigma^2}\right)^{\frac{3}{4}}
			\delta(t)
			e^{-i\vb{p}_0\vdot\vb{x}}
			e^{-(\sigma\vb{x})^2}
			\left(
				\int\frac{\dd{q}}{2\pi}
				\exp\left\{
					-
					\frac{(q+i\sigma^2x)^2}{4\sigma^2}
				\right\}
			\right)^3
			\\
			&=
			\sqrt{2\omega(\vb{p}_0)}
			\left(\frac{2\pi}{\sigma^2}\right)^{\frac{3}{4}}
			\delta(t)
			e^{-i\vb{p}_0\vdot\vb{x}}
			e^{-(\sigma\vb{x})^2}
			\left(
				\frac{1}{2\pi}
				\sqrt{4\pi\sigma^2}
			\right)^3
			\\
			&=
			\sqrt{2\omega(\vb{p}_0)}
			\left(\frac{2\pi}{\sigma^2}\right)^{\frac{3}{4}}
			\delta(t)
			e^{-i\vb{p}_0\vdot\vb{x}}
			e^{-(\sigma\vb{x})^2}
			\left(
				\frac{\sigma^2}{\pi}
			\right)^\frac{3}{2}
			\\
			&=
			\sqrt{2\omega(\vb{p}_0)}
			\left(\frac{2\sigma^2}{\pi}\right)^{\frac{3}{4}}
			\delta(t)
			e^{-i\vb{p}_0\vdot\vb{x}}
			e^{-(\sigma\vb{x})^2}
			,
		\end{split}
	\end{equation*}
	hence, at time $t=0$, the wave packet is localized at $\vb{x}=0$ with spatial spread $\sigma$.
	Thereby the wave packet is perfectly localized in time, equivalent with knowing the momentum distribution of the wave packet at $t=0$ or $f_1(0,\vb{x})$.
	However, we know that there must be a time-frequency uncertainty.
	We find the time-frequency uncertainty in the coordinate representation of the wave function
	\begin{equation*}
		\begin{split}
			\psi_1(t,\vb{x})
			&=
			\int\frac{\dd[3]{p}}{(2\pi)^32p_0}
			\eval{
				f(p_0,\vb{p})^*
				e^{+ip_\mu x^\mu}
			}_{p_0=\omega(\vb{p})}
			\\
			&=
			2\omega(\vb{p}_0)
			\left(\frac{2\pi}{\sigma^2}\right)^{\frac{3}{2}}
			\int\frac{\dd[3]{p}}{(2\pi)^32\omega(\vb{p})}
			\exp\left\{
				-
				\frac{(\vb{p}-\vb{p}_0)^2}{4\sigma^2}
			\right\}
			e^{+i\omega(\vb{p})t-i\vb{p}\vdot\vb{x}}
			\\
			&=
			\left(\frac{2\pi}{\sigma^2}\right)^{\frac{3}{2}}
			\int\frac{\dd[3]{p}}{(2\pi)^3}
			\exp\left\{
				-
				\frac{(\vb{p}-\vb{p}_0)^2}{4\sigma^2}
				-
				i\vb{p}\vdot\vb{x}
				+
				i\vb{p}^2t
			\right\}
			\\
			&=
			\left(\frac{2\pi}{\sigma^2}\right)^{\frac{3}{2}}
			\int\frac{\dd[3]{q}}{(2\pi)^3}
			\exp\left\{
				-
				\frac{\vb{q}^2}{4\sigma^2}
				-
				i(\vb{q}+\vb{p}_0)\vdot\vb{x}
				+
				i(\vb{q}+\vb{p}_0)^2t
			\right\}
			\\
			&=
			\left(\frac{2\pi}{\sigma^2}\right)^{\frac{3}{2}}
			e^{i\omega(\vb{p}_0)t-i\vb{p}_0\vdot\vb{x}}
			\int\frac{\dd[3]{q}}{(2\pi)^3}
			\exp\left\{
				-
				\frac{\vb{q}^2}{4\sigma^2}
				+
				i\vb{q}^2t
				+
				2i\vb{q}\vdot\vb{p}_0t
				-
				i\vb{q}\vdot\vb{x}
			\right\}
			\\
			&=
			\left(\frac{2\pi}{\sigma^2}\right)^{\frac{3}{2}}
			e^{i\omega(\vb{p}_0)t-i\vb{p}_0\vdot\vb{x}}
			\int\frac{\dd[3]{q}}{(2\pi)^3}
			\exp\left\{
				-
				\frac{
					(1-4i\sigma^2t)\vb{q}^2
					+
					22i(\sigma^2\vb{x}-2\sigma^2\vb{p}_0t)
					\vdot
					\vb{q}
				}{4\sigma^2}
			\right\}
			\\
			&=
			\left(\frac{2\pi}{\sigma^2}\right)^{\frac{3}{2}}
			\frac{e^{i\omega(\vb{p}_0)t-i\vb{p}_0\vdot\vb{x}}}{(1-4i\sigma^2t)^\frac{3}{2}}
			\int\frac{\dd[3]{v}}{(2\pi)^3}
			\exp\left\{
				-
				\frac{
					\vb{v}^2
					+
					2
					(2i\vb{b})
					\vdot
					\vb{v}
				}{4\sigma^2}
			\right\}
			\\
			&=
			\left(\frac{2\pi}{\sigma^2}\right)^{\frac{3}{2}}
			\frac{e^{i\omega(\vb{p}_0)t-i\vb{p}_0\vdot\vb{x}}}{(1-4i\sigma^2t)^\frac{3}{2}}
			\int\frac{\dd[3]{v}}{(2\pi)^3}
			\exp\left\{
				-
				\frac{
					\left(
						\vb{v}
						+
						2i\vb{b}
					\right)^2
					-
					(2i\vb{b})^2
				}{4\sigma^2}
			\right\}
			\\
			&=
			\left(\frac{2\pi}{\sigma^2}\right)^{\frac{3}{2}}
			\frac{e^{i\omega(\vb{p}_0)t-i\vb{p}_0\vdot\vb{x}}}{(1-4i\sigma^2t)^\frac{3}{2}}
			e^{-\vb{b}^2}
			\int\frac{\dd[3]{u}}{(2\pi)^3}
			\exp\left\{
				-
				\frac{\vb{u}^2}{4\sigma^2}
			\right\}
			\\
			&=
			\left(\frac{2\pi}{\sigma^2}\right)^{\frac{3}{2}}
			\frac{e^{i\omega(\vb{p}_0)t-i\vb{p}_0\vdot\vb{x}}}{(1-4i\sigma^2t)^\frac{3}{2}}
			e^{-\vb{b}^2}
			\left(\frac{\sigma^2}{\pi}\right)^\frac{3}{2}
			\\
			&=
			\left(2\right)^{\frac{3}{2}}
			\frac{e^{i\omega(\vb{p}_0)t-i\vb{p}_0\vdot\vb{x}}}{(1-4i\sigma^2t)^\frac{3}{2}}
			\exp\left\{
				-
				\sigma^4
				\frac{(\vb{x}-2\vb{p}_0t)^2}{1-4i\sigma^2 t}
			\right\}
		\end{split}
	\end{equation*}
	where we have
	\begin{equation*}
		\vb{b}
		=
		\sigma^2
		\frac{\vb{x}-2\vb{p}_0t}{\sqrt{1-4i\sigma^2t}}
		.
	\end{equation*}
	Comparing the smearing and wave function at $t=0$, we find
	\begin{align*}
		f_1(0,\vb{x})
		&=
		\sqrt{2\omega(\vb{p}_0)}
		\left(\frac{2\sigma^2}{\pi}\right)^\frac{3}{2}
		e^{-i\vb{p}_0\vdot\vb{x}}
		e^{-(\sigma^2\vb{x})^2}
		\\
		\psi_1(0,\vb{x})
		&=
		2^\frac{3}{4}
		e^{-i\vb{p}_0\vdot\vb{x}}
		e^{-(\sigma^2\vb{x})^2}
	\end{align*}
	and we conclude that
	\begin{equation*}
		f_1(0,\vb{x})
		=
		\sqrt{2\omega(\vb{p}_0)}
		\left(\frac{\sigma^2}{\pi}\right)^\frac{3}{4}
		\psi_1(0,\vb{x})
	\end{equation*}
	or just $f_1(0,\vb{x})\propto\psi_1(0,\vb{x})$ meaning that the smearing function is actually the (re-scaled) wave function at some initial time $t=0$ and the wave function has built-in time uncertainty.
\end{example}

\begin{definition}[Coherent state]
	The coherent state with spectrum $\vb{\alpha}(p_0,\vb{p})$ is
	\begin{equation}
		\label{eq:qmw_coherent_state}
		\ket{\vb{\alpha}}
		=
		\exp\left\{
			-
			\frac{1}{2}
			\comm{\hat{\vb{A}}_\perp^-[\vb{\alpha}]}{\hat{\vb{A}}_\perp^+[\vb{\alpha}]}
		\right\}
		\exp\left\{
			\hat{\vb{A}}_\perp^+[\vb{\alpha}]
		\right\}
		\ket{0}
		.
	\end{equation}
\end{definition}
%		\section{Time-dependent interactions}

\subsection{Time-evolution operator}

Let $\ket{\psi(t_0)}$ be a state at time $t_0$, then the time-evolution relates the state $\ket{\psi(t)}$ at some later time $t>t_0$ to $\ket{\psi(t_0)}$ via
\begin{equation}
	\ket{\psi(t)}
	=
	\hat{U}(t,t_0)
	\ket{\psi(t_0)}
	.
\end{equation}
Inserting $\ket{\psi(t)}$ into the Schrödinger equation leads to
\begin{equation}
	i\dv{t}
	\hat{U}(t,t_0)
	=
	\hat{H}(t)
	\hat{U}(t,t_0)
\end{equation}
which formal solution is the time-ordered exponential, see Ref.~\cite[p.~380]{Bartelmann2018},
\begin{equation}
	\hat{U}(t,t_0)
	=
	T\exp\left\{
		-i
		\int_{t_0}^t\dd{t^\prime}
		\hat{H}(t^\prime)
	\right\}
\end{equation}
where $T$ denotes the time-ordering symbol.
Only for simple time-dependent systems an exact time-evolution operator exists.
In contrast to the Dyson expansion, the Magnus expansion yields a unitary time-evolution operator even for finite order, in particular,
\begin{equation}
	\hat{U}(t,t_0)
	=
	\exp\left\{
		\sum_{n=1}
		\hat{\Omega}^{(n)}(t,t_0)
	\right\}
\end{equation}
where the first two expansion terms are given by
\begin{align}
	\hat{\Omega}^{(1)}(t,t_0)
	&=
	\frac{(-i)}{1!}
	\int_{t_0}^t\dd{t^\prime}
	\hat{H}(t^\prime)
	\\
	\hat{\Omega}^{(2)}(t,t_0)
	&=
	\frac{(-i)^2}{2!}
	\int_{t_0}^t\dd{t^\prime}
	\int_{t_0}^{t^\prime}\dd{t^{\prime\prime}}
	\comm{\hat{H}(t^\prime)}{\hat{H}(t^{\prime\prime})}
\end{align}
and represent time-ordering corrections, see Ref.~\cite{QuesadaMejia2015}.

\subsection{Interaction with a classical current field}

The Schrödinger-picture Hamiltonian describing the interaction of the Maxwell field $\hat{\vb{A}}$ with a classical current $\vb{j}$ is
\begin{equation}
	\hat{H}_\text{int}(t)
	=
	-
	\int_{\mathbb{R}^3}\dd[3]{x}
	\vb{j}(t,\vb{x})
	\vdot
	\hat{\vb{A}}(t,\vb{x})
	.
\end{equation}
Taking the spatial Fourier transform of the current
\begin{equation}
	\vb{j}(t,\vb{x})
	=
	\int_{\mathbb{R}^3}\frac{\dd[3]{q}}{(2\pi)^3}
	\vb{j}(t,\vb{q})
	e^{+i\vb{q}\vdot\vb{x}}
\end{equation}
and inserting the mode expansion, we find the interaction Hamiltonian to be
\begin{equation}
	\hat{H}_\text{int}(t)
	=
	-
	\sum_{\lambda=1,2}
	\int_{\mathbb{R}^3}\frac{\dd[3]{p}}{(2\pi)^3\sqrt{2\omega(\vb{p})}}
	\left\{
		j_\lambda(t,\vb{p})
		\hat{a}_\lambda(\vb{p})
		e^{-i\omega(\vb{p})t}
		+
		\text{h.c.}
	\right\}
\end{equation}
where we have the transverse current
\begin{equation}
	j_\lambda(q_0,\vb{p})
	=
	\vb{j}(q_0,\vb{p})
	\vdot
	\vu{e}_\lambda(\vb{p})
	.
\end{equation}
The first term in the Magnus expansion turns out to be
\begin{equation}
	\hat{\Omega}^{(1)}(t,t_0)
	=
	i
	\sum_{\lambda=1,2}
	\int_{\mathbb{R}^3}\frac{\dd[3]{p}}{(2\pi)^3\sqrt{2\omega(\vb{p})}}
	\left\{
		J_\lambda(t,t_0;\vb{p})
		\hat{a}_\lambda(\vb{p})
		+
		\text{h.c.}
	\right\}
\end{equation}
where we defined
\begin{equation}
	J_\lambda(t,t_0;\vb{p})
	=
	\int_{t_0}^t\dd{t^\prime}
	j_\lambda(t^\prime,\vb{p})
	e^{-i\omega(\vb{p})t^\prime}
	.
\end{equation}
For the second term in the Magnus expansion, we first evaluate the commutator
\begin{equation}
	\comm{\hat{H}(t^\prime)}{\hat{H}(t^{\prime\prime})}
	=
	i\sum_{\lambda=1,2}
	\int_{\mathbb{R}^3}\frac{\dd[3]{p}}{(2\pi)^3\omega(\vb{p})}
	\Im\left\{
		j_\lambda(t^\prime,\vb{p})
		j_\lambda(t^{\prime\prime},\vb{p})^*
		e^{-i\omega(\vb{p})(t^\prime-t^{\prime\prime})}
	\right\}
\end{equation}
and notice that it is complex valued, hence, all higher commutators vanish and the Magnus expansion with the first two terms is exact.
In summary, the second term of the Magnus expansion turns out to be
\begin{equation}
	\hat{\Omega}^{(2)}(t,t_0)
	=
	i\sum_{\lambda=1,2}
	\int_{\mathbb{R}^3}\frac{\dd[3]{p}}{(2\pi)^3\omega(\vb{p})}
	\Im\left\{
		J_\lambda(t_0,t^\prime;\vb{p})
		J_\lambda(t_0,t^{\prime\prime};\vb{p})^*
	\right\}
\end{equation}
The second term contributes a phase to the time-evolution operator.
As long as we consider a single current source, no interference of phases can occur and we can ignore the phase factor.
The exact time-evolution operator of the Maxwell field interacting with a classical source current therefore is
\begin{equation}
	\hat{U}(t,t_0)
	=
	\exp\left\{
		i\sum_{\lambda=1,2}
		\int_{\mathbb{R}^3}\frac{\dd[3]{p}}{(2\pi)^3\sqrt{2\omega(\vb{p})}}
		\left\{
			J_\lambda(t,t_0;\vb{p})
			\hat{a}_\lambda(\vb{p})
			+
			\text{h.c.}
		\right\}
	\right\}
\end{equation}
which equals the displacement operator for a time-dependent spectrum $\hat{D}[\alpha(t,t_0)]$.

\subsection{Interaction with a charged particle in a potential}

Based on Refs.~\cite[p.~687]{Mandel1995},~\cite[p.~128]{Cohen1992}

The Hamiltonian of the particle with charge $q$ and mass $m$
\begin{equation}
	\hat{H}_q
	=
	\frac{1}{2m}
	\hat{\vb{p}}^2
	+
	qV(\hat{\vb{x}})
\end{equation}
wherein $V$ is a classical external potential and the position and momentum operator satisfy the canonical commutation relations
\begin{align}
	\comm{\hat{x}_i}{\hat{p}_j}
	&=
	i\delta_{ij}
	&
	\comm{\hat{x}_i}{\hat{x}_j}
	&=
	0
	=
	\comm{\hat{p}_i}{\hat{p}_j}
	.
\end{align}
Interaction of the charged particle with an electromagnetic potential $\hat{\vb{A}}$ due to minimal coupling os obtained by the replacement~\cite{Itzykson2012}
\begin{equation}
	\hat{\vb{p}}
	\to
	\hat{\vb{p}}
	-
	q\vb{\hat{A}}
\end{equation}
and expanding the kinetic term of the particle's Hamiltonian
\begin{equation}
	\begin{split}
		\frac{1}{2m}
		\left(\hat{\vb{p}}-q\hat{\vb{A}}\right)^2
		&=
		\frac{1}{2m}
		\hat{\vb{p}}^2
		-
		\frac{q}{2m}
		\left(
			\hat{\vb{p}}
			\vdot
			\hat{\vb{A}}
			+
			\hat{\vb{A}}
			\vdot
			\hat{\vb{p}}
		\right)
		+
		\frac{q^2}{2m}
		\hat{\vb{A}}^2
		\\
		&=
		\frac{1}{2m}
		\hat{\vb{p}}^2
		-
		\frac{q}{m}
		\hat{\vb{p}}
		\vdot
		\hat{\vb{A}}
		+
		\frac{q^2}{2m}
		\hat{\vb{A}}^2
	\end{split}
\end{equation}
where we used that the particle's momentum and the Maxwell field operator commute in the Coulomb gauge, see Ref.~\cite[p.~687]{Mandel1995}.
The interaction term quadratic in the Maxwell field becomes relevant for very high intensities, see Ref.~\cite[p.~198]{Cohen1989} and Ref.~\cite[p.~689]{Mandel1995} for a detailed discussion, which we are not relevant for out use case.
We conclude the interaction Hamiltonian to be
\begin{equation}
	\hat{H}_\text{int}(t,\vb{x})
	=
	-\frac{q}{m}
	\hat{\vb{p}}(t)
	\vdot
	\hat{\vb{A}}(t,\vb{x})
	.
\end{equation}
In the dipole approximation, we consider the Maxwell field constant over the support of the particle wave function~\cite[p.~688]{Mandel1995} and thus
\begin{equation}
	\hat{\vb{A}}(t,\vb{x})
	\ket{\Psi}
	\approx
	\hat{\vb{A}}(t,\vb{x}_0)
	\ket{\Psi}
\end{equation}
with $\vb{x}_0$ being the particle's \gls{com}.

For an alternative approach on how the minimal coupling leads to the dipole interaction, see Ref.~\cite[p.~635]{Cohen1992}.

\subsection{Photodetection}

\subsubsection{Gerry and Knight}

Based on Ref.~\cite[p.~120]{Gerry2005}.

We have the dipole interaction
\begin{equation}
	\hat{H}_\text{int}(t,\vb{x})
	=
	-
	\hat{\vb{d}}
	\vdot
	\hat{\vb{E}}(\vb{x},t)
	=
	-
	\hat{\vb{d}}
	\vdot
	\left(
		\hat{\vb{E}}^{(+)}(\vb{x},t)
		+
		\hat{\vb{E}}^{(-)}(\vb{x},t)
	\right)
\end{equation}
wherein
\begin{equation}
	\hat{\vb{E}}^{(+)}(\vb{x},t)
	=
	i\sum_\lambda
	\int\frac{\dd[3]{p}}{(2\pi)^3\sqrt{2\omega}}
	\vu{e}_\lambda(\vb{p})
	\hat{a}_\lambda(\vb{p})
	e^{+i\vb{p}\vdot\vb{x}}
	\approx
	i\sum_\lambda
	\int\frac{\dd[3]{p}}{(2\pi)^3\sqrt{2\omega}}
	\vu{e}_\lambda(\vb{p})
	\hat{a}_\lambda(\vb{p})
\end{equation}
where the dipole approximation $\norm{\vb{p}\vdot\vb{x}}\ll1$ has been implemented using $e^{i\vb{p}\vdot\vb{x}}\approx1$.

The matrix element for the photoemission is
\begin{equation}
	\bra{e,f}\hat{H}_\text{int}\ket{g,i}
	=
	-
	\bra{e}\hat{\vb{d}}\ket{g}
	\bra{f}\hat{\vb{E}}^{(+)}(\vb{x},t)\ket{i}
	.
\end{equation}
We do not care about the final states of the field, hence we marginalize the probability of an absorption
\begin{equation}
	\sum_f\abs{\bra{f}\hat{\vb{E}}^{(+)}(\vb{x},t)\ket{i}}^2
	=
	\sum_f
	\bra{i}
	\hat{\vb{E}}^{(-)}
	\ketbra{f}
	\hat{\vb{E}}^{(+)}
	\ket{i}
	=
	\expval{\hat{\vb{E}}^{(-)}\vdot\hat{\vb{E}}^{(+)}}{i}
	.
\end{equation}
For a more general initial radiation state $\hat\rho_i=\sum_jp_j\ketbra{j}$, we can rewrite the previous equation as
\begin{equation}
	\tr\left\{
		\hat\rho_f
		\hat{\vb{E}}^{(-)}
		\vdot
		\hat{\vb{E}}^{(+)}
	\right\}
	.
\end{equation}

\subsubsection{Cohen-Tannoudji}

Based on Ref.~\cite[p.~128]{Cohen1992}.

\begin{equation}
	\hat{H}
	=
	\hat{H}_a
	+
	\hat{H}_f
	+
	\hat{H}_i
\end{equation}
wherein \textcolor{red}{need to show this!}
\begin{equation}
	\hat{H}_i
	=
	-
	\hat{\vb{d}}
	\vdot
	\hat{\vb{E}}
\end{equation}
Assuming the dipole moment and the electric field to be parallel, we find in the interaction picture
\begin{align}
	\hat{\vb{d}}(t)
	&=
	e^{+i\hat{H}_at}
	\hat{\vb{d}}(0)
	e^{-i\hat{H}_at}
	\\
	\hat{\vb{E}}(t)
	&=
	e^{+i\hat{H}_ft}
	\hat{\vb{E}}(0)
	e^{-i\hat{H}_ft}
	.
\end{align}
We have a photoelectron emission in time interval $\Delta t$ whenever an electron is excited from the ground state $\ket{g}$, i.e.,
\begin{equation}
	p_{\Delta t}
	=
	\sum_{f,e}
	\abs{\bra{f,e}\hat{U}(\Delta t)\ket{i,g}}^2
\end{equation}
where we marginalized the final states as we do not care about them, and the interaction time-evolution operator is
\begin{equation}
	\hat{U}(\Delta t)
	=
	\mathcal{T}_+
	\exp\left\{
		-i\int_0^{\Delta t}\dd{t^\prime}
		\hat{H}_\text{int}(t^\prime)
	\right\}
	.
\end{equation}
Expanding the time-evolution operator, we find
\begin{equation}
	p_{\Delta t}
	=
	\sum_{f\neq i,e\neq g}
	\int_0^{\Delta t}\dd{t^\prime}
	\int_0^{\Delta t}\dd{t^{\prime\prime}}
	\abs{\bra{f,e}\hat{H}_\text{int}(t^\prime)\hat{H}_\text{int}(t^{\prime\prime})\ket{i,g}}^2
\end{equation}
where the first term (zeroth order in $\hat{H}_\text{int}$ vanishes because of the orthogonality of the final and initial states, and the second term vanishes because the dipole moment operator is asymmetric~\cite[p.~131]{Cohen1992}.
Writing out the dipole and electric field operators, we find
\begin{equation}
	\begin{split}
		p_{\Delta t}
		&=
		\sum_{f\neq i,e\neq g}
		\int_0^{\Delta t}\dd{t^\prime}
		\int_0^{\Delta t}\dd{t^{\prime\prime}}
		\bra{i}
		\hat{\vb{E}}(t^\prime)
		\ketbra{f}
		\hat{\vb{E}}(t^{\prime\prime})
		\ket{i}
		\bra{g}
		\hat{\vb{d}}(t^\prime)
		\ketbra{e}
		\hat{\vb{d}}(t^{\prime\prime})
		\ket{g}
		\\
		&=
		\int_0^{\Delta t}\dd{t^\prime}
		\int_0^{\Delta t}\dd{t^{\prime\prime}}
		\bra{i}
		\hat{\vb{E}}(t^\prime)
		\left(
			\sum_f
			\ketbra{f}
		\right)
		\hat{\vb{E}}(t^{\prime\prime})
		\ket{i}
		\bra{g}
		\hat{\vb{d}}(t^\prime)
		\left(
			\sum_e
			\ketbra{e}
		\right)
		\hat{\vb{d}}(t^{\prime\prime})
		\ket{g}
		\\
		&=
		\int_0^{\Delta t}\dd{t^\prime}
		\int_0^{\Delta t}\dd{t^{\prime\prime}}
		\bra{i}
		\hat{\vb{E}}(t^\prime)
		\hat{\vb{E}}(t^{\prime\prime})
		\ket{i}
		\bra{g}
		\hat{\vb{d}}(t^\prime)
		\hat{\vb{d}}(t^{\prime\prime})
		\ket{g}
		\\
		&=
		\int_0^{\Delta t}\dd{t^\prime}
		\int_0^{\Delta t}\dd{t^{\prime\prime}}
		G_i(t^\prime,t^{\prime\prime})^*
		G_g(t^\prime,t^{\prime\prime})
	\end{split}
\end{equation}
where $G_i$ and $G_g$ are the two-time correlation functions of the atomic detector system and radiation field.

\subsubsection{Mandel and Wolf}

In the model presented in Ref.~\cite[p.~685]{Mandel1995}, the interaction Hamiltonian is~\cite[p.~689]{Mandel1995}
\begin{equation}
	\hat{H}_\text{int}(t)
	=
	-
	\hat{\vb{p}}(t)
	\vdot
	\hat{\vb{A}}(\vb{x}_0,t)
\end{equation}
wherein $\vb{x}_0$ is the detector atom's \gls{com}.
The interaction term can be transformed into the dipole moment operator $\hat{\vb{d}}$ and the dielectric displacement field operator $\hat{\vb{D}}$~\cite[p.~689]{Mandel1995} giving a similar interaction term as discussed by Cohen-Tannoudji.

We then consider a bound electron in a potential well. When bound, the electron state $\ket{\Psi_0}$ satisfies
\begin{equation}
	\hat{H}_a
	\ket{g}
	=
	E_g
	\ket{g}
	=
	-\omega_g
	\ket{g}
	.
\end{equation}
Let $\hat\rho_f$ be the radiation field state in the Schrödinger picture
\begin{equation}
	\hat\rho^{(S)}(t_0)
	=
	\ketbra{g}
	\otimes
	\hat\rho_f(t_0)
\end{equation}
then in the interaction picture, the state is~\cite[p.~685]{Mandel1995}
\begin{equation}
	\hat\rho^{(I)}(t)
	=
	e^{+i\hat{H}_0(t-t_0)}
	\hat\rho^{(S)}(t)
	e^{-i\hat{H}_0(t-t_0)}
\end{equation}
and the electron's momentum operator takes the form
\begin{equation}
	\hat{\vb{p}}(t)
	=
	e^{+i\hat{H}_a(t-t_0)}
	\hat{\vb{p}}
	e^{-i\hat{H}_a(t-t_0)}
\end{equation}
and the Maxwell field is
\begin{equation}
	\begin{split}
		\hat{\vb{A}}(\vb{x}_0,t)
		&=
		\hat{\vb{A}}^{(+)}(\vb{x}_0,t)
		+
		\hat{\vb{A}}^{(-)}(\vb{x}_0,t)
		\\
		&=
		\sum_{\lambda=1,2}
		\int_{\mathbb{R}^3}
		\frac{\dd[3]{p}}{(2\pi)^3\sqrt{2\omega(\vb{p})}}
		\hat{a}_\lambda(\vb{p})
		\boldsymbol{\varepsilon}_\lambda(\vb{p})
		e^{+i\vb{p}\vdot\vb{x}_0-i\omega(\vb{p})(t-t_0)}
		+
		\text{h.c.}
		.
	\end{split}
\end{equation}
In the interaction picture, the quantum state evolves according to
\begin{equation}
	\dv{\hat\rho^{(I)}}{t}
	=
	i\comm{\hat\rho^{(I)}}{\hat{H}^{(I)}_\text{int}}
\end{equation}
which can be solved by Magnus expansion.
For instance,
\begin{equation}
	\hat\rho^{(I)}(t)
	=
	\hat\rho(t_0)
	+
	i\int_{t_0}^t\dd{t^\prime}
	\comm{\hat\rho(t_0)}{\hat{H}_\text{int}(t^\prime)}
	+
	i^2
	\int_{t_0}^t\dd{t^\prime}
	\int_{t_0}^{t^\prime}\dd{t^{\prime\prime}}
	\dots
\end{equation}

The probability amplitude for the transition
\begin{equation}
	\ket{g,i}
	\to
	\ket{e,f}
\end{equation}
is equal to~\cite[p.~686]{Mandel1995}
\begin{equation}
	\begin{split}
		p(t_0,\Delta t)
		=
		\tr\left\{
			\hat\rho_{e,f}
			\hat\rho_{g,i}(t_0+\Delta t)
		\right\}
		&=
		\tr\left\{
			\hat\rho_{e,f}
			\hat\rho_{g,i}(t_0)
		\right\}
		\\
		&+
		\frac{1}{i}
		\tr\left\{
			\hat\rho_{e,f}
			\int_{t_0}^{t_0+\Delta t}
			\dd{t^\prime}
			\comm{\hat{H}_\text{int}(t^\prime)}{\hat\rho_{g,i}(t_0)}
		\right\}
		\\
		&+
		\frac{1}{i^2}
		\tr\left\{
			\hat\rho_{e,f}
			\int_{t_0}^{t_0+\Delta t}\dd{t^\prime}
			\int_{t_0}^{t^\prime}\dd{t^{\prime\prime}}
			\comm{\hat{H}_\text{int}(t^\prime)}{\comm{\hat{H}_\text{int}(t^{\prime\prime})}{\hat\rho_{g,i}(t_0)}}
		\right\}
	\end{split}
\end{equation}
the first two terms vanish and we have
\begin{equation}
	p(t_0,\Delta t)
	=
	\int_{t_0}^{t_0+\Delta t}\dd{t^\prime}
	\int_{t_0}^{t^\prime}\dd{t^{\prime\prime}}
	\expval{\hat{H}_\text{int}(t^\prime)\hat\rho(t_0)\hat{H}_\text{int}(t^{\prime\prime})}{e,f}
	+
	\text{c.c.}
\end{equation}
We now take the interaction Hamiltonian
\begin{equation}
	\hat{H}_\text{int}(t)
	=
	e^{+i\hat{H}_a(t-t_0)}
	\hat{\vb{p}}
	e^{-i\hat{H}_a(t-t_0)}
	\hat{\vb{A}}(\vb{x}_0,t)
\end{equation}
which we evaluate with \textcolor{red}{check this!}
\begin{equation}
	\begin{split}
		\expval{\hat{H}_\text{int}(t^\prime)\hat\rho(t_0)\hat{H}_\text{int}(t^{\prime\prime})}{e,f}
		&=
		\bra{e}\hat{p}_i\ket{g}
		\bra{g}\hat{p}_j\ket{e}
		e^{i(E-E_0)(t^\prime-t^{\prime\prime})}
		\\
		&\times
		\bra{f}
		\hat{A}_i(\vb{x}_0,t^\prime)
		\braket{i}
		\hat{A}_j(\vb{x}_0,t^{\prime\prime})
		\ket{f}
		+
		\text{c.c.}
	\end{split}
\end{equation}
Expanding the initial state in the coherent state basis
\begin{equation}
	\ket{i}
	=
	\int\dd[2]{\alpha}
	p_i(\alpha,t_0)
	\ketbra{\alpha}
\end{equation}
we can use the eigenvalue equation of the coherent state and sum over all final states to remove the final state dependency, i.e.,
\begin{equation}
	\sum_i
	p(t_0,\Delta t)
	=
	\int_{t_0}^{t_0+\Delta t}\dd{t^\prime}
	\int_{t_0}^{t^\prime}\dd{t^{\prime\prime}}
	\bra{e}\hat{p}_i\ket{g}
	\bra{g}\hat{p}_j\ket{e}
	e^{i(E-E_0)(t^\prime-t^{\prime\prime})}
	\expval{\hat{A}_i(\vb{x}_0,t^\prime)\hat{A}_j(\vb{x}_0,t^{\prime\prime})}
	+
	\text{c.c.}
\end{equation}
We then sum of all final electron states weighted by the density of states times the the probability of being collected by the detector and find
\begin{equation}
	P(t_0,\Delta t)
	=
	\int_{t_0}^{t_0+\Delta t}\dd{t^\prime}
	\int_{t_0}^{t^\prime}\dd{t^{\prime\prime}}
	k_{ij}(t^\prime-t^{\prime\prime})
	\expval{\hat{A}_i(\vb{x}_0,t^\prime)\hat{A}_j(\vb{x}_0,t^{\prime\prime})}
	+
	\text{c.c.}
\end{equation}
see Ref.~\cite[p.~694]{Mandel1995} for an explicit representation of the response function $k_{ij}$.
Employing normal-ordering of the Maxwell field operators, we find that the second term, the vacuum contribution, becomes zero Ref.~\cite[p.~694]{Mandel1995} and we can write
\begin{equation}
	P(t_0,\Delta t)
	=
	\int_{t_0}^{t_0+\Delta t}\dd{t^\prime}
	\int_{t_0}^{t^\prime}\dd{t^{\prime\prime}}
	k_{ij}(t^\prime-t^{\prime\prime})
	\expval{\colon\hat{A}_i(\vb{x}_0,t^\prime)\hat{A}_j(\vb{x}_0,t^{\prime\prime})\colon}
	+
	\text{c.c.}
\end{equation}
\textcolor{red}{Can we rewrite this in terms of electric field operators?} -> Yes, but we need to perform more approximations

\begin{equation}
	\begin{split}
		\colon
		\hat{A}_i(\vb{x}_0,t^\prime)
		\hat{A}_j(\vb{x}_0,t^{\prime\prime})
		\colon
		&=
		\colon
		\left[
			\hat{A}_i^{(+)}(\vb{x}_0,t^\prime)
			+
			\hat{A}_i^{(-)}(\vb{x}_0,t^\prime)
		\right]
		\left[
			\hat{A}_j^{(+)}(\vb{x}_0,t^{\prime\prime})
			+
			\hat{A}_j^{(-)}(\vb{x}_0,t^{\prime\prime})
		\right]
		\colon
		\\
		&\approx
		\colon
		\left[
			\hat{A}_i^{(+)}(\vb{x}_0,t^\prime)
			\hat{A}_j^{(-)}(\vb{x}_0,t^{\prime\prime})
			+
			\hat{A}_i^{(-)}(\vb{x}_0,t^\prime)
			\hat{A}_j^{(+)}(\vb{x}_0,t^{\prime\prime})
		\right]
		\colon
		\\
		&=
		\hat{A}_i^{(+)}(\vb{x}_0,t^\prime)
		\hat{A}_j^{(-)}(\vb{x}_0,t^{\prime\prime})
		+
		\hat{A}_j^{(+)}(\vb{x}_0,t^{\prime\prime})
		\hat{A}_i^{(-)}(\vb{x}_0,t^\prime)
	\end{split}
\end{equation}
\begin{equation}
	\begin{split}
		\hat{A}_i^{(+)}(\vb{x}_0,t^\prime)
		\hat{A}_j^{(-)}(\vb{x}_0,t^{\prime\prime})
		&=
		\left(
			\sum_{\lambda=1,2}
			\int\frac{\dd[3]{p}}{(2\pi)^3\sqrt{2\omega(\vb{p})}}
			\hat{a}_\lambda^\dagger(\vb{p})
			\vu{e}^i_\lambda(\vb{p})^*
			e^{+i\omega(\vb{p})t^\prime}
		\right)
		\\
		&\times
		\left(
			\sum_{\sigma=1,2}
			\int\frac{\dd[3]{q}}{(2\pi)^3\sqrt{2\omega(\vb{q})}}
			\hat{a}_\sigma(\vb{q})
			\vu{e}^j_\sigma(\vb{q})
			e^{-i\omega(\vb{q})t^{\prime\prime}}
		\right)
	\end{split}
\end{equation}
Let us further neglect polarization
\begin{equation}
	\begin{split}
		P(t_0,\Delta t)
		&=
		\int_{t_0}^{t_0+\Delta t}\dd{t^\prime}
		\int_{t_0}^{t^\prime}\dd{t^{\prime\prime}}
		k(t^\prime-t^{\prime\prime})
		\\
		&\times
		\expval{
			\hat{A}^{(+)}(\vb{x}_0,t^\prime)
			\hat{A}^{(-)}(\vb{x}_0,t^{\prime\prime})
			+
			\hat{A}^{(+)}(\vb{x}_0,t^{\prime\prime})
			\hat{A}^{(-)}(\vb{x}_0,t^\prime)
		}
		+
		\text{c.c.}
	\end{split}
\end{equation}

\subsubsection{Vogel}

Based on Ref.~\cite[p.~48]{Vogel2006}

The minimal-coupling Hamiltonian
\begin{equation}
	\hat{H}
	=
	\int\dd[3]{x}
	\int_0^\omega\dd{\omega}
	\hat{\vb{f}}^\dagger(\vb{x},\omega)
	\hat{\vb{f}}(\vb{x},\omega)
	+
	\sum_j\frac{\left(\hat{\vb{p}}_j-q_j\hat{\vb{A}}(\hat{\vb{x}}_j,t)\right)^2}{2m_j}
	+
	\hat{W}_\text{Coul}
\end{equation}
wherein $\hat{W}_\text{Coul}$ is the Coulomb interaction between the different charges.
For bound atomic states, we can expand the Maxwell field around the atom's \gls{com} and the interaction Hamiltonian takes the form
\begin{equation}
	\hat{H}_\text{int}
	=
	-
	\sum_j\frac{q_j}{m_j}
	\hat{\vb{p}}_j
	\vdot
	\hat{\vb{A}}(\vb{x}_j)
	+
	\sum_j\frac{q_j^2}{2m_j}
	\hat{\vb{A}}(\vb{x}_j)^2
\end{equation}
In the electric-dipole approximation we have
\begin{equation}
	\hat{H}_\text{int}
	\approx
	-
	\sum_j\frac{q_j}{m_j}
	\hat{\vb{p}}_j
	\vdot
	\hat{\vb{A}}(\vb{x}_j)	
\end{equation}
\textcolor{red}{fancy reasoning why the former interaction Hamiltonian is equivalent to} (maybe \cite[p.~691]{Mandel1995} or Cohen-Tanodji?)
\begin{equation}
	\hat{H}_\text{int}(t)
	\approx
	-
	\sum_j
	\hat{\vb{d}}_j
	\vdot
	\hat{\vb{E}}(\vb{x}_j,t)
	.
\end{equation}

Based on Ref.~\cite[p.~173]{Vogel2006}

The photoemission probability is equal to
\begin{equation}
	\begin{split}
		\abs{\bra{e,f}\hat{U}(t_0,t_0+\Delta t)\ket{g,i}}^2
		&=
		\bra{e,f}
		\hat{U}(t_0,t_0+\Delta t)
		\ket{g,i}
		\\
		&\times
		\bra{g,i}
		\hat{U}^\dagger(t_0,t_0+\Delta t)
		\ket{e,f}
		\\
		&=
		\tr\biggl\{
			\bra{e,f}
			\hat{U}(t_0,t_0+\Delta t)
			\ket{g,i}
			\\
			&\times
			\bra{g,i}
			\hat{U}^\dagger(t_0,t_0+\Delta t)
			\ket{e,f}
		\biggr\}
		\\
		&=
		\tr\biggl\{
			\ketbra{e,f}
			\hat{U}(t_0,t_0+\Delta t)
			\ketbra{g,i}
			\hat{U}^\dagger(t_0,t_0+\Delta t)
		\biggr\}
		\\
		&=
		\tr\left\{
			\hat\rho_{e,f}
			\hat\rho_{g,i}(t_0+\Delta t)
		\right\}
	\end{split}
\end{equation}
wherein the time-evolution operator is
\begin{equation}
	\hat{U}(t_0,t_0+\Delta t)
	=
	\mathcal{T}_+
	\exp\left\{
		-i
		\int_{t_0}^{t_0+\Delta t}\dd{t^\prime}
		\hat{H}_\text{int}(t^\prime)
	\right\}
\end{equation}
then, evaluating the transition amplitude
\begin{equation}
	\bra{g,f}
	\hat{U}(t_0,t_0+\Delta t)
	\ket{e,i}
	=
	\mathcal{T}_+
	\sum_{n=0}^\infty
	\frac{1}{n!}
	\bra{g,f}
	\left[
		i
		\int_{t_0}^{t_0+\Delta t}\dd{t^\prime}
		\vb{d}_{fg}
		\vdot
		\hat{\vb{E}}^{(+)}(\vb{x}_0,t^\prime)
		e^{i\omega_{fg}(t^\prime-t)}
	\right]^n
	\ket{e,i}
\end{equation}
\textcolor{red}{why do we only have $E^+$ here? How to derive this exactly?}

\subsection{Photodetection}

Let us consider the composite system of an atom with a single electron and a radiation field.
\textcolor{red}{Figure where we see an electron in a potential well and how radiation can excite it...}

In the ground state $\ket{g}$, the electron is bound and satisfies
\begin{equation}
	\hat{H}_a
	\ket{g}
	=
	E_g
	\ket{g}
	.
\end{equation}
For energies $E>0$, the electron is in one of many free excited state $\ket{e}$.
\textcolor{red}{photoelectric effect. why can this be used for photodiodes where there is no ionization happening. See Cohen-Tannoudji for explanation.}

Whenever, we ionize the atom, we can collect the free electron indicating a photo detection.
The probability that such an event occurs in the time interval $[t_0,t_0+\Delta t]$ is
\begin{equation}
	p_{e,f}(t_0,t_0+\Delta t)
	=
	\abs{
		\bra{e,f}
		\hat{U}(t_0,t_0+\Delta t)
		\ket{g,i}
	}^2
	.
\end{equation}
wherein $\hat{U}$ is the time-evolution operator.
We can recast the probability in the more general density operator formalism
\begin{equation}
	\begin{split}
		p_{e,f}(t_0,t_0+\Delta t)
		&=
		\bra{e,f}
		\hat{U}(t_0,t_0+\Delta t)
		\ketbra{g,i}
		\hat{U}^\dagger(t_0,t_0+\Delta t)
		\ket{e,f}
		\\
		&=
		\tr\left\{
			\bra{e,f}
			\hat{U}(t_0,t_0+\Delta t)
			\ketbra{g,i}
			\hat{U}^\dagger(t_0,t_0+\Delta t)
			\ket{e,f}
		\right\}
		\\
		&=
		\tr\left\{
			\ketbra{e,f}
			\hat{U}(t_0,t_0+\Delta t)
			\ketbra{g,i}
			\hat{U}^\dagger(t_0,t_0+\Delta t)
		\right\}
		\\
		&=
		\tr\left\{
			\hat\varrho
			\hat{U}(t_0,t_0+\Delta t)
			\hat\rho(t_0)
			\hat{U}^\dagger(t_0,t_0+\Delta t)
		\right\}
		\\
		&=
		\tr\left\{
			\hat\varrho
			\hat\rho(t_0+\Delta t)
		\right\}
	\end{split}
\end{equation}
where we used that the trace of a scalar is the scalar in the second line and the cyclic property of the trace in the third line.
The time-evolution operator is
\begin{equation}
	\hat{U}(t_0,t)
	=
	\mathcal{T}_+
	\exp\left\{
		-i
		\int_{t_0}^t\dd{t^\prime}
		\hat{H}_\text{int}(t^\prime)
	\right\}
\end{equation}
with the interaction Hamiltonian in the electric-dipole and rotating wave approximation being
\begin{equation}
	\hat{H}_\text{int}(t)
	=
	-
	\hat{\vb{p}}(t)
	\vdot
	\hat{\vb{A}}(\vb{x}_0,t)
	.
\end{equation}
Instead of the time-ordered exponential, we can use the Magnus expansion
\begin{align}
	\hat{U}(t_0,t)
	&=
	e^{\Omega(t_0,t)}
	&
	\Omega(t_0,t)
	&=
	\sum_{n=1}\Omega^{(n)}(t_0,t)
\end{align}
The time evolved quantum state is then
\begin{equation}
	\begin{split}
		\hat\rho(t_0+\Delta t)
		&=
		\hat{U}(t_0,t_0+\Delta t)
		\hat\rho(t_0)
		\hat{U}^\dagger(t_0,t_0+\Delta t)
		\\
		&=
		\hat\rho(t_0)
		+
		\comm{\Omega(t_0,t)}{\hat\rho(t_0)}
		+
		\frac{1}{2!}
		\comm{\Omega(t_0,t)}{\comm{\Omega(t_0,t)}{\hat\rho(t_0)}}
		+
		\dots
	\end{split}
\end{equation}
where we used the \gls{bch} formula.
We perform Magnus expansion up to the first term and find the perturbative solution
\begin{equation}
	\begin{split}
		\hat\rho(t_0+\Delta t)
		\approx
		\hat\rho(t_0)
		&+
		(-i)
		\int_{t_0}^{t_0+\Delta t}\dd{t_1}
		\comm{\hat{H}_\text{int}(t_1)}{\hat\rho(t_0)}
		\\
		&+
		\frac{(-i)^2}{2!}
		\int_{t_0}^{t_0+\Delta t}\dd{t_1}
		\int_{t_0}^{t_1}\dd{t_2}
		\comm{\hat{H}_\text{int}(t_1)}{\comm{\hat{H}_\text{int}(t_2)}{\hat\rho(t_0)}}
	\end{split}	
\end{equation}
and insert the expansion into the photoemission probability
\begin{equation}
	\begin{split}
		p_{e,f}(t_0,t_0+\Delta t)
		&=
		\tr\left\{
			\hat\varrho
			\hat\rho(t_0+\Delta t)
		\right\}
		\\
		&=
		\int_{t_0}^{t_0+\Delta t}\dd{t_1}
		\int_{t_0}^{t_1}\dd{t_2}
		\tr\left\{
			\hat{H}_\text{int}(t_1)
			\hat\rho(t_0)
			\hat{H}_\text{int}(t_2)
		\right\}
		+
		\text{c.c.}
		\\
		&=
		\bra{e}\hat{p}_i\ket{g}
		\bra{g}\hat{p}_j\ket{e}
		\int_{t_0}^{t_0+\Delta t}\dd{t_1}
		\int_{t_0}^{t_1}\dd{t_2}
		e^{i(E_e-E_g)(t_1-t_2)}
		\\
		&\times
		\expval{
			\hat{A}_j(\vb{x}_0,t_2)
			\hat\rho(t_0)
			\hat{A}_i(\vb{x}_0,t_1)
		}{i}
		+
		\text{c.c.}
	\end{split}
\end{equation}
\textcolor{red}{steps to second line missing!}
We are not interested in the final states and can integrate this degree of freedom.
Furthermore, we can assume an ensemble of independent detector atoms, so the probability for a photoemission is
\begin{equation}
	p(t_0,t_0+\Delta t)
	=
	\int_{t_0}^{t_0+\Delta t}\dd{t_1}
	\int_{t_0}^{t_1}\dd{t_2}
	k_{ij}(t_1-t_2)
	\expval{
		\hat{A}_j(\vb{x}_0,t_2)
		\hat{A}_i(\vb{x}_0,t_1)
	}
	+
	\text{c.c.}
\end{equation}
wherein $k_{ij}$ encodes the microscopic properties of the detector atom ensemble, see Ref.~\cite[p.~694]{Mandel1995}.
Expansion of the Maxwell field operator into positive and negative frequency parts as well as discarding high-frequency terms, we find~\cite[p.~698]{Mandel1995}.
\begin{equation}
	p(t_0,t_0+\Delta t)
	\approx
	\int_0^{\Delta t}\dd{t^\prime}
	\int_0^{t^\prime}\dd{t^{\prime\prime}}
	k_{ij}(t^{\prime}-t^{\prime\prime})
	\expval{
		\hat{A}_j^{(+)}(\vb{x}_0,t_0+t^{\prime\prime})
		\hat{A}_i^{(-)}(\vb{x}_0,t_0+t^{\prime})
	}
	+
	\text{c.c.}
	.
\end{equation}
Furthermore, performing the quasi-monochromatic approximation
\begin{align}
	\hat{A}_j^{(+)}(\vb{x}_0,t_0+t^{\prime\prime})
	&\approx
	\hat{A}_j^{(+)}(\vb{x}_0,t_0)
	e^{+i\omega_0t^{\prime\prime}}
	\\
	\hat{A}_j^{(-)}(\vb{x}_0,t_0+t^{\prime})
	&\approx
	\hat{A}_j^{(-)}(\vb{x}_0,t_0)
	e^{-i\omega_0t^{\prime}}
\end{align}
we find
\begin{equation}
	\begin{split}
		p(t_0,t_0+\Delta t)
		&\approx
		\expval{
			\hat{A}_j^{(+)}(\vb{x}_0,t_0)
			\hat{A}_i^{(-)}(\vb{x}_0,t_0)
		}
		\int_0^{\Delta t}\dd{t^\prime}
		\int_0^{t^\prime}\dd{t^{\prime\prime}}
		k_{ij}(t^{\prime}-t^{\prime\prime})
		e^{-i\omega_0(t^\prime-t^{\prime\prime})}
		+
		\text{c.c.}
		\\
		&=
		\expval{
			\hat{A}_j^{(+)}(\vb{x}_0,t_0)
			\hat{A}_i^{(-)}(\vb{x}_0,t_0)
		}
		\int_0^{\Delta t}\dd{t^\prime}
		\int_0^{t^\prime}\dd{\tau}
		k_{ij}(\tau)
		e^{-i\omega_0\tau}
		+
		\text{c.c.}
		\\
		&=
		\expval{
			\hat{A}_j^{(+)}(\vb{x}_0,t_0)
			\hat{A}_i^{(-)}(\vb{x}_0,t_0)
		}
		\int_0^{\Delta t}\dd{t^\prime}
		\int_{-t^\prime}^{t^\prime}\dd{\tau}
		k_{ij}(\tau)
		e^{-i\omega_0\tau}
		\\
		&\approx
		\expval{
			\hat{A}_j^{(+)}(\vb{x}_0,t_0)
			\hat{A}_i^{(-)}(\vb{x}_0,t_0)
		}
		k_{ij}(\omega_0)
		\Delta t
	\end{split}
\end{equation}
as in Ref.~\cite[p.~699]{Mandel1995} where $k_{ij}(\omega_0)$ is the frequency response of the detector atom ensemble.
We assume a detector equally sensitive to both polarizations and finally find
\begin{equation}
	p(t_0,t_0+\Delta t)
	\approx
	\expval{\hat{N}}
	k(\omega_0)
	\Delta t
\end{equation}
we can even relax the quasi-monochromatic approximation a bit by writing
\begin{equation}
	p(t_0,t_0+\Delta t)
	\approx
	\int\dd{\omega}
	k(\omega)
	\expval{\hat{n}(\omega)}
	\Delta t
	.
\end{equation}
		
%		\addcontentsline{toc}{section}{References}
%		\printbibliography[title=References]
%	\end{refsection}

\end{document}

\documentclass[
	a4paper,
	parskip,
	appendixprefix,
	chapterprefix,
	headings=big,
]{scrreprt}

\usepackage[utf8]{inputenc}
\usepackage[T1]{fontenc}
\usepackage{amsthm,amsmath,amssymb}
\usepackage{authblk}
\usepackage[english]{babel}
\usepackage[backend=biber]{biblatex}
\usepackage{booktabs}
%\usepackage{csquotes}
\usepackage[acronym,nonumberlist,toc]{glossaries}
\usepackage{hyperref}
\usepackage{graphicx}
\usepackage{cleveref}
\usepackage{physics}
\usepackage[section]{placeins}
\usepackage[separate-uncertainty=true]{siunitx}
\usepackage[mode=buildnew]{standalone}
%\usepackage{thmtools,thm-restate}
\usepackage{lmodern}
\usepackage{multirow}
%\usepackage{subcaption}
%\usepackage{wrapfig}
\usepackage{xcolor}

\addbibresource{literature.bib}

% add bibliography as section (not chapter)
% https://tex.stackexchange.com/questions/568580/make-the-bibliography-as-a-section-in-each-includ-chapter
\defbibheading{bibliography}[\bibname]{\section*{#1}}

% approximately proportional to symbol
% https://tex.stackexchange.com/questions/33538/how-to-get-an-approximately-proportional-to-symbol
\def\app#1#2{%
    \mathrel{%
        \setbox0=\hbox{$#1\sim$}%
        \setbox2=\hbox{%
            \rlap{\hbox{$#1\propto$}}%
            \lower1.1\ht0\box0%
        }%
        \raise0.25\ht2\box2%
    }%
}
\def\approxprop{\mathpalette\app\relax}

% overwrite real and imaginary part operators
\let\Re\undefined
\let\Im\undefined
\DeclareMathOperator{\Re}{\operatorname{Re}}
\DeclareMathOperator{\Im}{\operatorname{Im}}
% other functions
\DeclareMathOperator{\sinc}{\operatorname{sinc}}

\newcommand{\floor}[1]{\left\lfloor#1\right\rfloor}
\newcommand{\ceil}[1]{\left\lceil#1\right\rceil}

% transpose
% https://tex.stackexchange.com/questions/403104/small-caps-mathsf-font-for-writing-transpose-of-a-matrix
\newcommand{\trans}{{\scriptscriptstyle\mathsf{T}}}

% chapter appearence
% https://tex.stackexchange.com/questions/159502/koma-script-scrreprt-how-to-change-chapter-appearance-and-produce-a-chapter-bas
\addtokomafont{chapterprefix}{\raggedleft}
\renewcommand*{\chapterformat}{%
\mbox{\chapappifchapterprefix{\nobreakspace}%
\scalebox{3}{\color{gray}\thechapter\autodot}\enskip}}

% caption label linebreak
% https://tex.stackexchange.com/questions/196889/how-can-i-insert-a-line-break-after-the-captionlabel-koma-script-options
%\setcaphanging
%\setcapmargin{2ex}
%\setcapindent{0pt}
%\setcapwidth[c]{0.8\textwidth}
%\addtokomafont{caption}{\centering}
\setkomafont{captionlabel}{\sffamily}

% theorems
\newtheorem{theorem}{Theorem}[section]
\newtheorem{lemma}[theorem]{Lemma}
\newtheorem{corollary}[theorem]{Corollary}
\theoremstyle{definition}
\newtheorem{definition}{Definition}[section]
\newtheorem{conjecture}{Conjecture}[section]
\newtheorem{example}{Example}[section]
\theoremstyle{remark}
\newtheorem*{remark}{Remark}

% prefix equation numbers with section number
\numberwithin{equation}{section}

% optics
\newacronym{ar}{AR}{anti-reflective}
\newacronym{mzm}{MZM}{Mach-Zehnder modulator}
\newacronym{mzi}{MZI}{Mach-Zehnder interferometer}
\newacronym{bs}{BS}{beam splitter}
\newacronym{fc}{FC}{fiber coupler}
\newacronym{qe}{QE}{quantum efficiency}

% physics
\newacronym{dv}{DV}{discrete-variable}
\newacronym{cv}{CV}{continuous-variable}
\newacronym{dof}{DOF}{degrees of freedom}
\newacronym{eom}{EOM}{equation(s) of motion}
\newacronym{pbc}{PBC}{periodic boundary conditions}
\newacronym{bch}{BCH}{Baker-Campbell-Hausdorff}
\newacronym{ccr}{CCR}{canonical commutation relation}

% electrical engineering
\newacronym{asp}{ASP}{analog signal processing}
\newacronym{dsp}{DSP}{digital signal processing}
\newacronym{osp}{OSP}{optical signal processing}
\newacronym{lo}{LO}{local oscillator}
\newacronym{if}{IF}{intermediate frequency}
\newacronym{lp}{LP}{low-pass}
\newacronym{tia}{TIA}{transimpedance amplifier}
\newacronym{iq}{I/Q}{in-phase/quadrature}
\newacronym{psd}{PSD}{power spectral density}
\newacronym{snr}{SNR}{signal-to-noise ratio}
\newacronym{rc}{RC}{raised-cosine}
\newacronym{rrc}{RRC}{root-raised-cosine}
\newacronym{adc}{ADC}{analog-to-digital converter}
\newacronym{dac}{DAC}{digital-to-analog converter}
\newacronym{qam}{QAM}{quadrature amplitude modulation}
\newacronym{qpsk}{QPSK}{quadrature phase-shift keying}

% cryptography
\newacronym{aes}{AES}{advanced encryption standard}
\newacronym{cvqkd}{CV-QKD}{continuous-variable quantum-key distribution}
\newacronym{dvqkd}{DV-QKD}{discrete-variable quantum-key distribution}
\newacronym{gnfs}{GNFS}{generalized number field sieve}
\newacronym{mac}{MAC}{message authentication code}
\newacronym{ldpc}{LDPC}{low-density parity-check}
\newacronym{otp}{OTP}{one-time pad}
\newacronym{pkd}{PKD}{public-key distribution}
\newacronym{qkd}{QKD}{quantum-key distribution}
\newacronym{rsa}{RSA}{Rivest–Shamir–Adleman}
\newacronym{ecdh}{ECDH}{elliptic-curve Diffie-Hellman}
\newacronym{qber}{QBER}{quantum-bit error-rate}

\begin{document}
	\begin{titlepage}
		\begin{center}
			\large
   		    \textbf{\textsf{Experimental quantum optics}}\\
		    \vspace{0.8em}
			\huge
		    \textbf{\textsf{A theoretical framework for quantum optical communication - towards CV-QKD}}\\

		    \vspace{.8em}
			\begin{figure}[htb]
				\centering
				%\includestandalone{figures/pgfplots/phase-space}
			\end{figure}

		    \vspace{.6em}
		    \large
		    \textbf{Master thesis by}\\
   		    \vspace{.8em}
		    \large
			Bodo Kaiser\\
		    \vspace{.2em}
			\textit{bodo.kaiser@physik.uni-muenchen.de}

		    \usekomafont{date}
		    \today

		    \vspace{1.9em}
			\normalsize
			\begin{tabular}{ll}
			Internal supervisor: & Prof. Dr. Monika Aidelsburger \\
			External supervisor: & Dr. Hans Brunner \\
			\end{tabular}
		\end{center}
	\end{titlepage}
	\tableofcontents

	\chapter{Introduction}
	\begin{refsection}
		\textit{The following chapter presents the fundamental idea of quantum-key distribution and its many facets, usually overlooked in the introductory material.}
        \documentclass[tikz]{standalone}

\usetikzlibrary{positioning}

\begin{document}
	\begin{tikzpicture}[
		node distance=1em,
		block/.style={draw, very thick, fill=white, minimum height=18ex, minimum width=8em, text width=8em, align=center},
		super block/.style={draw, very thick, fill=white, minimum height=10ex, minimum width=32em, align=center},
	]
		\coordinate (in) at (0,0);
		\node [block, right=of in] (comm) {\textsf{Communication\\and\\security}};
		\node [block, right=of comm] (comp) {\textsf{Computation\\and\\simulation}};
		\node [block, right=of comp] (sens) {\textsf{Sensing\\and\\metrology}};
		\coordinate[right=of sens] (out);

		\node [super block, below=of comp] {\textsf{Basic science}};
		\node [super block, above=of comp] {\textsf{Application}};
	\end{tikzpicture}
\end{document}

        \documentclass[tikz]{standalone}

\usetikzlibrary{arrows.meta,positioning}

\begin{document}
	\begin{tikzpicture}[
		arrow/.style={-Latex},
		block/.style={draw, very thick, fill=white, minimum height=8ex, minimum width=3.5em},
	]
		\coordinate (in) at (0,0);
		\node (pla) [right=of in] {\textsf{Plain text}};
		\node (enc) [block, right=of pla] {\textsf{Encoder}};
		\node (dec) [block, right=8em of enc] {\textsf{Decoder}};
		\node (plb) [right=of dec] {\textsf{Plain text}};
		\coordinate[right=of plb] (out);
		
		\node (keya) [below=of enc] {\textsf{Secret key}};
		\node (keyb) [below=of dec] {\textsf{Secret key}};
		
		\draw[arrow] (pla) -- (enc);
		\draw[arrow] (enc) -- (dec) node[midway, fill=white] {\textsf{Cipher text}};
		\draw[arrow] (dec) -- (plb);
		\draw[arrow] (keya) -- (enc);
		\draw[arrow] (keyb) -- (dec);
	\end{tikzpicture}
\end{document}

        \section{Problem statement}

Shortcomings of the state of the art:
 Most \gls{qkd} literature is extremely specific to the authors background, e.g., quantum optics, communication engineering, quantum information theory, while neglecting the view points of other fields.
However, one needs to the integrate all the view points and agree on a common language to get the whole picture.

To the best of our knowledge, our work is the first complete description of a \gls{cvqkd} implementation incorporating a time-dependent continuous-mode description of quantum states.

On an intellectual level, the complete description of a \gls{cvqkd} device heavily challenges us our understanding of physics.
In particular, we found that there are many implicit assumptions made to carry classical results over to a quantum model.
        \documentclass[tikz]{standalone}

\usetikzlibrary{fit,positioning}
\pgfdeclarelayer{z1}
\pgfdeclarelayer{z2}
\pgfdeclarelayer{z3}
\pgfdeclarelayer{z4}
\pgfsetlayers{main,z1,z2,z3,z4}

\begin{document}
	\begin{tikzpicture}[
		box/.style={draw, very thick, rectangle, rounded corners, fill=white, align=center, inner xsep=12pt, inner ysep=4pt},
	]
		\begin{pgfonlayer}{z4}
			\node[box, minimum height=24pt] (A) {\textsf{Quantum field theory of light}};
		\end{pgfonlayer}
		\begin{pgfonlayer}{z3}
			\node[box, fit=(A), yshift=10pt, text depth=40pt, text width=180] (B) {\textsf{Interaction theory of optical components}};
		\end{pgfonlayer}
		\begin{pgfonlayer}{z2}
			\node[box, fit=(A), yshift=22pt, text depth=80pt, text width=220pt] {\textsf{Coherent state transmission system}};
		\end{pgfonlayer}
		\begin{pgfonlayer}{z1}
			\node[box, fit=(A), yshift=34pt, text width=280pt, text depth=120pt] {\textsf{Classical post-processing}};
		\end{pgfonlayer}
	\end{tikzpicture}
\end{document}

        \section{Conventions and notation}

% Minkowski space, four vectors
% why we use p instead of omega for modes -> to distinguish between frequency and momentum

\begin{align}
	f(t)
	=
	\int_{\mathbb{R}}\frac{\dd{\omega}}{2\pi}
	f(\omega)
	e^{+i\omega t}
	&&
	f(\omega)
	=
	\int_{\mathbb{R}}\dd{t}
	f(t)
	e^{-i\omega t}
\end{align}
\begin{align}
	f(\vb{x})
	=
	\int_{\mathbb{R}^3}\frac{\dd[3]{p}}{(2\pi)^3}
	f(\vb{p})
	e^{-i\vb{p}\vdot\vb{x}}
	&&
	f(\vb{p})
	=
	\int_{\mathbb{R}^3}\dd[3]{x}
	f(\vb{x})
	e^{+i\vb{p}\vdot\vb{x}}
\end{align}
	
		\addcontentsline{toc}{section}{References}
		\printbibliography[title=References]
	\end{refsection}
	
	\chapter{Quantum-key distribution}
	\textit{The following chapter presents the fundamental idea of quantum-key distribution and its many facets, usually overlooked in the introductory material. First, we dedicate our attention to \gls{dvqkd} as it is more familiar than its counterpart \gls{cvqkd} which we discuss in the second part. For \gls{dvqkd}, we focus on the most common BB84 protocol, which we first discuss on an abstract protocol level and then practical implementations, particularly the polarization and time-phase encoding. We hope to convey to the reader the difference between the protocol and the encoding, which is not apparent when considering the basic polarization-encoding BB84. However, the difference between protocol and encoding will become central in the later stages of the CV-QKD thesis.}
	\begin{refsection}
		\section{Overview}

Let $M\subseteq\mathbb{N}$ denote an index set, $\left\{\alpha_n\right\}_{n\in M}$ as well as $\left\{\beta_n\right\}_{n\in M}$ two complex sequences, referred to as symbols.
A coherent state transmission system comprises a transmitter, a quantum channel, and a receiver, see \Cref{fig:transmission_system}.
\begin{figure}[htb]
	\centering
	\includestandalone{figures/tikz/transmission-system}
	\caption{Coherent state transmission system comprising a transmitter, a quantum channel, and a receiver. The transmitter encodes the symbols $\left\{\alpha_n\right\}_{n\in M}$ onto a coherent state $\ket{\alpha(t)}$. The channel maps from $\ket{\alpha(t)}$ to a coherent state $\ket{\beta(t)}$. The receiver decodes a symbol sequence $\left\{\beta_n\right\}_{n\in M}$ from $\ket{\beta(t)}$.}\label{fig:transmission_system}
\end{figure}
The transmitter encodes the symbols $\left\{\alpha_n\right\}_{n\in M}$ onto a coherent state $\ket{\alpha(t)}$ and passes it to the quantum channel.
The quantum channel maps $\ket{\alpha(t)}$ to a coherent state $\ket{\beta(t)}$ at a receiver, which decodes the symbols $\left\{\beta_n\right\}_{n\in M}$ from the coherent state $\ket{\beta(t)}$.	
		\section{Discrete-variable}

% variants to DV-QKD (six state, decoy states, active/passive)
% active vs passive basis selection on the receiver (compare homodyne/heterodyne measurement with polarization filter vs. beam splitter + polarization filters)
% dv-qkd (BB84)
% qubit/spin-state system (Pauli eigenbasis)
% equations for phase-encoding QKD

In \gls{dvqkd}, the receiver performs a discrete measurement.
Many variants of \gls{dvqkd} exist. For an overview of protocols, see Ref.~\cite{Duvsek2006}.
The most popular protocol is BB84~\cite{Bennett1984} because of its simplicity and well-developed security proof~\cite{Shor2000}.
BB84 is frequently used as the foundation for more practical protocols which do not rely upon a (perfect) single-photon source.

We find an innate understanding of BB84's operating principle by considering the measurement of a general two-state quantum system: a qubit.
A qubit is an element $\ket{\psi}$ of a two-dimensional complex Hilbert space which absolute squared is normalized to one $\abs{\braket{\psi}}^2=1$.
We define $\ket{0},\ket{1}$ to be the standard basis and write a qubit state as $\ket{\psi}=c_1\ket{0}+c_2\ket{1}$ wherein $\abs{c_1}^2+\abs{c_2}^2=1$.
The state space then equals the surface of a three-dimensional unit sphere, the Bloch sphere, illustrated in \Cref{fig:bloch_sphere}.
\begin{figure}[htb]
	\centering
	\includestandalone{figures/tikz/bloch-sphere}
	\caption{Two-state quantum system in the Bloch sphere representation: A (pure) quantum state  $\ket{\psi}=c_1\ket{0}+c_2\ket{1}$ with $\abs{c_1}^2+\abs{c_2}^2=1$ takes a point on the surface of the (Bloch) sphere which has unit radius. By convention, the standard basis $\left\{\ket{0},\ket{1}\right\}$ is set to equal the $Z$ Pauli eigenbasis $\left\{\ket{z_+},\ket{z_-}\right\}$. The $Y,Z$ Pauli eigenbasis are both orthogonal to the $Z$ basis.}\label{fig:bloch_sphere}
\end{figure}
Contrary to Euclidean space, two antiparallel axes piercing the Bloch sphere correspond to orthogonal elements.
The Bloch sphere is embedded in three-dimensional space, and there are in total three independent bases.
The Pauli algebra relates to two-dimensional rotations and thereby describes state transitions on the Bloch sphere.
When working with the Bloch sphere, it is convenient to select the three Pauli eigenbasis as an orthogonal basis triplet.
By convention one identifies the $Z$ Pauli eigenbasis with the standard basis $\left\{\ket{0},\ket{1}\right\}$.

Having introduced the Bloch sphere, we can visually explain the main idea of \gls{dvqkd}:
Alice and Bob agree on two (or three) orthogonal bases and map the respective basis elements to the zero and one bit.
Alice now chooses randomly a basis and a vector of that basis which she sends to Bob.
Bob randomly chooses a basis to measure the state he received from Alice.
If Alice and Bob select the same basis, Bob can accurately decode Alice's key bit from the measurement.
Alice and Bob's probability of choosing the same basis for one transmission is one divided by the number of orthogonal bases Alice and Bob have agreed on, e.g., \SI{25}{\percent} if Alice and Bob agreed to use the $X$ and $Z$ Pauli eigenbasis, also called the \gls{qber}.
In the asymptotic limit of many transmissions, the \gls{qber} should approach the theoretical limit.
Otherwise, an opposing third party, Eve, might have tempered with the transmission.
\Cref{tab:dvqkd_transmission_sequence} displays a possible BB84 prepare-and-measure sequence where Alice and Bob agreed to use the $X$ and $Z$ bases.
\begin{table}[htb]
	\centering
	\begin{tabular}{llccccc}
		\toprule
		& & \multicolumn{5}{c}{Transmission} \\
		\cmidrule{3-7}
		Party & Step & 1 & 2 & 3 & 4 & 5 \\ 
		\midrule
		\multirow{3}{*}{Alice} & Initial key bit & \num{0} & \num{1} & \num{1} & \num{0} & \num{0} \\
		& State basis & $Z$ & $X$ & $X$ & $Z$ & $X$ \\
		& Prepared state & $\ket{z_+}$ & $\ket{x_-}$ & $\ket{x_-}$ & $\ket{z_+}$ & $\ket{x_+}$ \\
		\cmidrule{1-1}
		\multirow{3}{*}{Bob} & Measurement basis & $X$ & $Z$ & $X$ & $Z$ & $Z$ \\
		& Possible outcomes & \num{0},\num{1} & \num{0},\num{1} & \num{1} & \num{0} & \num{0},\num{1} \\
		& Sifted key bit & - & - & 1 & 0 & - \\
		\bottomrule
	\end{tabular}
	\caption{Possible prepare-and-measure sequence for BB84: Alice randomly selects an initial key bit \num{0} or \num{1} and a state basis $X$ or $Z$ where $X$ respective $Z$ denote the eigenbasis of the Pauli $\sigma_x$ respective $\sigma_z$ matrix. Alice's initial key bit and selected basis determine the quantum state she prepares and sends to Bob. Bob randomly chooses a measurement basis. Only if Alice's and Bob's basis agree, the key bit is not discarded.}\label{tab:dvqkd_transmission_sequence}
\end{table}
The base sifting where Alice or Bob announce their chosen basis is part of the classical post-processing.
The post-processing distills a secret key bit string from the partially secret and correlated bit string Alice and Bob obtained from the transmission, see \Cref{fig:dvqkd_post_processing} for details.
\begin{figure}[htb]
	\centering
	\includestandalone{figures/tikz/post-processing-dv}
	\caption{Post-processing procedure for \gls{dvqkd} protocols: First Alice's and Bob's key bits are correlated and partially secret. Key sifting discards key bits where Alice and Bob chose a different basis or where Bob did not detect an event while error correction removes the remaining differences between Alice's and Bob's partially secret key bits. Comparing the correlated key bits with the corrected key bits, Alice and Bob can estimate an error and thereby an upper bound on information loss to an adversary. Alice and Bob can remove the remaining partial information an adversary potentially has by performing privacy amplification. Finally, Alice and Bob share a secret key bit string.}\label{fig:dvqkd_post_processing}
\end{figure}

So far, we assumed Alice to send a qubit (state) to Bob, but we have not discussed the physical system underlying, i.e., encoding, the qubit.
\Cref{tab:dvqkd_encodings} lists different physical means to encode a qubit.
For example, the polarization of light behaves like a qubit system, with the three orthogonal bases being circular, linear, and diagonal polarization.
While BB84~\cite{Bennett1984} was initially proposed with polarization-encoding, other encodings appear more practicable when considering transmission properties of the physical quantum channel.
\begin{table}[htb]
	\centering	
	\begin{tabular}{lcc}
		\toprule
		& \multicolumn{2}{c}{Standard basis} \\
		\cmidrule{2-3}
		Encoding variable & $\ket{0}$ & $\ket{1}$ \\
		\midrule
		Polarization & Horizontal & Vertical \\
		Photon number & Vacuum & Single-photon \\
		Squeezing & Amplitude & Phase \\
		Time-bin & Early & Late \\
		Phase-bin & \SI{0}{\deg} & \SI{180}{\deg} \\
		\bottomrule
	\end{tabular}
	\caption{Encoding variables and their usual assigned basis state for two-state \gls{dvqkd}: The specific choice of the encoding variable does not matter as long as the encoding variable can be described as a two-state quantum (qubit) system.}\label{tab:dvqkd_encodings}
\end{table}

\FloatBarrier
\subsection{Polarization-encoding}

\FloatBarrier
\subsection{Time-phase-encoding}

In the following, we discuss the practical time-phase-encoding BB84 protocol and show its equivalence to the polarization-encoding BB84.
The idea of using phase-encoding was first proposed as part of the BB92 protocol~\cite{Bennett1992}.
The basic setup is illustrated in \Cref{fig:time_phase_encoding_bb84} and comprises a single-photon source and a first \gls{mzi} on Alice's side as well as a second \gls{mzi} and two single-photon detecctors on Bob's side.
\begin{figure}[htb]
	\centering
	\includestandalone{figures/pstricks/phase-encoding-bb84}
	\caption{Fiber-optical setup of the phase-encoding BB84 DV-QKD protocol: Alice creates an entangled photon state using a first \gls{mzi} with phase $\theta=0,\pi/2,\pi,3\pi/2$ and sends it to Bob. Bob detects the photon state using a second \gls{mzi} with phase $\phi=0,\pi$ and two single-photon detectors monitoring the outputs.}\label{fig:time_phase_encoding_bb84}
\end{figure}

To understand the time-phase encoding, we analyze the action of the (asymmetric) \gls{mzi} with variable phase $\varphi$ on a photon pulse $\ket{t_0}$ arriving at time $t_0$, see \Cref{fig:mzi_asym}.
\begin{figure}[htb]
    \centering
    \includegraphics{figures/pstricks/mzi-asym}
     \caption{Asymmetric \gls{mzi} adding a constant time delay and variable phase difference between the upper and lower path: A pulsed state enters the first beam splitter BS1 to the left and is split among a longer upper path and a shorter lower path. A first mirror M1 directs the pulse from the upper path to a phase shifter which adds a relative phase of $\varphi$ between the upper and lower path. A second mirror M2 directs the pulse from the phase shifter to a second beam splitter BS2 while the lower path is between BS1 and BS2.}\label{fig:mzi_asym}
\end{figure}
An ideal (lossless) and symmetric beam splitter transforms the single-photon input states into a superposition according to\footnote{See Ref.~\cite[p.~137]{Haroche2006} and Ref.~\cite[p.~143]{Gerry2005}}
\begin{align}
	\hat{U}_\text{BS}
	\ket{1,0}
	&=
	\frac{1}{\sqrt{2}}
	\left(\ket{1,0}+i\ket{0,1}\right)
	\\
	\hat{U}_\text{BS}
	\ket{0,1}
	&=
	\frac{1}{\sqrt{2}}
	\left(i\ket{1,0}+\ket{0,1}\right)
	.
\end{align}
Then, the first beam splitter BS1 in \Cref{fig:mzi_asym} (instantly) splits a photon pulse $\ket{t_0}$ arriving at $t_0$ into the superposition
\begin{equation}
	\hat{U}_\text{BS}
	\ket{t_0,0}
	=
	\frac{1}{\sqrt{2}}
	\left(\ket{t_0,0}+i\ket{0,t_0}\right)
\end{equation}
where the first mode corresponds to the upper and the second mode to the lower optical path in \Cref{fig:mzi_asym}.
The phase shifter adds a relative phase of $\varphi$ between the upper and lower path and the input state to the second beam splitter BS2 is
\begin{equation}
	\hat{U}_\text{PS}
	\hat{U}_\text{BS}
	\ket{t_0,0}
	=
	\frac{1}{\sqrt{2}}
	\left(
		\ket{t_0+\tau,0}
		+
		ie^{i\varphi}
		\ket{0,t_0+\tau+\Delta\tau}
	\right)
\end{equation}
wherein $\tau$ is the time delay the pulse accumulates over the short upper path and $\Delta\tau$ is the difference in time delay between the shorter, upper and longer, lower path.
The output state of BS2 is equal to the action of the \gls{mzi}
\begin{equation}
	\begin{split}
		\hat{U}_\text{MZM}
		\ket{t_0,0}
		&=
		\hat{U}_\text{BS}
		\hat{U}_\text{PS}
		\hat{U}_\text{BS}
		\ket{t_0,0}
		\\
		&=
		\frac{1}{2}
		\biggl[
			\left(
				\ket{t_0+\tau,0}
				+
				i\ket{0,t_0+\tau}
			\right)
			+
			ie^{i\varphi}
			\left(
				i\ket{t_0+\tau+\Delta\tau,0}
				+
				\ket{0,t_0+\tau+\Delta\tau}
			\right)
		\biggr]
		\\
		&=
		\frac{1}{2}
		\biggl[
			\ket{t_0+\tau,0}
			-
			e^{i\varphi}
			\ket{t_0+\tau+\Delta\tau,0}
			+
			i
			\left(
				\ket{0,t_0+\tau}
				+
				e^{i\varphi}
				\ket{0,t_0+\tau+\Delta\tau}
			\right)
		\biggr]
		.
	\end{split}
	\label{eq:mzi_asym}
\end{equation}

Back to the time-phase-encoding BB84 setup depicted in \Cref{fig:time_phase_encoding_bb84}, we note that Alice's transmitter consists of a single-photon source and an asymmetric \gls{mzi} where on output is dumped.
Therefore, Alice's states are parametrized by the relative phase $\theta$,
\begin{equation}
	\ket{t_0,\theta}
	=
	\frac{1}{\sqrt{2}}
	\left(
		\ket{t_0}
		-
		e^{i\theta}
		\ket{t_0+\Delta\tau}
	\right)
	,
\end{equation}
which is obtained from \cref{eq:mzi_asym} by absorbing the time delay of the shorter path $\tau$ into the time reference $t_0$ and projecting the first output mode of the \gls{mzi}.
By adding a time delay to the states of $\Delta\tau$, we receive the states for the time-delayed signal
\begin{align}
	\ket{t_1,\phi}_1
	&=
	\frac{1}{\sqrt{2}}
	\left(
		\ket{t_1}
		-
		e^{i\phi}
		\ket{t_1+\Delta\tau}
	\right)
	\\
	\ket{t_1,\phi}_2
	&=
	\frac{i}{\sqrt{2}}
	\left(
		\ket{t_1}
		+
		e^{i\phi}
		\ket{t_1+\Delta\tau}
	\right)
	\label{eq:tp_bb84_bob_nodelay}
\end{align}
where we again choose the time $t_1$ such that it cancels the time delay of the short path $\tau$.
If Bob receives a pulse with time delay $\Delta\tau$ at some time $t_1$, i.e., $\ket{t_1+\Delta\tau}$, then his \gls{mzi} provides the two detectors with the states
\begin{align}
	\ket{t_1+\Delta\tau,\phi}_1
	&=
	\frac{1}{\sqrt{2}}
	\left(
		\ket{t_1+\Delta\tau}
		-
		e^{i\phi}
		\ket{t_1+2\Delta\tau}
	\right)
	\\
	\ket{t_1+\Delta\tau,\phi}_2
	&=
	\frac{i}{\sqrt{2}}
	\left(
		\ket{t_1+\Delta\tau}
		+
		e^{i\phi}
		\ket{t_1+2\Delta\tau}
	\right)
	.
	\label{eq:tp_bb84_bob_delay}
\end{align}
We note that these are superpositions of states at three different time instances $0,\Delta\tau,2\Delta\tau$.
Alice's state is a superposition of \cref{eq:tp_bb84_bob_nodelay} and \cref{eq:tp_bb84_bob_delay}, so Bob's detectors will receive the states
\begin{align}
	\ket{\theta,\phi}_1
	&=
	\frac{1}{2}
	\left[
		\ket{\Delta\tau=0}
		-
		\left(
			e^{i\phi}
			+
			e^{i\theta}
		\right)
		\ket{\Delta\tau=1}
		+
		e^{i(\phi+\theta)}
		\ket{\Delta\tau=2}
	\right]
	\\
	\ket{\theta,\phi}_2
	&=
	\frac{i}{2}
	\left[
		\ket{\Delta\tau=0}
		+
		\left(
			e^{i\phi}
			-
			e^{i\theta}
		\right)
		\ket{\Delta\tau=1}
		-
		e^{i(\phi+\theta)}
		\ket{\Delta\tau=2}
	\right]
	\label{eq:tp_bb84_bob}
\end{align}
where we dropped the pulse time.
The no time delay and twice time delay states, $\ket{\Delta\tau=0}$ respective $\ket{\Delta\tau=2}$ occur with constant probability $1/4$ and thereby provide no information about the received state.
However, the superposition of one time delay states $\ket{\Delta\tau=1}$ have probability amplitudes $(e^{i\phi}\pm e^{i\theta})/2$.
We define the projector
\begin{equation}
	\hat{P}_{\Delta\tau=1}
	=
	\ketbra{\Delta\tau=1}
\end{equation}
and find the probability for a click at $\Delta\tau=1$ to be
\begin{align}
	p_1
	&=
	\trace{\hat\rho_1\hat{P}_{\Delta\tau=1}}
	=
	\frac{1}{2}
	\left(1+\cos(\theta-\phi)\right)
	\\
	p_2
	&=
	\trace{\hat\rho_2\hat{P}_{\Delta\tau=1}}
	=
	\frac{1}{2}
	\left(1-\textcolor{red}{\cos(\theta-\phi)}\right)
\end{align}
where we used the density matrices $\hat\rho_i={}_i\ketbra{\theta,\phi}_i$.
Restricting Alice's possible phases to $\theta=0,\pi,\pi/2,3\pi/2$ and Bob's possible phases to $0,\pi/2$, we find the click probabilities of the detectors as presented in \Cref{tab:tp_bb84_probabilities}.
\begin{table}[htb]
	\centering
	\begin{tabular}{ccccc}
		\toprule
		\multicolumn{3}{c}{Phase} & \multicolumn{2}{c}{Detector click probability} \\
		\cmidrule{1-3}
		\cmidrule{4-5}
		$\theta$ & $\phi$ & $\theta-\phi$ & $p_i(\theta-\phi)$ & $p_i(\theta-\phi)$ \\
		\midrule
		\multirow{2}{*}{$0$} & $0$ & $0$ & \SI{100}{\percent} & \SI{0}{\percent} \\
		& $\frac{\pi}{2}$ & $-\frac{\pi}{2}$ & \SI{50}{\percent} & \SI{50}{\percent} \\
		\cmidrule{1-3}
		\multirow{2}{*}{$\pi$} & $0$ & $\pi$ & \SI{0}{\percent} & \SI{100}{\percent} \\
		& $\frac{\pi}{2}$ & $\frac{\pi}{2}$ & \SI{50}{\percent} & \SI{50}{\percent} \\
		\cmidrule{1-3}
		\multirow{2}{*}{$\frac{\pi}{2}$} & $0$ & $\frac{\pi}{2}$ & \SI{50}{\percent} & \SI{50}{\percent} \\
		& $\frac{\pi}{2}$ & $0$ & \SI{100}{\percent} & \SI{0}{\percent} \\
		\cmidrule{1-3}
		\multirow{2}{*}{$\frac{3\pi}{2}$} & $0$ & $\frac{3\pi}{2}$ & \SI{50}{\percent} & \SI{50}{\percent} \\
		& $\frac{\pi}{2}$ & $\pi$ & \SI{0}{\percent} & \SI{100}{\percent} \\
		\bottomrule
	\end{tabular}
	\caption{Foobar}\label{tab:tp_bb84_probabilities}
\end{table}

\subsection{Equivalence between polarization- and time-phase-encoding}

		\section{Continuous-variable}

% citations:
% \cite{Scarani2009} The Security of Practical QKD
% \cite{Weedbrook2012,Ferraro2005} Gaussian quantum information (quantum channel, cv-qkd, measurements)
% \cite{Renner2009} - overview attacks, reduction of coherent attacks to collective attacks
% \cite{Grosshans2002} - first proposal of CV-QKD?
% \cite{Diamanti2016} - practical challenges for QKD / outlook (photonic integrationo, MIDI) and why DV-QKD is more practical
% \cite{Lo2014} Introduction to security proof of QKD (quantum hacking)
% \cite{Diamanti2015} security analysis of CV-QKD
% \cite{Lodewyck2007} complete description of CV-QKD device (theoretical and experimental)
% \cite{Laudenbach2018} notions of secrutiy, key rate, noise and detection models, parameter estimation
% \cite{Fung2010} security analysis of post-processing
% \cite{Renner2005} security of privacy amplification

% cv-qkd
% advantages of CV-QKD compared to DV-QKD
% protocol table
% equation for correlation between random variables (with figure of two gaussians)

\subsection{Phase-encoding}
\begin{figure}[htb]
	\centering
	\includestandalone{figures/pstricks/phase-encoding-cvqkd}
	\caption{Fiber-optical setup of the phase-encoding \gls{cvqkd} protocol:}
\end{figure}

\subsection{Amplitude-phase-encoding}

% dual homodyne
% tensor product input state

		\addcontentsline{toc}{section}{References}
		\printbibliography[title=References]
	\end{refsection}

	\chapter{Quantum field theory of light}
	\begin{refsection}
		\chapter*{Introduction}
\addcontentsline{toc}{chapter}{Introduction}

Optical communication enables humanity worldwide to share information in a split second, with companies like Huawei undergoing tremendous efforts to advance the frontiers.
In addition to incremental innovation increasing the performance and decreasing the cost of optical communication technology, we observe intensified activities towards disruptive innovations that challenge our present understanding of communication.
One such branch of activity is quantum optical communication, incorporating quantum aspects of light into classical communication and leading to novel communication technology like \gls{qkd}, which enables practical and secure key generation.
As a still young discipline, which emerged from two highly advanced fields, communication engineering and quantum physics, quantum communication lacks a unified description to which both communication engineers and quantum physicists agree.
The present thesis aims to resolve the seeming discrepancies between communication engineering and quantum physics by reviewing a practical implementation of a quantum communication system implementing a \gls{qkd} protocol.
In the process, we hope to develop a theoretical framework for quantum optical communication, incorporating quantum effects into classical communication, which has applicability beyond \gls{qkd}.

\subsection*{Problem statement}

To raise awareness of the challenges ahead, we review the best-known quantum theory of light, single-mode quantum optics, along with central ideas from classical communication and outline where these pictures conflict.

In single-mode quantum optics, we model monochromatic light with frequency $\omega_0$ as a quantum harmonic oscillator with unit mass, $m=1$, and Hamiltonian~\cite{Gerry2005,Fox2006}
\begin{equation}
	\hat{H}
	=
	\omega_0
	\hat{a}^\dagger
	\hat{a}
	,
\end{equation}
wherein $\hat{a}$ and $\hat{a}^\dagger$ are the quantum annihilation and creation operators, destroying or creating an excitation or "mode" of frequency $\omega_0$.
The electric field operator,
\begin{equation}
	\hat{E}(t,x)
	=
	\mathcal{E}_0
	\left(
		\hat{a}
		+
		\hat{a}^\dagger
	\right)
	\sin(\omega_0x)
	,
\end{equation}
wherein $\mathcal{E}_0$ has the interpretation of an electric field density, establishes the connection between the quantum harmonic oscillator and electromagnetic radiation, including light~\cite[p.~12]{Gerry2005}.
Two of the most important quantum states are the number and the coherent state,
\begin{align}
	\ket{n}
	&=
	\frac{1}{\sqrt{n!}}
	\left(\hat{a}^\dagger\right)^n
	\ket{0}
	&
	\ket{\alpha}
	&=
	\exp\left(-\frac{1}{2}\abs{\alpha}^2\right)
	\sum_{n=0}^\infty
	\frac{\alpha^n}{\sqrt{n!}}
	\ket{n}
	.
\end{align}
The number state is parametrized by a natural number $n\in\mathbb{N}_0$ counting the number excitations.
The coherent state is parametrized by a complex number $\alpha\in\mathbb{C}$ encoding amplitude and phase.
The expectation value of the electric field operator with respect to a coherent state,
\begin{equation}
	\bra{\alpha}
	\hat{E}(t)
	\ket{\alpha}
	=
	\sqrt{2}
	\abs{\alpha}
	\mathcal{E}_0
	\sin\left(\omega_0t-\theta\right)
	,
\end{equation}
equals a classical monochromatic wave with amplitude proportional to $\abs{\alpha}$ and phase $\theta$~\cite[p.~45]{Gerry2005}.

In communication engineering, light is primarily a means of transmitting signals that appear as frequency bands centered around an optical carrier frequency $\omega_c$, as illustrated in \Cref{fig:signal_spectrum}.
\begin{figure}[ht]
	\centering
	\includegraphics{figures/pgfplots/signal-spectrum}
	\caption{Receiver spectrum comprising multiple signal bands relative to a carrier frequency at $\omega=0$. At \SI{+100}{\mega\hertz}, the spectrum has a pilot tone broadened by phase noise. Centered at \SI{-25}{\mega\hertz}, the spectrum contains a first signal band with \SI{12.5}{\mega\hertz} bandwidth. Centered at \SI{-168.75}{\mega\hertz}, the spectrum contains a second signal band with \SI{12.5}{\mega\hertz} bandwidth. The remaining segments of the spectrum include mirror bands or disturbances.}\label{fig:signal_spectrum}
\end{figure}
The concept of frequency bands is extremely powerful as it allows the transmission of multiple independent signals around one carrier frequency.
More formally, we need to distinguish between base- and passband signals.
For a baseband signal $x_b(t)$, the signal power outside of the signal's bandwidth $B$ is negligible~\cite[p.~15]{Madhow2008}, i.e.,
\begin{align}
	\abs{x_b(\omega)}^2
	&\approx
	0
	&
	\abs{\omega}
	&>
	B/2
	.
\end{align}
For a passband signal $x_p(t)$, the signal power outside the signal's band around a carrier frequency $\omega_c$ is negligible~\cite[p.~16]{Madhow2008}, i.e.,
\begin{align}
	\abs{x_p(\omega)}^2
	&\approx
	0
	&
	\abs{\omega\pm\omega_c}
	&>
	B/2
	.	
\end{align}
\begin{figure}[ht]
	\centering
	\includegraphics{figures/tikz/up-conversion}
	\caption{Power spectrum illustrating up-conversion of a real-valued passband signal with bandwidth $B$ centered at $\omega_0$. Up-conversion by $\omega_c$ shifts the passband to $\omega_c+\omega_0$ and creates a mirror band at $\omega_c-\omega_0$.}\label{fig:up_conversion}
\end{figure}
A baseband signal can be up-converted to a passband signal at carrier frequency $\omega_c$ by shifting the spectrum by $\omega_c$ is known as up-conversion, see \Cref{fig:up_conversion}, and implemented by modulation.
Similar, a passband signal at carrier frequency $\omega_c$ is down-converted to a baseband signal by demodulation~\cite[p.~26]{Madhow2008}.

To sum up, single-mode quantum optics provides precise physical meaning to light, including quantum effects, although limited to monochromatic light.
On the other side, communication engineering provides a framework for efficiently constructing and transmitting signals.
For quantum optical communication, it is inevitable to welcome and incorporate both views.
For instance, people with a background in quantum optics but foreign to communication engineering often advocate the concept of "one state, one universe", where each quantum transmission is completely independent.
However, if we include practical considerations, like assuming a single transmission line, the picture of "one state, one universe" is plagued by several ambiguities.
For example, a single-mode quantum state has a single well-defined frequency $\omega_0$, which by Fourier uncertainty implies infinite temporal duration but makes information transmission absurd.
The typical counter-argument is that single-mode quantum optics implicitly assumes pulses with $\omega_0$ being the center frequency of the pulse.
While the counter-argument is technically valid, we must admit that it only raises new questions, such as bandwidth-limitations on the pulse parameters, all properly addressed in communication engineering.

The multi-mode quantum optics mentioned in popular quantum optics books~\cite{Gerry2005,Fox2006} are insufficient to represent continuous-time signals, and performing a continuum limit might not be correct if we consider the huge differences between linear algebra and functional analysis.
The advanced quantum optics literature~\cite{Vogel2006,Mandel1995} does sometimes use a continuous-mode formalism but does not explicitly investigate its properties.
We are only aware of two books~\cite{Loudon2000,Barnett2002} that explicitly present a continuous-mode theory of light but again open up new questions regarding the fundamental assumptions and justification thereof.
If we are willing to go one step deeper, we find answers in the quantum field theory literature~\cite{Peskin1995,Srednicki2007,Greiner2013,Itzykson2012}, but it is up to us to transfer these insights from particle physics to quantum optics applications.
We even have to go a bit deeper and look into mathematical quantum field theory~\cite{Streater2016,Bogoliubov1982,Bogolubov1989} to answer some questions.
Finally, we want to understand and upgrade quantum models of (electro-)optical components in the literature~\cite{Vogel2006,Leonhardt2003,Haroche2006,Mandel1995} to a mode continuum for comparison with the results from the optical communication community~\cite{Shapiro2009,Kikuchi2016}.

\subsection*{Thesis outline}

Our work is divided into four chapters.
In \Cref{ch:qkd}, we present an introduction to \gls{qkd}, emphasizing the similarities between the plethora of seemingly different protocols and attempting to argue why practical \gls{qkd} based on weak coherent states is effectively a coherent state communication system.
In the following three chapters, we construct our theoretical framework for quantum optical communication towards practical \gls{qkd}, starting from a general quantum theory of light, \Cref{ch:light}, over applying the quantum theory to describe the building blocks of coherent communication systems, \Cref{ch:components}, to an abstract description of a coherent state transmission system's signal-processing, \Cref{ch:system}.
While the thesis chapter structure supports a bottom-up approach, it is equally possible to read the thesis from the back to the front, revealing more and more details.
Likewise, it is possible to skip certain chapters and pare down to the chapter summary at the end of each chapter.
		\section{Maxwell field}

\subsection{Relativistic field theory}

The Maxwell Lagrangian with an external source $J^\mu(t,\vb{x})$ is
\begin{equation}
	\begin{split}
		\mathcal{L}
		&=
		-
		\frac{1}{4}
		F_{\mu\nu}
		F^{\mu\nu}
		+
		A_\mu J^\mu
		\\
		&=
		\frac{1}{2}
		\left(\partial_\mu A_\nu\right)
		\left(\partial^\mu A^\nu-\partial^\nu A^\mu\right)
		+
		A_\mu J^\mu
	\end{split}
	\label{eq:mw_lagrangian}
\end{equation}
with the relativistic Euler-Lagrange equation yielding
\begin{equation}
	0
	=
	\partial_\mu\pdv{\mathcal{L}}{(\partial_\mu A_\nu)}
	-
	\pdv{\mathcal{L}}{A_\nu}
	=
	\partial_\mu\partial^\mu A^\nu
	-
	\partial^\nu\partial_\mu A^\mu
	-
	J^\nu	
	\label{eq:mw_eom}	
\end{equation}

\subsection{Coulomb gauge}

The Maxwell Lagrangian is invariant under local gauge transformations
\begin{equation}
	A_\mu(t,\vb{x})
	\to
	A_\mu^\prime(t,\vb{x})
	=
	A_\mu(t,\vb{x})
	+
	\partial_\mu\Lambda(t,\vb{x})
	\label{eq:mw_local_gauge_transform}
\end{equation}
where $\Lambda(t,\vb{x})$ is a local gauge field.
For instance, the physical field-strength tensor transforms under \cref{eq:mw_local_gauge_transform} as
\begin{equation}
	\begin{split}
		F_{\mu\nu}
		\to
		F_{\mu\nu}^\prime
		&=
		\partial_\mu\left(A_\nu+\partial_\nu\Lambda\right)
		-
		\partial_\nu\left(A_\mu+\partial_\mu\Lambda\right)
		\\
		&=
		F_{\mu\nu}
		+
		\partial_\mu\partial_\nu\Lambda
		-
		\partial_\nu\partial_\mu\Lambda
		=
		F_{\mu\nu}
	\end{split}
	\label{eq:mw_field_strength_gauge_transform}.
\end{equation}
The invariance of the Maxwell field under local gauge transformation is used to remove a degree of freedom from the field using a gauge condition.
The most popular gauge conditions are the Lorentz gauge $\partial_\mu A^\mu=0$ and the Coulomb gauge $\div\vb{A}=0$.
While the Lorentz gauge is manifest Lorentz invariant it suffers from unphysical scalar and longitudinal polarization states destroying unitarity.
The Coulomb gauge is manifest unitarity but has to be imposed in every reference frame.
As Lorentz boosts are not of interest for us, we will adapt the Coulomb gauge condition.
The Coulomb gauge leaves a residual gauge freedom which allows us to choose $A_0=0$.\footnote{If external static sources are present, we need to be more careful with the residual gauge fixing.}

Applying the Coulomb gauge
\begin{align}
	\div\vb{A}(t,\vb{x})
	=
	0
	&&
	A_0
	=
	0
	\label{eq:mw_coulomb_gauge}
\end{align}
to the free equation of motion \cref{eq:mw_eom} yields the relativistic wave equation
\begin{equation}
	0
	=
	\partial_\mu\partial^\mu
	\vb{A}(t,\vb{x})
	=
	\partial_t^2
	\vb{A}(t,\vb{x})
	-
	\grad^2
	\vb{A}(t,\vb{x})
	\label{eq:mw_relatistic_wave}
\end{equation}
which is solved by plane-waves satisfying the massless dispersion relation
\begin{equation}
	\omega(\vb{p})
	=
	\norm{\vb{p}}
\end{equation}

\subsection{Mode decomposition}

As with the Klein-Gordon field, we start with the four-dimensional Fourier transform of $A^\mu(t,\vb{x})$, insert it into the free equations of motion ($J^\mu=0$), and perform the mode decomposition
\begin{equation}
	\vb{A}(t,\vb{x})
	=
	\sum_{\lambda=1,2}
	\int_{\mathbb{R}^3}\frac{}{(2\pi)^3\sqrt{2\omega(\vb{p})}}
	\left\{
		a_\lambda(\vb{p})
		\vb{\epsilon}_\lambda(\vb{p})
		e^{-ip_\mu x^\mu}
		+
		\text{c.c.}
	\right\}
	\label{eq:mw_ft}
\end{equation}
where we defined $a_\lambda(\vb{p})\vb{\epsilon}_\lambda(\vb{p})=\vb{A}(\omega(\vb{p}),\vb{p})$ and $\vb{\epsilon}_\lambda(\vb{p})$ denotes the polarization vectors.
For the mode decomposition to satisfy the Coulomb gauge, the polarization vectors $\vb{\epsilon}_\lambda(\vb{p})$ need to be orthogonal to the wave vector $\vb{p}$, i.e.,
\begin{equation}
	\vb{p}\vdot\vb{\epsilon}_\lambda(\vb{p})
	=
	0	
\end{equation}
Furthermore, we require the $\vb{p}/\norm{\vb{p}},\vb{\epsilon}_1(\vb{p}),\vb{\epsilon}_2(\vb{p})$ to form a orthonormal basis
\begin{equation}
	\vb{\epsilon}_i(\vb{p})
	\vdot
	\vb{\epsilon}_j(\vb{p})
	=
	\delta_{ij}
\end{equation}

\subsection{Maxwell equations}

The covariant inhomogeneous Maxwell equations are obtained from the equations of motion
\begin{equation}
	J^\nu
	=
	\partial_\mu F^{\mu\nu}
	=
	\partial_\mu\partial^\mu A^\nu
	-
	\partial^\nu\partial_\mu A^\mu
	\label{eq:mw_inhomo},
\end{equation}
and the covariant homogeneous Maxwell equations are a consequence of the Bianchi identity
\begin{equation}
	0
	=
	\partial_\mu\tilde{F}^{\mu\nu}
	\label{eq:mw_homo}
\end{equation}
where we defined $\tilde{F}^{\mu\nu}=\frac{1}{2}\varepsilon^{\mu\nu\alpha\beta}F_{\alpha\beta}$.

The components of the field-strength tensor $F^{\mu\nu}$ relate to the electromagnetic field components via
\begin{align}
	F^{0i}
	=
	-E^i
	&&
	F^{ij}
	=
	-\varepsilon^{ijk}B_k
	\label{eq:mw_em_components}.
\end{align}
Evaluating the time component of \cref{eq:mw_homo} yields the Gauss' law for magnetism
\begin{equation}
	\begin{split}
		0
		=
		\varepsilon_{0\lambda\mu\nu}\partial^\lambda F^{\mu\nu}
		&=
		\varepsilon_{0ijk}\partial^iF^{jk}
		\\
		&=
		-
		\varepsilon_{ijk}\varepsilon_{ljk}
		\partial^i B_l
		=
		2\partial_iB^i
	\end{split}
	\label{eq:mw_gauss_law_mag}
\end{equation}
and the spatial component yields Ampere's circuit law
\begin{equation}
	\begin{split}
		0
		=
		\varepsilon_{i\lambda\mu\nu}
		\partial^\lambda
		F^{\mu\nu}
		&=
		-
		\varepsilon_{ijk}
		\varepsilon^{ljk}
		\partial_t B_l
		-
		2\varepsilon_{ijk}
		\partial^jE^k
		\\
		&=
		\partial_tB_i
		+
		\varepsilon_{ijk}
		\partial^jE_k
	\end{split}
	\label{eq:mw_ampere_law}.
\end{equation}
The time component of the inhomogeneous covariant Maxwell equation \cref{eq:mw_inhomo} yields Gauss' law
\begin{equation}
	J^0
	=
	\rho
	=
	\partial_\mu F^{\mu\nu}
	=
	\partial_i E^i
	\label{eq:mw_gauss_law},
\end{equation}
and the spatial component yields Faraday's law of induction
\begin{equation}
	J^i
	=
	\partial_\mu F^{\mu i}
	=
	-\partial_t E^i
	+\varepsilon^{ijk}\partial_j B_k
	\label{eq:mw_faraday_law}.
\end{equation}
and we derived the vector Maxwell equations from first principles
\begin{align}
	\div\vb{E}
	=
	\rho
	&&
	\div\vb{B}
	=
	0
	\label{eq:mw_homo_vec}
	\\
	\curl\vb{E}
	=
	-
	\partial_t\vb{B}
	&&
	\curl\vb{B}
	=
	\vb{J}
	+
	\partial_t\vb{E}
	\label{eq:mw_inhomo_vec}
\end{align}

\subsection{Canonical quantization}

\subsection{Single-particle and coherent states}

\subsection{Electromagnetic field operators}

		\section{Quantum states}

In the previous section, we derived the relevant field operators for the Maxwell field encoding electromagnetism.
In the present section, we construct quantum states from these operators in a rather axiomatic approach as done in Ref.~\cite[p.~506]{Cohen2019} for the quantum harmonic oscillator, or in axiomatic field theory~\cite{Streater2016,Haag2012,Bogolubov1989}.

To keep the arguments and notation concise, we restrict the quantum state construction to one polarization mode of the Maxwell field.
The Maxwell field then effectively becomes a Klein-Gordon field with field operator
\begin{equation}
	\hat{A}(t,\vb{x})
	=
	\int\frac{\dd[3]{p}}{(2\pi)^3\sqrt{2\omega(\vb{p})}}
	\left\{
		\hat{a}(\vb{p})
		e^{-i\omega(\vb{p})t+i\vb{p}\vdot\vb{x}}
		+
		\hat{a}^\dagger(\vb{p})
		e^{+i\omega(\vb{p})t-i\vb{p}\vdot\vb{x}}
	\right\}
	\label{eq:klein_gordon_operator}
\end{equation}
wherein the annihilation and creation operator, $\hat{a}(\vb{p}),\hat{a}^\dagger(\vb{p})$, satisfy the \gls{ccr}
\begin{align}
	\comm{\hat{a}(\vb{p})}{\hat{a}^\dagger(\vb{q})}
	&=
	(2\pi)^3\delta^{(3)}
	\left(\vb{p}-\vb{q}\right)
	\\
	\comm{\hat{a}(\vb{p})}{\hat{a}(\vb{q})}
	&=
	\comm{\hat{a}^\dagger(\vb{p})}{\hat{a}^\dagger(\vb{q})}
	=
	0
	\label{eq:klein_gordon_canonical_commutation_relation}.
\end{align}
To extend the results back to two polarization modes, we can construct a two-dimensional tensor product space from the single polarization mode.

\subsection{Vacuum state}

The fundamental assumption our state construction relies upon is the existence of a unique, up to a constant phase factor, vacuum state $\ket{0}$ invariant under the unitary Poincaré transformation~\cite[p.~97]{Streater2016}
\begin{equation}
	\hat{U}(a,\Lambda)
	\ket{0}
	=
	\ket{0}
	\label{eq:vacuum_invariance}
\end{equation}
where $\Lambda$ denotes a Lorentz transformation and $a$ a spacetime translation.
The vacuum state is an element of a one-dimensional complex Hilbert space, $\mathcal{H}^{(0)}=\mathcal{H}(\mathbb{C})$, the zero-particle state space.
The generator of the unitary spacetime translation is the four-momentum operator $\hat{P}^\mu=\left(\hat{H},\vu{P}\right)$~\cite[p.~28]{Haag2012}
\begin{equation}
	\hat{U}(a)
	=
	\hat{U}(a,\mathbb{1})
	=
	e^{i\hat{P}_\mu a^\mu}
	\label{eq:spacetime_translation}
	.
\end{equation}
The invariance of the vacuum under spacetime translations, \cref{eq:spacetime_translation}, implies that the vacuum is a zero eigenstate to the Hamilton and momentum operator
\begin{align}
	\hat{H}
	\ket{0}
	&=
	0
	&
	\vu{P}
	\ket{0}
	&=
	\vb{0}
	\label{eq:vacuum_energy_momentum}
	.	
\end{align}
From the mode expansion of the Hamilton operator, \cref{eq:maxwell_hamilton_operator}, and the vacuum state being a zero eigenstate to the Hamilton operator, we conclude that the vacuum state is also a zero eigenstate to the annihilation operator
\begin{equation}
	\hat{a}(\vb{p})
	\ket{0}
	=
	0
	\label{eq:vacuum_annihilation}
	.
\end{equation}
In the next paragraph, we motivate why we understand the annihilation and creation operators as adding or removing a particle excitation to and from the field.
Under these circumstances, we can read \cref{eq:vacuum_annihilation} as destroying the vacuum state, ensuring no negative energy or negative particle number states.

\subsection{Particle states}

The motivation of why the annihilation and creation operators add or remove a particle with energy and momentum to and from the field follows from applying the commutators of the number and momentum operator with the creation operator
\begin{align}
	\comm{\hat{N}}{\hat{a}^\dagger(\vb{p})}
	&=
	\hat{a}^\dagger(\vb{p})
	&
	\comm{\vu{P}}{\hat{a}^\dagger(\vb{p})}
	&=
	\vb{p}
	\hat{a}^\dagger(\vb{p})
	.
\end{align}
Applying the vacuum state $\ket{0}$ to the right of the commutator equations, yields eigenvalue equations for the number and momentum operator
\begin{align}
	\hat{N}
	\hat{a}^\dagger(\vb{p})
	\ket{0}
	&=
	1
	\hat{a}^\dagger(\vb{p})
	\ket{0}
	&
	\vu{P}
	\hat{a}^\dagger(\vb{p})
	\ket{0}
	&=
	\vb{p}
	\hat{a}^\dagger(\vb{p})
	\ket{0}
	.
\end{align}
The eigenvalue equations suggest
\begin{equation}
	\ket{\vb{p}}
	=
	\hat{a}^\dagger(\vb{p})
	\ket{0}
	\label{eq:momentum_state}
\end{equation}
to be a single-particle state with momentum $\vb{p}$ and energy $\omega(\vb{p})$, a momentum state.~\cite[p.~23]{Peskin1995}.
Unfortunately, the inner product between two momentum states does not yield a complex number $\mathbb{C}$ but, a distribution,
\begin{equation}
	\braket{\vb{p}}{\vb{q}}
	=
	\bra{0}
	\comm{\hat{a}(\vb{p})}{\hat{a}^\dagger(\vb{q})}
	\ket{0}
	=
	(2\pi)^3
	\delta^{(3)}(\vb{p}-\vb{q})
\end{equation}
suggesting that something essential is missing in our description.

It makes sense to take a step back and recap some mathematical context regarding distributions.\footnote{See Ref.~\cite[p.~590]{Zeidler2016} and Ref.~\cite[p.~193]{Mukhanov2007} for a mathematical discussion of distributions in a physical context.}
One approach considers distributions as functionals, i.e., maps from a function space, e.g., the space of real-valued square-integrable functions $L^2(\mathbb{R})$, to real numbers $\mathbb{R}$.
Implicitly, we already used functionals when we considered the action integral
\begin{equation}
	\hat{S}\left[x(t)\right]
	=
	\int_{t_0}^{t_1}\dd{t}
	L\left(x(t),\dot{x}(t)\right)
	,
\end{equation}
wherein $L$ is some classical Lagrangian, evaluated for some finite time interval $[t_0,t_1]$ maps the trajectory $x(t)$ of a point particle to a real number $\mathbb{R}$.
Often a functional $A$ acting on a function $f$ is written
\begin{equation}
	A[f]
	=
	\int\dd{x}
	f(x)A(x)
	\label{eq:functional}
\end{equation}
wherein $A(x)$ is denoted the integration kernel representing the functional, which may be an ordinary function or a distribution.
For example, the delta distribution $\delta(x-y)$ is the integration kernel of the functional $\delta_y$
\begin{equation}
	\delta_{y}[f]
	=
	\int\dd{x}
	f(x)
	\delta(x-y)
	=
	f(y)
	.
\end{equation}
Fourier transforms are another class of functionals we already used frequently.
Linear functionals share many convenient properties with ordinary functions and physicists often skip the distinction.
In axiomatic quantum field theory, the quantum field operators are precisely defined as operator-valued tempered distributions mapping from the space of smearing or test functions $\mathcal{S}(\mathbb{R}^4)$ to the the set of operators defined on the corresponding Hilbert space $\mathcal{O}(\mathcal{H})$~\cite[p.~56]{Haag2012}.
A typical functional space for smearing functions is the Schwartz space, a subset of the space of square-integrable functions $L^2$, which rapidly fall off at infinity, a property which we exploit for partial integration with vanishing boundary terms of the action integral.\footnote{Typical Schwartz functions are Gaussian functions multiplied with a monomial, e.g., $x^ne^{-a\norm{x}^2}$ where $n\in\mathbb{N}_0$ and $a\in\mathbb{R}_+$.}

Let us reinterpret the (positive frequency) field operator with this mathematical background by considering its action on a smearing function, i.e.,
\begin{equation}
	\hat{A}^{(+)}[f]
	=
	\int\dd[4]{x}
	f(t,\vb{x})
	\hat{A}^{(+)}(t,\vb{x})
	=
	\int\frac{\dd[3]{p}}{(2\pi)^3\sqrt{2\omega(\vb{p})}}
	f\left(\omega(\vb{p}),\vb{p}\right)
	\hat{a}^\dagger(\vb{p})
\end{equation}
where we inserted the plane-wave expansion for the field and the spacetime Fourier transform for the smearing function and used the orthogonality of the Fourier modes in the second equation.
Applying the smeared positive frequency field operator to the vacuum state,
\begin{equation}
	\begin{split}
		\ket{1_f}
		=
		\hat{A}^{(+)}[f]
		\ket{0}
		,
	\end{split}
	\label{eq:single_particle_state}
\end{equation}
and comparing the result to the momentum state, \cref{eq:momentum_state}, we find the function $f$ to smear the function in momentum and spacetime space~\cite[p.~35]{Srednicki2007}.\footnote{Interestingly though, only momentum components satisfying the energy-momentum relation, \cref{eq:energy_momentum_relation}, contribute to momentum space.}
The inner product of two such smeared states yields a complex number $\mathbb{C}$
\begin{equation}
	\braket{1_f}{1_g}
	=
	\int\frac{\dd[3]{p}}{(2\pi)^32\omega(\vb{q})}
	f(\vb{p})^*
	g(\vb{p})
\end{equation}
implying the smeared state $\ket{1_f}$ being normalizable if we require the smearing function to satisfy
\begin{equation}
	\braket{1_f}{1_f}
	=
	\int\frac{\dd[3]{p}}{(2\pi)^32\omega(\vb{q})}
	\abs*{f(\vb{p})}^2
	=
	1
	\label{eq:spectrum_normalization}
	.
\end{equation}
With the normalization condition imposed, the smeared state $\ket{1_f}$ is an eigenstate of the number operator to eigenvalue one
\begin{equation}
	\hat{N}
	\ket{1_f}
	=
	1
	\ket{1_f}
\end{equation}
suggesting the smeared state $\ket{1_f}$ to be the physical single-particle state we were looking for.
The smeared state $\ket{1_f}$ is not an eigenstate of the energy and momentum operator anymore but has expectation values
\begin{align}
	\bra{1_f}
	\hat{H}
	\ket{1_f}
	&=
	\int\frac{\dd[3]{p}}{(2\pi)^3}
	\omega(\vb{p})
	\abs*{f\left(\omega(\vb{p}),\vb{p}\right)}^2
	\\
	\bra{1_f}
	\vu{P}
	\ket{1_f}
	&=
	\int\frac{\dd[3]{p}}{(2\pi)^3}
	\vb{p}
	\abs*{f\left(\omega(\vb{p}),\vb{p}\right)}^2
\end{align}
suggesting that the smearing function has the physical interpretation of a frequency spectrum.

Let $\ket{1_f}$ be a smeared particle state, then the wave function $\Psi(t,\vb{x})$ is the probability amplitude of finding the particle at a spacetime coordinate $(t,\vb{x})$~\cite[p.~24]{Peskin1995}, i.e.,
\begin{equation}
	\Psi(t,\vb{x})
	=
	\bra{0}
	\hat{A}(t,\vb{x})
	\ket{1_f}
	=
	\int\dd{t^\prime}\dd[3]{x^\prime}
	D(t-t^\prime,\vb{x}-\vb{x}^\prime)
	f(t^\prime,\vb{x}^\prime)
\end{equation}
wherein $f(t,\vb{x})$ is the spacetime representation of the (initial) smearing function and
\begin{equation}
	D(t,\vb{x})
	=
	\int\frac{\dd[3]{p}}{(2\pi)^32\omega(\vb{p})}
	e^{-i\omega(\vb{p})t+i\vb{p}\vdot\vb{x}}
\end{equation}
is the propagator as defined in Ref.~\cite[p.~27]{Peskin1995}.
Given the wave function, the relativistic probability current
\begin{equation}
	j_\mu(t,\vb{x})
	=
	2\Im\left\{
		\Psi(t,\vb{x})^*
		\partial_\mu
		\Psi(t,\vb{x})
	\right\}
\end{equation}
allows us to estimate the center-of-mass position and velocity of the particle, i.e.,
\begin{align}
	\expval{\vb{x}(t)}
	&=
	\int\dd[3]{x}
	\vb{x}
	\rho(t,\vb{x})
	&
	\expval{\vb{v}(t)}
	&=
	\int\dd[3]{x}
	\vb{j}(t,\vb{x})
\end{align}
wherein $\rho(t,\vb{x})=j_0(t,\vb{x})$ is the relativistic probability density.
For more details on the properties of relativistic wave packets, e.g., dispersion, see Ref.~\cite{Naumov2013} and Ref.~\cite{Naumov2009}.

To summarize our findings, we first motivated momentum eigenstates from the commutation algebra.
However, the momentum eigenstates are prone to mathematical inconsistencies, following that the momentum states are strictly speaking distributions, not functions.
Physically, the momentum states correspond to unphysical plane-waves.
A mathematical consistent single-particle state requires a momentum spectrum.
The momentum spectrum encodes many important physical properties like the localization and velocity of the particle.

\subsection{Fock space}

The single-particle state defined in \cref{eq:single_particle_state} is an element of the one-particle Hilbert space of square-integrable functions defined on three-dimensional space $\mathcal{H}^{(1)}=\mathcal{H}\left(L^2(\mathbb{R}^3)\right)$.
The generalization of the one-particle Hilbert space $\mathcal{H}^{(1)}$ to an $n$-particle Hilbert space $\mathcal{H}^{(n)}$ is the tensor product of one-particle Hilbert spaces
\begin{equation}
	\mathcal{H}^{(n)}
	=
	\bigotimes^n_{i=1}
	\mathcal{H}^{(1)}
	.
\end{equation}
Now, it is possible to have a superposition of, e.g., the vacuum state and a particle state
\begin{equation}
	\ket{\psi}
	=
	c_1
	\ket{0}
	+
	c_2
	\ket{1_f}
\end{equation}
with $c_1,c_2\in\mathbb{C}$ which means that we need to combine orthonormal $n$-particle states.
We first construct a tensor algebra over the Hilbert space $\mathcal{H}^{(1)}$ as the direct sum~\cite[p.~290]{Bogolubov1989}
\begin{equation}
	\bigoplus^\infty_{n=0}
	S_+
	\mathcal{H}^{(n)}
\end{equation}
wherein $S_+$ symmetrizes the Hilbert space for bosons.
Equipping the tensor algebra with an inner product and using the completeness of the $n$-particle Hilbert spaces, we obtain again a Hilbert space, named the symmetric Fock space $\mathcal{F}_+$~\cite[p.~35]{Haag2012}.

\subsection{Number states}

Applying the creation operator $\hat{A}^{(+)}[f]$ wherein $f$ is a smearing function or momentum spectrum satisfying the normalization condition, \cref{eq:spectrum_normalization}, suggests defining
\begin{equation}
	\ket{n_f}
	=
	\frac{1}{\sqrt{n!}}
	\hat{A}^{(+)}[f]^n
	\ket{0}
	\label{eq:number_state}
\end{equation}
as number state with spectrum $f$.\footnote{The factorial is required for normalization because bosons are indistinguishable.}
The positive and negative frequency field operators $\hat{A}^{(\pm)}(t,\vb{x})$ generalize the quantum harmonic annihilation and creation operators by adding or removing a particle with spectrum $f$ from the field
\begin{align}
	\hat{A}^{(+)}[f]
	\ket{n_f}
	&=
	\sqrt{n+1}
	\ket{{n+1}_f}
	\\
	\hat{A}^{(-)}[f]
	\ket{n_f}
	&=
	\sqrt{n}
	\ket{{n-1}_f}
	.
\end{align}
While the generalized number state $\ket{n_f}$ is still an eigenstate of the number operator to eigenvalue $n_f$,
\begin{equation}
	\hat{N}
	\ket{n_f}
	=
	n_f
	\ket{n_f}
	,
\end{equation}
and has energy and momentum expectation values
\begin{align}
	\bra{n_f}
	\hat{H}
	\ket{n_f}
	&=
	n
	\int\frac{\dd[3]{p}}{(2\pi)^3}
	\omega(\vb{p})
	\abs*{\frac{f\left(\omega(\vb{q}),\vb{q}\right)}{\sqrt{2\omega(\vb{p})}}}^2
	\\
	\bra{n_f}
	\vu{P}
	\ket{n_f}
	&=
	n
	\int\frac{\dd[3]{p}}{(2\pi)^3}
	\vb{p}
	\abs*{\frac{f\left(\omega(\vb{q}),\vb{q}\right)}{\sqrt{2\omega(\vb{p})}}}^2
	.
\end{align}
The expectation value and variance of the electric field operator are
\begin{align}
	\bra{n_f}
	\hat{E}(t,\vb{x})
	\ket{n_f}
	&=
	0
	\\
	\bra{n_f}
	\left(
		\Delta
		\hat{E}(t,\vb{x})
	\right)^2
	\ket{n_f}
	&=
	\frac{1}{2}
	\int\frac{\dd[3]{p}}{(2\pi)^3}
	\omega(\vb{p})
	+
	n
	\abs*{\Psi(t,\vb{x})}^2
	.
\end{align}
The electric field vanishes for our number states as known in quantum optics, see, for instance, Ref.~\cite{Gerry2005}, but the variance contains an additional term to the "vacuum fluctuations" from the momentum spectrum.
The vacuum fluctuations are in principle infinite, however, our detector is only able to detect a limited bandwidth which makes the vacuum fluctuations in practical applications finite again.

\subsection{Coherent states}

The interaction of a classical current $\vb{j}(t,\vb{x})$ with the Maxwell field operator in the Coulomb gauge $\vu{A}(t,\vb{x})$ is given by the interaction Hamiltonian
\begin{equation}
	\hat{H}_\text{int}(t)
	=
	-
	\int\dd[3]{x}
	\vb{j}(t,\vb{x})
	\vdot
	\hat{\vb{A}}(t,\vb{x})
	.
\end{equation}
Inserting the spatial Fourier transform of the current $\vb{j}(t,\vb{p})$ and the plane-wave expansion, \cref{eq:maxwell_positive_operator,eq:maxwell_negative_operator}, the interaction Hamiltonian becomes
\begin{equation}
	\hat{H}_\text{int}(t)
	=
	-
	\sum_{\lambda=1,2}
	\int\frac{\dd[3]{p}}{(2\pi)^3\sqrt{2\omega(\vb{p})}}
	\left\{
		\left(
			\vb{j}(t,\vb{p})^*
			\vdot
			\vu{e}_\lambda(\vb{p})
		\right)
		\hat{a}_\lambda(\vb{p})
		e^{-i\omega(\vb{p})t}
		+
		\text{H.c.}
	\right\}
	\label{eq:classical_current_interaction}
	.
\end{equation}
where we used the conjugate symmetry $\vb{j}(t,\vb{p})^*=\vb{j}(t,-\vb{p})$.

The effect of an interaction acting on a quantum state from time $t_0$ to $t$ is encoded in the time-evolution operator\footnote{See Ref.~\cite[p.~215]{Greiner2013} for an introduction into the time-evolution operator and interactions.}
\begin{equation}
	\hat{U}(t_0,t)
	=
	\mathcal{T}_+
	\exp\left\{
		-i
		\int_{t_0}^t\dd{t^\prime}
		\hat{H}_\text{int}(t^\prime)
	\right\}
	\label{eq:time_evolution_operator}
\end{equation}
wherein $\mathcal{T}_+$ denotes the time-ordering symbol.
The Magnus expansion presents a systematic approach in finding an explicit form of the time-evolution operator\footnote{See Ref.~\cite[p.~42]{QuesadaMejia2015}, for an introduction to the Magnus expansion with application to nonlinear processes.}, it is given by
\begin{equation}
	\hat{U}(t_0,t)
	=
	\exp\left\{
		\sum_{n=1}
		\Omega^{(n)}(t_0,t)
	\right\}
\end{equation}
wherein the first two terms are given by
\begin{align}
	\hat{\Omega}^{(1)}(t_0,t)
	&=
	-i
	\int_{t_0}^t\dd{t^\prime}
	\hat{H}_\text{int}(t^\prime)
	\\
	\hat{\Omega}^{(2)}(t_0,t)
	&=
	\frac{(-i)^2}{2!}
	\int_{t_0}^t\dd{t^\prime}
	\int_{t_0}^{t^\prime}\dd{t^{\prime\prime}}
	\comm{\hat{H}_\text{int}(t^\prime)}{\hat{H}_\text{int}(t^{\prime\prime})}
	.
\end{align}
For some interactions there exists no exact solution and we can truncate the expansion up to some finite term.
Compared to other expansions, e.g. the Neumann expansion, the truncated Magnus expansion is still unitary.

Let us apply the Magnus expansion to find the time-evolution operator corresponding to the interaction Hamiltonian of \cref{eq:classical_current_interaction}.
The first term of the Magnus expansion turns out to be
\begin{equation}
	\hat{\Omega}^{(1)}(t_0,t)
	=
	i
	\sum_{\lambda=1,2}
	\int\frac{\dd[3]{p}}{(2\pi)^3\sqrt{2\omega(\vb{p})}}
	\left\{
		J_\lambda(t,t_0;\vb{p})
		\hat{a}_\lambda(\vb{p})
		+
		\text{H.c.}
	\right\}
\end{equation}
where we defined the time-integrated current for polarization $\lambda$
\begin{equation}
	J_\lambda(t_0,t;\vb{p})
	=
	\int_{t_0}^t\dd{t^\prime}
	\left(
		\vb{j}(t,\vb{p})^*
		\vdot
		\vu{e}_\lambda(\vb{p})
	\right)
	e^{-i\omega(\vb{p})t^\prime}
	.
\end{equation}
The second term in the Magnus expansion turns out to be complex
\begin{equation}
	\hat{\Omega}^{(2)}(t_0,t)
	=
	i\sum_{\lambda=1,2}
	\int\frac{\dd[3]{p}}{(2\pi)^3\omega(\vb{p})}
	\Im\left\{
		J_\lambda(t_0,t^\prime;\vb{p})
		J_\lambda(t_0,t^{\prime\prime};\vb{p})^*
	\right\}
\end{equation}
which only contributes a phase to the time-evolution operator.
As the second commutator is complex-valued, and therefore commutes, higher order commutators vanish and the Magnus expansion is exact with the first two terms.
As long as we consider a single current source, no interference of phases can occur and we can ignore the complex phase originating from the second Magnus coefficient.
The time-evolution operator of the Maxwell field interacting with a classical source current therefore is~\cite[p.~168]{Itzykson2012}
\begin{equation}
	\hat{U}(t_0,t)
	=
	\exp\left\{
		i\sum_{\lambda=1,2}
		\int\frac{\dd[3]{p}}{(2\pi) ^3\sqrt{2\omega(\vb{p})}}
		\left\{
			J_\lambda(t,t_0;\vb{p})
			\hat{a}_\lambda(\vb{p})
			+
			\text{H.c.}
		\right\}
	\right\}
	.
\end{equation}
Neglecting the polarization 
\begin{equation}
	\hat{U}(t_0,t)
	=
	\exp\left\{
		\int\frac{\dd[3]{p}}{(2\pi) ^3\sqrt{2\omega(\vb{p})}}
		\left\{
			i
			J(t,t_0;\vb{p})
			\hat{a}(\vb{p})
			-
			\text{H.c.}
		\right\}
	\right\}
\end{equation}
we identify the time-evolution operator with the generalization of the displacement operator from quantum optics~\cite[p.~47]{Barnett2002}
\begin{equation}
	\lim_{t\to\infty}\hat{U}(-t,+t)
	=
	\hat{D}\left[-iJ(\vb{p})\right]
\end{equation}
where we take the generalized displacement operator to be
\begin{equation}
	\begin{split}
		\hat{D}[\alpha]
		&=
		\exp\left\{
			\hat{A}^{(+)}[\alpha]
			-
			\hat{A}^{(-)}[\alpha^*]
		\right\}
		\\
		&=
		\exp\left\{
			\int\frac{\dd[3]{p}}{(2\pi) ^3\sqrt{2\omega(\vb{p})}}
			\left\{
				\alpha(\vb{p})
				\hat{a}^\dagger(\vb{p})
				-
				\alpha(\vb{p})^*
				\hat{a}(\vb{p})
			\right\}
		\right\}
	\end{split}
	\label{eq:displacement_operator}
\end{equation}
where we identified the generalized field creation and annihilation operators, $\hat{A}^{(+)}[\alpha]$ and $\hat{A}^{(-)}[\alpha^*]$.
Noting that
\begin{equation}
	\comm{\hat{A}^{(+)}[\alpha]}{\hat{A}^{(-)}[\alpha^*]}
	=
	\int\frac{\dd[3]{p}}{(2\pi)^32\omega(\vb{p})}
	\abs*{\alpha(\vb{p})}^2
\end{equation}
we can employ the \gls{bch} formula as in Ref.~\cite[p.~48]{Barnett2002} to write the displacement operator in normal-order
\begin{equation}
	\hat{D}[\alpha]
	=
	\exp\left\{
		-
		\frac{1}{2}
		\comm{\hat{A}^{(+)}[\alpha]}{\hat{A}^{(-)}[\alpha^*]}
	\right\}
	\exp\left\{
		+
		\hat{A}^{(+)}[\alpha]
	\right\}
	\exp\left\{
		-
		\hat{A}^{(-)}[\alpha]
	\right\}
	.
\end{equation}

Now, let us discuss some properties of the displacement operator.
The product of two different displacements is equal to the sum of the displacement times a suppression factor depending on the overlap for the displacement, i.e.,
\begin{equation}
	\begin{split}
		\hat{D}[\alpha]
		\hat{D}[\beta]
		&=
		\hat{D}[\alpha+\beta]
		\exp\left\{
			-
			\frac{1}{2}
			\comm{\hat{A}^{(+)}[\alpha]}{\hat{A}^{(-)}[\beta^*]}
			+
			\frac{1}{2}
			\comm{\hat{A}^{(+)}[\beta]}{\hat{A}^{(-)}[\alpha^*]}
		\right\}
		\\
		&=
		\hat{D}[\alpha+\beta]
		\exp\left\{
			-
			\frac{1}{2}
			\int\frac{\dd{p}}{(2\pi)^32\omega(\vb{p})}
			\left\{
				\alpha(\vb{p})
				\beta(\vb{p})^*
				-
				\alpha(\vb{p})^*
				\beta(\vb{p})
			\right\}
		\right\}
		.
	\end{split}
\end{equation}
Using the product formula, we can quickly show that the displacement operator is unitary
\begin{equation}
	\hat{D}[\alpha]
	\hat{D}[\alpha]^\dagger
	=
	\hat{D}[\alpha]
	\hat{D}[-\alpha]
	=
	\mathbb{1}
\end{equation}
which is not too surprising given the time-evolution is unitary.

The radiation emitted by a classical current is coherent, suggesting to identify the quantum state produced by a classical current as the coherent state
\begin{equation}
	\ket{\alpha}
	=
	\hat{D}[\alpha]
	\ket{0}
	=
	\exp\left\{
		-
		\frac{1}{2}
		\comm{\hat{A}^{(+)}[\alpha]}{\hat{A}^{(-)}[\alpha^*]}
	\right\}
	\exp\left\{
		\hat{A}^{(+)}[\alpha]
	\right\}
	\ket{0}
	\label{eq:coherent_state}
\end{equation}
where we used that the exponential of the generalized annihilation operator acting on the vacuum state produces the vacuum state.
Coherent states are non-orthogonal, i.e.,
\begin{equation}
	\begin{split}
		\braket{\alpha}{\beta}
		&=
		\exp\left\{
			-
			\frac{1}{2}
			\comm{\hat{A}^{(+)}[\alpha]}{\hat{A}^{(-)}[\alpha^*]}
			-
			\frac{1}{2}
			\comm{\hat{A}^{(+)}[\beta]}{\hat{A}^{(-)}[\beta^*]}
			+
			\comm{\hat{A}^{(+)}[\beta]}{\hat{A}^{(-)}[\alpha^*]}
		\right\}
		\\
		&=
		\exp\left\{
			-
			\frac{1}{2}
			\int\frac{\dd{p}}{(2\pi)^32\omega(\vb{p})}
			\left\{
				\abs*{\alpha(\vb{p})}^2
				+
				\abs*{\beta(\vb{p})}^2
				-
				2\alpha(\vb{p})^*\beta(\vb{p})
			\right\}
		\right\}
		\label{eq:coherent_state_inner}
		.
	\end{split}
\end{equation}
The coherent state is an eigenstate to the annihilation operator
\begin{equation}
	\hat{a}(\vb{p})
	\ket{\alpha}
	=
	\frac{\alpha(\vb{p})}{\sqrt{2\omega(\vb{p})}}
	\ket{\alpha}
	\label{eq:coherent_state_annihilation}
\end{equation}
which makes it simple to derive the expectation value of the Hamiltonian operator
\begin{equation}
	\bra{\alpha}
	\hat{H}
	\ket{\alpha}
	=
	\int\frac{\dd{p}}{(2\pi)^3}
	\omega(\vb{p})
	\abs*{\frac{\alpha(\vb{p})}{\sqrt{2\omega(\vb{p})}}}^2
	\label{eq:coherent_state_hamilton_expval}
\end{equation}
and its variance
\begin{equation}
	\bra{\alpha}
	\left(
		\Delta
		\hat{H}
	\right)^2
	\ket{\alpha}
	=
	\int\frac{\dd{p}}{(2\pi)^3}
	\omega(\vb{p})^2
	\abs*{\frac{\alpha(\vb{p})}{\sqrt{2\omega(\vb{p})}}}^2
	\label{eq:coherent_state_hamilton_variance}
	.
\end{equation}
For the number operator we find the mean to equal the variance
\begin{align}
	\bra{\alpha}
	\hat{N}
	\ket{\alpha}
	&=
	\int\frac{\dd{p}}{(2\pi)^3}
	\abs*{\frac{\alpha(\vb{p})}{\sqrt{2\omega(\vb{p})}}}^2
	=
	\overline{n}
	\label{eq:coherent_state_number_expval}
	\\
	\bra{\alpha}
	\left(
		\Delta
		\hat{N}
	\right)^2
	\ket{\alpha}
	&=
	\overline{n}^2
	\label{eq:coherent_state_number_variance}
	,
\end{align}
i.e., the photon number to be Poisson distributed, which follows simply by setting $\omega(\vb{p})=1$ in the results obtained for the Hamilton operator.
The expectation value of the electric field operator reads
\begin{equation}
	\begin{split}
		\bra{\alpha}
		\hat{E}(t,\vb{x})
		\ket{\alpha}
		&=
		\int\frac{\dd[3]{p}}{(2\pi)^3}
		\sqrt{2\omega(\vb{p})}
		\Im\left\{
			\alpha(\vb{p})
			e^{-i\omega(\vb{p})t+i\vb{p}\vdot\vb{x}}
		\right\}
		\\
		&=
		\int\frac{\dd[3]{p}}{(2\pi)^3}
		\sqrt{2\omega(\vb{p})}
		\abs*{\alpha(\vb{p})}
		\sin\left(
			\vb{p}\vdot\vb{x}
			-
			\omega(\vb{p})t
			+
			\varphi
		\right)
	\end{split}
	\label{eq:coherent_state_electric_expval}
\end{equation}
where we used the polar representation $\alpha(\vb{p})=\abs*{\alpha(\vb{p})}e^{i\varphi}$.
The inner product of a coherent state with a number state yields
\begin{equation}
	\braket{n_f}{\alpha}
	=
	\frac{1}{\sqrt{n!}}
	e^{-\overline{n}/2}
	\left(
		\int\frac{\dd{p}}{(2\pi)^32\omega(\vb{p})}
		f\left(\omega(\vb{p}),\vb{p}\right)^*
		\alpha(\vb{p})
	\right)^n
	\label{eq:coherent_state_inner_number}
	.
\end{equation}
We have derived the generalized coherent state from the interaction of the Maxwell field with a classical current where we found the time-evolution operator to yield the displacement operator.
The generalized coherent state and displacement operators share the same properties as their single-mode quantum optics counterparts which is not too surprising given that the modes are independent of another.
		\section{Time-dependent interactions}

\subsection{Time-evolution operator}

Let $\ket{\psi(t_0)}$ be a state at time $t_0$, then the time-evolution relates the state $\ket{\psi(t)}$ at some later time $t>t_0$ to $\ket{\psi(t_0)}$ via
\begin{equation}
	\ket{\psi(t)}
	=
	\hat{U}(t,t_0)
	\ket{\psi(t_0)}
	.
\end{equation}
Inserting $\ket{\psi(t)}$ into the Schrödinger equation leads to
\begin{equation}
	i\dv{t}
	\hat{U}(t,t_0)
	=
	\hat{H}(t)
	\hat{U}(t,t_0)
\end{equation}
which formal solution is the time-ordered exponential, see Ref.~\cite[p.~380]{Bartelmann2018},
\begin{equation}
	\hat{U}(t,t_0)
	=
	T\exp\left\{
		-i
		\int_{t_0}^t\dd{t^\prime}
		\hat{H}(t^\prime)
	\right\}
\end{equation}
where $T$ denotes the time-ordering symbol.
Only for simple time-dependent systems an exact time-evolution operator exists.
In contrast to the Dyson expansion, the Magnus expansion yields a unitary time-evolution operator even for finite order, in particular,
\begin{equation}
	\hat{U}(t,t_0)
	=
	\exp\left\{
		\sum_{n=1}
		\hat{\Omega}^{(n)}(t,t_0)
	\right\}
\end{equation}
where the first two expansion terms are given by
\begin{align}
	\hat{\Omega}^{(1)}(t,t_0)
	&=
	\frac{(-i)}{1!}
	\int_{t_0}^t\dd{t^\prime}
	\hat{H}(t^\prime)
	\\
	\hat{\Omega}^{(2)}(t,t_0)
	&=
	\frac{(-i)^2}{2!}
	\int_{t_0}^t\dd{t^\prime}
	\int_{t_0}^{t^\prime}\dd{t^{\prime\prime}}
	\comm{\hat{H}(t^\prime)}{\hat{H}(t^{\prime\prime})}
\end{align}
and represent time-ordering corrections, see Ref.~\cite{QuesadaMejia2015}.

\subsection{Interaction with a classical current field}

The Schrödinger-picture Hamiltonian describing the interaction of the Maxwell field $\hat{\vb{A}}$ with a classical current $\vb{j}$ is
\begin{equation}
	\hat{H}_\text{int}(t)
	=
	-
	\int_{\mathbb{R}^3}\dd[3]{x}
	\vb{j}(t,\vb{x})
	\vdot
	\hat{\vb{A}}(t,\vb{x})
	.
\end{equation}
Taking the spatial Fourier transform of the current
\begin{equation}
	\vb{j}(t,\vb{x})
	=
	\int_{\mathbb{R}^3}\frac{\dd[3]{q}}{(2\pi)^3}
	\vb{j}(t,\vb{q})
	e^{+i\vb{q}\vdot\vb{x}}
\end{equation}
and inserting the mode expansion, we find the interaction Hamiltonian to be
\begin{equation}
	\hat{H}_\text{int}(t)
	=
	-
	\sum_{\lambda=1,2}
	\int_{\mathbb{R}^3}\frac{\dd[3]{p}}{(2\pi)^3\sqrt{2\omega(\vb{p})}}
	\left\{
		j_\lambda(t,\vb{p})
		\hat{a}_\lambda(\vb{p})
		e^{-i\omega(\vb{p})t}
		+
		\text{h.c.}
	\right\}
\end{equation}
where we have the transverse current
\begin{equation}
	j_\lambda(q_0,\vb{p})
	=
	\vb{j}(q_0,\vb{p})
	\vdot
	\vu{e}_\lambda(\vb{p})
	.
\end{equation}
The first term in the Magnus expansion turns out to be
\begin{equation}
	\hat{\Omega}^{(1)}(t,t_0)
	=
	i
	\sum_{\lambda=1,2}
	\int_{\mathbb{R}^3}\frac{\dd[3]{p}}{(2\pi)^3\sqrt{2\omega(\vb{p})}}
	\left\{
		J_\lambda(t,t_0;\vb{p})
		\hat{a}_\lambda(\vb{p})
		+
		\text{h.c.}
	\right\}
\end{equation}
where we defined
\begin{equation}
	J_\lambda(t,t_0;\vb{p})
	=
	\int_{t_0}^t\dd{t^\prime}
	j_\lambda(t^\prime,\vb{p})
	e^{-i\omega(\vb{p})t^\prime}
	.
\end{equation}
For the second term in the Magnus expansion, we first evaluate the commutator
\begin{equation}
	\comm{\hat{H}(t^\prime)}{\hat{H}(t^{\prime\prime})}
	=
	i\sum_{\lambda=1,2}
	\int_{\mathbb{R}^3}\frac{\dd[3]{p}}{(2\pi)^3\omega(\vb{p})}
	\Im\left\{
		j_\lambda(t^\prime,\vb{p})
		j_\lambda(t^{\prime\prime},\vb{p})^*
		e^{-i\omega(\vb{p})(t^\prime-t^{\prime\prime})}
	\right\}
\end{equation}
and notice that it is complex valued, hence, all higher commutators vanish and the Magnus expansion with the first two terms is exact.
In summary, the second term of the Magnus expansion turns out to be
\begin{equation}
	\hat{\Omega}^{(2)}(t,t_0)
	=
	i\sum_{\lambda=1,2}
	\int_{\mathbb{R}^3}\frac{\dd[3]{p}}{(2\pi)^3\omega(\vb{p})}
	\Im\left\{
		J_\lambda(t_0,t^\prime;\vb{p})
		J_\lambda(t_0,t^{\prime\prime};\vb{p})^*
	\right\}
\end{equation}
The second term contributes a phase to the time-evolution operator.
As long as we consider a single current source, no interference of phases can occur and we can ignore the phase factor.
The exact time-evolution operator of the Maxwell field interacting with a classical source current therefore is
\begin{equation}
	\hat{U}(t,t_0)
	=
	\exp\left\{
		i\sum_{\lambda=1,2}
		\int_{\mathbb{R}^3}\frac{\dd[3]{p}}{(2\pi)^3\sqrt{2\omega(\vb{p})}}
		\left\{
			J_\lambda(t,t_0;\vb{p})
			\hat{a}_\lambda(\vb{p})
			+
			\text{h.c.}
		\right\}
	\right\}
\end{equation}
which equals the displacement operator for a time-dependent spectrum $\hat{D}[\alpha(t,t_0)]$.

\subsection{Interaction with a charged particle in a potential}

Based on Refs.~\cite[p.~687]{Mandel1995},~\cite[p.~128]{Cohen1992}

The Hamiltonian of the particle with charge $q$ and mass $m$
\begin{equation}
	\hat{H}_q
	=
	\frac{1}{2m}
	\hat{\vb{p}}^2
	+
	qV(\hat{\vb{x}})
\end{equation}
wherein $V$ is a classical external potential and the position and momentum operator satisfy the canonical commutation relations
\begin{align}
	\comm{\hat{x}_i}{\hat{p}_j}
	&=
	i\delta_{ij}
	&
	\comm{\hat{x}_i}{\hat{x}_j}
	&=
	0
	=
	\comm{\hat{p}_i}{\hat{p}_j}
	.
\end{align}
Interaction of the charged particle with an electromagnetic potential $\hat{\vb{A}}$ due to minimal coupling os obtained by the replacement~\cite{Itzykson2012}
\begin{equation}
	\hat{\vb{p}}
	\to
	\hat{\vb{p}}
	-
	q\vb{\hat{A}}
\end{equation}
and expanding the kinetic term of the particle's Hamiltonian
\begin{equation}
	\begin{split}
		\frac{1}{2m}
		\left(\hat{\vb{p}}-q\hat{\vb{A}}\right)^2
		&=
		\frac{1}{2m}
		\hat{\vb{p}}^2
		-
		\frac{q}{2m}
		\left(
			\hat{\vb{p}}
			\vdot
			\hat{\vb{A}}
			+
			\hat{\vb{A}}
			\vdot
			\hat{\vb{p}}
		\right)
		+
		\frac{q^2}{2m}
		\hat{\vb{A}}^2
		\\
		&=
		\frac{1}{2m}
		\hat{\vb{p}}^2
		-
		\frac{q}{m}
		\hat{\vb{p}}
		\vdot
		\hat{\vb{A}}
		+
		\frac{q^2}{2m}
		\hat{\vb{A}}^2
	\end{split}
\end{equation}
where we used that the particle's momentum and the Maxwell field operator commute in the Coulomb gauge, see Ref.~\cite[p.~687]{Mandel1995}.
The interaction term quadratic in the Maxwell field becomes relevant for very high intensities, see Ref.~\cite[p.~198]{Cohen1989} and Ref.~\cite[p.~689]{Mandel1995} for a detailed discussion, which we are not relevant for out use case.
We conclude the interaction Hamiltonian to be
\begin{equation}
	\hat{H}_\text{int}(t,\vb{x})
	=
	-\frac{q}{m}
	\hat{\vb{p}}(t)
	\vdot
	\hat{\vb{A}}(t,\vb{x})
	.
\end{equation}
In the dipole approximation, we consider the Maxwell field constant over the support of the particle wave function~\cite[p.~688]{Mandel1995} and thus
\begin{equation}
	\hat{\vb{A}}(t,\vb{x})
	\ket{\Psi}
	\approx
	\hat{\vb{A}}(t,\vb{x}_0)
	\ket{\Psi}
\end{equation}
with $\vb{x}_0$ being the particle's \gls{com}.

For an alternative approach on how the minimal coupling leads to the dipole interaction, see Ref.~\cite[p.~635]{Cohen1992}.

\subsection{Photodetection}

\subsubsection{Gerry and Knight}

Based on Ref.~\cite[p.~120]{Gerry2005}.

We have the dipole interaction
\begin{equation}
	\hat{H}_\text{int}(t,\vb{x})
	=
	-
	\hat{\vb{d}}
	\vdot
	\hat{\vb{E}}(\vb{x},t)
	=
	-
	\hat{\vb{d}}
	\vdot
	\left(
		\hat{\vb{E}}^{(+)}(\vb{x},t)
		+
		\hat{\vb{E}}^{(-)}(\vb{x},t)
	\right)
\end{equation}
wherein
\begin{equation}
	\hat{\vb{E}}^{(+)}(\vb{x},t)
	=
	i\sum_\lambda
	\int\frac{\dd[3]{p}}{(2\pi)^3\sqrt{2\omega}}
	\vu{e}_\lambda(\vb{p})
	\hat{a}_\lambda(\vb{p})
	e^{+i\vb{p}\vdot\vb{x}}
	\approx
	i\sum_\lambda
	\int\frac{\dd[3]{p}}{(2\pi)^3\sqrt{2\omega}}
	\vu{e}_\lambda(\vb{p})
	\hat{a}_\lambda(\vb{p})
\end{equation}
where the dipole approximation $\norm{\vb{p}\vdot\vb{x}}\ll1$ has been implemented using $e^{i\vb{p}\vdot\vb{x}}\approx1$.

The matrix element for the photoemission is
\begin{equation}
	\bra{e,f}\hat{H}_\text{int}\ket{g,i}
	=
	-
	\bra{e}\hat{\vb{d}}\ket{g}
	\bra{f}\hat{\vb{E}}^{(+)}(\vb{x},t)\ket{i}
	.
\end{equation}
We do not care about the final states of the field, hence we marginalize the probability of an absorption
\begin{equation}
	\sum_f\abs{\bra{f}\hat{\vb{E}}^{(+)}(\vb{x},t)\ket{i}}^2
	=
	\sum_f
	\bra{i}
	\hat{\vb{E}}^{(-)}
	\ketbra{f}
	\hat{\vb{E}}^{(+)}
	\ket{i}
	=
	\expval{\hat{\vb{E}}^{(-)}\vdot\hat{\vb{E}}^{(+)}}{i}
	.
\end{equation}
For a more general initial radiation state $\hat\rho_i=\sum_jp_j\ketbra{j}$, we can rewrite the previous equation as
\begin{equation}
	\tr\left\{
		\hat\rho_f
		\hat{\vb{E}}^{(-)}
		\vdot
		\hat{\vb{E}}^{(+)}
	\right\}
	.
\end{equation}

\subsubsection{Cohen-Tannoudji}

Based on Ref.~\cite[p.~128]{Cohen1992}.

\begin{equation}
	\hat{H}
	=
	\hat{H}_a
	+
	\hat{H}_f
	+
	\hat{H}_i
\end{equation}
wherein \textcolor{red}{need to show this!}
\begin{equation}
	\hat{H}_i
	=
	-
	\hat{\vb{d}}
	\vdot
	\hat{\vb{E}}
\end{equation}
Assuming the dipole moment and the electric field to be parallel, we find in the interaction picture
\begin{align}
	\hat{\vb{d}}(t)
	&=
	e^{+i\hat{H}_at}
	\hat{\vb{d}}(0)
	e^{-i\hat{H}_at}
	\\
	\hat{\vb{E}}(t)
	&=
	e^{+i\hat{H}_ft}
	\hat{\vb{E}}(0)
	e^{-i\hat{H}_ft}
	.
\end{align}
We have a photoelectron emission in time interval $\Delta t$ whenever an electron is excited from the ground state $\ket{g}$, i.e.,
\begin{equation}
	p_{\Delta t}
	=
	\sum_{f,e}
	\abs{\bra{f,e}\hat{U}(\Delta t)\ket{i,g}}^2
\end{equation}
where we marginalized the final states as we do not care about them, and the interaction time-evolution operator is
\begin{equation}
	\hat{U}(\Delta t)
	=
	\mathcal{T}_+
	\exp\left\{
		-i\int_0^{\Delta t}\dd{t^\prime}
		\hat{H}_\text{int}(t^\prime)
	\right\}
	.
\end{equation}
Expanding the time-evolution operator, we find
\begin{equation}
	p_{\Delta t}
	=
	\sum_{f\neq i,e\neq g}
	\int_0^{\Delta t}\dd{t^\prime}
	\int_0^{\Delta t}\dd{t^{\prime\prime}}
	\abs{\bra{f,e}\hat{H}_\text{int}(t^\prime)\hat{H}_\text{int}(t^{\prime\prime})\ket{i,g}}^2
\end{equation}
where the first term (zeroth order in $\hat{H}_\text{int}$ vanishes because of the orthogonality of the final and initial states, and the second term vanishes because the dipole moment operator is asymmetric~\cite[p.~131]{Cohen1992}.
Writing out the dipole and electric field operators, we find
\begin{equation}
	\begin{split}
		p_{\Delta t}
		&=
		\sum_{f\neq i,e\neq g}
		\int_0^{\Delta t}\dd{t^\prime}
		\int_0^{\Delta t}\dd{t^{\prime\prime}}
		\bra{i}
		\hat{\vb{E}}(t^\prime)
		\ketbra{f}
		\hat{\vb{E}}(t^{\prime\prime})
		\ket{i}
		\bra{g}
		\hat{\vb{d}}(t^\prime)
		\ketbra{e}
		\hat{\vb{d}}(t^{\prime\prime})
		\ket{g}
		\\
		&=
		\int_0^{\Delta t}\dd{t^\prime}
		\int_0^{\Delta t}\dd{t^{\prime\prime}}
		\bra{i}
		\hat{\vb{E}}(t^\prime)
		\left(
			\sum_f
			\ketbra{f}
		\right)
		\hat{\vb{E}}(t^{\prime\prime})
		\ket{i}
		\bra{g}
		\hat{\vb{d}}(t^\prime)
		\left(
			\sum_e
			\ketbra{e}
		\right)
		\hat{\vb{d}}(t^{\prime\prime})
		\ket{g}
		\\
		&=
		\int_0^{\Delta t}\dd{t^\prime}
		\int_0^{\Delta t}\dd{t^{\prime\prime}}
		\bra{i}
		\hat{\vb{E}}(t^\prime)
		\hat{\vb{E}}(t^{\prime\prime})
		\ket{i}
		\bra{g}
		\hat{\vb{d}}(t^\prime)
		\hat{\vb{d}}(t^{\prime\prime})
		\ket{g}
		\\
		&=
		\int_0^{\Delta t}\dd{t^\prime}
		\int_0^{\Delta t}\dd{t^{\prime\prime}}
		G_i(t^\prime,t^{\prime\prime})^*
		G_g(t^\prime,t^{\prime\prime})
	\end{split}
\end{equation}
where $G_i$ and $G_g$ are the two-time correlation functions of the atomic detector system and radiation field.

\subsubsection{Mandel and Wolf}

In the model presented in Ref.~\cite[p.~685]{Mandel1995}, the interaction Hamiltonian is~\cite[p.~689]{Mandel1995}
\begin{equation}
	\hat{H}_\text{int}(t)
	=
	-
	\hat{\vb{p}}(t)
	\vdot
	\hat{\vb{A}}(\vb{x}_0,t)
\end{equation}
wherein $\vb{x}_0$ is the detector atom's \gls{com}.
The interaction term can be transformed into the dipole moment operator $\hat{\vb{d}}$ and the dielectric displacement field operator $\hat{\vb{D}}$~\cite[p.~689]{Mandel1995} giving a similar interaction term as discussed by Cohen-Tannoudji.

We then consider a bound electron in a potential well. When bound, the electron state $\ket{\Psi_0}$ satisfies
\begin{equation}
	\hat{H}_a
	\ket{g}
	=
	E_g
	\ket{g}
	=
	-\omega_g
	\ket{g}
	.
\end{equation}
Let $\hat\rho_f$ be the radiation field state in the Schrödinger picture
\begin{equation}
	\hat\rho^{(S)}(t_0)
	=
	\ketbra{g}
	\otimes
	\hat\rho_f(t_0)
\end{equation}
then in the interaction picture, the state is~\cite[p.~685]{Mandel1995}
\begin{equation}
	\hat\rho^{(I)}(t)
	=
	e^{+i\hat{H}_0(t-t_0)}
	\hat\rho^{(S)}(t)
	e^{-i\hat{H}_0(t-t_0)}
\end{equation}
and the electron's momentum operator takes the form
\begin{equation}
	\hat{\vb{p}}(t)
	=
	e^{+i\hat{H}_a(t-t_0)}
	\hat{\vb{p}}
	e^{-i\hat{H}_a(t-t_0)}
\end{equation}
and the Maxwell field is
\begin{equation}
	\begin{split}
		\hat{\vb{A}}(\vb{x}_0,t)
		&=
		\hat{\vb{A}}^{(+)}(\vb{x}_0,t)
		+
		\hat{\vb{A}}^{(-)}(\vb{x}_0,t)
		\\
		&=
		\sum_{\lambda=1,2}
		\int_{\mathbb{R}^3}
		\frac{\dd[3]{p}}{(2\pi)^3\sqrt{2\omega(\vb{p})}}
		\hat{a}_\lambda(\vb{p})
		\boldsymbol{\varepsilon}_\lambda(\vb{p})
		e^{+i\vb{p}\vdot\vb{x}_0-i\omega(\vb{p})(t-t_0)}
		+
		\text{h.c.}
		.
	\end{split}
\end{equation}
In the interaction picture, the quantum state evolves according to
\begin{equation}
	\dv{\hat\rho^{(I)}}{t}
	=
	i\comm{\hat\rho^{(I)}}{\hat{H}^{(I)}_\text{int}}
\end{equation}
which can be solved by Magnus expansion.
For instance,
\begin{equation}
	\hat\rho^{(I)}(t)
	=
	\hat\rho(t_0)
	+
	i\int_{t_0}^t\dd{t^\prime}
	\comm{\hat\rho(t_0)}{\hat{H}_\text{int}(t^\prime)}
	+
	i^2
	\int_{t_0}^t\dd{t^\prime}
	\int_{t_0}^{t^\prime}\dd{t^{\prime\prime}}
	\dots
\end{equation}

The probability amplitude for the transition
\begin{equation}
	\ket{g,i}
	\to
	\ket{e,f}
\end{equation}
is equal to~\cite[p.~686]{Mandel1995}
\begin{equation}
	\begin{split}
		p(t_0,\Delta t)
		=
		\tr\left\{
			\hat\rho_{e,f}
			\hat\rho_{g,i}(t_0+\Delta t)
		\right\}
		&=
		\tr\left\{
			\hat\rho_{e,f}
			\hat\rho_{g,i}(t_0)
		\right\}
		\\
		&+
		\frac{1}{i}
		\tr\left\{
			\hat\rho_{e,f}
			\int_{t_0}^{t_0+\Delta t}
			\dd{t^\prime}
			\comm{\hat{H}_\text{int}(t^\prime)}{\hat\rho_{g,i}(t_0)}
		\right\}
		\\
		&+
		\frac{1}{i^2}
		\tr\left\{
			\hat\rho_{e,f}
			\int_{t_0}^{t_0+\Delta t}\dd{t^\prime}
			\int_{t_0}^{t^\prime}\dd{t^{\prime\prime}}
			\comm{\hat{H}_\text{int}(t^\prime)}{\comm{\hat{H}_\text{int}(t^{\prime\prime})}{\hat\rho_{g,i}(t_0)}}
		\right\}
	\end{split}
\end{equation}
the first two terms vanish and we have
\begin{equation}
	p(t_0,\Delta t)
	=
	\int_{t_0}^{t_0+\Delta t}\dd{t^\prime}
	\int_{t_0}^{t^\prime}\dd{t^{\prime\prime}}
	\expval{\hat{H}_\text{int}(t^\prime)\hat\rho(t_0)\hat{H}_\text{int}(t^{\prime\prime})}{e,f}
	+
	\text{c.c.}
\end{equation}
We now take the interaction Hamiltonian
\begin{equation}
	\hat{H}_\text{int}(t)
	=
	e^{+i\hat{H}_a(t-t_0)}
	\hat{\vb{p}}
	e^{-i\hat{H}_a(t-t_0)}
	\hat{\vb{A}}(\vb{x}_0,t)
\end{equation}
which we evaluate with \textcolor{red}{check this!}
\begin{equation}
	\begin{split}
		\expval{\hat{H}_\text{int}(t^\prime)\hat\rho(t_0)\hat{H}_\text{int}(t^{\prime\prime})}{e,f}
		&=
		\bra{e}\hat{p}_i\ket{g}
		\bra{g}\hat{p}_j\ket{e}
		e^{i(E-E_0)(t^\prime-t^{\prime\prime})}
		\\
		&\times
		\bra{f}
		\hat{A}_i(\vb{x}_0,t^\prime)
		\braket{i}
		\hat{A}_j(\vb{x}_0,t^{\prime\prime})
		\ket{f}
		+
		\text{c.c.}
	\end{split}
\end{equation}
Expanding the initial state in the coherent state basis
\begin{equation}
	\ket{i}
	=
	\int\dd[2]{\alpha}
	p_i(\alpha,t_0)
	\ketbra{\alpha}
\end{equation}
we can use the eigenvalue equation of the coherent state and sum over all final states to remove the final state dependency, i.e.,
\begin{equation}
	\sum_i
	p(t_0,\Delta t)
	=
	\int_{t_0}^{t_0+\Delta t}\dd{t^\prime}
	\int_{t_0}^{t^\prime}\dd{t^{\prime\prime}}
	\bra{e}\hat{p}_i\ket{g}
	\bra{g}\hat{p}_j\ket{e}
	e^{i(E-E_0)(t^\prime-t^{\prime\prime})}
	\expval{\hat{A}_i(\vb{x}_0,t^\prime)\hat{A}_j(\vb{x}_0,t^{\prime\prime})}
	+
	\text{c.c.}
\end{equation}
We then sum of all final electron states weighted by the density of states times the the probability of being collected by the detector and find
\begin{equation}
	P(t_0,\Delta t)
	=
	\int_{t_0}^{t_0+\Delta t}\dd{t^\prime}
	\int_{t_0}^{t^\prime}\dd{t^{\prime\prime}}
	k_{ij}(t^\prime-t^{\prime\prime})
	\expval{\hat{A}_i(\vb{x}_0,t^\prime)\hat{A}_j(\vb{x}_0,t^{\prime\prime})}
	+
	\text{c.c.}
\end{equation}
see Ref.~\cite[p.~694]{Mandel1995} for an explicit representation of the response function $k_{ij}$.
Employing normal-ordering of the Maxwell field operators, we find that the second term, the vacuum contribution, becomes zero Ref.~\cite[p.~694]{Mandel1995} and we can write
\begin{equation}
	P(t_0,\Delta t)
	=
	\int_{t_0}^{t_0+\Delta t}\dd{t^\prime}
	\int_{t_0}^{t^\prime}\dd{t^{\prime\prime}}
	k_{ij}(t^\prime-t^{\prime\prime})
	\expval{\colon\hat{A}_i(\vb{x}_0,t^\prime)\hat{A}_j(\vb{x}_0,t^{\prime\prime})\colon}
	+
	\text{c.c.}
\end{equation}
\textcolor{red}{Can we rewrite this in terms of electric field operators?} -> Yes, but we need to perform more approximations

\begin{equation}
	\begin{split}
		\colon
		\hat{A}_i(\vb{x}_0,t^\prime)
		\hat{A}_j(\vb{x}_0,t^{\prime\prime})
		\colon
		&=
		\colon
		\left[
			\hat{A}_i^{(+)}(\vb{x}_0,t^\prime)
			+
			\hat{A}_i^{(-)}(\vb{x}_0,t^\prime)
		\right]
		\left[
			\hat{A}_j^{(+)}(\vb{x}_0,t^{\prime\prime})
			+
			\hat{A}_j^{(-)}(\vb{x}_0,t^{\prime\prime})
		\right]
		\colon
		\\
		&\approx
		\colon
		\left[
			\hat{A}_i^{(+)}(\vb{x}_0,t^\prime)
			\hat{A}_j^{(-)}(\vb{x}_0,t^{\prime\prime})
			+
			\hat{A}_i^{(-)}(\vb{x}_0,t^\prime)
			\hat{A}_j^{(+)}(\vb{x}_0,t^{\prime\prime})
		\right]
		\colon
		\\
		&=
		\hat{A}_i^{(+)}(\vb{x}_0,t^\prime)
		\hat{A}_j^{(-)}(\vb{x}_0,t^{\prime\prime})
		+
		\hat{A}_j^{(+)}(\vb{x}_0,t^{\prime\prime})
		\hat{A}_i^{(-)}(\vb{x}_0,t^\prime)
	\end{split}
\end{equation}
\begin{equation}
	\begin{split}
		\hat{A}_i^{(+)}(\vb{x}_0,t^\prime)
		\hat{A}_j^{(-)}(\vb{x}_0,t^{\prime\prime})
		&=
		\left(
			\sum_{\lambda=1,2}
			\int\frac{\dd[3]{p}}{(2\pi)^3\sqrt{2\omega(\vb{p})}}
			\hat{a}_\lambda^\dagger(\vb{p})
			\vu{e}^i_\lambda(\vb{p})^*
			e^{+i\omega(\vb{p})t^\prime}
		\right)
		\\
		&\times
		\left(
			\sum_{\sigma=1,2}
			\int\frac{\dd[3]{q}}{(2\pi)^3\sqrt{2\omega(\vb{q})}}
			\hat{a}_\sigma(\vb{q})
			\vu{e}^j_\sigma(\vb{q})
			e^{-i\omega(\vb{q})t^{\prime\prime}}
		\right)
	\end{split}
\end{equation}
Let us further neglect polarization
\begin{equation}
	\begin{split}
		P(t_0,\Delta t)
		&=
		\int_{t_0}^{t_0+\Delta t}\dd{t^\prime}
		\int_{t_0}^{t^\prime}\dd{t^{\prime\prime}}
		k(t^\prime-t^{\prime\prime})
		\\
		&\times
		\expval{
			\hat{A}^{(+)}(\vb{x}_0,t^\prime)
			\hat{A}^{(-)}(\vb{x}_0,t^{\prime\prime})
			+
			\hat{A}^{(+)}(\vb{x}_0,t^{\prime\prime})
			\hat{A}^{(-)}(\vb{x}_0,t^\prime)
		}
		+
		\text{c.c.}
	\end{split}
\end{equation}

\subsubsection{Vogel}

Based on Ref.~\cite[p.~48]{Vogel2006}

The minimal-coupling Hamiltonian
\begin{equation}
	\hat{H}
	=
	\int\dd[3]{x}
	\int_0^\omega\dd{\omega}
	\hat{\vb{f}}^\dagger(\vb{x},\omega)
	\hat{\vb{f}}(\vb{x},\omega)
	+
	\sum_j\frac{\left(\hat{\vb{p}}_j-q_j\hat{\vb{A}}(\hat{\vb{x}}_j,t)\right)^2}{2m_j}
	+
	\hat{W}_\text{Coul}
\end{equation}
wherein $\hat{W}_\text{Coul}$ is the Coulomb interaction between the different charges.
For bound atomic states, we can expand the Maxwell field around the atom's \gls{com} and the interaction Hamiltonian takes the form
\begin{equation}
	\hat{H}_\text{int}
	=
	-
	\sum_j\frac{q_j}{m_j}
	\hat{\vb{p}}_j
	\vdot
	\hat{\vb{A}}(\vb{x}_j)
	+
	\sum_j\frac{q_j^2}{2m_j}
	\hat{\vb{A}}(\vb{x}_j)^2
\end{equation}
In the electric-dipole approximation we have
\begin{equation}
	\hat{H}_\text{int}
	\approx
	-
	\sum_j\frac{q_j}{m_j}
	\hat{\vb{p}}_j
	\vdot
	\hat{\vb{A}}(\vb{x}_j)	
\end{equation}
\textcolor{red}{fancy reasoning why the former interaction Hamiltonian is equivalent to} (maybe \cite[p.~691]{Mandel1995} or Cohen-Tanodji?)
\begin{equation}
	\hat{H}_\text{int}(t)
	\approx
	-
	\sum_j
	\hat{\vb{d}}_j
	\vdot
	\hat{\vb{E}}(\vb{x}_j,t)
	.
\end{equation}

Based on Ref.~\cite[p.~173]{Vogel2006}

The photoemission probability is equal to
\begin{equation}
	\begin{split}
		\abs{\bra{e,f}\hat{U}(t_0,t_0+\Delta t)\ket{g,i}}^2
		&=
		\bra{e,f}
		\hat{U}(t_0,t_0+\Delta t)
		\ket{g,i}
		\\
		&\times
		\bra{g,i}
		\hat{U}^\dagger(t_0,t_0+\Delta t)
		\ket{e,f}
		\\
		&=
		\tr\biggl\{
			\bra{e,f}
			\hat{U}(t_0,t_0+\Delta t)
			\ket{g,i}
			\\
			&\times
			\bra{g,i}
			\hat{U}^\dagger(t_0,t_0+\Delta t)
			\ket{e,f}
		\biggr\}
		\\
		&=
		\tr\biggl\{
			\ketbra{e,f}
			\hat{U}(t_0,t_0+\Delta t)
			\ketbra{g,i}
			\hat{U}^\dagger(t_0,t_0+\Delta t)
		\biggr\}
		\\
		&=
		\tr\left\{
			\hat\rho_{e,f}
			\hat\rho_{g,i}(t_0+\Delta t)
		\right\}
	\end{split}
\end{equation}
wherein the time-evolution operator is
\begin{equation}
	\hat{U}(t_0,t_0+\Delta t)
	=
	\mathcal{T}_+
	\exp\left\{
		-i
		\int_{t_0}^{t_0+\Delta t}\dd{t^\prime}
		\hat{H}_\text{int}(t^\prime)
	\right\}
\end{equation}
then, evaluating the transition amplitude
\begin{equation}
	\bra{g,f}
	\hat{U}(t_0,t_0+\Delta t)
	\ket{e,i}
	=
	\mathcal{T}_+
	\sum_{n=0}^\infty
	\frac{1}{n!}
	\bra{g,f}
	\left[
		i
		\int_{t_0}^{t_0+\Delta t}\dd{t^\prime}
		\vb{d}_{fg}
		\vdot
		\hat{\vb{E}}^{(+)}(\vb{x}_0,t^\prime)
		e^{i\omega_{fg}(t^\prime-t)}
	\right]^n
	\ket{e,i}
\end{equation}
\textcolor{red}{why do we only have $E^+$ here? How to derive this exactly?}

\subsection{Photodetection}

Let us consider the composite system of an atom with a single electron and a radiation field.
\textcolor{red}{Figure where we see an electron in a potential well and how radiation can excite it...}

In the ground state $\ket{g}$, the electron is bound and satisfies
\begin{equation}
	\hat{H}_a
	\ket{g}
	=
	E_g
	\ket{g}
	.
\end{equation}
For energies $E>0$, the electron is in one of many free excited state $\ket{e}$.
\textcolor{red}{photoelectric effect. why can this be used for photodiodes where there is no ionization happening. See Cohen-Tannoudji for explanation.}

Whenever, we ionize the atom, we can collect the free electron indicating a photo detection.
The probability that such an event occurs in the time interval $[t_0,t_0+\Delta t]$ is
\begin{equation}
	p_{e,f}(t_0,t_0+\Delta t)
	=
	\abs{
		\bra{e,f}
		\hat{U}(t_0,t_0+\Delta t)
		\ket{g,i}
	}^2
	.
\end{equation}
wherein $\hat{U}$ is the time-evolution operator.
We can recast the probability in the more general density operator formalism
\begin{equation}
	\begin{split}
		p_{e,f}(t_0,t_0+\Delta t)
		&=
		\bra{e,f}
		\hat{U}(t_0,t_0+\Delta t)
		\ketbra{g,i}
		\hat{U}^\dagger(t_0,t_0+\Delta t)
		\ket{e,f}
		\\
		&=
		\tr\left\{
			\bra{e,f}
			\hat{U}(t_0,t_0+\Delta t)
			\ketbra{g,i}
			\hat{U}^\dagger(t_0,t_0+\Delta t)
			\ket{e,f}
		\right\}
		\\
		&=
		\tr\left\{
			\ketbra{e,f}
			\hat{U}(t_0,t_0+\Delta t)
			\ketbra{g,i}
			\hat{U}^\dagger(t_0,t_0+\Delta t)
		\right\}
		\\
		&=
		\tr\left\{
			\hat\varrho
			\hat{U}(t_0,t_0+\Delta t)
			\hat\rho(t_0)
			\hat{U}^\dagger(t_0,t_0+\Delta t)
		\right\}
		\\
		&=
		\tr\left\{
			\hat\varrho
			\hat\rho(t_0+\Delta t)
		\right\}
	\end{split}
\end{equation}
where we used that the trace of a scalar is the scalar in the second line and the cyclic property of the trace in the third line.
The time-evolution operator is
\begin{equation}
	\hat{U}(t_0,t)
	=
	\mathcal{T}_+
	\exp\left\{
		-i
		\int_{t_0}^t\dd{t^\prime}
		\hat{H}_\text{int}(t^\prime)
	\right\}
\end{equation}
with the interaction Hamiltonian in the electric-dipole and rotating wave approximation being
\begin{equation}
	\hat{H}_\text{int}(t)
	=
	-
	\hat{\vb{p}}(t)
	\vdot
	\hat{\vb{A}}(\vb{x}_0,t)
	.
\end{equation}
Instead of the time-ordered exponential, we can use the Magnus expansion
\begin{align}
	\hat{U}(t_0,t)
	&=
	e^{\Omega(t_0,t)}
	&
	\Omega(t_0,t)
	&=
	\sum_{n=1}\Omega^{(n)}(t_0,t)
\end{align}
The time evolved quantum state is then
\begin{equation}
	\begin{split}
		\hat\rho(t_0+\Delta t)
		&=
		\hat{U}(t_0,t_0+\Delta t)
		\hat\rho(t_0)
		\hat{U}^\dagger(t_0,t_0+\Delta t)
		\\
		&=
		\hat\rho(t_0)
		+
		\comm{\Omega(t_0,t)}{\hat\rho(t_0)}
		+
		\frac{1}{2!}
		\comm{\Omega(t_0,t)}{\comm{\Omega(t_0,t)}{\hat\rho(t_0)}}
		+
		\dots
	\end{split}
\end{equation}
where we used the \gls{bch} formula.
We perform Magnus expansion up to the first term and find the perturbative solution
\begin{equation}
	\begin{split}
		\hat\rho(t_0+\Delta t)
		\approx
		\hat\rho(t_0)
		&+
		(-i)
		\int_{t_0}^{t_0+\Delta t}\dd{t_1}
		\comm{\hat{H}_\text{int}(t_1)}{\hat\rho(t_0)}
		\\
		&+
		\frac{(-i)^2}{2!}
		\int_{t_0}^{t_0+\Delta t}\dd{t_1}
		\int_{t_0}^{t_1}\dd{t_2}
		\comm{\hat{H}_\text{int}(t_1)}{\comm{\hat{H}_\text{int}(t_2)}{\hat\rho(t_0)}}
	\end{split}	
\end{equation}
and insert the expansion into the photoemission probability
\begin{equation}
	\begin{split}
		p_{e,f}(t_0,t_0+\Delta t)
		&=
		\tr\left\{
			\hat\varrho
			\hat\rho(t_0+\Delta t)
		\right\}
		\\
		&=
		\int_{t_0}^{t_0+\Delta t}\dd{t_1}
		\int_{t_0}^{t_1}\dd{t_2}
		\tr\left\{
			\hat{H}_\text{int}(t_1)
			\hat\rho(t_0)
			\hat{H}_\text{int}(t_2)
		\right\}
		+
		\text{c.c.}
		\\
		&=
		\bra{e}\hat{p}_i\ket{g}
		\bra{g}\hat{p}_j\ket{e}
		\int_{t_0}^{t_0+\Delta t}\dd{t_1}
		\int_{t_0}^{t_1}\dd{t_2}
		e^{i(E_e-E_g)(t_1-t_2)}
		\\
		&\times
		\expval{
			\hat{A}_j(\vb{x}_0,t_2)
			\hat\rho(t_0)
			\hat{A}_i(\vb{x}_0,t_1)
		}{i}
		+
		\text{c.c.}
	\end{split}
\end{equation}
\textcolor{red}{steps to second line missing!}
We are not interested in the final states and can integrate this degree of freedom.
Furthermore, we can assume an ensemble of independent detector atoms, so the probability for a photoemission is
\begin{equation}
	p(t_0,t_0+\Delta t)
	=
	\int_{t_0}^{t_0+\Delta t}\dd{t_1}
	\int_{t_0}^{t_1}\dd{t_2}
	k_{ij}(t_1-t_2)
	\expval{
		\hat{A}_j(\vb{x}_0,t_2)
		\hat{A}_i(\vb{x}_0,t_1)
	}
	+
	\text{c.c.}
\end{equation}
wherein $k_{ij}$ encodes the microscopic properties of the detector atom ensemble, see Ref.~\cite[p.~694]{Mandel1995}.
Expansion of the Maxwell field operator into positive and negative frequency parts as well as discarding high-frequency terms, we find~\cite[p.~698]{Mandel1995}.
\begin{equation}
	p(t_0,t_0+\Delta t)
	\approx
	\int_0^{\Delta t}\dd{t^\prime}
	\int_0^{t^\prime}\dd{t^{\prime\prime}}
	k_{ij}(t^{\prime}-t^{\prime\prime})
	\expval{
		\hat{A}_j^{(+)}(\vb{x}_0,t_0+t^{\prime\prime})
		\hat{A}_i^{(-)}(\vb{x}_0,t_0+t^{\prime})
	}
	+
	\text{c.c.}
	.
\end{equation}
Furthermore, performing the quasi-monochromatic approximation
\begin{align}
	\hat{A}_j^{(+)}(\vb{x}_0,t_0+t^{\prime\prime})
	&\approx
	\hat{A}_j^{(+)}(\vb{x}_0,t_0)
	e^{+i\omega_0t^{\prime\prime}}
	\\
	\hat{A}_j^{(-)}(\vb{x}_0,t_0+t^{\prime})
	&\approx
	\hat{A}_j^{(-)}(\vb{x}_0,t_0)
	e^{-i\omega_0t^{\prime}}
\end{align}
we find
\begin{equation}
	\begin{split}
		p(t_0,t_0+\Delta t)
		&\approx
		\expval{
			\hat{A}_j^{(+)}(\vb{x}_0,t_0)
			\hat{A}_i^{(-)}(\vb{x}_0,t_0)
		}
		\int_0^{\Delta t}\dd{t^\prime}
		\int_0^{t^\prime}\dd{t^{\prime\prime}}
		k_{ij}(t^{\prime}-t^{\prime\prime})
		e^{-i\omega_0(t^\prime-t^{\prime\prime})}
		+
		\text{c.c.}
		\\
		&=
		\expval{
			\hat{A}_j^{(+)}(\vb{x}_0,t_0)
			\hat{A}_i^{(-)}(\vb{x}_0,t_0)
		}
		\int_0^{\Delta t}\dd{t^\prime}
		\int_0^{t^\prime}\dd{\tau}
		k_{ij}(\tau)
		e^{-i\omega_0\tau}
		+
		\text{c.c.}
		\\
		&=
		\expval{
			\hat{A}_j^{(+)}(\vb{x}_0,t_0)
			\hat{A}_i^{(-)}(\vb{x}_0,t_0)
		}
		\int_0^{\Delta t}\dd{t^\prime}
		\int_{-t^\prime}^{t^\prime}\dd{\tau}
		k_{ij}(\tau)
		e^{-i\omega_0\tau}
		\\
		&\approx
		\expval{
			\hat{A}_j^{(+)}(\vb{x}_0,t_0)
			\hat{A}_i^{(-)}(\vb{x}_0,t_0)
		}
		k_{ij}(\omega_0)
		\Delta t
	\end{split}
\end{equation}
as in Ref.~\cite[p.~699]{Mandel1995} where $k_{ij}(\omega_0)$ is the frequency response of the detector atom ensemble.
We assume a detector equally sensitive to both polarizations and finally find
\begin{equation}
	p(t_0,t_0+\Delta t)
	\approx
	\expval{\hat{N}}
	k(\omega_0)
	\Delta t
\end{equation}
we can even relax the quasi-monochromatic approximation a bit by writing
\begin{equation}
	p(t_0,t_0+\Delta t)
	\approx
	\int\dd{\omega}
	k(\omega)
	\expval{\hat{n}(\omega)}
	\Delta t
	.
\end{equation}
		\section*{Summary}
\addcontentsline{toc}{section}{Summary}

In the present chapter, we introduced \gls{qkd} as an example for quantum optical communication and as a mechanism for practical and secure key distribution, which, together with classical symmetric ciphers, enables secure communication with means to estimate information leakage.
We then analyzed the quantum transmission phase of qubit- and boson-based \gls{qkd} protocols generating correlated information between the receiver and transmitter.
\begin{table}[htb]
	\centering	
	\begin{tabular}{lcc}
		\toprule
			& Qubit-based & Boson-based \\
		\midrule
			Visualization & Bloch sphere & Phase space \\
			Hilbert space (dim) & Finite (two) & Countable (infinite) \\
			Measurement operator & $\vb{\hat{S}}(\vb{n})=\hat{S}_in^i$ & $\hat{X}(\vartheta)=\frac{1}{\sqrt{2}}\left(\hat{a}e^{-i\vartheta}+\hat{a}^\dagger e^{+i\vartheta}\right)$ \\
			Standard basis & $\left\{\ket{0},\ket{1}\right\}$ & $\left\{\ket{x},\ket{p}\colon x,p\in\mathbb{R}\right\}$ \\
		\bottomrule
	\end{tabular}
	\caption{Comparison of qubit- and boson-based \gls{qkd} protocols.}\label{tab:qkd_comparison}
\end{table}
By introducing the concept of qubit- and boson-based \gls{qkd} protocols, with their respective key properties summarized in \Cref{tab:qkd_comparison}, we formalized the concept of DV- and CV-QKD.
Additionally, we formulated the concept of a logical and an encoding quantum system allowing us to encode qubits onto number or coherent states, which might share some similarity with the concept of symbols and pulse-shaping in classical signal-processing.
Because the technology to prepare and measure coherent states is highly mature, most practical QKD implementations, regardless of qubit- or boson-based, transmit weak coherent states.
\begin{figure}[htb]
	\centering
	\includegraphics{figures/tikz/qkd-protocol}
	\caption{An abstract \gls{qkd} protocol comprises a binary encoder, a logical quantum system, a binary decoder, and some post-processing. The binary encoder maps bits onto a quantum state of the logical quantum system. The binary decoder extracts the bits from the logical quantum system. The logical quantum system is a subspace of a larger physical quantum system. The state encoder and decoder map between the logical and physical quantum states.}\label{fig:qkd_protocol}
\end{figure}
Given the correlated data from the quantum transmission, we compiled classical methods to distill a shared secret key between the transmitter and receiver, known as classical post-processing. 
The classical post-processing maps the discrete or continuous data from the transmission sequence to binary symbols, corrects errors, discards failed data blocks, and removes information from the partially secret key using privacy amplification.
Finally, we roughly outlined some ideas for the security analysis of QKD.

The concept of a logical and en encoding quantum layer in QKD needs further investigation but might open up new more general protocols and simplify security proofs.
Concerning our thesis, our investigations suggest developing our theoretical framework for quantum optical communication towards a coherent state transmission system.

		\addcontentsline{toc}{section}{References}
		\printbibliography[title=References]
	\end{refsection}
	
	\chapter{Interaction theory of optical components}
	\textit{The following chapter presents an in-depth discussion of the classical and quantum characteristics of important optical components. In particular, we need to describe optical couplers, modulators and detectors in using a continuous-mode time-dependent quantum theory.}
	\begin{refsection}
		\section{Laser}

% Use semiclassical laser theory as described in Wolf & Mandel. p. 929 to derive an expression for the classical current.
% Use the classical current to give coherent state.
		\section{Time-dependent mode coupler}

The mode coupler is a generalization of the beam splitter and is described by the Hamiltonian
\begin{equation}
	\hat{H}
	=
	\hat{H}_a
	+
	\hat{H}_b
	+
	\hat{H}_{ab}(t)
\end{equation}
where
\begin{align}
	\hat{H}_a
	=
	\int_{\mathbb{R}^3}\frac{\dd[3]{p}}{(2\pi)^3}
	\omega(\vb{p})
	\hat{a}^\dagger(\vb{p})
	\hat{a}(\vb{p})
	&&
	\hat{H}_b
	=
	\int_{\mathbb{R}^3}\frac{\dd[3]{p}}{(2\pi)^3}
	\omega(\vb{p})
	\hat{b}^\dagger(\vb{p})
	\hat{b}(\vb{p})
\end{align}
are bosonic free field Hamiltonians and
\begin{equation}
	\hat{H}_{ab}(t)
	=
	-
	\int\dd[3]{x}
	g(t,\vb{x})
	\hat{A}(t,\vb{x})
	\hat{B}(t,\vb{x})
\end{equation}
is the most general linear interaction between these fields.
Expanding the fields into their positive and negative frequency parts and only keeping the mixed terms
\begin{equation}
	\begin{split}
		\hat{H}_{ab}
		&=
		-
		\int\dd[3]{x}
		g(t,\vb{x})
		\left(
			\hat{A}^{(+)}(t,\vb{x})
			+
			\hat{A}^{(-)}(t,\vb{x})
		\right)
		\left(
			\hat{B}^{(+)}(t,\vb{x})
			+
			\hat{B}^{(-)}(t,\vb{x})
		\right)
		\\
		&\approx
		-
		\int\dd[3]{x}
		g(t,\vb{x})
		\left\{
			\hat{A}^{(-)}(t,\vb{x})
			\hat{B}^{(+)}(t,\vb{x})
			+
			\hat{A}^{(+)}(t,\vb{x})
			\hat{B}^{(-)}(t,\vb{x})
		\right\}
	\end{split}
\end{equation}
is known as the rotating wave approximation, see~\cite[p.~158]{Gardiner2000}.
The rotating wave approximation is typically justified by the argument that the non-mixed terms have higher frequency and are therefore highly oscillatory.
Additionally, Haroche~\cite[p.~127]{Haroche2006} argues that the high-frequency terms do not conserve energy as two particles are created or destroyed in one process.
However, such processes are, in general, possible in the nonlinear regime~\cite{QuesadaMejia2015}.
Using the definition of the positive and negative frequency operators, we find
\begin{equation}
	\hat{H}_{ab}
	=
	-
	\int_{\mathbb{R}^3}
	\frac{\dd[3]{p}}{(2\pi)^3\sqrt{2\omega(\vb{p})}}
	\frac{\dd[3]{p}}{(2\pi)^3\sqrt{2\omega(\vb{q})}}
	\hat{a}(\vb{p})
	e^{-i\omega(\vb{p})t}
	g(t,\vb{p}-\vb{q})
	\hat{b}^\dagger(\vb{q})
	e^{+i\omega(\vb{q})t}
	+
	\text{h.c.}
\end{equation}
wherein $g(t,\vb{p}-\vb{q})$ is the spatial Fourier transform of the coupling $g(t,\vb{x})$.
To preserve momentum, we expect the coupling to be strongly peaked where the incoming matches the outgoing momentum, i.e.,
\begin{equation}
	g(t,\vb{p}-\vb{q})
	\approx
	(2\pi)^3
	\delta^{(3)}(\vb{q}-\vb{p})
	g(t)
\end{equation}
and the coupling Hamiltonian reduces to
\begin{equation}
	\hat{H}_{ab}
	=
	-
	g(t)
	\int_{\mathbb{R}^3}
	\frac{\dd[3]{p}}{(2\pi)^32\omega(\vb{p})}
	\hat{a}(\vb{p})
	\hat{b}^\dagger(\vb{p})
	+
	\text{h.c.}
\end{equation}
from which we can calculate the terms of the Magnus expansion, i.e.,
\begin{equation}
	\Omega^{(1)}(t,t_0)
	=
	iG(t,t_0)
	\int_{\mathbb{R}^3}
	\frac{\dd[3]{p}}{(2\pi)^32\omega(\vb{p})}
	\hat{a}(\vb{p})
	\hat{b}^\dagger(\vb{p})
	-
	\text{h.c.}
\end{equation}
where we defined
\begin{equation}
	G(t,t_0)
	=
	\int_{t_0}^t
	\dd{t^\prime}
	g(t^\prime)
	.
\end{equation}
For the second term in the Magnus expansion, we first calculate the commutator
\begin{equation}
	\comm{\hat{H}_{ab}(t^\prime)}{\hat{H}_{ab}(t^{\prime\prime})}
	=
	g(t^\prime)^*
	g(t^{\prime\prime})
	\int_{\mathbb{R}^3}
	\frac{\dd[3]{p}}{(2\pi)^3(2\omega(\vb{p}))^2}
	\left[
		\hat{n}_a(\vb{p})
		-
		\hat{n}_b(\vb{p})
	\right]
	-
	\text{h.c.}
\end{equation}
where we used the number density operators, e.g., $\hat{n}_a(\vb{p})=\hat{a}^\dagger(\vb{p})\hat{a}(\vb{p})$.
Using the commutators, we find
\begin{equation}
	\comm{\hat{H}_{ab}(t^\prime)}{\hat{H}_{ab}(t^{\prime\prime})}
	=
	2i\Im\left\{
		g(t^\prime)^*
		g(t^{\prime\prime})
	\right\}
	\int_{\mathbb{R}^3}
	\frac{\dd[3]{p}}{(2\pi)^3(2\omega(\vb{p}))^2}
	\left[
		\hat{n}_a(\vb{p})
		-
		\hat{n}_b(\vb{p})
	\right]
	+
	\text{const}
\end{equation}
and we ignore the constant factor as it only contributes to a constant phase.
We then find the second term of the Magnus expansion to be
\begin{equation}
	\Omega^{(2)}(t,t_0)
	=
	-i
	\int_{t_0}^t
	\dd{t^\prime}
	\Im\left\{
		g(t^\prime)
		G(t^\prime,t_0)^*
	\right\}
	\int_{\mathbb{R}^3}
	\frac{\dd[3]{p}}{(2\pi)^3(2\omega(\vb{p}))^2}
	\left[
		\hat{n}_a(\vb{p})
		-
		\hat{n}_b(\vb{p})
	\right]
\end{equation}
which is some product of the coupling correlation and an integral of the the difference in photon number between the modes.
While the difference in photon number can change for a time-dependent state, we ignore the second Magnus term as it contributes an overall phase to the combined output state, however, for mode couplers used in parallel these phase shifts need to be accounted for!
In addition, higher order Magnus terms might become significant for high light intensities.
We conclude that the unitary action of a time-dependent mode coupler is
\begin{equation}
	\hat{U}(t,t_0)
	=
	\exp\left\{
		iG(t,t_0)
		\int_{\mathbb{R}^3}
		\frac{\dd[3]{p}}{(2\pi)^32\omega(\vb{p})}
		\biggl[
			\hat{a}(\vb{p})
			\hat{b}^\dagger(\vb{p})
			+
			\hat{a}^\dagger(\vb{p})
			\hat{b}(\vb{p})
		\biggr]
	\right\}
\end{equation}
which can be considered the field-theoretic time-dependent generalization of the beam splitter action reported in Ref.~\cite{Leonhardt2003,Haroche2006}.

\subsection{Beam splitter}

\subsection{Phase-shifter}
		\section{Modulators}

The (electro-optical) modulators allow us to encode an electrical signal onto an optical carrier, and in that sense, are the interface between the electrical and optical domains.
The domain crossover occurs at the electro-optical phase modulator, which we attempt to describe as a (quantum) nonlinear mixing process mediated by the dielectric, in a first part.
In a second part, we construct, in the optical domain, a complex, in-phase and quadrature, amplitude modulator using \gls{mzi}s driven by electro-optical phase modulators.

\subsection{Phase modulator based on nonlinear frequency conversion}

The electro-optical phase modulator relates an electric signal to a phase shift of an optical signal by changing the refractive index of its transmission medium.
There are many ways to change the refractive index, such as applying an electric field, the charge carrier density, the temperature, or the mechanical strain of the optical transmission medium.
Changing the refractive index directly through an electric field is, for many applications, the most convenient.
The two most relevant electro-optical effects are the Pockels and Kerr effects~\cite[Ch.~18]{Saleh2007}.
The Pockels effect labels a linear change, while the Kerr effect characterizes a quadratic refractive index change in terms of an external electric field.
If a material exhibits an electro-optical effect depends strongly on the crystal symmetry~\cite[p.~237]{Yariv1984}.
In the following, we focus on the linear electro-optic effect, the Pockels effect, which is present in noncentrosymmetric crystals~\cite[p.~2]{Boyd2020}, e.g., lithium niobate, and commonly used in integrated telecommunication modulators.

The Pockels cell is a Pockels medium embedded inside a plate capacitor (\Cref{fig:pockels_cell}) and constitutes a primitive embodiment of a phase modulator.
\begin{figure}[htb]
    \centering
    \includegraphics{figures/tikz/pockels-cell}
    \caption{Pockels cell of length $l$ and diameter $d$: The quantum optical input field, $\hat{t}_1(t)$, enters the cell from the left. Inside the Pockels cell, an electric quantum field, $\hat{E}_2(t)$, created by applying a voltage to the capacitor plates (black rectangles), couples with the input field and produces an quantum optical output field, $\hat{E}_3(t)$.}\label{fig:pockels_cell}
\end{figure}
Applying a voltage signal $V(t)$ to the capacitor plates creates an electric field inside the cell, depicted by the black, low-frequency sine wave in \Cref{fig:pockels_cell}.
An optical input field, $\hat{E}_1(\omega)$, approaches from the left couples to $\hat{E}_1(t)$ over the length of the Pockels cell, $l$.
The optical output field, $\hat{E}_3(t)$, exiting the cell to the right contains frequency components from both the optical input field, $\hat{E}_1(t)$, and the electric field inside the cell, $\hat{E}_2(t)$.

% usual description via refractive index not possible!
% Ref.~\cite{Horoshko2018}, which we briefly summarize.
% which reduces to the phase-rotation operator proposed in Ref.~\cite[p.~38]{Leonhardt2010} and Ref.~\cite[p.~103]{Vogel2006} in the single-mode limit and time-independent $\varphi$.
% The energy of the electromagnetic field inside a dielectric medium is symbolically~\cite[p.~124]{Jackson2007}
%where $v_\text{gr}$ is the group velocity in the quasi-monochromatic approximation~\cite[p.~211]{Jackson2007}.

%\footnote{An introduction to the phase-rotation operator is found in Ref.~\cite[p.~38]{Leonhardt2010} and Ref.~\cite[p.~103]{Vogel2006}.}

The interaction Hamiltonian~\cite[p.~1070]{Mandel1995}
\begin{equation}
	\begin{split}
		\hat{H}_\text{int}(t)
		&=
		\frac{1}{3}
		\int\dd[3]{x}
		\int\frac{\dd{\omega_1}}{2\pi}
		\int\frac{\dd{\omega_2}}{2\pi}
		\int\frac{\dd{\omega_3}}{2\pi}
		\\
		&\times
		\chi_{ijk}(\omega_3=\omega_1+\omega_2;\omega_1-\omega_2,\omega_2)
		\hat{E}^i(\omega_3,\vb{x})
		\hat{E}^j(\omega_1-\omega_2,\vb{x})
		\hat{E}^k(\omega_1,\vb{x})
	\end{split}
\end{equation}

\subsection{Amplitude modulator based on Mach-Zehnder interferometer}

The \gls{mzm} uses two phase modulators to perform amplitude modulation through interference.
The \gls{mzm} enables electrically-driven amplitude modulation if the phase modulators are driven electrically, for instance, using the Pockels effect as discussed previously.
\begin{figure}[htb]
	\centering
	\includestandalone{figures/pstricks/mzi-symmetric}
	\caption{Symmetric \gls{mzm} using free-space optics comprising two balanced \gls{bs}, BS1 and BS2, two mirrors, M1 and M2, and two phase shifters, PS1 and PS2. The input Fourier amplitudes, $\alpha_1(\omega)$ and $\alpha_2(\omega)$, enter BS1 and are split into an upper and lower path. The upper path receives a phase shift of $\phi_1(\omega)+\pi$ from PS1 and M1 before entering BS2 from the top. The lower path receives a phase shift of $\phi_2(\omega)+\pi$ from M2 and PS2 before entering BS2 from the left. BS2 recombines the phase-shifted upper and lower path into the output Fourier amplitudes $\alpha_1^\prime(\omega)$ and $\alpha_2^\prime(\omega)$.}\label{fig:mzi_symmetric}
\end{figure}
\Cref{fig:mzi_symmetric} shows a free-space optics setup of a symmetric \gls{mzi}\footnote{The \gls{mzi} is a static, i.e., time-independent, \gls{mzm}.} with one signal input; the other input being in the vacuum state.
The most crucial components of the \gls{mzi} are a splitter, a coupler, and two independent phase modulators.
The splitter divides the input light into two branches.
Each branch adds a relative phase shift, $\phi_1(\omega)$ and $\phi_2(\omega)$, using an independently  driven phase modulator, PS1 and PS2.
The coupler recombines both branches into two outputs.
Two cubic beam splitters implement the splitter (BS1) and the coupler (BS2) for our free-space setup.
For additional beam alignment, our free-space setup utilizes two mirrors (M1 and M2).

To find the effect of the \gls{mzm} on a coherent input state, we combine the actions of its individual optical components.
Approximating each of the optical components an ideal optical coupler, for which we showed that the Fourier amplitudes transform according to a unitary matrix, lets us write the output Fourier amplitudes of the \gls{mzm} as the matrix product
\begin{equation}
	\vb{\alpha}^\prime(\omega)
	=
	U_\text{MZM}(\omega)
	\vb{\alpha}(\omega)
	=
	U_\text{BS2}(\omega)
	U_\text{PS}(\omega)
	U_\text{BS1}(\omega)
	\vb{\alpha}(\omega)
	\label{eq:mzm_fourier}
	,
\end{equation}
i.e., the matrix transform of the symmetric \gls{mzm}, $U_\text{MZM}$, is equal to the matrix product of the second beam splitter's, the phase shifts', and the first beam splitter's matrix transform, $U_\text{BS2}U_\text{PS}U_\text{BS1}$.
Ignoring the relative phases between the individual components, we use the beam splitter transforms
\begin{align}
	U_\text{BS1}
	&=
	\frac{1}{\sqrt{2}}
	\begin{pmatrix}
		1 & i \\
		i & 1
	\end{pmatrix}
	&
	U_\text{BS2}
	&=
	\frac{1}{\sqrt{2}}
	\begin{pmatrix}
		i & 1 \\
		1 & i
	\end{pmatrix}
	,
\end{align}
corresponding to a perfect cubic beam splitter with a single dielectric layer~\cite[p.~139]{Gerry2005}, where we exchanged the rows for consistency with the input labels.
The matrix encoding the phase shifts from the phase modulation $\phi_1(\omega),\phi_2(\omega)$ and the reflection at the mirrors M1 and M2, $\pi$ is
\begin{equation}
	U_\text{PS}(\omega)
	=
	\begin{pmatrix}
		ie^{i\phi_1(\omega)} & 0 \\
		0 & ie^{i\phi_2(\omega)}
	\end{pmatrix}
\end{equation}
Performing the matrix multiplication and writing the exponentials as trigonometric functions, we find the matrix transform of the symmetric \gls{mzi} to be
\begin{equation}
	U_\text{MZM}(\omega)
	=
	-
	\begin{pmatrix}
		\cos\left(\frac{\phi_2(\omega)-\phi_1(\omega)}{2}\right) & \sin\left(\frac{\phi_2(\omega)-\phi_1(\omega)}{2}\right) \\
		-\sin\left(\frac{\phi_2(\omega)-\phi_1(\omega)}{2}\right) & \cos\left(\frac{\phi_2(\omega)-\phi_1(\omega)}{2}\right)
	\end{pmatrix}
	e^{i\frac{\phi_1(\omega)+\phi_2(\omega)}{2}}
	\label{eq:mzm_matrix1}
	.
\end{equation}
Comparing \cref{eq:mzm_matrix1} with the unitary matrix decomposition \cref{eq:unitary_matrix} suggests that accounting for relative phase would not change the main characteristics significantly or could be compensated by offsetting $\phi_1(\omega)$ or $\phi_2(\omega)$.
It appears useful to define the common-mode and differential-mode phases
\begin{align}
	\phi_+(\omega)
	&=
	\frac{\phi_2(\omega)+\phi_1(\omega)}{2}
	&
	\phi_-(\omega)
	&=
	\frac{\phi_2(\omega)-\phi_1(\omega)}{2}
\end{align}
for which the matrix transform simplifies to
\begin{equation}
	U_\text{MZM}(\omega)
	=
	-
	\begin{pmatrix}
		\cos\phi_-(\omega) & \sin\phi_-(\omega) \\
		-\sin\phi_-(\omega) & \cos\phi_-(\omega)
	\end{pmatrix}
	e^{i\phi_+(\omega)}
	\label{eq:mzm_matrix2}
\end{equation}
and we note that the common-mode phase $\phi_+(\omega)$ adds a global phase shift of $\phi_+(\omega)$ while the differential-mode phase $\phi_-(\omega)$ changes the splitting ratios at the output.

Our result is therefore analog to our result for the spectral filter.
\textcolor{red}{Is this actually correct? Who says that $\phi(\omega)$ and not that $\phi(t)$?}

\begin{figure}[htb]
    \centering
    \includegraphics{figures/tikz/iqm}
    \caption{Integrated \gls{iqm} using three \gls{mzm} arms: A coherent input amplitude, $\alpha(t)$, is split into an upper and lower branch. The upper and lower branches comprise an integrated \gls{mzm} that performs amplitude modulation with the in-phase and quadrature signal, $I(t)$ respectively $Q(t)$. The integrated \gls{mzm} consists of a hexagonal-shaped waveguide with an inside signal electrode and two outer grounds. The outputs of the in-phase and quadrature modulated form a third \gls{mzm} used to set a relative phase shift of $\pi$ between the in-phase and quadrature signals, yielding the coherent output amplitude $\alpha^\prime(t)$.}\label{fig:iqm}
\end{figure}
		\section{Detector}

\subsection{Quantum measurement}

Results we expect from POVM?

\subsection{Classical sink}

Classic electric current from a coherent state.

\subsection{Harmonic oscillator}

Quantum harmonic oscillator coupled to Klein-Gordon field.

\subsection{Ensemble harmonic oscillator}

Ensemble of quantum harmonic oscillators.
Quantum stochastics.
		
		\addcontentsline{toc}{section}{References}
		\printbibliography[title=References]
	\end{refsection}

	\chapter{Coherent state transmission system}
	\begin{refsection}
		\textit{The following chapter describes a coherent state transmission system from a signal-processing point of view, bridging the gap between the quantum theory of optical components and the quantum information theory of the \gls{cvqkd} protocol. In particular, we extend the standard description of a digital transmission system~\cite{Gallager2008,Nossek2015,Oppenheim1989,Proakis2007} to coherent light states following the setup described in Ref.~\cite{Brunner2017}.}
		\section{Overview}

Let $M\subseteq\mathbb{N}$ denote an index set, $\left\{\alpha_n\right\}_{n\in M}$ as well as $\left\{\beta_n\right\}_{n\in M}$ two complex sequences, referred to as symbols.
A coherent state transmission system comprises a transmitter, a quantum channel, and a receiver, see \Cref{fig:transmission_system}.
\begin{figure}[htb]
	\centering
	\includestandalone{figures/tikz/transmission-system}
	\caption{Coherent state transmission system comprising a transmitter, a quantum channel, and a receiver. The transmitter encodes the symbols $\left\{\alpha_n\right\}_{n\in M}$ onto a coherent state $\ket{\alpha(t)}$. The channel maps from $\ket{\alpha(t)}$ to a coherent state $\ket{\beta(t)}$. The receiver decodes a symbol sequence $\left\{\beta_n\right\}_{n\in M}$ from $\ket{\beta(t)}$.}\label{fig:transmission_system}
\end{figure}
The transmitter encodes the symbols $\left\{\alpha_n\right\}_{n\in M}$ onto a coherent state $\ket{\alpha(t)}$ and passes it to the quantum channel.
The quantum channel maps $\ket{\alpha(t)}$ to a coherent state $\ket{\beta(t)}$ at a receiver, which decodes the symbols $\left\{\beta_n\right\}_{n\in M}$ from the coherent state $\ket{\beta(t)}$.
		\section{Transmitter}

\subsection{Mach-Zehnder modulator}

\begin{figure}[htb]
	\centering
	\includestandalone{figures/pstricks/mzi-symmetric}
	\caption{Free-space setup of a symmetric \gls{mzi}: The input light field enters a first beam splitter BS1 from the left. The light field exits BS1 to the right and the bottom. Right of BS1, a first phase shifter adds a relative phase of $\varphi_1$. Right of the first phase shifter, a first mirror M1 reflects the light to the bottom, hitting a second beam splitter BS2 from the top. Below BS1, a second mirror M2 directs the light to the right, where a second phase shifter adds a relative phase of $\varphi_2$, and the light hits BS2 from the left.}
\end{figure}
\begin{equation}
	\begin{split}
		\begin{pmatrix}
			\hat{a}_1^\prime \\
			\hat{a}_2^\prime
		\end{pmatrix}
		&=
		\frac{1}{\sqrt{2}}
		\begin{pmatrix}
			i & 1 \\
			1 & i
		\end{pmatrix}
		\begin{pmatrix}
			ie^{i\varphi_1} & 0 \\
			0 & ie^{i\varphi_2}
		\end{pmatrix}
		\frac{1}{\sqrt{2}}
		\begin{pmatrix}
			1 & i \\
			i & 1
		\end{pmatrix}
		\begin{pmatrix}
			\hat{a}_1 \\
			\hat{a}_2
		\end{pmatrix}
		\\
		&=
		e^{i\frac{\varphi_1+\varphi_2}{2}+i\pi}
		\begin{pmatrix}
			\cos(\frac{\varphi_2-\varphi_1}{2}) & \sin(\frac{\varphi_2-\varphi_1}{2}) \\
			-\sin(\frac{\varphi_2-\varphi_1}{2}) & \cos(\frac{\varphi_2-\varphi_1}{2})
		\end{pmatrix}
		\begin{pmatrix}
			\hat{a}_1 \\
			\hat{a}_2
		\end{pmatrix}
	\end{split}
\end{equation}

\begin{table}[htb]
	\centering
	\begin{tabular}{lcc}
		\toprule
		Configuration & Phase relation & Modulation \\
		\midrule
		Push-pull & $\varphi_1=-\varphi_2$ & Amplitude \\
		Push-push & $\varphi_1=+\varphi_2$ & Phase \\
		\bottomrule
	\end{tabular}
	\caption{Configurations of a symmetric \gls{mzm}.}
\end{table}

\begin{equation}
	\begin{pmatrix}
		\hat{a}_1^\prime \\
		\hat{a}_2^\prime
	\end{pmatrix}
		\begin{pmatrix}
			\cos(\theta) & \sin(\theta) \\
			-\sin(\theta) & \cos(\theta)
		\end{pmatrix}
		\begin{pmatrix}
			\hat{a}_1 \\
			\hat{a}_2
		\end{pmatrix}
\end{equation}

\begin{equation}
	\hat{U}(\theta)
	=
	e^{i\theta\left(
		\hat{a}_1
		\hat{a}_2^\dagger
		-
		\hat{a}_1^\dagger
		\hat{a}_2
	\right)}
\end{equation}

\begin{align}
	\hat{a}_1^\prime
	&=
	\hat{U}^\dagger(\theta)
	\hat{a}_1
	\hat{U}(\theta)
	=
	\cos(\theta)
	\hat{a}_1
	+
	\sin(\theta)
	\hat{a}_2
	\\
	\hat{a}_2^\prime
	&=
	\hat{U}^\dagger(\theta)
	\hat{a}_2
	\hat{U}(\theta)
	=
	\cos(\theta)
	\hat{a}_2
	-
	\sin(\theta)
	\hat{a}_1
\end{align}

\subsection{I/Q modulator}

\begin{figure}[htb]
	\centering
	\includestandalone{figures/pstricks/iqm}
	\caption{\gls{iq}-modulator comprising three \gls{mzm} and connected by an input and output fiber. MZM 1 and MZM 2, are used in push-pull configuration for \gls{am}. MZM 3 is used in push-push configuration for setting the relative phase between the upper and lower branch. Vacuum in- and outputs are indicated by the red dashed fiber.}
\end{figure}

\begin{equation}
	\begin{pmatrix}
		\hat{a}_1^\prime \\
		\hat{a}_2^\prime
	\end{pmatrix}
	=
	\frac{1}{\sqrt{2}}
	\begin{pmatrix}
		 1 & ie^{+i\varphi_1} \\
		 ie^{-i\varphi_1} & 1
	\end{pmatrix}
	\begin{pmatrix}
		\hat{a}_1 \\
		\hat{a}_2
	\end{pmatrix}
\end{equation}

\begin{equation}
	\begin{pmatrix}
		\hat{a}_3^{\prime\prime} \\
		\hat{a}_1^{\prime\prime}
	\end{pmatrix}
	=
	\begin{pmatrix}
		 \cos\theta_1 & \sin\theta_1 \\
		 -\sin\theta_1 & \cos\theta_1
	\end{pmatrix}
	\begin{pmatrix}
		\hat{a}_3^\prime \\
		\hat{a}_1^\prime
	\end{pmatrix}
\end{equation}

\begin{equation}
	\begin{pmatrix}
		\hat{a}_1^{\prime\prime} \\
		\hat{a}_2^{\prime\prime} \\
		\hat{a}_3^{\prime\prime} \\
		\hat{a}_4^{\prime\prime}
	\end{pmatrix}
	=
	\begin{pmatrix}
		 \cos\theta_1 & 0 & -\sin\theta_1 & 0 \\
		 0 & \cos\theta_2 & 0 & \sin\theta_2 \\
		 \sin\theta_1 & 0 & \cos\theta_1 & 0 \\
		 0 & -\sin\theta_2 & 0 & \cos\theta_2 \\
	\end{pmatrix}
	\begin{pmatrix}
		\hat{a}_1^{\prime} \\
		\hat{a}_2^{\prime} \\
		\hat{a}_3^{\prime} \\
		\hat{a}_4^{\prime}
	\end{pmatrix}
\end{equation}

\begin{equation}
	\begin{pmatrix}
		\hat{a}_2^{\prime\prime} \\
		\hat{a}_4^{\prime\prime}
	\end{pmatrix}
	=
	\begin{pmatrix}
		 \cos\theta_2 & \sin\theta_2 \\
		 -\sin\theta_2 & \cos\theta_2
	\end{pmatrix}
	\begin{pmatrix}
		\hat{a}_2^\prime \\
		\hat{a}_4^\prime
	\end{pmatrix}
\end{equation}


\begin{equation}
	\begin{split}
		\begin{pmatrix}
			\hat{a}_1^{\prime\prime\prime\prime} \\
			\hat{a}_2^{\prime\prime\prime\prime}
		\end{pmatrix}
		&=
		\frac{1}{\sqrt{2}}
		\begin{pmatrix}
			1 & ie^{+i\varphi_2} \\
			ie^{-i\varphi_2} & 1
		\end{pmatrix}
		\begin{pmatrix}
			\hat{a}_1^{\prime\prime\prime} \\
			\hat{a}_2^{\prime\prime\prime}
		\end{pmatrix}
		\\
		&=
		\frac{1}{\sqrt{2}}
		\begin{pmatrix}
			1 & ie^{+i\varphi_2} \\
			ie^{-i\varphi_2} & 1
		\end{pmatrix}
		\begin{pmatrix}
			e^{+i\phi} & 0 \\
			0 & e^{-i\phi}
		\end{pmatrix}
		\begin{pmatrix}
			\hat{a}_1^{\prime\prime} \\
			\hat{a}_2^{\prime\prime}
		\end{pmatrix}
		\\
		&=
		\frac{1}{\sqrt{2}}
		\begin{pmatrix}
			e^{+i\phi} & ie^{+i(\varphi_2-\phi)} \\
			ie^{-i(\varphi_2-\phi)} & e^{-i\phi}
		\end{pmatrix}
		\begin{pmatrix}
			\hat{a}_1^{\prime\prime} \\
			\hat{a}_2^{\prime\prime}
		\end{pmatrix}
	\end{split}
\end{equation}
		\section{Receiver}
\FloatBarrier

We introduced the receiver as a component decoding a sequence of complex symbols, $\left\{\beta_n\in\mathbb{C}\colon n\in I\right\}$, from a coherent state $\ket{\beta(t)}$.
As with the transmitter, we want our receiver to be software-defined, meaning that the predominant signal processing is in the digital domain and we need to down-convert the optical signal.
\begin{figure}[htb]
	\centering
	\includegraphics{figures/tikz/software-defined-receiver}
	\caption{Block diagram of the receiver's signal processing domains. The analog electrical signals $v(t)$ and $w(t)$ are demodulated from the quadratures of the optical coherent state $\ket{\beta(t)}$, and then converted to the digital signals $v[m]$ and $w[m]$ from which the \gls{dsp} decodes the symbol sequence $\{\beta_n\in\mathbb{C}\colon n\in I\}$.}\label{fig:software_defined_receiver}
\end{figure}
\Cref{fig:software_defined_receiver} illustrates the signal flow across the different domains inside the receiver.
Unlike the transmitter up-conversion, the receiver's down-conversion leaves us some design freedom.
In particular, it is possible to reduce the hardware complexity significantly by introducing an intermediate frequency, which carries over to the digital domain.

\FloatBarrier
\subsection{Down-conversion}

% outlook down-conversion freedoms

\begin{figure}[htb]
	\centering
	\includegraphics{figures/circuitikz/down-conversion-single}
	\caption{Block diagram of single quadrature down-conversion. The signal $z(t)$ is mixed with the \gls{lo} $\sin(\omega_lt+\vartheta)$. The output of the mixing is filtered by a \gls{lp} to produce the down-converted signal $u(t)$.}\label{fig:down_conversion_single}
\end{figure}
The mixing of the real-valued passband signal,
\begin{equation}
	\begin{split}
		z(t)
		=
		\int_{-\infty}^{+\infty}\frac{\dd{\omega}}{2\pi}
		z(\omega)
		e^{+i\omega t}
		&=
		\int_0^{+\infty}\frac{\dd{\omega}}{2\pi}
		z(\omega)
		e^{+i\omega t}
		+
		\int_{-\infty}^0\frac{\dd{\omega}}{2\pi}
		z(\omega)
		e^{+i\omega t}
		\\
		&=
		\int_0^{+\infty}\frac{\dd{\omega}}{2\pi}
		z(\omega)
		e^{+i\omega t}
		-
		\int_{\infty}^0\frac{\dd{\omega}}{2\pi}
		z(-\omega)
		e^{-i\omega t}
		\\
		&=
		\int_0^{+\infty}\frac{\dd{\omega}}{2\pi}
		\left[
			z(\omega)
			e^{+i\omega t}
			+
			z(\omega)^*
			e^{-i\omega t}
		\right]
		,
	\end{split}
\end{equation}
with the \gls{lo} signal
\begin{equation}
	\sin(\omega_lt+\vartheta)
	=
	\frac{1}{2i}
	\left[
		e^{+i(\omega_lt+\vartheta)}
		-
		e^{-i(\omega_lt+\vartheta)}
	\right]
\end{equation}
produces a high- and low-frequency band
\begin{equation}
	\begin{split}
		z(t)
		\sin(\omega_lt+\vartheta)
		&=
		\Im
		\int_0^\infty\frac{\dd{\omega}}{2\pi}
		\left[
			e^{+i(\omega-\omega_l)t}
			+
			e^{+i(\omega+\omega_l)t}
		\right]
		z(\omega)
		e^{+i\vartheta}
		\\
		&=
		\Im
		\int_0^\infty\frac{\dd{\omega}}{2\pi}
		\left[
			z(\omega+\omega_l)
			+
			z(\omega-\omega_l)
		\right]
		e^{+i\omega t+i\vartheta}
		.
	\end{split}
\end{equation}
The high-frequency band is highly suppressed by the ideal \gls{lp},
\begin{equation}
	\begin{split}
		v(t)
		&=
		\Im
		\int_0^{B/2}\frac{\dd{\omega}}{2\pi}
		\left[
			z(\omega+\omega_l)
			+
			z(\omega-\omega_l)
		\right]
		e^{+i(\omega t+\vartheta)}
		\\
		&\approx
		\Im
		\int_0^{B/2}\frac{\dd{\omega}}{2\pi}
		z(\omega-\omega_l)
		e^{+i(\omega t+\vartheta)}
		,
	\end{split}
	\label{eq:down_conversion_imag}
\end{equation}
wherein $B$ denotes the effective detector bandwidth.

% compare result with derived photocurrent for balanced detection
% problem of only one quadrature ->
% reference to active and passive coherent receiver from QKD chapter

\begin{figure}[htb]
	\centering
	\includegraphics{figures/circuitikz/down-conversion-dual}
	\caption{Block diagram of dual quadrature down-conversion. The signal $z(t)$ is divided with equal power among an upper and a lower branch. The upper branch is mixed with the phase shifted $\gls{lo}$ signal $\cos(\omega_ct+\vartheta)$. The lower branch is mixed with \gls{lo} signal $\sin(\omega_ct+\vartheta)$. The mixer outputs are filtered separately by a \gls{lp} yielding the down-converted signals $v(t)$ and $w(t)$.}\label{fig:down_conversion_dual}
\end{figure}
\Cref{fig:down_conversion_dual} shows the block diagram corresponding to down-conversion of both quadratures.
The lower branch is equal to the result of the single quadrature downconversion after accounting for the power loss due to the signal split, i.e.,
\begin{equation}
	u(t)
	=
	\Re
	\int_0^B\frac{\dd{\omega}}{2\pi}
	z(\omega+\omega_l)
	e^{+i(\omega t+\vartheta)}
	\label{eq:down_conversion_real}	
\end{equation}

% alternative: heterodyne receiver with IF and digital down-conversion

\begin{table}[htb]
  \centering
  \begin{tabular}{lccccc}
    \toprule
    Receiver design & Homodyne (single) & Homodyne (dual) & Heterodyne \\
    \midrule
    Balanced detectors & \num{1} & \num{2} & \num{1} \\
    Quadratures & \num{1} & \num{2} & \num{2} \\
    Optical complexity & Low & High & Low \\
    Signal bandwidth & High & High & Low \\
    \gls{lo} synchronization & Frequency and phase & Frequency & Bandwidth \\
    \bottomrule
  \end{tabular}
  \caption{Comparison of receiver implementations according to Ref.~\cite{Brunner2017}: The single quadrature homodyne detection offers low optical complexity and high bandwidth but only resolves one of two quadratures and required frequency and phase synchronization of the \gls{lo}. The dual quadrature homodyne detection resolves both quadratures with high bandwidth but requires two balanced detectors increasing the optical complexity and phase synchronization of the \gls{lo}. The heterodyne detection schemes resolves both quadratures with low complexity and no requirements on \gls{lo} synchronization at the cost of signal bandwidth.}\label{tab:receivers}
\end{table}

% heterodyne receiver generalizes (single) homodyne receiver

\FloatBarrier
\subsection{Symbol decoding}

\begin{figure}[htb]
	\centering
	\includegraphics{figures/circuitikz/symbol-decoding}
	\caption{foo}
\end{figure}

\begin{figure}[htb]
	\centering
	\includegraphics{figures/pgfplots/rx-frequency}
	\caption{foo}
\end{figure}

\begin{figure}[htb]
	\centering
	\includegraphics{figures/pgfplots/rx-rand-time}
	\caption{foo}
\end{figure}

		\addcontentsline{toc}{section}{References}
		\printbibliography[title=References]
	\end{refsection}
	
	\chapter{Classical post-processing}
	\begin{refsection}
		\textit{The following chapter describes a \gls{cvqkd} system by extending the previously discussed coherent state transmission system with the missing pieces to create a shared secret key from the correlated measurements of the coherent states.}
		\section{Overview}

Let $M\subseteq\mathbb{N}$ denote an index set, $\left\{\alpha_n\right\}_{n\in M}$ as well as $\left\{\beta_n\right\}_{n\in M}$ two complex sequences, referred to as symbols.
A coherent state transmission system comprises a transmitter, a quantum channel, and a receiver, see \Cref{fig:transmission_system}.
\begin{figure}[htb]
	\centering
	\includestandalone{figures/tikz/transmission-system}
	\caption{Coherent state transmission system comprising a transmitter, a quantum channel, and a receiver. The transmitter encodes the symbols $\left\{\alpha_n\right\}_{n\in M}$ onto a coherent state $\ket{\alpha(t)}$. The channel maps from $\ket{\alpha(t)}$ to a coherent state $\ket{\beta(t)}$. The receiver decodes a symbol sequence $\left\{\beta_n\right\}_{n\in M}$ from $\ket{\beta(t)}$.}\label{fig:transmission_system}
\end{figure}
The transmitter encodes the symbols $\left\{\alpha_n\right\}_{n\in M}$ onto a coherent state $\ket{\alpha(t)}$ and passes it to the quantum channel.
The quantum channel maps $\ket{\alpha(t)}$ to a coherent state $\ket{\beta(t)}$ at a receiver, which decodes the symbols $\left\{\beta_n\right\}_{n\in M}$ from the coherent state $\ket{\beta(t)}$.
		\section{Reconciliation}

% classical post-processing in DV-QKD
% classical post-processing in CV-QKD (differences)
		\section{Error correction}

% LDPC codes
		\section{Privacy amplification}

% XOR-ing using Toeplitz matrices

		\addcontentsline{toc}{section}{References}
		\printbibliography[title=References]
	\end{refsection}

	\chapter{Conclusion and outlook}
	\begin{refsection}
		\chapter*{Conclusion and outlook}
\addcontentsline{toc}{chapter}{Conclusion and outlook}

% security proof using logical quantum channel
% odfm-based protocols

\begin{figure}[htb]
	\centering
	\includestandalone{figures/circuitikz/ofdm}
	\caption{\Gls{cvqkd} transmission system from a signal-processing perspective.}
\end{figure}


\begin{enumerate}
	\item Security proofs (logical quantum channel)
	\item New protocols inspired from telecommunications (ODFM)
	\item Frequency-entangled states (Generalization of two-mode squeezing)
\end{enumerate}

		\addcontentsline{toc}{section}{References}
		\printbibliography[title=References]
	\end{refsection}

	\appendix

\end{document}

\documentclass[a4paper,appendixprefix,parskip]{scrreprt}

\usepackage[utf8]{inputenc}
\usepackage{amsthm}
\usepackage{amsmath}
\usepackage{amssymb}
\usepackage{authblk}
\usepackage[english]{babel}
\usepackage[backend=biber]{biblatex}
\usepackage{booktabs}
\usepackage{csquotes}
\usepackage[acronym,nonumberlist,toc]{glossaries}
\usepackage{hyperref}
\usepackage{cleveref}
\usepackage{physics}
\usepackage{siunitx}
\usepackage[subpreambles=true]{standalone}
\usepackage{subfig}

\addbibresource{references/articles.bib}
\addbibresource{references/books.bib}

% add bibliography as section (not chapter)
% https://tex.stackexchange.com/questions/568580/make-the-bibliography-as-a-section-in-each-included-chapter
\defbibheading{bibliography}[\bibname]{\section*{#1}}

% custom math commands (transpose, Re, ...)
% approximately proportional to symbol
% https://tex.stackexchange.com/questions/33538/how-to-get-an-approximately-proportional-to-symbol
\def\app#1#2{%
    \mathrel{%
        \setbox0=\hbox{$#1\sim$}%
        \setbox2=\hbox{%
            \rlap{\hbox{$#1\propto$}}%
            \lower1.1\ht0\box0%
        }%
        \raise0.25\ht2\box2%
    }%
}
\def\approxprop{\mathpalette\app\relax}

% overwrite real and imaginary part operators
\let\Re\undefined
\let\Im\undefined
\DeclareMathOperator{\Re}{\operatorname{Re}}
\DeclareMathOperator{\Im}{\operatorname{Im}}
% other functions
\DeclareMathOperator{\sinc}{\operatorname{sinc}}

% transpose
% https://tex.stackexchange.com/questions/403104/small-caps-mathsf-font-for-writing-transpose-of-a-matrix
\newcommand{\trans}{\scriptscriptstyle\mathsf{T}}

% theorems
\newtheorem{theorem}{Theorem}[section]
\newtheorem{lemma}[theorem]{Lemma}
\newtheorem{corollary}[theorem]{Corollary}
\theoremstyle{definition}
\newtheorem{definition}{Definition}[section]
\newtheorem{conjecture}{Conjecture}[section]
\newtheorem{example}{Example}[section]
\theoremstyle{remark}
\newtheorem*{remark}{Remark}

% prefix equation numbers with section number
\numberwithin{equation}{section}

% optics
\newacronym{ar}{AR}{anti-reflective}
\newacronym{mzm}{MZM}{Mach-Zehnder modulator}
\newacronym{bs}{BS}{beam splitter}
\newacronym{fc}{FC}{fiber coupler}
\newacronym{qe}{QE}{quantum efficiency}

% physics
\newacronym{dv}{DV}{discrete-variable}
\newacronym{cv}{CV}{continuous-variable}
\newacronym{dof}{DOF}{degrees of freedom}
\newacronym{eom}{EOM}{equation(s) of motion}
\newacronym{pbc}{PBC}{periodic boundary conditions}
\newacronym{bch}{BCH}{Baker-Campbell-Hausdorff}
\newacronym{ccr}{CCR}{canonical commutation relation}

% signal processing
\newacronym{dsp}{DSP}{digital signal processing}
\newacronym{lo}{LO}{local oscillator}
\newacronym{if}{IF}{intermediate frequency}
\newacronym{lp}{LP}{low-pass}
\newacronym{adc}{ADC}{analog-to-digital converter}
\newacronym{dac}{DAC}{digital-to-analog converter}
\newacronym{qam}{QAM}{quadrature amplitude modulation}
\newacronym{qpsk}{QPSK}{quadrature phase-shift keying}

% quantum-key distribution
\newacronym{qkd}{QKD}{quantum-key distribution}
\newacronym{dvqkd}{DV-QKD}{discrete-variable quantum-key distribution}
\newacronym{cvqkd}{CV-QKD}{continuous-variable quantum-key distribution}

\title{A theoretical framework for quantum optical communication - towards CV-QKD}
\author{Bodo Kaiser}
\affil{\textit{bodo.kaiser@huawei.com}}

\begin{document}

	\maketitle
	\tableofcontents

	\chapter{Introduction}
	\begin{refsection}	
		\addcontentsline{toc}{section}{References}
		\printbibliography[title=References]
	\end{refsection}

	\chapter{Quantum field theory of light}
	\begin{refsection}
		\section{Relativistic field theory}

\rem
\begin{proof}
	According to the action principle, the dynamics of the field are determined by the equations of motion which can be found by the relativistic Euler-Lagrange equations
	\begin{equation*}
		0
		=
		\partial_\mu\pdv{\mathcal{L}}{(\partial_\mu\phi)}
		-
		\pdv{\mathcal{L}}{\phi}
		=
		\left(
			\partial_\mu\partial^\mu
			+
			m^2
		\right)
		\phi(t,\vb{x})
		.
	\end{equation*}
	Assuming the existence of the Klein-Gordon field's Fourier representation\footnote{See Ref.~\cite[p.~341]{Cohen2019} for a definition of a Minkowski Fourier transform.}
	\begin{equation*}
		\phi(t,\vb{x})
		=
		\int_{\mathbb{R}^3}\frac{\dd[3]{p}}{(2\pi)^3}
		\phi(t,\vb{p})
		e^{-i\vb{p}\vdot\vb{x}}
		=
		\int_{\mathbb{R}^4}\frac{\dd[4]{p}}{(2\pi)^4}
		\phi(p_0,\vb{p})
		e^{+ip_\mu x^\mu}
		,
	\end{equation*}
	the equation of motion in momentum space reduces to
	\begin{equation*}
		0
		=
		\left(
			ip_\mu ip^\mu
			+
			m^2
		\right)
		\phi(p_0,\vb{p})
		=
		-
		\left(
			p_0^2
			-
			\omega(\vb{p})^2
		\right)
		\phi(p_0,\vb{p})
	\end{equation*}
	which is satisfied if $p_0=\pm\omega(\vb{p})$.
\end{proof}

\kgmodeexp
\begin{proof}
	There are two approaches to prove \cref{thm:kg_mode_expansion}.
	In a first approach, we calculate with the proposed Fourier expansion
	\begin{equation*}
		\begin{split}
			\partial_\mu
			\partial^\mu
			\phi(t,\vb{x})
			&=
			\int\frac{\dd[3]{p}}{(2\pi)^3\sqrt{2\omega(\vb{p})}}
			\left\{
				a^*(\vb{p})
				\partial_\mu
				\partial^\mu
				e^{-ip_\mu x^\mu}
				+
				\text{c.c.}
			\right\}_{p_0=\omega(\vb{p})}
			\\
			&=
			\int\frac{\dd[3]{p}}{(2\pi)^3\sqrt{2\omega(\vb{p})}}
			\left\{
				a^*(\vb{p})
				\left(
					-
					p_\mu
					p^\mu
				\right)
				e^{-ip_\mu x^\mu}
				+
				\text{c.c.}
			\right\}_{p_0=\omega(\vb{p})}
		\end{split}
	\end{equation*}
	where $p_\mu p^\mu=\omega(\vb{p})^2-\vb{p}=m^2$ and therefore the equation of motion
	\begin{equation*}
			\partial_\mu
			\partial^\mu
			\phi(t,\vb{x})
			=
			-
			m^2
			\phi(t,\vb{x})
	\end{equation*}
	is satisfied.
	While the first approach successfully shows why the theorem is true, it is not obvious how to arrive at the Fourier expansion.
	Therefor, a second approach starts with the Fourier transform of the Klein-Gordon field
	\begin{equation*}
		\phi(t,\vb{x})
		=
		\int_V\frac{\dd[4]{p}}{(2\pi)^4}
		\phi(p_0,\vb{p})
		e^{+ip_\mu x^\mu}
	\end{equation*}
	where the integration domain is constrained to the momentum lightcone $V$ and therefore $\omega(\vb{p})^2=p_0^2$ is automatically satisfied.
	We are left to rewrite the constrained integration domain
	\begin{equation*}
		\phi(t,\vb{x})
		=
		\int_{\mathbb{R}^4}\frac{\dd[4]{p}}{(2\pi)^3}
		\delta^{(1)}\left(p_0^2-\omega(\vb{p})^2\right)
		\phi(p_0,\vb{p})
		e^{+ip_\mu x^\mu}
	\end{equation*}
	and using the composition property of the delta distribution
	\begin{equation*}
		\delta^{(1)}\left(p_0^2-\omega(\vb{p})^2\right)
		=
		\frac{
			\delta^{(1)}\left(p_0-\omega(\vb{p})\right)
			+
			\delta^{(1)}\left(p_0+\omega(\vb{p})\right)
		}{\sqrt{2\omega(\vb{p})}}
	\end{equation*}
	which leaves us with
	\begin{equation*}
		\phi(t,\vb{x})
		=
		\int_{\mathbb{R}^3}\frac{\dd[3]{p}}{(2\pi)^3\sqrt{2\omega(\vb{p})}}
		\biggl\{
			\phi(\omega(\vb{p}),\vb{p})
			e^{+i\omega(\vb{p})t}
			+
			\phi(-\omega(\vb{p}),\vb{p})
			e^{-i\omega(\vb{p})t}
		\biggr\}
		e^{-i\vb{p}\vdot\vb{x}}
		.
	\end{equation*}
	We now only need to perform the substitution $\vb{p}\to-\vb{p}$ on the second term
	\begin{equation*}
		\begin{split}
			\phi(t,\vb{x})
			&=
			\int_{\mathbb{R}^3}\frac{\dd[3]{p}}{(2\pi)^3\sqrt{2\omega(\vb{p})}}
			\biggl\{
				\phi(\omega(\vb{p}),\vb{p})
				e^{+i\omega(\vb{p})t}
				e^{-i\vb{p}\vdot\vb{x}}
				+
				\phi(-\omega(\vb{p}),\vb{p})
				e^{-i\omega(\vb{p})t}
				e^{-i\vb{p}\vdot\vb{x}}
			\biggr\}
			\\
			&=
			\int_{\mathbb{R}^3}\frac{\dd[3]{p}}{(2\pi)^3\sqrt{2\omega(\vb{p})}}
			\biggl\{
				\phi(\omega(\vb{p}),\vb{p})
				e^{+i\omega(\vb{p})t}
				e^{-i\vb{p}\vdot\vb{x}}
				+
				\phi(-\omega(\vb{p}),-\vb{p})
				e^{-i\omega(\vb{p})t}
				e^{+i\vb{p}\vdot\vb{x}}
			\biggr\}
			\\
			&=
			\int_{\mathbb{R}^3}\frac{\dd[3]{p}}{(2\pi)^3\sqrt{2\omega(\vb{p})}}
			\biggl\{
				\phi(\omega(\vb{p}),\vb{p})
				e^{+ip_\mu x^\mu}
				+
				\phi(\omega(\vb{p}),\vb{p})^*
				e^{-ip_\mu x^\mu}
			\biggr\}_{p_0=\omega(\vb{p})}
		\end{split}
	\end{equation*}
	and use the conjugate symmetry of the Fourier amplitudes
	\begin{equation*}
		\phi(p_0,\vb{p})^*
		=
		\phi(-p_0,-\vb{p})
	\end{equation*}
	which is present because $\phi(t,\vb{x})$ is real-valued.
\end{proof}

\Cref{thm:kg_energy_density}, \Cref{thm:kg_energy_momentum}, and \Cref{thm:kg_conjugate_momentum} are found in standard literature on relativistic field theory, for instance, in Ref.~\cite{Peskin1995}.

\section{Canonical quantization}

\qkgcommac
\begin{proof}
	The equal-time commutation relations are valid for any time $t$ so also for $t=0$, inserting \cref{eq:qkg_pos} into \cref{eq:qkg_comm_pm}, we find
	\begin{equation*}
		\begin{split}
			0
			=
			\comm{\hat\phi(0,\vb{x})}{\hat\phi(0,\vb{y})}
			&=
			\int\frac{\dd[3]{p}}{(2\pi)^3}
			\frac{1}{\sqrt{2\omega(\vb{p})}}
			\int\frac{\dd[3]{q}}{(2\pi)^3}
			\frac{1}{\sqrt{2\omega(\vb{q})}}
			\\
			&\times
			\comm{
				\hat{a}(\vb{p})
				e^{+i\vb{p}\vdot\vb{x}}
				+
				\hat{a}^\dagger(\vb{p})
				e^{-i\vb{p}\vdot\vb{x}}
			}{
				\hat{a}(\vb{q})
				e^{+i\vb{q}\vdot\vb{y}}
				+
				\hat{a}^\dagger(\vb{q})
				e^{-i\vb{q}\vdot\vb{y}}
			}
			,
		\end{split}
	\end{equation*}
		and for \cref{eq:qkg_mom}, we find
	\begin{equation*}
		\begin{split}
			0
			=
			\comm{\hat\phi(0,\vb{x})}{\hat\phi(0,\vb{y})}
			&=
			\int\frac{\dd[3]{p}}{(2\pi)^3}
			\left(
				-i
				\sqrt{\frac{\omega(\vb{p})}{2}}
			\right)
			\int\frac{\dd[3]{q}}{(2\pi)^3}
			\left(
				-i
				\sqrt{\frac{\omega(\vb{q})}{2}}
			\right)
			\\
			&\times
			\comm{
				\hat{a}(\vb{p})
				e^{+i\vb{p}\vdot\vb{x}}
				-
				\hat{a}^\dagger(\vb{p})
				e^{-i\vb{p}\vdot\vb{x}}
			}{
				\hat{a}(\vb{q})
				e^{+i\vb{q}\vdot\vb{y}}
				-
				\hat{a}^\dagger(\vb{q})
				e^{-i\vb{q}\vdot\vb{y}}
			}
			.
		\end{split}
	\end{equation*}
	The double integral is zero iff the integrand is zero, therefore
	\begin{align*}
		\comm{
			\hat{a}(\vb{p})
			e^{+i\vb{p}\vdot\vb{x}}
			+
			\hat{a}^\dagger(\vb{p})
			e^{-i\vb{p}\vdot\vb{x}}
		}{
			\hat{a}(\vb{q})
			e^{+i\vb{q}\vdot\vb{y}}
			+
			\hat{a}^\dagger(\vb{q})
			e^{-i\vb{q}\vdot\vb{y}}
		}
		&=
		0
		\\
		\comm{
			\hat{a}(\vb{p})
			e^{+i\vb{p}\vdot\vb{x}}
			-
			\hat{a}^\dagger(\vb{p})
			e^{-i\vb{p}\vdot\vb{x}}
		}{
			\hat{a}(\vb{q})
			e^{+i\vb{q}\vdot\vb{y}}
			-
			\hat{a}^\dagger(\vb{q})
			e^{-i\vb{q}\vdot\vb{y}}
		}
		&=
		0
		.
	\end{align*}
	Adding and subtracting these equations from another implies
	\begin{equation*}
		\comm{\hat{a}(\vb{p})}{\hat{a}(\vb{q})}
		=
		0
		=
		\comm{\hat{a}^\dagger(\vb{p})}{\hat{a}^\dagger(\vb{q})}
		.
	\end{equation*}
	Finally, inserting \cref{eq:qkg_pos} and \cref{eq:qkg_mom} into \cref{eq:qkg_comm_pm}, reveals
	\begin{equation*}
		\begin{split}
			i\delta^{(3)}(\vb{x}-\vb{y})
			=
			\comm{\hat\phi(0,\vb{x})}{\hat\pi(0,\vb{y})}
			&=
			\int\frac{\dd[3]{p}}{(2\pi)^3}
			\frac{1}{\sqrt{2\omega(\vb{p})}}
			\int\frac{\dd[3]{q}}{(2\pi)^3}
			\left(
				-i
				\sqrt{\frac{\omega(\vb{q})}{2}}
			\right)
			\\
			&\times
			\comm{
				\hat{a}(\vb{p})
				e^{+i\vb{p}\vdot\vb{x}}
				+
				\hat{a}^\dagger(\vb{p})
				e^{-i\vb{p}\vdot\vb{x}}
			}{
				\hat{a}(\vb{q})
				e^{+i\vb{q}\vdot\vb{y}}
				-
				\hat{a}^\dagger(\vb{q})
				e^{-i\vb{p}\vdot\vb{y}}
			}
			\\
			&=
			\int\frac{\dd[3]{p}}{(2\pi)^3}
			\int\frac{\dd[3]{q}}{(2\pi)^3}
			\left(
				-
				\frac{i}{2}
				\sqrt{\frac{\omega(\vb{q})}{\omega(\vb{p})}}
			\right)
			\\
			&\times
			\left\{
				-
				\comm{\hat{a}(\vb{p})}{\hat{a}^\dagger(\vb{q})}
				e^{+i\vb{p}\vdot\vb{x}}
				e^{-i\vb{q}\vdot\vb{y}}
				+
				\comm{\hat{a}^\dagger(\vb{p})}{\hat{a}(\vb{q})}
				e^{-i\vb{p}\vdot\vb{x}}
				e^{+i\vb{q}\vdot\vb{y}}
			\right\}
		\end{split}
	\end{equation*}
	which is satisfied for
	\begin{equation*}
		\comm{\hat{a}(\vb{p})}{\hat{a}^\dagger(\vb{q})}
		=
		(2\pi)^3
		\delta^{(3)}(\vb{q}-\vb{p})
		.
	\end{equation*}
\end{proof}

\qkgcommpn
\begin{proof}
	That positive respective negative frequency Klein-Gordon operators commute follows from \Cref{thm:qkg_comm_ac}.
	Using \Cref{thm:qkg_comm_ac} for the remaining commutator
	\begin{equation*}
		\begin{split}
			\comm{\hat\phi^+(x^\mu)}{\hat\phi^-(y^\mu)}
			&=
			\int\frac{\dd[3]{p}}{(2\pi)^3\sqrt{2\omega(\vb{p})}}
			\int\frac{\dd[3]{q}}{(2\pi)^3\sqrt{2\omega(\vb{q})}}
			\comm{\hat{a}(\vb{p})}{\hat{a}^\dagger(\vb{p})}
			\eval{
				e^{-ip_\mu x^\mu}
				e^{+iq_\mu y^\mu}
			}_{\substack{p_0=\omega(\vb{p})\\q_0=\omega(\vb{q})}}
			\\
			&=
			\int\frac{\dd[3]{p}}{(2\pi)^32\omega(\vb{p})}
			\eval{
				e^{-ip_\mu(x^\mu-y^\mu)}
			}_{p_0=\omega(\vb{p})}
			=
			D(x^\mu-y^\mu)
		\end{split}
	\end{equation*}
	recovers the initial claim after identification of the Klein-Gordon propagator.
\end{proof}
\qkgpropcorr
\begin{proof}
	Decomposing the Klein-Gordon operator into positive and negative frequency and using \Cref{thm:vacuum_state_pn}, we find
	\begin{equation*}
		\begin{split}
			\expval{\hat\phi(x^\mu)\hat\phi(y^\mu)}{0}
			&=
			\expval{\hat\phi^+(x^\mu)\hat\phi^-(y^\mu)}{0}
			\\
			&=
			\expval{\comm{\hat\phi^+(x^\mu)}{\hat\phi^-(y^\mu)}}{0}
			\\
			&=
			\expval{D(x^\mu-y^\mu)}{0}
			\\
			&=
			D(x^\mu-y^\mu)
		\end{split}
	\end{equation*}
	where we used \Cref{thm:qkg_comm_pn}.
\end{proof}

\section{Relativistic wave packets}

\nonrelativisticgaussianmom
\begin{proof}
	For massless particles, we have $p_\mu p^\mu=0=k_\mu k^\mu$ and \cref{eq:covariant_gaussian_spectrum} reduces to
	\begin{equation*}
		f(\vb{p})
		\propto
		\exp\left\{
			-
			\frac{\omega(\vb{p})\omega(\vb{k})-\vb{p}\vdot\vb{k}}{2\sigma^2}
		\right\}
		.
	\end{equation*}
	For $\omega(\vb{p})\ll\sigma$, we perform a Taylor expansion around $\vb{k}$
	\begin{equation*}
		\omega(\vb{p})
		=
		\omega(\vb{k})
		+
		\omega^\prime(\vb{k})
		\vdot
		(\vb{p}-\vb{k})
		+
		\frac{1}{2}
		(\vb{p}-\vb{k})^\trans
		\omega^{\prime\prime}(\vb{k})
		(\vb{p}-\vb{k})
		+
		\order{\vb{p}^3}
	\end{equation*}
	where $\omega^\prime(\vb{k})=\vb{k}/\omega(\vb{k})$ denotes the gradient and $\omega^{\prime\prime}(\vb{k})$ the Hessian of $\omega(\vb{p})$ evaluated at $\vb{k}$ and further reducing the spectrum to
	\begin{equation*}
		f(\vb{p})
		\propto
		\exp\left\{
			-
			\frac{
				(\vb{p}-\vb{k})^\trans
				\omega(\vb{k})
				\omega^{\prime\prime}(\vb{k})
				(\vb{p}-\vb{k})
			}{4\sigma^2}
		\right\}
		.
	\end{equation*}
	The expressions
	\begin{equation*}
		\omega(\vb{k})
		\omega^{\prime\prime}(\vb{k})_{ij}
		=
		\omega(\vb{k})
		\pdv{\omega(\vb{k})}{k_i}{k_j}
		=
		\delta_{ij}
		-
		\frac{k_ik_j}{\omega(\vb{k})^2}
		=
		\delta_{ij}
		-
		\frac{k_ik_j}{\vb{k}^2}
	\end{equation*}
	turns out to be equal to the transverse projector $P_T(\vb{k})$ which projects out the components orthogonal to $\vb{k}$.
	Inserting the transverse project, we find the non-relativistic Gaussian spectrum to be approximately equal to
	\begin{equation*}
		f(\vb{p})
		\propto
		\exp\left\{
			-
			\frac{
				(\vb{p}-\vb{k})^\trans
				P_T(\vb{k})
				(\vb{p}-\vb{k})
			}{4\sigma^2}
		\right\}
		=
		\exp\left\{
			-
			\frac{
				\vb{p}_T^2
			}{4\sigma^2}
		\right\}
		.
	\end{equation*}
	The final result is confirmed by Ref.~\cite[eq.~(25)]{Naumov2013} when removing the longitudinal momentum component.
\end{proof}

\section{Number states}

\qkgnumbersingleinner
\begin{proof}
	Inserting the definition of the number state for $n=1$
	\begin{equation*}
		\begin{split}
			\braket{1_f}{1_g}
			&=
			\expval{
				\left(
					\int\frac{\dd[3]{p}}{(2\pi)^3\sqrt{2\omega(\vb{p})}}
					f(\vb{p})
					\hat{a}^\dagger(\vb{p})
				\right)^\dagger
				\left(
					\int\frac{\dd[3]{q}}{(2\pi)^3\sqrt{2\omega(\vb{q})}}
					g(\vb{q})
					\hat{a}^\dagger(\vb{q})
				\right)
			}{0}
			\\
			&=
			\int\frac{\dd[3]{p}}{(2\pi)^3\sqrt{2\omega(\vb{p})}}
			f(\vb{p})^*
			\int\frac{\dd[3]{q}}{(2\pi)^3\sqrt{2\omega(\vb{q})}}
			g(\vb{q})
			\expval{\hat{a}(\vb{p})\hat{a}^\dagger(\vb{q})}{0}
			\\
			&=
			\int\frac{\dd[3]{p}}{(2\pi)^3\sqrt{2\omega(\vb{p})}}
			f(\vb{p})^*
			\int\frac{\dd[3]{q}}{(2\pi)^3\sqrt{2\omega(\vb{q})}}
			g(\vb{q})
			\comm{\hat{a}(\vb{p})}{\hat{a}^\dagger(\vb{q})}
			\braket{0}
			\\
			&=
			\int\frac{\dd[3]{p}}{(2\pi)^3\sqrt{2\omega(\vb{p})}}
			f(\vb{p})^*
			\int\frac{\dd[3]{q}}{(2\pi)^3\sqrt{2\omega(\vb{q})}}
			g(\vb{q})
			(2\pi)^3\delta^{(3)}(\vb{q}-\vb{p})
			\\
			&=
			\int\frac{\dd[3]{p}}{(2\pi)^32\omega(\vb{p})}
			f(\vb{p})^*
			g(\vb{p})
		\end{split}
	\end{equation*}
	where we used $\hat{a}(\vb{p})\ket{0}=0$ and $\braket{0}=1$ to rewrite the expectation value as commutator.
\end{proof}

\qkgnumbersmearing
\begin{proof}
	Comparing \cref{eq:qkg_number_state_smearing} and \cref{eq:qkg_number_state} leaves us with showing
	\begin{equation*}
		\int\dd[4]{x}
		f(x^\mu)
		\hat\phi^-(x^\mu)
		=
		\int\frac{\dd[3]{p}}{(2\pi)^3\sqrt{2\omega(\vb{p})}}
		f(\vb{p})
		\hat{a}^\dagger(\vb{p})
		.
	\end{equation*}
	Inserting the negative frequency Klein-Gordon operator, \cref{eq:qkg_positive_negative_frequency}, into the left-hand side yields
	\begin{equation*}
		\begin{split}
			\int\dd[4]{x}
			f(x^\mu)
			\hat\phi^-(x^\mu)
			&=
			\int\dd[4]{x}
			f(x^\mu)
			\int\frac{\dd[3]{p}}{(2\pi)^3\sqrt{2\omega(\vb{p})}}
			\eval{
				e^{+ip_\mu x^\mu}
				\hat{a}^\dagger(\vb{p})
			}_{p_0=\omega(\vb{p})}
			\\
			&=
			\int\frac{\dd[3]{p}}{(2\pi)^3\sqrt{2\omega(\vb{p})}}
			\left(
				\int\dd[4]{x}
				f(x^\mu)
				e^{+ip_\mu x^\mu}
			\right)_{p_0=\omega(\vb{p})}
			\hat{a}^\dagger(\vb{p})
			\\
			&=
			\int\frac{\dd[3]{p}}{(2\pi)^3\sqrt{2\omega(\vb{p})}}
			\eval{f(p^\mu)}_{p_0=\omega(\vb{p})}
			\hat{a}^\dagger(\vb{p})
			\\
			&=
			\int\frac{\dd[3]{p}}{(2\pi)^3\sqrt{2\omega(\vb{p})}}
			f\left(\omega(\vb{p}),\vb{p}\right)
			\hat{a}^\dagger(\vb{p})
		\end{split}
	\end{equation*}
	which equals the right-hand side when we take $f(\vb{p})=f\left(\omega(\vb{p}),\vb{p}\right)$.
\end{proof}
\qkgcommsmearedpn
\begin{proof}
	Proof by induction for $n\in\mathbb{N}$.
	\begin{enumerate}
		\item Induction start ($n=1$):
		\begin{equation*}
			\begin{split}
				\comm{\hat\phi^+[f]}{\hat\phi^-[g]}
				&=
				\int\dd[4]{x}f(x^\mu)
				\int\dd[4]{y}g(y^\mu)
				\comm{\hat\phi^+(x^\mu)}{\hat\phi^-(y^\mu)}
				\\
				&=
				\int\dd[4]{x}f(x^\mu)
				\int\dd[4]{y}g(y^\mu)
				D(x^\mu-y^\mu)
				\\
				&=
				\int\dd[4]{x}f(x^\mu)
				\int\dd[4]{y}g(y^\mu)
				\int\frac{\dd[3]{p}}{(2\pi)^32\omega(\vb{p})}
				\eval{e^{-ip_\mu(x^\mu-y^\mu)}}_{p_0=\omega(\vb{p})}
				\\
				&=
				\int\frac{\dd[3]{p}}{(2\pi)^32\omega(\vb{p})}
				\left(
					\int\dd[4]{x}
					f(x^\mu)
					e^{+ip_\mu x^\mu}
				\right)^*_{p_0=\omega(\vb{p})}
				\left(
					\int\dd[4]{y}
					g(y^\mu)
					e^{+ip_\mu y^\mu}
				\right)_{p_0=\omega(\vb{p})}
				\\
				&=
				\int\frac{\dd[3]{p}}{(2\pi)^32\omega(\vb{p})}
				f\left(\omega(\vb{p}),\vb{p}\right)^*
				g\left(\omega(\vb{p}),\vb{p}\right)
				\\
				&=
				\braket{1_f}{1_g}
			\end{split}
		\end{equation*}
		where we used \Cref{thm:qkg_comm_pn}, the four-Fourier transform and the definition of the propagator, alternatively, using the commutation relation of the annihilation and creation operator
		\begin{equation*}
			\begin{split}
				\comm{\hat\phi^+[f]}{\hat\phi^-[g]}
				&=
				\int\frac{\dd[3]{p}}{(2\pi)^3\sqrt{2\omega(\vb{p})}}
				f(\vb{p})^*
				\int\frac{\dd[3]{q}}{(2\pi)^3\sqrt{2\omega(\vb{q})}}
				g(\vb{q})
				\comm{\hat{a}(\vb{p})}{\hat{a}^\dagger(\vb{q})}
				\\
				&=
				\int\frac{\dd[3]{p}}{(2\pi)^3\sqrt{2\omega(\vb{p})}}
				f(\vb{p})^*
				\int\frac{\dd[3]{q}}{(2\pi)^3\sqrt{2\omega(\vb{q})}}
				g(\vb{q})
				(2\pi)^3\delta^{(3)}(\vb{q}-\vb{p})
				\\
				&=
				\int\frac{\dd[3]{p}}{(2\pi)^32\omega(\vb{p})}
				f(\vb{p})^*
				g(\vb{p})
				\\
				&=
				\braket{1_f}{1_g}
			\end{split}
		\end{equation*}
		which also equals $\braket{1_f}{1_g}$.
		\item Induction step ($n\to n+1$):
		\begin{equation*}
			\begin{split}
				\comm{\hat\phi^+[f]}{\hat\phi^-[g]^{n+1}}
				&=
				\comm{\hat\phi^+[f]}{\hat\phi^-[g]\hat\phi^-[g]^n}
				\\
				&=
				\hat\phi^-[g]
				\comm{\hat\phi^+[f]}{\hat\phi^-[g]^n}
				+
				\comm{\hat\phi^+[f]}{\hat\phi^-[g]}
				\hat\phi^-[g]^n
				\\
				&=
				\hat\phi^-[g]
				n\braket{1_f}{1_g}
				\hat\phi^-[g]^{n-1}
				+
				\braket{1_f}{1_g}
				\hat\phi^-[g]^n
				\\
				&=
				(n+1)
				\braket{1_f}{1_g}
				\hat\phi^-[g]^n
				.
			\end{split}
		\end{equation*}
	\end{enumerate}
\end{proof}
\qkgsmearedpos
\begin{proof}
	Inserting the definitions and using \Cref{thm:qkg_comm_smeared_pn}, we find
	\begin{equation*}
		\begin{split}
			\hat\phi^+[g]
			\ket{n_f}
			&=
			\frac{1}{\sqrt{n!}}
			\hat\phi^+[g]
			\hat\phi^-[f]^n
			\ket{0}
			\\
			&=
			\frac{1}{\sqrt{n!}}
			\comm{\hat\phi^+[g]}{\hat\phi^-[f]^n}
			\ket{0}
			\\
			&=
			\frac{1}{\sqrt{n!}}
			n
			\braket{1_g}{1_f}
			\hat\phi^-[f]^{n-1}
			\ket{0}
			\\
			&=
			\sqrt{n}
			\braket{1_g}{1_f}
			\ket{n-1_f}
			.
		\end{split}
	\end{equation*}
\end{proof}
\qkgsmearedpn
\begin{proof}
	The left equation follows from \Cref{thm:qkg_smeared_pos} for $f=g$, the right equation follows from
	\begin{equation*}
		\begin{split}
			\hat\phi^-[f]
			\ket{n_f}
			&=
			\frac{1}{\sqrt{n!}}
			\hat\phi^-[f]
			\hat\phi^-[f]^n
			\ket{0}
			\\
			&=
			\sqrt{n+1}
			\frac{1}{\sqrt{(n+1)!}}
			\hat\phi^-[f]^{n+1}
			\ket{0}
			\\
			&=
			\sqrt{n+1}
			\ket{n+1_f}
			.
		\end{split}
	\end{equation*}
\end{proof}

\qkgnumbereigenstate
\begin{proof}
	From \Cref{thm:qkg_comm_smeared_pn}, the left-hand side can be written
	\begin{equation*}
		\comm{\hat\phi^+[f]}{\hat\phi^-[g]^n}
		=
		\int\frac{\dd[3]{p}}{(2\pi)^3\sqrt{2\omega(\vb{p})}}
		f(\vb{p})^*
		\comm{\hat{a}(\vb{p})}{\hat\phi^-[g]^n}
		,
	\end{equation*}
	and the right-hand side can be written
	\begin{equation*}
		\begin{split}
			n
			\braket{1_f}{1_g}
			\hat\phi^-[g]^{n-1}
			&=
			n
			\int\frac{\dd[3]{p}}{(2\pi)^32\omega(\vb{p})}
			f(\vb{p})^*
			g(\vb{p})
			\hat\phi^-[g]^{n-1}
			\\
			&=
			\int\frac{\dd[3]{p}}{(2\pi)^3\sqrt{2\omega(\vb{p})}}
			f(\vb{p})^*
			n
			\frac{g(\vb{p})}{\sqrt{2\omega(\vb{p})}}
			\hat\phi^-[g]^{n-1}
		\end{split}
		.
	\end{equation*}
	Comparing the former and latter, we conclude
	\begin{equation*}
		\comm{\hat{a}(\vb{p})}{\hat\phi^-[g]^n}		
		=
		n
		\frac{g(\vb{p})}{\sqrt{2\omega(\vb{p})}}
		\hat\phi^-[g]^{n-1}
		,
	\end{equation*}
	and we can easily show the eigenvalue equation
	\begin{equation*}
		\begin{split}
			\hat{N}
			\ket{n_f}
			&=
			\int\frac{\dd[3]{p}}{(2\pi)^3}
			\hat{a}^\dagger(\vb{p})
			\hat{a}(\vb{p})
			\frac{1}{\sqrt{n!}}
			\hat\phi^-[f]^n
			\ket{0}
			\\
			&=
			\frac{1}{\sqrt{n!}}
			\int\frac{\dd[3]{p}}{(2\pi)^3}
			\hat{a}^\dagger(\vb{p})
			\comm{\hat{a}(\vb{p})}{\hat\phi^-[f]^n}
			\ket{0}
			\\
			&=
			\frac{1}{\sqrt{n!}}
			\int\frac{\dd[3]{p}}{(2\pi)^3}
			\hat{a}^\dagger(\vb{p})
			n
			\frac{f(\vb{p})}{\sqrt{2\omega(\vb{p})}}
			\hat\phi^-[f]^{n-1}
			\ket{0}
			\\
			&=
			\sqrt{n}
			\int\frac{\dd[3]{p}}{(2\pi)^3\sqrt{2\omega(\vb{p})}}
				f(\vb{p})
			\hat{a}^\dagger(\vb{p})
			\frac{1}{\sqrt{(n-1)!}}
			\hat\phi^-[f]^{n-1}
			\ket{0}
			\\
			&=
			\sqrt{n}
			\hat\phi^-[f]
			\ket{n-1_f}
			=
			n
			\ket{n_f}
		\end{split}
	\end{equation*}
	where we used \Cref{thm:qkg_smeared_pn} in the last step.
\end{proof}

\qkgnumberinnerproduct
\begin{proof}
	First, we show
	\begin{equation*}
		\braket{n_f}{n_g}
		=
		\left(
			\int\frac{\dd[3]{p}}{(2\pi)^32\omega(\vb{p})}
			f(\vb{p})^*
			g(\vb{p})
		\right)^n
	\end{equation*}
	and then we show $\braket{n_f}{m_g}=0$ iff $n\neq m$.
	\begin{itemize}
		\item Induction start ($n=1$):
		\\
		See \Cref{thm:qkg_number_state_inner_single}.
		\item Induction step ($n\to n+1$):
		\\
		\begin{equation*}
			\begin{split}
				\braket{n+1_f}{n+1_g}
				&=
				\frac{1}{(n+1)!}
				\expval{\hat\phi^+[f]^{n+1}\hat\phi^-[g]^{n+1}}{0}
			\end{split}
		\end{equation*}
	\end{itemize}
\end{proof}

\section{Coherent states}

\qkgdisplacementordered
\begin{proof}
	See Ref.~\cite[p.~48]{Barnett2002}.
\end{proof}
\qkgdisplacementproduct
\begin{proof}
	Inserting the definitions of the normal-ordered displacement operator from \Cref{thm:qkg_displacement_smeared}, we find
	\begin{equation*}
		\begin{split}
			\hat{D}[\alpha]
			\hat{D}[\beta]
			&=
			\left(
				\exp\left\{
					-
					\frac{1}{2}
					\comm{\hat\phi^+[\alpha]}{\hat\phi^-[\alpha]}
				\right\}
				\exp\left\{
					+\hat\phi^-[\alpha]
				\right\}
				\exp\left\{
					-\hat\phi^+[\alpha]
				\right\}
			\right)
			\\
			&\times
			\left(
				\exp\left\{
					-
					\frac{1}{2}
					\comm{\hat\phi^+[\beta]}{\hat\phi^-[\beta]}
				\right\}
				\exp\left\{
					+\hat\phi^-[\beta]
				\right\}
				\exp\left\{
					-\hat\phi^+[\beta]
				\right\}
			\right)
			\\
			&=
			\exp\left\{
				-
				\frac{1}{2}
				\comm{\hat\phi^+[\alpha]}{\hat\phi^-[\alpha]}
				-
				\frac{1}{2}
				\comm{\hat\phi^+[\beta]}{\hat\phi^-[\beta]}
			\right\}
			\\
			&\times
			\exp\left\{
				\hat\phi^-[\alpha]
			\right\}
			\left(
				\exp\left\{
					\hat\phi^+[-\alpha]
				\right\}
				\exp\left\{
					\hat\phi^-[\beta]
				\right\}
			\right)
			\exp\left\{
				\hat\phi^+[-\beta]
			\right\}
			.
		\end{split}
	\end{equation*}
	Because the commutator of the Klein-Gordon field operators are complex numbers, we can use the reduced Baker-Campbell-Hausdorff formula\footnote{The reduced BCH formula requires $\comm{\hat{X}}{\comm{\hat{X}}{\hat{Y}}}=0$.}
	\begin{equation*}
		\exp\hat{X}
		\exp\hat{Y}
		=
		\exp\left\{
			\hat{X}
			+
			\hat{Y}
			+
			\frac{1}{2}
			\comm{\hat{X}}{\hat{Y}}
		\right\}
	\end{equation*}
	where $\hat{X}=\hat\phi^+[-\alpha]$ and $\hat{Y}=\hat\phi^-[\beta]$.
	Using the Baker-Campbell-Hausdorff formula twice, we find
	\begin{equation*}
		\begin{split}
			\exp\hat{X}
			\exp\hat{Y}
			&=
			\exp\left\{
				\hat{X}
				+
				\hat{Y}
				+
				\frac{1}{2}
				\comm{\hat{X}}{\hat{Y}}
			\right\}
			\\
			&=
			\exp\left\{
				\hat{Y}
				+
				\hat{X}
			\right\}
			\exp\left\{
				\frac{1}{2}
				\comm{\hat{X}}{\hat{Y}}
			\right\}
			\\
			&=
			\exp\hat{Y}
			\exp\hat{X}
			\exp\left\{
				-
				\frac{1}{2}
				\comm{\hat{Y}}{\hat{X}}
			\right\}
			\exp\left\{
				+
				\frac{1}{2}
				\comm{\hat{X}}{\hat{Y}}
			\right\}
			\\
			&=
			\exp\hat{Y}
			\exp\hat{X}
			\exp\left\{
				\comm{\hat{X}}{\hat{Y}}
			\right\}
		\end{split}
	\end{equation*}
	which we insert back into the first equation
	\begin{equation*}
		\begin{split}
			\hat{D}[\alpha]
			\hat{D}[\beta]
			&=
			\exp\left\{
				-
				\frac{1}{2}
				\comm{\hat\phi^+[\alpha]}{\hat\phi^-[\alpha]}
				-
				\frac{1}{2}
				\comm{\hat\phi^+[\beta]}{\hat\phi^-[\beta]}
			\right\}
			\exp\left\{
				\hat\phi^-[\alpha]
			\right\}
			\\
			&\times
			\left(
				\exp\left\{
					\hat\phi^-[\beta]
				\right\}
				\exp\left\{
					\hat\phi^+[-\alpha]
				\right\}
				\exp\left\{
					\comm{\hat\phi^+[-\alpha]}{\hat\phi^-[\beta]}
				\right\}
			\right)
			\exp\left\{
				\hat\phi^+[-\beta]
			\right\}
			\\
			&=
			\exp\left\{
				-
				\frac{1}{2}
				\comm{\hat\phi^+[\alpha]}{\hat\phi^-[\alpha]}
				-
				\frac{1}{2}
				\comm{\hat\phi^+[\beta]}{\hat\phi^-[\beta]}
				+
				\comm{\hat\phi^+[-\alpha]}{\hat\phi^-[\beta]}
			\right\}
			\\
			&\times
			\exp\left\{
				+
				\hat\phi^-[\alpha+\beta]
			\right\}
			\exp\left\{
				-
				\hat\phi^+[\alpha+\beta]
			\right\}
			.
		\end{split}
	\end{equation*}
	To identify the normal-ordered displacement operator again, we are left to summarize the commutators
	\begin{equation*}
		\begin{split}
			&\
			-
			\frac{1}{2}
			\comm{\hat\phi^+[\alpha]}{\hat\phi^-[\alpha]}
			-
			\frac{1}{2}
			\comm{\hat\phi^+[\beta]}{\hat\phi^-[\beta]}
			+
			\comm{\hat\phi^+[-\alpha]}{\hat\phi^-[\beta]}
			\\
			=&\
			-
			\frac{1}{2}
			\comm{\hat\phi^+[\alpha]}{\hat\phi^-[\alpha]}
			-
			\frac{1}{2}
			\comm{\hat\phi^+[\beta]}{\hat\phi^-[\beta]}
			-
			\comm{\hat\phi^+[\alpha]}{\hat\phi^-[\beta]}
			\\
			=&\
			-
			\frac{1}{2}
			\comm{\hat\phi^+[\alpha]}{\hat\phi^-[\alpha]}
			-
			\frac{1}{2}
			\comm{\hat\phi^+[\alpha]}{\hat\phi^-[\beta]}
			-
			\frac{1}{2}
			\comm{\hat\phi^+[\beta]}{\hat\phi^-[\beta]}
			-
			\frac{1}{2}
			\comm{\hat\phi^+[\alpha]}{\hat\phi^-[\beta]}
			\\
			=&\
			-
			\frac{1}{2}
			\comm{\hat\phi^+[\alpha]}{\hat\phi^-[\alpha+\beta]}
			-
			\frac{1}{2}
			\comm{\hat\phi^+[\beta]}{\hat\phi^-[\beta]}
			-
			\frac{1}{2}
			\comm{\hat\phi^+[\alpha]}{\hat\phi^-[\beta]}
			\\
			=&\
			-
			\frac{1}{2}
			\comm{\hat\phi^+[\alpha]}{\hat\phi^-[\alpha+\beta]}
			-
			\frac{1}{2}
			\comm{\hat\phi^+[\beta]}{\hat\phi^-[\alpha+\beta]}
			\\
			&\
			+
			\frac{1}{2}
			\comm{\hat\phi^+[\beta]}{\hat\phi^-[\alpha+\beta]}
			-
			\frac{1}{2}
			\comm{\hat\phi^+[\beta]}{\hat\phi^-[\beta]}
			-
			\frac{1}{2}
			\comm{\hat\phi^+[\alpha]}{\hat\phi^-[\beta]}
			\\
			=&\
			-
			\frac{1}{2}
			\comm{\hat\phi^+[\alpha+\beta]}{\hat\phi^-[\alpha+\beta]}
			+
			\frac{1}{2}
			\comm{\hat\phi^+[\beta]}{\hat\phi^-[\alpha]}
			\\
			&\
			+
			\frac{1}{2}
			\comm{\hat\phi^+[\beta]}{\hat\phi^-[\beta]}
			-
			\frac{1}{2}
			\comm{\hat\phi^+[\beta]}{\hat\phi^-[\beta]}
			-
			\frac{1}{2}
			\comm{\hat\phi^+[\alpha]}{\hat\phi^-[\beta]}
			\\
			=&\
			-
			\frac{1}{2}
			\comm{\hat\phi^+[\alpha+\beta]}{\hat\phi^-[\alpha+\beta]}
			+
			\frac{1}{2}
			\comm{\hat\phi^+[\beta]}{\hat\phi^-[\alpha]}
			-
			\frac{1}{2}
			\comm{\hat\phi^+[\alpha]}{\hat\phi^-[\beta]}
		\end{split}
	\end{equation*}
	where we extensively made use of the (multi-)linearity.
	\begin{equation*}
		\begin{split}
			\hat{D}[\alpha]
			\hat{D}[\beta]
			&=
			\exp\left\{
				-
				\frac{1}{2}
				\comm{\hat\phi^+[\alpha+\beta]}{\hat\phi^-[\alpha+\beta]}
			\right\}
			\exp\left\{
				+
				\hat\phi^-[\alpha+\beta]
			\right\}
			\exp\left\{
				-
				\hat\phi^+[\alpha+\beta]
			\right\}
			\\
			&\times
			\exp\left\{
				-
				\frac{1}{2}
				\comm{\hat\phi^+[\alpha]}{\hat\phi^-[\beta]}
				+
				\frac{1}{2}
				\comm{\hat\phi^+[\beta]}{\hat\phi^-[\alpha]}
			\right\}
			\\
			&=
			\hat{D}[\alpha+\beta]
			\exp\left\{
				-
				\frac{1}{2}
				\comm{\hat\phi^+[\alpha]}{\hat\phi^-[\beta]}
				+
				\frac{1}{2}
				\comm{\hat\phi^+[\beta]}{\hat\phi^-[\alpha]}
			\right\}
		\end{split}
	\end{equation*}
\end{proof}
\qkgdisplacementunitary
\begin{proof}
	Using the normal-ordered displacement operator, we show $\hat{D}[\alpha]^\dagger=\hat{D}[-\alpha]$
	\begin{equation*}
		\begin{split}
			\hat{D}[\alpha]^\dagger
			&=
			\left(
				\exp\left\{
					-
					\frac{1}{2}
					\comm{\hat\phi^+[\alpha]}{\hat\phi^-[\alpha]}
				\right\}
				\exp\left\{
					+
					\hat\phi^-[\alpha]
				\right\}
				\exp\left\{
					-
					\hat\phi^+[\alpha]
				\right\}
			\right)^\dagger
			\\
			&=
			\exp\left\{
				-
				\frac{1}{2}
				\comm{\hat\phi^+[\alpha]}{\hat\phi^-[\alpha]}
			\right\}
			\exp\left\{
				+
				\hat\phi^-[\alpha]
			\right\}^\dagger
			\exp\left\{
				-
				\hat\phi^+[\alpha]
			\right\}^\dagger
			\\
			&=
			\exp\left\{
				-
				\frac{1}{2}
				\comm{\hat\phi^+[\alpha]}{\hat\phi^-[\alpha]}
			\right\}
			\exp\left\{
				+
				\hat\phi^+[\alpha]
			\right\}
			\exp\left\{
				-
				\hat\phi^-[\alpha]
			\right\}
			\\
			&=
			\exp\left\{
				-
				\frac{1}{2}
				\comm{\hat\phi^+[-\alpha]}{\hat\phi^-[-\alpha]}
			\right\}
			\exp\left\{
				-
				\hat\phi^+[-\alpha]
			\right\}
			\exp\left\{
				+
				\hat\phi^-[-\alpha]
			\right\}
			\\
			&=
			\hat{D}[-\alpha]
			.
		\end{split}
	\end{equation*}
	\Cref{thm:qkg_displacement_product} lets us evaluate the product of two displacement operators
	\begin{equation*}
		\hat{D}[\alpha]^\dagger
		\hat{D}[\alpha]
		=
		\hat{D}[-\alpha]
		\hat{D}[\alpha]
		=
		\hat{D}[-\alpha+\alpha]
		=
		\mathbb{I}
		.
	\end{equation*}
	We conclude that $\hat{D}[\alpha]^\dagger$ is the inverse of the displacement operator and therefore the displacement operator is unitary.
\end{proof}

\qkgcoherenteigenstate
\begin{proof}
	We use the series representation of the operator exponential and \cref{th:normal_ordered_a1_cn} to move the annihilation operator to the right
	\begin{equation*}
		\begin{split}
			&
			\hat{a}(\vb{p})
			\exp\left\{
				-i
				\int\frac{\dd[3]{p}\alpha(\vb{p})}{(2\pi)^3\sqrt{2\omega(\vb{p})}}
				\hat{a}^\dagger(\vb{p})
			\right\}
			\ket{0}
			\\
			=&\
			\hat{a}(\vb{p})
			\sum_{n=0}^\infty
			\frac{1}{n!}
			\left(
				-i
				\int\frac{\dd[3]{p}\alpha(\vb{p})}{(2\pi)^3\sqrt{2\omega(\vb{p})}}
				\hat{a}^\dagger(\vb{p})
			\right)^n
			\ket{0}
			\\
			=&\
			\sum_{n=0}^\infty
			\frac{1}{n!}
			(-i)^n
			\int\frac{\dd[3]{p_1}\alpha(\vb{p}_1)}{(2\pi)^3\sqrt{2\omega(\vb{p}_1)}}
			\dots
			\int\frac{\dd[3]{p_n}\alpha(\vb{p}_n)}{(2\pi)^3\sqrt{2\omega(\vb{p}_n)}}
			\hat{a}(\vb{p})
			\prod_{j=1}^n
			\hat{a}^\dagger(\vb{p}_j)
			\ket{0}
			\\
			=&\
			\sum_{n=1}^\infty
			\frac{1}{n!}
			(-i)^n
			\int\frac{\dd[3]{p_1}\alpha(\vb{p}_1)}{(2\pi)^3\sqrt{2\omega(\vb{p}_1)}}
			\dots
			\int\frac{\dd[3]{p_n}\alpha(\vb{p}_n)}{(2\pi)^3\sqrt{2\omega(\vb{p}_n)}}
			\sum_{i=1}^n
			(2\pi)^3
			\delta^{(3)}(\vb{p}_i-\vb{p})
			\prod_{\substack{j=1\\j\neq i}}^n
			\hat{a}^\dagger(\vb{p}_j)
			\ket{0}
			\\
			=&\
			\sum_{n=1}^\infty
			\frac{1}{n!}
			(-i)^n
			\sum_{i=1}^n
			\frac{\alpha(\vb{p})}{\sqrt{2\omega(\vb{p})}}
			\prod_{\substack{j=1\\j\neq i}}^n
			\int\frac{\dd[3]{p_j}\alpha(\vb{p}_j)}{(2\pi)^3\sqrt{2\omega(\vb{p}_j)}}
			\hat{a}^\dagger(\vb{p}_j)
			\ket{0}
			\\
			=&\
			\sum_{n=1}^\infty
			\frac{1}{(n-1)!}
			(-i)^n
			\frac{\alpha(\vb{p})}{\sqrt{2\omega(\vb{p})}}
			\left(
				\int\frac{\dd[3]{p}\alpha(\vb{p})}{(2\pi)^3\sqrt{2\omega(\vb{p})}}
				\hat{a}^\dagger(\vb{p})
			\right)^n
			\ket{0}
			\\
			=&\
			\frac{\alpha(\vb{p})}{\sqrt{2\omega(\vb{p})}}
			\sum_{n=0}^\infty
			\frac{1}{n!}
			\left(
				-i
				\int\frac{\dd[3]{p}\alpha(\vb{p})}{(2\pi)^3\sqrt{2\omega(\vb{p})}}
				\hat{a}^\dagger(\vb{p})
			\right)^n
			\ket{0}
			\\
			=&\
			\frac{\alpha(\vb{p})}{\sqrt{2\omega(\vb{p})}}
			\exp\left\{
				-i
				\int\frac{\dd[3]{p}\alpha(\vb{p})}{(2\pi)^3\sqrt{2\omega(\vb{p})}}
				\hat{a}^\dagger(\vb{p})
			\right\}
			\ket{0}
		\end{split}
	\end{equation*}
	and to obtain the eigenvalue equation we are left to multiply both sides with
	\begin{equation*}
		\exp\left\{
			-
			\frac{1}{2}
			\int\frac{\dd[3]{p}}{(2\pi)^3\sqrt{2\omega(\vb{p})}}
			\abs{\alpha(\vb{p})}^2
		\right\}
		.
	\end{equation*}
\end{proof}

\qkgcoherentenergy
\begin{proof}
	For the first moment of the energy observable, we insert the definitions
	\begin{equation*}
		\begin{split}
			\expval{\hat{H}}{\alpha}
			&=
			\int\frac{\dd[3]{p}}{(2\pi)^3}
			\omega(\vb{p})
			\expval{\hat{a}^\dagger(\vb{p})\hat{a}(\vb{p})}{\alpha}
			\\
			&=
			\int\frac{\dd[3]{p}}{(2\pi)^3}
			\omega(\vb{p})
			\expval{\frac{\alpha(\vb{p})^*}{2\omega(\vb{p})}\frac{\alpha(\vb{p})}{2\omega(\vb{p})}}{\alpha}
			\\
			&=
			\int\frac{\dd[3]{p}}{(2\pi)^32\omega(\vb{p})}
			\omega(\vb{p})
			\abs{\alpha(\vb{p})}^2
		\end{split}
	\end{equation*}
	and use the eigenvalue equation. For the second moment, we again use the definitions and the eigenvalue equation
	\begin{equation*}
		\begin{split}
			\expval{\hat{H}^2}{\alpha}
			&=
			\int\frac{\dd[3]{p}_1}{(2\pi)^3}
			\int\frac{\dd[3]{p}_2}{(2\pi)^3}
			\omega(\vb{p}_1)
			\omega(\vb{p}_2)
			\expval{
				\hat{a}^\dagger(\vb{p}_1)
				\hat{a}(\vb{p}_1)
				\hat{a}^\dagger(\vb{p}_2)
				\hat{a}(\vb{p}_2)
			}{\alpha}
			\\
			&=
			\int\frac{\dd[3]{p}_1}{(2\pi)^3\sqrt{2\omega(\vb{p}_1)}}
			\int\frac{\dd[3]{p}_2}{(2\pi)^3\sqrt{2\omega(\vb{p}_2)}}
			\omega(\vb{p}_1)
			\omega(\vb{p}_2)
			\alpha(\vb{p}_1)^*
			\alpha(\vb{p}_2)
			\expval{
				\hat{a}(\vb{p}_1)
				\hat{a}^\dagger(\vb{p}_2)
			}{\alpha}
			\\
			&=
			\int\frac{\dd[3]{p}_1}{(2\pi)^3\sqrt{2\omega(\vb{p}_1)}}
			\int\frac{\dd[3]{p}_2}{(2\pi)^3\sqrt{2\omega(\vb{p}_2)}}
			\omega(\vb{p}_1)
			\omega(\vb{p}_2)			
			\alpha(\vb{p}_1)^*
			\alpha(\vb{p}_2)
			\\
			&\times
			\expval{
				(2\pi)^3
				\delta^{(3)}(\vb{p}_2-\vb{p}_1)
				+
				\hat{a}^\dagger(\vb{p}_2)
				\hat{a}(\vb{p}_1)
			}{\alpha}
			\\
			&=
			\int\frac{\dd[3]{p}}{(2\pi)^32\omega(\vb{p}_1)}
			\omega(\vb{p})^2
			\abs{\alpha(\vb{p})}^2
			+
			\left(
				\int\frac{\dd[3]{p}_1}{(2\pi)^32\omega(\vb{p})}
				\omega(\vb{p})
				\abs{\alpha(\vb{p})}^2
			\right)^2
		\end{split}
	\end{equation*}
\end{proof}

\qkgcoherentnumber
\begin{proof}
	The number observable is a special case of \Cref{thm:qkg_coherent_state_energy} for $\omega(\vb{p})=1$.
\end{proof}

\qkgcoherentnumberinnerproduct
\begin{proof}
	Expanding the exponential series and noting that the coefficients have the same algebraic form as a $m$-particle state with spectrum $\alpha(\vb{p})$\footnote{Except for the coherent spectrum $\alpha(\vb{p})$ being unbound.}, we find
	\begin{equation*}
		\braket{n_f}{\alpha}
		=
		\sum_{m=0}^\infty
		\frac{1}{\sqrt{m!}}
		\braket{n_f}{m_\alpha}
		e^{-\overline{n}/2}
		=
		\frac{1}{\sqrt{m!}}
		\left(
			\int\frac{\dd[3]{p}}{(2\pi)^32\omega(\vb{p})}
			f(\vb{p})^*
			\alpha(\vb{p})
		\right)^n
		e^{-\overline{n}/2}
	\end{equation*}
\end{proof}
\qkgcoherentinnerproduct
\begin{proof}
	\begin{equation*}
		\begin{split}
			\braket{\alpha}{\beta}
			&=
			\expval{
				\hat{D}[\alpha]^\dagger
				\hat{D}[\beta]
			}{0}
			\\
			&=
			\expval{
				\hat{D}[-\alpha]
				\hat{D}[\beta]
			}{0}
			\\
			&=
			\expval{
				\hat{D}[\beta-\alpha]
			}{0}
			\exp\left\{
				-
				\frac{1}{2}
				\comm{\hat\phi^+[-\alpha]}{\hat\phi^-[\beta]}
				+
				\frac{1}{2}
				\comm{\hat\phi^+[\beta]}{\hat\phi^-[-\alpha]}
			\right\}
		\end{split}
	\end{equation*}
	where we used \Cref{thm:qkg_displacement_product}.
	Using \Cref{thm:qkg_coherent_state_number_state_inner_product} with $n=0$ reveals
	\begin{equation*}
		\expval{\hat{D}[\beta-\alpha]}{0}
		=
		\braket{0}{-\alpha+\beta}
		=
		\exp\left\{
			-
			\frac{1}{2}
			\comm{\hat\phi^+[-\alpha+\beta]}{\hat\phi^-[-\alpha+\beta]}
		\right\}
	\end{equation*}
	and threfore
	\begin{equation*}
		\begin{split}
			\braket{\alpha}{\beta}
			&=
			\exp\left\{
				-
				\frac{1}{2}
				\comm{\hat\phi^+[-\alpha+\beta]}{\hat\phi^-[-\alpha+\beta]}
				-
				\frac{1}{2}
				\comm{\hat\phi^+[-\alpha]}{\hat\phi^-[\beta]}
				+
				\frac{1}{2}
				\comm{\hat\phi^+[\beta]}{\hat\phi^-[-\alpha]}
			\right\}
			\\
			&=
			\exp\biggl\{
				-
				\frac{1}{2}
				\comm{\hat\phi^+[\alpha]}{\hat\phi^-[\alpha]}
				+
				\frac{1}{2}
				\comm{\hat\phi^+[\alpha]}{\hat\phi^-[\beta]}
				+
				\frac{1}{2}
				\comm{\hat\phi^+[\beta]}{\hat\phi^-[\alpha]}
				\\
				&\ \ \ \ \ \ \ \ \ \
				-
				\frac{1}{2}
				\comm{\hat\phi^+[\beta]}{\hat\phi^-[\beta]}
				+
				\frac{1}{2}
				\comm{\hat\phi^+[\alpha]}{\hat\phi^-[\beta]}
				-
				\frac{1}{2}
				\comm{\hat\phi^+[\beta]}{\hat\phi^-[\alpha]}
			\biggr\}
			\\
			&=
			\exp\left\{
				-
				\frac{1}{2}
				\comm{\hat\phi^+[\alpha]}{\hat\phi^-[\alpha]}
				+
				\comm{\hat\phi^+[\alpha]}{\hat\phi^-[\beta]}
				-
				\frac{1}{2}
				\comm{\hat\phi^+[\beta]}{\hat\phi^-[\beta]}
			\right\}
			\\
			&=
			\exp\left\{
				-
				\frac{1}{2}
				\int\frac{\dd[3]{p}}{(2\pi)^32\omega(\vb{p})}
				\left\{
					\abs{\alpha(\vb{p})}^2
					+
					\abs{\beta(\vb{p})}^2
					-
					2\alpha(\vb{p})\beta(\vb{p})^*
				\right\}
			\right\}
			.
		\end{split}
	\end{equation*}
\end{proof}

\qkgsinglewavefunction
\begin{proof}
	of the coordinate wave function
	\begin{equation*}
		\psi(t,\vb{x})
		=
		\bra{0}
		\hat\phi(t,\vb{x})
		\ket{1_f}
		=
		\bra{0}
		\hat\phi^+(t,\vb{x})
		\ket{1_f}
		.
	\end{equation*}
	Inserting the definition of the single-particle number state yields
	\begin{equation*}
		\begin{split}
			\psi(t,\vb{x})
			&=
			\int\frac{\dd[3]{p}}{(2\pi)^3\sqrt{2\omega(\vb{p})}}
			e^{-ip_\mu x^\mu}
			\expval{
				\hat{a}(\vb{p})
				\int\frac{\dd[3]{q}}{(2\pi)^3\sqrt{2\omega(\vb{q})}}
				f(\vb{q})
				\hat{a}^\dagger(\vb{p})
			}{0}
			\\
			&=
			\int\frac{\dd[3]{p}}{(2\pi)^3\sqrt{2\omega(\vb{p})}}
			\int\frac{\dd[3]{q}}{(2\pi)^3\sqrt{2\omega(\vb{q})}}
			e^{-ip_\mu x^\mu}
			f(\vb{q})
			\expval{
				\hat{a}(\vb{p})
				\hat{a}^\dagger(\vb{p})
			}{0}
			\\
			&=
			\int\frac{\dd[3]{p}}{(2\pi)^3\sqrt{2\omega(\vb{p})}}
			\int\frac{\dd[3]{q}}{(2\pi)^3\sqrt{2\omega(\vb{q})}}
			e^{-ip_\mu x^\mu}
			f(\vb{q})
			(2\pi)^3
			\delta^{(3)}(\vb{q}-\vb{p})
			\\
			&=
			\int\frac{\dd[3]{p}}{(2\pi)^32\omega(\vb{p})}
			e^{-ip_\mu x^\mu}
			f(\vb{p})
		\end{split}
	\end{equation*}
	in agreement with Ref.~\cite[eq.~4]{Naumov2013}.
\end{proof}
\begin{lemma}
	The probability current of a single-particle number state is
	\begin{equation}
		j_\mu(t,\vb{x})
		=
		\int\frac{\dd[3]{p}}{(2\pi)^32\omega(\vb{p})}
		\int\frac{\dd[3]{q}}{(2\pi)^32\omega(\vb{q})}
		q_\mu
		2\Re\left\{
			f(\vb{p})^*
			f(\vb{q})
			e^{-i(q_\mu-p_\mu)x^\mu}
		\right\}
	\end{equation}
\end{lemma}
\qkgsinglegroupvelocity
\begin{proof}
	The probability current is\footnote{See Ref.~\cite[p.~18]{Peskin1995} for a derivation from Noether's theorem}
	\begin{equation*}
		j^\mu(t,\vb{x})
		=
		i
		\bigl\{
			\psi(t,\vb{x})
			\partial^\mu
			\psi(t,\vb{x})^*
			-
			\psi(t,\vb{x})^*
			\partial^\mu
			\psi(t,\vb{x})
		\bigr\}
	\end{equation*}
	and can be written
	\begin{equation*}
		j^\mu(t,\vb{x})
		=
		2\frac{\psi(t,\vb{x})\partial^\mu\psi(t,\vb{x})^*-\text{c.c.}}{2i}
		=
		2\Im\left\{
			\psi(t,\vb{x})
			\partial^\mu
			\psi(t,\vb{x})^*
		\right\}
		.
	\end{equation*}
	We proceed with the argument of the imaginary part
	\begin{equation*}
		\begin{split}
			\psi(t,\vb{x})^*
			\partial^\mu
			\psi(t,\vb{x})
			&=
			\int\frac{\dd[3]{p}}{(2\pi)^32\omega(\vb{p})}
			e^{-ip_\mu x^\mu}
			f(\vb{p})
			\partial^\mu
			\int\frac{\dd[3]{q}}{(2\pi)^32\omega(\vb{q})}
			e^{+iq_\mu x^\mu}
			f(\vb{q})^*
			\\
			&=
			\int\frac{\dd[3]{p}}{(2\pi)^32\omega(\vb{p})}
			\int\frac{\dd[3]{q}}{(2\pi)^32\omega(\vb{q})}
			iq^\mu
			f(\vb{p})
			f(\vb{q})^*
			e^{-i(p_\mu-q_\mu)x^\mu}
		\end{split}
		.
	\end{equation*}
	Inserting the argument back into the probability current and using $\Im{iz}=\Re{z}$, we find
	\begin{equation*}
		j^\mu(t,\vb{x})
		=
		\int\frac{\dd[3]{p}}{(2\pi)^32\omega(\vb{p})}
		\int\frac{\dd[3]{q}}{(2\pi)^32\omega(\vb{q})}
		2\Re\left\{
			q^\mu
			f(\vb{p})
			f(\vb{q})^*
			e^{-i(p_\mu-q_\mu)x^\mu}
		\right\}
		.
	\end{equation*}
	Writing out the real part and relabeling the integration variables in the second term yields
	\begin{equation*}
		j^\mu(t,\vb{x})
		=
		\int\frac{\dd[3]{p}}{(2\pi)^32\omega(\vb{p})}
		\int\frac{\dd[3]{q}}{(2\pi)^32\omega(\vb{q})}
		\left\{
			q^\mu
			+
			p^\mu
		\right\}
		f(\vb{p})
		f(\vb{q})^*
		e^{-i(p_\mu-q_\mu)x^\mu}
		.
	\end{equation*}	
	The last result is in agreement with Ref.~\cite[eqs.~36,37]{Naumov2013} if one further assumes $f(\vb{p})$ to be real.
\end{proof}
\qkgsinglelocalization
\begin{proof}
	No explicit proof, result claimed in Ref.~\cite[eq.~38]{Naumov2013}.
\end{proof}
		\section{Maxwell field}

\subsection{Relativistic field theory}

The Maxwell Lagrangian with an external source $J^\mu(t,\vb{x})$ is
\begin{equation}
	\begin{split}
		\mathcal{L}
		&=
		-
		\frac{1}{4}
		F_{\mu\nu}
		F^{\mu\nu}
		+
		A_\mu J^\mu
		\\
		&=
		\frac{1}{2}
		\left(\partial_\mu A_\nu\right)
		\left(\partial^\mu A^\nu-\partial^\nu A^\mu\right)
		+
		A_\mu J^\mu
	\end{split}
	\label{eq:mw_lagrangian}
\end{equation}
with the relativistic Euler-Lagrange equation yielding
\begin{equation}
	0
	=
	\partial_\mu\pdv{\mathcal{L}}{(\partial_\mu A_\nu)}
	-
	\pdv{\mathcal{L}}{A_\nu}
	=
	\partial_\mu\partial^\mu A^\nu
	-
	\partial^\nu\partial_\mu A^\mu
	-
	J^\nu	
	\label{eq:mw_eom}	
\end{equation}

\subsection{Coulomb gauge}

The Maxwell Lagrangian is invariant under local gauge transformations
\begin{equation}
	A_\mu(t,\vb{x})
	\to
	A_\mu^\prime(t,\vb{x})
	=
	A_\mu(t,\vb{x})
	+
	\partial_\mu\Lambda(t,\vb{x})
	\label{eq:mw_local_gauge_transform}
\end{equation}
where $\Lambda(t,\vb{x})$ is a local gauge field.
For instance, the physical field-strength tensor transforms under \cref{eq:mw_local_gauge_transform} as
\begin{equation}
	\begin{split}
		F_{\mu\nu}
		\to
		F_{\mu\nu}^\prime
		&=
		\partial_\mu\left(A_\nu+\partial_\nu\Lambda\right)
		-
		\partial_\nu\left(A_\mu+\partial_\mu\Lambda\right)
		\\
		&=
		F_{\mu\nu}
		+
		\partial_\mu\partial_\nu\Lambda
		-
		\partial_\nu\partial_\mu\Lambda
		=
		F_{\mu\nu}
	\end{split}
	\label{eq:mw_field_strength_gauge_transform}.
\end{equation}
The invariance of the Maxwell field under local gauge transformation is used to remove a degree of freedom from the field using a gauge condition.
The most popular gauge conditions are the Lorentz gauge $\partial_\mu A^\mu=0$ and the Coulomb gauge $\div\vb{A}=0$.
While the Lorentz gauge is manifest Lorentz invariant it suffers from unphysical scalar and longitudinal polarization states destroying unitarity.
The Coulomb gauge is manifest unitarity but has to be imposed in every reference frame.
As Lorentz boosts are not of interest for us, we will adapt the Coulomb gauge condition.
The Coulomb gauge leaves a residual gauge freedom which allows us to choose $A_0=0$.\footnote{If external static sources are present, we need to be more careful with the residual gauge fixing.}

Applying the Coulomb gauge
\begin{align}
	\div\vb{A}(t,\vb{x})
	=
	0
	&&
	A_0
	=
	0
	\label{eq:mw_coulomb_gauge}
\end{align}
to the free equation of motion \cref{eq:mw_eom} yields the relativistic wave equation
\begin{equation}
	0
	=
	\partial_\mu\partial^\mu
	\vb{A}(t,\vb{x})
	=
	\partial_t^2
	\vb{A}(t,\vb{x})
	-
	\grad^2
	\vb{A}(t,\vb{x})
	\label{eq:mw_relatistic_wave}
\end{equation}
which is solved by plane-waves satisfying the massless dispersion relation
\begin{equation}
	\omega(\vb{p})
	=
	\norm{\vb{p}}
\end{equation}

\subsection{Mode decomposition}

As with the Klein-Gordon field, we start with the four-dimensional Fourier transform of $A^\mu(t,\vb{x})$, insert it into the free equations of motion ($J^\mu=0$), and perform the mode decomposition
\begin{equation}
	\vb{A}(t,\vb{x})
	=
	\sum_{\lambda=1,2}
	\int_{\mathbb{R}^3}\frac{}{(2\pi)^3\sqrt{2\omega(\vb{p})}}
	\left\{
		a_\lambda(\vb{p})
		\vb{\epsilon}_\lambda(\vb{p})
		e^{-ip_\mu x^\mu}
		+
		\text{c.c.}
	\right\}
	\label{eq:mw_ft}
\end{equation}
where we defined $a_\lambda(\vb{p})\vb{\epsilon}_\lambda(\vb{p})=\vb{A}(\omega(\vb{p}),\vb{p})$ and $\vb{\epsilon}_\lambda(\vb{p})$ denotes the polarization vectors.
For the mode decomposition to satisfy the Coulomb gauge, the polarization vectors $\vb{\epsilon}_\lambda(\vb{p})$ need to be orthogonal to the wave vector $\vb{p}$, i.e.,
\begin{equation}
	\vb{p}\vdot\vb{\epsilon}_\lambda(\vb{p})
	=
	0	
\end{equation}
Furthermore, we require the $\vb{p}/\norm{\vb{p}},\vb{\epsilon}_1(\vb{p}),\vb{\epsilon}_2(\vb{p})$ to form a orthonormal basis
\begin{equation}
	\vb{\epsilon}_i(\vb{p})
	\vdot
	\vb{\epsilon}_j(\vb{p})
	=
	\delta_{ij}
\end{equation}

\subsection{Maxwell equations}

The covariant inhomogeneous Maxwell equations are obtained from the equations of motion
\begin{equation}
	J^\nu
	=
	\partial_\mu F^{\mu\nu}
	=
	\partial_\mu\partial^\mu A^\nu
	-
	\partial^\nu\partial_\mu A^\mu
	\label{eq:mw_inhomo},
\end{equation}
and the covariant homogeneous Maxwell equations are a consequence of the Bianchi identity
\begin{equation}
	0
	=
	\partial_\mu\tilde{F}^{\mu\nu}
	\label{eq:mw_homo}
\end{equation}
where we defined $\tilde{F}^{\mu\nu}=\frac{1}{2}\varepsilon^{\mu\nu\alpha\beta}F_{\alpha\beta}$.

The components of the field-strength tensor $F^{\mu\nu}$ relate to the electromagnetic field components via
\begin{align}
	F^{0i}
	=
	-E^i
	&&
	F^{ij}
	=
	-\varepsilon^{ijk}B_k
	\label{eq:mw_em_components}.
\end{align}
Evaluating the time component of \cref{eq:mw_homo} yields the Gauss' law for magnetism
\begin{equation}
	\begin{split}
		0
		=
		\varepsilon_{0\lambda\mu\nu}\partial^\lambda F^{\mu\nu}
		&=
		\varepsilon_{0ijk}\partial^iF^{jk}
		\\
		&=
		-
		\varepsilon_{ijk}\varepsilon_{ljk}
		\partial^i B_l
		=
		2\partial_iB^i
	\end{split}
	\label{eq:mw_gauss_law_mag}
\end{equation}
and the spatial component yields Ampere's circuit law
\begin{equation}
	\begin{split}
		0
		=
		\varepsilon_{i\lambda\mu\nu}
		\partial^\lambda
		F^{\mu\nu}
		&=
		-
		\varepsilon_{ijk}
		\varepsilon^{ljk}
		\partial_t B_l
		-
		2\varepsilon_{ijk}
		\partial^jE^k
		\\
		&=
		\partial_tB_i
		+
		\varepsilon_{ijk}
		\partial^jE_k
	\end{split}
	\label{eq:mw_ampere_law}.
\end{equation}
The time component of the inhomogeneous covariant Maxwell equation \cref{eq:mw_inhomo} yields Gauss' law
\begin{equation}
	J^0
	=
	\rho
	=
	\partial_\mu F^{\mu\nu}
	=
	\partial_i E^i
	\label{eq:mw_gauss_law},
\end{equation}
and the spatial component yields Faraday's law of induction
\begin{equation}
	J^i
	=
	\partial_\mu F^{\mu i}
	=
	-\partial_t E^i
	+\varepsilon^{ijk}\partial_j B_k
	\label{eq:mw_faraday_law}.
\end{equation}
and we derived the vector Maxwell equations from first principles
\begin{align}
	\div\vb{E}
	=
	\rho
	&&
	\div\vb{B}
	=
	0
	\label{eq:mw_homo_vec}
	\\
	\curl\vb{E}
	=
	-
	\partial_t\vb{B}
	&&
	\curl\vb{B}
	=
	\vb{J}
	+
	\partial_t\vb{E}
	\label{eq:mw_inhomo_vec}
\end{align}

\subsection{Canonical quantization}

\subsection{Single-particle and coherent states}

\subsection{Electromagnetic field operators}

		
		\addcontentsline{toc}{section}{References}
		\printbibliography[title=References]
	\end{refsection}
	
	\chapter{Quantum optical signal-processing}
	\begin{refsection}
		\section{Coherent source}

In the first chapter, we showed that a classical source $\vb{J}(t,\vb{x})$ generates a coherent state via the $\hat{S}$ operator from the vacuum
\begin{equation}
	\begin{split}
		\ket{\alpha(\vb{p})}
		&=
		\exp\left\{
			-i\int\dd[4]{x}
			\vb{J}(t,\vb{x})
			\vdot
			\vb{A}(t,\vb{x})
		\right\}
		\ket{0}
		\\
		&=
		\exp\left\{
			-\frac{1}{2}
			\sum_{\lambda=1,2}
			\int\dd[3]{p}
			\boldsymbol{\varepsilon}_\lambda(\vb{p})
			\vdot
			\vb{J}(\vb{p})
		\right\}
		\\
		&\times
		\exp\left\{
			\sum_{\lambda=1,2}
			\int\dd[3]{p}
			\boldsymbol{\varepsilon}_\lambda(\vb{p})
			\vdot
			\vb{J}(\vb{p})
			\hat{a}_\lambda^\dagger(\vb{p})
		\right\}
		\ket{0}
	\end{split}
\end{equation}
and we are left to find a mechanism for $\vb{J}(\vb{p})$.

\subsection{Oscillating electron}

\subsection{Hydrogen atom}

\subsection{Harmonic oscillators}
		\section{Waveguides}

So far, we discussed the free-field in which the states disperse quickly as seen in the first chapter.
Therefore, most optical communication is conducted using waveguide.

We can amend the developed formalism by replacing the linear dispersion relation of the Maxwell field by a dispersion relation matching the waveguides transmission properties.

\subsection{Fiber}

Optical fibers are dielectric cylindrical waveguides.

\subsection{Loss}

Loss needs to be modeled by the exchange with an environment
		\section{Mode couplers}

Let $\ket{\boldsymbol{\alpha}_1}$ be a first coherent state $\vb{j}_1$ be a first classical current induced by absorption of the coherent state, i.e.,
\begin{equation*}
	\vb{j}_1(p^\mu)
	=
	i\boldsymbol{\alpha}_1(p^\mu)
	.
\end{equation*}
The first classical current is coupled to a second classical current by a classical transfer function $g(x^\mu)$
\begin{equation*}
	\vb{j}_2(x^\mu)
	=
	\int\dd[4]{y}
	g(x^\mu-y^\mu)
	\vb{j}_1(y^\mu)
\end{equation*}
which emits a second coherent state $\ket{\boldsymbol{\alpha}_2}$ with
\begin{equation*}
	\begin{split}
		\boldsymbol{\alpha}_2(p^\mu)
		=
		-i\vb{j}_2(p^\mu)
		&=
		-i
		\int\dd[4]{x}
		\vb{j}_2(x^\mu)
		e^{ip_\nu x^\nu}
		\\
		&=
		-i
		\int\dd[4]{x}
		g(x^\mu-y^\mu)
		e^{ip_\nu x^\nu}
		\int\dd[4]{y}
		\vb{j}_1(y^\mu)
		\\
		&=
		-i
		\int\dd[4]{x}
		g(x^\mu)
		e^{ip_\nu x^\nu}
		\int\dd[4]{y}
		i\boldsymbol{\alpha}_1(y^\mu)
		e^{ip_\nu y^\nu}
		\\
		&=
		g(p^\mu)
		\boldsymbol{\alpha}_1(p^\mu)
		.
	\end{split}
\end{equation*}
Alternatively, a quantum mode coupler with transfer function $g_{\lambda\lambda^\prime}$ is described by the interaction Lagrangian
\begin{equation}
	L_\text{int}
	=
	\sum_{\lambda,\lambda^\prime=1,2}
	\int\frac{\dd[3]{p}}{(2\pi)^32\omega(\vb{p})}
	\left\{
		g_{\lambda\lambda^\prime}\left(\omega(\vb{p}),\vb{p}\right)
		\hat{a}_{\text{out},\lambda^\prime}^\dagger(\vb{p})
		\hat{a}_{\text{in},\lambda}(\vb{p})
		+
		\text{h.c.}
	\right\}
	.
\end{equation}
In terms of smeared positive and negative frequency Maxwell operators, the interaction Lagrangian takes the form~\cite[p.~130]{Haroche2006}
\begin{equation*}
	\begin{split}
		L_\text{int}
		&=
		\int\dd[4]{x}
		\mathcal{L}_\text{int}
		=
		\hat{\vb{A}}_\text{out}^+[g_\text{out}]
		\hat{\vb{A}}_\text{in}^-[g_\text{in}]
		+
		\text{h.c.}
		\\
		&=
		\int\frac{\dd[3]{p}}{(2\pi)^3\sqrt{2\omega(\vb{p})}}
		\sum_{\lambda=1,2}
		g_{\text{out},\lambda}\left(\omega(\vb{p}),\vb{p}\right)^*
		\hat{a}_{\text{out},\lambda}^\dagger(\vb{p})
		\\
		&\times
		\int\frac{\dd[3]{q}}{(2\pi)^3\sqrt{2\omega(\vb{q})}}
		\sum_{\lambda^\prime=1,2}
		g_{\text{in},\lambda^\prime}\left(\omega(\vb{q}),\vb{q}\right)
		\hat{a}_{\text{in},\lambda^\prime}(\vb{q})
		+
		\text{h.c.}
		\\
		&=
		\int\frac{\dd[3]{p}}{(2\pi)^32\omega(\vb{p})}
		\sum_{\lambda,\lambda^\prime=1,2}
		g_{\lambda,\lambda^\prime}\left(\omega(\vb{p}),\vb{p}\right)
		\hat{a}_{\text{out},\lambda}^\dagger(\vb{p})
		\hat{a}_{\text{in},\lambda^\prime}(\vb{p})
	\end{split}
\end{equation*}
where we used
\begin{equation*}
	g_{\lambda,\lambda^\prime}\left(\omega(\vb{p}),\vb{p}\right)
	=
	(2\pi)^3\delta^{(3)}(\vb{q}-\vb{p})
	g_{\text{out},\lambda}\left(\omega(\vb{p}),\vb{p}\right)^*
	g_{\text{in},\lambda^\prime}\left(\omega(\vb{q}),\vb{q}\right)	
\end{equation*}
which appears somewhat reasonable when we consider first-order momentum transfer.
The scattering operator is
\begin{equation*}
	\begin{split}
		\hat{S}
		=
		T\exp\left\{
			iL_\text{int}
		\right\}
		&=
		T\exp\left\{
			i
			\hat{\vb{A}}_\text{out}^+[g_\text{out}]
			\hat{\vb{A}}_\text{in}^-[g_\text{in}]
			+
			i
			\hat{\vb{A}}_\text{in}^-[g_\text{in}]
			\hat{\vb{A}}_\text{out}^+[g_\text{out}]
		\right\}
		\\
		&=
		\sum_{n=0}^\infty
		\frac{i^n}{n!}
		T\left[
			\hat{\vb{A}}_\text{out}^+[g_\text{out}]
			\hat{\vb{A}}_\text{in}^-[g_\text{in}]
			+
			\hat{\vb{A}}_\text{in}^-[g_\text{in}]
			\hat{\vb{A}}_\text{out}^+[g_\text{out}]
		\right]^n
		\\
		&=
		\sum_{n=0}^\infty
		\frac{i^n}{n!}
		\sum_{m=0}^n
		\binom{n}{m}
		T\left[
			\hat{\vb{A}}_\text{out}^+[g_\text{out}]
			\hat{\vb{A}}_\text{in}^-[g_\text{in}]
		\right]^m
		T\left[
			\hat{\vb{A}}_\text{in}^-[g_\text{in}]
			\hat{\vb{A}}_\text{out}^+[g_\text{out}]
		\right]^{n-m}
	\end{split}
\end{equation*}
The first power term is
\begin{equation*}
	\begin{split}
		&
		T\left[
			\hat{\vb{A}}_\text{out}^+[g_\text{out}]
			\hat{\vb{A}}_\text{in}^-[g_\text{in}]
		\right]^m
		\\
		=&\
		T\left[
			\int\dd[4]{x}
			g_\text{out}(x^\mu)
			\hat{\vb{A}}_\text{out}^+(x^\mu)
			\int\dd[4]{y}
			g_\text{in}(y^\mu)
			\hat{\vb{A}}_\text{in}^-(y^\mu)
		\right]^m
		\\
		=&\
		\int\dd[4]{x_1}
		\int\dd[4]{y_1}
		g_\text{out}(x^\mu_1)
		g_\text{in}(y^\mu_1)
		\dots
		\int\dd[4]{x_m}
		\int\dd[4]{y_m}
		g_\text{out}(x^\mu_m)
		g_\text{in}(y^\mu_n)
		T\left[
			\hat{\vb{A}}_\text{out}^+(x^\mu)^m
			\hat{\vb{A}}_\text{in}^-(y^\mu)^m
		\right]
	\end{split}
\end{equation*}

\subsection{Phase modulator}

The action describing the absorption and emission of a field while adding a possibly time-dependent phase-shift is
\begin{equation*}
	\begin{split}
		S
		&=
		\frac{1}{2}
		\int\dd[4]{x}
		\int\dd[4]{y}
		\hat\phi_\text{out}^+(x^\mu)
		G(x^\mu-y^\mu)
		\hat\phi_\text{in}^-(y^\mu)
		\\
		&=
		\frac{1}{2}
		\int\frac{\dd[3]{p}}{(2\pi)^3}
		G(p^\mu)
		\int\dd[4]{x}
		\hat\phi_\text{out}^+(x^\mu)
		e^{-ip_\nu x^\nu}
		\int\dd[4]{y}
		\hat\phi_\text{in}^-(y^\mu)
		e^{+ip_\nu y^\nu}
		\\
		&=
		\frac{1}{2}
		\int\frac{\dd[3]{p}}{(2\pi)^32\omega(\vb{p})}
		G\left(\omega(\vb{p}),\vb{p}\right)
		\hat{a}_\text{out}^\dagger(\vb{p})
		\hat{a}_\text{in}(\vb{p})
	\end{split}
\end{equation*}
and the corresponding scattering operator becomes
\begin{equation*}
	\hat{S}
	=
	T\left[
		\exp\left\{
			iS
		\right\}
	\right]
	=
	T\left[
		\exp\left\{
			\frac{i}{2}
			\int\frac{\dd[3]{p}}{(2\pi)^32\omega(\vb{p})}
			G\left(\omega(\vb{p}),\vb{p}\right)
			\hat{a}_\text{out}^\dagger(\vb{p})
			\hat{a}_\text{in}(\vb{p})
		\right\}
	\right]
\end{equation*}
		\include{chapters/quantum-optical-signal-processing/coherent-detector}
		
		\addcontentsline{toc}{section}{References}
		\printbibliography[title=References]	
	\end{refsection}

	\chapter{Example: CV-QKD}
	\begin{refsection}
		\section{Idea and operating principle}
		\section{Protocol}
		\subsection{Recociliation}
		\subsection{Classical post-processing}
		\section{Coherent state transmitter}
		\section{Coherent state receiver}
		\section{Squeezed state attack}
		\section{Protocol extension}
		
		\addcontentsline{toc}{section}{References}
		\printbibliography[title=References]
	\end{refsection}

	\chapter{Conclusion and outlook}
	\begin{refsection}	
		\addcontentsline{toc}{section}{References}
		\printbibliography[title=References]
	\end{refsection}

	\appendix

	\chapter{Harmonic oscillator}
	\begin{refsection}
		\section{Harmonic oscillator}

\blockcquote{Sidney Coleman}{The career of a young theoretical physicist consists of treating the harmonic oscillator in ever-increasing levels of abstraction.}

\subsection{Examples and quadratic approximation}

To illustrate Coleman's quote, let us consider two at first glance very different physical systems showing the same dynamics.
The left-hand side of \Cref{fig:ho} depicts a mechanical oscillator comprising a spring and a mass $m$ at different times.
The right-hand side of \Cref{fig:ho} depicts an electrical oscillator comprising an inductor with inductance $L$, a capacitor with capacitance $C$, and a current $i$.
\begin{figure}[htb]
    \centering
    \subfloat[\centering Mechanical oscillator]{\includestandalone[mode=buildnew]{figures/tikz/oscillator-mechanical}}
    \qquad
    \subfloat[\centering Electrical oscillator]{\includestandalone[mode=buildnew]{figures/tikz/oscillator-electrical}}
    \caption{Two embodiments of a oscillator based on classical mechanics (a) and electronic circuit theory (b).}\label{fig:ho}
\end{figure}
If we assume the spring of the mechanical oscillator to exert a force proportional to $q(t)$ on the point mass $m$\footnote{For small deviations from the equilibrium, $\abs{q(t)}\ll1$, experiments justify a linear force for the spring.}, Newtonian mechanics relate the spring's force to the particle's acceleration
\begin{equation}
    m\ddot{q}
    =
    F
    =
    -kq
    \label{eq:ho_newton}
\end{equation}
wherein $k$ is the proportionality factor of the spring's force, known as the spring constant.
From circuit theory, we know that the effect of an inductance $L$ and capacitance $C$ on their respective current and voltage
\begin{align}
    V_L
    =
    L\dv{I_L}{t}
    ,&&
    I_C
    =
    C\dv{V_C}{t}
    .
\end{align}
Kirchhoff's current law tells us that the same current $I=I_L=I_C$ flows through the inductor and the capacitor while Kirchhoff's voltage law that the sum of the voltage drops $V_L+V_C$ is equal to zero, hence
\begin{equation}
    \dv[2]{I}{t}
    =
    \frac{1}{L}\dv{V_C}{t}
    =
    -\frac{1}{L}\dv{V_L}{t}
    =
    -\frac{1}{LC}I
    \label{eq:ho_current}.
\end{equation}
On first sight, \cref{eq:ho_newton} and \cref{eq:ho_current} appear quite different.
If we define $\omega_0=k/m$ for the mechanical and $\omega_0=1/\sqrt{LC}$ for the electrical oscillator, and adjust our notation, \cref{eq:ho_newton} and \cref{eq:ho_current} take the form
\begin{align}
    \ddot{q}
    =
    -\omega_0^2q
    ,&&
    \ddot{I}
    =
    -\omega_0^2I
    \label{eq:ho_me}
\end{align}
which is remarkable as the fundamental physics behind the electrical and mechanical oscillator are very different.
Integration of the harmonic oscillator's equations of motion, \cref{eq:ho_me}, yields
\begin{equation}
    E
    =
    \frac{1}{2}m\dot{q}^2
    +
    \frac{1}{2}m\omega_0^2q^2
    \label{eq:ho_energy}
\end{equation}
where the integration constant $E$ has the interpretation of being the energy available to the system.

Most physical problems reduce to the harmonic oscillator after performing a series expansion up of the potential in a dynamical variable up to quadratic (or harmonic) order.\footnote{The harmonic oscillator is the most simple non-trivial system for which there exists a closed solution.}
To clarify, let us consider the energy of a physical system with potential $V(q)$
\begin{equation}
    E
    =
    \frac{1}{2}m^2\dot{q}^2
    +
    V(q)
    \label{eq:pp_energy}.
\end{equation}
We perform a Taylor series expansion of the potential around some configuration $q_0$ minimizing the potential energy
\begin{equation}
    V(q)
    =
    \sum_{n=1}\frac{1}{n!}\eval{\dv[n]{V}{q}}_{q_0}(q-q_0)^n
    =
    \frac{1}{2}\eval{\dv[2]{V}{q}}_{q_0}(q-q_0)^2
    +
    \mathcal{O}(q^3)
    \label{eq:pp_potential}
\end{equation}
and recover the potential energy of the harmonic oscillator given in \cref{eq:ho_energy}.
In physics we mainly deal with continuously differentiable functions\footnote{Discontinuities in motion are unphysical in the sense that they require infinite energy. An exception being macroscopic or stochastic dynamics, e.g., Brownian motion.} for which a Taylor series exists.
If we find an equilibrium configuration of our physical system, a finite expansion around such point yields in most cases a very good approximation of the problem, and the harmonic solution to the quadratic term describes the dominating dynamics.

\subsection{Fourier analysis of the equation of motion}

A closed solution to the differential equation encoding the harmonic dynamics
\begin{equation}
    \ddot{q}
    =
    -\omega_0^2q
    \label{eq:ho_eom}
\end{equation}
can be found using Fourier analysis.
Physical systems have finite energy, \Cref{eq:ho_energy}, the dynamical variable $q(t)$ is square-integrable\footnote{Let $q^*$ be a global maximum, then we have $E=\frac{k}{2}{q^*}^2$, and thus $\int_0^T\dd{t}q(t)^2<\int_0^T2E/k=2ET/k<\infty$, i.e., $q(t)$ is square-integrable on $[0,T]$.} and the (inverse) Fourier transform
\begin{equation}
    q(t)
    =
    \int_0^T\dd{\omega}e^{i\omega t}q(\omega)
    \label{eq:ho_ft}
\end{equation}
exists.
Inserting \cref{eq:ho_ft} into \cref{eq:ho_eom} reduces the differential equation, \cref{eq:ho_eom}, to the algebraic equation
\begin{equation}
    -\omega^2q(\omega)
    =-\omega_0^2q(\omega)
    \label{eq:ho_eom_ft}
\end{equation}
which is solved by
\begin{equation}
    q(\omega)
    =
    c_1\delta(\omega-\omega_0)
    +
    c_2\delta(\omega+\omega_0)
    \label{eq:ho_eom_ft_sol}.
\end{equation}
\Cref{eq:ho_eom_ft_sol} corresponds to a spectrum of two fundamental frequencies $\pm\omega_0$ with complex weights $c_1,c_2\in\mathbb{C}$.
The requirement that $q(t)$ is real constraints the spectrum
\begin{equation}
    q^*(\omega)
    =
    q(-\omega)
\end{equation}
and the spectral weights of \cref{eq:ho_eom_ft_sol} must fulfill $c_1=c_2^*$.
We introduce $A,\varphi\in\mathbb{R}$ and rewrite the spectrum, \cref{eq:ho_eom_ft_sol}, to be real
\begin{equation}
    q(\omega)
    =
    \frac{A}{2}
    \left\{
    \delta(\omega-\omega_0)e^{+i\varphi}
    +
    \delta(\omega+\omega_0)e^{-i\varphi}
    \right\}
    \label{eq:ho_eom_ft_sol_fin}.
\end{equation}
By inserting \cref{eq:ho_eom_ft_sol_fin} into \cref{eq:ho_ft}, we find our way back to the time domain
\begin{equation}
    q(t)
    =
    \frac{A}{2}
    \int\dd{\omega}e^{i\omega t}q(\omega)
    =
    A\cos(\omega_0t+\varphi)
    \label{eq:ho_eom_sol}.
\end{equation}
Two initial conditions, for instance, $q(0),q(T)$, fix the free parameters $A$ and $\varphi$, and \cref{eq:ho_eom_sol} describes the displacement for all subsequent times.

\subsection{Lagrange and Hamilton formalism}

So far, we have used Newtonian mechanics to find the equation of motion of the mechanical oscillator.
Now, we discuss the harmonic oscillator in the language of the Lagrangian and Hamiltonian formalism which turn out to be essential for subsequent steps.

The Lagrangian of the harmonic oscillator is defined as the difference between the kinetic and the potential energy
\begin{equation}
    L
    =
    \frac{1}{2}m\dot{q}^2
    -
    \frac{1}{2}m\omega_0^2q^2
    \label{eq:ho_lagrangian}.
\end{equation}
In contrast to the Newtonian approach, where we need to account for every relevant force, the Lagrangian approach only requires knowledge of the kinetic and potential energies, which are more intuitive to suspect.\footnote{After choosing a convenient set of coordinates, the Lagrangian approach also reproduces inertial forces.}
Given the Lagrange function, finding the equation of motion reduces to calculating the derivatives in the Euler-Lagrange equation
\begin{equation}
    0
    =
    \dv{t}\pdv{L}{\dot{q}}
    -
    \pdv{L}{q}
    =
    m\ddot{q}+m\omega_0^2q
    \label{eq:ho_euler_lagrange}
\end{equation}
reproducing the equation of motion found by Newton's force force law and electrical circuit analysis \cref{eq:ho_me}.

In the Hamilton approach to mechanics, we replace the time-derivative of the dynamical variable $\dot{q}$ by a new dynamical variable, the canonical momentum
\begin{equation}
    p
    =
    \pdv{L}{\dot q}
    \label{eq:canonical_momentum}.
\end{equation}
Using the Legendre transformation
\begin{equation}
    H(p,q)
    =
    p\dot{q}(p,q)
    -
    L(p,q)
    \label{eq:legendre_transform},
\end{equation}
we can cast the Lagrange function $L$ to the Hamilton function $H$.
The Hamilton function is a function of the canonical momentum $p$ and position $q$ known as conjugate variables.
The dynamics of the conjugate variables are determined by the Hamilton equations of motion
\begin{align}
    \dot{q}
    =
    +\pdv{H}{p}
    =
    +\left\{p,H\right\}
    ,&&
    \dot{p}
    =
    -\pdv{H}{q}
    =
    -\left\{q,H\right\}
    \label{eq:hamilton_eom}
\end{align}
where we introduced the Poisson bracket operator $\left\{\cdot,\cdot\right\}$.
The action of the Poisson bracket operator acting on two functions $f(p,q,t),g(p,q,t)$ is defined by
\begin{equation}
    \left\{f,g\right\}
    =
    \pdv{f}{q}\pdv{g}{p}
    -
    \pdv{f}{p}\pdv{g}{q}
    \label{eq:poisson_bracket}.
\end{equation}
The Poisson bracket operator allows us to replace differential by algebraic equations.
For example, we can rewrite the time evolution of a function $f(p,q)$ as
\begin{equation}
    \dv{f}{t}
    =
    \pdv{f}{q}\dv{q}{t}
    +
    \pdv{f}{p}\dv{p}{t}
    =
    \pdv{f}{q}\pdv{H}{p}
    -
    \pdv{f}{p}\pdv{H}{q}
    =
    \left\{f,H\right\}
    \label{eq:hamilton_time_evol}.
\end{equation}
Introducing the Poisson bracket operator may appear artificial at first but it turns out that replacing the Poisson brackets with commutators leads us to quantum mechanics.

For the harmonic oscillator, the canonical momentum is equal to the physical momentum $p=m\dot{q}$\footnote{Generally, the canonical momentum is not equal to the physical momentum, e.g., the canonical momentum of a particle with electrical charge $e$ in an electric field given by the vector potential $A$ is $p-eA$.} and the Hamilton function
\begin{equation}
    H
    =
    \frac{p^2}{2m}
    +
    \frac{m\omega_0^2}{2}q^2
    \label{eq:ho_hamilton}
\end{equation}
is equal to the energy, \cref{eq:ho_energy}, rewritten in terms of the physical momentum $p=m\dot{q}$.
The Hamilton equations of motion, \cref{eq:hamilton_eom} for the harmonic oscillator yield
\begin{align}
    \dot{q}
    =
    \frac{p}{m},
    &&
    \dot{p}
    =
    -m\omega_0^2q
    \label{eq:ho_hamilton_eom}.
\end{align}
With the time derivative of the position variable $\ddot{q}=\dot{p}/m=\omega_0^2q$ we recover the equation of motion we derived with Newton, \cref{eq:ho_newton}, and the Lagrange formalism, \cref{eq:ho_eom}.

\subsection{Canonical quantization and dynamical pictures}

In the canonical quantization prescription, we promote the conjugate variables $q(t),p(t)$ to linear operators $\hat{q}(t),\hat{p}(t)$\footnote{In contrast to variables which can be thought of as real functions, linear operators map between vector spaces.} satisfying the commutation relation
\begin{equation}
    \comm{\hat{q}}{\hat{p}}
    =
    \hat{q}\hat{p}
    -
    \hat{p}\hat{q}
    =
    i
    \label{eq:comm_pm},
\end{equation}
and thereby implementing the measurement uncertainty in quantum mechanics.\footnote{In quantum mechanics, a measurement yields an eigenvalue of an (self-adjoint) operator. If operators do not commute, they cannot be simultaneously diagonalized, thus, there exists no shared eigenbasis, and the measurement order is not independent.}
Replacing the Poisson brackets in \cref{eq:hamilton_time_evol} with commutators yields us the time-evolution of an operator $\hat{A}(\hat{q},\hat{p})$ in the Heisenberg picture
\begin{equation}
    \dv{\hat{A}}{t}
    =
    \comm{\hat{H}}{\hat{A}}
    \label{eq:heisenberg_time_evol}
\end{equation}
which is solved by unitary transformation with the time evolution operator $\hat{U}$
\begin{equation}
    \hat{A}(t)
    =
    \hat{U}^\dagger(t)
    \hat{A}(0)
    \hat{U}(t)
    =
    e^{+i\hat{H}t}\hat{A}(0)e^{-i\hat{H}t}
    \label{eq:heisenberg_time_evol_sol}.
\end{equation}
In the Heisenberg picture, operators are time-dependent but states are not.
In the Schrödinger picture, the time evolution operator $\hat{U}$ is part of the state
\begin{equation}
    \ket{\psi(t)}
    =
    \hat{U}(t)\ket{\psi(0)}
    =
    e^{-i\hat{H}t}\ket{\psi(0)}
    \label{eq:schroedinger_state}.
\end{equation}
and the operator itself is time-independent.\footnote{Heisenberg and Schrödinger picture predict the same expectation values $\expval{\hat{A}(t)}{\psi(0)}=\expval{\hat{U}^\dagger(t)\hat{A}(0)\hat{U}}{\psi(0)}=\expval{\hat{A}(0)}{\psi(t)}$.}
Taking the derivative w.r.t. time of \cref{eq:schroedinger_state} yields the Schrödinger equation
\begin{equation}
    \pdv{t}\ket{\psi(t)}
    =
    \hat{H}\ket{\psi(t)}
    \label{eq:schroedinger_time_evol}.
\end{equation}
Solving field dynamics in the Schrödinger picture requires infinite-dimensional differential operators, which are mathematical problematic compared to the Heisenberg picture's algebraic relations~\cite[p.~19]{Fulling1989}.\footnote{Ref.~\cite[p.~19]{Fulling1989} gives an in-depth comparison of both pictures.}
However, it is possible to solve the field dynamics in the Heisenberg picture and transform it into the Schrödinger picture to gain additional insights.

\subsection{Hermite polynomials as solution to the Schrödinger equation}

We perform the canonical quantization procedure and obtain the quantum Hamiltonian
\begin{equation}
    \hat{H}
    =
    \frac{\hat{p}^2}{2m}
    +
    \frac{m\omega_0^2}{2}\hat{q}^2
    \label{eq:qho_hamilton}.
\end{equation}
The conservation of energy suggests the existence of an energy eigenbasis
\begin{equation}
    \hat{H}\ket{n}
    =
    E_n\ket{n}
    \label{eq:qho_eigen_energy}.
\end{equation}
Using some length mathematical arguments and the canonical commutation relation, \cref{eq:comm_pm}, it is possible to derive~\cite[p.~27]{Mukhanov2007}
\begin{equation}
    \hat{p}\ket{q}
    =
    -i\pdv{q}\ket{q}
    \label{eq:qho_eigen_mp}.
\end{equation}
Taking the self-adjoint of \cref{eq:qho_eigen_energy}, then applying $\ket{q}$ to the right, and using \cref{eq:qho_eigen_mp}, we can find the solutions to the one-dimensional quantum harmonic oscillator to solving the differential equation
\begin{equation}
    E_n\braket{n}{q}
    =
    \bra{n}\hat{H}\ket{q}
    =
    \left(-\frac{1}{2m}\pdv[2]{q}+\frac{m\omega_0^2}{2}q^2\right)\braket{n}{q}
    \label{eq:qho_pdv}.
\end{equation}
The solution to the differential equation is~\cite[p.~51]{Griffiths2017}
\begin{equation}
    \psi_n(q)
    =
    \braket{n}{q}
    =
    \left(\frac{m\omega_0}{\pi}\right)^{1/4}
    \frac{1}{\sqrt{2^nn!}}
    H_n\left(\sqrt{m\omega_0}q\right)
    \exp\left\{-\frac{m\omega_0}{2}q^2\right\}
    \label{eq:qho_pdv_sol}
\end{equation}
wherein $H_n$ denotes the $n$th Hermite polynomial.
It is common practice to read \cref{eq:qho_pdv_sol} as the position representation of the wave function corresponding to the $n$th energy level.
Momentum and position representation of the wave function relate by the Fourier transform\footnote{Instead of using a position state to derive \cref{eq:qho_pdv}, we could have used a momentum state by using the momentum state equivalent of \cref{eq:qho_eigen_mp} $q\ket{p}=i\pdv{q}\ket{q}$.}
\begin{equation}
    \psi_n(p)
    =
    \bra{n}\left(\int\dd{q}\ketbra{q}{q}\right)\ket{p}
    =
    \int\dd{q}\braket{q}{p}\braket{n}{q}
    =
    \int\frac{\dd{q}}{\sqrt{2\pi}}\psi_n(q)e^{ipq}
    \label{eq:qho_pdv_sol_mom}
\end{equation}
where $\braket{q}{p}=e^{ipq}/\sqrt{2\pi}$ is the solution to the differential equation one obtains after applying $\bra{p}$ to \cref{eq:qho_eigen_mp} and choosing a normalization compatible with the symmetric Fourier transform.
The time evolution is encoded in the time evolution operator, for instance, in the Schrödinger picture, \cref{eq:schroedinger_time_evol},
\begin{equation}
    \psi_n(q,t)
    =
    \psi_n(q)
    e^{-iE_nt}
    .
\end{equation}

\subsection{Energy eigenstates, creation and annihilation operators}

While \cref{eq:qho_pdv_sol} represents the probability amplitude dynamics in terms of position (and momentum) variables, the energy spectrum remains unknown.
If we could factor the quantum harmonic oscillator's Hamiltonian as a product of operators, finding the energy spectrum reduces to solving an eigenvalue equation.
It turns out that
\begin{align}
    \hat{x}
    =
    \sqrt{\frac{1}{2m\omega_0}}
    \left(\hat{a}^\dagger+\hat{a}\right)
    &&
    \hat{p}
    =
    i\sqrt{\frac{m\omega_0}{2}}
    \left(\hat{a}^\dagger-\hat{a}\right)
    \label{eq:qho_pm_ac}
\end{align}
respective
\begin{align}
    \hat{a}
    =
    \frac{1}{\sqrt{2m\omega_0}}
    \left(m\omega_0\hat{x}+i\hat{p}\right)
    &&
    \hat{a}^\dagger
    =
    \frac{1}{\sqrt{2m\omega_0}}
    \left(m\omega_0\hat{x}-i\hat{p}\right)
    \label{eq:qho_ac_pm}
\end{align}
is such a factorization known as the annihilation and creation operator.
Before we cast the Hamiltonian in terms of these new operators, we need to find the commutator between them
\begin{equation}
    \comm{\hat{a}}{\hat{a}^\dagger}
    =
    \hat{a}\hat{a}^\dagger
    -
    \hat{a}^\dagger\hat{a}
    =
    1
    \label{eq:qho_comm_ac}
\end{equation}
by inserting \cref{eq:qho_ac_pm} and using the commutator of the position and momentum operator, \cref{eq:comm_pm}.
Inserting \cref{eq:qho_pm_ac} into \cref{eq:qho_hamilton} yields
\begin{equation}
    \hat{H}
    =
    \omega_0\left(\hat{a}^\dagger\hat{a}+\frac{1}{2}\right)
    \label{eq:qho_hamilton_ac}
\end{equation}
after using the commutator, \cref{eq:qho_comm_ac}.
The eigenspectrum of $\hat{a}^\dagger\hat{a}$ is equal to the natural numbers including zero $\mathbb{N}_0$~\cite[p.~506]{Cohen2019} which suggests to name
\begin{equation}
    \hat{N}
    =
    \hat{a}^\dagger\hat{a}
    \label{eq:qho_number_operator}
\end{equation}
the number operator and the number state $\ket{n}$ an eigenstate thereof.
The number operator commutes with the Hamilton operator, thus, the number state is also an energy eigenstate
\begin{equation}
    E_n\ket{n}
    =
    \hat{H}\ket{n}
    =
    \omega_0\left(n+\frac{1}{2}\right)
    \label{eq:qho_number_energy}
\end{equation}
which answers the question of the energy spectrum of the quantum harmonic oscillator.

However, there is more about the annihilation and creation operator than finding the energy spectrum, in particular, we have not yet explained why they are called annihilation and creation operator.
Let us consider the eigenvalue of the number operator $\hat{N}$ with respect to $\hat{a}^\dagger\ket{n}$
\begin{equation}
    \hat{N}\hat{a}^\dagger\ket{n}
    =
    \left(\comm{\hat{N}}{\hat{a}^\dagger}+\hat{a}^\dagger\hat{N}\right)\ket{n}
    =
    (n+1)\hat{a}^\dagger\ket{n}
    ,
\end{equation}
i.e., $\hat{a}^\dagger\ket{n}\propto\ket{n+1}$.
We fix the proportionality constant by requiring $n=\expval{\hat{a}^\dagger\hat{a}}{n}$ and summarize
\begin{equation}
    \hat{a}^\dagger\ket{n}
    =
    \sqrt{n+1}\ket{n}
    \label{eq:qho_creation}.
\end{equation}
We conclude that $\hat{a}^\dagger$ is the creation operator because it creates an additional excitation when acting on a number state $\ket{n}$.
In the same sense, we find that
\begin{equation}
    \hat{a}\ket{n}
    =
    \sqrt{n}\ket{n-1}
    \label{eq:qho_annhiliation}
\end{equation}
explains why $\hat{a}$ is named the annihilation operator.
Consistency of the number states being natural numbers requires the annihilation operator to destroy the ground (or vacuum) state
\begin{equation}
    \hat{a}\ket{0}
    =
    0
    \label{eq:qho_vacuum_annihilation}.
\end{equation}
Applying the creation operator iteratively, lets us construct a number state from the vacuum state via
\begin{equation}
    \frac{(\hat{a}^\dagger)^n}{\sqrt{n!}}\ket{0}
    =
    \ket{n}
    \label{eq:qho_creation_number}.
\end{equation}
Annihilation and creation operators form an algebra that generalizes to highly advanced Hilbert spaces and will be used in the subsequent sections.

\subsection{Coherent states from an external classical source}
		
		\addcontentsline{toc}{section}{References}
		\printbibliography[title=References]
	\end{refsection}

	\chapter{Classical beam splitter}
	\begin{refsection}
		\include{chapters/appendix/classical-beam-splitter}
		
		\addcontentsline{toc}{section}{References}
		\printbibliography[title=References]
	\end{refsection}
	
	\chapter{Klein-Gordon observables}
	\begin{refsection}
		\section{Single-particle}

\subsection{Number, energy and momentum observables}

We define the operator
\begin{equation}
	\hat{A}
	=
	g(\vb{k})
	\hat{a}^\dagger(\vb{k})
	\hat{a}(\vb{k})
\end{equation}
where $g$ is allowed to be vector-valued, e.g., $g(\vb{p})=\vb{p}$.
To calculate the first two moments of $\hat{A}$, we need the auxiliary results
\begin{equation}
	\begin{split}
		\expval{\hat{a}(\vb{q})\hat{A}\hat{a}^\dagger(\vb{p})}{0}
		&=
		\int\frac{\dd[3]{k}}{(2\pi)^3}
		g(\vb{k})
		\expval{\hat{a}(\vb{q})\hat{a}^\dagger(\vb{k})\hat{a}(\vb{k})\hat{a}^\dagger(\vb{p})}{0}
		\\
		&=
		\int\frac{\dd[3]{k}}{(2\pi)^3}
		g(\vb{k})
		\expval{\hat{a}(\vb{q})\hat{a}^\dagger(\vb{k})}{0}
		(2\pi)^3\delta^{(3)}(\vb{k}-\vb{p})
		\\
		&=
		g(\vb{p})
		\expval{\hat{a}(\vb{q})\hat{a}^\dagger(\vb{p})}{0}
		\\
		&=
		g(\vb{p})
		(2\pi)^3\delta^{(3)}(\vb{q}-\vb{p})
	\end{split}
\end{equation}
and
\begin{equation}
	\begin{split}
		\expval{\hat{a}(\vb{q})\hat{A}^2\hat{a}^\dagger(\vb{p})}{0}
		&=
		\int\frac{\dd[3]{k_1}}{(2\pi)^3}
		\int\frac{\dd[3]{k_2}}{(2\pi)^3}
		g(\vb{k}_1)g(\vb{k}_2)
		\\
		&\times
		\expval{
			\hat{a}(\vb{q})
			\hat{a}^\dagger(\vb{k}_1)
			\hat{a}(\vb{k}_1)
			\hat{a}^\dagger(\vb{k}_2)
			\hat{a}(\vb{k}_2)
			\hat{a}^\dagger(\vb{p})
		}{0}
		\\
		&=
		\int\frac{\dd[3]{k_1}}{(2\pi)^3}
		\int\frac{\dd[3]{k_2}}{(2\pi)^3}
		g(\vb{k}_1)g(\vb{k}_2)
		\\
		&\times
		(2\pi)^3\delta^{(3)}(\vb{k}_1-\vb{q})
		\expval{
			\hat{a}(\vb{k}_1)
			\hat{a}^\dagger(\vb{k}_2)
		}{0}
		(2\pi)^3\delta^{(3)}(\vb{k}_2-\vb{p})
		\\
		&=
		g(\vb{q})g(\vb{p})
		\expval{
			\hat{a}(\vb{q})
			\hat{a}^\dagger(\vb{p})
		}{0}
		\\
		&=
		g(\vb{p})^2
		(2\pi)^3\delta^{(3)}(\vb{q}-\vb{p})
	\end{split}
	.
\end{equation}
The first moment then turns out to be
\begin{equation}
	\begin{split}
		\expval{\hat{A}}{f}
		&=
		\int\frac{\dd[3]{p}}{(2\pi)^3\sqrt{2\omega(\vb{p})}}
		\int\frac{\dd[3]{q}}{(2\pi)^3\sqrt{2\omega(\vb{q})}}
		f(\vb{p})f(\vb{q})^*
		\expval{\hat{a}(\vb{q})\hat{A}\hat{a}^\dagger(\vb{p})}{0}
		\\
		&=
		\int\frac{\dd[3]{p}}{(2\pi)^3\sqrt{2\omega(\vb{p})}}
		\int\frac{\dd[3]{q}}{(2\pi)^3\sqrt{2\omega(\vb{q})}}
		f(\vb{p})f(\vb{q})^*
		g(\vb{p})
		(2\pi)^3\delta^{(3)}(\vb{q}-\vb{p})
		\\
		&=
		\int\frac{\dd[3]{p}}{(2\pi)^32\omega(\vb{p})}
		g(\vb{p})
		\abs{f(\vb{p})}^2
	\end{split}
\end{equation}
and the second moment is
\begin{equation}
	\begin{split}
		\expval{\hat{A}^2}{f}
		&=
		\int\frac{\dd[3]{p}}{(2\pi)^3\sqrt{2\omega(\vb{p})}}
		\int\frac{\dd[3]{q}}{(2\pi)^3\sqrt{2\omega(\vb{q})}}
		f(\vb{p})f(\vb{q})^*
		\expval{\hat{a}(\vb{q})\hat{A}^2\hat{a}^\dagger(\vb{p})}{0}
		\\
		&=
		\int\frac{\dd[3]{p}}{(2\pi)^3\sqrt{2\omega(\vb{p})}}
		\int\frac{\dd[3]{q}}{(2\pi)^3\sqrt{2\omega(\vb{q})}}
		f(\vb{p})f(\vb{q})^*
		g(\vb{p})^2
		(2\pi)^3\delta^{(3)}(\vb{q}-\vb{p})
		\\
		&=
		\int\frac{\dd[3]{p}}{(2\pi)^32\omega(\vb{p})}
		g(\vb{p})^2
		\abs{f(\vb{p})}^2
	\end{split}
	.
\end{equation}
We can now identify $g(\vb{p})$ with $1,\omega(\vb{p}),\vb{p}$ to obtain the first two moments of number, Hamilton, or momentum operator.

\subsection{Center-of-mass position and dispersion}

The coordinate the wave function is
\begin{equation}
	\begin{split}
		\psi(t,\vb{x})
		&=
		\int\frac{\dd[3]{p}}{(2\pi)^3\sqrt{2\omega(\vb{p})}}
		\int\frac{\dd[3]{q}}{(2\pi)^3\sqrt{2\omega(\vb{q})}}
		f(\vb{p})e^{iq_\mu x^\mu}
		\expval{\hat{a}(\vb{q})\hat{a}^\dagger(\vb{p})}{0}
		\\
		&=
		\int\frac{\dd[3]{p}}{(2\pi)^3\sqrt{2\omega(\vb{p})}}
		\int\frac{\dd[3]{q}}{(2\pi)^3\sqrt{2\omega(\vb{q})}}
		f(\vb{p})e^{iq_\mu x^\mu}
		(2\pi)^3\delta^{(3)}(\vb{q}-\vb{p})
		\\
		&=
		\int\frac{\dd[3]{p}}{(2\pi)^32\omega(\vb{p})}
		f(\vb{p})e^{ip_\mu x^\mu}
	\end{split}
\end{equation}
the probability current in terms of the wave function is
\begin{equation}
	\begin{split}
		j_\mu(t,\vb{x})
		&=
		2
		\Im\left\{
			\psi(t,\vb{x})
			\partial_\mu
			\psi(t,\vb{x})
		\right\}
		\\
		&=
		2
		\Im\left\{
			\int\frac{\dd[3]{p}}{(2\pi)^32\omega(\vb{p})}
			f(\vb{p})^*e^{-ip_\mu x^\mu}
			\int\frac{\dd[3]{q}}{(2\pi)^32\omega(\vb{q})}
			f(\vb{q})iq_\mu e^{+iq_\mu x^\mu}
		\right\}
		\\
		&=
		2
		\int\frac{\dd[3]{p}}{(2\pi)^32\omega(\vb{p})}
		\int\frac{\dd[3]{q}}{(2\pi)^32\omega(\vb{q})}
		\Im\left\{
			f(\vb{p})^*
			f(\vb{q})
			iq_\mu
			e^{-i(p_\mu-q_\mu)x^\mu}
		\right\}
		\\
		&=
		2
		\int\frac{\dd[3]{p}}{(2\pi)^32\omega(\vb{p})}
		\int\frac{\dd[3]{q}}{(2\pi)^32\omega(\vb{q})}
		q_\mu
		\Re\left\{
			f(\vb{p})^*
			f(\vb{q})
			e^{-i(p_\mu-q_\mu)x^\mu}
		\right\}
		\\
		&=
		\int\frac{\dd[3]{p}}{(2\pi)^32\omega(\vb{p})}
		\int\frac{\dd[3]{q}}{(2\pi)^32\omega(\vb{q})}
		\left\{
			q_\mu
			f(\vb{p})
			f(\vb{q})^*
			e^{+i(p_\mu-q_\mu)x^\mu}
			+
			q_\mu
			f(\vb{p})^*
			f(\vb{q})
			e^{-i(p_\mu-q_\mu)x^\mu}
		\right\}
		\\
		&=
		\int\frac{\dd[3]{p}}{(2\pi)^32\omega(\vb{p})}
		\int\frac{\dd[3]{q}}{(2\pi)^32\omega(\vb{q})}
		\left\{
			q_\mu
			f(\vb{p})
			f(\vb{q})^*
			e^{+i(p_\mu-q_\mu)x^\mu}
			+
			p_\mu
			f(\vb{q})^*
			f(\vb{p})
			e^{-i(q_\mu-p_\mu)x^\mu}
		\right\}
		\\
		&=
		\int\frac{\dd[3]{p}}{(2\pi)^32\omega(\vb{p})}
		\int\frac{\dd[3]{q}}{(2\pi)^32\omega(\vb{q})}
		\left\{
			q_\mu
			+
			p_\mu
		\right\}
		f(\vb{q})^*
		f(\vb{p})
		e^{-i(q_\mu-p_\mu)x^\mu}
	\end{split}
\end{equation}

\subsubsection{Center-of-mass position}

The mean center-of-mass can be expressed in terms of the velocity
\begin{equation}
	\begin{split}
		\overline{\vb{x}}
		&=
		\int\dd[3]{x}\vb{x}\rho(t,\vb{x})
		\\
		&=
		\int\frac{\dd[3]{p}}{(2\pi)^32\omega(\vb{p})}
		\int\frac{\dd[3]{q}}{(2\pi)^32\omega(\vb{q})}
		\left\{
			\omega(\vb{p})
			+
			\omega(\vb{q})
		\right\}
		\\
		&\times
		f(\vb{q})^*
		f(\vb{p})
		e^{-i\left(\omega(\vb{q})-\omega(\vb{p})\right)t}
		\int\dd[3]{x}
		\vb{x}
		e^{+i(\vb{q}-\vb{p})\vdot\vb{x}}
		\\
		&=
		\int\frac{\dd[3]{p}}{(2\pi)^32\omega(\vb{p})}
		\int\frac{\dd[3]{k}}{(2\pi)^32\omega(\vb{k})}
		\left\{
			\omega(\vb{p})
			+
			\omega(\vb{p}+\vb{k})
		\right\}
		\\
		&\times
		f(\vb{p}+\vb{k})^*
		f(\vb{p})
		e^{-i\left(\omega(\vb{p}+\vb{k})-\omega(\vb{p})\right)t}
		\int\dd[3]{x}
		\vb{x}
		e^{+i(\vb{k})\vdot\vb{x}}
		\\
		&=
		-i
		\int\frac{\dd[3]{p}}{(2\pi)^32\omega(\vb{p})}
		f(\vb{p})
		e^{+i\omega(\vb{p})t}
		\\
		&\times
		\int\dd[3]{k}
		\frac{\omega(\vb{p})+\omega(\vb{p}+\vb{k})}{2\omega(\vb{p}+\vb{k})}
		f(\vb{p}+\vb{k})^*
		e^{-i\omega(\vb{p}+\vb{k})t}
		\grad_{\vb{k}}
		\delta^{(3)}(\vb{k})
	\end{split}
	\label{qkg:mean_position}
\end{equation}
we start with the second term, because we can reuse a part of it
\begin{equation}
	\begin{split}
		&
		\frac{1}{2}
		\int\dd[3]{k}
		f(\vb{p}+\vb{k})^*
		e^{-i\omega(\vb{p}+\vb{k})t}
		\grad_{\vb{k}}
		\delta^{(3)}(\vb{k})
		\\
		=&\
		-
		\frac{1}{2}
		\eval{\grad_{\vb{k}}}_{\vb{k}=0}
		\left\{
			f(\vb{p}+\vb{k})^*
			e^{-i\omega(\vb{p}+\vb{k})t}
		\right\}
		\\
		=&\
		-
		\frac{1}{2}
		e^{-i\omega(\vb{p})t}
		\left\{
			\left(\grad_{\vb{k}}f(\vb{p}+\vb{k})\right)^*
			-
			it\grad_{\vb{k}}\omega(\vb{p}+\vb{k})
		\right\}_{\vb{k}=0}
		\\
		=&\
		-
		\frac{1}{2}
		e^{-i\omega(\vb{p})t}
		\frac{\vb{p}}{\norm{\vb{p}}}
		\vdot\grad_{\vb{k}}
		\left\{
			f(\vb{p})^*
			-
			i\grad_{\vb{p}}\omega(\vb{p})t
		\right\}
	\end{split}
\end{equation}
and then the first term
\begin{equation}
	\begin{split}
		&
		\frac{\omega(\vb{p})}{2}
		\int\dd[3]{k}
		\frac{
			f(\vb{p}+\vb{k})^*
			e^{-i\omega(\vb{p}+\vb{k})t}
		}{2\omega(\vb{p}+\vb{k})}
		\grad_{\vb{k}}
		\delta^{(3)}(\vb{k})
		\\
		=&
		-
		\frac{\omega(\vb{p})}{2}
		\eval{\grad_{\vb{k}}}_{\vb{k}=0}
		\frac{
			f(\vb{p}+\vb{k})^*
			e^{-i\omega(\vb{p}+\vb{k})t}
		}{\omega(\vb{p}+\vb{k})}
		\\
		=&
		-
		\frac{\omega(\vb{p})}{2}
		\frac{
			\omega(\vb{p})
			\eval{\grad_{\vb{k}}}_{\vb{k}=0}
			f(\vb{p}+\vb{k})^*
			e^{-i\omega(\vb{p}+\vb{k})t}
			-
			f(\vb{p})^*
			e^{-i\omega(\vb{p})t}
			\eval{\grad_{\vb{k}}}_{\vb{k}=0}
			\omega(\vb{p}+\vb{k})
		}{\omega(\vb{p})^2}
		\\
		=&
		-
		\frac{1}{2}
		\eval{\grad_{\vb{k}}}_{\vb{k}=0}
		f(\vb{p}+\vb{k})^*
		e^{-i\omega(\vb{p}+\vb{k})t}
		+
		\frac{1}{2\omega(\vb{p})}
		f(\vb{p})^*
		e^{-i\omega(\vb{p})t}
		\eval{\grad_{\vb{k}}}_{\vb{k}=0}
		\omega(\vb{p}+\vb{k})
		\\
		=&\
		-
		\frac{1}{2}
		e^{-i\omega(\vb{p})t}
		\frac{\vb{p}}{\norm{\vb{p}}}
		\vdot\grad_{\vb{k}}
		\left\{
			f(\vb{p})^*
			-
			i\grad_{\vb{p}}\omega(\vb{p})t
		\right\}
		+
		\frac{1}{2\omega(\vb{p})}
		f(\vb{p})^*
		e^{-i\omega(\vb{p})t}
		\frac{\vb{p}}{\norm{\vb{p}}}\vdot
		\eval{\grad_{\vb{p}}}
		\omega(\vb{p})
	\end{split}
\end{equation}

where we used
\begin{equation}
	\int\dd[3]{x}
	\vb{x}
	e^{+i\vb{k}\vdot\vb{x}}
	=
	\left(-i\grad_{\vb{k}}\right)
	\int\dd[3]{x}
	e^{+i\vb{k}\vdot\vb{x}}
	=
	\left(-i\grad_{\vb{k}}\right)
	(2\pi)^3
	\delta^{(3)}(\vb{k})
\end{equation}
and
\begin{equation}
	\int\dd[3]{k}
	f(\vb{k})
	\grad_{\vb{k}}(\vb{k})
	\delta^{(3)}(\vb{k})
	=
	-
	\int\dd[3]{k}
	\delta^{(3)}(\vb{k})
	\grad_{\vb{k}}(\vb{k})
	f(\vb{k})
	=
	-
	f^\prime(0)
\end{equation}


\section{Coherent state}

		
		\addcontentsline{toc}{section}{References}
		\printbibliography[title=References]
	\end{refsection}

\end{document}
\documentclass[a4paper,appendixprefix,parskip]{scrreprt}

\usepackage[utf8]{inputenc}
\usepackage{amsthm}
\usepackage{amsmath}
\usepackage{amssymb}
\usepackage{authblk}
\usepackage[english]{babel}
\usepackage[backend=biber]{biblatex}
\usepackage{booktabs}
\usepackage{csquotes}
\usepackage[acronym,nonumberlist,toc]{glossaries}
\usepackage{hyperref}
\usepackage{cleveref}
\usepackage{physics}
\usepackage[separate-uncertainty=true]{siunitx}
\usepackage[subpreambles=true]{standalone}
\usepackage{subfig}
\usepackage{xcolor}

\addbibresource{references/articles.bib}
\addbibresource{references/books.bib}

% add bibliography as section (not chapter)
% https://tex.stackexchange.com/questions/568580/make-the-bibliography-as-a-section-in-each-included-chapter
\defbibheading{bibliography}[\bibname]{\section*{#1}}

% custom math commands (transpose, Re, ...)
% prefix equation numbers with section number
\numberwithin{equation}{section}

\DeclarePairedDelimiter{\ceil}{\lceil}{\rceil}
\DeclarePairedDelimiter{\floor}{\lfloor}{\rfloor}
\DeclarePairedDelimiter{\abs}{\lvert}{\rvert}
\DeclarePairedDelimiter{\norm}{\lVert}{\rVert}
\DeclarePairedDelimiter{\bra}{\langle}{\rvert}
\DeclarePairedDelimiter{\ket}{\lvert}{\rangle}
\DeclarePairedDelimiter{\expval}{\langle}{\rangle}
\DeclarePairedDelimiter{\norder}{\mathcolon}{\mathcolon}
\DeclarePairedDelimiter{\anorder}{\typecolon}{\typecolon}
	
\newcommand{\laplace}{\mbfnabla^2}
\newcommand{\trans}{{\scriptscriptstyle\mathsf{T}}}

\newcommand{\vdot}{\cdot}
\newcommand{\vcross}{\vectimes}
\newcommand{\vb}[1]{\symbfup{#1}}
\newcommand{\vu}[1]{\hat{\vb{#1}}}
\newcommand*\dd[2][\relax]{\mathop{\ifx\relax#1\odif{#2}\else \odif[order={#1}]{#2}\fi\,}}

\newcommand{\vacuum}{\ket*{\vb{0}}}

\DeclareMathOperator{\trace}{Tr}
\DeclareMathOperator{\sinc}{sinc}

\AtBeginDocument{
	\let\Re\relax
	\let\Im\relax
	\DeclareMathOperator{\Re}{Re}
	\DeclareMathOperator{\Im}{Im}

	\renewcommand{\div}{\mathop{\mbfnabla\vdot}}
	\newcommand{\curl}{\mathop{\mbfnabla\vectimes}}
}

\DeclarePairedDelimiterX{\comm}[2]{[}{]}{#1,#2}

\DeclarePairedDelimiterX{\braket}[2]{\langle}{\rangle}{#1\delimsize\vert#2}
\DeclarePairedDelimiterX{\ketbra}[1]{\lvert}{\rvert}{#1\rangle\delimsize\langle#1}


% prefix equation numbers with section number
\numberwithin{equation}{section}

% optics
\newacronym{ar}{AR}{anti-reflective}
\newacronym{mzm}{MZM}{Mach-Zehnder modulator}
\newacronym{bs}{BS}{beam splitter}
\newacronym{fc}{FC}{fiber coupler}
\newacronym{qe}{QE}{quantum efficiency}

% physics
\newacronym{dv}{DV}{discrete-variable}
\newacronym{cv}{CV}{continuous-variable}
\newacronym{dof}{DOF}{degrees of freedom}
\newacronym{eom}{EOM}{equation(s) of motion}
\newacronym{pbc}{PBC}{periodic boundary conditions}
\newacronym{bch}{BCH}{Baker-Campbell-Hausdorff}
\newacronym{ccr}{CCR}{canonical commutation relation}

% signal processing
\newacronym{dsp}{DSP}{digital signal processing}
\newacronym{lo}{LO}{local oscillator}
\newacronym{if}{IF}{intermediate frequency}
\newacronym{lp}{LP}{low-pass}
\newacronym{adc}{ADC}{analog-to-digital converter}
\newacronym{dac}{DAC}{digital-to-analog converter}
\newacronym{qam}{QAM}{quadrature amplitude modulation}
\newacronym{qpsk}{QPSK}{quadrature phase-shift keying}

% quantum-key distribution
\newacronym{qkd}{QKD}{quantum-key distribution}
\newacronym{dvqkd}{DV-QKD}{discrete-variable quantum-key distribution}
\newacronym{cvqkd}{CV-QKD}{continuous-variable quantum-key distribution}

\title{A theoretical framework for quantum optical communication - towards CV-QKD}
\author{Bodo Kaiser}
\affil{\textit{bodo.kaiser@huawei.com}}

\begin{document}

	\maketitle
	\tableofcontents

	\chapter{Introduction}
	\begin{refsection}	
		\addcontentsline{toc}{section}{References}
		\printbibliography[title=References]
	\end{refsection}

	\chapter{Quantum field theory of light}
	\begin{refsection}
		\section{Klein-Gordon field}

Choosing a reference frame in which the polarization basis is independent of the momentum, we find that
\begin{equation}
	\hat{\vb{A}}(t,\vb{x})
	=
	\sum_{\lambda=1,2}
	c_\lambda
	\hat\phi(t,\vb{x})
	\boldsymbol{\varepsilon}_\lambda
\end{equation}
with $c_1,c_2\in\mathbb{C}$ and $\abs{c_1}^2+\abs{c_2}^2=1$.
Alternatively, we can perform the replacement
\begin{equation}
	\hat{a}(\vb{p})
	\to
	\sum_{\lambda=1,2}
	\hat{a}_\lambda(\vb{p})
	\boldsymbol{\varepsilon}_\lambda(\vb{p})
\end{equation}
in our results obtained with the Klein-Gordon field to transfer the results to the Maxwell field in the Coulomb gauge.

\subsection{Relativistic field theory}

\begin{definition}[Klein-Gordon Lagrangian]
	The Lorentz-invariant Lagrangian density
	\begin{equation}
		\mathcal{L}
		=
		\frac{1}{2}
		\left(\partial_\mu\phi\right)
		\left(\partial^\mu\phi\right)
		-
		\frac{1}{2}
		m^2\phi^2
		\label{eq:kg_lagrangian}
	\end{equation}
	describes a real scalar field $\phi(t,\vb{x})$ with mass $m>0$, the (massive) Klein-Gordon field.
\end{definition}
\begin{theorem}[Relativistic energy-momentum relation]\label{th:relativistic_energy_momentum}
	Excitations of the Klein-Gordon field satisfy the relativistic energy-momentum relation
	\begin{equation}
		\omega(\vb{p})
		=
		\sqrt{\vb{p}^2+m^2}
		=
		E(\vb{p})
		\label{eq:energy_momentum_relation}
	\end{equation}
	and the physical momentum space is constrained to the momentum lightcone
	\begin{equation}
		V
		=
		\left\{
			(p_0,\vb{p})\in\mathbb{R}^4
			\mid
			p_0^2=\omega(\vb{p})^2
		\right\}
		\label{eq:momentum_lightcone}
		.
	\end{equation}
\end{theorem}
\begin{theorem}[Mode expansion of the Klein-Gordon field]\label{thm:kg_fourier_expansion}
	The Fourier expansion of the Klein-Gordon field
	\begin{equation}
		\phi(t,\vb{x})
		=
		\int_{\mathbb{R}^3}\frac{\dd[3]{p}}{(2\pi)^3\sqrt{2\omega(\vb{p})}}
		\biggl\{
			a(\vb{p})^*
			e^{-ip_\mu x^\mu}
			+
			a(\vb{p})
			e^{+ip_\mu x^\mu}
		\biggr\}_{p_0=\omega(\vb{p})}
	\end{equation}
	satisfies the equations of motion for any choice of $a(\vb{p})=\phi(\omega(\vb{p}),\vb{p})^*$.
\end{theorem}
From now on assume $p_0=\omega(\vb{p})$ if not stated otherwise.
\begin{corollary}[Conjugate momentum density of the Klein-Gordon field]
	The conjugate momentum density of the Klein-Gordon field is
	\begin{equation}
		\pi(t,\vb{x})
		=
		-i
		\int\frac{\dd[3]{p}}{(2\pi)^3}
		\sqrt{2\omega(\vb{p})}
		\left\{
			a(\vb{p})
			e^{-ip_\mu x^\mu}
			-
			a(\vb{p})^*
			e^{+ip_\mu x^\mu}
		\right\}
	\end{equation}
	which follows directly from the Legendre transformation
	\begin{equation}
		\pi(t,\vb{x})
		=
		\partial_t\pdv{\mathcal{L}}{(\partial_t\phi)}
		=
		\partial_t\phi(t,\vb{x})
	\end{equation}
	fundamental to Hamiltonian mechanics.
\end{corollary}
\begin{definition}[Energy-momentum tensor]
	We define the energy-momentum tensor
	\begin{equation}
		T^{\mu\nu}
		=
		\pdv{\mathcal{L}}{(\partial_\mu\phi)}\partial^\nu\phi
		-
		g^{\mu\nu}\mathcal{L}
		\label{eq:energy_momentum_tensor}
	\end{equation}
	which encodes the field's observables in its components.
	For instance, the energy density is encoded in the
	\begin{equation*}
		T^{00}
		=
		\frac{1}{2}
		\left(\partial_t\phi\right)^2
		+
		\frac{1}{2}
		\left(\grad\phi\right)^2
		+
		\frac{1}{2}
		\left(m\phi\right)^2		
	\end{equation*}
	component and the momentum density in the
	\begin{equation*}
		T^{0i}
		=
		-\pi\partial_i\phi		
	\end{equation*}
	component, see Ref.~\cite{Peskin1995}.
\end{definition}
\begin{lemma}\label{thm:kg_total_energy_momentum}
	The total energy and momentum of the Klein-Gordon field,
	\begin{align}
		H
		=
		\int\frac{\dd[3]{p}}{(2\pi)^3}
		\omega(\vb{p})\abs{a(\vb{p})}^2
		&&
		\vb{P}
		=
		\int\frac{\dd[3]{p}}{(2\pi)^3}
		\vb{p}\abs{a(\vb{p})}^2
		\label{eq:kg_energy_momentum}
		,
	\end{align}
	are the energy and momentum weighted by the mode amplitudes.
\end{lemma}

\subsection{Canonical quantization}

\begin{definition}[Canonical quantization]
	In the canonical quantization procedure, the dynamical variables are promoted to operators satisfying the equal-time canonical commutation relations
	\begin{align}
		\comm{\hat\phi(\vb{x})}{\hat\pi(\vb{y})}
		&=
		i\delta^{(3)}(\vb{x}-\vb{y})
		\\
		\comm{\hat\phi(\vb{x})}{\hat\phi(\vb{y})}
		&=
		\comm{\hat\pi(\vb{x})}{\hat\pi(\vb{y})}
		=
		0
		\label{eq:qkg_comm_pm}
		.
	\end{align}
\end{definition}
\begin{theorem}[Klein-Gordon field operators]
	The Klein-Gordon field operators are
	\begin{align}
		\hat\phi(t,\vb{x})
		&=
		\int\frac{\dd[3]{p}}{(2\pi)^3}
		\frac{1}{\sqrt{2\omega(\vb{p})}}
		\left\{
			\hat{a}(\vb{p})
			e^{-ip_\mu x^\mu}
			+
			\hat{a}^\dagger(\vb{p})
			e^{+ip_\mu x^\mu}
		\right\}
		\label{eq:qkg_pos}
		\\
		\hat\pi(t,\vb{x})
		&=
		\int\frac{\dd[3]{p}}{(2\pi)^3}
		\left(-i\sqrt{2\omega(\vb{p})}\right)
		\left\{
			\hat{a}(\vb{p})
			e^{-ip_\mu x^\mu}
			-
			\hat{a}^\dagger(\vb{p})
			e^{+ip_\mu x^\mu}
		\right\}
		\label{eq:qkg_mom}
		,
	\end{align}
	where $\hat\phi(t,\vb{x})$ and $\hat\pi(t,\vb{x})$ satisfy the equal-time canonical commutation relations.
\end{theorem}
\begin{theorem}\label{thm:kg_comm_ac}
	The annihilation and creation operator of the Klein-Gordon field obey the commutation relations
	\begin{align}
		\comm{\hat{a}(\vb{p})}{\hat{a}^\dagger(\vb{q})}
		&=
		(2\pi)^3
		\delta^{(3)}(\vb{p}-\vb{q})
		\\
		\comm{\hat{a}^\dagger(\vb{p})}{\hat{a}^\dagger(\vb{q})}
		&=
		\comm{\hat{a}(\vb{p})}{\hat{a}(\vb{q})}
		=
		0
		\label{eq:kg_comm_ac}
		.
	\end{align}	
\end{theorem}
\begin{definition}[Normal-ordered product]
	The normal-ordered product of annihilation and creation operators
	\begin{equation}
		N\left\{
			\left(
				\hat{a}^\dagger
				\hat{a}
			\right)^l
		\right\}
		=
		\left(\hat{a}^\dagger\right)^l
		\hat{a}^l
	\end{equation}
	places all creation operators to the left and all annihilation operators to the right.
\end{definition}
\begin{definition}[Correspondence principle]
	The correspondence principle is a prescription to find the quantum from the classical observables by promoting the dynamical variables to operators in normal-order.\footnote{See Ref.~\cite[p.~20]{Mukhanov2007} for details on the problem of operator ordering.}
\end{definition}
\begin{corollary}
	The total energy and momentum operators of the Klein-Gordon field are
	\begin{align}
		\hat{H}
		=
		\int\frac{\dd[3]{p}}{(2\pi)^3}
		\omega(\vb{p})\hat{a}^\dagger(\vb{p})\hat{a}(\vb{p})
		&&
		\hat{\vb{P}}
		=
		\int\frac{\dd[3]{p}}{(2\pi)^3}
		\vb{p}\hat{a}^\dagger(\vb{p})\hat{a}(\vb{p})
		\label{eq:qkg_energy_momentum}
		.
	\end{align}
\end{corollary}
\begin{corollary}
	The number operator is the unweighted part of the energy and momentum operators
	\begin{equation}
		\hat{N}
		=
		\int\frac{\dd[3]{p}}{(2\pi)^3}
		\hat{a}^\dagger(\vb{p})
		\hat{a}(\vb{p})
		\label{eq:qkg_number}
		.
	\end{equation}
\end{corollary}

\subsection{Propagators}

The section is based on Ref.~\cite[p.~26]{Peskin1995} and can be skipped if one is not interested in the proofs of the coherent state and interactions.
\begin{definition}[Propagator]
	We define the propagator as
	\begin{equation}
		D(x^\mu-y^\mu)
		=
		\int\frac{\dd[3]{p}}{(2\pi)^32\omega(\vb{p})}
		e^{-ip_\mu (x^\mu-y^\mu)}
		.
	\end{equation}
\end{definition}
\begin{definition}[Positive and negative frequency Klein-Gordon field operator]
	The positive and negative frequency decomposition of the Klein-Gordon field operator
	\begin{equation}
		\hat\phi(t,\vb{x})
		=
		\hat\phi^+(t,\vb{x})
		+
		\hat\phi^-(t,\vb{x})
	\end{equation}
	wherein the positive and negative frequency Klein-Gordon field operators
	\begin{equation}
		\begin{split}
			\hat\phi^+(t,\vb{x})
			&=
			\int\frac{\dd[3]{p}}{(2\pi)^3\sqrt{2\omega(\vb{p})}}
			e^{-ip_\mu x^\mu}
			\hat{a}(\vb{p})
			\\
			\hat\phi^-(t,\vb{x})
			&=
			\int\frac{\dd[3]{p}}{(2\pi)^3\sqrt{2\omega(\vb{p})}}
			e^{+ip_\mu x^\mu}
			\hat{a}^\dagger(\vb{p})
		\end{split}
		\label{eq:qkg_positive_negative_frequency}
	\end{equation}
	are related by the hermitian conjugate $\hat\phi^-(t,\vb{x})=\hat\phi^+(t,\vb{x})^\dagger$.
\end{definition}
\begin{lemma}\label{thm:qkg_full_comm_pn_comm}
	The positive and negative frequency Klein-Gordon field operators satisfy
	\begin{align}
		\comm{\hat\phi^+(x^\mu)}{\hat\phi^-(y^\mu)}
		&=
		D(x^\mu-y^\mu)
		\\
		\comm{\hat\phi^+(x^\mu)}{\hat\phi^+(y^\mu)}
		&=
		\comm{\hat\phi^-(x^\mu)}{\hat\phi^-(y^\mu)}
		=
		0
	\end{align}
\end{lemma}
\begin{lemma}\label{thm:qkg_propagator_kg_comm}
	The commutator of the Klein-Gordon operators is equal to
	\begin{equation}
		\comm{\hat\phi(x^\mu)}{\hat\phi(y^\mu)}
		=
		D(x^\mu-y^\mu)
		-
		D(y^\mu-x^\mu)
	\end{equation}
\end{lemma}
\begin{lemma}\label{thm:qkg_propagator_correlation_function}
	The propagator is equal to the correlation function
	\begin{equation}
		D(x^\mu-y^\mu)
		=
		\expval{\hat\phi(x^\mu)\hat\phi(y^\mu)}{0}
	\end{equation}
\end{lemma}
\begin{definition}
	The retarded propagator is defined to be
	\begin{equation}
		D_R(x^\mu-y^\mu)
		=
		\theta(x^0-y^0)
		\comm{\hat\phi(x^\mu)}{\hat\phi(y^\mu)}
		=
		\begin{cases}
			\comm{\hat\phi(x^\mu)}{\hat\phi(y^\mu)}	& \text{if}\ x^0<y^0
			\\
			0 & \text{otherwise}
		\end{cases}
		,
	\end{equation}
	the advanced propagator is defined to be
	\begin{equation}
		D_A(x^\mu-y^\mu)
		=
		\theta(x^0-y^0)
		\comm{\hat\phi(x^\mu)}{\hat\phi(y^\mu)}
		=
		\begin{cases}
			\comm{\hat\phi(x^\mu)}{\hat\phi(y^\mu)}	& \text{if}\ x^0>y^0
			\\
			0 & \text{otherwise}
		\end{cases}
		,
	\end{equation}
	and the Feynman propagator is defined to be
	\begin{equation}
		\begin{split}
			D_F(x^\mu-y^\mu)
			&=
			\theta(x^0-y^0)
			\expval{\hat\phi(x^\mu)\hat\phi(y^\mu)}{0}
			+
			\theta(y^0-x^0)
			\expval{\hat\phi(y^\mu)\hat\phi(x^\mu)}{0}
			\\
			&=
			\begin{cases}
				D(x^\mu-y^\mu) & \text{if}\ x^0>y^0 \\
				D(y^\mu-x^\mu) & \text{if}\ x^0<y^0 \\
				\frac{1}{2}
				D(x^\mu-y^\mu)
				+
				\frac{1}{2}
				D(y^\mu-x^\mu)
				& \text{if}\ x^0=y^0
			\end{cases}
		\end{split}
		.
	\end{equation}
\end{definition}
\begin{definition}
	The time-ordered product of two operators $\hat{A}(x^\mu),\hat{B}(y^\mu)$ is defined to be
	\begin{equation}
		\begin{split}
			\expval{T\left\{\hat{A}(x^\mu)\hat{B}(y^\mu)\right\}}{0}
			&=
			\theta(x^0-y^0)
			\expval{\hat{A}(x^\mu)\hat{B}(y^\mu)}{0}
			\\
			&+
			\theta(y^0-x^0)
			\expval{\hat{B}(y^\mu)\hat{A}(x^\mu)}{0}
		\end{split}
		.		
	\end{equation}
\end{definition}
\begin{corollary}
	The Feynman propagator is equal to the time-ordered correlation function
	\begin{equation}
		D_F(x^\mu-y^\mu)
		=
		\expval{T\left\{\hat\phi(x^\mu)\hat\phi(y^\mu)\right\}}{0}
		.
	\end{equation}
\end{corollary}
\begin{lemma}\label{thm:propagator_kg_solution}
	The propagators are solutions to the Klein-Gordon equation of a point-source
	\begin{equation}
		\left(
			\partial_\mu
			\partial^\mu
			+
			m^2
		\right)
		D(x^\mu-y^\mu)
		=
		-i\delta^{(4)}(x^\mu-y^\mu)
	\end{equation}
	but to different boundary conditions.
\end{lemma}
\begin{corollary}
	The Fourier transform of the propagators $D(p^\mu)$ satisfies
	\begin{equation}
		\left(
			-
			p_\mu p^\mu
			+
			m^2
		\right)
		D(p^\mu)
		=
		-i
		.
	\end{equation}
\end{corollary}

\subsection{Coordinate wave function properties}

\begin{definition}[Coordinate wave function]
	The equivalent of a coordinate wave function given a state $\ket{\psi}$ and a field operator $\hat\phi$ is
	\begin{equation}
		\psi(t,\vb{x})
		=
		\bra{0}\hat\phi(t,\vb{x})\ket{\psi}
		.
	\end{equation}
\end{definition}
\begin{lemma}\label{thm:coordinate_wave_function_simplified}
	The coordinate wave function reduces to
	\begin{equation}
		\psi(t,\vb{x})
		=
		\bra{0}\hat\phi(t,\vb{x})\ket{\psi}
		=
		\int\frac{\dd[3]{p}}{(2\pi)^3\sqrt{2\omega(\vb{p})}}
		e^{-ip_\mu x^\mu}
		\bra{0}
		\hat{a}(\vb{p})
		\ket{\psi}
		.
	\end{equation}
\end{lemma}
\begin{definition}[Probability current]
	The manifest Lorentz-covariant probability current\footnote{The probability current is the conserved Noether's current of the Klein-Gordon field, see Ref.~\cite[p.~18]{Peskin1995}.} is
	\begin{equation}
		j_\mu(t,\vb{x})
		=
		2
		\Im\left\{
			\psi(t,\vb{x})^*
			\partial_\mu
			\psi(t,\vb{x})
		\right\}
		\label{eq:qkg_probability_current}
	\end{equation}
	where $\psi(t,\vb{x})$ is a coordinate wave function.
	The time component $j^0(t,\vb{x})$ is equal to the probability density $\rho(t,\vb{x})$ and the spatial components $j^i(t,\vb{x})$ are equal to the probability current.
\end{definition}
\begin{definition}[Localization]
	The center-of-mass position of the probability density,
	\begin{equation}
		\expval{\vb{x}(t)}
		=
		\int\dd[3]{x}
		\vb{x}
		\rho(t,\vb{x})
		,
	\end{equation}
	allows us to make statements about the localization of a field excitation.\footnote{There exists no position operator free of contradictions in quantum field theory!}
\end{definition}
\begin{definition}[Group velocity]
	The total probability current
	\begin{equation}
		\expval{\vb{v}(t)}
		=
		\int\dd[3]{x}
		\vb{j}(t,\vb{x})
		\label{eq:group_velocity}
	\end{equation}
	equals the group velocity of a field excitation.
\end{definition}
\begin{definition}[Spatial dispersion]
	The spatial dispersion is equal to the variance of the position weighted by the probability density
	\begin{equation}
		\sigma_x(t)^2
		=
		\expval{\vb{x}(t)^2}
		-
		\expval{\vb{x}(t)}^2
	\end{equation}
	and quantifies the spatial spread of a wave packet.
\end{definition}
		\section{Quantization of the Maxwell field in the Coulomb gauge}

\subsection{Relativistic field theory}

The Lagrangian of the Maxwell field $A^\mu(t,\vb{x})$ reads~\cite[p.~339]{Srednicki2007}
\begin{equation}
	\mathcal{L}
	=
	\frac{1}{2}
	(\partial_\mu A_\nu)
	\left(
		\partial^\nu A^\mu
		-
		\partial^\mu A^\nu
	\right)
\end{equation}
and the covariant generalization of the Euler-Lagrange equations
\begin{equation}
	0
	=
	\partial_\mu
	\pdv{\mathcal{L}}{(\partial_\mu A_\nu)}
	-
	\pdv{\mathcal{L}}{A_\nu}
	=
	\partial_\mu\partial^\mu A^\nu
	-
	\partial^\nu\partial_\mu A^\mu
\end{equation}
leads to the free equations of motion.
We ignore static charges $A_0=0$ and employ the Coulomb gauge $\partial_iA^i=0$ in which the Maxwell field is transverse.
The equations of motion simplify to relativistic wave equation
\begin{equation}
	0
	=
	\partial_\mu\partial^\mu\vb{A}
	=
	\partial_t^2\vb{A}
	-
	\laplacian\vb{A}
	.
\end{equation}

\subsection{Mode decomposition}

In momentum space the transverse field $\vb{A}$ reads
\begin{equation}
	\vb{A}(t,\vb{x})
	=
	\int_{\mathbb{R}^4}\frac{\dd[4]{p}}{(2\pi)^4}
	\vb{A}(p_0,\vb{p})
	e^{ip_0t-i\vb{p}\vdot\vb{x}}
	.
\end{equation}
In momentum space, we can construct a polarization basis $\vu{e}_1(\vb{p}),\vu{e}_2(\vb{p})$ being transverse
\begin{equation}
	\vb{p}\vdot\vu{e}_\lambda(\vb{p})
	=
	0
	,
\end{equation}
orthonormal
\begin{equation}
	\vu{e}_\lambda(\vb{p})
	\vdot
	\vu{e}_{\lambda^\prime}(\vb{p})
	=
	\delta_{\lambda,\lambda^\prime}
\end{equation}
and complete
\begin{equation}
	\sum_{\lambda=1,2}
	\vu{e}_\lambda^i(\vb{p})
	\vu{e}_{\lambda^\prime}^j(\vb{p})
	=
	\delta^{ij}
	-
	\frac{p^ip^j}{\vb{p}^2}
	=
	P_\perp^{ij}(\vb{p})
\end{equation}
with $P_\perp(\vb{p})$ being the transverse projector.
Expressing $\vb{A}(p_0,\vb{p})$ in the polarization basis, we find
\begin{equation}
	\vb{A}(p_0,\vb{p})
	=
	\sum_{\lambda=1,2}
	A_\lambda(p_0,\vb{p})
	\vu{e}_\lambda(\vb{p})
\end{equation}
and the mode decomposition reads
\begin{equation}
	\vb{A}(t,\vb{x})
	=
	\sum_{\lambda=1,2}
	\int_{\mathbb{R}^4}\frac{\dd[4]{p}}{(2\pi)^4}
	A_\lambda(p_0,\vb{p})
	e^{ip_0t-i\vb{p}\vdot\vb{x}}
	\vu{e}_\lambda(\vb{p})
	.
\end{equation}
Inserting the mode decomposition into the relativistic wave equation, we recover the relativistic energy-momentum relation for massless particles
\begin{equation}
	E(\vb{p})
	=
	\omega(\vb{p})
	=
	\vb{p}
	.
\end{equation}
Hence, if the Fourier modes $a_\lambda(p_0,\vb{p})$ satisfy the relativistic energy-momentum relation, $\vb{A}(t,\vb{x})$ satisfies the relativistic wave equation.
We enforce the mode decomposition to satisfy the energy-momentum relation by constraining the integration domain to
\begin{equation}
	V_p
	=
	\left\{
		(p_0,\vb{p})\in\mathbb{R}^4
		\colon
		p_0^2
		=
		\omega(\vb{p})^2
	\right\}
\end{equation}
or equivalent, adding a factor
\begin{equation}
	(2\pi)
	\delta^{(1)}\left(p_0^2-\omega(\vb{p})^2\right)
	=
	(2\pi)
	\frac{
		\delta^{(1)}\left(p_0-\omega(\vb{p})\right)
		-
		\delta^{(1)}\left(p_0+\omega(\vb{p})\right)
	}{\sqrt{2\omega(\vb{p})}}
\end{equation}
to the integrand.
Finally, we arrive at
\begin{equation}
	\vb{A}(t,\vb{x})
	=
	\sum_{\lambda=1,2}
	\int_{\mathbb{R}^3}\frac{\dd[3]{p}}{(2\pi)^3\sqrt{2\omega(\vb{p})}}
	\left\{
		a_\lambda(\vb{p})
		e^{i\omega(\vb{p})t-i\vb{p}\vdot\vb{x}}
		\vu{e}_\lambda(\vb{p})
		+
		\text{c.c.}
	\right\}
\end{equation}
where we defined
\begin{equation}
	a_\lambda(\vb{p})
	=
	A_\lambda\left(\omega(\vb{p}),\vb{p}\right)
\end{equation}
and used the conjugate symmetry of the Fourier amplitudes $a_\lambda(-\vb{p})=a_\lambda(\vb{p})^*$.

\subsection{Canonical quantization}
		\section{Quantum states, operators and expectation values}

\subsection{Vacuum state}

\subsection{Positive and negative frequency operators}

% TODO: cite operator-valued distributions, smeared fields

The positive and negative frequency operators of the Maxwell field
\begin{align}
	\hat{\vb{A}}^{(-)}
	&=
	\sum_{\lambda=1,2}
	\int_{\mathbb{R}^3}
	\frac{\dd[3]{p}}{(2\pi)^3\sqrt{2\omega(\vb{p})}}
	\hat{a}_\lambda(\vb{p})
	\vu{e}_\lambda(\vb{p})
	\eval{e^{-ip_\mu x^\mu}}_{p_0=\omega(\vb{p})}
	\\
	\hat{\vb{A}}^{(+)}
	&=
	\sum_{\lambda=1,2}
	\int_{\mathbb{R}^3}
	\frac{\dd[3]{p}}{(2\pi)^3\sqrt{2\omega(\vb{p})}}
	\hat{a}_\lambda(\vb{p})^\dagger
	\vu{e}_\lambda(\vb{p})^*
	\eval{e^{+ip_\mu x^\mu}}_{p_0=\omega(\vb{p})}
\end{align}
are operator-valued distributions.

\subsection{Number state}

\subsection{Displacement operator}

\begin{equation}
	\hat{D}[\alpha]
	=
	\exp\left\{
		\int\frac{\dd[3]{p}}{(2\pi)^3\sqrt{2\omega(\vb{p})}}
		\left\{
			\alpha(\vb{p})
			\hat{a}^\dagger(\vb{p})
			-
			\alpha(\vb{p})^*
			\hat{a}(\vb{p})
		\right\}
	\right\}
\end{equation}

\subsection{Coherent state}

\begin{equation}
	\begin{split}
		\ket{\alpha}
		&=
		\exp\left\{
			-
			\frac{1}{2}
			\int\frac{\dd[3]{p}}{(2\pi)^32\omega(\vb{p})}
			\abs{\alpha(\vb{p})}^2
		\right\}
		\\
		&\times
		\exp\left\{
			\int\frac{\dd[3]{p}}{(2\pi)^3\sqrt{2\omega(\vb{p})}}
			\alpha(\vb{p})
			\hat{a}(\vb{p})^\dagger
		\right\}
		\ket{0}
	\end{split}
\end{equation}

\subsection{Quadrature operator}

We define the generalized quadrature operator by 
\begin{equation}
	\hat{X}(\theta)
	=
	\int_{\mathbb{R}^3}
	\frac{\dd[3]{p}}{(2\pi)^3}
	\frac{1}{\sqrt{2}}
	\left\{
		\hat{a}(\vb{p})
		e^{-i\theta}
		+
		\hat{a}^\dagger(\vb{p})
		e^{+i\theta}
	\right\}
\end{equation}
where the prefactor ensures that the commutator takes the standard form
\begin{equation}
	\comm{\hat{X}(\theta)}{\hat{X}(\theta+\Delta\theta)}
	=
	\frac{i}{2}
	\sin(\Delta\theta)
	V_p
\end{equation}
and $V_p$ is the momentum space volume
\begin{equation}
	V_p
	=
	\int_{\mathbb{R}^3}\frac{\dd[3]{p}}{(2\pi)^3}
	=
	\frac{4\pi}{(2\pi)^3}
	\int_0^\Lambda\dd{p}p^2
	=
	\frac{\Lambda^3}{6\pi^2}
\end{equation}
where we introduced the cut-off momentum $\Lambda$ for regularization.

The Robertson uncertainty relation yields a lower bound for the product of the variances
\begin{equation}
	\expval{\left(\Delta\hat{X}(\theta)\right)^2}
	\expval{\left(\Delta\hat{X}(\theta+\Delta\theta)\right)^2}
	\geq
	\frac{1}{4}
	\sin(\Delta\theta)^2
	V_p^2
	.
\end{equation}
The uncertainty is maximal for $\Delta\theta=\pi/2$.
The coherent state is a minimal uncertainty state in the sense that
\begin{align}
	\expval{\hat{X}(\theta)}{\alpha}
	=
	\sqrt{2}
	\int_{\mathbb{R}^3}
	\frac{\dd[3]{p}}{(2\pi)^3\sqrt{2\omega(\vb{p})}}
	\Re\left\{
		\alpha(\vb{p})
		e^{-i\theta}
	\right\}
	&&
	\expval{\left(\Delta\hat{X}(\theta)\right)^2}{\alpha}
	=
	\frac{1}{2}
	V_p
\end{align}

\subsection{Electromagnetic field operator}
		\section{Time-dependent interactions}

\subsection{Time-evolution operator}

Let $\ket{\psi(t_0)}$ be a state at time $t_0$, then the time-evolution relates the state $\ket{\psi(t)}$ at some later time $t>t_0$ to $\ket{\psi(t_0)}$ via
\begin{equation}
	\ket{\psi(t)}
	=
	\hat{U}(t,t_0)
	\ket{\psi(t_0)}
	.
\end{equation}
Inserting $\ket{\psi(t)}$ into the Schrödinger equation leads to
\begin{equation}
	i\dv{t}
	\hat{U}(t,t_0)
	=
	\hat{H}(t)
	\hat{U}(t,t_0)
\end{equation}
which formal solution is the time-ordered exponential, see Ref.~\cite[p.~380]{Bartelmann2018},
\begin{equation}
	\hat{U}(t,t_0)
	=
	T\exp\left\{
		-i
		\int_{t_0}^t\dd{t^\prime}
		\hat{H}(t^\prime)
	\right\}
\end{equation}
where $T$ denotes the time-ordering symbol.
Only for simple time-dependent systems an exact time-evolution operator exists.
In contrast to the Dyson expansion, the Magnus expansion yields a unitary time-evolution operator even for finite order, in particular,
\begin{equation}
	\hat{U}(t,t_0)
	=
	\exp\left\{
		\sum_{n=1}
		\hat{\Omega}^{(n)}(t,t_0)
	\right\}
\end{equation}
where the first two expansion terms are given by
\begin{align}
	\hat{\Omega}^{(1)}(t,t_0)
	&=
	\frac{(-i)}{1!}
	\int_{t_0}^t\dd{t^\prime}
	\hat{H}(t^\prime)
	\\
	\hat{\Omega}^{(2)}(t,t_0)
	&=
	\frac{(-i)^2}{2!}
	\int_{t_0}^t\dd{t^\prime}
	\int_{t_0}^{t^\prime}\dd{t^{\prime\prime}}
	\comm{\hat{H}(t^\prime)}{\hat{H}(t^\prime)}
\end{align}
and represent time-ordering corrections, see Ref.~\cite{QuesadaMejia2015}.

\subsection{Interaction with classical current}

The Schrödinger-picture Hamiltonian describing the interaction of the Maxwell field $\hat{\vb{A}}$ with a classical current $\vb{j}$ is
\begin{equation}
	\hat{H}_\text{int}(t)
	=
	-
	\vb{j}(t,\vb{x})
	\vdot
	\hat{\vb{A}}(t,\vb{x})
	.
\end{equation}
Inserting the mode expansion
		\section{Connection to quantum optics}

Here we reduce the results of the previous sections to a continuous-mode description of (linear polarized) light, i.e., frequency range limited to the optical domain.
Assumptions and limitations:
\begin{enumerate}
	\item Fixed reference frame (at rest), i.e., no Lorentz boosts allowed.
	\item Light is linearly polarized.
	\item \textcolor{red}{No distribution of the momentum}
\end{enumerate}

\begin{equation}
	\hat{N}
	=
	\int\dd{\omega}
	\hat{a}^\dagger(\omega)
	\hat{a}(\omega)
\end{equation}

\begin{equation}
	\hat{E}
	=
	i
	\int\dd{\omega}
	\omega
	\hat{a}^\dagger(\omega)
	\hat{a}(\omega)
\end{equation}

\begin{equation}
	\ket{f}
	=
	\int\dd{\omega}
	f(\omega)
	\hat{a}(\omega)
\end{equation}

\begin{equation}
	\ket{\alpha}
	=
	\exp\left\{
		-
		\frac{1}{2}
		\int\dd{\omega}
		\abs{\alpha(\omega)}^2
	\right\}
	\exp\left\{
		-
		\int\dd{\omega}
		\alpha(\omega)
		\hat{a}^\dagger(\omega)
	\right\}
	\ket{0}
\end{equation}
		
		\addcontentsline{toc}{section}{References}
		\printbibliography[title=References]
	\end{refsection}
	
	\chapter{Quantum optical communication}
	\begin{refsection}
		\section{Coherent source}

In the first chapter, we showed that a classical source $\vb{J}(t,\vb{x})$ generates a coherent state via the $\hat{S}$ operator from the vacuum
\begin{equation}
	\begin{split}
		\ket{\alpha(\vb{p})}
		&=
		\exp\left\{
			-i\int\dd[4]{x}
			\vb{J}(t,\vb{x})
			\vdot
			\vb{A}(t,\vb{x})
		\right\}
		\ket{0}
		\\
		&=
		\exp\left\{
			-\frac{1}{2}
			\sum_{\lambda=1,2}
			\int\dd[3]{p}
			\boldsymbol{\varepsilon}_\lambda(\vb{p})
			\vdot
			\vb{J}(\vb{p})
		\right\}
		\\
		&\times
		\exp\left\{
			\sum_{\lambda=1,2}
			\int\dd[3]{p}
			\boldsymbol{\varepsilon}_\lambda(\vb{p})
			\vdot
			\vb{J}(\vb{p})
			\hat{a}_\lambda^\dagger(\vb{p})
		\right\}
		\ket{0}
	\end{split}
\end{equation}
and we are left to find a mechanism for $\vb{J}(\vb{p})$.

\subsection{Oscillating electron}

\subsection{Hydrogen atom}

\subsection{Harmonic oscillators}
		\section{Waveguides}

So far, we discussed the free-field in which the states disperse quickly as seen in the first chapter.
Therefore, most optical communication is conducted using waveguide.

We can amend the developed formalism by replacing the linear dispersion relation of the Maxwell field by a dispersion relation matching the waveguides transmission properties.

\subsection{Fiber}

Optical fibers are dielectric cylindrical waveguides.

\subsection{Loss}

Loss needs to be modeled by the exchange with an environment
		\section{Mode coupler}

\subsection{Modulator}

\subsubsection{Sampling}

		\section{Modulator}
		\section{Coherent detector}

		
		\addcontentsline{toc}{section}{References}
		\printbibliography[title=References]	
	\end{refsection}

	\chapter{Example: CV-QKD}
	\begin{refsection}
		\section{Idea and operating principle}
		\section{Protocol}
		\subsection{Recociliation}
		\subsection{Classical post-processing}
		\section{Coherent state transmitter}
		\section{Coherent state receiver}
		\section{Squeezed state attack}
		\section{Protocol extension}
		
		\addcontentsline{toc}{section}{References}
		\printbibliography[title=References]
	\end{refsection}

	\chapter{Conclusion and outlook}
	\begin{refsection}	
		\addcontentsline{toc}{section}{References}
		\printbibliography[title=References]
	\end{refsection}

	\appendix

	\chapter{Harmonic oscillator}
	\begin{refsection}
		\section{Harmonic oscillator}

\blockcquote{Sidney Coleman}{The career of a young theoretical physicist consists of treating the harmonic oscillator in ever-increasing levels of abstraction.}

\subsection{Examples and quadratic approximation}

To illustrate Coleman's quote, let us consider two at first glance very different physical systems showing the same dynamics.
The left-hand side of \Cref{fig:ho} depicts a mechanical oscillator comprising a spring and a mass $m$ at different times.
The right-hand side of \Cref{fig:ho} depicts an electrical oscillator comprising an inductor with inductance $L$, a capacitor with capacitance $C$, and a current $i$.
\begin{figure}[htb]
    \centering
    \subfloat[\centering Mechanical oscillator]{\includestandalone[mode=buildnew]{figures/tikz/oscillator-mechanical}}
    \qquad
    \subfloat[\centering Electrical oscillator]{\includestandalone[mode=buildnew]{figures/tikz/oscillator-electrical}}
    \caption{Two embodiments of a oscillator based on classical mechanics (a) and electronic circuit theory (b).}\label{fig:ho}
\end{figure}
If we assume the spring of the mechanical oscillator to exert a force proportional to $q(t)$ on the point mass $m$\footnote{For small deviations from the equilibrium, $\abs{q(t)}\ll1$, experiments justify a linear force for the spring.}, Newtonian mechanics relate the spring's force to the particle's acceleration
\begin{equation}
    m\ddot{q}
    =
    F
    =
    -kq
    \label{eq:ho_newton}
\end{equation}
wherein $k$ is the proportionality factor of the spring's force, known as the spring constant.
From circuit theory, we know that the effect of an inductance $L$ and capacitance $C$ on their respective current and voltage
\begin{align}
    V_L
    =
    L\dv{I_L}{t}
    ,&&
    I_C
    =
    C\dv{V_C}{t}
    .
\end{align}
Kirchhoff's current law tells us that the same current $I=I_L=I_C$ flows through the inductor and the capacitor while Kirchhoff's voltage law that the sum of the voltage drops $V_L+V_C$ is equal to zero, hence
\begin{equation}
    \dv[2]{I}{t}
    =
    \frac{1}{L}\dv{V_C}{t}
    =
    -\frac{1}{L}\dv{V_L}{t}
    =
    -\frac{1}{LC}I
    \label{eq:ho_current}.
\end{equation}
On first sight, \cref{eq:ho_newton} and \cref{eq:ho_current} appear quite different.
If we define $\omega_0=k/m$ for the mechanical and $\omega_0=1/\sqrt{LC}$ for the electrical oscillator, and adjust our notation, \cref{eq:ho_newton} and \cref{eq:ho_current} take the form
\begin{align}
    \ddot{q}
    =
    -\omega_0^2q
    ,&&
    \ddot{I}
    =
    -\omega_0^2I
    \label{eq:ho_me}
\end{align}
which is remarkable as the fundamental physics behind the electrical and mechanical oscillator are very different.
Integration of the harmonic oscillator's equations of motion, \cref{eq:ho_me}, yields
\begin{equation}
    E
    =
    \frac{1}{2}m\dot{q}^2
    +
    \frac{1}{2}m\omega_0^2q^2
    \label{eq:ho_energy}
\end{equation}
where the integration constant $E$ has the interpretation of being the energy available to the system.

Most physical problems reduce to the harmonic oscillator after performing a series expansion up of the potential in a dynamical variable up to quadratic (or harmonic) order.\footnote{The harmonic oscillator is the most simple non-trivial system for which there exists a closed solution.}
To clarify, let us consider the energy of a physical system with potential $V(q)$
\begin{equation}
    E
    =
    \frac{1}{2}m^2\dot{q}^2
    +
    V(q)
    \label{eq:pp_energy}.
\end{equation}
We perform a Taylor series expansion of the potential around some configuration $q_0$ minimizing the potential energy
\begin{equation}
    V(q)
    =
    \sum_{n=1}\frac{1}{n!}\eval{\dv[n]{V}{q}}_{q_0}(q-q_0)^n
    =
    \frac{1}{2}\eval{\dv[2]{V}{q}}_{q_0}(q-q_0)^2
    +
    \mathcal{O}(q^3)
    \label{eq:pp_potential}
\end{equation}
and recover the potential energy of the harmonic oscillator given in \cref{eq:ho_energy}.
In physics we mainly deal with continuously differentiable functions\footnote{Discontinuities in motion are unphysical in the sense that they require infinite energy. An exception being macroscopic or stochastic dynamics, e.g., Brownian motion.} for which a Taylor series exists.
If we find an equilibrium configuration of our physical system, a finite expansion around such point yields in most cases a very good approximation of the problem, and the harmonic solution to the quadratic term describes the dominating dynamics.

\subsection{Fourier analysis of the equation of motion}

A closed solution to the differential equation encoding the harmonic dynamics
\begin{equation}
    \ddot{q}
    =
    -\omega_0^2q
    \label{eq:ho_eom}
\end{equation}
can be found using Fourier analysis.
Physical systems have finite energy, \Cref{eq:ho_energy}, the dynamical variable $q(t)$ is square-integrable\footnote{Let $q^*$ be a global maximum, then we have $E=\frac{k}{2}{q^*}^2$, and thus $\int_0^T\dd{t}q(t)^2<\int_0^T2E/k=2ET/k<\infty$, i.e., $q(t)$ is square-integrable on $[0,T]$.} and the (inverse) Fourier transform
\begin{equation}
    q(t)
    =
    \int_0^T\dd{\omega}e^{i\omega t}q(\omega)
    \label{eq:ho_ft}
\end{equation}
exists.
Inserting \cref{eq:ho_ft} into \cref{eq:ho_eom} reduces the differential equation, \cref{eq:ho_eom}, to the algebraic equation
\begin{equation}
    -\omega^2q(\omega)
    =-\omega_0^2q(\omega)
    \label{eq:ho_eom_ft}
\end{equation}
which is solved by
\begin{equation}
    q(\omega)
    =
    c_1\delta(\omega-\omega_0)
    +
    c_2\delta(\omega+\omega_0)
    \label{eq:ho_eom_ft_sol}.
\end{equation}
\Cref{eq:ho_eom_ft_sol} corresponds to a spectrum of two fundamental frequencies $\pm\omega_0$ with complex weights $c_1,c_2\in\mathbb{C}$.
The requirement that $q(t)$ is real constraints the spectrum
\begin{equation}
    q^*(\omega)
    =
    q(-\omega)
\end{equation}
and the spectral weights of \cref{eq:ho_eom_ft_sol} must fulfill $c_1=c_2^*$.
We introduce $A,\varphi\in\mathbb{R}$ and rewrite the spectrum, \cref{eq:ho_eom_ft_sol}, to be real
\begin{equation}
    q(\omega)
    =
    \frac{A}{2}
    \left\{
    \delta(\omega-\omega_0)e^{+i\varphi}
    +
    \delta(\omega+\omega_0)e^{-i\varphi}
    \right\}
    \label{eq:ho_eom_ft_sol_fin}.
\end{equation}
By inserting \cref{eq:ho_eom_ft_sol_fin} into \cref{eq:ho_ft}, we find our way back to the time domain
\begin{equation}
    q(t)
    =
    \frac{A}{2}
    \int\dd{\omega}e^{i\omega t}q(\omega)
    =
    A\cos(\omega_0t+\varphi)
    \label{eq:ho_eom_sol}.
\end{equation}
Two initial conditions, for instance, $q(0),q(T)$, fix the free parameters $A$ and $\varphi$, and \cref{eq:ho_eom_sol} describes the displacement for all subsequent times.

\subsection{Lagrange and Hamilton formalism}

So far, we have used Newtonian mechanics to find the equation of motion of the mechanical oscillator.
Now, we discuss the harmonic oscillator in the language of the Lagrangian and Hamiltonian formalism which turn out to be essential for subsequent steps.

The Lagrangian of the harmonic oscillator is defined as the difference between the kinetic and the potential energy
\begin{equation}
    L
    =
    \frac{1}{2}m\dot{q}^2
    -
    \frac{1}{2}m\omega_0^2q^2
    \label{eq:ho_lagrangian}.
\end{equation}
In contrast to the Newtonian approach, where we need to account for every relevant force, the Lagrangian approach only requires knowledge of the kinetic and potential energies, which are more intuitive to suspect.\footnote{After choosing a convenient set of coordinates, the Lagrangian approach also reproduces inertial forces.}
Given the Lagrange function, finding the equation of motion reduces to calculating the derivatives in the Euler-Lagrange equation
\begin{equation}
    0
    =
    \dv{t}\pdv{L}{\dot{q}}
    -
    \pdv{L}{q}
    =
    m\ddot{q}+m\omega_0^2q
    \label{eq:ho_euler_lagrange}
\end{equation}
reproducing the equation of motion found by Newton's force force law and electrical circuit analysis \cref{eq:ho_me}.

In the Hamilton approach to mechanics, we replace the time-derivative of the dynamical variable $\dot{q}$ by a new dynamical variable, the canonical momentum
\begin{equation}
    p
    =
    \pdv{L}{\dot q}
    \label{eq:canonical_momentum}.
\end{equation}
Using the Legendre transformation
\begin{equation}
    H(p,q)
    =
    p\dot{q}(p,q)
    -
    L(p,q)
    \label{eq:legendre_transform},
\end{equation}
we can cast the Lagrange function $L$ to the Hamilton function $H$.
The Hamilton function is a function of the canonical momentum $p$ and position $q$ known as conjugate variables.
The dynamics of the conjugate variables are determined by the Hamilton equations of motion
\begin{align}
    \dot{q}
    =
    +\pdv{H}{p}
    =
    +\left\{p,H\right\}
    ,&&
    \dot{p}
    =
    -\pdv{H}{q}
    =
    -\left\{q,H\right\}
    \label{eq:hamilton_eom}
\end{align}
where we introduced the Poisson bracket operator $\left\{\cdot,\cdot\right\}$.
The action of the Poisson bracket operator acting on two functions $f(p,q,t),g(p,q,t)$ is defined by
\begin{equation}
    \left\{f,g\right\}
    =
    \pdv{f}{q}\pdv{g}{p}
    -
    \pdv{f}{p}\pdv{g}{q}
    \label{eq:poisson_bracket}.
\end{equation}
The Poisson bracket operator allows us to replace differential by algebraic equations.
For example, we can rewrite the time evolution of a function $f(p,q)$ as
\begin{equation}
    \dv{f}{t}
    =
    \pdv{f}{q}\dv{q}{t}
    +
    \pdv{f}{p}\dv{p}{t}
    =
    \pdv{f}{q}\pdv{H}{p}
    -
    \pdv{f}{p}\pdv{H}{q}
    =
    \left\{f,H\right\}
    \label{eq:hamilton_time_evol}.
\end{equation}
Introducing the Poisson bracket operator may appear artificial at first but it turns out that replacing the Poisson brackets with commutators leads us to quantum mechanics.

For the harmonic oscillator, the canonical momentum is equal to the physical momentum $p=m\dot{q}$\footnote{Generally, the canonical momentum is not equal to the physical momentum, e.g., the canonical momentum of a particle with electrical charge $e$ in an electric field given by the vector potential $A$ is $p-eA$.} and the Hamilton function
\begin{equation}
    H
    =
    \frac{p^2}{2m}
    +
    \frac{m\omega_0^2}{2}q^2
    \label{eq:ho_hamilton}
\end{equation}
is equal to the energy, \cref{eq:ho_energy}, rewritten in terms of the physical momentum $p=m\dot{q}$.
The Hamilton equations of motion, \cref{eq:hamilton_eom} for the harmonic oscillator yield
\begin{align}
    \dot{q}
    =
    \frac{p}{m},
    &&
    \dot{p}
    =
    -m\omega_0^2q
    \label{eq:ho_hamilton_eom}.
\end{align}
With the time derivative of the position variable $\ddot{q}=\dot{p}/m=\omega_0^2q$ we recover the equation of motion we derived with Newton, \cref{eq:ho_newton}, and the Lagrange formalism, \cref{eq:ho_eom}.

\subsection{Canonical quantization and dynamical pictures}

In the canonical quantization prescription, we promote the conjugate variables $q(t),p(t)$ to linear operators $\hat{q}(t),\hat{p}(t)$\footnote{In contrast to variables which can be thought of as real functions, linear operators map between vector spaces.} satisfying the commutation relation
\begin{equation}
    \comm{\hat{q}}{\hat{p}}
    =
    \hat{q}\hat{p}
    -
    \hat{p}\hat{q}
    =
    i
    \label{eq:comm_pm},
\end{equation}
and thereby implementing the measurement uncertainty in quantum mechanics.\footnote{In quantum mechanics, a measurement yields an eigenvalue of an (self-adjoint) operator. If operators do not commute, they cannot be simultaneously diagonalized, thus, there exists no shared eigenbasis, and the measurement order is not independent.}
Replacing the Poisson brackets in \cref{eq:hamilton_time_evol} with commutators yields us the time-evolution of an operator $\hat{A}(\hat{q},\hat{p})$ in the Heisenberg picture
\begin{equation}
    \dv{\hat{A}}{t}
    =
    \comm{\hat{H}}{\hat{A}}
    \label{eq:heisenberg_time_evol}
\end{equation}
which is solved by unitary transformation with the time evolution operator $\hat{U}$
\begin{equation}
    \hat{A}(t)
    =
    \hat{U}^\dagger(t)
    \hat{A}(0)
    \hat{U}(t)
    =
    e^{+i\hat{H}t}\hat{A}(0)e^{-i\hat{H}t}
    \label{eq:heisenberg_time_evol_sol}.
\end{equation}
In the Heisenberg picture, operators are time-dependent but states are not.
In the Schrödinger picture, the time evolution operator $\hat{U}$ is part of the state
\begin{equation}
    \ket{\psi(t)}
    =
    \hat{U}(t)\ket{\psi(0)}
    =
    e^{-i\hat{H}t}\ket{\psi(0)}
    \label{eq:schroedinger_state}.
\end{equation}
and the operator itself is time-independent.\footnote{Heisenberg and Schrödinger picture predict the same expectation values $\expval{\hat{A}(t)}{\psi(0)}=\expval{\hat{U}^\dagger(t)\hat{A}(0)\hat{U}}{\psi(0)}=\expval{\hat{A}(0)}{\psi(t)}$.}
Taking the derivative w.r.t. time of \cref{eq:schroedinger_state} yields the Schrödinger equation
\begin{equation}
    \pdv{t}\ket{\psi(t)}
    =
    \hat{H}\ket{\psi(t)}
    \label{eq:schroedinger_time_evol}.
\end{equation}
Solving field dynamics in the Schrödinger picture requires infinite-dimensional differential operators, which are mathematical problematic compared to the Heisenberg picture's algebraic relations~\cite[p.~19]{Fulling1989}.\footnote{Ref.~\cite[p.~19]{Fulling1989} gives an in-depth comparison of both pictures.}
However, it is possible to solve the field dynamics in the Heisenberg picture and transform it into the Schrödinger picture to gain additional insights.

\subsection{Hermite polynomials as solution to the Schrödinger equation}

We perform the canonical quantization procedure and obtain the quantum Hamiltonian
\begin{equation}
    \hat{H}
    =
    \frac{\hat{p}^2}{2m}
    +
    \frac{m\omega_0^2}{2}\hat{q}^2
    \label{eq:qho_hamilton}.
\end{equation}
The conservation of energy suggests the existence of an energy eigenbasis
\begin{equation}
    \hat{H}\ket{n}
    =
    E_n\ket{n}
    \label{eq:qho_eigen_energy}.
\end{equation}
Using some length mathematical arguments and the canonical commutation relation, \cref{eq:comm_pm}, it is possible to derive~\cite[p.~27]{Mukhanov2007}
\begin{equation}
    \hat{p}\ket{q}
    =
    -i\pdv{q}\ket{q}
    \label{eq:qho_eigen_mp}.
\end{equation}
Taking the self-adjoint of \cref{eq:qho_eigen_energy}, then applying $\ket{q}$ to the right, and using \cref{eq:qho_eigen_mp}, we can find the solutions to the one-dimensional quantum harmonic oscillator to solving the differential equation
\begin{equation}
    E_n\braket{n}{q}
    =
    \bra{n}\hat{H}\ket{q}
    =
    \left(-\frac{1}{2m}\pdv[2]{q}+\frac{m\omega_0^2}{2}q^2\right)\braket{n}{q}
    \label{eq:qho_pdv}.
\end{equation}
The solution to the differential equation is~\cite[p.~51]{Griffiths2017}
\begin{equation}
    \psi_n(q)
    =
    \braket{n}{q}
    =
    \left(\frac{m\omega_0}{\pi}\right)^{1/4}
    \frac{1}{\sqrt{2^nn!}}
    H_n\left(\sqrt{m\omega_0}q\right)
    \exp\left\{-\frac{m\omega_0}{2}q^2\right\}
    \label{eq:qho_pdv_sol}
\end{equation}
wherein $H_n$ denotes the $n$th Hermite polynomial.
It is common practice to read \cref{eq:qho_pdv_sol} as the position representation of the wave function corresponding to the $n$th energy level.
Momentum and position representation of the wave function relate by the Fourier transform\footnote{Instead of using a position state to derive \cref{eq:qho_pdv}, we could have used a momentum state by using the momentum state equivalent of \cref{eq:qho_eigen_mp} $q\ket{p}=i\pdv{q}\ket{q}$.}
\begin{equation}
    \psi_n(p)
    =
    \bra{n}\left(\int\dd{q}\ketbra{q}{q}\right)\ket{p}
    =
    \int\dd{q}\braket{q}{p}\braket{n}{q}
    =
    \int\frac{\dd{q}}{\sqrt{2\pi}}\psi_n(q)e^{ipq}
    \label{eq:qho_pdv_sol_mom}
\end{equation}
where $\braket{q}{p}=e^{ipq}/\sqrt{2\pi}$ is the solution to the differential equation one obtains after applying $\bra{p}$ to \cref{eq:qho_eigen_mp} and choosing a normalization compatible with the symmetric Fourier transform.
The time evolution is encoded in the time evolution operator, for instance, in the Schrödinger picture, \cref{eq:schroedinger_time_evol},
\begin{equation}
    \psi_n(q,t)
    =
    \psi_n(q)
    e^{-iE_nt}
    .
\end{equation}

\subsection{Energy eigenstates, creation and annihilation operators}

While \cref{eq:qho_pdv_sol} represents the probability amplitude dynamics in terms of position (and momentum) variables, the energy spectrum remains unknown.
If we could factor the quantum harmonic oscillator's Hamiltonian as a product of operators, finding the energy spectrum reduces to solving an eigenvalue equation.
It turns out that
\begin{align}
    \hat{x}
    =
    \sqrt{\frac{1}{2m\omega_0}}
    \left(\hat{a}^\dagger+\hat{a}\right)
    &&
    \hat{p}
    =
    i\sqrt{\frac{m\omega_0}{2}}
    \left(\hat{a}^\dagger-\hat{a}\right)
    \label{eq:qho_pm_ac}
\end{align}
respective
\begin{align}
    \hat{a}
    =
    \frac{1}{\sqrt{2m\omega_0}}
    \left(m\omega_0\hat{x}+i\hat{p}\right)
    &&
    \hat{a}^\dagger
    =
    \frac{1}{\sqrt{2m\omega_0}}
    \left(m\omega_0\hat{x}-i\hat{p}\right)
    \label{eq:qho_ac_pm}
\end{align}
is such a factorization known as the annihilation and creation operator.
Before we cast the Hamiltonian in terms of these new operators, we need to find the commutator between them
\begin{equation}
    \comm{\hat{a}}{\hat{a}^\dagger}
    =
    \hat{a}\hat{a}^\dagger
    -
    \hat{a}^\dagger\hat{a}
    =
    1
    \label{eq:qho_comm_ac}
\end{equation}
by inserting \cref{eq:qho_ac_pm} and using the commutator of the position and momentum operator, \cref{eq:comm_pm}.
Inserting \cref{eq:qho_pm_ac} into \cref{eq:qho_hamilton} yields
\begin{equation}
    \hat{H}
    =
    \omega_0\left(\hat{a}^\dagger\hat{a}+\frac{1}{2}\right)
    \label{eq:qho_hamilton_ac}
\end{equation}
after using the commutator, \cref{eq:qho_comm_ac}.
The eigenspectrum of $\hat{a}^\dagger\hat{a}$ is equal to the natural numbers including zero $\mathbb{N}_0$~\cite[p.~506]{Cohen2019} which suggests to name
\begin{equation}
    \hat{N}
    =
    \hat{a}^\dagger\hat{a}
    \label{eq:qho_number_operator}
\end{equation}
the number operator and the number state $\ket{n}$ an eigenstate thereof.
The number operator commutes with the Hamilton operator, thus, the number state is also an energy eigenstate
\begin{equation}
    E_n\ket{n}
    =
    \hat{H}\ket{n}
    =
    \omega_0\left(n+\frac{1}{2}\right)
    \label{eq:qho_number_energy}
\end{equation}
which answers the question of the energy spectrum of the quantum harmonic oscillator.

However, there is more about the annihilation and creation operator than finding the energy spectrum, in particular, we have not yet explained why they are called annihilation and creation operator.
Let us consider the eigenvalue of the number operator $\hat{N}$ with respect to $\hat{a}^\dagger\ket{n}$
\begin{equation}
    \hat{N}\hat{a}^\dagger\ket{n}
    =
    \left(\comm{\hat{N}}{\hat{a}^\dagger}+\hat{a}^\dagger\hat{N}\right)\ket{n}
    =
    (n+1)\hat{a}^\dagger\ket{n}
    ,
\end{equation}
i.e., $\hat{a}^\dagger\ket{n}\propto\ket{n+1}$.
We fix the proportionality constant by requiring $n=\expval{\hat{a}^\dagger\hat{a}}{n}$ and summarize
\begin{equation}
    \hat{a}^\dagger\ket{n}
    =
    \sqrt{n+1}\ket{n}
    \label{eq:qho_creation}.
\end{equation}
We conclude that $\hat{a}^\dagger$ is the creation operator because it creates an additional excitation when acting on a number state $\ket{n}$.
In the same sense, we find that
\begin{equation}
    \hat{a}\ket{n}
    =
    \sqrt{n}\ket{n-1}
    \label{eq:qho_annhiliation}
\end{equation}
explains why $\hat{a}$ is named the annihilation operator.
Consistency of the number states being natural numbers requires the annihilation operator to destroy the ground (or vacuum) state
\begin{equation}
    \hat{a}\ket{0}
    =
    0
    \label{eq:qho_vacuum_annihilation}.
\end{equation}
Applying the creation operator iteratively, lets us construct a number state from the vacuum state via
\begin{equation}
    \frac{(\hat{a}^\dagger)^n}{\sqrt{n!}}\ket{0}
    =
    \ket{n}
    \label{eq:qho_creation_number}.
\end{equation}
Annihilation and creation operators form an algebra that generalizes to highly advanced Hilbert spaces and will be used in the subsequent sections.
		
		\addcontentsline{toc}{section}{References}
		\printbibliography[title=References]
	\end{refsection}

	\chapter{Classical beam splitter}
	\begin{refsection}
		Optical combiners and splitters, including the beam splitter and fiber coupler, are essential for many sophisticated experiments and devices.
For example, in quantum information theory, the beam splitter is an experimental realization of a quantum NOT gate~\cite{Adami1998}.
While an optical combiner superimposes (mixes) two optical signals, an optical splitter divides an input's power among two outputs.
From a physical perspective, the distinction between optical combiners and splitters becomes obsolete: the optical combiner and splitter are special cases of a linear passive optical component with two in- and outputs where either one in- or output is blocked.
More interestingly, a quantum mechanical description strictly requires two in- and outputs where one of the in- or outputs is assumed to be in the vacuum state.

A plethora of approaches towards the (quantum) beam splitter exists~\cite{Leonhardt2010,Gerry2005,Loudon2000}, ending with a unitary transform relating the electrical output to the input fields.\footnote{The optical beams are electromagnetic waves, yet given the electric field component, the magnetic field component follows from Maxwell's equations.}
We find that most of these descriptions are based on rather strong, not well-stated assumptions, leading to misconception and confusion, for example, regarding the phases and symmetry of the beam splitter.
In addition, we miss the link to imperfect implementations and the generalization to fiber components.

The chapter starts with a detailed review of the classical free-ray beam splitter, followed by a short review of the fiber coupler. After that, we present a generic model of a quantum mode coupler and provide extensions of the beam splitter and fiber coupler to quantum effects.

\section{Classical free-ray beam splitter}\label{sec:beam_splitter}


We start the review of the classical beam splitter with typical embodiments as a free-ray component.
We will find that there are only a few fundamental constraints on a generic beam splitter transformation.
Nevertheless, given the materials and dimensions, it is possible to model the transformation properties correctly.
Being sensitized for the difficulties of finding a general transformation for the beam splitter, we move on to an in-depth discussion of the ideal beam splitter described by a unitary transformation.
Finally, we end with a presentation of the scattering parameters of a generic but real beam splitter.

\subsection{Cubic, plate, and pellicle beam splitters}

The most commonly employed (free-ray) designs of the beam splitter are the cubic, plate, and pellicle beam splitters, see \Cref{fig:beam_splitter_types}.
\begin{figure}[htb]
    \centering
    \includestandalone[mode=buildnew]{figures/tikz/beam-splitter-types}
    \caption{Cubic (a), plate (b), and pellicle (c) beam splitter.}\label{fig:beam_splitter_types}
\end{figure}
The cubic beam splitter is made of two triangular prisms. The interface between the two prisms is finished with a dielectric coating.
The outward-facing surface of the prisms is grafted with an \gls{ar} coating.\footnote{The incident angle of the electric field is perpendicular to the surface of the cubic beam splitter. As the reflection angle is equal to the incidence angle, we have back-reflection of the input fields. Back-reflection is discussed in \cref{sec:beam_splitter_scattering_parameters}.}
Compared to the plate and pellicle beam splitter, the cubic beam splitter has the highest loss and aberrations.
Nonetheless, the cubic beam splitter is a popular choice because of its simple integration into an optical setup.\footnote{The cubic beam splitter is significantly less sensitive to angular misalignment and mechanical vibrations.}
The pellicle beam splitter consists of a few micrometer thin membrane, optionally with a one-sided coating.
The small thickness of the membrane makes the pellicle beam splitter the most performant in terms of absorption and aberrations among the three designs.
Furthermore, the pellicle embodiment of the beam splitter supports a broad wavelength range.
The drawback of using a pellicle beam splitter is that they are highly susceptible to acoustic noise and require a precise angle to the incident beam.
The plate beam splitter is like a thick pellicle beam splitter made of glass.
Compared to the cubic and pellicle beam splitter's performance, the plate beam splitter is in-between concerning absorption, aberrations, and wavelength range.

Given two input beams, represented by the electrical field components $E_1(t)$ and $E_2(t)$, a beam splitter gives rise to two output beams, represented by the electrical field components $E_1^\prime(t)$ and $E_2^\prime(t)$.
\Cref{fig:beam_splitter_cube_plate} illustrates how the input and output fields emerge in the cubic and plate beam splitter.
The two input fields $E_1(t),E_2(t)$ are incident to the cubic and plate beam splitter's left and top, while the two output electric fields $E_1^\prime(t),E_2^\prime(t)$ exit the beam splitter at the right and bottom.
\begin{figure}[htb]
    \centering
    \includestandalone[mode=buildnew]{figures/tikz/beam-splitter-cubic-plate}
    \caption{Cubic (left) and plate beam splitter (right) with two input beams represented by the electrical field components $E_1(t),E_2(t)$ and two output beam represented by the electrical field components $E_1^\prime(t),E_2^\prime(t)$. The frequency-dependent complex coefficients $r(\omega),r(\omega)^\prime$ and $t(\omega),t(\omega)^\prime$ fully characterize the reflection and transmission properties of the beam splitter.}\label{fig:beam_splitter_cube_plate}
\end{figure}
We assume the beam splitter to be a linear passive optical device, which suggests writing the relation between input and output fields given in the complex representation $\mathcal{E}(\omega)$ of the electric field component\footnote{The physical electrical field component $E(t)$ relates to the complex field $\mathcal{E}(t)$ via $E(t)=\Re\mathcal{E}(t)$. In the complex representation, phase shifts take the simple form of multiplications with a complex exponential. The frequency dependence coefficients are frequency-dependent, the linear transformation only holds in frequency space.} in the frequency domain\footnote{The frequency dependency of the reflection and transmission coefficients requires us to perform the linear transformation in frequency space.} as
\begin{equation}
    \begin{pmatrix}
        \mathcal{E}_1^\prime(\omega)
        \\
        \mathcal{E}_2^\prime(\omega)
    \end{pmatrix}
    =
    \begin{pmatrix}
        t(\omega) & r^\prime(\omega)
        \\
        r(\omega) & t^\prime(\omega)
    \end{pmatrix}
    \begin{pmatrix}
        \mathcal{E}_1(\omega)
        \\
        \mathcal{E}_2(\omega)
    \end{pmatrix}
    \label{eq:beam_splitter_coefficients_transform}
\end{equation}
wherein $r(\omega),r^\prime(\omega)$ and $t(\omega),t^\prime(\omega)$ are the complex reflection respective transmission coefficients of the beam splitter.
The absolute value of the transmission $\abs{t(\omega)},\abs{t^\prime(\omega)}$ and reflection coefficients $\abs{r(\omega)},\abs{r^\prime(\omega)}$ determine the splitting ratio of the input power among the outputs.
The complex phase factor of the reflection and transmission coefficients characterize the phase shifts with which the input fields contribute to the output fields.

The beam splitter is a passive device, therefore, the optical output power is bound by the input power
\begin{equation}
    \abs{\mathcal{E}_1^\prime(\omega)}^2
    +
    \abs{\mathcal{E}_2^\prime(\omega)}^2
    \leq
    \abs{\mathcal{E}_1(\omega)}^2
    +
    \abs{\mathcal{E}_2(\omega)}^2
    \label{eq:beam_splitter_passive}.
\end{equation}
Inserting \cref{eq:beam_splitter_coefficients_transform} into \cref{eq:beam_splitter_passive} and setting the one of the input fields to zero, constraints the coefficients to
\begin{align}
    \abs{r(\omega)}^2+\abs{t(\omega)}^2\leq1,
    &&
    \abs{r^\prime(\omega)}^2+\abs{t^\prime(\omega)}^2\leq1
    \label{eq:beam_splitter_coefficients_constraint}.
\end{align}
Except for \cref{eq:beam_splitter_coefficients_constraint}, there are no further constraints to the coefficients describing a linear passive optical four-port in \cref{eq:beam_splitter_coefficients_transform}.
% TODO: is this even true? Don't we already assume with eq. (3.1.1) that there is no back-reflection?
Theorists often claim the transformation in \cref{eq:beam_splitter_coefficients_transform} to be symmetric (or reciprocal) due to Maxwell's equations~\cite[p.~129]{Haroche2006}.
Linear passive optical two-ports are only reciprocal for optical system with a single dielectric layer~\cite{Potton2004} but most physical beam splitters comprise multiple dielectric layers.\footnote{For example, cubic beam splitters have a coating followed by optical cement between the prisms breaking reciprocal symmetry of the system.}

Using classical wave optics and perfect knowledge of the dimensions and material properties, it is possible to derive exact expressions of the complex reflection $r(\omega),r^\prime(\omega)$ and transmission coefficients $t(\omega),t^\prime(\omega)$.
For example, Hénault~\cite{Henault2015} derived an exact expression for the reflected and transmitted amplitudes of a plate beam splitter with one input and a single dielectric layer.
Likewise, Hamilton~\cite{Hamilton2000} discusses the cubic beam splitter with two inputs and different coatings.
In general, the complex reflection and transmission coefficients need to account for multiple reflections at different dielectric layers inside the beam splitter.

We conclude that given exact knowledge of the design parameters, we can find a physical model describing the complex reflection and transmission coefficients.
Unfortunately, these parameters are rarely available and are subject to manufacturing variations.
The situation worsens for fiber optics, where only limited physical parameters are accessible without breaking the device.
% TODO: conclusion?

\subsection{Unitary transformation of an ideal beam splitter}

We define an ideal beam splitter as a linear passive optical four-port that is
\begin{enumerate}
    \item lossless (the energy of the input fields is equal to the energy of the output fields), and
    \item independent of frequency (for the frequency range of the input fields).
\end{enumerate}
In this case, the transformation relating the output with the input fields is unitary.\footnote{Unitary transformations preserve the inner product and $\abs{\mathcal{E}_1}^2+\abs{\mathcal{E}_2}^2=\abs{\mathcal{E}^\prime_1}^2+\abs{\mathcal{E}^\prime_2}^2$ is the norm of the input and output vectors induced by the inner product. As a consequence, a unitary transform $\hat{U}$ satisfies $\hat{U}^{-1}=\hat{U}^\dagger$.}
In general, the unitary transformation describing a lossless beam splitter has four \gls{dof}.\footnote{The beam splitter is described by a linear map $\hat{U}\colon\mathbb{C}^2\to\mathbb{C}^2$ which has eight \gls{dof}. $\hat{U}\hat{U}^\dagger=\mathbb{1}$ removes four \gls{dof}.}
Different parametrizations towards a two-dimensional unitary transformation exist.
In the previous section, we used a parametrization, \cref{eq:beam_splitter_coefficients_transform}, based on the polar representation of the complex reflection and transmission coefficients.
Alternatively, a parametrization involving the Pauli matrices is used in the literature, see, for instance, Ref.~\cite{Zeilinger1981}.
In the following, we use a parametrization introduced by Leonhardt~\cite[p.~92]{Leonhardt2010}
\begin{equation}
    \begin{pmatrix}
        \mathcal{E}_1^\prime(t)
        \\
        \mathcal{E}_2^\prime(t)
    \end{pmatrix}
    =
    e^{i\Lambda}
    \begin{pmatrix}
        e^{+i\Phi} & 0
        \\
        0 & e^{-i\Phi}
    \end{pmatrix}
    \begin{pmatrix}
        \cos\Theta & \sin\Theta
        \\
        -\sin\Theta & \cos\Theta
    \end{pmatrix}
    \begin{pmatrix}
        e^{+i\Psi} & 0
        \\
        0 & e^{-i\Psi}
    \end{pmatrix}
    \begin{pmatrix}
        \mathcal{E}_1(t)
        \\
        \mathcal{E}_2(t)
    \end{pmatrix}
    \label{eq:beam_splitter_unitary_transform}
\end{equation}
parametrized by the angles $\Lambda,\Theta,\Psi,\Phi\in\mathbb{R}$.
The parametrization of \cref{eq:beam_splitter_unitary_transform} allows for a physical interpretation without assumptions of a physical model.
More precisely, the matrix product in \cref{eq:beam_splitter_unitary_transform} reads as the composition of a phase transform of $\pm\Psi$ applied to the input fields, a mixing of the input fields with angle $\Theta$, and a phase transform of $\pm\Phi$ applied to the mixed fields, in addition, the output fields pick up a global phase $\Lambda$ from the optical path length common to the beams.\footnote{Even the global phase $\Lambda$ becomes local, and thereby measurable, if we consider the beam splitter to be only part of a subsystem, see \cref{sec:mach_zehnder_modulator}.}

If we carry out the matrix multiplication in \cref{eq:beam_splitter_unitary_transform}, we find
\begin{equation}
    \begin{pmatrix}
        \mathcal{E}_1^\prime(t)
        \\
        \mathcal{E}_2^\prime(t)
    \end{pmatrix}
    =
    e^{i\Lambda}
    \begin{pmatrix}
        e^{i(+\Phi+\Psi)}\cos\Theta & e^{i(+\Phi-\Psi)}\sin\Theta
        \\
        -e^{i(-\Psi+\Phi)}\sin\Theta & e^{i(-\Psi-\Phi)}\cos\Theta
    \end{pmatrix}
    \begin{pmatrix}
        \mathcal{E}_1(t)
        \\
        \mathcal{E}_2(t)
    \end{pmatrix},
\end{equation}
we introduce $\phi=(\Phi+\Psi)/2,\psi=(\Phi-\Psi)/2$ and write
\begin{equation}
    \begin{pmatrix}
        \mathcal{E}_1^\prime(t)
        \\
        \mathcal{E}_2^\prime(t)
    \end{pmatrix}
    =
    e^{i\Lambda}
    \begin{pmatrix}
        e^{+i\psi/2}\cos\Theta & e^{i\phi/2}\sin\Theta
        \\
        e^{-i\phi/2+i\pi}\sin\Theta & e^{-i\psi/2}\cos\Theta
    \end{pmatrix}
    \begin{pmatrix}
        \mathcal{E}_1(t)
        \\
        \mathcal{E}_2(t)
    \end{pmatrix}
    \label{eq:beam_splitter_unitary_transform_expanded}.
\end{equation}
For $\Theta=0$ and $\Theta=\pi/2$ there is no mixing, yet, there is a subtle difference between these cases.
In the case of $\Theta=0$, \cref{eq:beam_splitter_unitary_transform_expanded} becomes
\begin{equation}
    \begin{pmatrix}
        \mathcal{E}_1^\prime(t)
        \\
        \mathcal{E}_2^\prime(t)
    \end{pmatrix}
    =
    e^{i\Lambda}
    \begin{pmatrix}
        e^{+i\psi/2} & 0
        \\
        0 & e^{-i\psi/2}
    \end{pmatrix}
    \begin{pmatrix}
        \mathcal{E}_1(t)
        \\
        \mathcal{E}_2(t)
    \end{pmatrix}
    =
    e^{i\Lambda}
    \begin{pmatrix}
        \mathcal{E}_1(t)e^{+i\psi/2}
        \\
        \mathcal{E}_2(t)e^{-i\psi/2}
    \end{pmatrix}
    \label{eq:beam_splitter_unitary_transform_transmission},
\end{equation}
in other words, an additional phase difference of $\psi$ is added between the output fields.
In the case of $\Theta=\pi/2$, \cref{eq:beam_splitter_unitary_transform_expanded} becomes
\begin{equation}
    \begin{pmatrix}
        \mathcal{E}_1^\prime(t)
        \\
        \mathcal{E}_2^\prime(t)
    \end{pmatrix}
    =
    e^{i\Lambda}
    \begin{pmatrix}
        0 & e^{+i\phi/2}
        \\
        e^{-i\psi/2+i\pi} & 0
    \end{pmatrix}
    \begin{pmatrix}
        \mathcal{E}_1(t)
        \\
        \mathcal{E}_2(t)
    \end{pmatrix}
    =
    e^{i\Lambda}
    \begin{pmatrix}
        \mathcal{E}_2(t)e^{+i\phi/2}
        \\
        \mathcal{E}_1(t)e^{-i\phi/2+i\pi}
    \end{pmatrix}
    \label{eq:beam_splitter_unitary_transform_reflection},
\end{equation}
and a phase difference of $\pi-\phi$ is added between the output fields.
The difference between these two phase shifts between the output fields yields $\phi+\psi=\pi$.
If we identify $\Theta=0$ with the transmitted and $\Theta=\pi/2$ with the reflected field\footnote{Only the physical models have a notion of reflectance and transmittance. For the unitary transformation there is no fundamental difference between $\Theta=0,\pi/2$ as the $\pi$ phase can be moved around by reparamtrization.}, we recover the statement~\cite{Zeilinger1981} that the phase shift between the transmitted and reflected fields of an ideal beam splitter adds up to $\pi$.\footnote{The statement should not be confused with the statement that the phase difference between the output fields are phase-shifted by $\pi$.}

Evidently, the phases $\psi,\phi$ are a source of confusion and require a thorough examination.
For simplicity, we assume a balanced lossless beam splitter which distributes the input power equally among the outputs.
The corresponding transformation is obtained by setting $\Theta=\pi/4$ in \cref{eq:beam_splitter_unitary_transform}
\begin{equation}
    \begin{pmatrix}
        \mathcal{E}_1^\prime(t)
        \\
        \mathcal{E}_2^\prime(t)
    \end{pmatrix}
    =
    \frac{e^{i\Lambda}}{\sqrt{2}}
    \begin{pmatrix}
        e^{i\phi} & e^{i\psi}
        \\
        -e^{-i\psi} & e^{-i\phi}
    \end{pmatrix}
    \begin{pmatrix}
        \mathcal{E}_1(t)
        \\
        \mathcal{E}_2(t)
    \end{pmatrix}
    =
    \frac{e^{i\Lambda}}{\sqrt{2}}
    \begin{pmatrix}
        \mathcal{E}_1(t)e^{i\phi}+\mathcal{E}_2(t)e^{i\psi}
        \\
        \mathcal{E}_1(t)e^{-i\psi+i\pi}+\mathcal{E}_2(t)e^{-i\phi}
    \end{pmatrix}
    \label{eq:beam_splitter_unitary_transform_balanced}.
\end{equation}
\Cref{eq:beam_splitter_unitary_transform_balanced} cannot be simplified without further assumptions\footnote{In principle, the dimensions, materials and coatings of a physical beam splitter can be chosen such that $\psi,\phi,\Lambda$ can take arbitrary values.}.
In the quantum optics literature, for instance, Ref.~\cite[p.~138]{Gerry2005}, we frequently find the claim that
\begin{equation}
    \begin{pmatrix}
        \mathcal{E}_1^\prime(t)
        \\
        \mathcal{E}_2^\prime(t)
    \end{pmatrix}
    =
    \frac{1}{\sqrt{2}}
    \begin{pmatrix}
        1 & i\\
        i & 1
    \end{pmatrix}
    \begin{pmatrix}
        \mathcal{E}_1(t)
        \\
        \mathcal{E}_2(t)
    \end{pmatrix}
    \label{eq:beam_splitter_unitary_transform_balanced_qo}
\end{equation}
describes an ideal balanced beam splitter.
Moreover, the quantum communication literature, for example, Ref.~\cite{Shapiro2009}, uses
\begin{equation}
    \begin{pmatrix}
        \mathcal{E}_1^\prime(t)
        \\
        \mathcal{E}_2^\prime(t)
    \end{pmatrix}
    =
    \frac{1}{\sqrt{2}}
    \begin{pmatrix}
        1 & -1\\
        1 & 1
    \end{pmatrix}
    \begin{pmatrix}
        \mathcal{E}_1(t)
        \\
        \mathcal{E}_2(t)
    \end{pmatrix}
    \label{eq:beam_splitter_unitary_transform_balanced_qc},
\end{equation}
which has the merit that we can write $\mathcal{E}_\pm(t)=\left(\mathcal{E}_1(t)\pm\mathcal{E}_2(t)\right)/\sqrt{2}$, as the transformation of an ideal balanced beam splitter.
While \cref{eq:beam_splitter_unitary_transform_balanced_qo} corresponds to $\Lambda=0,\psi=0,\phi=\pi/2$ in \cref{eq:beam_splitter_unitary_transform_balanced}, and \cref{eq:beam_splitter_unitary_transform_balanced_qc} can be obtained with $\Lambda=0,\phi=0,\psi=\pi$, we rarely find a justification to remove three physical \gls{dof}.
If we assume an ideal beam splitter to be
\begin{enumerate}
    \item no other fields except for the input fields to be in the system, and
    \item symmetric with respect to the input fields,
\end{enumerate}
it is possible to derive \cref{eq:beam_splitter_unitary_transform_balanced_qo} from \cref{eq:beam_splitter_unitary_transform_balanced}.
The first assumption lets us remove two \gls{dof}: first, $\Lambda$ is a global phase that cannot be measured, and we are free to choose $\Lambda$ as convenient, second, we are allowed to choose a common phase origin.
Let us explicitly write $\mathcal{E}_1(t)e^{i\delta_1}$ and $\mathcal{E}_2(t)e^{i\delta_2}$.
By adjusting the global phase $\Lambda$ accordingly, this is equivalent to $\mathcal{E}_1(t)e^{i\Delta\delta}$ and $\mathcal{E}_2(t)$ wherein $\Delta\delta=\delta_1-\delta_2$, and we can rewrite \cref{eq:beam_splitter_unitary_transform_balanced} as
\begin{equation}
    \begin{pmatrix}
        \mathcal{E}_1^\prime(t)
        \\
        \mathcal{E}_2^\prime(t)
    \end{pmatrix}
    =
    \frac{1}{\sqrt{2}}
    \begin{pmatrix}
        e^{i(\phi/2+\Delta\theta)} & e^{i\psi/2}
        \\
        -e^{-i(\psi/2-\Delta\theta)} & e^{-i\phi/2}
    \end{pmatrix}
    \begin{pmatrix}
        \mathcal{E}_1(t)
        \\
        \mathcal{E}_2(t)
    \end{pmatrix}
    \label{eq:beam_splitter_unitary_transform_balanced_phase_origin}.
\end{equation}
By the second assumption, $e^{i\psi/2}=-e^{i(\psi/2-\Delta\delta)}$, or equivalently, $\psi=\pi/2+\Delta\delta/2$ and \cref{eq:beam_splitter_unitary_transform_balanced_phase_origin} becomes
\begin{equation}
    \begin{pmatrix}
        \mathcal{E}_1^\prime(t)
        \\
        \mathcal{E}_2^\prime(t)
    \end{pmatrix}
    =
    \frac{e^{i\Delta\delta/2}}{\sqrt{2}}
    \begin{pmatrix}
        e^{i(\phi+\Delta\delta/2)} & i
        \\
        i & e^{-i(\phi+\Delta\delta/2)}
    \end{pmatrix}
    \begin{pmatrix}
        \mathcal{E}_1(t)
        \\
        \mathcal{E}_2(t)
    \end{pmatrix}
    .
\end{equation}
Removing the global phase and choosing the phase reference such that $\phi+\Delta\delta=0$ finally yields \cref{eq:beam_splitter_unitary_transform_balanced_qo}.
The assumption of a symmetric beam splitter is not justified for most physical beam splitters\footnote{See the discussion of physcial beam splitter types in \cref{sec:beam_splitter}.}.
Furthermore, the requirement that the optical system comprises only two beams massively restricts the use of \cref{eq:beam_splitter_unitary_transform_balanced_qo}.
That said, the assumptions of our ideal beam splitter are already very strong and in contrast to \cref{eq:beam_splitter_unitary_transform_balanced}, \cref{eq:beam_splitter_unitary_transform_balanced_qo} and \cref{eq:beam_splitter_unitary_transform_balanced_qc} allow simple calculations.

\subsection{Scattering parameters of a real beam splitter}

% TODO: cite reference where reflection and transmission properties are measured

\begin{figure}[htb]
    \centering
    \includestandalone[mode=buildnew]{figures/tikz/four-port}
    \caption{In- and outputs of a beam splitter used for the scattering parameter model.}\label{fig:beam_splitter_scattering_parameters}
\end{figure}

\begin{equation}
    \begin{pmatrix}
        \mathcal{E}_{1,\text{out}} \\
        \mathcal{E}_{2,\text{out}} \\
        \mathcal{E}_{3,\text{out}} \\
        \mathcal{E}_{4,\text{out}}
    \end{pmatrix}
    =
    T
    \begin{pmatrix}
        \mathcal{E}_{1,\text{in}} \\
        \mathcal{E}_{2,\text{in}} \\
        \mathcal{E}_{3,\text{in}} \\
        \mathcal{E}_{4,\text{in}}
    \end{pmatrix}
\end{equation}
where $T\in\mathbb{C}^{4\times4}$.

\begin{figure}[htb]
    \centering
    \includegraphics{figures/pstricks/optical-network-analyzer.pdf}
    \caption{Experimental setup of an optical network analyzer measuring the scattering parameters of a real beam splitter.}\label{fig:beam_splitter_network_analyzer}
\end{figure}

		
		\addcontentsline{toc}{section}{References}
		\printbibliography[title=References]
	\end{refsection}
	
	\chapter{Proofs}
	\begin{refsection}
		\section{Klein-Gordon field}

\subsection{Relativistic field theory}

\begin{delayedproof}{th:relativistic_energy_momentum}
	According to the action principle, the dynamics of the field are determined by the equations of motion which can be found by the relativistic Euler-Lagrange equations
	\begin{equation*}
		0
		=
		\partial_\mu\pdv{\mathcal{L}}{(\partial_\mu\phi)}
		-
		\pdv{\mathcal{L}}{\phi}
		=
		\left(
			\partial_\mu\partial^\mu
			+
			m^2
		\right)
		\phi(t,\vb{x})
		.
	\end{equation*}
	Assuming the existence of the Klein-Gordon field's Fourier representation\footnote{See Ref.~\cite[p.~341]{Cohen2019} for a definition of a Minkowski Fourier transform.}
	\begin{equation*}
		\phi(t,\vb{x})
		=
		\int_{\mathbb{R}^3}\frac{\dd[3]{p}}{(2\pi)^3}
		\phi(t,\vb{p})
		e^{-i\vb{p}\vdot\vb{x}}
		=
		\int_{\mathbb{R}^4}\frac{\dd[4]{p}}{(2\pi)^4}
		\phi(p_0,\vb{p})
		e^{+ip_\mu x^\mu}
		,
	\end{equation*}
	the equation of motion in momentum space reduces to
	\begin{equation*}
		0
		=
		\left(
			ip_\mu ip^\mu
			+
			m^2
		\right)
		\phi(p_0,\vb{p})
		=
		-
		\left(
			p_0^2
			-
			\omega(\vb{p})^2
		\right)
		\phi(p_0,\vb{p})
	\end{equation*}
	which is satisfied if $p_0=\pm\omega(\vb{p})$.
\end{delayedproof}
\begin{delayedproof}{thm:kg_fourier_expansion}
	There are two approaches to prove \cref{thm:kg_fourier_expansion}.
	In a first approach, we calculate with the proposed Fourier expansion
	\begin{equation*}
		\begin{split}
			\partial_\mu
			\partial^\mu
			\phi(t,\vb{x})
			&=
			\int\frac{\dd[3]{p}}{(2\pi)^3\sqrt{2\omega(\vb{p})}}
			\left\{
				a^*(\vb{p})
				\partial_\mu
				\partial^\mu
				e^{-ip_\mu x^\mu}
				+
				\text{c.c.}
			\right\}
			\\
			&=
			\int\frac{\dd[3]{p}}{(2\pi)^3\sqrt{2\omega(\vb{p})}}
			\left\{
				a^*(\vb{p})
				\left(
					-
					p_\mu
					p^\mu
				\right)
				e^{-ip_\mu x^\mu}
				+
				\text{c.c.}
			\right\}
		\end{split}
	\end{equation*}
	where $p_\mu p^\mu=\omega(\vb{p})^2-\vb{p}=m^2$ and therefore the equation of motion
	\begin{equation*}
			\partial_\mu
			\partial^\mu
			\phi(t,\vb{x})
			=
			-
			m^2
			\phi(t,\vb{x})
	\end{equation*}
	is satisfied.
	While the first approach successfully shows why the theorem is true, it is not obvious how to arrive at the Fourier expansion.
	Therefor, a second approach starts with the Fourier transform of the Klein-Gordon field
	\begin{equation*}
		\phi(t,\vb{x})
		=
		\int_V\frac{\dd[4]{p}}{(2\pi)^4}
		\phi(p_0,\vb{p})
		e^{+ip_\mu x^\mu}
	\end{equation*}
	where the integration domain is constrained to the momentum lightcone $V$ and therefore $\omega(\vb{p})^2=p_0^2$ is automatically satisfied.
	We are left to rewrite the constrained integration domain
	\begin{equation*}
		\phi(t,\vb{x})
		=
		\int_{\mathbb{R}^4}\frac{\dd[4]{p}}{(2\pi)^3}
		\delta^{(1)}\left(p_0^2-\omega(\vb{p})^2\right)
		\phi(p_0,\vb{p})
		e^{+ip_\mu x^\mu}
	\end{equation*}
	and using the composition property of the delta distribution
	\begin{equation*}
		\delta^{(1)}\left(p_0^2-\omega(\vb{p})^2\right)
		=
		\frac{
			\delta^{(1)}\left(p_0-\omega(\vb{p})\right)
			+
			\delta^{(1)}\left(p_0+\omega(\vb{p})\right)
		}{\sqrt{2\omega(\vb{p})}}
	\end{equation*}
	which leaves us with
	\begin{equation*}
		\phi(t,\vb{x})
		=
		\int_{\mathbb{R}^3}\frac{\dd[3]{p}}{(2\pi)^3\sqrt{2\omega(\vb{p})}}
		\biggl\{
			\phi(\omega(\vb{p}),\vb{p})
			e^{+i\omega(\vb{p})t}
			+
			\phi(-\omega(\vb{p}),\vb{p})
			e^{-i\omega(\vb{p})t}
		\biggr\}
		e^{-i\vb{p}\vdot\vb{x}}
		.
	\end{equation*}
	We now only need to perform the substitution $\vb{p}\to-\vb{p}$ on the second term
	\begin{equation*}
		\begin{split}
			\phi(t,\vb{x})
			&=
			\int_{\mathbb{R}^3}\frac{\dd[3]{p}}{(2\pi)^3\sqrt{2\omega(\vb{p})}}
			\biggl\{
				\phi(\omega(\vb{p}),\vb{p})
				e^{+i\omega(\vb{p})t}
				e^{-i\vb{p}\vdot\vb{x}}
				+
				\phi(-\omega(\vb{p}),\vb{p})
				e^{-i\omega(\vb{p})t}
				e^{-i\vb{p}\vdot\vb{x}}
			\biggr\}
			\\
			&=
			\int_{\mathbb{R}^3}\frac{\dd[3]{p}}{(2\pi)^3\sqrt{2\omega(\vb{p})}}
			\biggl\{
				\phi(\omega(\vb{p}),\vb{p})
				e^{+i\omega(\vb{p})t}
				e^{-i\vb{p}\vdot\vb{x}}
				+
				\phi(-\omega(\vb{p}),-\vb{p})
				e^{-i\omega(\vb{p})t}
				e^{+i\vb{p}\vdot\vb{x}}
			\biggr\}
			\\
			&=
			\int_{\mathbb{R}^3}\frac{\dd[3]{p}}{(2\pi)^3\sqrt{2\omega(\vb{p})}}
			\biggl\{
				\phi(\omega(\vb{p}),\vb{p})
				e^{+ip_\mu x^\mu}
				+
				\phi(\omega(\vb{p}),\vb{p})^*
				e^{-ip_\mu x^\mu}
			\biggr\}_{p_0=\omega(\vb{p})}
		\end{split}
	\end{equation*}
	and use the conjugate symmetry of the Fourier amplitudes
	\begin{equation*}
		\phi(p_0,\vb{p})^*
		=
		\phi(-p_0,-\vb{p})
	\end{equation*}
	which is present because $\phi(t,\vb{x})$ is real-valued.
\end{delayedproof}
\begin{delayedproof}{thm:kg_total_energy_momentum}
	Perform a spatial integration over the energy and momentum densities, see Ref.~\cite{Peskin1995}.
\end{delayedproof}
\begin{delayedproof}{thm:kg_comm_ac}
	Insert the Klein-Gordon field operators in terms of the annihilation and creation operators, \cref{eq:qkg_pos} and \cref{eq:qkg_mom}, into the field commutation relations \cref{eq:qkg_comm_pm}.
\end{delayedproof}

\subsection{Coordinate wave function properties}

\begin{delayedproof}{thm:coordinate_wave_function_simplified}
	The Klein-Gordon operator acting on the vacuum state reduces to the negative frequency Klein-Gordon operator
	\begin{equation*}
		\hat\phi(t,\vb{x})
		\ket{0}
		=
		\hat\phi^-(t,\vb{x})
		\ket{0}
	\end{equation*}
	because the annihilation operator destroy the vacuum $\hat{a}(\vb{p})\ket{0}=0$.
	We are left to take hermitian conjugate
	\begin{equation*}
		\begin{split}
			\bra{0}
			\hat\phi^+(t,\vb{x})
			&=
			\left(
				\hat\phi^-(t,\vb{x})
				\ket{0}
			\right)^\dagger
			\\
			&=
			\left(
				\int\frac{\dd[3]{p}}{(2\pi)^3\sqrt{2\omega(\vb{p})}}
				e^{+ip_\mu x^\mu}
				\hat{a}^\dagger(\vb{p})
				\ket{0}
			\right)^\dagger
			\\
			&=
			\int\frac{\dd[3]{p}}{(2\pi)^3\sqrt{2\omega(\vb{p})}}
			e^{-ip_\mu x^\mu}
			\bra{0}
			\hat{a}(\vb{p})
		\end{split}
	\end{equation*}
	and insert the definition of the negative frequency Klein-Gordon operator.
\end{delayedproof}

\section{Quantum states}

\subsection{Momentum state}

\begin{delayedproof}{thm:momentum_state_non_normalizable}
	The inner product of two momentum states is
	\begin{equation*}
		\braket{\vb{p}}{\vb{q}}
		=
		\sqrt{2\omega(\vb{p})}
		\sqrt{2\omega(\vb{q})}
		\expval{\hat{a}(\vb{p})\hat{a}^\dagger(\vb{q})}{0}
		=
		2\omega(\vb{p})
		(2\pi)^3
		\delta^{(3)}(\vb{q}-\vb{p})
	\end{equation*}
	where we used the commutation relation and $\hat{a}(\vb{p})\ket{0}=0$.
	For $\vb{p}=\vb{q}$ the inner product is proportional to $\delta^{(3)}(0)$ which has no consistent mathematical definition.
\end{delayedproof}
\begin{delayedproof}{thm:momentum_state_eigenstate}
	\begin{equation*}
		\begin{split}
			\hat{\vb{P}}
			\ket{\vb{p}}
			&=
			\int\frac{\dd[3]{q}}{(2\pi)^3}
			\vb{q}
			\hat{a}^\dagger(\vb{q})
			\hat{a}(\vb{q})
			\sqrt{2\omega(\vb{p})}
			\hat{a}^\dagger(\vb{p})
			\ket{0}
			\\
			&=
			\int\frac{\dd[3]{q}}{(2\pi)^3}
			\vb{q}
			\sqrt{2\omega(\vb{p})}
			\hat{a}^\dagger(\vb{q})
			(2\pi)^3
			\delta^{(3)}(\vb{q}-\vb{p})
			\ket{0}
			\\
			&=
			\vb{p}
			\sqrt{2\omega(\vb{p})}
			\hat{a}^\dagger(\vb{p})
			\ket{0}
			=
			\vb{p}
			\ket{\vb{p}}
		\end{split}
	\end{equation*}
\end{delayedproof}
\begin{delayedproof}{thm:momentum_state_wave_function}
	We use \Cref{thm:coordinate_wave_function_simplified} and insert the definition of the momentum state
	\begin{equation*}
		\begin{split}
			\psi(t,\vb{x})
			&=
			\int\frac{\dd[3]{p}}{(2\pi)^3\sqrt{2\omega(\vb{p})}}
			e^{-ip_\mu x^\mu}
			\bra{0}
			\hat{a}(\vb{p})
			\ket{\vb{q}}
			\\
			&=
			\int\frac{\dd[3]{p}}{(2\pi)^3\sqrt{2\omega(\vb{p})}}
			e^{-ip_\mu x^\mu}
			\sqrt{2\omega(\vb{q})}
			\expval{
				\hat{a}(\vb{p})
				\hat{a}^\dagger(\vb{p})
			}{0}
			\\
			&=
			e^{-iq_\mu x^\mu}
		\end{split}
		.
	\end{equation*}
\end{delayedproof}

\subsection{Single-particle number state}

\begin{delayedproof}{thm:single_particle_number_state_smeared_kg}
	First, we note that the positive frequency Klein-Gordon operator vanishes,
	\begin{equation*}
		\int\dd[4]{x}
		f(x)
		\hat\phi(x)
		\ket{0}
		=
		\int\dd[4]{x}
		f(x)
		\hat\phi^-(x)
		\ket{0}
		,
	\end{equation*}
	because $\hat{a}(\vb{p})\ket{0}=0$, then we insert the definition of the negative frequency part \cref{eq:qkg_positive_negative_frequency}
	\begin{equation*}
		\begin{split}
			\int\dd[4]{x}
			f(x)
			\hat\phi^-(x)
			\ket{0}
			&=
			\int\dd[4]{x}
			f(x)
			\int\frac{\dd[3]{p}}{(2\pi)^3\sqrt{2\omega(\vb{p})}}
			e^{+ip_\mu x^\mu}
			\hat{a}^\dagger(\vb{p})
			\ket{0}
			\\
			&=
			\int\frac{\dd[3]{p}}{(2\pi)^3\sqrt{2\omega(\vb{p})}}
			\left(
				\int\dd[4]{x}
				f(x)
				e^{+ip_\mu x^\mu}
			\right)
			\hat{a}^\dagger(\vb{p})
			\ket{0}
			\\
			&=
			\int\frac{\dd[3]{p}}{(2\pi)^3\sqrt{2\omega(\vb{p})}}
			f(\vb{p})
			\hat{a}^\dagger(\vb{p})
			\ket{0}
			\\
			&=
			\ket{1_f}
		\end{split}
	\end{equation*}
	and recognize the single-particle number state.
\end{delayedproof}
\begin{delayedproof}{th:single_particle_number_states_inner_product}
	Simply inserting the definition and using the commutation relations
	\begin{equation*}
		\begin{split}
			\bra{1_g}\ket{1_f}
			&=
			\int\frac{\dd[3]{p}}{(2\pi)^3\sqrt{2\omega(\vb{p})}}
			\int\frac{\dd[3]{q}}{(2\pi)^3\sqrt{2\omega(\vb{q})}}
			f(\vb{p})g(\vb{q})^*
			\expval{\hat{a}(\vb{q})\hat{a}^\dagger(\vb{p})}{0}
			\\
			&=
			\int\frac{\dd[3]{p}}{(2\pi)^3\sqrt{2\omega(\vb{p})}}
			\int\frac{\dd[3]{q}}{(2\pi)^3\sqrt{2\omega(\vb{q})}}
			f(\vb{p})g(\vb{q})^*
			(2\pi)^3\delta^{(3)}(\vb{q}-\vb{p})
			\\
			&=
			\int\frac{\dd[3]{p}}{(2\pi)^32\omega(\vb{p})}
			f(\vb{p})g(\vb{p})^*
		\end{split}
	\end{equation*}
	shows the equality.
\end{delayedproof}
\begin{delayedproof}{thm:single_particle_number_state_normalization}
	The probabilistic interpretation of quantum mechanics requires the absolute square of the probability amplitude to equal one, i.e.,
	\begin{equation*}
		\abs{\braket{1_f}}^2
		=
		1
		.
	\end{equation*}
	From \Cref{th:single_particle_number_states_inner_product}, we know that
	\begin{equation*}
		\braket{1_f}
		=
		\int\frac{\dd[3]{p}}{(2\pi)^32\omega(\vb{p})}
		\abs{f(\vb{p})}^2
	\end{equation*}
	which is real-valued. Combining both insights, we must conclude
	\begin{equation*}
		\braket{1_f}
		=
		\int\frac{\dd[3]{p}}{(2\pi)^32\omega(\vb{p})}
		\abs{f(\vb{p})}^2
		=
		1
		.
	\end{equation*}
\end{delayedproof}
\begin{delayedproof}{thm:single_particle_number_state_number_eigenstate}
	Inserting the definitions and carefully applying the commutation relations
	\begin{equation*}
		\begin{split}
			\hat{N}
			\ket{1_f}
			&=
			\int\frac{\dd[3]{p}}{(2\pi)^3}
			\hat{a}^\dagger(\vb{p})
			\hat{a}(\vb{p})
			\int\frac{\dd[3]{q}}{(2\pi)^3\sqrt{2\omega(\vb{q})}}
			f(\vb{q})
			\hat{a}^\dagger(\vb{q})
			\ket{0}
			\\
			&=
			\int\frac{\dd[3]{p}}{(2\pi)^3}
			\int\frac{\dd[3]{q}}{(2\pi)^3\sqrt{2\omega(\vb{q})}}
			f(\vb{q})
			\hat{a}^\dagger(\vb{p})
			\hat{a}(\vb{p})
			\hat{a}^\dagger(\vb{q})
			\ket{0}
			\\
			&=
			\int\frac{\dd[3]{p}}{(2\pi)^3}
			\int\frac{\dd[3]{q}}{(2\pi)^3\sqrt{2\omega(\vb{q})}}
			f(\vb{q})
			\hat{a}^\dagger(\vb{p})
			(2\pi)^3
			\delta^{(3)}(\vb{q}-\vb{p})
			\ket{0}
			\\
			&=
			\int\frac{\dd[3]{p}}{(2\pi)^3\sqrt{2\omega(\vb{p})}}
			f(\vb{p})
			\hat{a}^\dagger(\vb{p})
			\ket{0}
			\\
			&=
			1
			\ket{1_f}
		\end{split}
		,
	\end{equation*}
	we identify the single-particle state which has trivial eigenvalue $1$.
\end{delayedproof}
\begin{delayedproof}{thm:single_particle_number_state_energy}
	Using the auxiliary result
	\begin{equation*}
		\begin{split}
			\hat{H}
			\ket{1_f}
			&=
			\int\frac{\dd[3]{p}}{(2\pi)^3}
			\omega(\vb{p})
			\hat{a}^\dagger(\vb{p})
			\hat{a}(\vb{p})
			\int\frac{\dd[3]{q}}{(2\pi)^3\sqrt{2\omega(\vb{q})}}
			f(\vb{q})
			\hat{a}^\dagger(\vb{q})
			\ket{0}
			\\
			&=
			\int\frac{\dd[3]{q}}{(2\pi)^3\sqrt{2\omega(\vb{q})}}
			f(\vb{q})
			\int\frac{\dd[3]{p}}{(2\pi)^3}
			\omega(\vb{p})
			\hat{a}^\dagger(\vb{p})
			\hat{a}(\vb{p})
			\hat{a}^\dagger(\vb{q})
			\ket{0}
			\\
			&=
			\int\frac{\dd[3]{q}}{(2\pi)^3\sqrt{2\omega(\vb{q})}}
			f(\vb{q})
			\int\frac{\dd[3]{p}}{(2\pi)^3}
			\omega(\vb{p})
			\hat{a}^\dagger(\vb{p})
			(2\pi)^3
			\delta^{(3)}(\vb{p}-\vb{q})
			\ket{0}
			\\
			&=
			\int\frac{\dd[3]{q}}{(2\pi)^3\sqrt{2\omega(\vb{q})}}
			f(\vb{q})
			\omega(\vb{q})
			\hat{a}^\dagger(\vb{q})
			\ket{0}
		\end{split}
		,
	\end{equation*}
	the first moment turns out to be
	\begin{equation*}
		\begin{split}
			\expval{\hat{H}}{1_f}
			&=
			\int\frac{\dd[3]{p}}{(2\pi)^3\sqrt{2\omega(\vb{p})}}
			f(\vb{p})^*
			\bra{0}
			\hat{a}(\vb{p})
			\int\frac{\dd[3]{q}}{(2\pi)^3\sqrt{2\omega(\vb{q})}}
			f(\vb{q})
			\omega(\vb{q})
			\hat{a}^\dagger(\vb{q})
			\ket{0}
			\\
			&=
			\int\frac{\dd[3]{p}}{(2\pi)^3\sqrt{2\omega(\vb{p})}}
			\int\frac{\dd[3]{q}}{(2\pi)^3\sqrt{2\omega(\vb{q})}}
			\omega(\vb{q})
			f(\vb{p})^*
			f(\vb{q})
			\expval{
				\hat{a}(\vb{p})
				\hat{a}^\dagger(\vb{q})
			}{0}
			\\
			&=
			\int\frac{\dd[3]{p}}{(2\pi)^32\omega(\vb{p})}
			\omega(\vb{p})
			\abs{f(\vb{p})}^2
		\end{split}
	\end{equation*}
	in agreement with Ref.~\cite[eqs.~10 and 11]{Naumov2013}.
	The second moment turns out as
	\begin{equation*}
		\begin{split}
			\expval{\hat{H}^2}{1_f}
			&=
			\int\frac{\dd[3]{p}}{(2\pi)^3\sqrt{2\omega(\vb{p})}}
			f(\vb{p})^*
			\omega(\vb{p})
			\bra{0}
			\hat{a}(\vb{p})
			\int\frac{\dd[3]{q}}{(2\pi)^3\sqrt{2\omega(\vb{q})}}
			f(\vb{q})
			\omega(\vb{q})
			\hat{a}^\dagger(\vb{q})
			\ket{0}
			\\
			&=
			\int\frac{\dd[3]{p}}{(2\pi)^3\sqrt{2\omega(\vb{p})}}
			\int\frac{\dd[3]{q}}{(2\pi)^3\sqrt{2\omega(\vb{q})}}
			\omega(\vb{p})
			\omega(\vb{q})
			f(\vb{p})^*
			f(\vb{q})
			\expval{
				\hat{a}(\vb{p})
				\hat{a}^\dagger(\vb{q})
			}{0}
			\\
			&=
			\int\frac{\dd[3]{p}}{(2\pi)^32\omega(\vb{p})}
			\omega(\vb{p})^2
			\abs{f(\vb{p})}^2
		\end{split}
		.
	\end{equation*}
\end{delayedproof}
\begin{delayedproof}{thm:single_particle_number_state_wave_function}
	of the coordinate wave function
	\begin{equation*}
		\psi(t,\vb{x})
		=
		\bra{0}
		\hat\phi(t,\vb{x})
		\ket{1_f}
		=
		\bra{0}
		\hat\phi^+(t,\vb{x})
		\ket{1_f}
		.
	\end{equation*}
	Inserting the definition of the single-particle number state yields
	\begin{equation*}
		\begin{split}
			\psi(t,\vb{x})
			&=
			\int\frac{\dd[3]{p}}{(2\pi)^3\sqrt{2\omega(\vb{p})}}
			e^{-ip_\mu x^\mu}
			\expval{
				\hat{a}(\vb{p})
				\int\frac{\dd[3]{q}}{(2\pi)^3\sqrt{2\omega(\vb{q})}}
				f(\vb{q})
				\hat{a}^\dagger(\vb{p})
			}{0}
			\\
			&=
			\int\frac{\dd[3]{p}}{(2\pi)^3\sqrt{2\omega(\vb{p})}}
			\int\frac{\dd[3]{q}}{(2\pi)^3\sqrt{2\omega(\vb{q})}}
			e^{-ip_\mu x^\mu}
			f(\vb{q})
			\expval{
				\hat{a}(\vb{p})
				\hat{a}^\dagger(\vb{p})
			}{0}
			\\
			&=
			\int\frac{\dd[3]{p}}{(2\pi)^3\sqrt{2\omega(\vb{p})}}
			\int\frac{\dd[3]{q}}{(2\pi)^3\sqrt{2\omega(\vb{q})}}
			e^{-ip_\mu x^\mu}
			f(\vb{q})
			(2\pi)^3
			\delta^{(3)}(\vb{q}-\vb{p})
			\\
			&=
			\int\frac{\dd[3]{p}}{(2\pi)^32\omega(\vb{p})}
			e^{-ip_\mu x^\mu}
			f(\vb{p})
		\end{split}
	\end{equation*}
	in agreement with Ref.~\cite[eq.~4]{Naumov2013}.
\end{delayedproof}
\begin{lemma}
	The probability current of a single-particle number state is
	\begin{equation}
		j_\mu(t,\vb{x})
		=
		\int\frac{\dd[3]{p}}{(2\pi)^32\omega(\vb{p})}
		\int\frac{\dd[3]{q}}{(2\pi)^32\omega(\vb{q})}
		q_\mu
		2\Re\left\{
			f(\vb{p})^*
			f(\vb{q})
			e^{-i(q_\mu-p_\mu)x^\mu}
		\right\}
	\end{equation}
\end{lemma}
\begin{proof}
	The probability current is\footnote{See Ref.~\cite[p.~18]{Peskin1995} for a derivation from Noether's theorem}
	\begin{equation*}
		j^\mu(t,\vb{x})
		=
		i
		\bigl\{
			\psi(t,\vb{x})
			\partial^\mu
			\psi(t,\vb{x})^*
			-
			\psi(t,\vb{x})^*
			\partial^\mu
			\psi(t,\vb{x})
		\bigr\}
	\end{equation*}
	and can be written
	\begin{equation*}
		j^\mu(t,\vb{x})
		=
		2\frac{\psi(t,\vb{x})\partial^\mu\psi(t,\vb{x})^*-\text{c.c.}}{2i}
		=
		2\Im\left\{
			\psi(t,\vb{x})
			\partial^\mu
			\psi(t,\vb{x})^*
		\right\}
		.
	\end{equation*}
	We proceed with the argument of the imaginary part
	\begin{equation*}
		\begin{split}
			\psi(t,\vb{x})^*
			\partial^\mu
			\psi(t,\vb{x})
			&=
			\int\frac{\dd[3]{p}}{(2\pi)^32\omega(\vb{p})}
			e^{-ip_\mu x^\mu}
			f(\vb{p})
			\partial^\mu
			\int\frac{\dd[3]{q}}{(2\pi)^32\omega(\vb{q})}
			e^{+iq_\mu x^\mu}
			f(\vb{q})^*
			\\
			&=
			\int\frac{\dd[3]{p}}{(2\pi)^32\omega(\vb{p})}
			\int\frac{\dd[3]{q}}{(2\pi)^32\omega(\vb{q})}
			iq^\mu
			f(\vb{p})
			f(\vb{q})^*
			e^{-i(p_\mu-q_\mu)x^\mu}
		\end{split}
		.
	\end{equation*}
	Inserting the argument back into the probability current and using $\Im{iz}=\Re{z}$, we find
	\begin{equation*}
		j^\mu(t,\vb{x})
		=
		\int\frac{\dd[3]{p}}{(2\pi)^32\omega(\vb{p})}
		\int\frac{\dd[3]{q}}{(2\pi)^32\omega(\vb{q})}
		2\Re\left\{
			q^\mu
			f(\vb{p})
			f(\vb{q})^*
			e^{-i(p_\mu-q_\mu)x^\mu}
		\right\}
		.
	\end{equation*}
	Writing out the real part and relabeling the integration variables in the second term yields
	\begin{equation*}
		j^\mu(t,\vb{x})
		=
		\int\frac{\dd[3]{p}}{(2\pi)^32\omega(\vb{p})}
		\int\frac{\dd[3]{q}}{(2\pi)^32\omega(\vb{q})}
		\left\{
			q^\mu
			+
			p^\mu
		\right\}
		f(\vb{p})
		f(\vb{q})^*
		e^{-i(p_\mu-q_\mu)x^\mu}
		.
	\end{equation*}	
	The last result is in agreement with Ref.~\cite[eqs.~36,37]{Naumov2013} if one further assumes $f(\vb{p})$ to be real.
\end{proof}
\begin{delayedproof}{thm:single_particle_number_state_group_velocity}
	No explicit proof, result claimed in Ref.~\cite[eq.~38]{Naumov2013}.
\end{delayedproof}

\subsection{Multi-particle number state}

\begin{lemma}\label{th:normal_ordered_a1_cn}
	Let $\hat{a}(\vb{p}),\hat{a}^\dagger(\vb{p})$ be the annihilation and creation operator of the Klein-Gordon field satisfying the canonical commutation relations then for $n\in\mathbb{N}$
	\begin{equation}
		\hat{a}(\vb{p})
		\prod_{j=1}^n
		\hat{a}^\dagger(\vb{q}_j)
		\ket{0}
		=
		\sum_{i=1}^n
		(2\pi)^3
		\delta^{(3)}(\vb{q}_i-\vb{p})
		\prod_{\substack{j=1\\j\neq i}}^n
		\hat{a}^\dagger(\vb{q}_j)
		\ket{0}
		\label{eq:normal_ordered_a1_cn}
		.
	\end{equation}
\end{lemma}
\begin{proof}
	Induction start $n=1$:
	\begin{equation*}
		\hat{a}(\vb{p})
		\prod_{j=1}^1
		\hat{a}^\dagger(\vb{q}_j)
		\ket{0}
		=
		\hat{a}(\vb{p})
		\hat{a}^\dagger(\vb{q}_1)
		\ket{0}
		=
		(2\pi)^3
		\delta^{(3)}(\vb{q}_1-\vb{p})
		\ket{0}
	\end{equation*}
	where we used the commutation relation and $\hat{a}(\vb{p})\ket{0}=0$.

	Induction step $n\to n+1$:
	\begin{equation*}
		\begin{split}
			\hat{a}(\vb{p})
			\prod_{j=1}^{n+1}
			\hat{a}^\dagger(\vb{q}_j)
			\ket{0}
			&=
			\hat{a}(\vb{p})
			\left(
				\prod_{j=1}^n
				\hat{a}^\dagger(\vb{q}_j)
			\right)
			\hat{a}^\dagger(\vb{q}_{n+1})
			\ket{0}
			\\
			&=
			\hat{a}(\vb{p})
			\hat{a}^\dagger(\vb{q}_{n+1})
			\left(
				\prod_{j=1}^n
				\hat{a}^\dagger(\vb{q}_j)
			\right)
			\ket{0}
			\\
			&=
			(2\pi)^3
			\delta^{(3)}(\vb{q}_{n+1}-\vb{p})
			\prod_{j=1}^n
			\hat{a}^\dagger(\vb{q}_j)
			\ket{0}
			+
			\hat{a}^\dagger(\vb{q}_{n+1})
			\hat{a}(\vb{p})
			\prod_{j=1}^n
			\hat{a}^\dagger(\vb{q}_j)
			\ket{0}
			\\
			&=
			(2\pi)^3
			\delta^{(3)}(\vb{q}_{n+1}-\vb{p})
			\prod_{j=1}^n
			\hat{a}^\dagger(\vb{q}_j)
			\ket{0}
			+
			\sum_{i=1}^n
			(2\pi)^3
			\delta^{(3)}(\vb{q}_i-\vb{p})
			\prod_{\substack{j=1\\j\neq i}}^{n+1}
			\hat{a}^\dagger(\vb{q}_j)
			\ket{0}
			\\
			&=
			\sum_{i=1}^{n+1}
			(2\pi)^3
			\delta^{(3)}(\vb{q}_i-\vb{p})
			\prod_{\substack{j=1\\j\neq i}}^{n+1}
			\hat{a}^\dagger(\vb{q}_j)
			\ket{0}
		\end{split}
	\end{equation*}
	where we used that creation operators commute in the first line, the canonical commutation relation in the second line, and the induction hypothesis in the third line.
\end{proof}
\begin{lemma}\label{thm:anti_normal_expvalue}
	Let $n,m\in\mathbb{N}$ then
	\begin{equation}
		\expval{
			\hat{a}(\vb{p}_1)
			\dots
			\hat{a}(\vb{p}_n)
			\hat{a}^\dagger(\vb{q}_1)
			\dots
			\hat{a}^\dagger(\vb{q}_m)
		}{0}
		=
		\delta_{nm}
		\sum_{\pi\in\textrm{perm}}
		\prod^n_{i=1}
		(2\pi)^3
		\delta^{(3)}(\vb{p}_i-\vb{q}_{\pi(i)})
	\end{equation}
	where the sum is over all pairwise permutations of $(i,j)\in\left\{1,\dots,n\right\}^2$.
\end{lemma}
\begin{proof}
	Assuming $s=n-m>0$, then
	\begin{equation*}
		\expval{
			\hat{a}(\vb{p}_1)
			\dots
			\hat{a}(\vb{p}_n)
			\hat{a}^\dagger(\vb{q}_1)
			\dots
			\hat{a}^\dagger(\vb{q}_m)
		}{0}
		\propto
		\expval{
			\hat{a}(\vb{p}_{i_1})
			\dots
			\hat{a}(\vb{p}_{i_s})
		}{0}
		=
		0
	\end{equation*}
	because of $\hat{a}(\vb{p})\ket{0}=0$. The case $s<0$ follows directly from the hermitian conjugate.
	We conclude that the expectation value is only non-zero iff $n=m$.
	
	Induction start $n=1$:
	\begin{equation*}
		\expval{
			\hat{a}(\vb{p}_1)
			\hat{a}^\dagger(\vb{q}_1)
			\dots
			\hat{a}^\dagger(\vb{q}_m)
		}{0}
		=
		\delta_{m1}
		\expval{
			\hat{a}(\vb{p}_1)
			\hat{a}^\dagger(\vb{q}_1)
		}{0}
		=
		\delta_{m1}
		(2\pi)^3
		\delta^{(3)}(\vb{q}_1-\vb{p}_1)
	\end{equation*}
	
	Induction step $n\to n+1$:
	\begin{equation*}
		\begin{split}
			&\
			\expval{
				\hat{a}(\vb{p}_1)
				\dots
				\hat{a}(\vb{p}_{n+1})
				\hat{a}^\dagger(\vb{q}_1)
				\dots
				\hat{a}^\dagger(\vb{q}_{m+1})
			}{0}
			\\
			=&\
			\delta_{nm}
			\expval{
				\hat{a}(\vb{p}_1)
				\dots
				\hat{a}(\vb{p}_{n+1})
				\hat{a}^\dagger(\vb{q}_{n+1})
				\hat{a}^\dagger(\vb{q}_1)
				\dots
				\hat{a}^\dagger(\vb{q}_n)
			}{0}
			\\
			=&\
			\delta_{nm}
			(2\pi)^3
			\delta^{(3)}(\vb{q}_{n+1}-\vb{p}_{n+1})
			\expval{
				\hat{a}(\vb{p}_1)
				\dots
				\hat{a}(\vb{p}_n)
				\hat{a}^\dagger(\vb{q}_1)
				\dots
				\hat{a}^\dagger(\vb{q}_n)
			}{0}
			\\
			+&\
			\delta_{nm}
			\expval{
				\hat{a}(\vb{p}_1)
				\dots
				\hat{a}(\vb{p}_n)
				\hat{a}^\dagger(\vb{q}_{n+1})
				\hat{a}(\vb{p}_{n+1})
				\hat{a}^\dagger(\vb{q}_1)
				\dots
				\hat{a}^\dagger(\vb{q}_n)
			}{0}
		\end{split}
	\end{equation*}
	the second term can be further simplified to
	\begin{equation*}
		\begin{split}
			&\
			\expval{
				\hat{a}(\vb{p}_1)
				\dots
				\hat{a}(\vb{p}_n)
				\hat{a}^\dagger(\vb{q}_{n+1})
				\hat{a}(\vb{p}_{n+1})
				\hat{a}^\dagger(\vb{q}_1)
				\dots
				\hat{a}^\dagger(\vb{q}_n)
			}{0}
			\\
			=&\
			\left(
				\hat{a}(\vb{q}_{n+1})
				\hat{a}^\dagger(\vb{p}_1)
				\dots
				\hat{a}^\dagger(\vb{p}_n)
				\ket{0}
			\right)^\dagger
			\left(
				\hat{a}(\vb{p}_{n+1})
				\hat{a}^\dagger(\vb{q}_1)
				\dots
				\hat{a}^\dagger(\vb{q}_n)
				\ket{0}
			\right)
			\\
			=&\
			\left(
				\sum^n_{i=1}
				(2\pi)^3
				\delta^{(3)}(\vb{p}_i-\vb{q}_{n+1})
				\prod_{\substack{j=1\\j\neq i}}^n
				\hat{a}^\dagger(\vb{p}_j)
				\ket{0}
			\right)^\dagger
			\left(
				\sum^n_{l=1}
				(2\pi)^3
				\delta^{(3)}(\vb{q}_l-\vb{p}_{n+1})
				\prod_{\substack{k=1\\k\neq l}}^n
				\hat{a}^\dagger(\vb{p}_k)
				\ket{0}
			\right)
			\\
			=&\
			\sum^n_{i,l=1}
			\delta^{(3)}(\vb{p}_i-\vb{q}_{n+1})
			\delta^{(3)}(\vb{q}_l-\vb{p}_{n+1})
			\expval{
				\left(
					\prod_{\substack{j=1\\j\neq i}}^n
					\hat{a}(\vb{p}_j)
				\right)
				\left(
					\prod_{\substack{k=1\\k\neq l}}^n
					\hat{a}^\dagger(\vb{p}_k)
				\right)
			}{0}
		\end{split}
	\end{equation*}
	applying the induction hypothesis and inserting the second term back, we complete the induction step as all permutations are accounted for.
\end{proof}
\begin{delayedproof}{thm:multi_particle_number_state_inner_product}
	Using \Cref{thm:anti_normal_expvalue} and noting that the sum over the permutations compensates for the factorials, we find
	\begin{equation*}
		\begin{split}
			\braket{n_f}{m_g}
			&=
			\expval{
				\frac{1}{\sqrt{n!}}
				\left(
					\int\frac{\dd[3]{p}}{(2\pi)^3\sqrt{2\omega(\vb{p})}}
					f(\vb{p})^*
					\hat{a}(\vb{p})
				\right)^n
				\frac{1}{\sqrt{m!}}
				\left(
					\int\frac{\dd[3]{q}}{(2\pi)^3\sqrt{2\omega(\vb{q})}}
					g(\vb{q})
					\hat{a}^\dagger(\vb{q})
				\right)^m
			}{0}
			\\
			&=
			\int\frac{\dd[3]{p_1}}{(2\pi)^3\sqrt{2\omega(\vb{p}_1)}}
			\dots
			\int\frac{\dd[3]{p_n}}{(2\pi)^3\sqrt{2\omega(\vb{p}_n)}}
			\int\frac{\dd[3]{q_1}}{(2\pi)^3\sqrt{2\omega(\vb{q}_1)}}
			\dots
			\int\frac{\dd[3]{q_m}}{(2\pi)^3\sqrt{2\omega(\vb{q}_m)}}
			\\
			&\times
			\frac{1}{\sqrt{n!}}
			f(\vb{p}_1)^*
			\dots
			f(\vb{p}_n)^*
			\frac{1}{\sqrt{m!}}
			g(\vb{q}_1)
			\dots
			g(\vb{q}_m)
			\expval{
				\hat{a}(\vb{p}_1)
				\dots
				\hat{a}(\vb{p}_n)
				\hat{a}^\dagger(\vb{q}_1)
				\dots
				\hat{a}^\dagger(\vb{q}_m)
			}{0}
			\\
			&=
			\delta_{nm}
			\int\frac{\dd[3]{p_1}}{(2\pi)^32\omega(\vb{p}_1)}
			f(\vb{p}_1)^*
			g(\vb{p}_1)
			\dots
			\int\frac{\dd[3]{p_n}}{(2\pi)^32\omega(\vb{p}_n)}
			f(\vb{p}_n)^*
			g(\vb{p}_n)
			\\
			&=
			\left(
				\int\frac{\dd[3]{p}}{(2\pi)^32\omega(\vb{p})}
				f(\vb{p})^*
				g(\vb{p})
			\right)^n
		\end{split}
	\end{equation*}
\end{delayedproof}

\subsection{Coherent state}

\begin{delayedproof}{thm:displacement_operator_normal_ordered}
	See Ref.~\cite[p.~48]{Barnett2002}.
\end{delayedproof}
\begin{delayedproof}{thm:coherent_state_annihilation_eigenvalue}
	We use the series representation of the operator exponential and \cref{th:normal_ordered_a1_cn} to move the annihilation operator to the right
	\begin{equation*}
		\begin{split}
			&
			\hat{a}(\vb{p})
			\exp\left\{
				-i
				\int\frac{\dd[3]{p}\alpha(\vb{p})}{(2\pi)^3\sqrt{2\omega(\vb{p})}}
				\hat{a}^\dagger(\vb{p})
			\right\}
			\ket{0}
			\\
			=&\
			\hat{a}(\vb{p})
			\sum_{n=0}^\infty
			\frac{1}{n!}
			\left(
				-i
				\int\frac{\dd[3]{p}\alpha(\vb{p})}{(2\pi)^3\sqrt{2\omega(\vb{p})}}
				\hat{a}^\dagger(\vb{p})
			\right)^n
			\ket{0}
			\\
			=&\
			\sum_{n=0}^\infty
			\frac{1}{n!}
			(-i)^n
			\int\frac{\dd[3]{p_1}\alpha(\vb{p}_1)}{(2\pi)^3\sqrt{2\omega(\vb{p}_1)}}
			\dots
			\int\frac{\dd[3]{p_n}\alpha(\vb{p}_n)}{(2\pi)^3\sqrt{2\omega(\vb{p}_n)}}
			\hat{a}(\vb{p})
			\prod_{j=1}^n
			\hat{a}^\dagger(\vb{p}_j)
			\ket{0}
			\\
			=&\
			\sum_{n=1}^\infty
			\frac{1}{n!}
			(-i)^n
			\int\frac{\dd[3]{p_1}\alpha(\vb{p}_1)}{(2\pi)^3\sqrt{2\omega(\vb{p}_1)}}
			\dots
			\int\frac{\dd[3]{p_n}\alpha(\vb{p}_n)}{(2\pi)^3\sqrt{2\omega(\vb{p}_n)}}
			\sum_{i=1}^n
			(2\pi)^3
			\delta^{(3)}(\vb{p}_i-\vb{p})
			\prod_{\substack{j=1\\j\neq i}}^n
			\hat{a}^\dagger(\vb{p}_j)
			\ket{0}
			\\
			=&\
			\sum_{n=1}^\infty
			\frac{1}{n!}
			(-i)^n
			\sum_{i=1}^n
			\frac{\alpha(\vb{p})}{\sqrt{2\omega(\vb{p})}}
			\prod_{\substack{j=1\\j\neq i}}^n
			\int\frac{\dd[3]{p_j}\alpha(\vb{p}_j)}{(2\pi)^3\sqrt{2\omega(\vb{p}_j)}}
			\hat{a}^\dagger(\vb{p}_j)
			\ket{0}
			\\
			=&\
			\sum_{n=1}^\infty
			\frac{1}{(n-1)!}
			(-i)^n
			\frac{\alpha(\vb{p})}{\sqrt{2\omega(\vb{p})}}
			\left(
				\int\frac{\dd[3]{p}\alpha(\vb{p})}{(2\pi)^3\sqrt{2\omega(\vb{p})}}
				\hat{a}^\dagger(\vb{p})
			\right)^n
			\ket{0}
			\\
			=&\
			\frac{\alpha(\vb{p})}{\sqrt{2\omega(\vb{p})}}
			\sum_{n=0}^\infty
			\frac{1}{n!}
			\left(
				-i
				\int\frac{\dd[3]{p}\alpha(\vb{p})}{(2\pi)^3\sqrt{2\omega(\vb{p})}}
				\hat{a}^\dagger(\vb{p})
			\right)^n
			\ket{0}
			\\
			=&\
			\frac{\alpha(\vb{p})}{\sqrt{2\omega(\vb{p})}}
			\exp\left\{
				-i
				\int\frac{\dd[3]{p}\alpha(\vb{p})}{(2\pi)^3\sqrt{2\omega(\vb{p})}}
				\hat{a}^\dagger(\vb{p})
			\right\}
			\ket{0}
		\end{split}
	\end{equation*}
	and to obtain the eigenvalue equation we are left to multiply both sides with
	\begin{equation*}
		\exp\left\{
			-
			\frac{1}{2}
			\int\frac{\dd[3]{p}}{(2\pi)^3\sqrt{2\omega(\vb{p})}}
			\abs{\alpha(\vb{p})}^2
		\right\}
		.
	\end{equation*}
\end{delayedproof}
\begin{delayedproof}{thm:coherent_state_energy_observable}
	For the first moment of the energy observable, we insert the definitions
	\begin{equation*}
		\begin{split}
			\expval{\hat{H}}{\alpha}
			&=
			\int\frac{\dd[3]{p}}{(2\pi)^3}
			\omega(\vb{p})
			\expval{\hat{a}^\dagger(\vb{p})\hat{a}(\vb{p})}{\alpha}
			\\
			&=
			\int\frac{\dd[3]{p}}{(2\pi)^3}
			\omega(\vb{p})
			\expval{\frac{\alpha(\vb{p})^*}{2\omega(\vb{p})}\frac{\alpha(\vb{p})}{2\omega(\vb{p})}}{\alpha}
			\\
			&=
			\int\frac{\dd[3]{p}}{(2\pi)^32\omega(\vb{p})}
			\omega(\vb{p})
			\abs{\alpha(\vb{p})}^2
		\end{split}
	\end{equation*}
	and use the eigenvalue equation. For the second moment, we again use the definitions and the eigenvalue equation
	\begin{equation*}
		\begin{split}
			\expval{\hat{H}^2}{\alpha}
			&=
			\int\frac{\dd[3]{p}_1}{(2\pi)^3}
			\int\frac{\dd[3]{p}_2}{(2\pi)^3}
			\omega(\vb{p}_1)
			\omega(\vb{p}_2)
			\expval{
				\hat{a}^\dagger(\vb{p}_1)
				\hat{a}(\vb{p}_1)
				\hat{a}^\dagger(\vb{p}_2)
				\hat{a}(\vb{p}_2)
			}{\alpha}
			\\
			&=
			\int\frac{\dd[3]{p}_1}{(2\pi)^3\sqrt{2\omega(\vb{p}_1)}}
			\int\frac{\dd[3]{p}_2}{(2\pi)^3\sqrt{2\omega(\vb{p}_2)}}
			\omega(\vb{p}_1)
			\omega(\vb{p}_2)
			\alpha(\vb{p}_1)^*
			\alpha(\vb{p}_2)
			\expval{
				\hat{a}(\vb{p}_1)
				\hat{a}^\dagger(\vb{p}_2)
			}{\alpha}
			\\
			&=
			\int\frac{\dd[3]{p}_1}{(2\pi)^3\sqrt{2\omega(\vb{p}_1)}}
			\int\frac{\dd[3]{p}_2}{(2\pi)^3\sqrt{2\omega(\vb{p}_2)}}
			\omega(\vb{p}_1)
			\omega(\vb{p}_2)			
			\alpha(\vb{p}_1)^*
			\alpha(\vb{p}_2)
			\\
			&\times
			\expval{
				(2\pi)^3
				\delta^{(3)}(\vb{p}_2-\vb{p}_1)
				+
				\hat{a}^\dagger(\vb{p}_2)
				\hat{a}(\vb{p}_1)
			}{\alpha}
			\\
			&=
			\int\frac{\dd[3]{p}}{(2\pi)^32\omega(\vb{p}_1)}
			\omega(\vb{p})^2
			\abs{\alpha(\vb{p})}^2
			+
			\left(
				\int\frac{\dd[3]{p}_1}{(2\pi)^32\omega(\vb{p})}
				\omega(\vb{p})
				\abs{\alpha(\vb{p})}^2
			\right)^2
		\end{split}
	\end{equation*}
\end{delayedproof}
\begin{delayedproof}{thm:coherent_state_number_observable}
	The number observable is a special case of \Cref{thm:coherent_state_energy_observable} for $\omega(\vb{p})=1$.
\end{delayedproof}
\begin{delayedproof}{thm:coherent_state_number_state_inner_product}
	Expanding the exponential series and noting that the coefficients have the same algebraic form as a $m$-particle state with spectrum $\alpha(\vb{p})$\footnote{Except for the coherent spectrum $\alpha(\vb{p})$ being unbound.}, we find
	\begin{equation*}
		\braket{n_f}{\alpha}
		=
		\sum_{m=0}^\infty
		\frac{1}{\sqrt{m!}}
		\braket{n_f}{m_\alpha}
		e^{-\overline{n}/2}
		=
		\frac{1}{\sqrt{m!}}
		\left(
			\int\frac{\dd[3]{p}}{(2\pi)^32\omega(\vb{p})}
			f(\vb{p})^*
			\alpha(\vb{p})
		\right)^n
		e^{-\overline{n}/2}
	\end{equation*}
\end{delayedproof}

\section{Interactions}

\subsection{Dynamical pictures}

\begin{delayedproof}{thm:heisenberg_schroedinger_equivalence}
	Heisenberg and Schrödinger picture are unitarily equivalent by the Stone-von Neumann theorem.
	Let $\hat{O}^H(t)$ be an observable and $\ket{\psi}^H$ a state in the Heisenberg picture, then the expectation values agree
	\begin{equation*}
		\bra{\psi}^H
		\hat{O}^H(t)
		\ket{\psi}^H
		=
		\bra{\psi(0)}^H
		e^{+i\hat{H}t}
		\hat{O}^H(0)
		e^{-i\hat{H}t}
		\ket{\psi(0)}^H
		=
		\bra{\psi(t)}^S
		\hat{O}^S
		\ket{\psi(t)}^S
	\end{equation*}
	where we used \Cref{thm:heisenberg_eom_sol} and \Cref{thm:schroedinger_eom_sol}.
\end{delayedproof}
\begin{delayedproof}{thm:heisenberg_schroedinger_canonical_transformation}
	Let $\hat{A}^H,\hat{B}^H,\hat{C}^H$ be operators in the Heisenberg picture satisfying the commutation relation
	\begin{equation*}
		\comm{\hat{A}^H}{\hat{B}^H}
		=
		\hat{C}^H
	\end{equation*}
	then in the Schrödinger picture, we find
	\begin{equation*}
		\begin{split}
			\comm{\hat{A}^S}{\hat{B}^S}
			&=
			e^{-i\hat{H}t}
			\left\{
				\hat{A}^H
				e^{+i\hat{H}t}
				e^{-i\hat{H}t}
				\hat{B}^H
				-	
				\hat{B}^H
				e^{-i\hat{H}t}
				e^{+i\hat{H}t}
				\hat{A}^H
			\right\}
			e^{+i\hat{H}t}
			\\
			&=
			e^{-i\hat{H}t}
			\comm{\hat{A}^H}{\hat{B}^H}
			e^{+i\hat{H}t}
			\\
			&=
			e^{-i\hat{H}t}
			\hat{C}^H
			e^{+i\hat{H}t}
			=
			\hat{C}^S
		\end{split}
	\end{equation*}
	from Ref.~\cite[p.~213]{Greiner2013}.
\end{delayedproof}
\begin{delayedproof}{thm:dirac_schroedinger_eom}
	See Ref.~\cite[p.~214]{Greiner2013}.
\end{delayedproof}

\subsection{Time-evolution operator}

\begin{delayedproof}{thm:time_evolution_int}
	We integrate \cref{eq:time_evolution_diff} over the interval $[t_0,t]$
	\begin{equation*}
		\int_t^{t_0}\dd{t_1}
		\hat{H}_\text{int}^D(t_1,t_0)
		\hat{U}_D(t_1,t_0)
		=
		i\int_t^{t_0}\dd{t_1}
		\partial_{t_1}
		\hat{U}_D(t_1,t_0)
		=
		i\mathbb{I}
		-
		i\hat{U}_D(t,t_0)
	\end{equation*}
	where we used the fundamental theorem of calculus and the boundary condition $\hat{U}^D(t_0,t_0)=\mathbb{I}$ in the third line.
	We are left to rearrange the terms to find \cref{eq:time_evolution_int}.
\end{delayedproof}
\begin{delayedproof}{thm:time_evolution_exp_sol}
	Use \Cref{thm:time_ordered_integral} with $\hat{A}=\hat{H}_\text{int}^D$ to rewrite the iterative integral solution \Cref{thm:time_evolution_iter_sol}.
	A proof that \cref{eq:time_evolution_sol} satisfies \cref{eq:time_evolution_diff} can be found in Ref.~\cite[p.~219]{Greiner2013}.
\end{delayedproof}

\subsection{Displacement operator from classical source interaction}

\begin{delayedproof}{thm:displacement_scattering_operator_equivalence}
	The double integral evaluates to
	\begin{equation*}
		\iint\dd[4]{x}\dd[4]{y}
		J(x)
		D(x-y)
		J(y)
		=
		\int\frac{\dd[3]{p}}{(2\pi)^32\omega(\vb{p})}
		\abs{J(\vb{p})}^2
	\end{equation*}
	see Ref.~\cite[p.~26]{Zee2010}.
	Furthermore, decomposing the Klein-Gordon operator into positive and negative frequency parts
	\begin{equation*}
		\int\dd[4]{x}
		J(x)
		\hat\phi(x)
		=
		\int\dd[4]{x}
		J(x)
		\hat\phi^+(x)
		+
		\int\dd[4]{x}
		J(x)
		\hat\phi^-(x)
		=
		\hat\phi^+[J]
		+
		\hat\phi^-[J]
		,
	\end{equation*}
	we note that the second term is the hermitian conjugate of the first.
	Similar to what we did for the single-particle number state, we rewrite the integral using the Fourier transform of the source
	\begin{equation*}
		\begin{split}
			\hat\phi^+[J]
			=
			\int\dd[4]{x}
			J(x)
			\hat\phi^+(x)
			&=
			\int\dd[4]{x}
			J(x)
			\int\frac{\dd[3]{p}}{(2\pi)^3\sqrt{2\omega(\vb{p})}}
			e^{-ip_\mu x^\mu}
			\hat{a}^\dagger(\vb{p})
			\\
			&=
			\int\frac{\dd[3]{p}}{(2\pi)^3\sqrt{2\omega(\vb{p})}}
			\left(
				\int\dd[4]{x}
				J(x)
				e^{-ip_\mu x^\mu}
			\right)
			\hat{a}^\dagger(\vb{p})
			\\
			&=
			\int\frac{\dd[3]{p}}{(2\pi)^3\sqrt{2\omega(\vb{p})}}
			J(\vb{p})
			\hat{a}^\dagger(\vb{p})
		\end{split}
		.
	\end{equation*}
	We therefore find
	\begin{equation*}
		\begin{split}
			\hat{S}
			&=
			N\exp\left\{
				\hat\phi^+[iJ]
				-
				\hat\phi^-[iJ]
			\right\}
			\exp\left\{
				\int\dd[4]{x}\dd[4]{y}
				J(x)
				D(x-y)
				J(y)
			\right\}
			\\
			&=
			\exp\left\{
				-
				\int\frac{\dd[3]{p}}{(2\pi)^3\sqrt{2\omega(\vb{p})}}
				iJ(\vb{p})
				\hat{a}^\dagger(\vb{p})
			\right\}
			\exp\left\{
				+
				\int\frac{\dd[3]{p}}{(2\pi)^3\sqrt{2\omega(\vb{p})}}
				iJ(\vb{p})
				\hat{a}^\dagger(\vb{p})
			\right\}
			\\
			&\times
			\exp\left\{
				-
				\frac{1}{2}
				\int\frac{\dd[3]{p}}{(2\pi)^32\omega(\vb{p})}
				\abs{J(\vb{p})}^2
			\right\}
			=
			\hat{D}[iJ]
		\end{split}
	\end{equation*}
	which shows that $\hat{S}=\hat{D}[iJ]$.
\end{delayedproof}

\section{Maxwell field in the Coulomb gauge}

\subsection{Lagrangian and gauge fixing}

\begin{delayedproof}{thm:mw_local_gauge_invariance}
	The physical field-strength tensor transforms under \cref{eq:mw_local_gauge_transform} as
	\begin{equation*}
		\begin{split}
			F_{\mu\nu}
			\to
			F_{\mu\nu}^\prime
			&=
			\partial_\mu\left(A_\nu+\partial_\nu\Lambda\right)
			-
			\partial_\nu\left(A_\mu+\partial_\mu\Lambda\right)
			\\
			&=
			F_{\mu\nu}
			+
			\partial_\mu\partial_\nu\Lambda
			-
			\partial_\nu\partial_\mu\Lambda
			=
			F_{\mu\nu}
		\end{split}
	\end{equation*}	
\end{delayedproof}
\begin{lemma}
	The field-strength tensor is antisymmetric $F^{\mu\nu}=-F^{\nu\mu}$.
\end{lemma}
\begin{proof}
	\begin{equation*}
		F^{\mu\nu}
		=
		\partial^\mu A^\nu
		-
		\partial^\nu A^\mu
		=
		-
		\left(
			\partial^\nu A^\mu
			-
			\partial^\mu A^\nu
		\right)
		=
		-
		F^{\nu\mu}
	\end{equation*}
\end{proof}

\subsection{Maxwell equations}

\begin{delayedproof}{thm:tensor_maxwell_equations}
	The inhomogeneous Maxwell equations are the equations of motion which are found from the Euler-Lagrange equation
	\begin{equation*}
		0
		=
		\partial_\mu
		\pdv{\mathcal{L}}{(\partial_\mu A_\nu)}
		-
		\pdv{\mathcal{L}}{A_\nu}
		=
		-
		\partial_\mu
		F^{\mu\nu}
		+
		J^\nu
	\end{equation*}
	where we used
	\begin{equation*}
		\begin{split}
			\pdv{\mathcal{L}}{(\partial_\mu A_\nu)}
			&=
			\pdv{\mathcal{L}}{F_{\alpha\beta}}
			\pdv{F_{\alpha\beta}}{(\partial_\mu A_\nu)}
			\\
			&=
			-
			\frac{1}{4}
			\pdv{(F_{\sigma\rho}F^{\sigma\rho})}{F_{\alpha\beta}}
			\pdv{(\partial_\alpha A_\beta-\partial_\beta A_\alpha)}{(\partial_\mu A_\nu)}
			\\
			&=
			-
			\frac{1}{2}
			F^{\alpha\beta}
			\left(
				\delta_\alpha^\mu
				\delta_\beta^\nu
				-
				\delta_\alpha^\nu
				\delta_\beta^\mu
			\right)
			\\
			&=
			-
			\frac{1}{2}
			F^{\mu\nu}
			+
			\frac{1}{2}
			F^{\nu\mu}
			=
			-
			F^{\mu\nu}			
		\end{split}
	\end{equation*}
	The homogeneous Maxwell equations are a consequence of the Bianchi identity and antisymmetry of $F^{\mu\nu}$.
\end{delayedproof}
\begin{delayedproof}{thm:vector_maxwell_equations}
	Evaluating the time component of \cref{eq:mw_homo_vec} yields the Gauss' law for magnetism
	\begin{equation*}
		\begin{split}
			0
			=
			\varepsilon_{0\lambda\mu\nu}\partial^\lambda F^{\mu\nu}
			&=
			\varepsilon_{0ijk}\partial^iF^{jk}
			\\
			&=
			-
			\varepsilon_{ijk}\varepsilon_{ljk}
			\partial^i B_l
			=
			2\partial_iB^i
		\end{split}
		\label{eq:mw_gauss_law_mag}
	\end{equation*}
	and the spatial component yields Ampere's circuit law
	\begin{equation*}
		\begin{split}
			0
			=
			\varepsilon_{i\lambda\mu\nu}
			\partial^\lambda
			F^{\mu\nu}
			&=
			-
			\varepsilon_{ijk}
			\varepsilon^{ljk}
			\partial_t B_l
			-
			2\varepsilon_{ijk}
			\partial^jE^k
			\\
			&=
			\partial_tB_i
			+
			\varepsilon_{ijk}
			\partial^jE_k
		\end{split}
		\label{eq:mw_ampere_law}.
	\end{equation*}
	The time component of the inhomogeneous covariant Maxwell equation \cref{eq:mw_inhomo} yields Gauss' law
	\begin{equation*}
		J^0
		=
		\rho
		=
		\partial_\mu F^{\mu\nu}
		=
		\partial_i E^i
		\label{eq:mw_gauss_law},
	\end{equation*}
	and the spatial component yields Faraday's law of induction
	\begin{equation*}
		J^i
		=
		\partial_\mu F^{\mu i}
		=
		-\partial_t E^i
		+\varepsilon^{ijk}\partial_j B_k
		\label{eq:mw_faraday_law}.
	\end{equation*}
	and we derived the vector Maxwell equations from first principles.
\end{delayedproof}
		
		\addcontentsline{toc}{section}{References}
		\printbibliography[title=References]
	\end{refsection}

\end{document}
\documentclass[
	a4paper,
	parskip,
	appendixprefix,
	chapterprefix,
	headings=big,
]{scrreprt}

\usepackage[utf8]{inputenc}
\usepackage[T1]{fontenc}
\usepackage{amsthm,amsmath,amssymb}
\usepackage{authblk}
\usepackage[english]{babel}
\usepackage[backend=biber]{biblatex}
\usepackage{booktabs}
%\usepackage{csquotes}
\usepackage[acronym,nonumberlist,toc]{glossaries}
\usepackage{hyperref}
\usepackage{graphicx}
\usepackage{cleveref}
\usepackage{physics}
\usepackage[section]{placeins}
\usepackage[separate-uncertainty=true]{siunitx}
\usepackage[mode=buildnew]{standalone}
%\usepackage{thmtools,thm-restate}
\usepackage{lmodern}
\usepackage{multirow}
%\usepackage{subcaption}
%\usepackage{wrapfig}
\usepackage{xcolor}

\addbibresource{literature.bib}

% add bibliography as section (not chapter)
% https://tex.stackexchange.com/questions/568580/make-the-bibliography-as-a-section-in-each-includ-chapter
\defbibheading{bibliography}[\bibname]{\section*{#1}}

% approximately proportional to symbol
% https://tex.stackexchange.com/questions/33538/how-to-get-an-approximately-proportional-to-symbol
\def\app#1#2{%
    \mathrel{%
        \setbox0=\hbox{$#1\sim$}%
        \setbox2=\hbox{%
            \rlap{\hbox{$#1\propto$}}%
            \lower1.1\ht0\box0%
        }%
        \raise0.25\ht2\box2%
    }%
}
\def\approxprop{\mathpalette\app\relax}

% overwrite real and imaginary part operators
\let\Re\undefined
\let\Im\undefined
\DeclareMathOperator{\Re}{\operatorname{Re}}
\DeclareMathOperator{\Im}{\operatorname{Im}}
% other functions
\DeclareMathOperator{\sinc}{\operatorname{sinc}}

\newcommand{\floor}[1]{\left\lfloor#1\right\rfloor}
\newcommand{\ceil}[1]{\left\lceil#1\right\rceil}

% transpose
% https://tex.stackexchange.com/questions/403104/small-caps-mathsf-font-for-writing-transpose-of-a-matrix
\newcommand{\trans}{{\scriptscriptstyle\mathsf{T}}}

% diagonal double arrows
% https://tex.stackexchange.com/questions/145006/nearrow-and-swarrow-together
\DeclareRobustCommand{\neswarrow}{
  \mathrel{\text{\ooalign{$\swarrow$\cr$\nearrow$}}}%
}
\DeclareRobustCommand{\nwsearrow}{
  \mathrel{\text{\ooalign{$\searrow$\cr$\nwarrow$}}}%
}

% chapter appearence
% https://tex.stackexchange.com/questions/159502/koma-script-scrreprt-how-to-change-chapter-appearance-and-produce-a-chapter-bas
\addtokomafont{chapterprefix}{\raggedleft}
\renewcommand*{\chapterformat}{%
\mbox{\chapappifchapterprefix{\nobreakspace}%
\scalebox{3}{\color{gray}\thechapter\autodot}\enskip}}

% caption label linebreak
% https://tex.stackexchange.com/questions/196889/how-can-i-insert-a-line-break-after-the-captionlabel-koma-script-options
%\setcaphanging
%\setcapmargin{2ex}
%\setcapindent{0pt}
%\setcapwidth[c]{0.8\textwidth}
%\addtokomafont{caption}{\centering}
\setkomafont{captionlabel}{\sffamily}

% theorems
\newtheorem{theorem}{Theorem}[section]
\newtheorem{lemma}[theorem]{Lemma}
\newtheorem{corollary}[theorem]{Corollary}
\theoremstyle{definition}
\newtheorem{definition}{Definition}[section]
\newtheorem{conjecture}{Conjecture}[section]
\newtheorem{example}{Example}[section]
\theoremstyle{remark}
\newtheorem*{remark}{Remark}

% prefix equation numbers with section number
\numberwithin{equation}{section}

% optics
\newacronym{ar}{AR}{anti-reflective}
\newacronym{mzm}{MZM}{Mach-Zehnder modulator}
\newacronym{mzi}{MZI}{Mach-Zehnder interferometer}
\newacronym{bs}{BS}{beam splitter}
\newacronym{fc}{FC}{fiber coupler}
\newacronym{qe}{QE}{quantum efficiency}

% physics
\newacronym{dv}{DV}{discrete-variable}
\newacronym{cv}{CV}{continuous-variable}
\newacronym{dof}{DOF}{degrees of freedom}
\newacronym{eom}{EOM}{equation(s) of motion}
\newacronym{pbc}{PBC}{periodic boundary conditions}
\newacronym{bch}{BCH}{Baker-Campbell-Hausdorff}
\newacronym{ccr}{CCR}{canonical commutation relation}

% electrical engineering
\newacronym{asp}{ASP}{analog signal processing}
\newacronym{dsp}{DSP}{digital signal processing}
\newacronym{osp}{OSP}{optical signal processing}
\newacronym{lo}{LO}{local oscillator}
\newacronym{if}{IF}{intermediate frequency}
\newacronym{lp}{LP}{low-pass}
\newacronym{tia}{TIA}{transimpedance amplifier}
\newacronym{iq}{I/Q}{in-phase/quadrature}
\newacronym{psd}{PSD}{power spectral density}
\newacronym{snr}{SNR}{signal-to-noise ratio}
\newacronym{rc}{RC}{raised-cosine}
\newacronym{rrc}{RRC}{root-raised-cosine}
\newacronym{adc}{ADC}{analog-to-digital converter}
\newacronym{dac}{DAC}{digital-to-analog converter}
\newacronym{qam}{QAM}{quadrature amplitude modulation}
\newacronym{qpsk}{QPSK}{quadrature phase-shift keying}

% cryptography
\newacronym{aes}{AES}{advanced encryption standard}
\newacronym{cvqkd}{CV-QKD}{continuous-variable quantum-key distribution}
\newacronym{dvqkd}{DV-QKD}{discrete-variable quantum-key distribution}
\newacronym{gnfs}{GNFS}{generalized number field sieve}
\newacronym{mac}{MAC}{message authentication code}
\newacronym{ldpc}{LDPC}{low-density parity-check}
\newacronym{otp}{OTP}{one-time pad}
\newacronym{pkd}{PKD}{public-key distribution}
\newacronym{qkd}{QKD}{quantum-key distribution}
\newacronym{rsa}{RSA}{Rivest–Shamir–Adleman}
\newacronym{ecdh}{ECDH}{elliptic-curve Diffie-Hellman}
\newacronym{qber}{QBER}{quantum-bit error-rate}
\newacronym{dps}{DPS}{differential-phase-shift}

\begin{document}
	\begin{titlepage}
		\begin{center}
			\large
   		    \textbf{\textsf{Experimental quantum optics}}\\
		    \vspace{0.8em}
			\huge
		    \textbf{\textsf{A theoretical framework for quantum optical communication - towards CV-QKD}}\\

		    \vspace{.8em}
			\begin{figure}[htb]
				\centering
				%\includestandalone{figures/pgfplots/phase-space}
			\end{figure}

		    \vspace{.6em}
		    \large
		    \textbf{Master thesis by}\\
   		    \vspace{.8em}
		    \large
			Bodo Kaiser\\
		    \vspace{.2em}not
			\textit{bodo.kaiser@physik.uni-muenchen.de}

		    \usekomafont{date}
		    \today

		    \vspace{1.9em}
			\normalsize
			\begin{tabular}{ll}
			Internal supervisor: & Prof. Dr. Monika Aidelsburger \\
			External supervisor: & Dr. Hans Brunner \\
			\end{tabular}
		\end{center}
	\end{titlepage}
	\tableofcontents

	\chapter{Introduction}
	\begin{refsection}
		\textit{The following chapter presents the fundamental idea of quantum-key distribution and its many facets, usually overlooked in the introductory material.}
        \section{Quantum science and technology}

We are in the acceleration stage of a second quantum revolution.
While the first quantum revolution unveiled us the - often counterintuitive - rules of quantum mechanics, the second quantum revolution breaches a new level of understanding and control of applied quantum systems.
Notable results of the last century include, for instance, the widespread adaption of the laser, the transistor, and nuclear-magnetic-resonance tomography.
Recently, the second quantum revolution gained further momentum by attracting widespread public and political interest.
Quantum technologies are seen as critical technologies for the upcoming decades, and public institutions fear losing the technological race.

% TODO: citations
% \cite{MacQuarrie2020} - The emerging commercial landscape of quantum computing
% \cite{Acin2018} - The quantum technologies roadmap: a European community view
% \cite{Downling2003} - Quantum technology: the second quantum revolution
% \cite{Alex2021} - Quantum Technologies: A Review of the Patent Landscape
% \cite{Wehner2018} - Quantum internet: A vision for the road ahead

\begin{figure}[htb]
	\centering
	\includestandalone{figures/tikz/quantum-technologies}
	\caption{Relevant quantum technologies for the European quantum flagship bla.}
\end{figure}

\begin{figure}[htb]
	\centering
	\includestandalone{figures/pgfplots/keywords}
	\caption{Keyword trends in Google Scholar}
\end{figure}

\subsection{Quantum communication}

% TODO: why QKD? (most mature of all "new" quantum technologies)

% actually quantum communication -> quantum key-distribution
% most mature quantum technology
% commercial adaption (name some examples, QKD networks)
        \section{Secure-key distribution}

Two spatially distanced parties, Alice and Bob, share a communication channel that allows them to send and receive information.
We assume the communication channel to be a one-to-one map between Alice and Bob, i.e., information is received as transmitted, for instance, by employing some error correction, e.g., LDPC codes~\cite{Gallager1962}.
How do we enable secure communication between Alice and Bob?
To promote the communication channel to be secure against an adversarial third party, Eve, we need to extend the communication protocol to assure
\begin{enumerate}
	\item \textbf{confidentiality}, and
	\item \textbf{integrity}.
\end{enumerate}
Confidentiality ensures that Eve cannot read messages and is implemented using symmetric encryption like the \gls{otp}~\cite{Shannon1949} or the more practical \gls{aes}~\cite{Daemen1999}.
Integrity ensures that Eve cannot alter messages in transit and is implemented using a \gls{mac}, e.g., universal hash functions~\cite{Carter1979}.
\begin{figure}[htb]
	\centering
	\includestandalone{figures/tikz/secure-communication}
	\caption{Secure communication between Alice and Bob using a classical channel to which an adversarial third party, Eve, has read and write access: A plaintext $p$ is encrypted with a secret key $k$ through a symmetric encryption function $e(k,p)$ yielding a ciphertext $c$. The ciphertext is amended by a hashing function $h(k,c)$ yielding the message $m$. The message is sent through an error-free channel to Bob. Bob checks the integrity of the message using his copy of the secret key $k$ and the hash inside the message $m$. If he finds the message $m$ to be genuine, he removes the hash to obtain the ciphertext $c$ which he decrypts using his key $k$ and the decryption function $d(k,c)$. After decryption, Bob has access to the plaintext $p$ initially prepared by Alice.}\label{fig:secure_communication}
\end{figure}
\Cref{fig:secure_communication} depicts a communication between Alice and Bob secure against a third adversarial party, Eve.
The plaintext is first encrypted and then appended a \gls{mac} as advised by Ref.~\cite{Kohno2003,Krawczyk2001,Bellare2000}.

Given a truly random and secret key $k$ shared between Alice and Bob such a system has been proven to be eternal secure when using the \gls{otp} encryption scheme~\cite{Shannon1949}.
However, one needs to employ a practical mechanism to generate and distribute the key securely.

\FloatBarrier
\subsection{Public-key distribution}

The standard attempt to solve the secret key distribution problem is to use an asymmetric cipher.
Asymmetric ciphers like RSA, comprise a public key for encryption and a private key for decryption.
Using such an asymmetric cipher, secret key distribution between Alice and Bob involves the following steps:
\begin{enumerate}
	\item Bob generates a public and a private key.
	\item Alice generates a secret key.
	\item Bob discloses the public key to Alice.
	\item Alice encrypts the secret key with the public key and sends it to Bob.
	\item Bob decrypts Alice secret key with the private key.
\end{enumerate}
Assuming the asymmetric cipher to be secure, Alice and Bob now share a secret key.
\begin{figure}[htb]
	\centering
	\includestandalone{figures/tikz/pkd-system}
	\caption{\Gls{pkd} system used to create a shared key between between two spatially distanced parties, Alice and Bob.}\label{fig:pkd_system}
\end{figure}
Public-key distribution algorithms, for instance, Diffie-Hellman~\cite{Diffie1976} and variants thereof, e.g., \gls{ecdh}, are heavily employed in todays internet for key agreement.
One of the advantages of public-key distribution is that it is software-defined, allowing for easy installation and upgrades.
The core principle behind asymmetric ciphers is the concept of one-way functions:
Functions that are easy to compute but difficult to invert.
A typical class of one-way functions appears in prime number factorization: Given two prime numbers $p$ and $q$, it is easy to compute the product $pq$ but difficult to factorize $pq$ into its prime numbers $p$ and $q$.
Here, easy and difficult refer to the computational complexity, which gives an upper bound of the best (known) algorithm to solve the problem.
So far, there has been no mathematical proof that one-way functions are indeed computational complex, and it may be that we are just not yet aware of efficient algorithms placing asymmetric cryptography at all in a questionable position.
With the advent of quantum computers, new algorithms with significant speed up to their classical counterpart were found.
Shor, for example, presented an algorithm for prime number factorization in polynomial time~\cite{Shor1994} - the same task used by the Diffie-Hellman key exchange.
\begin{figure}[htb]
	\centering
	\includestandalone{figures/pgfplots/runtime}
	\caption{Computational runtimes for prime number factorization algorithms: The most efficient known classical algorithm is the \gls{gnfs}~\cite{Lenstra1993} and Shor's algorithm~\cite{Shor1994}.}
\end{figure}
Consequently, post-quantum algorithms have been proposed~\cite{Bernstein2017,Chen2016} to use a different class of one-way functions where no efficient (quantum) algorithm is (yet) known.
As long as there is no proof for the computational complexity for a specific family of one-way functions, public-key distribution is only safe for key lengths that are not computationally feasible.
However, what is not computationally feasible today might be tomorrow, and it is conceivable that encrypted communication recorded today will be broken in the feature with specialized hardware.
Therefore, public-key distribution is not considered to be forward secure.

\FloatBarrier
\subsection{Quantum-key distribution}

The chance of public-key distribution becoming a critical vulnerability in secure communication infrastructure inflamed a demand for a practical and information-theoretical provable alternative key distribution.
\gls{qkd} claims to be such a practical and information-theoretical secure key distribution.
In summary, \gls{qkd} exploits the inherent uncertainty in measuring non-orthogonal quantum states for shared random number generation:
\begin{enumerate}
	\item Alice samples some random variables $\alpha_1,\dots,\alpha_n$ and encodes them onto a quantum state $\ket{\psi(t)}$.
	\item Bob performs measurements of the quantum state $\ket{\psi(t)}$ yielding $\beta_1,\dots,\beta_n$.
	\item Alice's $\alpha_1,\dots,\alpha_n$ and Bob's $\beta_1,\dots,\beta_n$ are correlated random variables from which a shared secret key can be distilled using classical algorithms (post-processing).
\end{enumerate}
On a more practical level, a typical \gls{qkd} system comprises a transmitter, a receiver, a quantum and classical communication channel as illustrated in \Cref{fig:qkd_system}.
\begin{figure}[htb]
	\centering
	\includestandalone{figures/tikz/qkd-system}
	\caption{\Gls{qkd} system used to create a shared key between between two spatially distanced parties, Alice and Bob, secret to a third party, Eve. Eve has full access to the quantum channel but can only read information from the classical authenticated channel.}\label{fig:qkd_system}
\end{figure}
A potential adverse eavesdropper, Eve, has full access to the quantum channel, i.e., can intercept and temper with quantum states sent by Alice, but has only read access to the classical channel.
More precisely, the classical channel is an authenticated channel to prevent man-in-the-middle attacks.
Authentication can be implemented classically using \gls{mac} as discussed for the secure communication system.
% assumptions to Eve and what kind of attacks she can do (coherent, ?)
% explain superoperator (effect of quantum channel)

% entanglement- vs. prepare-and-measure -> EB used for proofs but otherwise PM used practical
% comparison cv/dv on a quantum level (possible quantum states) and on a practical level (fiber, free-space) -> CV and DV depends on state space of the receiver
\begin{table}[htb]
	\centering
	\begin{tabular}{lll}
		\toprule
		Encoding variable & State space & Information support \\
		\midrule
		Polarization & Finite & Polarization state \\
		Photon number & Countable & Number state \\
		Quadratures & Uncountable & Coherent state \\
		Squeezing & Uncountable & Squeezed state \\
		Time-bin & Finite & Time of arrival \\
		Phase-bin & Finite & Phase \\
		\bottomrule
	\end{tabular}
	\caption{Encoding variables to use for \gls{qkd}: Variables with uncountable state space are considered for \gls{cvqkd} while variables with (possibly restricted) countable state space are considered for \gls{dvqkd}.}
\end{table}

\subsection{Taxonomy of QKD protocols}

Before diving into specific \gls{qkd} protocols, we introduce a taxonomy of \gls{qkd} protocols.
By identifying common protocol features, allows for a reductionist model of \gls{qkd}.

The literature typically distinguishes between \gls{dvqkd}, \gls{cvqkd}, and \gls{dpsqkd}.
Ignoring \gls{dpsqkd}, it is unclear what exact features unambiguously differentiate between \gls{cvqkd} and \gls{dvqkd}.
For instance, most practical \gls{dvqkd} protocols use weak coherent states~\cite{Duvsek2006}, which are anything but discrete.
Fr this reason, the accepted opinion considers a protocol discrete when using a single-photon and continuous when using a coherent detector.
This view has been recently challenged by proposing a BB84-like protocol using coherent detection~\cite{Qi2021}.
\begin{figure}[htb]
	\centering
	\includestandalone[scale=0.8]{figures/tikz/qkd-classification}
	\caption{Common features among \gls{qkd} protocols: Detection, physical encoding, logical state space, measurement basis selection and schema.}\label{fig:qkd_classification}
\end{figure}
We propose a more subtle classification based on definite protocol features.
\Cref{fig:qkd_classification} illustrates common features we identified among \gls{qkd} protocols that can be uniquely determined.
Many of these features are implementation details such as physical encoding, measurement basis selection, and detection.
In contrast, other features are more fundamental, like the logical Hilbert space or the schema.
We will discuss and exemplify these features in the next two sections.

In the next two sections, we present \gls{qkd} protocols that differ by their logical Hilbert space.
The logical Hilbert space defines the quantum system assumed on an abstract protocol level, e.g., a qubit or boson, and is a more precise concept than discrete and continuous~\cite[p.~2]{Weedbrook2012}.
The logical quantum system does not need to be equal to the quantum system physically encoding the quantum states, for example, a weak coherent state mimicking a single-photon.
        \section{Problem statement}

% no complete and connected description of DV-QKD (because not discussed much in academia)
% standard quantum optics formalism neither time-dependent nor frequency continuous
% signal-processing uses continuous frequency spectras while quantum optics uses single frequency modes
% no complete fully-connected description of a practical QKD device reported so far
% intersections of different disciplines: signal-processing, quantum optics, quantum information theory, quantum field theory
        \documentclass[tikz]{standalone}

%\usepackage{amsmath}
%\usepackage{physics}

%\usetikzlibrary{arrows.meta,backgrounds,fit,positioning}

\begin{document}
	\begin{tikzpicture}[
		block/.style={draw, very thick, fill=white, minimum height=8ex, minimum width=3.5em},
	]
		\coordinate (in) at (0,0);
		
	\end{tikzpicture}
\end{document}

        \section{Conventions and notation}

% Minkowski space, four vectors
% why we use p instead of omega for modes -> to distinguish between frequency and momentum

\begin{align}
	f(t)
	=
	\int_{\mathbb{R}}\frac{\odif{\omega}}{2\pi}
	f(\omega)
	e^{+i\omega t}
	&&
	f(\omega)
	=
	\int_{\mathbb{R}}\odif{t}
	f(t)
	e^{-i\omega t}
\end{align}
\begin{align}
	f(\vb{x})
	=
	\int_{\mathbb{R}^3}\frac{\odif[order={3}]{p}}{(2\pi)^3}
	f(\vb{p})
	e^{-i\vb{p}\vdot\vb{x}}
	&&
	f(\vb{p})
	=
	\int_{\mathbb{R}^3}\odif[order={3}]{x}
	f(\vb{x})
	e^{+i\vb{p}\vdot\vb{x}}
\end{align}

We use $\hbar=1$, i.e., vacuum quadrature variance is equal to $1/2$, see Ref.~\cite[p.~4]{Weedbrook2012}.
	
		\addcontentsline{toc}{section}{References}
		\printbibliography[title=References]
	\end{refsection}
	
	\chapter{Quantum-key distribution}
	\begin{refsection}
		The present chapter identifies common characteristics of \gls{qkd} protocols and attempts to formalize the notion of a \gls{qkd} protocol.
We test the proposed formula with qubit-based \gls{qkd} protocols like BB84 and the six-state protocol as well as Gaussian \gls{cvqkd} protocols.
The chapter ends with a summary and literature review regarding practical \gls{qkd} protocols' post-processing and security analysis.

\Cref{fig:qkd_protocol} illustrates our proposed notion of a \gls{qkd} protocol, with the feature being a logical quantum system from which the random bits are encoded and decoded.
The logical quantum system is a subspace of the physical quantum system.
The physical quantum system depends strongly on the physical implementation and quantum encoding.
\begin{figure}[htb]
	\centering
	\includestandalone{figures/tikz/qkd-protocol}
	\caption{A \gls{qkd} protocol comprises a binary encoder, a logical quantum system, and a binary decoder. The binary encoder maps bits $\vb{b}\in\{0,1\}^n$ onto a quantum state of the logical quantum system $\ket{\psi}$. The binary decoder extracts the bits $\vb{b}$ back from the quantum state $\ket{\psi^\prime}$. The logical quantum system is a subspace of a larger physical quantum system. The state encoder and decoder map between the logical and physical quantum states.}\label{fig:qkd_protocol}
\end{figure}
The distinction between logical and quantum systems is vital to separate the implementation and security concerns.
Many security proofs show equivalence between the physical implementation and the logical system to use an established security proof.
One should keep in mind that such a separation implicitly assumes no loopholes from the particular implementation.

\Cref{fig:qkd_classification} illustrates common features among the \gls{qkd} protocols.
For an overview of \gls{qkd} protocols, see Ref.~\cite{Duvsek2006}.
\begin{figure}[htb]
	\centering
	\includestandalone{figures/tikz/qkd-classification}
	\caption{Common features among \gls{qkd} protocols: Detection, physical encoding, logical state space, measurement basis selection and schema.}\label{fig:qkd_classification}
\end{figure}
Every \gls{qkd} system requires a detector, e.g., a coherent detector or a single-photon (click) detector.
The detection does not necessarily imply the dimension of the logical state space as BB84 has been implemented with coherent detection~\cite{Qi2021}.
Concerning measurement basis selection, Bob can either actively choose a random measurement basis for every transmission or passively measure all (orthogonal) bases for measurement basis selection.
We will cover both active and passive measurement basis selection in the discussion of the polarization-encoding BB84 protocol.
Finally, the \gls{qkd} schema determines if either Alice prepares a state and sends it to Bob for measurement (prepare-and-measure) or if Alice and Bob share an entangled state (entanglement-based).
Most practical \gls{qkd} implementations use prepare-and-measure.
On a theoretical level, both schemas are equivalent, and security proofs are often more convenient in an entanglement-based setting.	
		\section{Qubit-based protocols}

Many \gls{dvqkd} protocols, e.g., the BB84~\cite{Bennett1984}, BB92, or the six-state protocol~\cite{Bechmann1999}, are qubit-based in that the logical quantum system underlying the key generation is a two-state quantum system, a qubit.

A qubit state $\ket{\psi}$ is an element of a two-dimensional complex Hilbert space with norm one, i.e., $\abs{\braket{\psi}{\psi}}^2=1$.
In the qubit basis $\{\ket{0},\ket{1}\}$, a generic qubit state takes the form
\begin{align}
	\ket{\psi}
	=
	c_1\ket{0}
	+
	c_2\ket{1}
	&&
	\text{with}\
	\abs{c_1}^2
	+
	\abs{c_2}^2
	=
	1
	.
\end{align}
\begin{table}[htb]
	\centering	
	\begin{tabular}{lcc}
		\toprule
		\multirow{2}{*}{Encoding variable} & \multicolumn{2}{c}{Standard basis} \\
		\cmidrule{2-3}
		& $\ket{0}$ & $\ket{1}$ \\
		\midrule
		Polarization & Horizontal & Vertical \\
		Photon number & Vacuum & Single-photon \\
		Time-bin & Early & Late \\
		Phase-bin & \SI{0}{\deg} & \SI{180}{\deg} \\
		\bottomrule
	\end{tabular}
	\caption{Possible physical systems to encode a qubit with possible choices for the standard basis elements.}\label{tab:qubit_encodings}
\end{table}
\Cref{tab:qubit_encodings} lists different quantum systems which allow encoding of a qubit.
To encode a qubit the actual quantum systems does not have to be two-dimensional.
For example, the photon Fock space is countable.
Still, by restricting the basis elements to the vacuum and single-photon state, we have a qubit.
Similar, we can partition the continuous time and phase parameters of a quantum system to separate bins.

A useful visualization of qubit states is the Bloch sphere, see \Cref{fig:bloch_sphere}.
\begin{figure}[htb]
	\centering
	\includegraphics{figures/tikz/bloch-sphere}
	\caption{Two-state quantum system in the Bloch sphere representation. The Bloch sphere is a three-dimensional sphere with unit radius. A pure quantum state  $\ket{\psi}$ resides on its surface of a sphere.}\label{fig:bloch_sphere}
\end{figure}
The Bloch sphere is a unit sphere embedded in three-dimensional space.
Pure quantum states are elements on the surface of the Bloch sphere.
Two antiparallel vectors correspond to orthogonal states.
Typically, the standard axis in $\mathbb{R}^3$ are assigned to the three orthogonal Pauli eigenbases.
A generic quantum state $\ket{\psi}$ in a certain basis can be found by projection.
\Cref{fig:state_space_qubit} shows the projection among the $X$ and $Z$ Pauli eigenbases.
\begin{figure}[htb]
	\centering
	\includegraphics{figures/pgfplots/state-space-qubit}
	\caption{Two-dimensional state space spanned by the $X,Z$ Pauli eigenbases: Projecting the $\ket{x_-}$ state onto the $Z$ eigenbasis yields a constant probability amplitude of $1/\sqrt{2}$.}\label{fig:state_space_qubit}
\end{figure}
In quantum mechanics, the projection coefficients, i.e., the inner product of two states, correspond to the probability amplitude of measuring the given state in particular basis.
We can formalize this concept by introducing the generalized spin operator
\begin{equation}
	\hat{S}(\vu{n})
	=
	\vb{\hat{S}}\vdot\vu{n}
	=
	\frac{1}{2}
	\hat{\sigma}_jn^j
	=
	\begin{pmatrix}
		n_3 & n_1 - in_2 \\
		n_1 + in_2 & -n+3
	\end{pmatrix}
\end{equation}
wherein $\vu{n}\in\mathbb{R}^3$ is a unit norm vector and $\hat{\sigma}_j$ is the $j$th Pauli matrix.
Let $\ket{\pm,\vu{n}}$ be the eigenstate of the generalized spin operator $\hat{S}(\vu{n})$ to eigenvalues $\pm1/2$, i.e.,
\begin{equation}
	\hat{S}(\vu{n})
	\ket{\pm,\vu{n}}
	=
	\pm\frac{1}{2}
	\ket{\pm,\vu{n}},
\end{equation}
then for $\vu{n}=\vu{e}_j$ we obtain the $\hat{\sigma}_j$ Pauli eigenstates, i.e.,
\begin{align}
	\hat{S}_x
	\ket{x_\pm}
	=	
	\vb{\hat{S}}(\vu{e}_x)
	\ket{\pm,\vu{e}_x}
	&=
	\pm\frac{1}{2}
	\ket{\pm,\vu{e}_x}
	=
	\pm\frac{1}{2}
	\ket{x_\pm}
	\\
	\hat{S}_y
	\ket{y_\pm}
	=	
	\vb{\hat{S}}(\vu{e}_y)
	\ket{\pm,\vu{e}_y}
	&=
	\pm\frac{1}{2}
	\ket{\pm,\vu{e}_y}
	=
	\pm\frac{1}{2}
	\ket{y_\pm}
	\\
	\hat{S}_z
	\ket{z_\pm}
	=	
	\vb{\hat{S}}(\vu{e}_z)
	\ket{\pm,\vu{e}_z}
	&=
	\pm\frac{1}{2}
	\ket{\pm,\vu{e}_z}
	=
	\pm\frac{1}{2}
	\ket{z_\pm}
	.
\end{align}
By convention one identifies the $Z$ Pauli eigenbasis with the qubit basis $\left\{\ket{0},\ket{1}\right\}$.

Having introduced the concept of basis projections and the spin operator, we can discuss the BB84 (six state) protocol for which Alice and Bob must agree on two (or three) orthogonal bases\footnote{BB92 using non-orthogonal bases can be implemented by using the generalized spin operator with non-orthogonal vectors.} and a mapping between the basis states and some bit sequence, then
\begin{enumerate}
	\item Alice encodes her bits into the state $\ket{\psi}$ and sends it to Bob.
	\item Bob receives the state $\ket{\psi}$ from Alice and performs a measurement decoding some bits.
\end{enumerate}
If Alice and Bob select the same basis, Bob can accurately decode Alice's key bit from the measurement.
Alice and Bob's probability of choosing the same basis for one transmission is one divided by the number of orthogonal bases Alice and Bob have agreed on, e.g., \SI{50}{\percent} if Alice and Bob agreed to use the $X$ and $Z$ Pauli eigenbasis, also called the \gls{qber}.
In the asymptotic limit of many transmissions, the \gls{qber} should approach the theoretical limit.
Otherwise, an opposing third party, Eve, might have tempered with the transmission.
\Cref{tab:qubit_transmission_sequence} displays a possible transmission sequence between Alice and Bob.
Alice randomly selects an initial key bit \num{0} or \num{1} and a state basis $X$ or $Z$ where $X$ respective $Z$ denote the eigenbasis of the Pauli $\sigma_x$ respective $\sigma_z$ matrix. Alice's initial key bit and selected basis determine the quantum state she prepares and sends to Bob. Bob randomly chooses a measurement basis. Only if Alice's and Bob's basis agree, the key bit is not discarded.
\begin{table}[htb]
	\centering
	\begin{tabular}{llccccc}
		\toprule
		& & \multicolumn{5}{c}{Transmission} \\
		\cmidrule{3-7}
		Party & Step & 1 & 2 & 3 & 4 & 5 \\ 
		\midrule
		\multirow{3}{*}{Alice} & Initial key bit & \num{0} & \num{1} & \num{1} & \num{0} & \num{0} \\
		& State basis & $Z$ & $X$ & $X$ & $Z$ & $X$ \\
		& Prepared state & $\ket{z_+}$ & $\ket{x_-}$ & $\ket{x_-}$ & $\ket{z_+}$ & $\ket{x_+}$ \\
		\cmidrule{1-1}
		\multirow{3}{*}{Bob} & Measurement basis & $X$ & $Z$ & $X$ & $Z$ & $Z$ \\
		& Possible outcomes & \num{0},\num{1} & \num{0},\num{1} & \num{1} & \num{0} & \num{0},\num{1} \\
		& Sifted outcomes & - & - & 1 & 0 & - \\
		\bottomrule
	\end{tabular}
	\caption{Possible transmission sequence for qubit-based \gls{qkd} illustrating how Alice encodes a key bit into a qubit state and Bob attempt to decode.}\label{tab:qubit_transmission_sequence}
\end{table}
After the transmission sequence, Alice and Bob hold a partially correlated and partially secret bit string from which they can distill a shared secret bit string using classical post-processing.

\FloatBarrier
\subsection{Polarization-encoding BB84}

In polarization-encoding BB84, the polarization of light is used as physical quantum system to encode the logical qubit system.
Let $\ket{\leftrightarrow}$ and $\ket{\updownarrow}$ denote the horizontal respective vertical polarization states forming the rectilinear basis.
Let $\ket{\nwsearrow}$ and $\ket{\neswarrow}$ denote the left- and right-diagonal polarization states forming the diagonal basis.
Let $\ket{\acwopencirclearrow}$ and $\ket{\cwopencirclearrow}$ denote the left- and right-circular polarization states forming the circular basis.
We can express the diagonal and circular basis elements in terms of the rectilinear basis elements:
\begin{align}
	\ket{\nwsearrow}
	&=
	\frac{1}{\sqrt{2}}
	\left(
		\ket{\leftrightarrow}
		+
		\ket{\updownarrow}
	\right)
	&
	\ket{\neswarrow}
	&=
	\frac{1}{\sqrt{2}}
	\left(
		\ket{\leftrightarrow}
		-
		\ket{\updownarrow}
	\right)
	\\
	\ket{\acwopencirclearrow}
	&=
	\frac{1}{\sqrt{2}}
	\left(
		\ket{\leftrightarrow}
		+
		i\ket{\updownarrow}
	\right)
	&
	\ket{\cwopencirclearrow}
	&=
	\frac{1}{\sqrt{2}}
	\left(
		\ket{\leftrightarrow}
		-
		i\ket{\updownarrow}
	\right)
\end{align}
For clarity, we restrict the following discussion to qubit-based \gls{qkd} protocols where two orthogonal bases are used, e.g., rectilinear and diagonal.
Other protocols exist that use three orthogonal bases (six-state protocol) or even non-orthogonal bases.

A possible optical setup to implement such polarization-encoding is depicted in \Cref{fig:polarization_encoding_active}.
Alice configures her linear polarizer to select a basis element of the rectilinear or diagonal polarization basis. Bob receives Alice's polarization state through the polarization controlled quantum channel. He rotates a rectilinear polarized beam splitter by either \SI{0}{\degree} or \SI{45}{\degree} to detect either rectilinear or diagonal polarized photons with his two single-photon detectors placed at the beam splitter output.
\begin{figure}[htb]
	\centering
	\includegraphics{figures/pstricks/bb84-polarization-active}
	\caption{Optical setup to implement polarization-encoding BB84 with active basis selection. The transmitter comprises an \gls{sps} and a polarizer connected to a fiber. The receiver comprises a \gls{pc}, a \gls{fc}, a rotatable \gls{pbs} with two \gls{spd}s at its outputs.}\label{fig:polarization_encoding_active}
\end{figure}
Alice selects one of four polarization states $\ket{\leftrightarrow},\ket{\updownarrow},\ket{\nwsearrow},\ket{\neswarrow}$ by adjusting her linear polarizer to one of four angles $\theta\in\{0,\pi,\pi/2,3\pi/2\}$.
We can write Alice's state as
\begin{equation}
	\ket{\theta}
	=
	\frac{1}{\sqrt{2}}
	\left(
		\ket{\acwopencirclearrow}
		+
		e^{i\theta}
		\ket{\cwopencirclearrow}
	\right)
	.
\end{equation}
Unrotated, Bob's rectilinear polarized beam splitter monitored by two single-photon detectors is equivalent to the \gls{povm} for detecting rectilinear-polarized light
\begin{equation}
	\biggl\{
		\hat{P}_{\leftrightarrow}
		=
		\ketbra{\leftrightarrow},
		\;
		\hat{P}_{\updownarrow}
		=
		\ketbra{\updownarrow}
	\biggr\}
	.
\end{equation}
Rotated by \SI{45}{\degree}, Bob's rectilinear polarized beam splitter monitored by two single-photon detectors is equivalent to the \gls{povm} for detecting diagonal-polarized light
\begin{equation}
	\biggl\{
		\hat{P}_{\nwsearrow}
		=
		\ketbra{\nwsearrow},
		\;
		\hat{P}_{\neswarrow}
		=
		\ketbra{\neswarrow}
	\biggr\}
	.
\end{equation}
Instead of Bob actively selecting the measurement basis, he can passively let the quantum randomness decide by splitting the photon with an unpolarized beam splitter towards a rectilinear and diagonal polarization detector.
\Cref{fig:polarization_encoding_passive} shows an optical setup implementing polarization-encoding BB84 with passive measurement basis selection.
While Alice's transmitter setup is unchanged to the previous setup, Bob now has two polarization detectors.
One polarization detector comprises a rectilinear-polarized beam splitter and two single-photon detectors.
Another polarization detector comprises a diagonal-polarized beam splitter.
\begin{figure}[htb]
	\centering
	\includegraphics{figures/pstricks/bb84-polarization-passive}
	\caption{Optical setup to implement polarization-encoding BB84 with passive basis selection. The transmitter comprises a \gls{sps} and a polarizer connected to a fiber. The receiver comprises a \gls{pc}, a \gls{fc}, an unpolarized \gls{bs}, a rectilinear \gls{pbs} with two \gls{spd}s at the outputs, and a diagonal \gls{pbs} with two \gls{spd}s at the outputs. The receiver connects with the fiber through the \gls{pc}. The \gls{fc} couples the output of the \gls{pc} with the \gls{bs} which splits the beam among the two \gls{pbs}s.}\label{fig:polarization_encoding_passive}
\end{figure}
The \gls{povm} describing Bob's measurement with passive basis selection is
\begin{equation}
	\biggl\{
		\frac{1}{2}\hat{P}_{\leftrightarrow},
		\;
		\frac{1}{2}\hat{P}_{\updownarrow},
		\;
		\frac{1}{2}\hat{P}_{\nwsearrow},
		\;
		\frac{1}{2}\hat{P}_{\neswarrow}
	\biggr\}
	.
\end{equation}
Bob may still have inconclusive measurements.
For instance, if he receives a horizontal polarization state $\ket{\leftrightarrow}$ and the photon chooses the path towards the diagonal polarization detector, the clicks among the two single-photon detectors are equally distributed.
\begin{table}[htb]
	\centering
	\begin{tabular}{cccc}
		\toprule
		Prepared & Measurement & \multicolumn{2}{c}{Click probability} \\
		state & basis & $p_1$ & $p_2$ \\
		\midrule
		\multirow{2}{*}{$\ket{\updownarrow}$} & Rectilinear & \SI{100}{\percent} & \SI{0}{\percent} \\
		& Diagonal & \SI{50}{\percent} & \SI{50}{\percent} \\
		\cmidrule{1-1}
		\multirow{2}{*}{$\ket{\leftrightarrow}$} & Rectilinear & \SI{0}{\percent} & \SI{100}{\percent} \\
		& Diagonal & \SI{50}{\percent} & \SI{50}{\percent} \\
		\cmidrule{1-1}
		\multirow{2}{*}{$\ket{\nwsearrow}$} & Rectilinear & \SI{50}{\percent} & \SI{50}{\percent} \\
		& Diagonal & \SI{100}{\percent} & \SI{0}{\percent} \\
		\cmidrule{1-1}
		\multirow{2}{*}{$\ket{\neswarrow}$} & Rectilinear & \SI{50}{\percent} & \SI{50}{\percent} \\
		& Diagonal & \SI{0}{\percent} & \SI{100}{\percent} \\
		\bottomrule
	\end{tabular}
	\caption{Click probabilities for the polarization-encoding BB84 with active measurement basis selection depending on the states Alice prepared and the measurement bases Bob selected.}\label{tab:bb84_polarization_clicks}
\end{table}
The polarization of light is a qubit and we can simply relabel the polarization states with the Pauli eigenstates, i.e.,
\begin{align}
	\ket{\nwsearrow}
	&=
	\ket{x_+}
	&
	\ket{\acwopencirclearrow}
	&=
	\ket{y_+},
	&
	\ket{\leftrightarrow}
	&=
	\ket{z_+},
	\\
	\ket{\neswarrow}
	&=
	\ket{x_-}
	&
	\ket{\cwopencirclearrow}
	&=
	\ket{y_-}
	&
	\ket{\updownarrow}
	&=
	\ket{z_-}
\end{align}
to show equivalence to the general qubit description.

\FloatBarrier
\subsection{Time-phase-encoding BB84}

In the following, we discuss the practical time-phase-encoding BB84 protocol and show its equivalence to the polarization-encoding BB84.
The idea of using phase-encoding was first proposed as part of the BB92 protocol~\cite{Bennett1992}.
The basic setup is illustrated in \Cref{fig:qubit_time_phase_active}:
Alice creates an entangled photon state using a first \gls{mzi} with phase $\theta\in\{0,\pi/2,\pi,3\pi/2\}$ and sends it to Bob. Bob detects the photon state using a second \gls{mzi} with phase $\phi\in\{0,\pi/2\}$ and two single-photon detectors monitoring the outputs.
\begin{figure}[htb]
	\centering
	\includegraphics{figures/pstricks/bb84-time-phase-active}
	\caption{Fiber-optical setup of the phase-encoding BB84 using active basis selection. The transmitter comprises a \gls{sps} and a first asymmetric \gls{mzi} with variable phase-shift $\theta$. The transmitter is connected to the receiver through a fiber. The receiver comprises a second asymmetric \gls{mzi} with variable phase-shift $\phi$. The first and second \gls{mzi} are both made of two fiber couplers with a variable phase-shifter in the longer optical path. The outputs of the second asymmetric \gls{mzi} is monitored by two \gls{spd}s.}\label{fig:qubit_time_phase_active}
\end{figure}

To understand the time-phase encoding, we analyze the action of the asymmetric \gls{mzi} with variable phase $\varphi$ on a photon pulse $\ket{t_0}$ arriving at time $t_0$, see \Cref{fig:mzi_asymmetric}.
\begin{figure}[htb]
    \centering
    \includegraphics{figures/pstricks/mzi-asymmetric}
     \caption{Asymmetric \gls{mzi} adding a constant time delay and variable phase difference between the upper and lower path. A pulsed state enters the first \gls{bs}, BS1, to the left and is split among a longer upper path and a shorter lower path. A first mirror M1 directs the pulse from the upper path to a phase shifter which adds a relative phase of $\varphi$ between the upper and lower path. A second mirror M2 directs the pulse from the phase shifter to a second \gls{bs}, BS2, while the lower path is between BS1 and BS2.}\label{fig:mzi_asymmetric}
\end{figure}
An ideal (lossless) and symmetric beam splitter transforms the single-photon input states into a superposition according to\footnote{See Ref.~\cite[p.~137]{Haroche2006} and Ref.~\cite[p.~143]{Gerry2005}}
\begin{align}
	\hat{U}_\text{BS}
	\ket{1,0}
	&=
	\frac{1}{\sqrt{2}}
	\left(\ket{1,0}+i\ket{0,1}\right)
	\\
	\hat{U}_\text{BS}
	\ket{0,1}
	&=
	\frac{1}{\sqrt{2}}
	\left(i\ket{1,0}+\ket{0,1}\right)
	.
\end{align}
Then, the first beam splitter BS1 in \Cref{fig:mzi_asymmetric} (instantly) splits a photon pulse $\ket{t_0}$ arriving at $t_0$ into the superposition
\begin{equation}
	\hat{U}_\text{BS}
	\ket{t_0,0}
	=
	\frac{1}{\sqrt{2}}
	\left(\ket{t_0,0}+i\ket{0,t_0}\right)
\end{equation}
where the first mode corresponds to the upper and the second mode to the lower optical path in \Cref{fig:mzi_asymmetric}.
The phase shifter adds a relative phase of $\varphi$ between the upper and lower path and the input state to the second beam splitter BS2 is
\begin{equation}
	\hat{U}_\text{PS}
	\hat{U}_\text{BS}
	\ket{t_0,0}
	=
	\frac{1}{\sqrt{2}}
	\left(
		\ket{t_0+\tau,0}
		+
		ie^{i\varphi}
		\ket{0,t_0+\tau+\Delta\tau}
	\right)
\end{equation}
wherein $\tau$ is the time delay the pulse accumulates over the short upper path and $\Delta\tau$ is the difference in time delay between the shorter, lower and longer, upper path.
The output state of BS2 is equal to the action of the \gls{mzi}
\begin{equation}
	\begin{split}
		\hat{U}_\text{MZM}
		\ket{t_0,0}
		&=
		\hat{U}_\text{BS}
		\hat{U}_\text{PS}
		\hat{U}_\text{BS}
		\ket{t_0,0}
		\\
		&=
		\frac{1}{2}
		\biggl[
			\left(
				\ket{t_0+\tau,0}
				+
				i\ket{0,t_0+\tau}
			\right)
			+
			ie^{i\varphi}
			\left(
				i\ket{t_0+\tau+\Delta\tau,0}
				+
				\ket{0,t_0+\tau+\Delta\tau}
			\right)
		\biggr]
		\\
		&=
		\frac{1}{2}
		\biggl[
			\ket{t_0+\tau,0}
			-
			e^{i\varphi}
			\ket{t_0+\tau+\Delta\tau,0}
			+
			i
			\left(
				\ket{0,t_0+\tau}
				+
				e^{i\varphi}
				\ket{0,t_0+\tau+\Delta\tau}
			\right)
		\biggr]
		.
	\end{split}
	\label{eq:mzi_asymmetric}
\end{equation}
Ignoring the vacuum state, we project the product state in \cref{eq:mzi_asymmetric} onto each of the output modes and obtain
\begin{align}
	\ket{t_1,\phi}_1
	&=
	\frac{1}{\sqrt{2}}
	\left(
		\ket{t_1}
		-
		e^{i\varphi}
		\ket{t_1+\Delta\tau}
	\right)
	\label{eq:mzi_asymmetric_right}
	\\
	\ket{t_1,\phi}_2
	&=
	\frac{i}{\sqrt{2}}
	\left(
		\ket{t_1}
		+
		e^{i\varphi}
		\ket{t_1+\Delta\tau}
	\right)
	\label{eq:mzi_asymmetric_down}
	,
\end{align}
wherein $t_1=t_0+\tau$.

Back to the time-phase-encoding BB84 setup depicted in \Cref{fig:qubit_time_phase_active}, we note that Alice's transmitter consists of a single-photon source and an asymmetric \gls{mzi} where one output is dumped.
Therefore, Alice's states are parametrized by the relative phase $\theta$,
\begin{equation}
	\ket{t_0,\theta}
	=
	\frac{1}{\sqrt{2}}
	\left(
		\ket{t_0}
		-
		e^{i\theta}
		\ket{t_0+\Delta\tau}
	\right)
	,
\end{equation}
which equals the first output mode of the \gls{mzi}, \cref{eq:mzi_asymmetric_right}, adapting the new time reference $t_1\to t_0$.
If Bob receives a pulse with time delay $\Delta\tau$ at some time $t_1$, i.e., $\ket{t_1+\Delta\tau}$, then his \gls{mzi} provides the two detectors with the states
\begin{align}
	\ket{t_1+\Delta\tau,\phi}_1
	&=
	\frac{1}{\sqrt{2}}
	\left(
		\ket{t_1+\Delta\tau}
		-
		e^{i\phi}
		\ket{t_1+2\Delta\tau}
	\right)
	\\
	\ket{t_1+\Delta\tau,\phi}_2
	&=
	\frac{i}{\sqrt{2}}
	\left(
		\ket{t_1+\Delta\tau}
		+
		e^{i\phi}
		\ket{t_1+2\Delta\tau}
	\right)
	.
	\label{eq:qubit_time_phase_bob}
\end{align}
We note that these are superpositions of states at three different time instances $0,\Delta\tau,2\Delta\tau$.
We drop the pulse time and introduce the state
\begin{equation}
	\ket{\Delta\tau=m}
	=
	\ket{t_1+m\Delta\tau}
\end{equation}
corresponding to the $m$th detection time slot.
In the new notation, Bob's detectors receive a superposition of Alice's states, \cref{eq:qubit_time_phase_bob} and \cref{eq:qubit_time_phase_bob_delayed},
\begin{equation}
	\ket{\theta,\phi}_\pm
	=
	\frac{c_\pm(\theta-\phi)}{\sqrt{2}}
	\biggl[
		\ket{\Delta\tau=0}
		\mp
		\left(
			e^{i\phi}
			\pm
			e^{i\theta}
		\right)
		\ket{\Delta\tau=1}
		\pm
		e^{i(\phi+\theta)}
		\ket{\Delta\tau=2}
	\biggr]
	\label{eq:qubit_time_phase_bob_delayed}
\end{equation}
with phase-dependent normalization constant
\begin{equation}
	c_\pm(\theta-\phi)
	=
	\frac{1}{\sqrt{2\pm\cos(\theta-\phi)}}
	.
\end{equation}
Using the \gls{povm} for detecting a click at time slot $m$,
\begin{equation}
	\left\{
		\hat{P}_m
		=
		\ketbra{\Delta\tau=m}
	\right\}_{m=0,1,2}
	,
\end{equation}
we find the click probabilities for the Bob's detectors to equal
\begin{equation}
	\begin{split}
		p_{\pm,m}
		&=
		\trace{\hat\rho_\pm\hat{P}_m}
		=
		\expval{\hat{P}_m}{\theta,\phi}_{\pm}
		\\
		&=
		\begin{cases}
			\abs{c_\pm(\theta-\phi)}^2\left(1\pm\cos(\theta-\phi)\right) & m=1 \\
			\abs{c_\pm(\theta-\phi)}^2\frac{1}{2} & m=0,2
		\end{cases}
		.
	\end{split}
\end{equation}
If we configure the detectors to trigger only on the $m=1$ time slot, we find the probability for a click of the plus and minus detectors to be
\begin{equation}
	p_\pm(\theta-\phi)
	=
	\frac{1}{2}
	\left[
		1
		\pm
		\cos(\theta-\phi)
	\right]
	.
\end{equation}
\Cref{tab:bb84_time_phase_clicks} summarizes the click probability of the plus and minus detectors triggered on the $m=1$ time slot for a restricted choice of phases:
Alice choosing $\theta\in\{0,\pi\}$ corresponds to choosing the $Z$ eigenbasis while $\theta\in\{\pi/2,3\pi/2\}$ corresponds to her choosing the $X$ eigenbasis. Bob using no phase shift $\phi=0$ corresponds to a selection of the $X$ basis while Bob adding a phase shift of $\phi=\pi/2$ corresponds to selection of $X$ as measurement basis. Only if Alice and Bob choose the same basis, Bob's click is perfectly correlated with Alice's choice for a basis element. Otherwise, it is completely random.
\begin{table}[htb]
	\centering
	\begin{tabular}{cccc}
		\toprule
		\multicolumn{2}{c}{Phase} & \multicolumn{2}{c}{Detector click probability} \\
%		\cmidrule{1-2}
%		\cmidrule{3-4}
		$\theta$ & $\phi$ & $p_1(\theta-\phi)$ & $p_2(\theta-\phi)$ \\
		\midrule
		\multirow{2}{*}{$0$} & $0$ & \SI{100}{\percent} & \SI{0}{\percent} \\
		& $\pi/2$ & \SI{50}{\percent} & \SI{50}{\percent} \\
		\cmidrule{1-1}
		\multirow{2}{*}{$\pi$} & $0$ & \SI{0}{\percent} & \SI{100}{\percent} \\
		& $\pi/2$ & \SI{50}{\percent} & \SI{50}{\percent} \\
		\cmidrule{1-1}
		\multirow{2}{*}{$\pi/2$} & $0$ & \SI{50}{\percent} & \SI{50}{\percent} \\
		& $\pi/2$ & \SI{100}{\percent} & \SI{0}{\percent} \\
		\cmidrule{1-1}
		\multirow{2}{*}{$3\pi/2$} & $0$ & \SI{50}{\percent} & \SI{50}{\percent} \\
		& $\pi/2$ & \SI{0}{\percent} & \SI{100}{\percent} \\
		\bottomrule
	\end{tabular}
	\caption{Click probabilities for the time-phase-encoding BB84 protocol depending on the \gls{mzi} phase angles set by Alice and Bob.}\label{tab:bb84_time_phase_clicks}
\end{table}
Comparing the click probabilities of \Cref{tab:bb84_time_phase_clicks} with the probabilities of the qubit-based BB84, suggests equivalence of the time-phase-encoding BB84 with the more general qubit-based description of BB84.
While we can identify Alice's state in the time basis in terms of the $Y$ qubit basis,
\begin{equation}
	\ket{t_0,\theta}
	=
	\frac{1}{\sqrt{2}}
	\left(
		\ket{t_0}
		-
		e^{i\theta}
		\ket{t_0+\Delta\tau}
	\right)
	=
	\frac{1}{\sqrt{2}}
	\left(
		\ket{y_+}
		-
		e^{i\theta}
		\ket{y_-}
	\right)
	=
	\ket{\theta}
	,
\end{equation}
the receiver side cannot simply be relabeled into the qubit-based description:
Bob's Hilbert space, spanned by the three time slot states, $\ket{\Delta\tau=m}_{m=0,1,2}$, has one additional dimension compared to the qubit Hilbert space.
Such complication can be addressed using "squashing"~\cite{Beaudry2008,Gittsovich2014}:
We first find a unitary transformation for the input mode of Bob's receiver.
Second, we show that the \gls{povm} yields the same probability distribution as the qubit-based description for all possible quantum states.
The number state basis $\left\{\ket{n}\right\}_{n\in\mathbb{N}_0}$ is complete and countable allowing a proof by induction.
It is important to show equivalence for all number states as Eve's not limited to the single-photon state.
		\section{Gaussian-based protocols}

In \gls{cvqkd}, the receiver performs a continuous-value measurement, i.e., the set of possible outcomes is uncountable.
In comparison to \gls{dvqkd}, \gls{cvqkd} generates random bits faster but with higher error.
Grosshans initially proposed \gls{cvqkd} in 2002 as an alternative to DV-QKD~\cite{Grosshans2002}.
Today, CV-QKD has grown into solid competition with DV-QKD~\cite{Diamanti2016}.
One significant advantage of CV-QKD over DV-QKD is that the CV-QKD receiver uses standard telecommunication components allowing for cheap high-performance integration.

\subsection{Coherent-encoding}

The generalized quadrature operator in the single-mode quantum optics is~\cite[p.~36]{Barnett2002}
\begin{equation}
	\hat{X}(\theta)
	=
	\frac{1}{\sqrt{2}}
	\left(
		\hat{a}
		e^{-i\theta}
		+
		\hat{a}^\dagger
		e^{+i\theta}
	\right)
	.
\end{equation}
For a single-mode coherent state $\ket{\alpha}$, we find mean expectation value
\begin{equation}
	\expval{\hat{X}(\theta)}{\alpha}
	=
	\frac{1}{\sqrt{2}}
	\left(
		\alpha
		e^{-i\theta}
		+
		\alpha^*
		e^{+i\theta}
	\right)
	=
	\sqrt{2}
	\Re\left(
		\alpha
		e^{-i\theta}
	\right)
	=
	\sqrt{2}
	\abs{\alpha}
	\cos(\phi-\theta)
\end{equation}
where we used the complex polar representation $\alpha=\abs{\alpha}e^{i\phi}$ and variance~\cite[p.~59]{Barnett2002}
\begin{equation}
	\expval{\left(\Delta\hat{X}(\theta)\right)^2}{\alpha}
	=
	\frac{1}{2}
	.
\end{equation}

\begin{figure}[htb]
	\centering
	\includestandalone{figures/pgfplots/gaussian-coherent-default}
	\caption{Fiber-optical setup of the phase-encoding \gls{cvqkd} protocol:}
\end{figure}

% relevance of Gaussian RV to CV-QKD?

% citations:
% \cite{Weedbrook2012,Ferraro2005} Gaussian quantum information (quantum channel, cv-qkd, measurements)
% \cite{Lodewyck2007} complete description of CV-QKD device (theoretical and experimental)

% measurement of a Gaussian variable (1d)
% protocol table
% post-processing

\begin{figure}[htb]
	\centering
	\includestandalone{figures/pstricks/gaussian-coherent-transmitter}
	\caption{Fiber-optical setup of the phase-encoding \gls{cvqkd} protocol:}
\end{figure}

\begin{figure}[htb]
	\centering
	\includestandalone{figures/pstricks/gaussian-coherent-receiver-active}
	\caption{Fiber-optical setup of the phase-encoding \gls{cvqkd} protocol:}
\end{figure}

\begin{figure}[htb]
	\centering
	\includestandalone{figures/pstricks/gaussian-coherent-receiver-passive}
	\caption{Fiber-optical setup of the phase-encoding \gls{cvqkd} protocol:}
\end{figure}

% dual homodyne

% tensor product input state
		\section{Post-processing}
\begin{figure}[htb]
	\centering
	\includestandalone{figures/tikz/post-processing}
	\caption{First, the raw data from the quantum transmission is partitioned into data for calibration and data for key generation. The parameter estimation estimates an upper bound of Eve's information and the channel characteristics from the raw calibration data. If Eve's information exceeds a certain threshold, the protocol is aborted, or the current data frame is discarded. Symbol mapping, including basis sifting, transforms the raw key data to correlated key data. Information reconciliation corrects errors in the correlated key data and discards data where error correction failed. Eve's information on the partially secret key is reduced to epsilon using privacy amplification. Finally, Alice and Bob verify that the post-processing was successful by comparing a hash of their secret key. If the hashes mismatch, the protocol is aborted, or the transmission block discarded.}\label{fig:post_processing}
\end{figure}

% citations
\cite{Silberhorn2002} % post-selection mechanism to mitigate beam splitter attack

\FloatBarrier
\subsection{Symbol mapping}

\cite{Leverrier2008} % multidimensional (sphere) 
\cite{Elkouss2011} % simpler reconciliation scheme

\FloatBarrier
\subsection{Information reconciliation}

Information reconciliation summarizes methods required for Alice and Bob to agree on shared data.
It includes error correction, and discarding of data failed to correct.

Let us first consider procedures for error correction.
Error correction is a subdiscipline of coding theory, or more precisely, channel coding, which studies the arrangement of data for efficient and reliable transmission, see \Cref{fig:error_correction_codes}.
The following discussion is a very brief introduction to binary linear codes based of Ref.~\cite{MacKay2003,Mildenberger2013}.
\begin{figure}[htb]
	\centering
	\includestandalone{figures/tikz/error-correction-codes}
	\caption{Taxonomy of codes in coding theory with emphasis on linear block codes for error correction: The linear block codes are distinguished by the constraints on their generator matrix.}\label{fig:error_correction_codes}
\end{figure}
A $(n,k)$ binary linear code encodes $k$-bit messagewords into $n$-bit codewords.
The additional $n-k$ check bits are used to detect and correct errors, e.g., bit flips.
In general, it is impossible to correct for all errors although practical linear block codes closely approach the theoretical (Shannon) limit set by the noisy-channel coding theorem.

Let $\vb{m}\in\{0,1\}^{1\times k}$ be a messageword, then the generator matrix
\begin{equation}
	G
	=
	\begin{pmatrix}[c|c]
		I_k & P
	\end{pmatrix}
	=
	\begin{pmatrix}[cccc|cccc]
		1 & 0 & \cdots & 0 & p_{1,1} & p_{1,2} & \cdots & p_{1,m} \\
		0 & 1 & \cdots & 0 & p_{2,1} & p_{2,2} & \cdots & p_{2,m}\\
		\vdots & \vdots  & \ddots & \vdots & \vdots & \vdots & \ddots & \vdots \\
		0 & 0 & \cdots & 1 & p_{n-m,1} & p_{n-m,2} & \cdots & p_{n-m,m} \\
	\end{pmatrix}
	\in\{0,1\}^{k\times n}
	,
\end{equation}
wherein $I_k\in\{0,1\}^{k\times k}$ denotes an identity matrix and $P\in\{0,1\}^{k\times(n-k)}$ denotes the (parity) check matrix,
encodes the messageword $\vb{m}$ into the codeword
\begin{equation}
	\vb{x}
	=
	\vb{m}G
	\in\{0,1\}^{1\times n}
\end{equation}
with the matrix multiplication being defined on the binary field $\mathbb{F}_2$\footnote{Alternatively, we can define the addition and multiplication on the real field with modulo two.}.
The explicit form of the generator matrix depends on the linear block code.
For instance, the $(n,1)$ repetition code has the generator matrix
\begin{equation}
	G
	=
	\begin{pmatrix}
		1 & 1 & \cdots & 1
	\end{pmatrix}
	\in\{0,1\}^{1\times n}
\end{equation}
and the $(7,4)$ Hamming code has the generator matrix
\begin{equation}
	G
	=
	\begin{pmatrix}
		1 & 0 & 0 & 0 & 1 & 1 & 0 \\
		0 & 1 & 0 & 0 & 1 & 0 & 1 \\
		0 & 0 & 1 & 0 & 0 & 1 & 1 \\
		0 & 0 & 0 & 1 & 1 & 1 & 1 \\
	\end{pmatrix}
	\in\{0,1\}^{4\times 7}
	.
\end{equation}
For the post-processing, we use \gls{ldpc}~\cite{Gallager1962} where the generator matrix is a sparse random matrix.
\Cref{tab:repetition_codewords} and \Cref{tab:hamming_codewords} show the possible codewords for the $(3,1)$ repetition and $(7,4)$ Hamming code.
From \Cref{tab:repetition_codewords}, we note that the repetition code repeats the message word $n-k$ times.
The $(n,1)$ repetition code is able to detect all bit errors except the error when all bits are flipped and correct up to $\floor{(n-1)/2}$ bit errors~\cite[p.~5]{MacKay2003}.
\begin{table}[htb]
	\centering
	\begin{tabular}{c|c|cc}
		\toprule
		Nr. & Information & \multicolumn{2}{c}{Check} \\
		\midrule
			1 & 0 & 0 & 0 \\
			2 & 1 & 1 & 1 \\
		\bottomrule
	\end{tabular}
	\caption{Possible codewords for $(3,1)$ repetition code.}\label{tab:repetition_codewords}
\end{table}
The $(7,4)$ Hamming code is a more efficient block code which uses parity checks to detect and correct single-bit errors~\cite[p.~10]{MacKay2003}.
\begin{table}[htb]
	\centering
	\begin{tabular}{c|cccc|ccc}
		\toprule
		Nr. & \multicolumn{4}{c}{Information} & \multicolumn{3}{c}{Check} \\
		\midrule
			1 & 0 & 0 & 0 & 0 & 0 & 0 & 0 \\
			2 & 0 & 0 & 0 & 1 & 1 & 1 & 1 \\
			3 & 0 & 0 & 1 & 0 & 1 & 1 & 0 \\
			4 & 0 & 0 & 1 & 1 & 0 & 0 & 1 \\
			5 & 0 & 1 & 0 & 0 & 1 & 0 & 1 \\
			6 & 0 & 1 & 0 & 1 & 0 & 1 & 0 \\
			7 & 0 & 1 & 1 & 0 & 0 & 1 & 1 \\
			8 & 0 & 1 & 1 & 1 & 1 & 0 & 0 \\
			9 & 1 & 0 & 0 & 0 & 0 & 1 & 1 \\
			10 & 1 & 0 & 0 & 1 & 1 & 0 & 0 \\
			11 & 1 & 0 & 1 & 0 & 1 & 0 & 1 \\
			12 & 1 & 0 & 1 & 1 & 0 & 1 & 0 \\
			13 & 1 & 1 & 0 & 0 & 1 & 1 & 0 \\
			14 & 1 & 1 & 0 & 1 & 0 & 0 & 1 \\
			15 & 1 & 1 & 1 & 0 & 0 & 0 & 0 \\
			16 & 1 & 1 & 1 & 1 & 1 & 1 & 1 \\
		\bottomrule
	\end{tabular}
	\caption{Possible codewords for $(7,4)$ Hamming code~\cite[p.~109]{Mildenberger2013}.}\label{tab:hamming_codewords}
\end{table}
The received codeword $\vb{y}$ of a linear channel equals the sent codeword $\vb{x}$ plus a noise (row) vector $\vb{n}$, i.e.,
\begin{equation}
	\vb{y}
	=
	\vb{x}
	+
	\vb{n}
	\in\{0,1\}^n
	.
\end{equation}
The noise vector introduces bit flips according to an assumed channel model, for example, the binary symmetric channel where a single bit flip occurs with probability $p$, see, e.g., Ref.~\cite[p.~148]{MacKay2003}.
To detect errors of a received codeword $\vb{y}$, one uses the parity-check matrix
\begin{equation}
	H
	=
	\begin{pmatrix}[c|c]
		-P^\trans & I_{n-k}
	\end{pmatrix}
	=
	\begin{pmatrix}
		p_{1,1} & p_{2,1} & \cdots & p_{n-m,1} & 1 & 0 & \cdots & 0 \\
		p_{1,2} & p_{2,2} & \cdots & p_{n-m,2} & 0 & 1 & \cdots & 0 \\
		\vdots & \vdots & \ddots & \vdots & \vdots & \vdots & \ddots & \vdots \\
		p_{1,m} & p_{2,m} & \cdots & p_{n-m,m} & 0 & 0 & \cdots & 1 \\
	\end{pmatrix}
	\in\{0,1\}^{(n-k)\times n}
\end{equation}
where we used $-p_{i,j}=p_{i,j}$ for elements of the binary field $p_{i,j}\in\mathbb{F}_2$.
The parity-check matrix is orthogonal to the generator matrix~\cite[p.~95]{Mildenberger2013}, i.e.,
\begin{equation}
	GH^\trans
	=
	\vb{0}
	=
	HG^\trans
	.
\end{equation}
The orthogonality between generator and parity-check matrix implies that for the received codeword $\vb{y}$, the parity-check matrix yields, a binary vector called the syndrom (column) vector
\begin{equation}
	\vb{s}
	=
	H\vb{y}^\trans
	=
	HG^\trans\vb{m}^\trans
	+
	H\vb{n}^\trans
	=
	H\vb{n}^\trans
\end{equation}
which only depends on the noise (row) vector $\vb{n}$.
If the block code does not detect any error, we have $\vb{s}=0$.
\begin{table}[htb]
	\centering
	\begin{tabular}{c|cccc|ccc}
		\toprule
		Nr. & \multicolumn{4}{c}{Received codeword} & \multicolumn{3}{c}{Syndrome} \\
		\midrule
			1 & 0 & 0 & 0 & 0 & 0 & 0 & 0 \\
			2 & 0 & 0 & 0 & 1 & 0 & 0 & 1 \\
			3 & 0 & 0 & 1 & 0 & 0 & 1 & 0 \\
			4 & 0 & 0 & 1 & 1 & 0 & 1 & 1 \\
			5 & 0 & 1 & 0 & 0 & 1 & 0 & 0 \\
			6 & 0 & 1 & 0 & 1 & 1 & 0 & 1 \\
			7 & 0 & 1 & 1 & 0 & 1 & 1 & 0 \\
			8 & 0 & 1 & 1 & 1 & 1 & 1 & 1 \\
			9 & 1 & 0 & 0 & 0 & 1 & 1 & 1 \\
			10 & 1 & 0 & 0 & 1 & 1 & 1 & 0 \\
			11 & 1 & 0 & 1 & 0 & 1 & 0 & 1 \\
			12 & 1 & 0 & 1 & 1 & 1 & 0 & 0 \\
			13 & 1 & 1 & 0 & 0 & 0 & 1 & 1 \\
			14 & 1 & 1 & 0 & 1 & 0 & 1 & 0 \\
			15 & 1 & 1 & 1 & 0 & 0 & 0 & 1 \\
			16 & 1 & 1 & 1 & 1 & 0 & 0 & 0 \\
		\bottomrule
	\end{tabular}
	\caption{Possible syndroms for the $(3,1)$ repetition code: The first and last row are correct codewords (without noise) and yield a zero syndrome indicating no error. All other received codewords contain bit flips and thereby non-zero syndroms.}\label{tab:repetition_syndroms}
\end{table}
\Cref{tab:repetition_syndroms} lists the possible syndroms for the $(3,1)$ repetition code.
To correct the error, one looks up the calculated syndrom in an error correction table.
For example, \Cref{tab:hamming_correction} lists the bit-error corrections assigned to each syndrom of the $(7,4)$ Hamming code.
\begin{table}[htb]
	\centering
	\begin{tabular}{cccccccc}
		\toprule
		Syndrome $\vb{s}$ & $001$ & $010$ & $011$ & $100$ & $101$ & $110$ & $111$ \\
		Unflip bit & $y_7$ & $y_6$ & $y_4$ & $y_5$ & $y_1$ & $y_2$ & $y_3$ \\
		\bottomrule
	\end{tabular}
	\caption{Bit-error correction lookup table for the $(7,4)$ Hamming code~\cite[p.~11]{MacKay2003}.}\label{tab:hamming_correction}
\end{table}

So far, we have implicitly assumed the (parity) check bits to be transmitted together with the data bits such that errors can be directly detected and corrected (forward error correction).
Forward error correction does not make sense for \gls{qkd} as the data bits are a result of the post-processing and not directly transmitted.
Instead of exchanging the check bits together with a data index as part of the error correction in \gls{qkd}, one instead directly calculates the syndromes and transmits these depending on the implementation details.

There cannot exist a perfect error correction protocol as it is always possible that bits are flipped such that a different valid codeword is received.
However, we have no use of data blocks where error correction have failed and we can simply discard them.
To identify these data blocks, Alice and Bob exchange hashes of their data blocks to verify success of the error correction.

\FloatBarrier
\subsection{Privacy amplification}

The final step of most \gls{qkd} protocols is privacy amplification which removes Eve's information from the key.
More formally, let us assume a key of length $n$, $s\in\{0,1\}^n$, and that Eve has partial information over the key equivalent to $k$ bits.
For privacy amplification, Alice and Bob need to agree on a map $f\colon\{0,1\}^n\to\{0,1\}^r$ with $r\leq n-k$ which extracts Eve's information from the key.

Bennett and Brassard proposed privacy amplification four years after BB84 in 1988.
An information-secure proof that privacy amplification is possible was published one year later, now known as the leftover-hash-lemma~\cite{Impagliazzo1989} and later extended to the quantum leftover-hash-lemma~\cite{Renner2005}.
In 1995, Bennett and Brassard generalized the privacy amplification outside of \gls{qkd}~\cite{Bennett1995}.

In practice, privacy amplification is performed by randomly XOR the bits of a key.
Let us illustrate this using a partially secret key $(s_1,s_2,s_3)\in\{0,1\}^3$ where Eve knows the value of $s_3$.
After privacy amplification the initial key could for example $(s_1\oplus s_3,s_2\oplus s_3)$ or $(s_1\oplus s_2,s_2\oplus s_3)$ and Eve's knowledge of the key bit $s_3$ does not help her infering a bit of the new secret key.

For further understanding, it may be helpful to consider the following Gedankenexperiment:
Let us assume Alice and Bob generate a key by flipping a coin $n$ times and assigning heads to zero and tails to one.
Eve having no information about the key corresponds to Alice and Bob using an unbiased coin yielding heads and tails with equal probability.
Eve having full information about the key corresponds to Alice and Bob using a biased coin always yielding, e.g., heads.
Therefore, we can use the probability of the coin yielding heads, $p\in[0,1]$, to indicate Eve's information.
If we now start to XOR all outcomes, we see the probability for obtaining either a one or zero to approach $1/2$.
More precisely, let $X_1,X_2,\dots,X_N$ be \gls{iid} Bernoulli random variables with probability $p$, $X_i\sim\text{Bern}(p)$, then the probability that XOR of all outcomes equals one is
\begin{equation}
	\mathbb{P}\left[
		\text{"XOR of outcomes equals one"}
	\right]
	=
	\mathbb{P}\left[
		\sum_{n=1}^NX_n\bmod2
		=
		1
	\right]
	.
\end{equation}
The sum of \gls{iid} Bernoulli random variables $X_1,\dots,X_n$ equals a Binomial random variable $Y=\sum_{n=1}^NX_n\sim\text{Binom}(N,p)$, yielding
\begin{equation}
	\mathbb{P}\left[
		Y\bmod2
		=
		1
	\right]
	=
	\sum^N_{k\text{ odd}}
	\mathbb{P}\left[Y=k\right]
	=
	\sum^N_{k\text{ odd}}
	\binom{N}{k}
	p^k(1-p)^{N-k}
	\label{eq:xoring_prop}
	.
\end{equation}
Using the identity from Ref.~\cite{stackexchange2528974}, we can simplify \cref{eq:xoring_prop} to
\begin{equation}
	\mathbb{P}\left[
		Y\bmod2
		=
		1
	\right]
	=
	\frac{1}{2}
	\left(
		1-(2p-1)^N
	\right)
	\xrightarrow{N\to\infty}
	\frac{1}{2}
\end{equation}
where limit is true for $p\in]0,1[$ so even for $p=0.9999$.
Moreover, for any $\varepsilon>0$ we can choose $N$ such that $\mathbb{P}\left[Y\bmod2=1\right]-1/2<\varepsilon$, meaning that we can arbitrarily reduce Eve's information by XORing more coin flips.

For practical applications, privacy amplification operates on the class of linear maps,
\begin{equation}
	\begin{split}
		f\colon\{0,1\}^n&\to \{0,1\}^{n-k}\\
		\vb{s}&\mapsto M\vb{s}\bmod{2}
	\end{split}
\end{equation}
where $M\in\{0,1\}^{r\times n}$ is a random boolean matrix with at least $k+1$ ones.
For construction of $M$ it is convenient to use Toeplitz matrices.
A $n\times m$ Toeplitz matrix $A$ has the form
\begin{equation}
	A
	=
	\begin{pmatrix}
		a_{i,j}
	\end{pmatrix}
	=
	\begin{pmatrix}
		a_{i-j}
	\end{pmatrix}
	=
	\begin{pmatrix}
		a_0 & a_{-1} & \cdots & a_{1-m} \\
		a_1 & a_0 & \cdots & a_{2-m} \\
		\vdots & \ddots & \cdots & \vdots \\
		a_{n-1} & a_{n-2} & \cdots & a_{n-m} \\
	\end{pmatrix}
	.
\end{equation}
For example, if Alice and Bob estimate Eve to know three bits, they could agree on the Toeplitz coefficients $(a_{-4},a_{-3},\dots,a_1,a_2)=(0,1,1,1,0,1,0)$ yielding the mapping
\begin{equation}
	f(\vb{s})
	=
	\begin{pmatrix}
		0 & 1 & 1 & 1 & 0 \\
		1 & 0 & 1 & 1 & 1 \\
		0 & 1 & 0 & 1 & 1 \\
	\end{pmatrix}
	\begin{pmatrix}
		0 \\
		1 \\
		1 \\
		0 \\
		1 \\
	\end{pmatrix}
	\bmod{2}
\end{equation}
which for the key $\vb{s}=(0,1,1,0,1)$ would yield the secret ket $\vb{s}^\prime=(0,0,0)$.
		\section{Security analysis}

% assumptions about Eve
% outline security proof for entanglement-based QKD
% equivalence of CV-QKD with entangelement-based
% upper and lower error bounds

\begin{figure}[htb]
	\centering
	\includestandalone{figures/tikz/qkd-eve}
	\caption{Eavesdroppers Eve access to the quantum and authenticated classical channel.}
\end{figure}


		\addcontentsline{toc}{section}{References}
		\printbibliography[title=References]
	\end{refsection}

	\chapter{Classical transmission system}
	\begin{refsection}
		The present chapter identifies common characteristics of \gls{qkd} protocols and attempts to formalize the notion of a \gls{qkd} protocol.
We test the proposed formula with qubit-based \gls{qkd} protocols like BB84 and the six-state protocol as well as Gaussian \gls{cvqkd} protocols.
The chapter ends with a summary and literature review regarding practical \gls{qkd} protocols' post-processing and security analysis.

\Cref{fig:qkd_protocol} illustrates our proposed notion of a \gls{qkd} protocol, with the feature being a logical quantum system from which the random bits are encoded and decoded.
The logical quantum system is a subspace of the physical quantum system.
The physical quantum system depends strongly on the physical implementation and quantum encoding.
\begin{figure}[htb]
	\centering
	\includestandalone{figures/tikz/qkd-protocol}
	\caption{A \gls{qkd} protocol comprises a binary encoder, a logical quantum system, and a binary decoder. The binary encoder maps bits $\vb{b}\in\{0,1\}^n$ onto a quantum state of the logical quantum system $\ket{\psi}$. The binary decoder extracts the bits $\vb{b}$ back from the quantum state $\ket{\psi^\prime}$. The logical quantum system is a subspace of a larger physical quantum system. The state encoder and decoder map between the logical and physical quantum states.}\label{fig:qkd_protocol}
\end{figure}
The distinction between logical and quantum systems is vital to separate the implementation and security concerns.
Many security proofs show equivalence between the physical implementation and the logical system to use an established security proof.
One should keep in mind that such a separation implicitly assumes no loopholes from the particular implementation.

\Cref{fig:qkd_classification} illustrates common features among the \gls{qkd} protocols.
For an overview of \gls{qkd} protocols, see Ref.~\cite{Duvsek2006}.
\begin{figure}[htb]
	\centering
	\includestandalone{figures/tikz/qkd-classification}
	\caption{Common features among \gls{qkd} protocols: Detection, physical encoding, logical state space, measurement basis selection and schema.}\label{fig:qkd_classification}
\end{figure}
Every \gls{qkd} system requires a detector, e.g., a coherent detector or a single-photon (click) detector.
The detection does not necessarily imply the dimension of the logical state space as BB84 has been implemented with coherent detection~\cite{Qi2021}.
Concerning measurement basis selection, Bob can either actively choose a random measurement basis for every transmission or passively measure all (orthogonal) bases for measurement basis selection.
We will cover both active and passive measurement basis selection in the discussion of the polarization-encoding BB84 protocol.
Finally, the \gls{qkd} schema determines if either Alice prepares a state and sends it to Bob for measurement (prepare-and-measure) or if Alice and Bob share an entangled state (entanglement-based).
Most practical \gls{qkd} implementations use prepare-and-measure.
On a theoretical level, both schemas are equivalent, and security proofs are often more convenient in an entanglement-based setting.
		\section{Transmitter}                                                                                                                                                                                                                                                                                                                                                                                                                                                                                                                                                                                                                                                                                                                     

We introduced the transmitter as a component encoding a sequence of complex symbols, $\left\{\alpha_n\in\mathbb{C}\colon n\in I\right\}$, onto a coherent state $\ket{\alpha(t)}$.
Efficient transmission through the channel and effective receiver detection impose additional constraints on the space of useful coherent states.
Together with practical considerations, these constraints lead to the particular design embodiment of the transmitter we will discuss.

First and foremost, the channel and receiver limit the spectrum of useful coherent states.
For instance, the receiver has limited bandwidth to detect the signal with signal power outside that bandwidth being lost.
Apart from that, the physical channel only shows favorable transmission properties over a certain frequency range, outside the signal is strongly suppressed and distorted.\footnote{For instance, the C-band, spanning wavelengths from \SI{1530}{\nano\meter} to \SI{1565}{\nano\meter}, is widely deployed for optical telecommunication.}
Additionally, different users may jointly use the same physical channel, and using the available bandwidth efficiently while reducing interference between users, requires the signal bandwidth to be minimal.
\begin{figure}[htb]
	\centering
	\includegraphics{figures/tikz/baseband-passband}
	\caption{Power spectrum showing the relationship between a real-valued baseband and passband signal. Both base- and passband signals have bandwidth $B$. The baseband signal is centered at $\omega=0$. The passband is shifted by $\omega_0$ and has a conjugate symmetric counterpart at $-\omega_0$.}\label{fig:baseband_passband}
\end{figure}
To illustrate the signal and channel bandwidth, we introduce the concept of baseband and passband signals\cite[p.~26]{Madhow2008}.
\Cref{fig:baseband_passband} depicts the power spectrum of a baseband and passband signal each with bandwidth $B$.
The spectrum of the baseband signal is centered at zero frequency, $\omega=0$, while the spectrum of the passband signal is located at $\pm\omega_0$.
For efficient use of the limited channel and receiver bandwidth, we want to minimize $B$ while keeping the \gls{snr} high.
In addition, we need to shift the baseband spectrum to an optical frequency $\omega_0$ for which the channel shows desirable transmission characteristics.
So from a signal processing point of view, we want the transmitter to
\begin{enumerate}
	\item first create a baseband signal with minimum bandwidth $B$ and
	\item then transfer it to a passband signal in the optical domain.
\end{enumerate}
In the following, we term the first step signal synthesis and the second step up-conversion.

Nowadays, the signal is almost exclusively constructed in the digital domain, and the analog part is limited to the digital-to-analog conversion.
Constructing the signal digitally allows for greater flexibility in the development process as the synthesis is mostly software-defined.
For up-conversion of the synthesized signal to the optical domain, we modulate the electric signal onto an optical carrier.
\begin{figure}[htb]
	\centering
	\includegraphics{figures/tikz/software-defined-transmitter}
	\caption{Block diagram of the transmitter's signal processing domains. The \gls{dsp} transforms a complex symbol sequence $\{\alpha_n\}_{n\in I}$ into two digital signals, $x^\prime[m]$ and $p^\prime[m]$, corresponding to the real and imaginary part. The \gls{dac} converts the digital signals to analog signals, $x(t)$ and $p(t)$ we modulate onto a coherent state $\ket{\alpha(t)}$.}\label{fig:software_defined_transmitter}
\end{figure}
\Cref{fig:software_defined_transmitter} illustrates how such a software-defined transmitter architecture applies to our coherent state transmission system.
The software-defined \gls{dsp} constructs the bandwidth-optimized digital signals $x[m]$ and $p[m]$, encoding the real and imaginary parts of the complex symbols.
The \gls{dac} stage converts the digital signals, $x[m]$ and $p[m]$, to bandwidth-limited analog signals $x(t)$ and $p(t)$.
Finally, the analog signals are modulated onto an optical carrier yielding a coherent state $\ket{\alpha(t)}$ which meets the bandwidth requirements of the channel.

\subsection{Symbol encoding in baseband}

To construct a bandwidth-optimized baseband signal, encoding the complex symbol sequence $\{\alpha_n\in\mathbb{C}\colon n\in I\}$, we first remark that the symbol sequence itself has no notion of time.
In contrast, a digital (time-discrete) signal, consisting of discrete samples, includes a time reference, the sample period $T_s$, denoting the temporal distance between two consecutive samples.
By defining the digital signal with samples equal to the symbols, $\alpha[n]=\alpha_n$, and introducing the symbol period $T_s$ as sample period, we find ourselves with the complete \gls{dsp} toolbox at our disposal.\footnote{Even if the \gls{dsp} itself does not work explicitly with the time reference $T_s$, we need time to give a meaningful interpretation of the signal between the steps.}
\begin{figure}[htb]
	\centering
	\includegraphics{figures/circuitikz/baseband-construction}
	\caption{Block diagram of the signal-processing for baseband construction. The digital signal $x[km]$ is upsampled by a factor $k$ to $x^\prime[m]$ and pulse-shaped by a \gls{rrc} filter to yield $x^{\prime\prime}[m]$. A \gls{dac} converts the pulse-shaped signal $x^{\prime\prime}[m]$ to the analog signal $x^\prime(t)$. Finally, the analog signal $x^\prime(t)$ is \gls{lp} filtered to yield the analog anti-aliased signal $x(t)$.}\label{fig:baseband_construction}
\end{figure}
\Cref{fig:baseband_construction} summarizes the essential \gls{dsp} steps including analog conversion of a real-valued digital signal $x[km]$ to construct a bandwidth-optimized analog baseband signal $x(t)$.\footnote{The baseband construction generalizes to a complex digital signal by applying the real-valued baseband construction separately to the real and imaginary part.}
The digital signal $x[km]$, containing the symbols, is first upsampled by an upsampling factor of $k$, adding $k$ zero-valued samples in between the original samples.
The pulse-shaping of the \gls{rrc} filter interpolates between the non-zero samples, the symbols, to reduce the effective signal bandwidth.
Finally, a \gls{dac} converts the digital signal $x^{\prime\prime}[m]$ to the analog signal $x^\prime(t)$ through infinite upsampling.
The analog signal $x^\prime(t)$ contains infinite aliases through the upsampling, which we remove by filtering $x^\prime(t)$ with a \gls{lp}, yielding the anti-aliased analog signal $x(t)$.
\begin{figure}[htb]
	\centering
	\includegraphics{figures/pgfplots/tx-unit-time}
	\caption{Baseband construction for a single unit symbol in the time domain. The symbol sequence $\{x_n\in\mathbb{R}\colon n\in I\}$ is represented by the digital signal $x[km]$ with sample period $T_s$ (first row). The digital signal $x[km]$ is upsampled to $x[m]$ by an upsampling factor of $k=2$ (second row). The upsampled signal $x[m]$ is pulse-shaped with a \gls{rrc} filter to yield $x^\prime[m]$ (third row). The pulse-shaped digital signal $x^\prime[m]$ is converted to the anti-aliased analog signal $x(t)$.}\label{fig:baseband_construction_unit_time}
\end{figure}
\Cref{fig:baseband_construction_unit_time} illustrates the time domain signals for each signal-processing step for a symbol sequence which contains only a single non-zero symbol with unit value.
We see very well how the upsampling increases the resolution of the digital signal in the time domain and how the pulse-shaping filter interpolates between the samples.
We also see that the pulse-shaping filter corresponds to a $\sinc$-like impulse response.
The similarity of the analog signal with a $\sinc$ pulse is not surprising since its \gls{rrc} square equals a rectangular filter with roll-off, i.e.,
\begin{equation}
	\abs{h_\text{rc}\left(f/f_s\right)}
	=
	\begin{cases}
		1 & \abs*{f/f_s}\leq(1-\alpha) \\
		\cos\left[\frac{\pi}{4\alpha}\left(\abs*{f/f_s}-1+\alpha\right)\right] & 1-\alpha\leq\abs*{f/f_s}\leq1+\alpha \\
		0 & \text{otherwise}
	\end{cases}
	,
\end{equation}
wherein $f_s=1/T_s$ is the symbol rate and $\alpha$ determines the roll-off.
\begin{figure}[htb]
	\centering
	\includegraphics{figures/pgfplots/tx-rand-time}
	\caption{Baseband construction for a random \gls{qpsk} symbol sequence in the time domain. The real and imaginary part are colored orange respectively green. The complex symbol sequence $\{\alpha_n\in\mathbb{C}\colon n\in I\}$ is represented by the digital signal $\alpha[km]$ with sample period $T_s$ (first row). The digital signal $\alpha[km]$ is upsampled to $\alpha^\prime[m]$ by an upsampling factor of $k=2$ (second row). The upsampled signal $\alpha^\prime[m]$ is pulse-shaped with a \gls{rrc} filter to yield $\alpha^{\prime\prime}[m]$ (third row). The pulse-shaped digital signal $\alpha^{\prime\prime}[m]$ is converted to the anti-aliased analog signal $\alpha(t)$.}\label{fig:baseband_construction_rand_time}
\end{figure}
\Cref{fig:baseband_construction_rand_time} illustrates the time domain signals for each signal-processing step for a random \gls{qpsk} symbol sequence.
\begin{figure}[htb]
	\centering
	\includegraphics{figures/pgfplots/tx-frequency}
	\caption{Baseband construction for a random \gls{qpsk} symbol sequence in the frequency domain. The power spectrum corresponding to the unit symbol sequence is colored orange, the \gls{qpsk} symbol sequence is colored green. Initially, the spectrum spans from $-1/2$ to $+1/2$ the normalized sampling frequency $f/f_s$ (first row). Upsampling by $k=2$ adds aliases left and right to the initial spectrum (second row). Pulse-shaping suppresses the left and ride flanks of the spectrum (third row). Analog conversion corresponds to infinite upsampling, adding infinite aliases left and right of the spectrum. (fourth row). Applying a \gls{lp} filter strongly suppresses the aliases (last row).}\label{fig:baseband_construction_freq}
\end{figure}
\Cref{fig:baseband_construction_freq} provides further inside into the signal-processing steps by presenting the power spectrum of the unit and \gls{qpsk} symbol sequences.
In the frequency domain, it is very clear to see how upsampling widens the spectrum without adding additional information.
We also see how the pulse-shaping filter shapes the upsampled spectrum, and the \gls{lp} filter suppresses aliases.

\FloatBarrier
\subsection{Up-conversion of baseband to passband}

We previously constructed the bandwidth-optimized baseband signals $x(t)$ and $p(t)$ encoding the real respective imaginary part of a complex symbol sequence $\{\alpha_n\in\mathbb{C}\colon n\in I\}$.
We now want to shift the spectrum of the baseband signals into the optical domain, i.e., transform the baseband to a passband signal centered around some optical carrier frequency $\omega_c$.
One way to shift a real signal in the frequency domain is to multiply it by a sine or cosine, effectively corresponding to amplitude modulation.
With the orthogonality of sine and cosine, it is possible to write two real baseband signals as one real passband signal, i.e.,
\begin{equation}
	x(t)
	\cos(\omega_ct)
	+
	p(t)
	\sin(\omega_ct)
	.
\end{equation}
Rewriting the sine as the real part of a complex exponential
\begin{equation}
	\sin(\omega_ct)
	=
	\cos(\omega_c t-\pi/2)
	=
	\Re\left\{e^{-i\omega_c t+i\pi/2}\right\}
	,
\end{equation}
we can rewrite the real passband signal as the real part of a complex signal,
\begin{equation}
	\begin{split}
		x(t)
		\cos(\omega_ct)
		+
		p(t)
		\sin(\omega_ct)
		&=
		\Re\left\{
			x(t)
			e^{-i\omega_ct}
			+
			p(t)
			e^{-i\omega_ct+i\pi/2}
		\right\}
		\\
		&=
		\Re\left\{
			\left[
				x(t)
				+
				ip(t)
			\right]
			e^{-i\omega_ct}
		\right\}
		,
	\end{split}
	\label{eq:real_complex_passband}
\end{equation}
where we can identify the complex baseband
\begin{equation}
	\alpha(t)
	=
	x(t)
	+
	ip(t)
	.
\end{equation}
The complex signal is often used when dealing with only a real signal since the up-conversion here corresponds to a simple multiplication by a complex exponent.
\begin{figure}[htb]
	\centering
	\includegraphics{figures/circuitikz/up-conversion}
	\caption{Block diagram illustrating up-conversion of two real-valued baseband signals, $x(t)$ and $p(t)$, to a real-valued passband signal, $\Re\left\{\alpha(t)e^{-i\omega_ct}\right\}$. A \gls{lo} with up-conversion frequency $\omega_c$ is split into two branches with a relative phase shift between the branches of $\pi/2$. One branch is mixed with the baseband signal $p(t)$ and the other branch is mixed with $x(t)$. The product of the mixing is added giving the passband signal.}\label{fig:up_conversion}
\end{figure}
Our derivation of the complex passband, \cref{eq:real_complex_passband}, already hints implementing efficient up-conversion of two real baseband signals by splitting a \gls{lo} with a relative phase shift of $\pi/2$.
If we replace the mixer in \Cref{fig:up_conversion} with electro-optical amplitude modulators, e.g.,\gls{mzm}, we find ourselves with an electro-optical \gls{iqm}.
Indeed, if we assume a single-mode laser as an input state, we find that the \gls{iqm} produces the output state
\begin{equation}
	\ket{e^{-i\omega_c t}}
	\to
	\ket{\alpha(t)e^{-i\omega_c t}}
\end{equation}
exactly as we found for the up-conversion.
		\section{Receiver}
\FloatBarrier

We introduced the receiver as a component decoding a sequence of complex symbols,
\begin{equation}
	\left\{
		\beta_n
		\in
		\mathbb{C}
		\colon
		n\in I
	\right\}
	,
\end{equation}
from a coherent state $\ket{\beta(t)}$.
Just like the transmitter, we want to keep the receiver software-defined.
\begin{figure}[htb]
	\centering
	\includegraphics{figures/tikz/software-defined-receiver}
	\caption{Block diagram of the receiver's signal processing domains. The analog electrical signals $u(t)$, and optional $v(t)$, are demodulated from the quadratures of the optical coherent state $\ket{\beta(t)}$, and then converted to the digital signals $u[m]$, and optional $v[m]$, from which the \gls{dsp} decodes the symbol sequence $\{\beta_n\in\mathbb{C}\colon n\in I\}$.}\label{fig:software_defined_receiver}
\end{figure}
\Cref{fig:software_defined_receiver} shows the signal processing of a possible software-defined receiver.
The coherent state is transferred from the optical via the analog to the digital.

\FloatBarrier
\subsection{Downconversion}

At the transmitter, we upconverted two real baseband signals to a real passband signal.
For the receiver, we discuss options involving one and two real baseband signals.
We first consider the simpler case of direct downconversion as depicted in \Cref{fig:downconversion_single}.
\begin{figure}[htb]
	\centering
	\includegraphics{figures/circuitikz/downconversion-single}
	\caption{Block diagram of single-quadrature downconversion. The signal $z(t)$ is mixed with the \gls{lo} signal $\cos(\omega_lt+\vartheta)$. The downconverted signal $u(t)$ is obtained by \gls{lp} filtering the output signal of the mixing.}\label{fig:downconversion_single}
\end{figure}
In direct downconversion, we mix a real-valued signal,
\begin{equation}
	\begin{split}
		z(t)
		=
		\int_{-\infty}^{+\infty}\frac{\dd{\omega}}{2\pi}
		z(\omega)
		e^{+i\omega t}
		&=
		\int_0^{+\infty}\frac{\dd{\omega}}{2\pi}
		z(\omega)
		e^{+i\omega t}
		+
		\int_{-\infty}^0\frac{\dd{\omega}}{2\pi}
		z(\omega)
		e^{+i\omega t}
		\\
		&=
		\int_0^{+\infty}\frac{\dd{\omega}}{2\pi}
		z(\omega)
		e^{+i\omega t}
		-
		\int_{+\infty}^0\frac{\dd{\omega}}{2\pi}
		z(-\omega)
		e^{-i\omega t}
		\\
		&=
		\int_0^{+\infty}\frac{\dd{\omega}}{2\pi}
		\left[
			z(\omega)
			e^{+i\omega t}
			+
			z(\omega)^*
			e^{-i\omega t}
		\right]
		\\
		&=
		\int_0^{+\infty}\frac{\dd{\omega}}{2\pi}
		2\Re\left[
			z(\omega)
			e^{+i\omega t}
		\right]
		,
	\end{split}
\end{equation}
where we used the conjugate symmetry, $z(-\omega)=z(\omega)^*$, of the Fourier transform of a real-valued function $z(t)$.
Multiplication with the \gls{lo} signal $\cos(\omega_lt+\vartheta)$, the mixing produces a high- and low-frequency band
\begin{equation}
	\begin{split}
		z(t)
		\cos(\omega_lt+\vartheta)
		&=
		2\Re
		\int_0^\infty\frac{\dd{\omega}}{2\pi}
		z(\omega)
		e^{+i\omega t}
		\cos(\omega_lt+\vartheta)
		\\
		&=
		\Re
		\int_0^\infty\frac{\dd{\omega}}{2\pi}
		z(\omega)
		e^{+i\omega t}
		\left[
			e^{+i(\omega_lt+\vartheta)}
			+
			e^{-i(\omega_lt+\vartheta)}
		\right]
		\\
		&=
		\Re
		\int_0^\infty\frac{\dd{\omega}}{2\pi}
		z(\omega)
		\left[
			e^{+i(\omega+\omega_l)t+i\vartheta}
			+
			e^{+i(\omega-\omega_l)t-i\vartheta}
		\right]
		\\
		&=
		\Re
		\int_{+\omega_l}^\infty\frac{\dd{\omega}}{2\pi}
		z(\omega-\omega_l)
		e^{+i\omega t+i\vartheta}
		\\
		&\qquad+
		\Re
		\int_{-\omega_l}^\infty\frac{\dd{\omega}}{2\pi}
		z(\omega+\omega_l)
		e^{+i\omega t-i\vartheta}
		.
	\end{split}
\end{equation}
However, we suppress the high-frequency band using an ideal \gls{lp} filter with bandwidth $B$,
\begin{equation}
	\begin{split}
		u(t)
		&=
		\Re
		\int_{-B/2}^{+B/2}\frac{\dd{\omega}}{2\pi}
		z(\omega+\omega_l)
		e^{+i\omega t-i\vartheta}
		\\
		&=
		\Re
		\int_{0}^{+B/2}\frac{\dd{\omega}}{2\pi}
		z(\omega+\omega_l)
		e^{+i\omega t-i\vartheta}
		+
		\Re
		\int_{-B/2}^{0}\frac{\dd{\omega}}{2\pi}
		z(\omega+\omega_l)
		e^{+i\omega t-i\vartheta}
		\\
		&=
		\Re
		\int_{0}^{+B/2}\frac{\dd{\omega}}{2\pi}
		z(\omega+\omega_l)
		e^{+i\omega t-i\vartheta}
		-
		\Re
		\int_{B/2}^{0}\frac{\dd{\omega}}{2\pi}
		z(\omega-\omega_l)^*
		e^{-i\omega t-i\vartheta}
		\\
		&=
		\Re
		\int_{0}^{+B/2}\frac{\dd{\omega}}{2\pi}
		\left[
			z(\omega+\omega_l)
			e^{+i\omega t}
			+
			z(\omega-\omega_l)^*
			e^{-i\omega t}
		\right]
		e^{-i\vartheta}
		,
	\end{split}
	\label{eq:downconversion_real}
\end{equation}
where we assumed $\omega_l\gg B/2$.
For $\vartheta=0$, the downconverted signal $v(t)$ is equal to projecting the real part of the complex input spectrum $z(\omega)$, losing the imaginary part's information.
Furthermore, when rewriting $u(t)$ as an integral over positive frequencies, i.e., frequencies we can measure, we find a second term mirroring the first term.
\begin{figure}[htb]
	\centering
	\includegraphics{figures/tikz/spectrum-downconversion-single}
	\caption{Power spectrum illustrating downconversion of a passband signal (solid spectrum) mixed with a \gls{lo} signal $\omega_l$ to the intermediate frequency $\omega_i$ (dashed spectrum) and measurement with bandwidth $B_d$ (dotted spectrum).}\label{fig:spectrum_downconversion_single}
\end{figure}
\Cref{fig:spectrum_downconversion_single} shows the downconversion of the signal $z(t)$ around the \gls{lo} at $\omega_l$ to the intermediate frequency $\omega_i$.
The actual measurement involved only positive frequencies up to the detector bandwidth $B/2$ causing the actual signal to be imposed with the mirrored spectrum.

Single-quadrature downconversion only reveals a real projection of the complex spectrum.
To conserve both quadratures, we need to split the input signal into two branches and perform single-quadrature downconversion with two orthogonal phase references of the \gls{lo}, see \Cref{fig:downconversion_dual}.
\begin{figure}[htb]
	\centering
	\includegraphics{figures/circuitikz/downconversion-dual}
	\caption{Block diagram of dual-quadrature downconversion. The signal $z(t)$ is divided equally into an upper and a lower branch. The upper branch is mixed with the phase shifted \gls{lo} signal $\cos(\omega_ct+\vartheta)$. The lower branch is mixed with \gls{lo} signal $\sin(\omega_ct+\vartheta)$. The mixer outputs are filtered separately by a \gls{lp} yielding the downconverted signals $u(t)$ and $v(t)$.}\label{fig:downconversion_dual}
\end{figure}
The signal of the upper branch $u(t)$ is equal to our result for the single-quadrature downconversion, \cref{eq:downconversion_real}.
The signal of the lower branch,
\begin{equation}
	\begin{split}
		v(t)
		&=
		\Im
		\int_{-B/2}^{+B/2}\frac{\dd{\omega}}{2\pi}
		z(\omega-\omega_l)
		e^{+i(\omega t+\vartheta)}
		\\
		&=
		\Im
		\int_{0}^{+B/2}\frac{\dd{\omega}}{2\pi}
		\left[
			z(\omega-\omega_l)
			e^{+i(\omega t+\vartheta)}
			+
			z(\omega+\omega_l)^*
			e^{-i(\omega t+\vartheta)}
		\right]
		,
	\end{split}
	\label{eq:downconversion_imag}	
\end{equation}
is simply obtained from \cref{eq:downconversion_real} by shifting the \gls{lo} phase reference by \SI{90}{\degree}, i.e., $\vartheta\to\vartheta+\pi/2$.
Regardless of the particular value of the \gls{lo} phase reference $\vartheta$, dual-quadrature downconversion recovers the complete information, the real and imaginary part, of the input signal spectrum $z(\omega)$.

We presented the electro-optical receiver setups implementing single- and dual-quadrature downconversion in \Cref{fig:coherent_receiver_active} and \Cref{fig:coherent_receiver_passive} in \Cref{ch:qkd}.
Essentially, the electro-optical setups combine the optical signal and \gls{lo} in an optical coupler and perform balanced detection on the coupler outputs for which derived the quantum theory in \Cref{sec:photodetectors}.
From a quantum viewpoint, balanced detection corresponds to a quadrature measurement at a particular frequency represented by the generalized quadrature operator, \cref{eq:quadrature_operator_generalized}.

\subsection{Homo- and heterodyning}

So far, we have not assumed any particular signal for the downconversion but treated the receiver as a spectrum analyzer.
If we now assume the input signal to be from the coherent-state transmitter $\ket{\beta(t)}$, \cref{eq:upconversion_dual}, the downconverted signals read
\begin{align}
	u(t)
	&=
	\Re
	\int_{-B/2}^{+B/2}\frac{\dd{\omega}}{2\pi}
	\beta(\omega-\omega_c+\omega_l)
	e^{+i(\omega t+\theta)}
	\label{eq:receiver_demod_real}
	\\
	v(t)
	&=
	\Im
	\int_{-B/2}^{+B/2}\frac{\dd{\omega}}{2\pi}
	\beta(\omega-\omega_c+\omega_l)
	e^{+i(\omega t+\theta)}
	\label{eq:receiver_demod_imag}
	,
\end{align}
wherein $\theta$ accounts for the phases of the up- and downconversion \glspl{lo}.
We define the the intermediate frequency as the difference between the transmitter and receiver \glspl{lo}, i.e.,
\begin{equation}
	\omega_i
	=
	\abs{\omega_c-\omega_l}
	<
	B_d/2
	,
\end{equation}
and distinguish between homodyning for zero intermediate frequency $\omega_i=0$, and otherwise, heterodyning $\omega_i\neq 0$.
\begin{figure}[htb]
	\centering
	\includegraphics{figures/tikz/spectrum-heterodyning}
	\caption{Power spectrum illustrating heterodyne detection. The passband signal with carrier frequency $\omega_c$ and bandwidth $B_s$ (dashed) is downconverted with the \gls{lo} frequency $\omega_l>\omega_c$. The downconverted spectrum is measured with bandwidth $B_d$ (solid), which contains the image band (dotted) from the right side of the \gls{lo}.}\label{fig:spectrum_heterodyning}
\end{figure}
In heterodyning, we downconvert the passband signal at $\omega_c$ to an intermediate frequency $\omega_i$.


The negative frequencies collapse onto the positive frequencies by being mirrored around zero frequency $\omega=0$.
In \Cref{fig:spectrum_heterodyning}, the left spectrum at $\omega_l$ (dashed) is mirrored 

\Cref{fig:spectrum_heterodyning} shows how the frequencies left of the $\omega_l$ fold onto the positive frequencies.


Furthermore, we distinguish between single- and dual-quadrature homodyning, depending if we perform single- or dual- quadrature downconversion.
If the detector bandwidth is large enough to cover the baseband signal at the intermediate frequency, we can resolve both quadratures of the incoming signal with heterodyning and single-quadrature downconversion.
\begin{table}[htb]
  \centering
  \begin{tabular}{lccccc}
    \toprule
    Scheme & Homodyne (single) & Homodyne (dual) & Heterodyne \\
    \midrule
    Balanced detectors & \num{1} & \num{2} & \num{1} \\
    Quadratures & \num{1} & \num{2} & \num{2} \\
    Intermediate frequency & \multicolumn{2}{c}{$\omega_i=0$} & $\omega_i\neq 0$ \\
    Optical complexity & Low & High & Low \\
    Signal bandwidth & High & High & Low \\
    \gls{lo} synchronization & Frequency and phase & Frequency & Bandwidth \\
    \bottomrule
  \end{tabular}
  \caption{Comparison of receiver schemes according to Ref.~\cite{Brunner2017}. The single-quadrature homodyne detection offers low optical complexity and high bandwidth but only resolves one of two quadratures and required frequency and phase synchronization of the \gls{lo}. The dual-quadrature homodyne detection resolves both quadratures with high bandwidth but requires two balanced detectors increasing the optical complexity and phase synchronization of the \gls{lo}. The heterodyne detection schemes resolves both quadratures with low complexity and no requirements on \gls{lo} synchronization at the cost of signal bandwidth.}\label{tab:receivers}
\end{table}
\Cref{tab:receivers} summarizes the characteristics between the single and dual homodyne and the heterodyne receiver schemes.
A strong advantage of the heterodyne receiver design is that both quadratures can be resolved with a single balanced detector, keeping the optical complexity low.

\begin{figure}[htb]
	\centering
	\includegraphics{figures/tikz/spectrum-homodyning}
	\caption{Power spectrum illustrating homodyne detection. The passband signal with carrier frequency $\omega_c$ and bandwidth $B_s$ (dashed) is downconverted with the \gls{lo} frequency, equal to the carrier frequency, $\omega_l=\omega_c$. The downconverted spectrum is measured with bandwidth $B_d$ (solid), which contains the mirror (dotted) from the right side of the \gls{lo}.}\label{fig:spectrum_homodyning}
\end{figure}

\FloatBarrier
\subsection{Symbol decoding}

We continue our receiver description, starting from the single-quadrature downconversion and assuming the more general heterodyning, which for $\omega_i=0$ reduces to single-quadrature homodyning.
\begin{figure}[htb]
	\centering
	\includegraphics{figures/circuitikz/symbol-decoding}
	\caption{Block diagram of the signal processing for the symbol decoding. The analog signal $u(t)$ is converted to the digital signal $u[m/(kl)]$. The real digital signal $u[m/(kl)]$ is multiplied with the complex exponential $\exp(i\omega_it)$, yielding the complex digital signal $\sigma[m/(kl)]$. $\sigma[m/(kl)]$ is downsampled by $l$ to yield the complex digital signal $\mu[m/k]$. $\mu[m/k]$ is pulse-shaped with the matched \gls{rrc} filter to yield the complex digital signal $\kappa[m/k]$. $\kappa[m/k]$ is downsampled to the complex digital signal $\beta[m]$ corresponding to the decoded symbol sequence.}\label{fig:symbol_decoding_blocks}
\end{figure}
\Cref{fig:symbol_decoding_blocks} summarizes the relevant signal processing for the symbol decoding.
The downconverted signal $u(t)$ corresponding to the real part of the received coherent-state spectrum $\beta(\omega)$, \cref{eq:receiver_demod_real}, is sampled by an \gls{adc}, yielding the digital signal $u[m/(kl)]$.
We remove the intermediate frequency in $u[m/(kl)]$ by multiplication with a complex exponential, i.e,
\begin{equation}
	\sigma\left[\frac{m}{kl}\right]
	=
	u\left[\frac{m}{kl}\right]
	e^{+2\pi i (m/kl) T_s}
	,
\end{equation}
making the signal complex-valued.
It follows a downsampling by $l$ of the signal such that we can apply the same \gls{rrc} filter, the matched filter, which we used in the symbol encoding to maximize \gls{snr}.
Finally, we downsample by $k$ to restore a digital signal corresponding to the symbol sequence.
\begin{figure}[htb]
	\centering
	\includegraphics{figures/pgfplots/symbol-decoding-frequency}
	\caption{Power spectrum of the symbol-decoding steps for a random \gls{qpsk} symbol-sequence. The demodulated signal is a real-valued passband signal centered at the intermediate frequency (first row). After digital downconversion we have a complex-valued baseband signal, centered at zero frequency (second row). Applying the matched \gls{rrc} filter completes the pulse-shaping (third row). Downsampling recovers the initial symbol band (last row).}\label{fig:symbol_decoding_frequency}
\end{figure}
\Cref{fig:symbol_decoding_frequency} illustrates how the symbol decoding is carried out in the frequency domain.
The demodulated signal spectrum is a passband signal at the intermediate frequency and downconversion reduces the passband to a baseband signal.
Completing the pulse-shaping with the matched filter increases the steepness of the flanks which are collapsed with aliasing by the final downsampling step.
\begin{figure}[htb]
	\centering
	\includegraphics{figures/pgfplots/symbol-decoding-time-qpsk}
	\caption{Signal amplitude of the symbol decoding steps for a random \gls{qpsk} symbol-sequence. The real-valued demodulated signal oscillates at the intermediate frequency (first row). Digital downconversion removed the intermediate frequency, yielding a complex signal (second row). Completing the pulse-shaping and downsampling by applying a matched \gls{rrc} filter (third row). Downsampling recovers the complex symbol sequence equal to the transmitted sequence (fourth and last row).}\label{fig:symbol_decoding_time}
\end{figure}
\Cref{fig:symbol_decoding_time}) shows the symbol decoding in the time domain.

		\addcontentsline{toc}{section}{References}
		\printbibliography[title=References]
	\end{refsection}


	\chapter{Quantum field theory of light}
	\begin{refsection}
		The present chapter identifies common characteristics of \gls{qkd} protocols and attempts to formalize the notion of a \gls{qkd} protocol.
We test the proposed formula with qubit-based \gls{qkd} protocols like BB84 and the six-state protocol as well as Gaussian \gls{cvqkd} protocols.
The chapter ends with a summary and literature review regarding practical \gls{qkd} protocols' post-processing and security analysis.

\Cref{fig:qkd_protocol} illustrates our proposed notion of a \gls{qkd} protocol, with the feature being a logical quantum system from which the random bits are encoded and decoded.
The logical quantum system is a subspace of the physical quantum system.
The physical quantum system depends strongly on the physical implementation and quantum encoding.
\begin{figure}[htb]
	\centering
	\includestandalone{figures/tikz/qkd-protocol}
	\caption{A \gls{qkd} protocol comprises a binary encoder, a logical quantum system, and a binary decoder. The binary encoder maps bits $\vb{b}\in\{0,1\}^n$ onto a quantum state of the logical quantum system $\ket{\psi}$. The binary decoder extracts the bits $\vb{b}$ back from the quantum state $\ket{\psi^\prime}$. The logical quantum system is a subspace of a larger physical quantum system. The state encoder and decoder map between the logical and physical quantum states.}\label{fig:qkd_protocol}
\end{figure}
The distinction between logical and quantum systems is vital to separate the implementation and security concerns.
Many security proofs show equivalence between the physical implementation and the logical system to use an established security proof.
One should keep in mind that such a separation implicitly assumes no loopholes from the particular implementation.

\Cref{fig:qkd_classification} illustrates common features among the \gls{qkd} protocols.
For an overview of \gls{qkd} protocols, see Ref.~\cite{Duvsek2006}.
\begin{figure}[htb]
	\centering
	\includestandalone{figures/tikz/qkd-classification}
	\caption{Common features among \gls{qkd} protocols: Detection, physical encoding, logical state space, measurement basis selection and schema.}\label{fig:qkd_classification}
\end{figure}
Every \gls{qkd} system requires a detector, e.g., a coherent detector or a single-photon (click) detector.
The detection does not necessarily imply the dimension of the logical state space as BB84 has been implemented with coherent detection~\cite{Qi2021}.
Concerning measurement basis selection, Bob can either actively choose a random measurement basis for every transmission or passively measure all (orthogonal) bases for measurement basis selection.
We will cover both active and passive measurement basis selection in the discussion of the polarization-encoding BB84 protocol.
Finally, the \gls{qkd} schema determines if either Alice prepares a state and sends it to Bob for measurement (prepare-and-measure) or if Alice and Bob share an entangled state (entanglement-based).
Most practical \gls{qkd} implementations use prepare-and-measure.
On a theoretical level, both schemas are equivalent, and security proofs are often more convenient in an entanglement-based setting.
		\section{Quantization of the Maxwell field in the Coulomb gauge}

\subsection{Relativistic field theory}

The Lagrangian of the Maxwell field $A^\mu(t,\vb{x})$ reads~\cite[p.~339]{Srednicki2007}
\begin{equation}
	\mathcal{L}
	=
	\frac{1}{2}
	(\partial_\mu A_\nu)
	\left(
		\partial^\nu A^\mu
		-
		\partial^\mu A^\nu
	\right)
\end{equation}
and the covariant generalization of the Euler-Lagrange equations
\begin{equation}
	0
	=
	\partial_\mu
	\pdv{\mathcal{L}}{(\partial_\mu A_\nu)}
	-
	\pdv{\mathcal{L}}{A_\nu}
	=
	\partial_\mu\partial^\mu A^\nu
	-
	\partial^\nu\partial_\mu A^\mu
\end{equation}
leads to the free equations of motion.
We ignore static charges $A_0=0$ and employ the Coulomb gauge $\partial_iA^i=0$ in which the Maxwell field is transverse.
The equations of motion simplify to relativistic wave equation
\begin{equation}
	0
	=
	\partial_\mu\partial^\mu\vb{A}
	=
	\partial_t^2\vb{A}
	-
	\laplacian\vb{A}
	.
\end{equation}

\subsection{Mode decomposition}

In momentum space the transverse field $\vb{A}$ reads
\begin{equation}
	\vb{A}(t,\vb{x})
	=
	\int_{\mathbb{R}^4}\frac{\dd[4]{p}}{(2\pi)^4}
	\vb{A}(p_0,\vb{p})
	e^{ip_0t-i\vb{p}\vdot\vb{x}}
	.
\end{equation}
In momentum space, we can construct a polarization basis $\vu{e}_1(\vb{p}),\vu{e}_2(\vb{p})$ being transverse
\begin{equation}
	\vb{p}\vdot\vu{e}_\lambda(\vb{p})
	=
	0
	,
\end{equation}
orthonormal
\begin{equation}
	\vu{e}_\lambda(\vb{p})
	\vdot
	\vu{e}_{\lambda^\prime}(\vb{p})
	=
	\delta_{\lambda,\lambda^\prime}
\end{equation}
and complete
\begin{equation}
	\sum_{\lambda=1,2}
	\vu{e}_\lambda^i(\vb{p})
	\vu{e}_{\lambda^\prime}^j(\vb{p})
	=
	\delta^{ij}
	-
	\frac{p^ip^j}{\vb{p}^2}
	=
	P_\perp^{ij}(\vb{p})
\end{equation}
with $P_\perp(\vb{p})$ being the transverse projector.
Expressing $\vb{A}(p_0,\vb{p})$ in the polarization basis, we find
\begin{equation}
	\vb{A}(p_0,\vb{p})
	=
	\sum_{\lambda=1,2}
	A_\lambda(p_0,\vb{p})
	\vu{e}_\lambda(\vb{p})
\end{equation}
and the mode decomposition reads
\begin{equation}
	\vb{A}(t,\vb{x})
	=
	\sum_{\lambda=1,2}
	\int_{\mathbb{R}^4}\frac{\dd[4]{p}}{(2\pi)^4}
	A_\lambda(p_0,\vb{p})
	e^{ip_0t-i\vb{p}\vdot\vb{x}}
	\vu{e}_\lambda(\vb{p})
	.
\end{equation}
Inserting the mode decomposition into the relativistic wave equation, we recover the relativistic energy-momentum relation for massless particles
\begin{equation}
	E(\vb{p})
	=
	\omega(\vb{p})
	=
	\vb{p}
	.
\end{equation}
Hence, if the Fourier modes $a_\lambda(p_0,\vb{p})$ satisfy the relativistic energy-momentum relation, $\vb{A}(t,\vb{x})$ satisfies the relativistic wave equation.
We enforce the mode decomposition to satisfy the energy-momentum relation by constraining the integration domain to
\begin{equation}
	V_p
	=
	\left\{
		(p_0,\vb{p})\in\mathbb{R}^4
		\colon
		p_0^2
		=
		\omega(\vb{p})^2
	\right\}
\end{equation}
or equivalent, adding a factor
\begin{equation}
	(2\pi)
	\delta^{(1)}\left(p_0^2-\omega(\vb{p})^2\right)
	=
	(2\pi)
	\frac{
		\delta^{(1)}\left(p_0-\omega(\vb{p})\right)
		-
		\delta^{(1)}\left(p_0+\omega(\vb{p})\right)
	}{\sqrt{2\omega(\vb{p})}}
\end{equation}
to the integrand.
Finally, we arrive at
\begin{equation}
	\vb{A}(t,\vb{x})
	=
	\sum_{\lambda=1,2}
	\int_{\mathbb{R}^3}\frac{\dd[3]{p}}{(2\pi)^3\sqrt{2\omega(\vb{p})}}
	\left\{
		a_\lambda(\vb{p})
		e^{i\omega(\vb{p})t-i\vb{p}\vdot\vb{x}}
		\vu{e}_\lambda(\vb{p})
		+
		\text{c.c.}
	\right\}
\end{equation}
where we defined
\begin{equation}
	a_\lambda(\vb{p})
	=
	A_\lambda\left(\omega(\vb{p}),\vb{p}\right)
\end{equation}
and used the conjugate symmetry of the Fourier amplitudes $a_\lambda(-\vb{p})=a_\lambda(\vb{p})^*$.

\subsection{Canonical quantization}
		\section{Quantum states, operators and expectation values}

\subsection{Vacuum state}

\subsection{Positive and negative frequency operators}

% TODO: cite operator-valued distributions, smeared fields

The positive and negative frequency operators of the Maxwell field
\begin{align}
	\hat{\vb{A}}^{(-)}
	&=
	\sum_{\lambda=1,2}
	\int_{\mathbb{R}^3}
	\frac{\dd[3]{p}}{(2\pi)^3\sqrt{2\omega(\vb{p})}}
	\hat{a}_\lambda(\vb{p})
	\vu{e}_\lambda(\vb{p})
	\eval{e^{-ip_\mu x^\mu}}_{p_0=\omega(\vb{p})}
	\\
	\hat{\vb{A}}^{(+)}
	&=
	\sum_{\lambda=1,2}
	\int_{\mathbb{R}^3}
	\frac{\dd[3]{p}}{(2\pi)^3\sqrt{2\omega(\vb{p})}}
	\hat{a}_\lambda(\vb{p})^\dagger
	\vu{e}_\lambda(\vb{p})^*
	\eval{e^{+ip_\mu x^\mu}}_{p_0=\omega(\vb{p})}
\end{align}
are operator-valued distributions.

\subsection{Number state}

\subsection{Displacement operator}

\begin{equation}
	\hat{D}[\alpha]
	=
	\exp\left\{
		\int\frac{\dd[3]{p}}{(2\pi)^3\sqrt{2\omega(\vb{p})}}
		\left\{
			\alpha(\vb{p})
			\hat{a}^\dagger(\vb{p})
			-
			\alpha(\vb{p})^*
			\hat{a}(\vb{p})
		\right\}
	\right\}
\end{equation}

\subsection{Coherent state}

\begin{equation}
	\begin{split}
		\ket{\alpha}
		&=
		\exp\left\{
			-
			\frac{1}{2}
			\int\frac{\dd[3]{p}}{(2\pi)^32\omega(\vb{p})}
			\abs{\alpha(\vb{p})}^2
		\right\}
		\\
		&\times
		\exp\left\{
			\int\frac{\dd[3]{p}}{(2\pi)^3\sqrt{2\omega(\vb{p})}}
			\alpha(\vb{p})
			\hat{a}(\vb{p})^\dagger
		\right\}
		\ket{0}
	\end{split}
\end{equation}

\subsection{Quadrature operator}

We define the generalized quadrature operator by 
\begin{equation}
	\hat{X}(\theta)
	=
	\int_{\mathbb{R}^3}
	\frac{\dd[3]{p}}{(2\pi)^3}
	\frac{1}{\sqrt{2}}
	\left\{
		\hat{a}(\vb{p})
		e^{-i\theta}
		+
		\hat{a}^\dagger(\vb{p})
		e^{+i\theta}
	\right\}
\end{equation}
where the prefactor ensures that the commutator takes the standard form
\begin{equation}
	\comm{\hat{X}(\theta)}{\hat{X}(\theta+\Delta\theta)}
	=
	\frac{i}{2}
	\sin(\Delta\theta)
	V_p
\end{equation}
and $V_p$ is the momentum space volume
\begin{equation}
	V_p
	=
	\int_{\mathbb{R}^3}\frac{\dd[3]{p}}{(2\pi)^3}
	=
	\frac{4\pi}{(2\pi)^3}
	\int_0^\Lambda\dd{p}p^2
	=
	\frac{\Lambda^3}{6\pi^2}
\end{equation}
where we introduced the cut-off momentum $\Lambda$ for regularization.

The Robertson uncertainty relation yields a lower bound for the product of the variances
\begin{equation}
	\expval{\left(\Delta\hat{X}(\theta)\right)^2}
	\expval{\left(\Delta\hat{X}(\theta+\Delta\theta)\right)^2}
	\geq
	\frac{1}{4}
	\sin(\Delta\theta)^2
	V_p^2
	.
\end{equation}
The uncertainty is maximal for $\Delta\theta=\pi/2$.
The coherent state is a minimal uncertainty state in the sense that
\begin{align}
	\expval{\hat{X}(\theta)}{\alpha}
	=
	\sqrt{2}
	\int_{\mathbb{R}^3}
	\frac{\dd[3]{p}}{(2\pi)^3\sqrt{2\omega(\vb{p})}}
	\Re\left\{
		\alpha(\vb{p})
		e^{-i\theta}
	\right\}
	&&
	\expval{\left(\Delta\hat{X}(\theta)\right)^2}{\alpha}
	=
	\frac{1}{2}
	V_p
\end{align}

\subsection{Electromagnetic field operator}
		\section{Time-dependent interactions}

\subsection{Time-evolution operator}

Let $\ket{\psi(t_0)}$ be a state at time $t_0$, then the time-evolution relates the state $\ket{\psi(t)}$ at some later time $t>t_0$ to $\ket{\psi(t_0)}$ via
\begin{equation}
	\ket{\psi(t)}
	=
	\hat{U}(t,t_0)
	\ket{\psi(t_0)}
	.
\end{equation}
Inserting $\ket{\psi(t)}$ into the Schrödinger equation leads to
\begin{equation}
	i\dv{t}
	\hat{U}(t,t_0)
	=
	\hat{H}(t)
	\hat{U}(t,t_0)
\end{equation}
which formal solution is the time-ordered exponential, see Ref.~\cite[p.~380]{Bartelmann2018},
\begin{equation}
	\hat{U}(t,t_0)
	=
	T\exp\left\{
		-i
		\int_{t_0}^t\dd{t^\prime}
		\hat{H}(t^\prime)
	\right\}
\end{equation}
where $T$ denotes the time-ordering symbol.
Only for simple time-dependent systems an exact time-evolution operator exists.
In contrast to the Dyson expansion, the Magnus expansion yields a unitary time-evolution operator even for finite order, in particular,
\begin{equation}
	\hat{U}(t,t_0)
	=
	\exp\left\{
		\sum_{n=1}
		\hat{\Omega}^{(n)}(t,t_0)
	\right\}
\end{equation}
where the first two expansion terms are given by
\begin{align}
	\hat{\Omega}^{(1)}(t,t_0)
	&=
	\frac{(-i)}{1!}
	\int_{t_0}^t\dd{t^\prime}
	\hat{H}(t^\prime)
	\\
	\hat{\Omega}^{(2)}(t,t_0)
	&=
	\frac{(-i)^2}{2!}
	\int_{t_0}^t\dd{t^\prime}
	\int_{t_0}^{t^\prime}\dd{t^{\prime\prime}}
	\comm{\hat{H}(t^\prime)}{\hat{H}(t^\prime)}
\end{align}
and represent time-ordering corrections, see Ref.~\cite{QuesadaMejia2015}.

\subsection{Interaction with classical current}

The Schrödinger-picture Hamiltonian describing the interaction of the Maxwell field $\hat{\vb{A}}$ with a classical current $\vb{j}$ is
\begin{equation}
	\hat{H}_\text{int}(t)
	=
	-
	\vb{j}(t,\vb{x})
	\vdot
	\hat{\vb{A}}(t,\vb{x})
	.
\end{equation}
Inserting the mode expansion
		\section{Connection to quantum optics}

Here we reduce the results of the previous sections to a continuous-mode description of (linear polarized) light, i.e., frequency range limited to the optical domain.
Assumptions and limitations:
\begin{enumerate}
	\item Fixed reference frame (at rest), i.e., no Lorentz boosts allowed.
	\item Light is linearly polarized.
	\item \textcolor{red}{No distribution of the momentum}
\end{enumerate}

\begin{equation}
	\hat{N}
	=
	\int\dd{\omega}
	\hat{a}^\dagger(\omega)
	\hat{a}(\omega)
\end{equation}

\begin{equation}
	\hat{E}
	=
	i
	\int\dd{\omega}
	\omega
	\hat{a}^\dagger(\omega)
	\hat{a}(\omega)
\end{equation}

\begin{equation}
	\ket{f}
	=
	\int\dd{\omega}
	f(\omega)
	\hat{a}(\omega)
\end{equation}

\begin{equation}
	\ket{\alpha}
	=
	\exp\left\{
		-
		\frac{1}{2}
		\int\dd{\omega}
		\abs{\alpha(\omega)}^2
	\right\}
	\exp\left\{
		-
		\int\dd{\omega}
		\alpha(\omega)
		\hat{a}^\dagger(\omega)
	\right\}
	\ket{0}
\end{equation}

		\addcontentsline{toc}{section}{References}
		\printbibliography[title=References]
	\end{refsection}
	
	\chapter{Quantum transmission system}
	\textit{The following chapter presents an in-depth discussion of the classical and quantum characteristics of important optical components. In particular, we need to describe optical couplers, modulators and detectors in using a continuous-mode time-dependent quantum theory.}
	\begin{refsection}
		\section{Laser}

The Hamiltonian describing a laser is given by
\begin{equation}
	\hat{H}
	=
	\hat{H}_a
	+
	\hat{H}_m
	+
	\hat{H}_{ae}
	+
	\hat{H}_{am}
	+
	\hat{H}_{me}
\end{equation}
where we have the free Hamiltonian of the atoms inside the cavity
\begin{equation}
	\hat{H}_a
	=
	\sum_{n=1}^N
	\frac{1}{2}
	\omega
	\hat\sigma_{z,n}
\end{equation}
which are approximated as independent two-level spin-like system and the free photon field
\begin{equation}
	\hat{H}_m
	=
	\int\dd{\omega}
	\omega
	\hat{a}^\dagger(\omega)
	\hat{a}(\omega)
\end{equation}
The environment is coupled to the photons
\begin{equation}
	\hat{H}_{me}
	=
	\Gamma_p
	\left(
		\hat{a}(\omega)
		+
		\hat{a}^\dagger(\omega)
	\right)
\end{equation}
and the atoms
\begin{equation}
	\hat{H}_{ae}
	=
	\sum_{n=1}^N
	\Gamma_{a,n}
	\left(
		\hat\sigma_{+,n}
		+
		\hat\sigma_{-,n}
	\right)
\end{equation}
The atoms are coupled with the photon field with
\begin{equation}
	\hat{H}_{am}
	=
	ig
	\sum_{n=1}^N
	\left(
		\hat{a}^\dagger(\omega)
		\hat\sigma_{-,n}
		-
		\hat{a}(\omega)
		\hat\sigma_{-,n}^\dagger
	\right)
\end{equation}
		\section{Coupler}

Optical couplers, including optical splitters, redistribute two optical inputs among two outputs and are an essential passive component for almost every setup.\footnote{The optical splitter is a special case of the optical coupler where one of the two optical inputs is zero, or, more precisely, the vacuum state.}

A plethora of approaches towards the (quantum) beam splitter exists~\cite{Leonhardt2010,Gerry2005,Loudon2000} but are vague on assumptions, which, if not discussed, lead to misconception and confusion, for instance, regarding the phase and energy conservation.
We would therefore approach the optical coupler, or beam splitter, from two directions:
First, we approach the free-ray beam splitter from an experimentalist's perspective, considering the reflection and transmission properties of a beam splitter.
Second, we discuss the fiber coupler from a theoretical mode-coupling perspective leading to an interaction Hamiltonian.
Of course, both paths lead to the same unitary transformations, which we present in matrix and operator form.
Finally, we discuss the input-output relations for coherent states and the interpretation of a frequency-dependent beam splitter as an optical filter.

\subsection{Beam splitter}

The most commonly employed (free-ray) designs of the beam splitter are the cubic, plate, and pellicle beam splitters, see \Cref{fig:beam_splitter_types}.
\begin{figure}[htb]
    \centering
    \includegraphics{figures/tikz/beam-splitter-types}
    \caption{Different types of free-ray beam splitters: (a) Cubic beam splitter made of two triangular prisms glued at their base (grey). (b) Plate beam splitter made of a dielectric plate. (c) Pellicle beam splitter made from a thin membrane.}\label{fig:beam_splitter_types}
\end{figure}
The cubic beam splitter is made of two triangular prisms.
The interface between the two prisms is finished with a dielectric coating.
The outward-facing surface of the prisms is grafted with an \gls{ar} coating.\footnote{The incident angle of the electric field is perpendicular to the surface of the cubic beam splitter. As the reflection angle is equal to the incidence angle, we have back-reflection of the input fields.}
he pellicle beam splitter consists of a few micrometer thin membrane, optionally with a one-sided coating.
The plate beam splitter is like a thick pellicle beam splitter made of glass.

To deduce the relation between the in- and output fields, we sequentially couple a laser pulse into each input while monitoring both outputs with a spectrum analyzer, see \Cref{fig:beam_splitter_inputs_outputs}.
Assuming the beam splitter to be an \gls{lti} system, knowing the spectral shape of the laser pulse lets us infer the frequency responses of the beam splitter.
\begin{figure}[htb]
    \centering
    \includegraphics{figures/tikz/beam-splitter-cubic-plate}
    \caption{Cubic (left) and plate beam splitter (right) with the two input fields, $\hat{E}_1(\omega)$ and $\hat{E}_2(\omega)$, and two output fields, $\hat{E}_1^\prime(\omega)$ and $\hat{E}_2^\prime(\omega)$, labelled by the momentum representation of the electric field operators.}\label{fig:beam_splitter_inputs_outputs}
\end{figure}
Invoking the superposition principle for electromagnetic waves, we find the frequency responses of the beam splitter to relate the electric fields by
\begin{equation}
    \begin{pmatrix}
        \expval{\hat{E}_1^\prime(\omega)} \\
        \expval{\hat{E}_2^\prime(\omega)}
    \end{pmatrix}
    =
    \begin{pmatrix}
        t(\omega) & r^\prime(\omega)
        \\
        r(\omega) & t^\prime(\omega)
    \end{pmatrix}
    \begin{pmatrix}
		\expval{\hat{E}_1(\omega)} \\
        \expval{\hat{E}_2(\omega)}
    \end{pmatrix}
    \label{eq:beam_splitter_expval}
\end{equation}
wherein $r(\omega),r^\prime(\omega)$ and $t(\omega),t^\prime(\omega)$ are the complex reflection respective transmission coefficients.
The absolute values of the transmission, $\abs{t(\omega)}$ and $\abs{t^\prime(\omega)}$, and reflection coefficients, $\abs{r(\omega)}$ and $\abs{r^\prime(\omega)}$, determine the splitting ratio of the input power among the outputs.
The complex phase factor of the reflection and transmission coefficients characterizes the phase shifts the output fields concerning the input fields.
The beam splitter is a passive device implying the output energy to be bound by the input energy
\begin{equation}
    \abs{\expval{\hat{E}_1^\prime(\omega)}}^2
    +
    \abs{\expval{\hat{E}_2^\prime(\omega)}}^2
    \leq
    \abs{\expval{\hat{E}_1(\omega)}}^2
    +
    \abs{\expval{\hat{E}_2(\omega)}}^2
    \label{eq:beam_splitter_passive}
    ,
\end{equation}
or equivalently, constraining the reflection and transmission coefficients by
\begin{align}
    \abs{r(\omega)}^2+\abs{t(\omega)}^2
    &\leq
    1,
    &
    \abs{r^\prime(\omega)}^2+\abs{t^\prime(\omega)}^2
    &\leq
    1
    \label{eq:beam_splitter_coefficients_constraint}
    .
\end{align}
The equality of these inequalities is only true for lossless devices for which there is no back-scattering.\footnote{Using an optical circulator it is in principle possible to measure all \num{16} scattering parameters.}
Sometimes, one finds the claim~\cite[p.~129]{Haroche2006} that the matrix transformation in \cref{eq:beam_splitter_expval} is required to be symmetric (or reciprocal) due to Maxwell's equations.
However, only optical systems with a single dielectric layer are reciprocal~\cite{Potton2004}, but most physical beam splitters comprise multiple dielectric layers.\footnote{For example, cubic beam splitters typically have a coating followed by optical cement between the prisms breaking reciprocal symmetry of the system.}
It is possible to derive exact expressions of the complex reflection, $r(\omega),r^\prime(\omega)$, and transmission coefficients, $t(\omega),t^\prime(\omega)$ using classical wave optics and perfect knowledge of the dimensions and material properties.
For example, Hénault~\cite{Henault2015} derived an exact expression for the reflected and transmitted amplitudes of a plate beam splitter with one input and a single dielectric layer.
Likewise, Hamilton~\cite{Hamilton2000} discusses the cubic beam splitter with two inputs and different coatings.
In general, the complex reflection and transmission coefficients need to account for multiple reflections at different dielectric layers inside the beam splitter.

Inserting the mode expansion of the electric field operators, \cref{eq:electric_operator,eq:electric_negative_operator,eq:electric_positive_operator}, and using the linearity of the device and the expectation value, we recover the transformation for the annihilation operators, sometimes referred to as quantum modes,
\begin{equation}
    \begin{pmatrix}
        \hat{a}_1^\prime(\omega) \\
        \hat{a}_2^\prime(\omega)
    \end{pmatrix}
    =
    \begin{pmatrix}
        t(\omega) & r^\prime(\omega)
        \\
        r(\omega) & t^\prime(\omega)
    \end{pmatrix}
    \begin{pmatrix}
        \hat{a}_1(\omega) \\
        \hat{a}_2(\omega)
    \end{pmatrix}
    \label{eq:beam_splitter_annihilation}
\end{equation}
in agreement with Refs.~\cite{Leonhardt2010,Gerry2005}.

\subsection{Mode coupler}

Contrary to the direct coupling in free-ray beam splitters, a fiber or waveguide coupler uses indirect coupling through the evanescent field.
The evanescent field of an electromagnetic field does not propagate but decays exponentially.
We often observe evanescent fields at the boundary of waveguiding structures.
One must bring the waveguides in proximity for the evanescent fields of two waveguided modes to couple efficiently.
The range where the waveguides are close is the interaction length $l$, see \Cref{fig:waveguide_coupler}.
Over the interaction length, the two energy of the field modes oscillates back and forth between the two waveguides.
\begin{figure}[htb]
    \centering
    \includegraphics{figures/tikz/waveguide-coupler}
    \caption{Waveguide coupler with input quantum modes $\hat{a}_1(\omega)$ and $\hat{a}_2(\omega)$ coupled evanescent over an interaction length $l$ yielding the output quantum modes $\hat{a}_1^\prime(\omega)$ and $\hat{a}_2^\prime(\omega)$.}\label{fig:waveguide_coupler}
\end{figure}
The weak coupling through evanescent fields is conceptionally similar to weakly coupled harmonic oscillators.
Haroché~\cite[p.~131]{Haroche2006} successfully exploits the analogy to derive the quantum beam splitter transform from interaction theory.
We generalize his approach for the mode continuum derived in the previous chapter.

Let $\hat{a}_1(\omega)$ and $\hat{a}_2(\omega)$ be the annihilation operators of the first and second waveguide modes.
The interaction Hamiltonian
\begin{equation}
	\hat{H}_\text{int}
	=
	-
	\int\frac{\dd{\omega}}{2\pi}
	\left\{
		g(\omega)
		\hat{a}_1(\omega)
		\hat{a}_2^\dagger(\omega)
		+
		g^*(\omega)
		\hat{a}_1^\dagger(\omega)
		\hat{a}_2(\omega)
	\right\}
	,
\end{equation}
wherein $g(\omega)$ is a complex-valued coupling parameter encoding the material and geometry of the coupler, describes the transitions of excitations between the first and the second mode.
As the interaction Hamiltonian is time-independent, all but the first term in the Magnus expansion vanish, and the time evolution operator is
\begin{equation}
	\hat{U}_\text{int}
	=
	\exp\left\{
		i
		\int\dd{t^\prime}
		\int\frac{\dd{\omega}}{2\pi}
		\left\{
			g(\omega)
			\hat{a}_1(\omega)
			\hat{a}_2^\dagger(\omega)
			+
			g^*(\omega)
			\hat{a}_1^\dagger(\omega)
			\hat{a}_2(\omega)
		\right\}
	\right\}
\end{equation}
wherein the time integration is over the duration of the interaction.
Assuming the interaction to be limited to the interaction length $l$, the interaction duration $T$ is approximately equal to the interaction length $l$ divided by the group velocity $v_g(\omega)$.
The group velocity depends on the materials of the coupler, suggesting redefining the coupling parameter to include the different interaction durations, i.e.,
\begin{equation}
	\hat{U}_\text{int}
	=
	\exp\left\{
		i
		\int\frac{\dd{\omega}}{2\pi}
		\theta(\omega)
		\left\{
			\hat{a}_1(\omega)
			\hat{a}_2^\dagger(\omega)
			e^{-i\varphi(\omega)}
			+
			\hat{a}_1^\dagger(\omega)
			\hat{a}_2(\omega)
			e^{+i\varphi(\omega)}
		\right\}
	\right\}
\end{equation}
where the real-valued couplings $\theta(\omega)$ and $\varphi(\omega)$ implicitly depend on the materials and geometry of the waveguide coupler and the interaction length $l$.
We define the generator
\begin{equation}
	\hat{G}
	=
	-i
	\int\frac{\dd{\omega}}{2\pi}
	\theta(\omega)
	\left\{
		\hat{a}_1(\omega)
		\hat{a}_2^\dagger(\omega)
		e^{-i\varphi(\omega)}
		+
		\hat{a}_1^\dagger(\omega)
		\hat{a}_2(\omega)
		e^{+i\varphi(\omega)}
	\right\}
\end{equation}
and calculate the commutator of the generator with the annihilation operators
\begin{align}
	\comm{\hat{G}}{\hat{a}_1(\omega)}
	&=
	i
	\theta(\omega)
	\hat{a}_2(\omega)
	e^{+i\varphi(\omega)}
	&
	\comm{\hat{G}}{\hat{a}_2(\omega)}
	&=
	i
	\theta(\omega)
	\hat{a}_1(\omega)
	e^{-i\varphi(\omega)}
	.
\end{align}
The transformed annihilation operators turn out to be\footnote{Strictly speaking, the annihilation operators in the interaction picture have an additional factor $e^{-i\omega t}$.},
\begin{equation}
	\begin{split}
		\hat{a}_1^\prime(\omega)
		&=
		\hat{U}_\text{int}^\dagger
		\hat{a}_1(\omega)
		\hat{U}_\text{int}
		=
		e^{+\hat{G}}
		\hat{a}_1(\omega)
		e^{-\hat{G}}
		\\
		&=
		\hat{a}_1
		+
		\comm{\hat{G}}{\hat{a}_1}
		+
		\frac{1}{2!}
		\comm{\hat{G}}{\comm{\hat{G}}{\hat{a}_1}}
		+
		\frac{1}{3!}
		\comm{\hat{G}}{\comm{\hat{G}}{\comm{\hat{G}}{\hat{a}_1}}}
		+
		\dots
		\\
		&=
		\hat{a}_1(\omega)
		+
		i\theta(\omega)
		\hat{a}_2(\omega)
		e^{+i\varphi(\omega)}
		+
		\frac{1}{2!}
		\left(i\theta(\omega)\right)^2
		\hat{a}_1(\omega)
		+
		\frac{1}{3!}
		\left(i\theta(\omega)\right)^3
		\hat{a}_2(\omega)
		e^{+i\varphi(\omega)}
		+
		\dots
		\\
		&=
		\cos\theta(\omega)
		\hat{a}_1(\omega)
		+
		i\sin\theta(\omega)
		\hat{a}_2(\omega)
		e^{+i\varphi(\omega)}
	\end{split}
\end{equation}
and
\begin{equation}
	\begin{split}
		\hat{a}_2^\prime(\omega)
		&=
		\hat{U}_\text{int}^\dagger
		\hat{a}_2(\omega)
		\hat{U}_\text{int}
		=
		e^{+\hat{G}}
		\hat{a}_2(\omega)
		e^{-\hat{G}}
		\\
		&=
		\hat{a}_2
		+
		\comm{\hat{G}}{\hat{a}_2}
		+
		\frac{1}{2!}
		\comm{\hat{G}}{\comm{\hat{G}}{\hat{a}_2}}
		+
		\frac{1}{3!}
		\comm{\hat{G}}{\comm{\hat{G}}{\comm{\hat{G}}{\hat{a}_2}}}
		+
		\dots
		\\
		&=
		\hat{a}_2(\omega)
		+
		i\theta(\omega)
		\hat{a}_1(\omega)
		e^{-i\varphi(\omega)}
		+
		\frac{1}{2!}
		\left(i\theta(\omega)\right)^2
		\hat{a}_2(\omega)
		+
		\frac{1}{3!}
		\left(i\theta(\omega)\right)^3
		\hat{a}_1(\omega)
		e^{-i\varphi(\omega)}
		+
		\dots
		\\
		&=
		\cos\theta(\omega)
		\hat{a}_2(\omega)
		+
		i\sin\theta(\omega)
		\hat{a}_1(\omega)
		e^{-i\varphi(\omega)}
		,
	\end{split}
\end{equation}
where we used a kind of \gls{bch} formula, in agreement with Ref.~\cite[p.~131]{Haroche2006}.
In matrix notation, the transformation of the annihilation operators reads
\begin{equation}
	\begin{pmatrix}
        \hat{a}_1^\prime(\omega) \\
        \hat{a}_2^\prime(\omega)
    \end{pmatrix}
    =
    U(\omega)
    \begin{pmatrix}
        \hat{a}_1(\omega) \\
        \hat{a}_2(\omega)
    \end{pmatrix}
    =
    \begin{pmatrix}
        \cos\theta(\omega) & i\sin\theta(\omega)e^{+i\varphi} 
        \\
        i\sin\theta(\omega)e^{-i\varphi} & \cos\theta(\omega)
    \end{pmatrix}
    \begin{pmatrix}
        \hat{a}_1(\omega) \\
        \hat{a}_2(\omega)
    \end{pmatrix}
    \label{eq:waveguide_coupler_transformation}
    .
\end{equation}
Comparison of the annihilation operator transformation for the waveguide coupler, \cref{eq:waveguide_coupler_transformation}, and the beam splitter, \cref{eq:beam_splitter_annihilation}, our waveguide result implies lossless coupling.
Lossless coupling is essential for the transformed annihilation operators to satisfy the \gls{ccr}~\cite[p.~38]{Gerry2005}.
Modeling an absorbing coupler requires four quantum modes, two annihilation operators for the field, and two for a bosonic reservoir, see Ref.~\cite[p.~210]{Vogel2006} for details.

\subsection{Unitary operator transform}

The derived transforms of the free-ray beam splitter and the fiber or waveguide coupler, \cref{eq:waveguide_coupler_transformation,eq:beam_splitter_annihilation}, have in common that they are two-dimensional unitary matrices, which is not surprising since a unitary matrix transform conserves energy.
The optical coupler transform being linear and unitary is not surprising since the coupler is a linear passive device, which we further assumed to be lossless.
It presents itself to take the unitary matrix transform as the defining property of an ideal optical coupler.

A general decomposition of a two-dimensional unitary matrix is the product~\cite[p.~95]{Leonhardt2010}
\begin{equation}
	U(\omega)
	=
	e^{i\Lambda/2}
	\begin{pmatrix}
		e^{+i\Phi/2} & 0 \\
		0 & e^{-i\Phi/2}
	\end{pmatrix}
	\begin{pmatrix}
		\cos(\Theta/2) & \sin(\Theta/2) \\
		-\sin(\Theta/2) & \cos(\Theta/2)
	\end{pmatrix}
	\begin{pmatrix}
		e^{+i\Psi/2} & 0 \\
		0 & e^{-i\Psi/2}
	\end{pmatrix}
	\label{eq:unitary_matrix}
\end{equation}
wherein we suppress the frequency-dependence of the real parameters, $\Lambda(\omega),\Theta(\omega),\Psi(\omega),\Phi(\omega)$, for clarity.
We can read the matrix decomposition, \cref{eq:unitary_matrix}, as first adding a global and relative phase shift, $\Lambda/2,\pm\Psi/2$, to the incident fields, rotating (mixing) the field amplitudes by the angle $\Theta/2$, and adding another relative phase shift of $\pm\Psi/2$ to the outgoing fields.

While the unitary matrix transforms the annihilation operators and the field amplitudes, it cannot transform a generic quantum state.
In the previous subsection, we found a time evolution operator $\hat{U}$ from linear mode coupling theory, which related to the unitary matrix transform $U$ via
\begin{equation}
	U(\omega)
	\begin{pmatrix}
		\hat{a}_1(\omega) \\
		\hat{a}_2(\omega)
	\end{pmatrix}
	=
	\begin{pmatrix}
		\hat{a}_1^\prime(\omega) \\
		\hat{a}_2^\prime(\omega)
	\end{pmatrix}
	=
	\begin{pmatrix}
		\hat{U}^\dagger\hat{a}_1(\omega)\hat{U} \\
		\hat{U}^\dagger\hat{a}_2(\omega)\hat{U}
	\end{pmatrix}
	=
	\hat{U}^\dagger
	\begin{pmatrix}
		\hat{a}_1(\omega) \\
		\hat{a}_2(\omega)
	\end{pmatrix}
	\hat{U}
	\label{eq:unitary_matrix_operator}
	.
\end{equation}
The unitary operators corresponding to the unitary matrix decomposition in \cref{eq:unitary_matrix} are the Jordan-Schwinger operators\footnote{Generalized to a frequency continuum from Ref.~\cite[p.~97]{Leonhardt2010}.}
\begin{align}
	\hat{L}_t
	&=
	\frac{1}{2}
	\int\frac{\dd{\omega}}{2\pi}
	\begin{pmatrix}
		\hat{a}_1(\omega) \\
		\hat{a}_2(\omega)
	\end{pmatrix}^\dagger
	\mathbb{1}_2
	\begin{pmatrix}
		\hat{a}_1(\omega) \\
		\hat{a}_2(\omega)
	\end{pmatrix}
	=
	\frac{1}{2}
	\int\frac{\dd{\omega}}{2\pi}
	\left(
		\hat{a}_1^\dagger(\omega)
		\hat{a}_1(\omega)
		+
		\hat{a}_2^\dagger(\omega)
		\hat{a}_2(\omega)
	\right)
	\\
	\hat{L}_x
	&=
	\frac{1}{2}
	\int\frac{\dd{\omega}}{2\pi}
	\begin{pmatrix}
		\hat{a}_1(\omega) \\
		\hat{a}_2(\omega)
	\end{pmatrix}^\dagger
	\sigma_x
	\begin{pmatrix}
		\hat{a}_1(\omega) \\
		\hat{a}_2(\omega)
	\end{pmatrix}
	=
	\frac{1}{2}
	\int\frac{\dd{\omega}}{2\pi}
	\left(
		\hat{a}_1^\dagger(\omega)
		\hat{a}_2(\omega)
		+
		\hat{a}_2^\dagger(\omega)
		\hat{a}_1(\omega)
	\right)
	\\
	\hat{L}_y
	&=
	\frac{1}{2}
	\int\frac{\dd{\omega}}{2\pi}
	\begin{pmatrix}
		\hat{a}_1(\omega) \\
		\hat{a}_2(\omega)
	\end{pmatrix}^\dagger
	\sigma_y
	\begin{pmatrix}
		\hat{a}_1(\omega) \\
		\hat{a}_2(\omega)
	\end{pmatrix}
	=
	\frac{i}{2}
	\int\frac{\dd{\omega}}{2\pi}
	\left(
		\hat{a}_2^\dagger(\omega)
		\hat{a}_1(\omega)
		-
		\hat{a}_1^\dagger(\omega)
		\hat{a}_2(\omega)
	\right)
	\\
	\hat{L}_z
	&=
	\frac{1}{2}
	\int\frac{\dd{\omega}}{2\pi}
	\begin{pmatrix}
		\hat{a}_1(\omega) \\
		\hat{a}_2(\omega)
	\end{pmatrix}^\dagger
	\sigma_z
	\begin{pmatrix}
		\hat{a}_1(\omega) \\
		\hat{a}_2(\omega)
	\end{pmatrix}
	=
	\frac{1}{2}
	\int\frac{\dd{\omega}}{2\pi}
	\left(
		\hat{a}_1^\dagger(\omega)
		\hat{a}_1(\omega)
		-
		\hat{a}_2^\dagger(\omega)
		\hat{a}_2(\omega)
	\right)
\end{align}
where $\sigma_1,\sigma_2,\sigma_3$ denote the two-dimensional Pauli matrices.
The Jordan-Schwinger operators satisfy the angular-momentum commutation algebra~\cite[p.~97]{Leonhardt2010}
\begin{align}
	\comm{\hat{L}_i}{\hat{L}_j}
	&=
	i\varepsilon_{ijk}\hat{L}^k
	&
	\comm{\hat{L}_t}{\hat{L}_i}
	&=
	0
\end{align}
and act as generator for the individual components of the matrix decomposition in \cref{eq:unitary_matrix}.
The generator of the unitary matrix, \cref{eq:unitary_matrix}, is
\begin{equation}
	\hat{U}
	=
	e^{i\Lambda\hat{L}_t}
	e^{i\Phi\hat{L}_z}
	e^{i\Theta\hat{L}_y}
	e^{i\Psi\hat{L}_z}
	\label{eq:unitary_operator}
	.
\end{equation}
The inverse of the unitary operator, \cref{eq:unitary_operator}, can be written
\begin{equation}
	\begin{split}
		\hat{U}(\Lambda,\Phi,\Theta,\Psi)^\dagger
		&=
		e^{-i\Psi\hat{L}_z}
		e^{-i\Theta\hat{L}_y}
		e^{-i\Phi\hat{L}_z}
		e^{-i\Lambda\hat{L}_t}
		\\
		&=
		e^{-i\Lambda\hat{L}_t}
		e^{-i\Psi\hat{L}_z}
		e^{-i\Theta\hat{L}_y}
		e^{-i\Phi\hat{L}_z}
		\\
		&=
		\hat{U}(-\Lambda,-\Psi,-\Theta,-\Phi)
		,
	\end{split}
	\label{eq:unitary_operator_inverse}
\end{equation}
where we used that $\hat{L}_t$ commutes with the other Jordan-Schwinger operators, and can be used to find the reversed transform,
\begin{equation}
	\hat{U}
	\begin{pmatrix}
		\hat{a}_1(\omega) \\
		\hat{a}_2(\omega)
	\end{pmatrix}
	\hat{U}^\dagger
	=
	U(\omega)^\dagger
	\begin{pmatrix}
		\hat{a}_1(\omega) \\
		\hat{a}_2(\omega)
	\end{pmatrix}
	\label{eq:unitary_matrix_operator_reverse}
	,
\end{equation}
 of the annihilation operators.

\subsection{Coherent state transform}

Let us now consider the action of the ideal coupler on the tensor product of coherent input states\footnote{Other quantum states typically produce entangled output states, see, for instance, Ref.~\cite{Windhager2011}, which is not of interest here.}
\begin{equation}
	\ket*{\vb{\alpha}(t)}
	=
	\ket*{\alpha_1(t),\alpha_2(t)}
	.
\end{equation}
The output states are given by applying the unitary evolution operator $\hat{U}$, e.g., \cref{eq:unitary_operator}, onto the input state
\begin{equation}
	\hat{U}
	\ket*{\vb{\alpha}(t)}
	=
	\hat{U}
	\hat{D}\left[\vb{\alpha}(t)\right]
	\hat{U}^\dagger
	\hat{U}
	\ket*{0,0}
	=
	\hat{U}
	\hat{D}\left[\vb{\alpha}(t)\right]
	\hat{U}^\dagger
	\vacuum
	,
\end{equation}
wherein we inserted $\mathbb{1}=\hat{U}^\dagger\hat{U}$ in the second step and we used the invariance of the vacuum state in the third step.
The transformed displacement operator reads\footnote{We adopt matrix notation as in Ref.~\cite[p.~206]{Vogel2006} to have our result independent of a particular choice of the unitary matrix.}
\begin{equation}
	\begin{split}
		\hat{U}
		\hat{D}\left[\vb{\alpha}(t)\right]
		\hat{U}^\dagger
		&=
		\hat{U}
		\exp\left\{
			\int\frac{\dd{\omega}}{2\pi}
			\left\{
				\vb{\alpha}(\omega)^\trans
				e^{-i\omega t}
				\begin{pmatrix}
					\hat{a}_1^\dagger(\omega) \\
					\hat{a}_2^\dagger(\omega)
				\end{pmatrix}
				-
				\vb{\alpha}(\omega)^\dagger
				e^{+i\omega t}
				\begin{pmatrix}
					\hat{a}_1(\omega) \\
					\hat{a}_2(\omega)
				\end{pmatrix}
			\right\}
		\right\}
		\hat{U}^\dagger
		\\
		&=
		\exp\left\{
			\int\frac{\dd{\omega}}{2\pi}
			\left\{
				\vb{\alpha}^\trans
				e^{-i\omega t}
				\hat{U}
				\begin{pmatrix}
					\hat{a}_1^\dagger(\omega) \\
					\hat{a}_2^\dagger(\omega)
				\end{pmatrix}
				\hat{U}^\dagger
				-
				\vb{\alpha}^\dagger
				e^{+i\omega t}
				\hat{U}
				\begin{pmatrix}
					\hat{a}_1(\omega) \\
					\hat{a}_2(\omega)
				\end{pmatrix}
				\hat{U}^\dagger
			\right\}
		\right\}
	\end{split}
	\label{eq:displacement_operator_transformed}
	,
\end{equation}
wherein we used the operator identity
\begin{equation}
	\hat{U}
	e^{\hat{A}}
	\hat{U}^\dagger
	=
	\sum_{n=0}^\infty
	\frac{1}{n!}
	\hat{U}
	\hat{A}^n
	\hat{U}^\dagger
	=
	\sum_{n=0}^\infty
	\frac{1}{n!}
	\hat{U}
	\hat{A}
	\hat{U}^\dagger
	\cdots
	\hat{U}
	\hat{A}
	\hat{U}^\dagger
	=
	\sum_{n=0}^\infty
	\frac{1}{n!}
	\left(
		\hat{U}
		\hat{A}
		\hat{U}^\dagger
	\right)^n
	=
	e^{\hat{U}\hat{A}\hat{U}^\dagger}
\end{equation}
in the second step to move the unitary operators into the argument of the exponential.
We already expressed the transformed annihilation operators in the second term of the exponential using the unitary matrix in \cref{eq:unitary_matrix_operator_reverse}.
The transformed creation operators in the first term can be brought into a similar form, i.e.,
\begin{equation}
	\begin{split}
		\hat{U}
		\begin{pmatrix}
			\hat{a}_1^\dagger(\omega) \\
			\hat{a}_2^\dagger(\omega)
		\end{pmatrix}
		\hat{U}^\dagger
		&=
		\begin{pmatrix}
			\left[
				\hat{U}
				\hat{a}_1(\omega)
				\hat{U}^\dagger
			\right]^\dagger \\
			\left[
				\hat{U}
				\hat{a}_2(\omega)
				\hat{U}^\dagger
			\right]^\dagger
		\end{pmatrix}
		=
		\left[
			\begin{pmatrix}
				\hat{U}
				\hat{a}_1(\omega)
				\hat{U}^\dagger
				\\
				\hat{U}
				\hat{a}_2(\omega)
				\hat{U}^\dagger
			\end{pmatrix}^\dagger
		\right]^\trans
		=
		\left[
			\left(
				U(\omega)
				\begin{pmatrix}
					\hat{a}_1(\omega) \\
					\hat{a}_2(\omega)
				\end{pmatrix}
			\right)^\dagger
		\right]^\trans		
		\\
		&=
		\left[
			\begin{pmatrix}
				\hat{a}_1(\omega)
				\\
				\hat{a}_2(\omega)
			\end{pmatrix}^\dagger
			U(\omega)
		\right]^\trans
		=
		U(\omega)^\trans
		\begin{pmatrix}
			\hat{a}_1^\dagger(\omega) \\
			\hat{a}_2^\dagger(\omega)
		\end{pmatrix}
		.
	\end{split}
\end{equation}
Inserting the previous result back into the transformed displacement operator, \cref{eq:displacement_operator_transformed}, we factor the unitary matrix to the Fourier amplitudes
\begin{equation}
	\begin{split}
		\hat{D}^\prime\left[\vb{\alpha}(t)\right]
		&=
		\exp\left\{
			\int\frac{\dd{\omega}}{2\pi}
			\left\{
				\vb{\alpha}(\omega)^\trans
				e^{-i\omega t}
				U(\omega)^\trans
				\begin{pmatrix}
					\hat{a}_1^\dagger(\omega) \\
					\hat{a}_2^\dagger(\omega)
				\end{pmatrix}
				-
				\vb{\alpha}(\omega)^\dagger
				e^{+i\omega t}
				U(\omega)^\dagger
				\begin{pmatrix}
					\hat{a}_1(\omega) \\
					\hat{a}_2(\omega)
				\end{pmatrix}
			\right\}
		\right\}
		\\
		&=
		\exp\left\{
			\int\frac{\dd{\omega}}{2\pi}
			\left\{
				\left(
					U(\omega)
					\vb{\alpha}(\omega)
				\right)^\trans
				e^{-i\omega t}
				\begin{pmatrix}
					\hat{a}_1^\dagger(\omega) \\
					\hat{a}_2^\dagger(\omega)
				\end{pmatrix}
				-
				\left(
					U(\omega)
					\vb{\alpha}(\omega)
				\right)^\dagger
				e^{+i\omega t}
				\begin{pmatrix}
					\hat{a}_1(\omega) \\
					\hat{a}_2(\omega)
				\end{pmatrix}
			\right\}
		\right\}
	\end{split}
\end{equation}
in agreement with Ref.~\cite[p.~210]{Vogel2006}.
The transformed Fourier amplitudes are given by the matrix product
\begin{equation}
	\vb{\alpha}^\prime(\omega)
	=
	U(\omega)
	\vb{\alpha}(\omega)
	\label{eq:coupler_coherent_amplitudes_frequency}
	.
\end{equation}
The product in frequency space implies a convolution in the time domain, i.e.,
\begin{equation}
	\vb{\alpha}^\prime(t)
	=
	\left(U\conv\vb{\alpha}\right)(t)
	=
	\int\frac{\dd{\omega}}{2\pi}
	U(\omega)
	\vb{\alpha}(\omega)
	e^{+i\omega t}
	\label{eq:coupler_coherent_amplitudes_time}
	,
\end{equation}
and we conclude that a tensor product of coherent states transforms under an ideal optical coupler according to
\begin{align}
	\hat{U}
	\ket*{\alpha(t)}
	&=
	\ket*{\left(U\conv\vb{\alpha}\right)(t)}	
	&
	U(t)
	&=
	\int\frac{\dd{\omega}}{2\pi}
	U(\omega)
	e^{+i\omega t}
	\label{eq:coupler_coherent_state}
\end{align}
wherein $U(\omega)$ is a two-dimensional unitary matrix characterizing the coupler.

The fact that the ideal optical coupler only transforms the amplitudes of the coherent input states is specific to coherent states.
The coherent states owe this special property due to having a Poisson number distribution and the Poisson distribution being memoryless.
As a consequence, the coherent output states are independent and consider a subsystem by performing a partial trace, e.g.,
\begin{equation}
	\trace_2\left\{
		\ketbra{\alpha,\beta}
	\right\}
	=
	\trace_2\left\{
		\ketbra{\alpha}
		\otimes
		\ketbra{\beta}
	\right\}
	=
	\ketbra{\alpha}
	\otimes
	\trace_2\left\{
		\ketbra{\beta}
	\right\}
	=
	\ketbra{\alpha}
	,
\end{equation}
where we used
\begin{equation}
	\trace_2\left\{
		\ketbra{\beta}
	\right\}
	=
	\sum_{n=0}^\infty
	\braket{n}{\beta}
	\braket{\beta}{n}
	=
	\sum_{n=0}^\infty
	\abs{\braket{n}{\beta}}^2
	=
	1
	,
\end{equation}
is equivalent to the projection of the subsystem
\begin{equation}
	\trace_2\left\{
		\ketbra{\alpha,\beta}
	\right\}
	=
	\ketbra{\alpha}
	=
	\hat{P}_1
	\ketbra{\alpha,\beta}
	\hat{P}_1
	.
\end{equation}
For non-coherent quantum states, it is not correct to project out a subsystem as the partial trace does not generally yield a mixed but a pure state.

\subsection{Spectral filter}

Our considerations have so far been quite general except for restricting ourselves to coherent input states.
We will now discuss two applications of our results:
First, we consider the special case of an optical being used as a splitter. 
Second, we consider a coupler as an optical filter relevant to signal processing and quantum information theory.

An ideal optical splitter redistributes the power of one input among two outputs and is a special case of the optical coupler with one input state being the vacuum state.
Using \cref{eq:coupler_coherent_amplitudes_frequency}, we find the transformed Fourier amplitudes to be
\begin{equation}
	\begin{split}
		\begin{pmatrix}
			\alpha_1^\prime(\omega) \\
			\alpha_2^\prime(\omega)
		\end{pmatrix}
		&=
		e^{i\Lambda/2}
		\begin{pmatrix}
			\cos(\Theta/2)
			e^{i\left(+\Phi+\Psi\right)/2}
			&
			\sin(\Theta/2)
			e^{i\left(+\Phi-\Psi\right)/2}
			\\
			-
			\sin(\Theta/2)
			e^{i\left(-\Phi+\Psi\right)/2}
			&
			\cos(\Theta/2)
			e^{i\left(-\Phi-\Psi\right)/2}
		\end{pmatrix}
		\begin{pmatrix}
			\alpha(\omega) \\
			0
		\end{pmatrix}
		\\
		&=
		\alpha(\omega)
		\begin{pmatrix}
			+
			\cos(\Theta/2)
			e^{+i\Phi/2}
			\\
			-
			\sin(\Theta/2)
			e^{-i\Phi/2}
		\end{pmatrix}
		e^{i(\Lambda+\Psi)/2}
	\end{split}
\end{equation}
wherein we again suppressed the frequency dependency of the splitting parameters.
Instead of choosing a parametrization for the splitting coefficients, which directly ensures energy conservation, we can more generally write
\begin{align}
	\begin{pmatrix}
		\alpha_1^\prime(\omega) \\
		\alpha_2^\prime(\omega)
	\end{pmatrix}
	&=
	\alpha(\omega)
	\begin{pmatrix}
		c_1(\omega) \\
		c_2(\omega)
	\end{pmatrix}
	&
	\abs{c_1(\omega)}^2
	+
	\abs{c_2(\omega)}^2
	&=
	1
	.
\end{align}
Assuming the Fourier transform of $c_1(\omega),c_2(\omega)$ to be well-defined, we find the coherent output states to be
\begin{equation}
	\hat{U}
	\ket*{\alpha(t),0}
	=
	\ket*{\left(c_1\conv\alpha\right)(t),\left(c_2\conv\alpha\right)(t)}
	\label{eq:splitter_state_convolution}
\end{equation}
according to \cref{eq:coupler_coherent_state}.
If we further assume the signal $\alpha(t)$ to be bandwidth-limited to $B$ and the splitting coefficients to be approximately constant over the bandwidth, i.e.,
\begin{align}
	c_1(\omega)
	&\approx
	c_1(\omega_0)
	&
	c_2(\omega)
	&\approx
	c_2(\omega_0)
	&
	\forall
	\omega
	&\in
	B
	,
\end{align}
the output state takes the simple form
\begin{equation}
	\hat{U}
	\ket*{\alpha(t),0}
	=
	\ket*{c_1\alpha(t),c_2\alpha(t)}
\end{equation}
and the splitter only redistributes the signal power among the outputs while leaving the signal itself unaltered.

\Cref{eq:splitter_state_convolution} already suggests the similarity of the ideal optical splitter with an optical filter, the difference being that only one output matters for the optical filter.
To remove the second output, we perform a partial trace over the second subsystem, equivalent to applying the projection operator to \cref{eq:splitter_state_convolution}, i.e.,
\begin{equation}
	\hat{P}_1
	\hat{U}
	\ket*{\alpha(t),0}
	=
	\ket*{\left(h\conv\alpha\right)(t)}
	\label{eq:filter_state}
\end{equation}
where we introduced the optical filter function $h=c_1$.
Ref.~\cite[p.~199]{Vogel2006} discusses a spectral filter made of a dielectric slab of thickness $l$ and refractive index $n$, see \Cref{fig:dielectric_filter}, with incoming and outgoing quantum modes to each side.
\begin{figure}[htb]
    \centering
    \includegraphics{figures/tikz/dielectric-filter}
    \caption{Dielectric slab of thickness $l$ and refractive index $n$ used as a spectral filter with incident quantum modes, denoted by the annihilation operators, $\hat{a}_1(\omega)$ from the left, and $\hat{a}_2(\omega)$ from the right, and outgoing quantum modes, $\hat{a}_1^\prime(\omega)$ to the left, and $\hat{a}_2(\omega)$ to the right.}\label{fig:dielectric_filter}
\end{figure}
Let $\hat{a}_1(\omega)$ be the signal mode approaching the dielectric slab from the left.
Let us assume that the mode $\hat{a}_2(\omega)$ is in vacuum and that we are only interested in the mode $\hat{a}_2^\prime(\omega)$, outgoing to the right.
Then the optical filter function in \cref{eq:filter_state} is equal to the transmission coefficient of the dielectric slab which is equal to~\cite[p.~199]{Vogel2006}
\begin{align}
	h(\omega)
	&=
	\frac{1-r^2}{1-r^2\exp(2i\omega nl)}
	\exp\left[-i(n-1)l\omega\right]
	&
	r^2
	&=
	\left(\frac{n-1}{n+1}\right)^2
	.
\end{align}
By carefully selecting the geometry and dielectric (layers), it should be possible to tailor the transmission coefficient in a specific bandwidth to implement a custom optical filter function $h$.
		\section{Modulators}

The (electro-optical) modulators allow us to encode an electrical signal onto an optical carrier, and in that sense, are the interface between the electrical and optical domains.
The domain crossover occurs at the electro-optical phase modulator, which we attempt to describe as a (quantum) nonlinear mixing process mediated by the dielectric, in a first part.
In a second part, we construct, in the optical domain, a complex, in-phase and quadrature, amplitude modulator using \gls{mzi}s driven by electro-optical phase modulators.

\subsection{Phase modulator}

In an electro-optical phase modulator, an electrical signal changes the refractive index of an optical transmission medium, causing a phase shift.
The linear electro-optic effect, also known as the Pockels effect, characterizes a linear refractive index change proportional to the amplitude of an external electric field~\cite[Ch.~18]{Saleh2007}.
It is present in noncentrosymmetric crystals~\cite[p.~2]{Boyd2020}, e.g., lithium niobate~\cite[p.~237]{Yariv1984}, commonly used in photonic integration.
\begin{figure}[htb]
    \centering
    \includegraphics{figures/tikz/phase-modulator}
    \caption{Traveling-wave electro-optical phase modulator comprising two electrodes of length $L$ (black): The electrodes are driven by a sinusoidal voltage signal creating an \gls{rf} field with frequency $\Omega_0$ between the electrodes. An optical field with frequency $\omega$ reaches the phase modulator from the left, and an optical field with multiple frequency components, $\omega,\omega\pm\Omega_0$, exits the phase modulator to the right.}\label{fig:phase_modulator}
\end{figure}
\Cref{fig:phase_modulator} depicts a traveling-wave electro-optical phase modulator.
The phase modulation signal is applied to the electrodes creating a traveling-wave \gls{rf} field between the two electrodes.
The linear-electro optical effect couples the different \gls{rf} and optical frequency components.
The output field contains sidebands, $\omega\pm\Omega_0$, at the modulation frequency, $\Omega_0$.

The modulation frequency, $\Omega$, is typically many magnitudes smaller than the optical frequency, $\omega$, and the phase-shift due to the refractive index change appears static from the optical domain.
In this case, we expect the complex amplitude of the optical signal leaving the phase modulator to be
\begin{equation}
	\alpha^\prime(t)
	=
	\alpha(t)
	e^{-i\varphi(t)}
	\label{eq:phase_modulation}
	,
\end{equation}
wherein $\alpha(t)$ is the initial amplitude and $\varphi(t)$ is some time-dependent phase.
To a good approximation, the phase modulator is an \gls{lti} system for which the time-dependent phase signal is a convolution
\begin{equation}
	\varphi(t)
	=
	\left(h\conv x\right)(t)
	=
	\int\dd{t^\prime}
	h(t^\prime)
	x(t-t^\prime)
\end{equation}
with $h(t)$ being the linear time-response of the phase modulator system and $x(t)$ being the voltage signal driving the modulator.
As a rough estimate of when the phase is effectively static concerning the optical field, we can compare the transmission time of the optical signal through the modulator with the period of the optical signal.\footnote{The transmission time $T$ is equal to the group velocity, $v_g(\omega)$, divided by the length of the modulator, $L$.}


While the former classical approach is perfectly sufficient, it does not fit well into our quantum mechanical framework regarding interactions and unitary operators.
In the following, we present the essence of Ref.~\cite{QuesadaMejia2015} and Ref.~\cite{Horoshko2018} to get an insight into quantum optical frequency conversion.
To simplify the discussion, we approximate the electric field (averages) inside the dielectric with free electric field operators at the cost of correctness relating, e.g., to phase-matching.
One particular challenge regarding time-dependent phase modulation is that there exists no simple representation in the frequency domain for arbitrary modulation signals. 
The best we can do is assume a sinusoidal modulation, e.g.,
\begin{equation}
	\varphi(t)
	=
	\beta_0
	\sin(\Omega_0t+\phi)
	\label{eq:sinusoidal_phase_signal}
	,
\end{equation}
for which we can use the Jacobi-Anger expansion~\cite[eq.~23]{Horoshko2018}
\begin{equation}
	e^{-i\beta_0\sin(\Omega_0t+\phi)}
	=
	\sum_{m\in\mathbb{Z}}
	J_m(\beta_0)
	e^{-im(\Omega_0t+\phi)}
	\label{eq:jacobi_anger_expansion}
\end{equation}
wherein $J_m(\beta_0)$ is the $m$th Bessel function of the first kind.
Inserting \cref{eq:sinusoidal_phase_signal} into \cref{eq:phase_modulation} and performing the Jacobi-Anger expansion, we identify the Fourier transform with
\begin{equation}
	\begin{split}
		\alpha^\prime(\omega)
		&=
		\sum_{m\in\mathbb{Z}}
		J_m(\beta_0)
		e^{-im\phi}
		\int\dd{t}
		\alpha(t)
		e^{-i(\omega+m\Omega_0)t}
		\\
		&=
		\sum_{m\in\mathbb{Z}}
		J_m(\beta_0)
		e^{-im\phi}
		\alpha(\omega+m\Omega_0)
		.
	\end{split}
	\label{eq:sinusoidal_phase_signal_fourier}
\end{equation}
Sinusoidal phase modulation with frequency $\Omega_0$ creates infinitely many sidebands around the carrier frequency $\omega$.
As we can write an arbitrary time-dependent signal as a sum of sinusoidals of different frequencies, we have to deal with an integral of infinite sums, making arbitrary phase-modulation untractable.

Creating new frequency components requires a nonlinear interaction, not unlike classical signal-processing, where a diode is used as a nonlinear element for frequency conversion.
In the appendix, particular \Cref{sec:frequency_convserion}, we discuss nonlinear processes mediated by the electric susceptibility of a nonabsorptive dielectric and find the
interaction Hamiltonian corresponding to frequency conversion to be
\begin{equation}
	\hat{H}_\text{int}
	=
	\int\frac{\dd{\omega}}{2\pi}
	\int\frac{\dd{\Omega}}{2\pi}
	g(\omega,\Omega)
	\hat{a}^\dagger(\omega)
	\hat{a}(\Omega)
	\hat{a}(\omega-\Omega)
	+
	\text{h.c.}
	\label{eq:frequency_conversion_interaction}
	,
\end{equation}
wherein $g(\omega,\Omega)$ encodes the geometry and properties of the dielectric material.
The coupling parameter $g(\omega,\Omega)$ is only non-zero for $\omega$ being an optical and $\Omega$ being an electrical frequency.
The operators in \cref{eq:frequency_conversion_interaction} describe the simultaneous absorption of an "electric" photon with \gls{rf} $\Omega$, $\hat{a}(\Omega)$, and an "optical" photon with frequency $\omega-\Omega$, $\hat{a}(\omega-\Omega)$, to create an "optical" photon at a higher frequency $\omega$, $\hat{a}^\dagger(\omega)$.
We are not interested in the quantum properties of the \gls{rf} field and assume it to be in a coherent state with complex amplitude $\beta(t)$
\begin{equation}
	\hat{H}_\text{int}
	=
	\int\frac{\dd{\omega}}{2\pi}
	\int\frac{\dd{\Omega}}{2\pi}
	g(\omega,\Omega)
	\beta(\Omega)
	\hat{a}^\dagger(\omega)
	\hat{a}(\omega-\Omega)
	+
	\text{h.c.}
	,
	\label{eq:frequency_conversion_interaction_approx}
\end{equation}
wherein $\beta(\Omega)$ is the Fourier amplitude corresponding to $\beta(t)$.
A sinusoidal modulation signal has a single frequency component,
\begin{align}
	\beta(t)
	&=
	\beta_0
	e^{i\Omega_0t}
	&
	\beta(\Omega)
	&=
	\beta_0
	(2\pi)
	\delta^{(1)}(\Omega-\Omega_0)
	,
\end{align}
Inserting the frequency representation of the sinusoidal modulation into the interaction Hamiltonian, \cref{eq:frequency_conversion_interaction_approx}, we find
\begin{equation}
	\hat{H}_\text{int}
	=
	\beta_0
	\int\frac{\dd{\omega}}{2\pi}
	g(\omega,\Omega_0)
	\hat{a}^\dagger(\omega)
	\hat{a}(\omega+\Omega_0)
	+
	\text{h.c.}
	\label{eq:frequency_conversion_interaction_sinusoidal}
	.
\end{equation}
Let the optical signal be limited to the bandwidth $B$, then
according to the mean-value theorem for integrals, there exists an $\omega_0\in B$ such that we can write
\begin{equation}
	\begin{split}
		\hat{H}_\text{int}
		&=
		\beta_0
		g(\omega_0,\Omega_0)
		\int\frac{\dd{\omega}}{2\pi}
		\hat{a}^\dagger(\omega)
		\hat{a}(\omega+\Omega_0)
		+
		\beta_0^*
		g(\omega_0,\Omega_0)^*
		\int\frac{\dd{\omega}}{2\pi}
		\hat{a}^\dagger(\omega)
		\hat{a}(\omega-\Omega_0)
		\\
		&=
		\beta_0
		g(\omega_0,\Omega_0)
		\int\frac{\dd{\omega}}{2\pi}
		\hat{a}^\dagger(\omega-\Omega_0)
		\hat{a}(\omega)
		+
		\beta_0^*
		g(\omega_0,\Omega_0)^*
		\int\frac{\dd{\omega}}{2\pi}
		\hat{a}^\dagger(\omega+\Omega_0)
		\hat{a}(\omega)
	\end{split}
	\label{eq:frequency_conversion_interaction_meanvalue}
\end{equation}
where we performed an integration variable substitution in the second step.

We define the up- and down-conversion operators for the frequency $\Omega_0$ as
\begin{equation}
	\hat{T}_\pm(\Omega_0)
	=
	\int\frac{\dd{\omega}}{2\pi}
	\hat{a}^\dagger(\omega\pm\Omega_0)
	\hat{a}(\omega)
	\label{eq:frequency_conversion_operator}
\end{equation}
which are related via the hermitian conjugate, $\hat{T}_+(\Omega_0)=\hat{T}_-(\Omega_0)^\dagger$, and satisfy the commutation relations
\begin{align}
	\comm{\hat{T}_\pm(\Omega_0)}{\hat{a}(\omega)}
	&=
	-
	\hat{a}(\omega\mp\Omega_0)
	&
	\comm{\hat{T}_\pm(\Omega_0)}{\hat{a}^\dagger(\omega)}
	&=
	+
	\hat{a}^\dagger(\omega\pm\Omega_0)
	,
\end{align}
which can be used to up- or donw-convert the frequency of a photon by $\Omega_0$.

The time-evolution operator corresponding to the interaction Hamiltonian in \cref{eq:frequency_conversion_interaction_meanvalue} turns out to be
\begin{equation}
	\begin{split}
		\hat{U}(\theta,\Omega_0,\varphi)
		=
		\exp\left\{
			-iT
			\hat{H}_\text{int}
		\right\}
		=
		\exp\left\{
			-i
			\frac{1}{2}
			\theta
			\left[
				\hat{T}_+(\Omega_0)
				e^{-i\varphi}
				+
				\hat{T}_-(\Omega_0)
				e^{+i\varphi}
			\right]
		\right\}
	\end{split}
	\label{eq:frequency_conversion_evolution}
\end{equation}
where we new parameters $\Theta,\varphi$ relate to the old coupling parameters via
\begin{equation}
	\beta_0^*
	g(\omega_0,\Omega_0)^*
	T
	=
	\frac{1}{2}
	\theta
	e^{+i\varphi}
\end{equation}
with $T$ being the interaction time approximately equal to the transmit time of the light through the phase modulator.\footnote{Strictly speaking, the transmit time is frequency-dependent requiring a frequency response filter instead of coupling constant.}
Before, we transform the annihilation operator, we define the generator
\begin{equation}
	\hat{G}(\varphi,\Omega_0)
	=
	\hat{T}_+(\Omega_0)
	e^{-i\varphi}
	+
	\hat{T}_-(\Omega_0)
	e^{+i\varphi}
	\label{eq:frequency_conversion_generator}
\end{equation}
which relates to the evolution operator via
\begin{equation}
	\hat{U}(\theta,\Omega_0,\varphi)
	=
	\exp\left\{
		-i\frac{1}{2}
		\theta
		\hat{G}(\varphi,\Omega_0)
	\right\}
	.
\end{equation}
The iterated commutators of the generator with the annihilation operator are
\begin{align}
	\comm{\hat{G}}{\hat{a}(\omega)}
	&=
	(-1)
	\left[
		\hat{a}(\omega-\Omega_0)
		e^{-i\varphi}
		+
		\hat{a}(\omega+\Omega_0)
		e^{+i\varphi}
	\right]
	\\
	\comm{\hat{G}}{\comm{\hat{G}}{\hat{a}(\omega)}}
	&=
	(-1)^2
	\left[
		\hat{a}(\omega-2\Omega_0)
		e^{-2i\varphi}
		+
		2\hat{a}(\omega)
		+
		\hat{a}(\omega+2\Omega_0)
		e^{+2i\varphi}
	\right]
	\\
	\comm{\hat{G}}{\comm{\hat{G}}{\comm{\hat{G}}{\hat{a}(\omega)}}}
	&=
	(-1)^3
	\bigl[
	\hat{a}(\omega-3\Omega_0)
	e^{-3i\varphi}
	+
	2^2
	\hat{a}(\omega-\Omega_0)
	e^{-i\varphi}
	\\
	&+
	\hat{a}(\omega+3\Omega_0)
	e^{+3i\varphi}
	+
	2^2
	\hat{a}(\omega+\Omega_0)
	e^{+i\varphi}
	\bigr]
\end{align}
and we can invoke the \gls{bch} formula, to transform the annihilation operator
\begin{equation}
	\begin{split}
		\hat{a}^\prime(\omega)
		&=
		\hat{U}(\theta,\Omega_0,\varphi)^\dagger
		\hat{a}(\omega)
		\hat{U}(\theta,\Omega_0,\varphi)
		\\
		&=
		\hat{a}(\omega)
		+
		\frac{i\theta}{2}
		\comm{\hat{G}}{\hat{a}(\omega)}
		+
		\frac{i^2\theta^2}{2^22!}
		\comm{\hat{G}}{\comm{\hat{G}}{\hat{a}(\omega)}}
		+
		\frac{i^3\theta^3}{2^33!}
		\comm{\hat{G}}{\comm{\hat{G}}{\dots}}
		+
		\dots
		\\
		&=
		\sum_{m\in\mathbb{Z}}
		J_m(\theta)
		e^{im(\varphi-\pi/2)}
		\hat{a}(\omega+m\Omega_0)
	\end{split}
	\label{eq:frequency_conversion_annihilation}
\end{equation}
in agreement with Ref.~\cite[eq.~40]{Horoshko2018}.

To transform the displacement operator, we need to evaluate
\begin{equation}
	\begin{split}
		\hat{U}(\theta,\Omega_0,\varphi)
		\hat{a}^\dagger(\omega)
		\hat{U}(\theta,\Omega_0,\varphi)^\dagger
		&=
		\left[
			\hat{U}(\theta,\Omega_0,\varphi)
			\hat{a}(\omega)
			\hat{U}(\theta,\Omega_0,\varphi)^\dagger	
		\right]^\dagger
		\\
		&=
		\left[
			\hat{U}(-\theta,\Omega_0,\varphi)^\dagger
			\hat{a}(\omega)
			\hat{U}(-\theta,\Omega_0,\varphi)
		\right]^\dagger
		\\
		&=
		\left[
			\sum_{m\in\mathbb{Z}}
			J_m(-\theta)
			e^{+im(\varphi-\pi/2)}
			\hat{a}(\omega+m\Omega_0)
		\right]^\dagger
		\\
		&=
		\sum_{m\in\mathbb{Z}}
		J_m(-\theta)
		e^{-im(\varphi-\pi/2)}
		\hat{a}^\dagger(\omega+m\Omega_0)
		\\
		&=
		\sum_{m\in\mathbb{Z}}
		J_m(\theta)
		e^{+im(\varphi+\pi/2)}
		\hat{a}^\dagger(\omega+m\Omega_0)
	\end{split}
\end{equation}
where we used
\begin{equation}
	J_m(-\theta)
	e^{-im(\varphi-\pi/2)}
	=
	(-1)^m
	J_m(\theta)
	e^{+im(\varphi-\pi/2)}
	=
	J_m(\theta)
	e^{+im(\varphi+\pi/2)}
\end{equation}
in the last step.
The transformed displacement operator then reads
\begin{align}
	\begin{split}
		\hat{D}^\prime\left[\alpha(t)\right]
		&=
		\hat{U}(\theta,\Omega_0,\varphi)
		\hat{D}\left[\alpha(t)\right]
		\hat{U}(\theta,\Omega_0,\varphi)^\dagger
		\\
		&=
		\exp\left\{
			\int\frac{\dd{\omega}}{2\pi}
			\left[
				\alpha(\omega)
				\hat{U}(\theta,\Omega_0,\varphi)
				\hat{a}^\dagger(\omega)
				\hat{U}(\theta,\Omega_0,\varphi)^\dagger
				-
				\text{h.c.}
			\right]
		\right\}
		\\
		&=
		\exp\left\{
			\int\frac{\dd{\omega}}{2\pi}
			\left[
				\sum_{m\in\mathbb{Z}}
				\alpha(\omega)
				J_m(\theta)
				e^{+im(\varphi+\pi/2)}
				\hat{a}^\dagger(\omega+m\Omega_0)
				-
				\text{h.c.}
			\right]
		\right\}
		\\
		&=
		\exp\left\{
			\int\frac{\dd{\omega}}{2\pi}
			\left[
				\sum_{m\in\mathbb{Z}}
				J_m(\theta)
				\alpha(\omega-m\Omega_0)
				e^{im(\varphi+\pi/2)}
				\hat{a}^\dagger(\omega)
				-
				\text{h.c.}
			\right]
		\right\}
		\\
		&=
		\exp\left\{
			\int\frac{\dd{\omega}}{2\pi}
			\left[
				\alpha^\prime(\omega)
				\hat{a}^\dagger(\omega)
				-
				\text{h.c.}
			\right]
		\right\}
	\end{split}	
\end{align}
where we identified the transformed Fourier amplitude with
\begin{equation}
	\alpha^\prime(\omega)
	=
	\sum_{m\in\mathbb{Z}}
	J_m(\theta)
	e^{im(\varphi+\pi/2)}
	\alpha(\omega-m\Omega_0)
	\label{eq:frequency_conversion_frequency_amplitude}
	.
\end{equation}
Written as a convolution, the transformed Fourier amplitude reads
\begin{equation}
	\alpha^\prime(\omega)
	=
	\int\frac{\dd{\omega^\prime}}{2\pi}
	h(\omega^\prime)
	\alpha(\omega-\omega^\prime)
	,
\end{equation}
where the convolution kernel is the Dirac train
\begin{equation}
	\sum_{m\in\mathbb{Z}}
	J_m(\theta)
	e^{im(\varphi+\pi/2)}
	(2\pi)
	\delta^{(1)}(\omega^\prime-m\Omega_0)
	\label{eq:phase_modulation_sinusoidal_kernel}
	.
\end{equation}
As the transformed Fourier amplitude is a convolution in frequency space, we expect a product in time space, i.e.,
\begin{equation}
	\begin{split}
		\alpha^\prime(t)
		=
		\int\frac{\dd{\omega}}{2\pi}
		\alpha^\prime(\omega)
		e^{-i\omega t}
		&=
		\int\frac{\dd{\omega}}{2\pi}
		\alpha(\omega-m\Omega_0)
		e^{-i\omega t}
		\sum_{m\in\mathbb{Z}}
		J_m(\theta)
		e^{im(\varphi+\pi/2)}
		\\
		&=
		\int\frac{\dd{\omega}}{2\pi}
		\alpha(\omega)
		e^{-i\omega t}
		\sum_{m\in\mathbb{Z}}
		J_m(\theta)
		e^{-im(\Omega_0t-\varphi-\pi/2)}
		\\
		&=
		\alpha(t)
		e^{-i\theta\sin(\Omega_0t-\varphi-\pi/2)}
		=
		\alpha(t)
		e^{+i\theta\cos(\Omega_0t-\varphi)}
	\end{split}
\end{equation}
where we again used the Jacobi-Anger expansion, \cref{eq:jacobi_anger_expansion}.
We conclude that a coherent state transform under sinusoidal phase-modulation as
\begin{equation}
	\hat{U}(\theta,\Omega_0,\varphi)
	\ket*{\alpha(t)}
	=
	\ket*{\alpha(t)e^{+i\theta\cos(\Omega_0t-\varphi)}}
	.
\end{equation}
We extend our result to signals of finite duration by noting that we can write such a signal as a sum of harmonics, i.e.,
\begin{equation}
	\varphi(t)
	=
	\sum_{k=0}^N
	\theta_k
	\cos(\Omega_kt-\varphi_k)
	,
\end{equation}
and apply a product of sinusoidal phase modulation operators, \cref{eq:frequency_conversion_evolution},
\begin{equation}
	\left(
		\prod^N_{k=0}
		\hat{U}(\theta_k,\Omega_k,\varphi_k)
	\right)
	\ket*{\alpha(t)}
	=
	\ket*{\alpha(t)e^{+i\varphi(t)}}
	.
\end{equation}
Although we cannot write down a closed solution for an arbitrary time-dependent signal in frequency space, we expect such a solution to exist and to be a convolution.

\subsection{Amplitude modulator}

The \gls{mzm} uses two phase modulators to perform amplitude modulation through interference.
\begin{figure}[htb]
	\centering
	\includegraphics{figures/pstricks/mzi-symmetric}
	\caption{Symmetric \gls{mzi} using free-space optics comprising two balanced \gls{bs}, BS1 and BS2, two mirrors, M1 and M2, and two phase modulators, PM1 and PM2. The input amplitudes, $\alpha_1(t)$ and $\alpha_2(t)$, enter BS1 and are split into an upper and lower path. The upper path receives a phase shift of $\phi_1(t)+\pi$ from PM1 and M1 before entering BS2 from the top. The lower path receives a phase shift of $\phi_2(t)+\pi$ from M2 and PM2 before entering BS2 from the left. BS2 recombines the phase-shifted upper and lower path into the output Fourier amplitudes $\alpha_1^\prime(t)$ and $\alpha_2^\prime(t)$.}\label{fig:mzi_symmetric}
\end{figure}
\Cref{fig:mzi_symmetric} shows a symmetric \gls{mzi}\footnote{We distinguish between \gls{mzi} and \gls{mzm}, whether it is an optical (\gls{mzi}) or integrated (\gls{mzm}) embodiment. However, there is no difference in the theoretical treatment.} using free-space optics with one signal input; the other input being in the vacuum state.
The most crucial components of the \gls{mzi} are a splitter, a coupler, and two independent phase modulators.
The splitter divides the input light into two branches, which are phase modulated with $\phi_1(t)$ and $\phi_2(t)$ by PM1 respectively PM2.
The coupler recombines both branches into two outputs.
Two cubic beam splitters implement the splitter (BS1) and the coupler (BS2) for our free-space setup.
For additional beam alignment, our free-space setup utilizes two mirrors (M1 and M2).

To find the effect of the \gls{mzm} on a coherent input state, we study the cumulative effect of the individual optical components.
One particular challenge is that the transformation of passive components is a convolution in the time domain.
In contrast, the transformation of active components is a convolution in the frequency domain.
A possible way forward is to invoke the narrow-bandwidth approximation and assume that the passive components are free of dispersion over the relevant optical bandwidth.
Under this simplifying assumption, the passive components are described by the ideal transformations
\begin{align}
	U_\text{BS1}
	&=
	\frac{1}{\sqrt{2}}
	\begin{pmatrix}
		1 & i \\
		i & 1
	\end{pmatrix}
	&
	U_\text{BS2}
	&=
	\frac{1}{\sqrt{2}}
	\begin{pmatrix}
		i & 1 \\
		1 & i
	\end{pmatrix}
	,
\end{align}
which are valid in both time and frequency space.\footnote{The particular choice corresponds to a perfect cubic beam splitter with a single dielectric layer~\cite[p.~139]{Gerry2005}, where we exchanged the rows of the second \gls{bs} for consistency with the input labels.}
The transformation of the active phase modulation is
\begin{equation}
	U_\text{PM}(t)
	=
	\begin{pmatrix}
		e^{i\phi_1(\omega)} & 0 \\
		0 & e^{i\phi_2(\omega)}
	\end{pmatrix}
	,
\end{equation}
where we ignored static phase shifts.\footnote{For practical applications, one calibrates the phase modulators with a static bias voltage to compensate for undesired static phase shifts.}
The composition of these transformations yields the transformation of the \gls{mzm}
\begin{equation}
	U_\text{MZM}(t)
	=
	U_\text{BS2}
	U_\text{PM}(t)
	U_\text{BS1}
	=
	\begin{pmatrix}
		\cos\phi_-(t) & +\sin\phi_-(t) \\
		-\sin\phi_-(t) & \cos\phi_-(t)
	\end{pmatrix}
	ie^{i\varphi_+(t)}
	\label{eq:mzm_matrix}
	,
\end{equation}
where we introduced the common-mode and differential-mode phase signals
\begin{align}
	\phi_+(t)
	&=
	\frac{\phi_2(t)+\phi_1(t)}{2}
	&
	\phi_-(t)
	&=
	\frac{\phi_2(t)-\phi_1(t)}{2}
	.
\end{align}
For the input state being a tensor product of a coherent and a vacuum state
\begin{equation}
	\ket{\vb{\alpha}(t)}
	=
	\ket{\alpha(t),0}
	,
\end{equation}
the matrix transformation of the \gls{mzm}, \cref{eq:mzm_matrix}, predicts the output amplitudes to be
\begin{equation}
	\alpha^\prime(t)
	=
	U_\text{MZM}(t)
	\alpha(t)
	=
	\alpha(t)
	\begin{pmatrix}
		\cos\phi_-(t) \\
		-\sin\phi_-(t)
	\end{pmatrix}
	ie^{i\varphi_+(t)}
	\label{eq:mzm_amplitude_time}
\end{equation}
where the common-mode phase signal $\varphi_+(t)$ changes the global phase of the output signal and the differential-mode signal $\varphi_-(t)$ changes the power splitting ratio of the outputs.
In the previous sections, we derived the unitary evolution operator corresponding to the unitary matrix transforms.
We therefore claim the existence of an unitary operator, $\hat{U}_\text{MZM}$, corresponding to the unitary matrix transform of \cref{eq:mzm_matrix}, where the action of such operator on a coherent and vacuum input state is
\begin{equation}
	\hat{U}_\text{MZM}(t)
	\ket*{\alpha(t),0}
	=
	\ket*{\alpha(t)\cos\phi_-(t)ie^{i\varphi_+(t)},\alpha(t)\sin\phi_-(t)ie^{i\varphi_+(t)}}
	.
\end{equation}
Usually, one output is monitored for bias control of the phase modulators, and the other output is used for further processing.
In this case, we can remove the other other output using a projection operator, $\hat{P}$, and we find
\begin{equation}
	\hat{P}
	\hat{U}_\text{MZM}(t)
	\ket*{\alpha(t),0}
	=
	\ket*{\alpha(t)\beta_\text{MZM}(t)}
	\label{eq:mzm_state}
	,
\end{equation}
where we defined the complex-valued amplitude modulation signal $\beta(t)$ with $\abs{\beta(t)}\leq 1$.

\Cref{eq:mzm_state} suggests that a \gls{mzm} with two independent phase modulators allows for complex amplitude modulation, i.e.,  modulation of in-phase and quadrature components of the input signal $\alpha(t)$.
\begin{figure}[htb]
    \centering
    \includegraphics{figures/tikz/iqm}
    \caption{Integrated \gls{iqm} using three \gls{mzm} arms: A coherent input amplitude, $\alpha(t)$, is split into an upper and lower branch. The upper and lower branches comprise an integrated \gls{mzm} that performs amplitude modulation with the in-phase and quadrature signal, $I(t)$ respectively $Q(t)$. The integrated \gls{mzm} consists of a hexagonal-shaped waveguide with an inside signal electrode and two outer grounds. The outputs of the in-phase and quadrature modulated form a third \gls{mzm} used to set a relative phase of $\Lambda$ between the in-phase and quadrature signals, yielding the coherent output amplitude $\alpha^\prime(t)$.}\label{fig:iqm}
\end{figure}
In practice, however, we the \gls{iqm} to be implemented using three \gls{mzm}s as depicted in \Cref{fig:iqm}.
The phases of the \gls{mzm} in an integrated \gls{iqm} are only driven differentially.
The upper branch modulates the in-phase component, $I(t)$, while the lower branch modulates the quadrature component, $Q(t)$.
A third \gls{mzm} adds a static relative phase $\Lambda$ between the in-phase and quadrature branch such that these branches are recombined at $\pi/2$.
For the coherent output state, we find
\begin{equation}
	\ket*{\alpha^\prime(t)}
	=
	\hat{U}_\text{IQM}(t)
	\ket*{\alpha(t)}
	=
	\ket*{\alpha(t)\beta_\text{IQM}(t)}
\end{equation}
wherein the complex amplitude modulation signal is now
\begin{equation}
	\beta_\text{IQM}(t)
	=
	I(t)
	+
	iQ(t)
	\label{eq:iqm_modulation}
	.
\end{equation}
While the \gls{iqm} is equivalent to a \gls{mzm} with two independent phase modulators for time-independent modulation, the \gls{mzm} cannot, in general, perform continuously phase modulation.
The reason being that the complex phase of the \gls{mzm}, \cref{eq:mzm_amplitude_time}, cannot be increased arbitrary for practical electro-optical phase modulators but must be brought back to some working point, which would require instantaneous jumps.
		\section{Detector}

\subsection{Photodiode}

\subsection{Homo- and heterodyne detectors}

\Cref{fig:balanced_detector_optics} illustrates the essential optical setup for balanced detection:
a 50:50 beam splitter (depicted by the square with the diagonal line) mixes two optical input fields $E_l,E_s$.
\begin{figure}[htb]
    \centering
    \includestandalone{figures/tikz/balanced-detector}
    \caption{Optical part of the balanced detector setup comprising a beam splitter and two photodetectors.}\label{fig:balanced_detector_optics}
\end{figure}
The intensities of the two optical output fields leaving the beam splitter, represented by $E_\pm$, are separately measured by two photodiodes (depicted by the filled half-circles).
The photodiodes output a current signal $i_\pm$ proportional to the registered light field's intensity.

$E_l$ is called the \gls{lo} and is assumed monochromatic, i.e.,
\begin{equation}
    E_l(t)
    =
    E_l(0)\cos(\omega_l t+\theta)
    \label{eq:efield_lo}.
\end{equation}
$E_s$ is a real passband signal with optical carrier frequency $\omega_c$
\begin{equation}
    E_s(t)
    =
    E_I(t)\cos(\omega_c t)-E_Q(t)\sin(\omega_c t)
    \label{eq:efield_signal}
\end{equation}
with $E_I(t),E_Q(t)$ being the quadrature components of the baseband signal of $E_s(t)$.
We have factored the relative phase $\theta$ between \gls{lo} and signal to $E_l(t)$.

From our discussions of the beam splitter in \cref{sec:beam_splitter}, we know that
\begin{equation}
    E_\pm(t)
    =
    \frac{E_s(t)\pm E_l(t)}{\sqrt{2}}
\end{equation}
up to some phase factors.

\Cref{fig:balanced_detector_electronics} depicts the electronic circuit of the balanced detector.
The balanced detector is characterized by two photodiodes PD+ and PD- connected in series and allowing current flow in one direction.
According to Kirchhoff's current law, the balanced current signal is equal to the difference of the individual photocurrents, i.e., $i_+-i_-$
In the present embodiment, we include an operational amplifier into the balanced detector configuration.
\begin{figure}[htb]
    \centering
    \includestandalone{figures/circuitikz/balanced-detector}
    \caption{Schematic of the balanced detector circuit.}\label{fig:balanced_detector_electronics}
\end{figure}
The operational amplifier is configured as transimpedance amplifier where the non-inverting input is grounded, the inverting input is feed with the input current signal, and the voltage output is coupled through a feedback impedance $Z_f$ with the inverting voltage input.
For an ideal transimpedance amplifier, the output voltage relates to the photocurrents via
\begin{equation}
    V_\text{out}
    =-Z_f(i_+-i_-).
\end{equation}
We end our analysis of the balanced detector to conclude that the photocurrent difference is our relevant quantity we need to discuss.

The photocurrents of the photodectors are proportional to the incident beam power which in turn is proportional to the squared amplitude of the electric field component
\begin{equation}
    i_\pm(t)
    \propto
    E_\pm(t)^2
    =
    \frac{1}{2}
    \bigl\{
        E_l(t)^2+E_s(t)^2\pm 2E_l(t)E_s(t)
    \bigr\}
    \label{eq:photocurrent_individual}.
\end{equation}
From \Cref{fig:balanced_detector_electronics} we know that the current signal of the balanced detector is equal to the difference of the individual photocurrents
\begin{equation}
    i(t)
    =
    i_+(t)-i_-(t)
    \propto
    2E_l(t)E_s(t)
    \label{eq:photocurrent_difference1}
\end{equation}
Inserting \cref{eq:efield_lo} and \cref{eq:efield_signal} into \cref{eq:photocurrent_difference1} yields
\begin{equation}
    i(t)
    \propto
    2E_l(0)\cos(\omega_lt+\theta)\bigl\{E_I(t)\cos(\omega_ct)-E_Q(t)\sin(\omega_ct)\bigr\}
    \label{eq:photocurrent_difference3}.
\end{equation}
Finally, by using the trigonometric identities
\begin{align}
    \cos(x)\cos(y)
    &=
    \frac{1}{2}\bigl(\cos(x+y)+\cos(x-y)\bigr)\\
    \cos(x)\sin(y)
    &=
    \frac{1}{2}\bigl(\sin(x+y)-\sin(x-y)\bigr)
\end{align}
we can rewrite \cref{eq:photocurrent_difference3} as
\begin{equation}
    i(t)
    \propto
    E_l(0)\biggl\{
        E_I(t)\bigl[\cos(\omega_+t+\theta)+\cos(\omega_-t+\theta)\bigr]
        +
        E_Q(t)\bigl[\sin(\omega_+t+\theta)-\sin(\omega_-t+\theta)\bigr]
    \biggr\}
    \label{eq:photocurrent_difference_final}
\end{equation}
where we have defined $\omega_\pm=\omega_l\pm\omega_c$.

So far we neglected the finite response-time of the photodiodes.
We assume that the photocurrent is attenuated by a low-pass with transfer function $h(\omega)$ and cut-off frequency $\omega_0$.
The bandwidth-limited balanced detector current signal $i^\prime(t)$ is obtained by convolution of the low-pass transfer function with the infinite bandwidth signal $i(t)$
\begin{equation}
    i^\prime(t)
    =
    (h*i)(t)
    =
    \int_\mathbb{R}\dd{\omega} h(\omega)i(\omega)e^{-i\omega t}
    \label{eq:photocurrent_difference_bandwidth_limited}
\end{equation}
where we invoked the convolution theorem for the last equal.
The Fourier transform of the unlimited bandwidth balanced detector signal is
\begin{align}
    i(\omega)
    \propto
    E_l(0)\int_\mathbb{R}\dd{t}e^{i\omega t}
    \biggl\{
        &E_I(t)\bigl[\cos(\omega_+t+\theta)+\cos(\omega_-t+\theta)\bigr]\\
       -&E_Q(t)\bigl[\sin(\omega_+t+\theta)-\sin(\omega_-t+\theta)\bigr]
    \biggr\}
    \label{eq:photocurrent_difference_spectrum1}.
\end{align}
We can express \cref{eq:photocurrent_difference_spectrum1} in terms of the spectral representation of the in-phase and quadrature components.
We show this exemplary for the in-phase component
\begin{align*}
    \int_\mathbb{R}\dd{t}E_I(t)e^{i\omega t}\cos(\omega_\pm t+\theta)
    &=
    \frac{e^{+i\theta}}{2}\underbrace{\int_\mathbb{R}\dd{t}E_I(t)e^{i(\omega+\omega_\pm)t}}_{E_I(\omega+\omega_\pm)}
    +
    \frac{e^{-i\theta}}{2}\underbrace{\int_\mathbb{R}\dd{t}E_I(t)e^{i(\omega-\omega_\pm)t}}_{E_I(\omega-\omega_\pm)}\\
    \int_\mathbb{R}\dd{t}E_Q(t)e^{i\omega t}\sin(\omega_\pm t+\theta)
    &=
    \frac{e^{+i\theta}}{2i}\underbrace{\int_\mathbb{R}\dd{t}E_Q(t)e^{i(\omega+\omega_\pm)t}}_{E_Q(\omega+\omega_\pm)}
    -
    \frac{e^{-i\theta}}{2i}\underbrace{\int_\mathbb{R}\dd{t}E_Q(t)e^{i(\omega-\omega_\pm)t}}_{E_Q(\omega-\omega_\pm)}
    .
\end{align*}
with this in mind \cref{eq:photocurrent_difference_spectrum1} becomes
\begin{align*}
    i(\omega)
    \propto
    E_l(0)
    \biggl\{
        &\frac{e^{+i\theta}}{2}E_I(\omega+\omega_+)+\frac{e^{-i\theta}}{2}E_I(\omega-\omega_+)\\
        +&\frac{e^{+i\theta}}{2}E_I(\omega+\omega_-)+\frac{e^{-i\theta}}{2}E_I(\omega-\omega_-)\\
        -&\frac{e^{+i\theta}}{2i}E_I(\omega+\omega_+)+\frac{e^{-i\theta}}{2i}E_I(\omega-\omega_+)\\
        +&\frac{e^{+i\theta}}{2i}E_I(\omega+\omega_-)-\frac{e^{-i\theta}}{2i}E_I(\omega-\omega_-)
    \biggr\}
\end{align*}
we factor out the phases
\begin{align*}
    i(\omega)
    \propto
    \frac{1}{2}
    E_l(0)
    \biggl\{
        &e^{+i\theta}
        \bigl[
            E_I(\omega+\omega_+)+E_I(\omega+\omega_-)+iE_Q(\omega+\omega_+)-iE_Q(\omega+\omega_-)
        \bigr]\\
        +
        &e^{-i\theta}
        \bigl[
            E_I(\omega-\omega_+)+E_I(\omega-\omega_-)-iE_Q(\omega-\omega_+)+iE_Q(\omega-\omega_-)
        \bigr]
    \biggr\},
\end{align*}
after defining the complex spectrum
\begin{equation}
    A_\pm(\omega)
    =
    E_I(\omega)\pm iE_Q(\omega)
    \label{eq:aux_spectrum},
\end{equation}
we find the compact expression
\begin{align}
    i(\omega)
    \propto
    \frac{1}{2}E_l(0)
    \biggl\{
        &e^{+i\theta}
        \bigl[
            A_+(\omega+\omega_+)+A_-(\omega+\omega_-)
        \bigr]\\
        +
        &e^{-i\theta}
        \bigl[
            A_-(\omega-\omega_+)+A_+(\omega-\omega_-)
        \bigr]
    \biggr\}
\end{align}
Inserting \cref{eq:photocurrent_difference_spectrum2} into \cref{eq:photocurrent_difference_bandwidth_limited}, we find
\begin{align}
    i^\prime(t)
    \propto
    \frac{1}{2}E_l(0)
    \int_\mathbb{R}\dd{\omega}
    h(\omega)
    \biggl\{
        &e^{+i\theta}
        \bigl[
            A_+(\omega+\omega_+)+A_-(\omega+\omega_-)
        \bigr]\\
        +
        &e^{-i\theta}
        \bigl[
            A_-(\omega-\omega_+)+A_+(\omega-\omega_-)
        \bigr]
    \biggr\}
    e^{-i\omega t}
    \label{eq:photocurrent_difference_spectrum3}.
\end{align}
Usually, the transimpedance amplifier has a smaller bandwidth than the photodiode, and we can approximate the photodiode's transfer function by
\begin{equation}
    h(\omega)
    \approx
    \begin{cases}
        1 & \text{if } -B<\omega<+B\\
        0 & \text{otherwise}
    \end{cases}
\end{equation}
and the balanced detector's bandwidth-limited current signal take the form
\begin{align}
    i^\prime(t)
    \approxprop
    \frac{1}{2}
    E_l(0)
    \int_{-B}^{+B}\dd{\omega}
    \biggl\{
        &e^{+i\theta}
        \bigl[
            A_+(\omega+\omega_+)+A_-(\omega+\omega_-)
        \bigr]\\
        +
        &e^{-i\theta}
        \bigl[
            A_-(\omega-\omega_+)+A_+(\omega-\omega_-)
        \bigr]
    \biggr\}
    e^{-i\omega t}.
\end{align}
For real photodiodes, we have $B\ll\omega_+$ which allows us to neglect the $A_+$ terms as they represent complex passband signals, i.e., their signal power is concentrated at $\pm\omega_+$,
\begin{align}
    i^\prime(t)
    &\approxprop
    \frac{1}{2}
    E_l(0)
    \int_{-B}^{+B}\dd{\omega}
    \biggl\{
        A_-(\omega+\omega_\text{IF})e^{+i\theta}
        +
        A_+(\omega-\omega_\text{IF})e^{-i\theta}
    \biggr\}
    e^{-i\omega t}
\end{align}
where we now refer to $\omega_-$ as the \gls{if} $\omega_\text{IF}$.
Using the variable substitution $\omega^\prime=\omega\pm\omega_\text{IF}$, we can write the previous equation as
\begin{align}
    i^\prime(t)
    \approxprop
    \frac{1}{2}
    E_l(0)
    \biggl\{
        &e^{+i(\omega_\text{IF}t+\theta)}
        \int_{-B+\omega_\text{IF}}^{+B-\omega_\text{IF}}\dd{\omega^\prime}
            A_-(\omega^\prime)e^{-i\omega^\prime t}\\
        +
        &e^{-i(\omega_\text{IF}t+\theta)}
        \int_{-B-\omega_\text{IF}}^{+B+\omega_\text{IF}}\dd{\omega^\prime}
            A_+(\omega^\prime)e^{-i\omega^\prime t}
    \biggr\}.
\end{align}
$A_\pm(\omega)$ inherits a symmetry from the conjugate symmetry of the quadrature baseband signals
\begin{equation}
    A_\pm(-\omega)
    =
    E_I(-\omega)\pm iE_Q(-\omega)
    =
    \bigl(
    E_I(\omega)\mp iE_Q(\omega)
    \bigr)^*
    =
    A_\mp^*(\omega)
    .
\end{equation}
With this symmetry in mind, we can rewrite the spectrum
\begin{align}
    \int_{-(B\mp\omega_\text{IF})}^{+(B\mp\omega_\text{IF})}\dd{\omega^\prime}
    A_\mp(\omega^\prime)e^{-i\omega^\prime t}
    =
    \int_0^{B\mp\omega_\text{IF}}\dd{\omega^\prime}
    \bigg\{
        A_\mp(\omega^\prime)e^{-i\omega^\prime t}
        +
        A_\pm^*(\omega^\prime)e^{+i\omega^\prime t}
    \bigg\},
\end{align}
and our balanced detector signal becomes
\begin{align}
    i^\prime(t)
    \approxprop
    \frac{1}{2}E_l(0)
    \biggl\{
        &e^{+i(\omega_\text{IF}t+\theta)}
        \int_{0}^{B-\omega_\text{IF}}\dd{\omega^\prime}
            \bigl[
                A_-(\omega^\prime)e^{-i\omega^\prime t}
                +
                A_+^*(\omega^\prime)e^{+i\omega^\prime t}
            \bigr]\\
            +
        &e^{-i(\omega_\text{IF}t+\theta)}
        \int_{0}^{B+\omega_\text{IF}}\dd{\omega^\prime}
            \bigl[
                A_+(\omega^\prime)e^{-i\omega^\prime t}
                +
                A_-^*(\omega^\prime)e^{+i\omega^\prime t}
            \bigr]
    \biggr\}.
\end{align}
If we ignore the \gls{if} in the integration domain
\begin{equation}
    \int_{-B}^{+B}\dd{\omega}A_\mp(\omega)e^{-i\omega t}
    =
    \underbrace{\int_{-B}^{+B}\dd{\omega}E_I(\omega)e^{-i\omega t}}_{\overline{E}_I(t)}
    \mp
    i\underbrace{\int_{-B}^{+B}\dd{\omega}E_Q(\omega)e^{-i\omega t}}_{\overline{E}_Q(t)}
\end{equation}
we recover the (bandwidth-limited) quadrature signals.
In summary, the bandwidth-limited balanced detection current signal turns out to be
\begin{align}
    i^\prime(t)
    &\approxprop
    \frac{1}{2}E_l(0)
    \biggl\{
        \bigl[
            \overline{E}_I(t)-i\overline{E}_Q(t)
        \bigr]
        e^{+i(\omega_\text{IF}t+\theta)}
        +
        \bigl[
            \overline{E}_I(t)+i\overline{E}_Q(t)
        \bigr]
        e^{-i(\omega_\text{IF}t+\theta)}
    \biggr\}\\
    &=
    E_l(0)
    \bigl\{
        \overline{E}_I(t)\cos(\omega_\text{IF}t+\theta)
        +
        \overline{E}_Q(t)\sin(\omega_\text{IF}t+\theta)
    \bigr\}
    \label{eq:balanced_detection_current_final}.
\end{align}

In the case of homodyne detection, $\omega_\text{IF}=0$, \cref{eq:balanced_detection_current_final} reduces to
\begin{equation}
    i^\prime(t)
    \propto
    E_l(0)
    \bigl\{
        \overline{E}_I(t)\cos(\theta)
        +
        \overline{E}_Q(t)\sin(\theta)
    \bigr\}.
\end{equation}
If the \gls{lo}'s phase $\theta$ runs in sync with the carrier phase $\theta=0$, the current signal is proportional to the bandwidth-limited in-phase baseband $\overline{E}_I(t)$.
If the \gls{lo}'s phase $\theta$ is shifted by $\theta=\pi$, the current signal is proportional to the bandwidth-limited quadrature baseband $\overline{E}_Q(t)$.

		
		\addcontentsline{toc}{section}{References}
		\printbibliography[title=References]
	\end{refsection}

	\chapter{Conclusion and outlook}
	\begin{refsection}
		\chapter*{Conclusion and outlook}
\addcontentsline{toc}{chapter}{Conclusion and outlook}

% answer to the research question
This thesis aimed to outline the incorporation of quantum aspects into optical communication.
By crafting a continuous-mode theory of light and applying it to describe electro-optical components essential for optical communication, we proposed a coherent-state transmission system as an extension of a classical optical transmission system to the quantum regime.
The suggested coherent-state transmission system offers a platform to explore and formulate novel communication protocols inspired by classical communication and quantum mechanics, especially practical \gls{qkd} protocols based on weak coherent states.

% reflection on the methodology
Next to the practical results of our work, we illuminate quantum aspects of light that have physical relevance but otherwise fall short, as our result on the generalized quadrature measurement shows.
From an abstract point of view, we can draw two key lessons from our work.
On the one hand, we need to be more careful in employing reductionism.
Some concepts are notoriously difficult to grasp, and there exists no shortcut to understanding them truly.
On the other hand, we need to be more encouraged to search for answers outside our traditional domain.
In deriving a continuous-mode quantum theory of light from quantum field theory, we consciously opted against extending single-mode quantum optics, which turned out to be a key factor in our research's success as there is more clarity in simplifying a complete theory as opposed to extending a simplified theory.
Of course, this does not mean disavowing established methods.
Rather it proves useful to switch perspectives from time to time.
For example, towards the end of our research, it turned out to be favorable to move away from a constructive bottom-up approach and work our way backward from the classical results.

% contribution to the literature
Our work is difficult to place into the existing literature, largely since our problem statement emerges from an applied industry-related setting.
The closest to our research is Shapiro's work regarding a quantum theory of optical communication~\cite{Shapiro2009}.
However, besides the photodetection, Shapiro approaches the topic from a quantum-information perspective and does neither consider a practical implementations nor a transmission setup.
Concerning the quantum theory of light, we are the first to our knowledge to present a continuous-mode theory rooted in quantum field theory and emphasizing the communication aspects.
The few existing references~\cite{Barnett2002,Loudon2000} do not attempt to justify their results but apply their formalism towards quantum optics.
For the quantum theory of electro-optical components, we have summarized and unified the existing literature.
Most importantly to mention here are the beam splitter~\cite{Haroche2006,Leonhardt2003,Vogel2006}, the photodetector~\cite{Vogel2006,Mandel1995,Shapiro2009}, and the phase modulator~\cite{Horoshko2018,QuesadaMejia2015}.

% limitation of our work
Some topics, e.g., quantum information and non-classical quantum states, require additional attention to complete a theoretical framework for quantum-optical communication not limited to practical \gls{qkd} using weak coherent states.
In addition, our description of the coherent-state transmission system misses a discussion of the quantum statistics and effects of a non-ideal quantum channel.

% outlook and recommendations
Even though our work leaves many open questions, it provides as many opportunities it provides, the most obvious being the transfer of classical communication protocols to \gls{qkd} like, for example, \gls{ofdm}~\cite{Bahrani2015}, which requires a continuous-mode theory of light.
Another potential research direction concerns security proofs for practical \gls{qkd} protocols based on weak coherent states.
In particular, it would be interesting to follow up on the concept of a logical quantum channel investigating the equivalence of a tensor-product coherent-state transmission system with our continuous-time coherent-state transmission system.
Such an equivalence, if confirmed, could be a useful tool to simplify existing and future security proofs.
Last but not least, it would be interesting to extend our theoretical framework to the transmission of non-classical quantum states and discuss if and how one can make sense of a communication system conveying not classical but quantum information.
One promising direction, which would benefit from our framework, is the transmission of frequency-entangled squeezed states, also known as broadband squeezed states~\cite{Vogel2006,Mandel1995}.

		\addcontentsline{toc}{section}{References}
		\printbibliography[title=References]
	\end{refsection}

	\appendix

\end{document}

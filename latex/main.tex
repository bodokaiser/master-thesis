\documentclass[a4paper,appendixprefix]{scrreprt}

\usepackage[utf8]{inputenc}
\usepackage{amsthm,amsmath,amssymb}
\usepackage{authblk}
\usepackage[english]{babel}
\usepackage[backend=biber]{biblatex}
\usepackage{booktabs}
%\usepackage{csquotes}
\usepackage[acronym,nonumberlist,toc]{glossaries}
\usepackage{hyperref}
\usepackage{cleveref}
\usepackage{physics}
\usepackage[separate-uncertainty=true]{siunitx}
%\usepackage[subpreambles=true]{standalone}
\usepackage{thmtools,thm-restate}
%\usepackage{subfig}
\usepackage{xcolor}

\addbibresource{references/articles.bib}
\addbibresource{references/books.bib}
\addbibresource{references/thesis.bib}

% add bibliography as section (not chapter)
% https://tex.stackexchange.com/questions/568580/make-the-bibliography-as-a-section-in-each-included-chapter
\defbibheading{bibliography}[\bibname]{\section*{#1}}

% approximately proportional to symbol
% https://tex.stackexchange.com/questions/33538/how-to-get-an-approximately-proportional-to-symbol
\def\app#1#2{%
    \mathrel{%
        \setbox0=\hbox{$#1\sim$}%
        \setbox2=\hbox{%
            \rlap{\hbox{$#1\propto$}}%
            \lower1.1\ht0\box0%
        }%
        \raise0.25\ht2\box2%
    }%
}
\def\approxprop{\mathpalette\app\relax}

% overwrite real and imaginary part operators
\let\Re\undefined
\let\Im\undefined
\DeclareMathOperator{\Re}{\operatorname{Re}}
\DeclareMathOperator{\Im}{\operatorname{Im}}
% other functions
\DeclareMathOperator{\sinc}{\operatorname{sinc}}

\newcommand{\floor}[1]{\left\lfloor#1\right\rfloor}
\newcommand{\ceil}[1]{\left\lceil#1\right\rceil}

% transpose
% https://tex.stackexchange.com/questions/403104/small-caps-mathsf-font-for-writing-transpose-of-a-matrix
\newcommand{\trans}{{\scriptscriptstyle\mathsf{T}}}

% theorems
\newtheorem{theorem}{Theorem}[section]
\newtheorem{lemma}[theorem]{Lemma}
\newtheorem{corollary}[theorem]{Corollary}
\theoremstyle{definition}
\newtheorem{definition}{Definition}[section]
\newtheorem{conjecture}{Conjecture}[section]
\newtheorem{example}{Example}[section]
\theoremstyle{remark}
\newtheorem*{remark}{Remark}

% prefix equation numbers with section number
\numberwithin{equation}{section}

% optics
\newacronym{ar}{AR}{anti-reflective}
\newacronym{mzm}{MZM}{Mach-Zehnder modulator}
\newacronym{bs}{BS}{beam splitter}
\newacronym{fc}{FC}{fiber coupler}
\newacronym{qe}{QE}{quantum efficiency}

% physics
\newacronym{dv}{DV}{discrete-variable}
\newacronym{cv}{CV}{continuous-variable}
\newacronym{dof}{DOF}{degrees of freedom}
\newacronym{eom}{EOM}{equation(s) of motion}
\newacronym{pbc}{PBC}{periodic boundary conditions}
\newacronym{bch}{BCH}{Baker-Campbell-Hausdorff}
\newacronym{ccr}{CCR}{canonical commutation relation}

% signal processing
\newacronym{dsp}{DSP}{digital signal processing}
\newacronym{lo}{LO}{local oscillator}
\newacronym{if}{IF}{intermediate frequency}
\newacronym{lp}{LP}{low-pass}
\newacronym{adc}{ADC}{analog-to-digital converter}
\newacronym{dac}{DAC}{digital-to-analog converter}
\newacronym{qam}{QAM}{quadrature amplitude modulation}
\newacronym{qpsk}{QPSK}{quadrature phase-shift keying}

% quantum-key distribution
\newacronym{qkd}{QKD}{quantum-key distribution}
\newacronym{dvqkd}{DV-QKD}{discrete-variable quantum-key distribution}
\newacronym{cvqkd}{CV-QKD}{continuous-variable quantum-key distribution}

\title{A theoretical framework for quantum optical communication - towards CV-QKD}
\author{Bodo Kaiser}
\affil{\textit{bodo.kaiser@huawei.com}}

\begin{document}

	\maketitle
	\tableofcontents

	\chapter{Introduction}
	\begin{refsection}
		\section{Quantum science and technology}
		\section{Motivation and problem statement}
        \section{Thesis overview and structure}
        \section{Notation and conventions}
	
		\addcontentsline{toc}{section}{References}
		\printbibliography[title=References]
	\end{refsection}
	
	\chapter{Continuous-variable quantum-key distribution}
	\begin{refsection}

		\addcontentsline{toc}{section}{References}
		\printbibliography[title=References]
	\end{refsection}

	\chapter{Quantum field theory of light}
	\begin{refsection}
		\section{Maxwell field}

\subsection{Relativistic field theory}

The Maxwell Lagrangian with an external source $J^\mu(t,\vb{x})$ is
\begin{equation}
	\begin{split}
		\mathcal{L}
		&=
		-
		\frac{1}{4}
		F_{\mu\nu}
		F^{\mu\nu}
		+
		A_\mu J^\mu
		\\
		&=
		\frac{1}{2}
		\left(\partial_\mu A_\nu\right)
		\left(\partial^\mu A^\nu-\partial^\nu A^\mu\right)
		+
		A_\mu J^\mu
	\end{split}
	\label{eq:mw_lagrangian}
\end{equation}
with the relativistic Euler-Lagrange equation yielding
\begin{equation}
	0
	=
	\partial_\mu\pdv{\mathcal{L}}{(\partial_\mu A_\nu)}
	-
	\pdv{\mathcal{L}}{A_\nu}
	=
	\partial_\mu\partial^\mu A^\nu
	-
	\partial^\nu\partial_\mu A^\mu
	-
	J^\nu	
	\label{eq:mw_eom}	
\end{equation}

\subsection{Coulomb gauge}

The Maxwell Lagrangian is invariant under local gauge transformations
\begin{equation}
	A_\mu(t,\vb{x})
	\to
	A_\mu^\prime(t,\vb{x})
	=
	A_\mu(t,\vb{x})
	+
	\partial_\mu\Lambda(t,\vb{x})
	\label{eq:mw_local_gauge_transform}
\end{equation}
where $\Lambda(t,\vb{x})$ is a local gauge field.
For instance, the physical field-strength tensor transforms under \cref{eq:mw_local_gauge_transform} as
\begin{equation}
	\begin{split}
		F_{\mu\nu}
		\to
		F_{\mu\nu}^\prime
		&=
		\partial_\mu\left(A_\nu+\partial_\nu\Lambda\right)
		-
		\partial_\nu\left(A_\mu+\partial_\mu\Lambda\right)
		\\
		&=
		F_{\mu\nu}
		+
		\partial_\mu\partial_\nu\Lambda
		-
		\partial_\nu\partial_\mu\Lambda
		=
		F_{\mu\nu}
	\end{split}
	\label{eq:mw_field_strength_gauge_transform}.
\end{equation}
The invariance of the Maxwell field under local gauge transformation is used to remove a degree of freedom from the field using a gauge condition.
The most popular gauge conditions are the Lorentz gauge $\partial_\mu A^\mu=0$ and the Coulomb gauge $\div\vb{A}=0$.
While the Lorentz gauge is manifest Lorentz invariant it suffers from unphysical scalar and longitudinal polarization states destroying unitarity.
The Coulomb gauge is manifest unitarity but has to be imposed in every reference frame.
As Lorentz boosts are not of interest for us, we will adapt the Coulomb gauge condition.
The Coulomb gauge leaves a residual gauge freedom which allows us to choose $A_0=0$.\footnote{If external static sources are present, we need to be more careful with the residual gauge fixing.}

Applying the Coulomb gauge
\begin{align}
	\div\vb{A}(t,\vb{x})
	=
	0
	&&
	A_0
	=
	0
	\label{eq:mw_coulomb_gauge}
\end{align}
to the free equation of motion \cref{eq:mw_eom} yields the relativistic wave equation
\begin{equation}
	0
	=
	\partial_\mu\partial^\mu
	\vb{A}(t,\vb{x})
	=
	\partial_t^2
	\vb{A}(t,\vb{x})
	-
	\grad^2
	\vb{A}(t,\vb{x})
	\label{eq:mw_relatistic_wave}
\end{equation}
which is solved by plane-waves satisfying the massless dispersion relation
\begin{equation}
	\omega(\vb{p})
	=
	\norm{\vb{p}}
\end{equation}

\subsection{Mode decomposition}

As with the Klein-Gordon field, we start with the four-dimensional Fourier transform of $A^\mu(t,\vb{x})$, insert it into the free equations of motion ($J^\mu=0$), and perform the mode decomposition
\begin{equation}
	\vb{A}(t,\vb{x})
	=
	\sum_{\lambda=1,2}
	\int_{\mathbb{R}^3}\frac{}{(2\pi)^3\sqrt{2\omega(\vb{p})}}
	\left\{
		a_\lambda(\vb{p})
		\vb{\epsilon}_\lambda(\vb{p})
		e^{-ip_\mu x^\mu}
		+
		\text{c.c.}
	\right\}
	\label{eq:mw_ft}
\end{equation}
where we defined $a_\lambda(\vb{p})\vb{\epsilon}_\lambda(\vb{p})=\vb{A}(\omega(\vb{p}),\vb{p})$ and $\vb{\epsilon}_\lambda(\vb{p})$ denotes the polarization vectors.
For the mode decomposition to satisfy the Coulomb gauge, the polarization vectors $\vb{\epsilon}_\lambda(\vb{p})$ need to be orthogonal to the wave vector $\vb{p}$, i.e.,
\begin{equation}
	\vb{p}\vdot\vb{\epsilon}_\lambda(\vb{p})
	=
	0	
\end{equation}
Furthermore, we require the $\vb{p}/\norm{\vb{p}},\vb{\epsilon}_1(\vb{p}),\vb{\epsilon}_2(\vb{p})$ to form a orthonormal basis
\begin{equation}
	\vb{\epsilon}_i(\vb{p})
	\vdot
	\vb{\epsilon}_j(\vb{p})
	=
	\delta_{ij}
\end{equation}

\subsection{Maxwell equations}

The covariant inhomogeneous Maxwell equations are obtained from the equations of motion
\begin{equation}
	J^\nu
	=
	\partial_\mu F^{\mu\nu}
	=
	\partial_\mu\partial^\mu A^\nu
	-
	\partial^\nu\partial_\mu A^\mu
	\label{eq:mw_inhomo},
\end{equation}
and the covariant homogeneous Maxwell equations are a consequence of the Bianchi identity
\begin{equation}
	0
	=
	\partial_\mu\tilde{F}^{\mu\nu}
	\label{eq:mw_homo}
\end{equation}
where we defined $\tilde{F}^{\mu\nu}=\frac{1}{2}\varepsilon^{\mu\nu\alpha\beta}F_{\alpha\beta}$.

The components of the field-strength tensor $F^{\mu\nu}$ relate to the electromagnetic field components via
\begin{align}
	F^{0i}
	=
	-E^i
	&&
	F^{ij}
	=
	-\varepsilon^{ijk}B_k
	\label{eq:mw_em_components}.
\end{align}
Evaluating the time component of \cref{eq:mw_homo} yields the Gauss' law for magnetism
\begin{equation}
	\begin{split}
		0
		=
		\varepsilon_{0\lambda\mu\nu}\partial^\lambda F^{\mu\nu}
		&=
		\varepsilon_{0ijk}\partial^iF^{jk}
		\\
		&=
		-
		\varepsilon_{ijk}\varepsilon_{ljk}
		\partial^i B_l
		=
		2\partial_iB^i
	\end{split}
	\label{eq:mw_gauss_law_mag}
\end{equation}
and the spatial component yields Ampere's circuit law
\begin{equation}
	\begin{split}
		0
		=
		\varepsilon_{i\lambda\mu\nu}
		\partial^\lambda
		F^{\mu\nu}
		&=
		-
		\varepsilon_{ijk}
		\varepsilon^{ljk}
		\partial_t B_l
		-
		2\varepsilon_{ijk}
		\partial^jE^k
		\\
		&=
		\partial_tB_i
		+
		\varepsilon_{ijk}
		\partial^jE_k
	\end{split}
	\label{eq:mw_ampere_law}.
\end{equation}
The time component of the inhomogeneous covariant Maxwell equation \cref{eq:mw_inhomo} yields Gauss' law
\begin{equation}
	J^0
	=
	\rho
	=
	\partial_\mu F^{\mu\nu}
	=
	\partial_i E^i
	\label{eq:mw_gauss_law},
\end{equation}
and the spatial component yields Faraday's law of induction
\begin{equation}
	J^i
	=
	\partial_\mu F^{\mu i}
	=
	-\partial_t E^i
	+\varepsilon^{ijk}\partial_j B_k
	\label{eq:mw_faraday_law}.
\end{equation}
and we derived the vector Maxwell equations from first principles
\begin{align}
	\div\vb{E}
	=
	\rho
	&&
	\div\vb{B}
	=
	0
	\label{eq:mw_homo_vec}
	\\
	\curl\vb{E}
	=
	-
	\partial_t\vb{B}
	&&
	\curl\vb{B}
	=
	\vb{J}
	+
	\partial_t\vb{E}
	\label{eq:mw_inhomo_vec}
\end{align}

\subsection{Canonical quantization}

\subsection{Single-particle and coherent states}

\subsection{Electromagnetic field operators}

		\section{Quantum states}

In the previous section, we derived the relevant field operators for the Maxwell field encoding electromagnetism.
In the present section, we construct quantum states from these operators in a rather axiomatic approach as done in Ref.~\cite[p.~506]{Cohen2019} for the quantum harmonic oscillator, or in axiomatic field theory~\cite{Streater2016,Haag2012,Bogolubov1989}.

To keep the arguments and notation concise, we restrict the quantum state construction to one polarization mode of the Maxwell field.
The Maxwell field then effectively becomes a Klein-Gordon field with field operator
\begin{equation}
	\hat{A}(t,\vb{x})
	=
	\int\frac{\dd[3]{p}}{(2\pi)^3\sqrt{2\omega(\vb{p})}}
	\left\{
		\hat{a}(\vb{p})
		e^{-i\omega(\vb{p})t+i\vb{p}\vdot\vb{x}}
		+
		\hat{a}^\dagger(\vb{p})
		e^{+i\omega(\vb{p})t-i\vb{p}\vdot\vb{x}}
	\right\}
	\label{eq:klein_gordon_operator}
\end{equation}
wherein the annihilation and creation operator, $\hat{a}(\vb{p}),\hat{a}^\dagger(\vb{p})$, satisfy the \gls{ccr}
\begin{align}
	\comm{\hat{a}(\vb{p})}{\hat{a}^\dagger(\vb{q})}
	&=
	(2\pi)^3\delta^{(3)}
	\left(\vb{p}-\vb{q}\right)
	\\
	\comm{\hat{a}(\vb{p})}{\hat{a}(\vb{q})}
	&=
	\comm{\hat{a}^\dagger(\vb{p})}{\hat{a}^\dagger(\vb{q})}
	=
	0
	\label{eq:klein_gordon_canonical_commutation_relation}.
\end{align}
To extend the results back to two polarization modes, we can construct a two-dimensional tensor product space from the single polarization mode.

\subsection{Vacuum state}

The fundamental assumption our state construction relies upon is the existence of a unique, up to a constant phase factor, vacuum state $\ket{0}$ invariant under the unitary Poincaré transformation~\cite[p.~97]{Streater2016}
\begin{equation}
	\hat{U}(a,\Lambda)
	\ket{0}
	=
	\ket{0}
	\label{eq:vacuum_invariance}
\end{equation}
where $\Lambda$ denotes a Lorentz transformation and $a$ a spacetime translation.
The vacuum state is an element of a one-dimensional complex Hilbert space, $\mathcal{H}^{(0)}=\mathcal{H}(\mathbb{C})$, the zero-particle state space.
The generator of the unitary spacetime translation is the four-momentum operator $\hat{P}^\mu=\left(\hat{H},\vu{P}\right)$~\cite[p.~28]{Haag2012}
\begin{equation}
	\hat{U}(a)
	=
	\hat{U}(a,\mathbb{1})
	=
	e^{i\hat{P}_\mu a^\mu}
	\label{eq:spacetime_translation}
	.
\end{equation}
The invariance of the vacuum under spacetime translations, \cref{eq:spacetime_translation}, implies that the vacuum is a zero eigenstate to the Hamilton and momentum operator
\begin{align}
	\hat{H}
	\ket{0}
	&=
	0
	&
	\vu{P}
	\ket{0}
	&=
	\vb{0}
	\label{eq:vacuum_energy_momentum}
	.	
\end{align}
From the mode expansion of the Hamilton operator, \cref{eq:maxwell_hamilton_operator}, and the vacuum state being a zero eigenstate to the Hamilton operator, we conclude that the vacuum state is also a zero eigenstate to the annihilation operator
\begin{equation}
	\hat{a}(\vb{p})
	\ket{0}
	=
	0
	\label{eq:vacuum_annihilation}
	.
\end{equation}
In the next paragraph, we motivate why we understand the annihilation and creation operators as adding or removing a particle excitation to and from the field.
Under these circumstances, we can read \cref{eq:vacuum_annihilation} as destroying the vacuum state, ensuring no negative energy or negative particle number states.

\subsection{Particle states}

The motivation of why the annihilation and creation operators add or remove a particle with energy and momentum to and from the field follows from applying the commutators of the number and momentum operator with the creation operator
\begin{align}
	\comm{\hat{N}}{\hat{a}^\dagger(\vb{p})}
	&=
	\hat{a}^\dagger(\vb{p})
	&
	\comm{\vu{P}}{\hat{a}^\dagger(\vb{p})}
	&=
	\vb{p}
	\hat{a}^\dagger(\vb{p})
	.
\end{align}
Applying the vacuum state $\ket{0}$ to the right of the commutator equations, yields eigenvalue equations for the number and momentum operator
\begin{align}
	\hat{N}
	\hat{a}^\dagger(\vb{p})
	\ket{0}
	&=
	1
	\hat{a}^\dagger(\vb{p})
	\ket{0}
	&
	\vu{P}
	\hat{a}^\dagger(\vb{p})
	\ket{0}
	&=
	\vb{p}
	\hat{a}^\dagger(\vb{p})
	\ket{0}
	.
\end{align}
The eigenvalue equations suggest
\begin{equation}
	\ket{\vb{p}}
	=
	\hat{a}^\dagger(\vb{p})
	\ket{0}
	\label{eq:momentum_state}
\end{equation}
to be a single-particle state with momentum $\vb{p}$ and energy $\omega(\vb{p})$, a momentum state.~\cite[p.~23]{Peskin1995}.
Unfortunately, the inner product between two momentum states does not yield a complex number $\mathbb{C}$ but, a distribution,
\begin{equation}
	\braket{\vb{p}}{\vb{q}}
	=
	\bra{0}
	\comm{\hat{a}(\vb{p})}{\hat{a}^\dagger(\vb{q})}
	\ket{0}
	=
	(2\pi)^3
	\delta^{(3)}(\vb{p}-\vb{q})
\end{equation}
suggesting that something essential is missing in our description.

It makes sense to take a step back and recap some mathematical context regarding distributions.\footnote{See Ref.~\cite[p.~590]{Zeidler2016} and Ref.~\cite[p.~193]{Mukhanov2007} for a mathematical discussion of distributions in a physical context.}
One approach considers distributions as functionals, i.e., maps from a function space, e.g., the space of real-valued square-integrable functions $L^2(\mathbb{R})$, to real numbers $\mathbb{R}$.
Implicitly, we already used functionals when we considered the action integral
\begin{equation}
	\hat{S}\left[x(t)\right]
	=
	\int_{t_0}^{t_1}\dd{t}
	L\left(x(t),\dot{x}(t)\right)
	,
\end{equation}
wherein $L$ is some classical Lagrangian, evaluated for some finite time interval $[t_0,t_1]$ maps the trajectory $x(t)$ of a point particle to a real number $\mathbb{R}$.
Often a functional $A$ acting on a function $f$ is written
\begin{equation}
	A[f]
	=
	\int\dd{x}
	f(x)A(x)
	\label{eq:functional}
\end{equation}
wherein $A(x)$ is denoted the integration kernel representing the functional, which may be an ordinary function or a distribution.
For example, the delta distribution $\delta(x-y)$ is the integration kernel of the functional $\delta_y$
\begin{equation}
	\delta_{y}[f]
	=
	\int\dd{x}
	f(x)
	\delta(x-y)
	=
	f(y)
	.
\end{equation}
Fourier transforms are another class of functionals we already used frequently.
Linear functionals share many convenient properties with ordinary functions and physicists often skip the distinction.
In axiomatic quantum field theory, the quantum field operators are precisely defined as operator-valued tempered distributions mapping from the space of smearing or test functions $\mathcal{S}(\mathbb{R}^4)$ to the the set of operators defined on the corresponding Hilbert space $\mathcal{O}(\mathcal{H})$~\cite[p.~56]{Haag2012}.
A typical functional space for smearing functions is the Schwartz space, a subset of the space of square-integrable functions $L^2$, which rapidly fall off at infinity, a property which we exploit for partial integration with vanishing boundary terms of the action integral.\footnote{Typical Schwartz functions are Gaussian functions multiplied with a monomial, e.g., $x^ne^{-a\norm{x}^2}$ where $n\in\mathbb{N}_0$ and $a\in\mathbb{R}_+$.}

Let us reinterpret the (positive frequency) field operator with this mathematical background by considering its action on a smearing function, i.e.,
\begin{equation}
	\hat{A}^{(+)}[f]
	=
	\int\dd[4]{x}
	f(t,\vb{x})
	\hat{A}^{(+)}(t,\vb{x})
	=
	\int\frac{\dd[3]{p}}{(2\pi)^3\sqrt{2\omega(\vb{p})}}
	f\left(\omega(\vb{p}),\vb{p}\right)
	\hat{a}^\dagger(\vb{p})
\end{equation}
where we inserted the plane-wave expansion for the field and the spacetime Fourier transform for the smearing function and used the orthogonality of the Fourier modes in the second equation.
Applying the smeared positive frequency field operator to the vacuum state,
\begin{equation}
	\begin{split}
		\ket{1_f}
		=
		\hat{A}^{(+)}[f]
		\ket{0}
		,
	\end{split}
	\label{eq:single_particle_state}
\end{equation}
and comparing the result to the momentum state, \cref{eq:momentum_state}, we find the function $f$ to smear the function in momentum and spacetime space~\cite[p.~35]{Srednicki2007}.\footnote{Interestingly though, only momentum components satisfying the energy-momentum relation, \cref{eq:energy_momentum_relation}, contribute to momentum space.}
The inner product of two such smeared states yields a complex number $\mathbb{C}$
\begin{equation}
	\braket{1_f}{1_g}
	=
	\int\frac{\dd[3]{p}}{(2\pi)^32\omega(\vb{q})}
	f(\vb{p})^*
	g(\vb{p})
\end{equation}
implying the smeared state $\ket{1_f}$ being normalizable if we require the smearing function to satisfy
\begin{equation}
	\braket{1_f}{1_f}
	=
	\int\frac{\dd[3]{p}}{(2\pi)^32\omega(\vb{q})}
	\abs*{f(\vb{p})}^2
	=
	1
	\label{eq:spectrum_normalization}
	.
\end{equation}
With the normalization condition imposed, the smeared state $\ket{1_f}$ is an eigenstate of the number operator to eigenvalue one
\begin{equation}
	\hat{N}
	\ket{1_f}
	=
	1
	\ket{1_f}
\end{equation}
suggesting the smeared state $\ket{1_f}$ to be the physical single-particle state we were looking for.
The smeared state $\ket{1_f}$ is not an eigenstate of the energy and momentum operator anymore but has expectation values
\begin{align}
	\bra{1_f}
	\hat{H}
	\ket{1_f}
	&=
	\int\frac{\dd[3]{p}}{(2\pi)^3}
	\omega(\vb{p})
	\abs*{f\left(\omega(\vb{p}),\vb{p}\right)}^2
	\\
	\bra{1_f}
	\vu{P}
	\ket{1_f}
	&=
	\int\frac{\dd[3]{p}}{(2\pi)^3}
	\vb{p}
	\abs*{f\left(\omega(\vb{p}),\vb{p}\right)}^2
\end{align}
suggesting that the smearing function has the physical interpretation of a frequency spectrum.

Let $\ket{1_f}$ be a smeared particle state, then the wave function $\Psi(t,\vb{x})$ is the probability amplitude of finding the particle at a spacetime coordinate $(t,\vb{x})$~\cite[p.~24]{Peskin1995}, i.e.,
\begin{equation}
	\Psi(t,\vb{x})
	=
	\bra{0}
	\hat{A}(t,\vb{x})
	\ket{1_f}
	=
	\int\dd{t^\prime}\dd[3]{x^\prime}
	D(t-t^\prime,\vb{x}-\vb{x}^\prime)
	f(t^\prime,\vb{x}^\prime)
\end{equation}
wherein $f(t,\vb{x})$ is the spacetime representation of the (initial) smearing function and
\begin{equation}
	D(t,\vb{x})
	=
	\int\frac{\dd[3]{p}}{(2\pi)^32\omega(\vb{p})}
	e^{-i\omega(\vb{p})t+i\vb{p}\vdot\vb{x}}
\end{equation}
is the propagator as defined in Ref.~\cite[p.~27]{Peskin1995}.
Given the wave function, the relativistic probability current
\begin{equation}
	j_\mu(t,\vb{x})
	=
	2\Im\left\{
		\Psi(t,\vb{x})^*
		\partial_\mu
		\Psi(t,\vb{x})
	\right\}
\end{equation}
allows us to estimate the center-of-mass position and velocity of the particle, i.e.,
\begin{align}
	\expval{\vb{x}(t)}
	&=
	\int\dd[3]{x}
	\vb{x}
	\rho(t,\vb{x})
	&
	\expval{\vb{v}(t)}
	&=
	\int\dd[3]{x}
	\vb{j}(t,\vb{x})
\end{align}
wherein $\rho(t,\vb{x})=j_0(t,\vb{x})$ is the relativistic probability density.
For more details on the properties of relativistic wave packets, e.g., dispersion, see Ref.~\cite{Naumov2013} and Ref.~\cite{Naumov2009}.

To summarize our findings, we first motivated momentum eigenstates from the commutation algebra.
However, the momentum eigenstates are prone to mathematical inconsistencies, following that the momentum states are strictly speaking distributions, not functions.
Physically, the momentum states correspond to unphysical plane-waves.
A mathematical consistent single-particle state requires a momentum spectrum.
The momentum spectrum encodes many important physical properties like the localization and velocity of the particle.

\subsection{Fock space}

The single-particle state defined in \cref{eq:single_particle_state} is an element of the one-particle Hilbert space of square-integrable functions defined on three-dimensional space $\mathcal{H}^{(1)}=\mathcal{H}\left(L^2(\mathbb{R}^3)\right)$.
The generalization of the one-particle Hilbert space $\mathcal{H}^{(1)}$ to an $n$-particle Hilbert space $\mathcal{H}^{(n)}$ is the tensor product of one-particle Hilbert spaces
\begin{equation}
	\mathcal{H}^{(n)}
	=
	\bigotimes^n_{i=1}
	\mathcal{H}^{(1)}
	.
\end{equation}
Now, it is possible to have a superposition of, e.g., the vacuum state and a particle state
\begin{equation}
	\ket{\psi}
	=
	c_1
	\ket{0}
	+
	c_2
	\ket{1_f}
\end{equation}
with $c_1,c_2\in\mathbb{C}$ which means that we need to combine orthonormal $n$-particle states.
We first construct a tensor algebra over the Hilbert space $\mathcal{H}^{(1)}$ as the direct sum~\cite[p.~290]{Bogolubov1989}
\begin{equation}
	\bigoplus^\infty_{n=0}
	S_+
	\mathcal{H}^{(n)}
\end{equation}
wherein $S_+$ symmetrizes the Hilbert space for bosons.
Equipping the tensor algebra with an inner product and using the completeness of the $n$-particle Hilbert spaces, we obtain again a Hilbert space, named the symmetric Fock space $\mathcal{F}_+$~\cite[p.~35]{Haag2012}.

\subsection{Number states}

Applying the creation operator $\hat{A}^{(+)}[f]$ wherein $f$ is a smearing function or momentum spectrum satisfying the normalization condition, \cref{eq:spectrum_normalization}, suggests defining
\begin{equation}
	\ket{n_f}
	=
	\frac{1}{\sqrt{n!}}
	\hat{A}^{(+)}[f]^n
	\ket{0}
	\label{eq:number_state}
\end{equation}
as number state with spectrum $f$.\footnote{The factorial is required for normalization because bosons are indistinguishable.}
The positive and negative frequency field operators $\hat{A}^{(\pm)}(t,\vb{x})$ generalize the quantum harmonic annihilation and creation operators by adding or removing a particle with spectrum $f$ from the field
\begin{align}
	\hat{A}^{(+)}[f]
	\ket{n_f}
	&=
	\sqrt{n+1}
	\ket{{n+1}_f}
	\\
	\hat{A}^{(-)}[f]
	\ket{n_f}
	&=
	\sqrt{n}
	\ket{{n-1}_f}
	.
\end{align}
While the generalized number state $\ket{n_f}$ is still an eigenstate of the number operator to eigenvalue $n_f$,
\begin{equation}
	\hat{N}
	\ket{n_f}
	=
	n_f
	\ket{n_f}
	,
\end{equation}
and has energy and momentum expectation values
\begin{align}
	\bra{n_f}
	\hat{H}
	\ket{n_f}
	&=
	n
	\int\frac{\dd[3]{p}}{(2\pi)^3}
	\omega(\vb{p})
	\abs*{\frac{f\left(\omega(\vb{q}),\vb{q}\right)}{\sqrt{2\omega(\vb{p})}}}^2
	\\
	\bra{n_f}
	\vu{P}
	\ket{n_f}
	&=
	n
	\int\frac{\dd[3]{p}}{(2\pi)^3}
	\vb{p}
	\abs*{\frac{f\left(\omega(\vb{q}),\vb{q}\right)}{\sqrt{2\omega(\vb{p})}}}^2
	.
\end{align}
The expectation value and variance of the electric field operator are
\begin{align}
	\bra{n_f}
	\hat{E}(t,\vb{x})
	\ket{n_f}
	&=
	0
	\\
	\bra{n_f}
	\left(
		\Delta
		\hat{E}(t,\vb{x})
	\right)^2
	\ket{n_f}
	&=
	\frac{1}{2}
	\int\frac{\dd[3]{p}}{(2\pi)^3}
	\omega(\vb{p})
	+
	n
	\abs*{\Psi(t,\vb{x})}^2
	.
\end{align}
The electric field vanishes for our number states as known in quantum optics, see, for instance, Ref.~\cite{Gerry2005}, but the variance contains an additional term to the "vacuum fluctuations" from the momentum spectrum.
The vacuum fluctuations are in principle infinite, however, our detector is only able to detect a limited bandwidth which makes the vacuum fluctuations in practical applications finite again.

\subsection{Coherent states}

The interaction of a classical current $\vb{j}(t,\vb{x})$ with the Maxwell field operator in the Coulomb gauge $\vu{A}(t,\vb{x})$ is given by the interaction Hamiltonian
\begin{equation}
	\hat{H}_\text{int}(t)
	=
	-
	\int\dd[3]{x}
	\vb{j}(t,\vb{x})
	\vdot
	\hat{\vb{A}}(t,\vb{x})
	.
\end{equation}
Inserting the spatial Fourier transform of the current $\vb{j}(t,\vb{p})$ and the plane-wave expansion, \cref{eq:maxwell_positive_operator,eq:maxwell_negative_operator}, the interaction Hamiltonian becomes
\begin{equation}
	\hat{H}_\text{int}(t)
	=
	-
	\sum_{\lambda=1,2}
	\int\frac{\dd[3]{p}}{(2\pi)^3\sqrt{2\omega(\vb{p})}}
	\left\{
		\left(
			\vb{j}(t,\vb{p})^*
			\vdot
			\vu{e}_\lambda(\vb{p})
		\right)
		\hat{a}_\lambda(\vb{p})
		e^{-i\omega(\vb{p})t}
		+
		\text{H.c.}
	\right\}
	\label{eq:classical_current_interaction}
	.
\end{equation}
where we used the conjugate symmetry $\vb{j}(t,\vb{p})^*=\vb{j}(t,-\vb{p})$.

The effect of an interaction acting on a quantum state from time $t_0$ to $t$ is encoded in the time-evolution operator\footnote{See Ref.~\cite[p.~215]{Greiner2013} for an introduction into the time-evolution operator and interactions.}
\begin{equation}
	\hat{U}(t_0,t)
	=
	\mathcal{T}_+
	\exp\left\{
		-i
		\int_{t_0}^t\dd{t^\prime}
		\hat{H}_\text{int}(t^\prime)
	\right\}
	\label{eq:time_evolution_operator}
\end{equation}
wherein $\mathcal{T}_+$ denotes the time-ordering symbol.
The Magnus expansion presents a systematic approach in finding an explicit form of the time-evolution operator\footnote{See Ref.~\cite[p.~42]{QuesadaMejia2015}, for an introduction to the Magnus expansion with application to nonlinear processes.}, it is given by
\begin{equation}
	\hat{U}(t_0,t)
	=
	\exp\left\{
		\sum_{n=1}
		\Omega^{(n)}(t_0,t)
	\right\}
\end{equation}
wherein the first two terms are given by
\begin{align}
	\hat{\Omega}^{(1)}(t_0,t)
	&=
	-i
	\int_{t_0}^t\dd{t^\prime}
	\hat{H}_\text{int}(t^\prime)
	\\
	\hat{\Omega}^{(2)}(t_0,t)
	&=
	\frac{(-i)^2}{2!}
	\int_{t_0}^t\dd{t^\prime}
	\int_{t_0}^{t^\prime}\dd{t^{\prime\prime}}
	\comm{\hat{H}_\text{int}(t^\prime)}{\hat{H}_\text{int}(t^{\prime\prime})}
	.
\end{align}
For some interactions there exists no exact solution and we can truncate the expansion up to some finite term.
Compared to other expansions, e.g. the Neumann expansion, the truncated Magnus expansion is still unitary.

Let us apply the Magnus expansion to find the time-evolution operator corresponding to the interaction Hamiltonian of \cref{eq:classical_current_interaction}.
The first term of the Magnus expansion turns out to be
\begin{equation}
	\hat{\Omega}^{(1)}(t_0,t)
	=
	i
	\sum_{\lambda=1,2}
	\int\frac{\dd[3]{p}}{(2\pi)^3\sqrt{2\omega(\vb{p})}}
	\left\{
		J_\lambda(t,t_0;\vb{p})
		\hat{a}_\lambda(\vb{p})
		+
		\text{H.c.}
	\right\}
\end{equation}
where we defined the time-integrated current for polarization $\lambda$
\begin{equation}
	J_\lambda(t_0,t;\vb{p})
	=
	\int_{t_0}^t\dd{t^\prime}
	\left(
		\vb{j}(t,\vb{p})^*
		\vdot
		\vu{e}_\lambda(\vb{p})
	\right)
	e^{-i\omega(\vb{p})t^\prime}
	.
\end{equation}
The second term in the Magnus expansion turns out to be complex
\begin{equation}
	\hat{\Omega}^{(2)}(t_0,t)
	=
	i\sum_{\lambda=1,2}
	\int\frac{\dd[3]{p}}{(2\pi)^3\omega(\vb{p})}
	\Im\left\{
		J_\lambda(t_0,t^\prime;\vb{p})
		J_\lambda(t_0,t^{\prime\prime};\vb{p})^*
	\right\}
\end{equation}
which only contributes a phase to the time-evolution operator.
As the second commutator is complex-valued, and therefore commutes, higher order commutators vanish and the Magnus expansion is exact with the first two terms.
As long as we consider a single current source, no interference of phases can occur and we can ignore the complex phase originating from the second Magnus coefficient.
The time-evolution operator of the Maxwell field interacting with a classical source current therefore is~\cite[p.~168]{Itzykson2012}
\begin{equation}
	\hat{U}(t_0,t)
	=
	\exp\left\{
		i\sum_{\lambda=1,2}
		\int\frac{\dd[3]{p}}{(2\pi) ^3\sqrt{2\omega(\vb{p})}}
		\left\{
			J_\lambda(t,t_0;\vb{p})
			\hat{a}_\lambda(\vb{p})
			+
			\text{H.c.}
		\right\}
	\right\}
	.
\end{equation}
Neglecting the polarization 
\begin{equation}
	\hat{U}(t_0,t)
	=
	\exp\left\{
		\int\frac{\dd[3]{p}}{(2\pi) ^3\sqrt{2\omega(\vb{p})}}
		\left\{
			i
			J(t,t_0;\vb{p})
			\hat{a}(\vb{p})
			-
			\text{H.c.}
		\right\}
	\right\}
\end{equation}
we identify the time-evolution operator with the generalization of the displacement operator from quantum optics~\cite[p.~47]{Barnett2002}
\begin{equation}
	\lim_{t\to\infty}\hat{U}(-t,+t)
	=
	\hat{D}\left[-iJ(\vb{p})\right]
\end{equation}
where we take the generalized displacement operator to be
\begin{equation}
	\begin{split}
		\hat{D}[\alpha]
		&=
		\exp\left\{
			\hat{A}^{(+)}[\alpha]
			-
			\hat{A}^{(-)}[\alpha^*]
		\right\}
		\\
		&=
		\exp\left\{
			\int\frac{\dd[3]{p}}{(2\pi) ^3\sqrt{2\omega(\vb{p})}}
			\left\{
				\alpha(\vb{p})
				\hat{a}^\dagger(\vb{p})
				-
				\alpha(\vb{p})^*
				\hat{a}(\vb{p})
			\right\}
		\right\}
	\end{split}
	\label{eq:displacement_operator}
\end{equation}
where we identified the generalized field creation and annihilation operators, $\hat{A}^{(+)}[\alpha]$ and $\hat{A}^{(-)}[\alpha^*]$.
Noting that
\begin{equation}
	\comm{\hat{A}^{(+)}[\alpha]}{\hat{A}^{(-)}[\alpha^*]}
	=
	\int\frac{\dd[3]{p}}{(2\pi)^32\omega(\vb{p})}
	\abs*{\alpha(\vb{p})}^2
\end{equation}
we can employ the \gls{bch} formula as in Ref.~\cite[p.~48]{Barnett2002} to write the displacement operator in normal-order
\begin{equation}
	\hat{D}[\alpha]
	=
	\exp\left\{
		-
		\frac{1}{2}
		\comm{\hat{A}^{(+)}[\alpha]}{\hat{A}^{(-)}[\alpha^*]}
	\right\}
	\exp\left\{
		+
		\hat{A}^{(+)}[\alpha]
	\right\}
	\exp\left\{
		-
		\hat{A}^{(-)}[\alpha]
	\right\}
	.
\end{equation}

Now, let us discuss some properties of the displacement operator.
The product of two different displacements is equal to the sum of the displacement times a suppression factor depending on the overlap for the displacement, i.e.,
\begin{equation}
	\begin{split}
		\hat{D}[\alpha]
		\hat{D}[\beta]
		&=
		\hat{D}[\alpha+\beta]
		\exp\left\{
			-
			\frac{1}{2}
			\comm{\hat{A}^{(+)}[\alpha]}{\hat{A}^{(-)}[\beta^*]}
			+
			\frac{1}{2}
			\comm{\hat{A}^{(+)}[\beta]}{\hat{A}^{(-)}[\alpha^*]}
		\right\}
		\\
		&=
		\hat{D}[\alpha+\beta]
		\exp\left\{
			-
			\frac{1}{2}
			\int\frac{\dd{p}}{(2\pi)^32\omega(\vb{p})}
			\left\{
				\alpha(\vb{p})
				\beta(\vb{p})^*
				-
				\alpha(\vb{p})^*
				\beta(\vb{p})
			\right\}
		\right\}
		.
	\end{split}
\end{equation}
Using the product formula, we can quickly show that the displacement operator is unitary
\begin{equation}
	\hat{D}[\alpha]
	\hat{D}[\alpha]^\dagger
	=
	\hat{D}[\alpha]
	\hat{D}[-\alpha]
	=
	\mathbb{1}
\end{equation}
which is not too surprising given the time-evolution is unitary.

The radiation emitted by a classical current is coherent, suggesting to identify the quantum state produced by a classical current as the coherent state
\begin{equation}
	\ket{\alpha}
	=
	\hat{D}[\alpha]
	\ket{0}
	=
	\exp\left\{
		-
		\frac{1}{2}
		\comm{\hat{A}^{(+)}[\alpha]}{\hat{A}^{(-)}[\alpha^*]}
	\right\}
	\exp\left\{
		\hat{A}^{(+)}[\alpha]
	\right\}
	\ket{0}
	\label{eq:coherent_state}
\end{equation}
where we used that the exponential of the generalized annihilation operator acting on the vacuum state produces the vacuum state.
Coherent states are non-orthogonal, i.e.,
\begin{equation}
	\begin{split}
		\braket{\alpha}{\beta}
		&=
		\exp\left\{
			-
			\frac{1}{2}
			\comm{\hat{A}^{(+)}[\alpha]}{\hat{A}^{(-)}[\alpha^*]}
			-
			\frac{1}{2}
			\comm{\hat{A}^{(+)}[\beta]}{\hat{A}^{(-)}[\beta^*]}
			+
			\comm{\hat{A}^{(+)}[\beta]}{\hat{A}^{(-)}[\alpha^*]}
		\right\}
		\\
		&=
		\exp\left\{
			-
			\frac{1}{2}
			\int\frac{\dd{p}}{(2\pi)^32\omega(\vb{p})}
			\left\{
				\abs*{\alpha(\vb{p})}^2
				+
				\abs*{\beta(\vb{p})}^2
				-
				2\alpha(\vb{p})^*\beta(\vb{p})
			\right\}
		\right\}
		\label{eq:coherent_state_inner}
		.
	\end{split}
\end{equation}
The coherent state is an eigenstate to the annihilation operator
\begin{equation}
	\hat{a}(\vb{p})
	\ket{\alpha}
	=
	\frac{\alpha(\vb{p})}{\sqrt{2\omega(\vb{p})}}
	\ket{\alpha}
	\label{eq:coherent_state_annihilation}
\end{equation}
which makes it simple to derive the expectation value of the Hamiltonian operator
\begin{equation}
	\bra{\alpha}
	\hat{H}
	\ket{\alpha}
	=
	\int\frac{\dd{p}}{(2\pi)^3}
	\omega(\vb{p})
	\abs*{\frac{\alpha(\vb{p})}{\sqrt{2\omega(\vb{p})}}}^2
	\label{eq:coherent_state_hamilton_expval}
\end{equation}
and its variance
\begin{equation}
	\bra{\alpha}
	\left(
		\Delta
		\hat{H}
	\right)^2
	\ket{\alpha}
	=
	\int\frac{\dd{p}}{(2\pi)^3}
	\omega(\vb{p})^2
	\abs*{\frac{\alpha(\vb{p})}{\sqrt{2\omega(\vb{p})}}}^2
	\label{eq:coherent_state_hamilton_variance}
	.
\end{equation}
For the number operator we find the mean to equal the variance
\begin{align}
	\bra{\alpha}
	\hat{N}
	\ket{\alpha}
	&=
	\int\frac{\dd{p}}{(2\pi)^3}
	\abs*{\frac{\alpha(\vb{p})}{\sqrt{2\omega(\vb{p})}}}^2
	=
	\overline{n}
	\label{eq:coherent_state_number_expval}
	\\
	\bra{\alpha}
	\left(
		\Delta
		\hat{N}
	\right)^2
	\ket{\alpha}
	&=
	\overline{n}^2
	\label{eq:coherent_state_number_variance}
	,
\end{align}
i.e., the photon number to be Poisson distributed, which follows simply by setting $\omega(\vb{p})=1$ in the results obtained for the Hamilton operator.
The expectation value of the electric field operator reads
\begin{equation}
	\begin{split}
		\bra{\alpha}
		\hat{E}(t,\vb{x})
		\ket{\alpha}
		&=
		\int\frac{\dd[3]{p}}{(2\pi)^3}
		\sqrt{2\omega(\vb{p})}
		\Im\left\{
			\alpha(\vb{p})
			e^{-i\omega(\vb{p})t+i\vb{p}\vdot\vb{x}}
		\right\}
		\\
		&=
		\int\frac{\dd[3]{p}}{(2\pi)^3}
		\sqrt{2\omega(\vb{p})}
		\abs*{\alpha(\vb{p})}
		\sin\left(
			\vb{p}\vdot\vb{x}
			-
			\omega(\vb{p})t
			+
			\varphi
		\right)
	\end{split}
	\label{eq:coherent_state_electric_expval}
\end{equation}
where we used the polar representation $\alpha(\vb{p})=\abs*{\alpha(\vb{p})}e^{i\varphi}$.
The inner product of a coherent state with a number state yields
\begin{equation}
	\braket{n_f}{\alpha}
	=
	\frac{1}{\sqrt{n!}}
	e^{-\overline{n}/2}
	\left(
		\int\frac{\dd{p}}{(2\pi)^32\omega(\vb{p})}
		f\left(\omega(\vb{p}),\vb{p}\right)^*
		\alpha(\vb{p})
	\right)^n
	\label{eq:coherent_state_inner_number}
	.
\end{equation}
We have derived the generalized coherent state from the interaction of the Maxwell field with a classical current where we found the time-evolution operator to yield the displacement operator.
The generalized coherent state and displacement operators share the same properties as their single-mode quantum optics counterparts which is not too surprising given that the modes are independent of another.
		\section{Time-dependent interactions}

\subsection{Time-evolution operator}

Let $\ket{\psi(t_0)}$ be a state at time $t_0$, then the time-evolution relates the state $\ket{\psi(t)}$ at some later time $t>t_0$ to $\ket{\psi(t_0)}$ via
\begin{equation}
	\ket{\psi(t)}
	=
	\hat{U}(t,t_0)
	\ket{\psi(t_0)}
	.
\end{equation}
Inserting $\ket{\psi(t)}$ into the Schrödinger equation leads to
\begin{equation}
	i\dv{t}
	\hat{U}(t,t_0)
	=
	\hat{H}(t)
	\hat{U}(t,t_0)
\end{equation}
which formal solution is the time-ordered exponential, see Ref.~\cite[p.~380]{Bartelmann2018},
\begin{equation}
	\hat{U}(t,t_0)
	=
	T\exp\left\{
		-i
		\int_{t_0}^t\dd{t^\prime}
		\hat{H}(t^\prime)
	\right\}
\end{equation}
where $T$ denotes the time-ordering symbol.
Only for simple time-dependent systems an exact time-evolution operator exists.
In contrast to the Dyson expansion, the Magnus expansion yields a unitary time-evolution operator even for finite order, in particular,
\begin{equation}
	\hat{U}(t,t_0)
	=
	\exp\left\{
		\sum_{n=1}
		\hat{\Omega}^{(n)}(t,t_0)
	\right\}
\end{equation}
where the first two expansion terms are given by
\begin{align}
	\hat{\Omega}^{(1)}(t,t_0)
	&=
	\frac{(-i)}{1!}
	\int_{t_0}^t\dd{t^\prime}
	\hat{H}(t^\prime)
	\\
	\hat{\Omega}^{(2)}(t,t_0)
	&=
	\frac{(-i)^2}{2!}
	\int_{t_0}^t\dd{t^\prime}
	\int_{t_0}^{t^\prime}\dd{t^{\prime\prime}}
	\comm{\hat{H}(t^\prime)}{\hat{H}(t^{\prime\prime})}
\end{align}
and represent time-ordering corrections, see Ref.~\cite{QuesadaMejia2015}.

\subsection{Interaction with a classical current field}

The Schrödinger-picture Hamiltonian describing the interaction of the Maxwell field $\hat{\vb{A}}$ with a classical current $\vb{j}$ is
\begin{equation}
	\hat{H}_\text{int}(t)
	=
	-
	\int_{\mathbb{R}^3}\dd[3]{x}
	\vb{j}(t,\vb{x})
	\vdot
	\hat{\vb{A}}(t,\vb{x})
	.
\end{equation}
Taking the spatial Fourier transform of the current
\begin{equation}
	\vb{j}(t,\vb{x})
	=
	\int_{\mathbb{R}^3}\frac{\dd[3]{q}}{(2\pi)^3}
	\vb{j}(t,\vb{q})
	e^{+i\vb{q}\vdot\vb{x}}
\end{equation}
and inserting the mode expansion, we find the interaction Hamiltonian to be
\begin{equation}
	\hat{H}_\text{int}(t)
	=
	-
	\sum_{\lambda=1,2}
	\int_{\mathbb{R}^3}\frac{\dd[3]{p}}{(2\pi)^3\sqrt{2\omega(\vb{p})}}
	\left\{
		j_\lambda(t,\vb{p})
		\hat{a}_\lambda(\vb{p})
		e^{-i\omega(\vb{p})t}
		+
		\text{h.c.}
	\right\}
\end{equation}
where we have the transverse current
\begin{equation}
	j_\lambda(q_0,\vb{p})
	=
	\vb{j}(q_0,\vb{p})
	\vdot
	\vu{e}_\lambda(\vb{p})
	.
\end{equation}
The first term in the Magnus expansion turns out to be
\begin{equation}
	\hat{\Omega}^{(1)}(t,t_0)
	=
	i
	\sum_{\lambda=1,2}
	\int_{\mathbb{R}^3}\frac{\dd[3]{p}}{(2\pi)^3\sqrt{2\omega(\vb{p})}}
	\left\{
		J_\lambda(t,t_0;\vb{p})
		\hat{a}_\lambda(\vb{p})
		+
		\text{h.c.}
	\right\}
\end{equation}
where we defined
\begin{equation}
	J_\lambda(t,t_0;\vb{p})
	=
	\int_{t_0}^t\dd{t^\prime}
	j_\lambda(t^\prime,\vb{p})
	e^{-i\omega(\vb{p})t^\prime}
	.
\end{equation}
For the second term in the Magnus expansion, we first evaluate the commutator
\begin{equation}
	\comm{\hat{H}(t^\prime)}{\hat{H}(t^{\prime\prime})}
	=
	i\sum_{\lambda=1,2}
	\int_{\mathbb{R}^3}\frac{\dd[3]{p}}{(2\pi)^3\omega(\vb{p})}
	\Im\left\{
		j_\lambda(t^\prime,\vb{p})
		j_\lambda(t^{\prime\prime},\vb{p})^*
		e^{-i\omega(\vb{p})(t^\prime-t^{\prime\prime})}
	\right\}
\end{equation}
and notice that it is complex valued, hence, all higher commutators vanish and the Magnus expansion with the first two terms is exact.
In summary, the second term of the Magnus expansion turns out to be
\begin{equation}
	\hat{\Omega}^{(2)}(t,t_0)
	=
	i\sum_{\lambda=1,2}
	\int_{\mathbb{R}^3}\frac{\dd[3]{p}}{(2\pi)^3\omega(\vb{p})}
	\Im\left\{
		J_\lambda(t_0,t^\prime;\vb{p})
		J_\lambda(t_0,t^{\prime\prime};\vb{p})^*
	\right\}
\end{equation}
The second term contributes a phase to the time-evolution operator.
As long as we consider a single current source, no interference of phases can occur and we can ignore the phase factor.
The exact time-evolution operator of the Maxwell field interacting with a classical source current therefore is
\begin{equation}
	\hat{U}(t,t_0)
	=
	\exp\left\{
		i\sum_{\lambda=1,2}
		\int_{\mathbb{R}^3}\frac{\dd[3]{p}}{(2\pi)^3\sqrt{2\omega(\vb{p})}}
		\left\{
			J_\lambda(t,t_0;\vb{p})
			\hat{a}_\lambda(\vb{p})
			+
			\text{h.c.}
		\right\}
	\right\}
\end{equation}
which equals the displacement operator for a time-dependent spectrum $\hat{D}[\alpha(t,t_0)]$.

\subsection{Interaction with a charged particle in a potential}

Based on Refs.~\cite[p.~687]{Mandel1995},~\cite[p.~128]{Cohen1992}

The Hamiltonian of the particle with charge $q$ and mass $m$
\begin{equation}
	\hat{H}_q
	=
	\frac{1}{2m}
	\hat{\vb{p}}^2
	+
	qV(\hat{\vb{x}})
\end{equation}
wherein $V$ is a classical external potential and the position and momentum operator satisfy the canonical commutation relations
\begin{align}
	\comm{\hat{x}_i}{\hat{p}_j}
	&=
	i\delta_{ij}
	&
	\comm{\hat{x}_i}{\hat{x}_j}
	&=
	0
	=
	\comm{\hat{p}_i}{\hat{p}_j}
	.
\end{align}
Interaction of the charged particle with an electromagnetic potential $\hat{\vb{A}}$ due to minimal coupling os obtained by the replacement~\cite{Itzykson2012}
\begin{equation}
	\hat{\vb{p}}
	\to
	\hat{\vb{p}}
	-
	q\vb{\hat{A}}
\end{equation}
and expanding the kinetic term of the particle's Hamiltonian
\begin{equation}
	\begin{split}
		\frac{1}{2m}
		\left(\hat{\vb{p}}-q\hat{\vb{A}}\right)^2
		&=
		\frac{1}{2m}
		\hat{\vb{p}}^2
		-
		\frac{q}{2m}
		\left(
			\hat{\vb{p}}
			\vdot
			\hat{\vb{A}}
			+
			\hat{\vb{A}}
			\vdot
			\hat{\vb{p}}
		\right)
		+
		\frac{q^2}{2m}
		\hat{\vb{A}}^2
		\\
		&=
		\frac{1}{2m}
		\hat{\vb{p}}^2
		-
		\frac{q}{m}
		\hat{\vb{p}}
		\vdot
		\hat{\vb{A}}
		+
		\frac{q^2}{2m}
		\hat{\vb{A}}^2
	\end{split}
\end{equation}
where we used that the particle's momentum and the Maxwell field operator commute in the Coulomb gauge, see Ref.~\cite[p.~687]{Mandel1995}.
The interaction term quadratic in the Maxwell field becomes relevant for very high intensities, see Ref.~\cite[p.~198]{Cohen1989} and Ref.~\cite[p.~689]{Mandel1995} for a detailed discussion, which we are not relevant for out use case.
We conclude the interaction Hamiltonian to be
\begin{equation}
	\hat{H}_\text{int}(t,\vb{x})
	=
	-\frac{q}{m}
	\hat{\vb{p}}(t)
	\vdot
	\hat{\vb{A}}(t,\vb{x})
	.
\end{equation}
In the dipole approximation, we consider the Maxwell field constant over the support of the particle wave function~\cite[p.~688]{Mandel1995} and thus
\begin{equation}
	\hat{\vb{A}}(t,\vb{x})
	\ket{\Psi}
	\approx
	\hat{\vb{A}}(t,\vb{x}_0)
	\ket{\Psi}
\end{equation}
with $\vb{x}_0$ being the particle's \gls{com}.

For an alternative approach on how the minimal coupling leads to the dipole interaction, see Ref.~\cite[p.~635]{Cohen1992}.

\subsection{Photodetection}

\subsubsection{Gerry and Knight}

Based on Ref.~\cite[p.~120]{Gerry2005}.

We have the dipole interaction
\begin{equation}
	\hat{H}_\text{int}(t,\vb{x})
	=
	-
	\hat{\vb{d}}
	\vdot
	\hat{\vb{E}}(\vb{x},t)
	=
	-
	\hat{\vb{d}}
	\vdot
	\left(
		\hat{\vb{E}}^{(+)}(\vb{x},t)
		+
		\hat{\vb{E}}^{(-)}(\vb{x},t)
	\right)
\end{equation}
wherein
\begin{equation}
	\hat{\vb{E}}^{(+)}(\vb{x},t)
	=
	i\sum_\lambda
	\int\frac{\dd[3]{p}}{(2\pi)^3\sqrt{2\omega}}
	\vu{e}_\lambda(\vb{p})
	\hat{a}_\lambda(\vb{p})
	e^{+i\vb{p}\vdot\vb{x}}
	\approx
	i\sum_\lambda
	\int\frac{\dd[3]{p}}{(2\pi)^3\sqrt{2\omega}}
	\vu{e}_\lambda(\vb{p})
	\hat{a}_\lambda(\vb{p})
\end{equation}
where the dipole approximation $\norm{\vb{p}\vdot\vb{x}}\ll1$ has been implemented using $e^{i\vb{p}\vdot\vb{x}}\approx1$.

The matrix element for the photoemission is
\begin{equation}
	\bra{e,f}\hat{H}_\text{int}\ket{g,i}
	=
	-
	\bra{e}\hat{\vb{d}}\ket{g}
	\bra{f}\hat{\vb{E}}^{(+)}(\vb{x},t)\ket{i}
	.
\end{equation}
We do not care about the final states of the field, hence we marginalize the probability of an absorption
\begin{equation}
	\sum_f\abs{\bra{f}\hat{\vb{E}}^{(+)}(\vb{x},t)\ket{i}}^2
	=
	\sum_f
	\bra{i}
	\hat{\vb{E}}^{(-)}
	\ketbra{f}
	\hat{\vb{E}}^{(+)}
	\ket{i}
	=
	\expval{\hat{\vb{E}}^{(-)}\vdot\hat{\vb{E}}^{(+)}}{i}
	.
\end{equation}
For a more general initial radiation state $\hat\rho_i=\sum_jp_j\ketbra{j}$, we can rewrite the previous equation as
\begin{equation}
	\tr\left\{
		\hat\rho_f
		\hat{\vb{E}}^{(-)}
		\vdot
		\hat{\vb{E}}^{(+)}
	\right\}
	.
\end{equation}

\subsubsection{Cohen-Tannoudji}

Based on Ref.~\cite[p.~128]{Cohen1992}.

\begin{equation}
	\hat{H}
	=
	\hat{H}_a
	+
	\hat{H}_f
	+
	\hat{H}_i
\end{equation}
wherein \textcolor{red}{need to show this!}
\begin{equation}
	\hat{H}_i
	=
	-
	\hat{\vb{d}}
	\vdot
	\hat{\vb{E}}
\end{equation}
Assuming the dipole moment and the electric field to be parallel, we find in the interaction picture
\begin{align}
	\hat{\vb{d}}(t)
	&=
	e^{+i\hat{H}_at}
	\hat{\vb{d}}(0)
	e^{-i\hat{H}_at}
	\\
	\hat{\vb{E}}(t)
	&=
	e^{+i\hat{H}_ft}
	\hat{\vb{E}}(0)
	e^{-i\hat{H}_ft}
	.
\end{align}
We have a photoelectron emission in time interval $\Delta t$ whenever an electron is excited from the ground state $\ket{g}$, i.e.,
\begin{equation}
	p_{\Delta t}
	=
	\sum_{f,e}
	\abs{\bra{f,e}\hat{U}(\Delta t)\ket{i,g}}^2
\end{equation}
where we marginalized the final states as we do not care about them, and the interaction time-evolution operator is
\begin{equation}
	\hat{U}(\Delta t)
	=
	\mathcal{T}_+
	\exp\left\{
		-i\int_0^{\Delta t}\dd{t^\prime}
		\hat{H}_\text{int}(t^\prime)
	\right\}
	.
\end{equation}
Expanding the time-evolution operator, we find
\begin{equation}
	p_{\Delta t}
	=
	\sum_{f\neq i,e\neq g}
	\int_0^{\Delta t}\dd{t^\prime}
	\int_0^{\Delta t}\dd{t^{\prime\prime}}
	\abs{\bra{f,e}\hat{H}_\text{int}(t^\prime)\hat{H}_\text{int}(t^{\prime\prime})\ket{i,g}}^2
\end{equation}
where the first term (zeroth order in $\hat{H}_\text{int}$ vanishes because of the orthogonality of the final and initial states, and the second term vanishes because the dipole moment operator is asymmetric~\cite[p.~131]{Cohen1992}.
Writing out the dipole and electric field operators, we find
\begin{equation}
	\begin{split}
		p_{\Delta t}
		&=
		\sum_{f\neq i,e\neq g}
		\int_0^{\Delta t}\dd{t^\prime}
		\int_0^{\Delta t}\dd{t^{\prime\prime}}
		\bra{i}
		\hat{\vb{E}}(t^\prime)
		\ketbra{f}
		\hat{\vb{E}}(t^{\prime\prime})
		\ket{i}
		\bra{g}
		\hat{\vb{d}}(t^\prime)
		\ketbra{e}
		\hat{\vb{d}}(t^{\prime\prime})
		\ket{g}
		\\
		&=
		\int_0^{\Delta t}\dd{t^\prime}
		\int_0^{\Delta t}\dd{t^{\prime\prime}}
		\bra{i}
		\hat{\vb{E}}(t^\prime)
		\left(
			\sum_f
			\ketbra{f}
		\right)
		\hat{\vb{E}}(t^{\prime\prime})
		\ket{i}
		\bra{g}
		\hat{\vb{d}}(t^\prime)
		\left(
			\sum_e
			\ketbra{e}
		\right)
		\hat{\vb{d}}(t^{\prime\prime})
		\ket{g}
		\\
		&=
		\int_0^{\Delta t}\dd{t^\prime}
		\int_0^{\Delta t}\dd{t^{\prime\prime}}
		\bra{i}
		\hat{\vb{E}}(t^\prime)
		\hat{\vb{E}}(t^{\prime\prime})
		\ket{i}
		\bra{g}
		\hat{\vb{d}}(t^\prime)
		\hat{\vb{d}}(t^{\prime\prime})
		\ket{g}
		\\
		&=
		\int_0^{\Delta t}\dd{t^\prime}
		\int_0^{\Delta t}\dd{t^{\prime\prime}}
		G_i(t^\prime,t^{\prime\prime})^*
		G_g(t^\prime,t^{\prime\prime})
	\end{split}
\end{equation}
where $G_i$ and $G_g$ are the two-time correlation functions of the atomic detector system and radiation field.

\subsubsection{Mandel and Wolf}

In the model presented in Ref.~\cite[p.~685]{Mandel1995}, the interaction Hamiltonian is~\cite[p.~689]{Mandel1995}
\begin{equation}
	\hat{H}_\text{int}(t)
	=
	-
	\hat{\vb{p}}(t)
	\vdot
	\hat{\vb{A}}(\vb{x}_0,t)
\end{equation}
wherein $\vb{x}_0$ is the detector atom's \gls{com}.
The interaction term can be transformed into the dipole moment operator $\hat{\vb{d}}$ and the dielectric displacement field operator $\hat{\vb{D}}$~\cite[p.~689]{Mandel1995} giving a similar interaction term as discussed by Cohen-Tannoudji.

We then consider a bound electron in a potential well. When bound, the electron state $\ket{\Psi_0}$ satisfies
\begin{equation}
	\hat{H}_a
	\ket{g}
	=
	E_g
	\ket{g}
	=
	-\omega_g
	\ket{g}
	.
\end{equation}
Let $\hat\rho_f$ be the radiation field state in the Schrödinger picture
\begin{equation}
	\hat\rho^{(S)}(t_0)
	=
	\ketbra{g}
	\otimes
	\hat\rho_f(t_0)
\end{equation}
then in the interaction picture, the state is~\cite[p.~685]{Mandel1995}
\begin{equation}
	\hat\rho^{(I)}(t)
	=
	e^{+i\hat{H}_0(t-t_0)}
	\hat\rho^{(S)}(t)
	e^{-i\hat{H}_0(t-t_0)}
\end{equation}
and the electron's momentum operator takes the form
\begin{equation}
	\hat{\vb{p}}(t)
	=
	e^{+i\hat{H}_a(t-t_0)}
	\hat{\vb{p}}
	e^{-i\hat{H}_a(t-t_0)}
\end{equation}
and the Maxwell field is
\begin{equation}
	\begin{split}
		\hat{\vb{A}}(\vb{x}_0,t)
		&=
		\hat{\vb{A}}^{(+)}(\vb{x}_0,t)
		+
		\hat{\vb{A}}^{(-)}(\vb{x}_0,t)
		\\
		&=
		\sum_{\lambda=1,2}
		\int_{\mathbb{R}^3}
		\frac{\dd[3]{p}}{(2\pi)^3\sqrt{2\omega(\vb{p})}}
		\hat{a}_\lambda(\vb{p})
		\boldsymbol{\varepsilon}_\lambda(\vb{p})
		e^{+i\vb{p}\vdot\vb{x}_0-i\omega(\vb{p})(t-t_0)}
		+
		\text{h.c.}
		.
	\end{split}
\end{equation}
In the interaction picture, the quantum state evolves according to
\begin{equation}
	\dv{\hat\rho^{(I)}}{t}
	=
	i\comm{\hat\rho^{(I)}}{\hat{H}^{(I)}_\text{int}}
\end{equation}
which can be solved by Magnus expansion.
For instance,
\begin{equation}
	\hat\rho^{(I)}(t)
	=
	\hat\rho(t_0)
	+
	i\int_{t_0}^t\dd{t^\prime}
	\comm{\hat\rho(t_0)}{\hat{H}_\text{int}(t^\prime)}
	+
	i^2
	\int_{t_0}^t\dd{t^\prime}
	\int_{t_0}^{t^\prime}\dd{t^{\prime\prime}}
	\dots
\end{equation}

The probability amplitude for the transition
\begin{equation}
	\ket{g,i}
	\to
	\ket{e,f}
\end{equation}
is equal to~\cite[p.~686]{Mandel1995}
\begin{equation}
	\begin{split}
		p(t_0,\Delta t)
		=
		\tr\left\{
			\hat\rho_{e,f}
			\hat\rho_{g,i}(t_0+\Delta t)
		\right\}
		&=
		\tr\left\{
			\hat\rho_{e,f}
			\hat\rho_{g,i}(t_0)
		\right\}
		\\
		&+
		\frac{1}{i}
		\tr\left\{
			\hat\rho_{e,f}
			\int_{t_0}^{t_0+\Delta t}
			\dd{t^\prime}
			\comm{\hat{H}_\text{int}(t^\prime)}{\hat\rho_{g,i}(t_0)}
		\right\}
		\\
		&+
		\frac{1}{i^2}
		\tr\left\{
			\hat\rho_{e,f}
			\int_{t_0}^{t_0+\Delta t}\dd{t^\prime}
			\int_{t_0}^{t^\prime}\dd{t^{\prime\prime}}
			\comm{\hat{H}_\text{int}(t^\prime)}{\comm{\hat{H}_\text{int}(t^{\prime\prime})}{\hat\rho_{g,i}(t_0)}}
		\right\}
	\end{split}
\end{equation}
the first two terms vanish and we have
\begin{equation}
	p(t_0,\Delta t)
	=
	\int_{t_0}^{t_0+\Delta t}\dd{t^\prime}
	\int_{t_0}^{t^\prime}\dd{t^{\prime\prime}}
	\expval{\hat{H}_\text{int}(t^\prime)\hat\rho(t_0)\hat{H}_\text{int}(t^{\prime\prime})}{e,f}
	+
	\text{c.c.}
\end{equation}
We now take the interaction Hamiltonian
\begin{equation}
	\hat{H}_\text{int}(t)
	=
	e^{+i\hat{H}_a(t-t_0)}
	\hat{\vb{p}}
	e^{-i\hat{H}_a(t-t_0)}
	\hat{\vb{A}}(\vb{x}_0,t)
\end{equation}
which we evaluate with \textcolor{red}{check this!}
\begin{equation}
	\begin{split}
		\expval{\hat{H}_\text{int}(t^\prime)\hat\rho(t_0)\hat{H}_\text{int}(t^{\prime\prime})}{e,f}
		&=
		\bra{e}\hat{p}_i\ket{g}
		\bra{g}\hat{p}_j\ket{e}
		e^{i(E-E_0)(t^\prime-t^{\prime\prime})}
		\\
		&\times
		\bra{f}
		\hat{A}_i(\vb{x}_0,t^\prime)
		\braket{i}
		\hat{A}_j(\vb{x}_0,t^{\prime\prime})
		\ket{f}
		+
		\text{c.c.}
	\end{split}
\end{equation}
Expanding the initial state in the coherent state basis
\begin{equation}
	\ket{i}
	=
	\int\dd[2]{\alpha}
	p_i(\alpha,t_0)
	\ketbra{\alpha}
\end{equation}
we can use the eigenvalue equation of the coherent state and sum over all final states to remove the final state dependency, i.e.,
\begin{equation}
	\sum_i
	p(t_0,\Delta t)
	=
	\int_{t_0}^{t_0+\Delta t}\dd{t^\prime}
	\int_{t_0}^{t^\prime}\dd{t^{\prime\prime}}
	\bra{e}\hat{p}_i\ket{g}
	\bra{g}\hat{p}_j\ket{e}
	e^{i(E-E_0)(t^\prime-t^{\prime\prime})}
	\expval{\hat{A}_i(\vb{x}_0,t^\prime)\hat{A}_j(\vb{x}_0,t^{\prime\prime})}
	+
	\text{c.c.}
\end{equation}
We then sum of all final electron states weighted by the density of states times the the probability of being collected by the detector and find
\begin{equation}
	P(t_0,\Delta t)
	=
	\int_{t_0}^{t_0+\Delta t}\dd{t^\prime}
	\int_{t_0}^{t^\prime}\dd{t^{\prime\prime}}
	k_{ij}(t^\prime-t^{\prime\prime})
	\expval{\hat{A}_i(\vb{x}_0,t^\prime)\hat{A}_j(\vb{x}_0,t^{\prime\prime})}
	+
	\text{c.c.}
\end{equation}
see Ref.~\cite[p.~694]{Mandel1995} for an explicit representation of the response function $k_{ij}$.
Employing normal-ordering of the Maxwell field operators, we find that the second term, the vacuum contribution, becomes zero Ref.~\cite[p.~694]{Mandel1995} and we can write
\begin{equation}
	P(t_0,\Delta t)
	=
	\int_{t_0}^{t_0+\Delta t}\dd{t^\prime}
	\int_{t_0}^{t^\prime}\dd{t^{\prime\prime}}
	k_{ij}(t^\prime-t^{\prime\prime})
	\expval{\colon\hat{A}_i(\vb{x}_0,t^\prime)\hat{A}_j(\vb{x}_0,t^{\prime\prime})\colon}
	+
	\text{c.c.}
\end{equation}
\textcolor{red}{Can we rewrite this in terms of electric field operators?} -> Yes, but we need to perform more approximations

\begin{equation}
	\begin{split}
		\colon
		\hat{A}_i(\vb{x}_0,t^\prime)
		\hat{A}_j(\vb{x}_0,t^{\prime\prime})
		\colon
		&=
		\colon
		\left[
			\hat{A}_i^{(+)}(\vb{x}_0,t^\prime)
			+
			\hat{A}_i^{(-)}(\vb{x}_0,t^\prime)
		\right]
		\left[
			\hat{A}_j^{(+)}(\vb{x}_0,t^{\prime\prime})
			+
			\hat{A}_j^{(-)}(\vb{x}_0,t^{\prime\prime})
		\right]
		\colon
		\\
		&\approx
		\colon
		\left[
			\hat{A}_i^{(+)}(\vb{x}_0,t^\prime)
			\hat{A}_j^{(-)}(\vb{x}_0,t^{\prime\prime})
			+
			\hat{A}_i^{(-)}(\vb{x}_0,t^\prime)
			\hat{A}_j^{(+)}(\vb{x}_0,t^{\prime\prime})
		\right]
		\colon
		\\
		&=
		\hat{A}_i^{(+)}(\vb{x}_0,t^\prime)
		\hat{A}_j^{(-)}(\vb{x}_0,t^{\prime\prime})
		+
		\hat{A}_j^{(+)}(\vb{x}_0,t^{\prime\prime})
		\hat{A}_i^{(-)}(\vb{x}_0,t^\prime)
	\end{split}
\end{equation}
\begin{equation}
	\begin{split}
		\hat{A}_i^{(+)}(\vb{x}_0,t^\prime)
		\hat{A}_j^{(-)}(\vb{x}_0,t^{\prime\prime})
		&=
		\left(
			\sum_{\lambda=1,2}
			\int\frac{\dd[3]{p}}{(2\pi)^3\sqrt{2\omega(\vb{p})}}
			\hat{a}_\lambda^\dagger(\vb{p})
			\vu{e}^i_\lambda(\vb{p})^*
			e^{+i\omega(\vb{p})t^\prime}
		\right)
		\\
		&\times
		\left(
			\sum_{\sigma=1,2}
			\int\frac{\dd[3]{q}}{(2\pi)^3\sqrt{2\omega(\vb{q})}}
			\hat{a}_\sigma(\vb{q})
			\vu{e}^j_\sigma(\vb{q})
			e^{-i\omega(\vb{q})t^{\prime\prime}}
		\right)
	\end{split}
\end{equation}
Let us further neglect polarization
\begin{equation}
	\begin{split}
		P(t_0,\Delta t)
		&=
		\int_{t_0}^{t_0+\Delta t}\dd{t^\prime}
		\int_{t_0}^{t^\prime}\dd{t^{\prime\prime}}
		k(t^\prime-t^{\prime\prime})
		\\
		&\times
		\expval{
			\hat{A}^{(+)}(\vb{x}_0,t^\prime)
			\hat{A}^{(-)}(\vb{x}_0,t^{\prime\prime})
			+
			\hat{A}^{(+)}(\vb{x}_0,t^{\prime\prime})
			\hat{A}^{(-)}(\vb{x}_0,t^\prime)
		}
		+
		\text{c.c.}
	\end{split}
\end{equation}

\subsubsection{Vogel}

Based on Ref.~\cite[p.~48]{Vogel2006}

The minimal-coupling Hamiltonian
\begin{equation}
	\hat{H}
	=
	\int\dd[3]{x}
	\int_0^\omega\dd{\omega}
	\hat{\vb{f}}^\dagger(\vb{x},\omega)
	\hat{\vb{f}}(\vb{x},\omega)
	+
	\sum_j\frac{\left(\hat{\vb{p}}_j-q_j\hat{\vb{A}}(\hat{\vb{x}}_j,t)\right)^2}{2m_j}
	+
	\hat{W}_\text{Coul}
\end{equation}
wherein $\hat{W}_\text{Coul}$ is the Coulomb interaction between the different charges.
For bound atomic states, we can expand the Maxwell field around the atom's \gls{com} and the interaction Hamiltonian takes the form
\begin{equation}
	\hat{H}_\text{int}
	=
	-
	\sum_j\frac{q_j}{m_j}
	\hat{\vb{p}}_j
	\vdot
	\hat{\vb{A}}(\vb{x}_j)
	+
	\sum_j\frac{q_j^2}{2m_j}
	\hat{\vb{A}}(\vb{x}_j)^2
\end{equation}
In the electric-dipole approximation we have
\begin{equation}
	\hat{H}_\text{int}
	\approx
	-
	\sum_j\frac{q_j}{m_j}
	\hat{\vb{p}}_j
	\vdot
	\hat{\vb{A}}(\vb{x}_j)	
\end{equation}
\textcolor{red}{fancy reasoning why the former interaction Hamiltonian is equivalent to} (maybe \cite[p.~691]{Mandel1995} or Cohen-Tanodji?)
\begin{equation}
	\hat{H}_\text{int}(t)
	\approx
	-
	\sum_j
	\hat{\vb{d}}_j
	\vdot
	\hat{\vb{E}}(\vb{x}_j,t)
	.
\end{equation}

Based on Ref.~\cite[p.~173]{Vogel2006}

The photoemission probability is equal to
\begin{equation}
	\begin{split}
		\abs{\bra{e,f}\hat{U}(t_0,t_0+\Delta t)\ket{g,i}}^2
		&=
		\bra{e,f}
		\hat{U}(t_0,t_0+\Delta t)
		\ket{g,i}
		\\
		&\times
		\bra{g,i}
		\hat{U}^\dagger(t_0,t_0+\Delta t)
		\ket{e,f}
		\\
		&=
		\tr\biggl\{
			\bra{e,f}
			\hat{U}(t_0,t_0+\Delta t)
			\ket{g,i}
			\\
			&\times
			\bra{g,i}
			\hat{U}^\dagger(t_0,t_0+\Delta t)
			\ket{e,f}
		\biggr\}
		\\
		&=
		\tr\biggl\{
			\ketbra{e,f}
			\hat{U}(t_0,t_0+\Delta t)
			\ketbra{g,i}
			\hat{U}^\dagger(t_0,t_0+\Delta t)
		\biggr\}
		\\
		&=
		\tr\left\{
			\hat\rho_{e,f}
			\hat\rho_{g,i}(t_0+\Delta t)
		\right\}
	\end{split}
\end{equation}
wherein the time-evolution operator is
\begin{equation}
	\hat{U}(t_0,t_0+\Delta t)
	=
	\mathcal{T}_+
	\exp\left\{
		-i
		\int_{t_0}^{t_0+\Delta t}\dd{t^\prime}
		\hat{H}_\text{int}(t^\prime)
	\right\}
\end{equation}
then, evaluating the transition amplitude
\begin{equation}
	\bra{g,f}
	\hat{U}(t_0,t_0+\Delta t)
	\ket{e,i}
	=
	\mathcal{T}_+
	\sum_{n=0}^\infty
	\frac{1}{n!}
	\bra{g,f}
	\left[
		i
		\int_{t_0}^{t_0+\Delta t}\dd{t^\prime}
		\vb{d}_{fg}
		\vdot
		\hat{\vb{E}}^{(+)}(\vb{x}_0,t^\prime)
		e^{i\omega_{fg}(t^\prime-t)}
	\right]^n
	\ket{e,i}
\end{equation}
\textcolor{red}{why do we only have $E^+$ here? How to derive this exactly?}

\subsection{Photodetection}

Let us consider the composite system of an atom with a single electron and a radiation field.
\textcolor{red}{Figure where we see an electron in a potential well and how radiation can excite it...}

In the ground state $\ket{g}$, the electron is bound and satisfies
\begin{equation}
	\hat{H}_a
	\ket{g}
	=
	E_g
	\ket{g}
	.
\end{equation}
For energies $E>0$, the electron is in one of many free excited state $\ket{e}$.
\textcolor{red}{photoelectric effect. why can this be used for photodiodes where there is no ionization happening. See Cohen-Tannoudji for explanation.}

Whenever, we ionize the atom, we can collect the free electron indicating a photo detection.
The probability that such an event occurs in the time interval $[t_0,t_0+\Delta t]$ is
\begin{equation}
	p_{e,f}(t_0,t_0+\Delta t)
	=
	\abs{
		\bra{e,f}
		\hat{U}(t_0,t_0+\Delta t)
		\ket{g,i}
	}^2
	.
\end{equation}
wherein $\hat{U}$ is the time-evolution operator.
We can recast the probability in the more general density operator formalism
\begin{equation}
	\begin{split}
		p_{e,f}(t_0,t_0+\Delta t)
		&=
		\bra{e,f}
		\hat{U}(t_0,t_0+\Delta t)
		\ketbra{g,i}
		\hat{U}^\dagger(t_0,t_0+\Delta t)
		\ket{e,f}
		\\
		&=
		\tr\left\{
			\bra{e,f}
			\hat{U}(t_0,t_0+\Delta t)
			\ketbra{g,i}
			\hat{U}^\dagger(t_0,t_0+\Delta t)
			\ket{e,f}
		\right\}
		\\
		&=
		\tr\left\{
			\ketbra{e,f}
			\hat{U}(t_0,t_0+\Delta t)
			\ketbra{g,i}
			\hat{U}^\dagger(t_0,t_0+\Delta t)
		\right\}
		\\
		&=
		\tr\left\{
			\hat\varrho
			\hat{U}(t_0,t_0+\Delta t)
			\hat\rho(t_0)
			\hat{U}^\dagger(t_0,t_0+\Delta t)
		\right\}
		\\
		&=
		\tr\left\{
			\hat\varrho
			\hat\rho(t_0+\Delta t)
		\right\}
	\end{split}
\end{equation}
where we used that the trace of a scalar is the scalar in the second line and the cyclic property of the trace in the third line.
The time-evolution operator is
\begin{equation}
	\hat{U}(t_0,t)
	=
	\mathcal{T}_+
	\exp\left\{
		-i
		\int_{t_0}^t\dd{t^\prime}
		\hat{H}_\text{int}(t^\prime)
	\right\}
\end{equation}
with the interaction Hamiltonian in the electric-dipole and rotating wave approximation being
\begin{equation}
	\hat{H}_\text{int}(t)
	=
	-
	\hat{\vb{p}}(t)
	\vdot
	\hat{\vb{A}}(\vb{x}_0,t)
	.
\end{equation}
Instead of the time-ordered exponential, we can use the Magnus expansion
\begin{align}
	\hat{U}(t_0,t)
	&=
	e^{\Omega(t_0,t)}
	&
	\Omega(t_0,t)
	&=
	\sum_{n=1}\Omega^{(n)}(t_0,t)
\end{align}
The time evolved quantum state is then
\begin{equation}
	\begin{split}
		\hat\rho(t_0+\Delta t)
		&=
		\hat{U}(t_0,t_0+\Delta t)
		\hat\rho(t_0)
		\hat{U}^\dagger(t_0,t_0+\Delta t)
		\\
		&=
		\hat\rho(t_0)
		+
		\comm{\Omega(t_0,t)}{\hat\rho(t_0)}
		+
		\frac{1}{2!}
		\comm{\Omega(t_0,t)}{\comm{\Omega(t_0,t)}{\hat\rho(t_0)}}
		+
		\dots
	\end{split}
\end{equation}
where we used the \gls{bch} formula.
We perform Magnus expansion up to the first term and find the perturbative solution
\begin{equation}
	\begin{split}
		\hat\rho(t_0+\Delta t)
		\approx
		\hat\rho(t_0)
		&+
		(-i)
		\int_{t_0}^{t_0+\Delta t}\dd{t_1}
		\comm{\hat{H}_\text{int}(t_1)}{\hat\rho(t_0)}
		\\
		&+
		\frac{(-i)^2}{2!}
		\int_{t_0}^{t_0+\Delta t}\dd{t_1}
		\int_{t_0}^{t_1}\dd{t_2}
		\comm{\hat{H}_\text{int}(t_1)}{\comm{\hat{H}_\text{int}(t_2)}{\hat\rho(t_0)}}
	\end{split}	
\end{equation}
and insert the expansion into the photoemission probability
\begin{equation}
	\begin{split}
		p_{e,f}(t_0,t_0+\Delta t)
		&=
		\tr\left\{
			\hat\varrho
			\hat\rho(t_0+\Delta t)
		\right\}
		\\
		&=
		\int_{t_0}^{t_0+\Delta t}\dd{t_1}
		\int_{t_0}^{t_1}\dd{t_2}
		\tr\left\{
			\hat{H}_\text{int}(t_1)
			\hat\rho(t_0)
			\hat{H}_\text{int}(t_2)
		\right\}
		+
		\text{c.c.}
		\\
		&=
		\bra{e}\hat{p}_i\ket{g}
		\bra{g}\hat{p}_j\ket{e}
		\int_{t_0}^{t_0+\Delta t}\dd{t_1}
		\int_{t_0}^{t_1}\dd{t_2}
		e^{i(E_e-E_g)(t_1-t_2)}
		\\
		&\times
		\expval{
			\hat{A}_j(\vb{x}_0,t_2)
			\hat\rho(t_0)
			\hat{A}_i(\vb{x}_0,t_1)
		}{i}
		+
		\text{c.c.}
	\end{split}
\end{equation}
\textcolor{red}{steps to second line missing!}
We are not interested in the final states and can integrate this degree of freedom.
Furthermore, we can assume an ensemble of independent detector atoms, so the probability for a photoemission is
\begin{equation}
	p(t_0,t_0+\Delta t)
	=
	\int_{t_0}^{t_0+\Delta t}\dd{t_1}
	\int_{t_0}^{t_1}\dd{t_2}
	k_{ij}(t_1-t_2)
	\expval{
		\hat{A}_j(\vb{x}_0,t_2)
		\hat{A}_i(\vb{x}_0,t_1)
	}
	+
	\text{c.c.}
\end{equation}
wherein $k_{ij}$ encodes the microscopic properties of the detector atom ensemble, see Ref.~\cite[p.~694]{Mandel1995}.
Expansion of the Maxwell field operator into positive and negative frequency parts as well as discarding high-frequency terms, we find~\cite[p.~698]{Mandel1995}.
\begin{equation}
	p(t_0,t_0+\Delta t)
	\approx
	\int_0^{\Delta t}\dd{t^\prime}
	\int_0^{t^\prime}\dd{t^{\prime\prime}}
	k_{ij}(t^{\prime}-t^{\prime\prime})
	\expval{
		\hat{A}_j^{(+)}(\vb{x}_0,t_0+t^{\prime\prime})
		\hat{A}_i^{(-)}(\vb{x}_0,t_0+t^{\prime})
	}
	+
	\text{c.c.}
	.
\end{equation}
Furthermore, performing the quasi-monochromatic approximation
\begin{align}
	\hat{A}_j^{(+)}(\vb{x}_0,t_0+t^{\prime\prime})
	&\approx
	\hat{A}_j^{(+)}(\vb{x}_0,t_0)
	e^{+i\omega_0t^{\prime\prime}}
	\\
	\hat{A}_j^{(-)}(\vb{x}_0,t_0+t^{\prime})
	&\approx
	\hat{A}_j^{(-)}(\vb{x}_0,t_0)
	e^{-i\omega_0t^{\prime}}
\end{align}
we find
\begin{equation}
	\begin{split}
		p(t_0,t_0+\Delta t)
		&\approx
		\expval{
			\hat{A}_j^{(+)}(\vb{x}_0,t_0)
			\hat{A}_i^{(-)}(\vb{x}_0,t_0)
		}
		\int_0^{\Delta t}\dd{t^\prime}
		\int_0^{t^\prime}\dd{t^{\prime\prime}}
		k_{ij}(t^{\prime}-t^{\prime\prime})
		e^{-i\omega_0(t^\prime-t^{\prime\prime})}
		+
		\text{c.c.}
		\\
		&=
		\expval{
			\hat{A}_j^{(+)}(\vb{x}_0,t_0)
			\hat{A}_i^{(-)}(\vb{x}_0,t_0)
		}
		\int_0^{\Delta t}\dd{t^\prime}
		\int_0^{t^\prime}\dd{\tau}
		k_{ij}(\tau)
		e^{-i\omega_0\tau}
		+
		\text{c.c.}
		\\
		&=
		\expval{
			\hat{A}_j^{(+)}(\vb{x}_0,t_0)
			\hat{A}_i^{(-)}(\vb{x}_0,t_0)
		}
		\int_0^{\Delta t}\dd{t^\prime}
		\int_{-t^\prime}^{t^\prime}\dd{\tau}
		k_{ij}(\tau)
		e^{-i\omega_0\tau}
		\\
		&\approx
		\expval{
			\hat{A}_j^{(+)}(\vb{x}_0,t_0)
			\hat{A}_i^{(-)}(\vb{x}_0,t_0)
		}
		k_{ij}(\omega_0)
		\Delta t
	\end{split}
\end{equation}
as in Ref.~\cite[p.~699]{Mandel1995} where $k_{ij}(\omega_0)$ is the frequency response of the detector atom ensemble.
We assume a detector equally sensitive to both polarizations and finally find
\begin{equation}
	p(t_0,t_0+\Delta t)
	\approx
	\expval{\hat{N}}
	k(\omega_0)
	\Delta t
\end{equation}
we can even relax the quasi-monochromatic approximation a bit by writing
\begin{equation}
	p(t_0,t_0+\Delta t)
	\approx
	\int\dd{\omega}
	k(\omega)
	\expval{\hat{n}(\omega)}
	\Delta t
	.
\end{equation}
		\section*{Summary}
\addcontentsline{toc}{section}{Summary}

In the present chapter, we introduced \gls{qkd} as an example for quantum optical communication and as a mechanism for practical and secure key distribution, which, together with classical symmetric ciphers, enables secure communication with means to estimate information leakage.
We then analyzed the quantum transmission phase of qubit- and boson-based \gls{qkd} protocols generating correlated information between the receiver and transmitter.
\begin{table}[htb]
	\centering	
	\begin{tabular}{lcc}
		\toprule
			& Qubit-based & Boson-based \\
		\midrule
			Visualization & Bloch sphere & Phase space \\
			Hilbert space (dim) & Finite (two) & Countable (infinite) \\
			Measurement operator & $\vb{\hat{S}}(\vb{n})=\hat{S}_in^i$ & $\hat{X}(\vartheta)=\frac{1}{\sqrt{2}}\left(\hat{a}e^{-i\vartheta}+\hat{a}^\dagger e^{+i\vartheta}\right)$ \\
			Standard basis & $\left\{\ket{0},\ket{1}\right\}$ & $\left\{\ket{x},\ket{p}\colon x,p\in\mathbb{R}\right\}$ \\
		\bottomrule
	\end{tabular}
	\caption{Comparison of qubit- and boson-based \gls{qkd} protocols.}\label{tab:qkd_comparison}
\end{table}
By introducing the concept of qubit- and boson-based \gls{qkd} protocols, with their respective key properties summarized in \Cref{tab:qkd_comparison}, we formalized the concept of DV- and CV-QKD.
Additionally, we formulated the concept of a logical and an encoding quantum system allowing us to encode qubits onto number or coherent states, which might share some similarity with the concept of symbols and pulse-shaping in classical signal-processing.
Because the technology to prepare and measure coherent states is highly mature, most practical QKD implementations, regardless of qubit- or boson-based, transmit weak coherent states.
\begin{figure}[htb]
	\centering
	\includegraphics{figures/tikz/qkd-protocol}
	\caption{An abstract \gls{qkd} protocol comprises a binary encoder, a logical quantum system, a binary decoder, and some post-processing. The binary encoder maps bits onto a quantum state of the logical quantum system. The binary decoder extracts the bits from the logical quantum system. The logical quantum system is a subspace of a larger physical quantum system. The state encoder and decoder map between the logical and physical quantum states.}\label{fig:qkd_protocol}
\end{figure}
Given the correlated data from the quantum transmission, we compiled classical methods to distill a shared secret key between the transmitter and receiver, known as classical post-processing. 
The classical post-processing maps the discrete or continuous data from the transmission sequence to binary symbols, corrects errors, discards failed data blocks, and removes information from the partially secret key using privacy amplification.
Finally, we roughly outlined some ideas for the security analysis of QKD.

The concept of a logical and en encoding quantum layer in QKD needs further investigation but might open up new more general protocols and simplify security proofs.
Concerning our thesis, our investigations suggest developing our theoretical framework for quantum optical communication towards a coherent state transmission system.

		\addcontentsline{toc}{section}{References}
		\printbibliography[title=References]
	\end{refsection}
	
	\chapter{Interactions with optical components}
	\begin{refsection}
		\section{Laser}

% Use semiclassical laser theory as described in Wolf & Mandel. p. 929 to derive an expression for the classical current.
% Use the classical current to give coherent state.
		\section{Mode couplers}

Let $\ket{\boldsymbol{\alpha}_1}$ be a first coherent state $\vb{j}_1$ be a first classical current induced by absorption of the coherent state, i.e.,
\begin{equation*}
	\vb{j}_1(p^\mu)
	=
	i\boldsymbol{\alpha}_1(p^\mu)
	.
\end{equation*}
The first classical current is coupled to a second classical current by a classical transfer function $g(x^\mu)$
\begin{equation*}
	\vb{j}_2(x^\mu)
	=
	\int\dd[4]{y}
	g(x^\mu-y^\mu)
	\vb{j}_1(y^\mu)
\end{equation*}
which emits a second coherent state $\ket{\boldsymbol{\alpha}_2}$ with
\begin{equation*}
	\begin{split}
		\boldsymbol{\alpha}_2(p^\mu)
		=
		-i\vb{j}_2(p^\mu)
		&=
		-i
		\int\dd[4]{x}
		\vb{j}_2(x^\mu)
		e^{ip_\nu x^\nu}
		\\
		&=
		-i
		\int\dd[4]{x}
		g(x^\mu-y^\mu)
		e^{ip_\nu x^\nu}
		\int\dd[4]{y}
		\vb{j}_1(y^\mu)
		\\
		&=
		-i
		\int\dd[4]{x}
		g(x^\mu)
		e^{ip_\nu x^\nu}
		\int\dd[4]{y}
		i\boldsymbol{\alpha}_1(y^\mu)
		e^{ip_\nu y^\nu}
		\\
		&=
		g(p^\mu)
		\boldsymbol{\alpha}_1(p^\mu)
		.
	\end{split}
\end{equation*}
Alternatively, a quantum mode coupler with transfer function $g_{\lambda\lambda^\prime}$ is described by the interaction Lagrangian
\begin{equation}
	L_\text{int}
	=
	\sum_{\lambda,\lambda^\prime=1,2}
	\int\frac{\dd[3]{p}}{(2\pi)^32\omega(\vb{p})}
	\left\{
		g_{\lambda\lambda^\prime}\left(\omega(\vb{p}),\vb{p}\right)
		\hat{a}_{\text{out},\lambda^\prime}^\dagger(\vb{p})
		\hat{a}_{\text{in},\lambda}(\vb{p})
		+
		\text{h.c.}
	\right\}
	.
\end{equation}
In terms of smeared positive and negative frequency Maxwell operators, the interaction Lagrangian takes the form~\cite[p.~130]{Haroche2006}
\begin{equation*}
	\begin{split}
		L_\text{int}
		&=
		\int\dd[4]{x}
		\mathcal{L}_\text{int}
		=
		\hat{\vb{A}}_\text{out}^+[g_\text{out}]
		\hat{\vb{A}}_\text{in}^-[g_\text{in}]
		+
		\text{h.c.}
		\\
		&=
		\int\frac{\dd[3]{p}}{(2\pi)^3\sqrt{2\omega(\vb{p})}}
		\sum_{\lambda=1,2}
		g_{\text{out},\lambda}\left(\omega(\vb{p}),\vb{p}\right)^*
		\hat{a}_{\text{out},\lambda}^\dagger(\vb{p})
		\\
		&\times
		\int\frac{\dd[3]{q}}{(2\pi)^3\sqrt{2\omega(\vb{q})}}
		\sum_{\lambda^\prime=1,2}
		g_{\text{in},\lambda^\prime}\left(\omega(\vb{q}),\vb{q}\right)
		\hat{a}_{\text{in},\lambda^\prime}(\vb{q})
		+
		\text{h.c.}
		\\
		&=
		\int\frac{\dd[3]{p}}{(2\pi)^32\omega(\vb{p})}
		\sum_{\lambda,\lambda^\prime=1,2}
		g_{\lambda,\lambda^\prime}\left(\omega(\vb{p}),\vb{p}\right)
		\hat{a}_{\text{out},\lambda}^\dagger(\vb{p})
		\hat{a}_{\text{in},\lambda^\prime}(\vb{p})
	\end{split}
\end{equation*}
where we used
\begin{equation*}
	g_{\lambda,\lambda^\prime}\left(\omega(\vb{p}),\vb{p}\right)
	=
	(2\pi)^3\delta^{(3)}(\vb{q}-\vb{p})
	g_{\text{out},\lambda}\left(\omega(\vb{p}),\vb{p}\right)^*
	g_{\text{in},\lambda^\prime}\left(\omega(\vb{q}),\vb{q}\right)	
\end{equation*}
which appears somewhat reasonable when we consider first-order momentum transfer.
The scattering operator is
\begin{equation*}
	\begin{split}
		\hat{S}
		=
		T\exp\left\{
			iL_\text{int}
		\right\}
		&=
		T\exp\left\{
			i
			\hat{\vb{A}}_\text{out}^+[g_\text{out}]
			\hat{\vb{A}}_\text{in}^-[g_\text{in}]
			+
			i
			\hat{\vb{A}}_\text{in}^-[g_\text{in}]
			\hat{\vb{A}}_\text{out}^+[g_\text{out}]
		\right\}
		\\
		&=
		\sum_{n=0}^\infty
		\frac{i^n}{n!}
		T\left[
			\hat{\vb{A}}_\text{out}^+[g_\text{out}]
			\hat{\vb{A}}_\text{in}^-[g_\text{in}]
			+
			\hat{\vb{A}}_\text{in}^-[g_\text{in}]
			\hat{\vb{A}}_\text{out}^+[g_\text{out}]
		\right]^n
		\\
		&=
		\sum_{n=0}^\infty
		\frac{i^n}{n!}
		\sum_{m=0}^n
		\binom{n}{m}
		T\left[
			\hat{\vb{A}}_\text{out}^+[g_\text{out}]
			\hat{\vb{A}}_\text{in}^-[g_\text{in}]
		\right]^m
		T\left[
			\hat{\vb{A}}_\text{in}^-[g_\text{in}]
			\hat{\vb{A}}_\text{out}^+[g_\text{out}]
		\right]^{n-m}
	\end{split}
\end{equation*}
The first power term is
\begin{equation*}
	\begin{split}
		&
		T\left[
			\hat{\vb{A}}_\text{out}^+[g_\text{out}]
			\hat{\vb{A}}_\text{in}^-[g_\text{in}]
		\right]^m
		\\
		=&\
		T\left[
			\int\dd[4]{x}
			g_\text{out}(x^\mu)
			\hat{\vb{A}}_\text{out}^+(x^\mu)
			\int\dd[4]{y}
			g_\text{in}(y^\mu)
			\hat{\vb{A}}_\text{in}^-(y^\mu)
		\right]^m
		\\
		=&\
		\int\dd[4]{x_1}
		\int\dd[4]{y_1}
		g_\text{out}(x^\mu_1)
		g_\text{in}(y^\mu_1)
		\dots
		\int\dd[4]{x_m}
		\int\dd[4]{y_m}
		g_\text{out}(x^\mu_m)
		g_\text{in}(y^\mu_n)
		T\left[
			\hat{\vb{A}}_\text{out}^+(x^\mu)^m
			\hat{\vb{A}}_\text{in}^-(y^\mu)^m
		\right]
	\end{split}
\end{equation*}

\subsection{Phase modulator}

The action describing the absorption and emission of a field while adding a possibly time-dependent phase-shift is
\begin{equation*}
	\begin{split}
		S
		&=
		\frac{1}{2}
		\int\dd[4]{x}
		\int\dd[4]{y}
		\hat\phi_\text{out}^+(x^\mu)
		G(x^\mu-y^\mu)
		\hat\phi_\text{in}^-(y^\mu)
		\\
		&=
		\frac{1}{2}
		\int\frac{\dd[3]{p}}{(2\pi)^3}
		G(p^\mu)
		\int\dd[4]{x}
		\hat\phi_\text{out}^+(x^\mu)
		e^{-ip_\nu x^\nu}
		\int\dd[4]{y}
		\hat\phi_\text{in}^-(y^\mu)
		e^{+ip_\nu y^\nu}
		\\
		&=
		\frac{1}{2}
		\int\frac{\dd[3]{p}}{(2\pi)^32\omega(\vb{p})}
		G\left(\omega(\vb{p}),\vb{p}\right)
		\hat{a}_\text{out}^\dagger(\vb{p})
		\hat{a}_\text{in}(\vb{p})
	\end{split}
\end{equation*}
and the corresponding scattering operator becomes
\begin{equation*}
	\hat{S}
	=
	T\left[
		\exp\left\{
			iS
		\right\}
	\right]
	=
	T\left[
		\exp\left\{
			\frac{i}{2}
			\int\frac{\dd[3]{p}}{(2\pi)^32\omega(\vb{p})}
			G\left(\omega(\vb{p}),\vb{p}\right)
			\hat{a}_\text{out}^\dagger(\vb{p})
			\hat{a}_\text{in}(\vb{p})
		\right\}
	\right]
\end{equation*}
		\documentclass[tikz]{standalone}

\usepackage{circuitikz}
\usepackage{physics}

\usetikzlibrary{calc,positioning}

\begin{document}
	\begin{tikzpicture}[
		node distance=4em,
		arrow/.style={-latex},
		block/.style={draw, very thick, fill=white, minimum height=10ex, minimum width=8em},
	]
		\coordinate (in) at (0,0);
		\node (pc) [block, right=of in, align=center] {$\Delta n(\omega,t)$};
		node[below right] {};	
		\node (osc) [vsourcesinshape, above=3em of pc, rotate=90, anchor=west] {};
		\coordinate [right=of pc] (out);
		
		\draw[arrow] (in) -- (pc) node[above, midway] {$\ket{\alpha(t)}$};
		\draw[arrow] (pc) -- (out) node[above, midway] {$\ket{\alpha^\prime(t)}$};
		\draw[arrow] (osc.west) -- (pc.north) node[left, midway] {$x(t)$};
	\end{tikzpicture}
\end{document}

		\section{Detector}

\subsection{Quantum measurement}

Results we expect from POVM?

\subsection{Classical sink}

Classic electric current from a coherent state.

\subsection{Harmonic oscillator}

Quantum harmonic oscillator coupled to Klein-Gordon field.

\subsection{Ensemble harmonic oscillator}

Ensemble of quantum harmonic oscillators.
Quantum stochastics.
		
		\addcontentsline{toc}{section}{References}
		\printbibliography[title=References]
	\end{refsection}

	\chapter{Coherent transmitter and receiver}
	\begin{refsection}
		\section{Overview and architecture}
		\section{Transmitter}
		\section{Receiver}
		
		\addcontentsline{toc}{section}{References}
		\printbibliography[title=References]
	\end{refsection}

	\chapter{Conclusion and outlook}
	\begin{refsection}	
		\addcontentsline{toc}{section}{References}
		\printbibliography[title=References]
	\end{refsection}

	\appendix

\end{document}

% approximately proportional to symbol
% https://tex.stackexchange.com/questions/33538/how-to-get-an-approximately-proportional-to-symbol
\def\app#1#2{%
    \mathrel{%
        \setbox0=\hbox{$#1\sim$}%
        \setbox2=\hbox{%
            \rlap{\hbox{$#1\propto$}}%
            \lower1.1\ht0\box0%
        }%
        \raise0.25\ht2\box2%
    }%
}
\def\approxprop{\mathpalette\app\relax}

% overwrite real and imaginary part operators
\let\Re\undefined
\let\Im\undefined
\DeclareMathOperator{\Re}{\operatorname{Re}}
\DeclareMathOperator{\Im}{\operatorname{Im}}
% other functions
\DeclareMathOperator{\sinc}{\operatorname{sinc}}

\newcommand{\floor}[1]{\left\lfloor#1\right\rfloor}
\newcommand{\ceil}[1]{\left\lceil#1\right\rceil}

% transpose
% https://tex.stackexchange.com/questions/403104/small-caps-mathsf-font-for-writing-transpose-of-a-matrix
\newcommand{\trans}{{\scriptscriptstyle\mathsf{T}}}

% theorems
\newtheorem{theorem}{Theorem}[section]
\newtheorem{lemma}[theorem]{Lemma}
\newtheorem{corollary}[theorem]{Corollary}
\theoremstyle{definition}
\newtheorem{definition}{Definition}[section]
\newtheorem{conjecture}{Conjecture}[section]
\newtheorem{example}{Example}[section]
\theoremstyle{remark}
\newtheorem*{remark}{Remark}
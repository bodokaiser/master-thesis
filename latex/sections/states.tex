\section{Quantum states and operators}

\subsection{Vacuum state}

\subsection{Positive-frequency operator}

\subsection{Number state}

\subsection{Displacement operator}

\begin{equation}
	\hat{D}[\alpha]
	=
	\exp\left\{
		\int\frac{\dd[3]{p}}{(2\pi)^3\sqrt{2\omega(\vb{p})}}
		\left\{
			\alpha(\vb{p})
			\hat{a}_{\vb{p}}^\dagger
			-
			\alpha(\vb{p})^*
			\hat{a}_{\vb{p}}
		\right\}
	\right\}
\end{equation}

\subsection{Coherent state}

\begin{equation}
	\ket{\alpha}
	=
	\exp\left\{
		-
		\frac{1}{2}
		\int\frac{\dd[3]{p}}{(2\pi)^32\omega(\vb{p})}
		\abs{\alpha(\vb{p})}^2
	\right\}
	\exp\left\{
		\int\frac{\dd[3]{p}}{(2\pi)^3}
		\frac{\alpha(\vb{p})}{\sqrt{2\omega(\vb{p})}}
		\hat{a}_{\vb{p}}^\dagger
	\right\}
	\ket{0}
\end{equation}

\subsection{Quadrature operator}

We define the generalized quadrature operator by 
\begin{equation}
	\hat{X}(\theta)
	=
	\frac{6\pi^2}{\Lambda^3}
	\int_{\mathbb{R}^3}
	\frac{\dd[3]{p}}{(2\pi)^3}
	\frac{1}{\sqrt{2}}
	\left\{
		\hat{a}_{\vb{p}}
		e^{-i\theta}
		+
		\hat{a}_{\vb{p}}
		e^{+i\theta}
	\right\}
\end{equation}
where the prefactor ensures that the commutator takes the standard form
\begin{equation}
	\comm{\hat{X}(\theta)}{\hat{X}(\theta+\Delta\theta)}
	=
	\frac{i}{2}
	\sin(\Delta\theta)
	.
\end{equation}
and $\Lambda$ is used as the cut-off momentum for the otherwise infinite momentum integral
\begin{equation}
	\int_{\mathbb{R}^3}\frac{\dd[3]{p}}{(2\pi)^3}
	=
	\frac{4\pi}{(2\pi)^3}
	\int_0^\Lambda\dd{p}p^2
	=
	\frac{\Lambda^3}{6\pi^2}
	.
\end{equation}
The Robertson uncertainty relation yields a lower bound for the variances
\begin{equation}
	\expval{\left(\Delta\hat{X}(\theta)\right)^2}
	\expval{\left(\Delta\hat{X}(\theta+\Delta\theta)\right)^2}
	\geq
	\frac{1}{4}
	\sin(\Delta\theta)^2
\end{equation}
and we conclude that there is maximum uncertainty for $\Delta\theta=\pi/2$.

\begin{equation}
	\expval{\hat{X}(\theta)}{\alpha}
	=
	\frac{\sqrt{2}6\pi^2}{\Lambda^3}
	\int_{\mathbb{R}^3}
	\frac{\dd[3]{p}}{(2\pi)^3}
	\Re{\frac{\alpha(\vb{p})e^{-i\theta}}{\sqrt{2\omega(\vb{p})}}}
\end{equation}
\begin{equation}
	\expval{\left(\Delta\hat{X}(\theta)\right)^2}{\alpha}
	=
	\frac{1}{2}
\end{equation}

\subsection{Electromagnetic field operator}
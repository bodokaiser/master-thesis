\section{Time-dependent mode coupler}

The mode coupler is a generalization of the beam splitter and is described by the Hamiltonian
\begin{equation}
	\hat{H}
	=
	\hat{H}_a
	+
	\hat{H}_b
	+
	\hat{H}_{ab}(t)
\end{equation}
where
\begin{align}
	\hat{H}_a
	=
	\int_{\mathbb{R}^3}\frac{\dd[3]{p}}{(2\pi)^3}
	\omega(\vb{p})
	\hat{a}^\dagger(\vb{p})
	\hat{a}(\vb{p})
	&&
	\hat{H}_b
	=
	\int_{\mathbb{R}^3}\frac{\dd[3]{p}}{(2\pi)^3}
	\omega(\vb{p})
	\hat{b}^\dagger(\vb{p})
	\hat{b}(\vb{p})
\end{align}
are bosonic free field Hamiltonians and
\begin{equation}
	\hat{H}_{ab}(t)
	=
	-
	\int\dd[3]{x}
	g(t,\vb{x})
	\hat{A}(t,\vb{x})
	\hat{B}(t,\vb{x})
\end{equation}
is the most general linear interaction between these fields.
Expanding the fields into their positive and negative frequency parts and only keeping the mixed terms
\begin{equation}
	\begin{split}
		\hat{H}_{ab}
		&=
		-
		\int\dd[3]{x}
		g(t,\vb{x})
		\left(
			\hat{A}^{(+)}(t,\vb{x})
			+
			\hat{A}^{(-)}(t,\vb{x})
		\right)
		\left(
			\hat{B}^{(+)}(t,\vb{x})
			+
			\hat{B}^{(-)}(t,\vb{x})
		\right)
		\\
		&\approx
		-
		\int\dd[3]{x}
		g(t,\vb{x})
		\left\{
			\hat{A}^{(-)}(t,\vb{x})
			\hat{B}^{(+)}(t,\vb{x})
			+
			\hat{A}^{(+)}(t,\vb{x})
			\hat{B}^{(-)}(t,\vb{x})
		\right\}
	\end{split}
\end{equation}
called the rotating wave approximation, see Ref.~\cite[p.~158]{Gardiner2000}, because the non-mixed terms are strongly oscillating.
An alternative argument given in Ref.~\cite[p.~127]{Haroche2006} justifies keeping only the mixed terms because the non-mixed terms do not conserve energy.

\subsection{Phase-shifter}

\subsection{Beam splitter}
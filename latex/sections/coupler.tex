\section{Generalized mode coupler}

The mode coupler is a generalization of the beam splitter and is described by the Hamiltonian
\begin{equation}
	\hat{H}
	=
	\hat{H}_0
	+
	\hat{H}_\text{int}(t)
\end{equation}
where we have the Hamiltonian of the scalar Maxwell field
\begin{equation}
	\hat{H}_0
	=
	\int_{\mathbb{R}^3}\frac{\dd[3]{p}}{(2\pi)^3}
	\omega(\vb{p})
	\hat{a}^\dagger(\vb{p})
	\hat{a}(\vb{p})
\end{equation}
and the interaction Hamiltonian
\begin{equation}
	\hat{H}_\text{int}(t)
	=
	-
	\int\dd[3]{x}
	g(t,\vb{x})
	\hat{A}(t,\vb{x})
	\hat{A}(t,\vb{x})
	.
\end{equation}
We expand the fields into their positive and negative frequency parts and only keep the mixed terms known as the rotating wave approximation~\cite[p.~158]{Gardiner2000}
\begin{equation}
	\hat{H}_\text{int}(t)
	=
	-
	\int\dd[3]{x}
	g(t,\vb{x})
	\left[
		\hat{A}^{(-)}(t,\vb{x})
		\hat{A}^{(+)}(t,\vb{x})
		+
		\hat{A}^{(+)}(t,\vb{x})
		\hat{A}^{(-)}(t,\vb{x})
	\right]
	.
\end{equation}
The rotating wave approximation is justified when for long time scales relative to the highly oscillating terms.
Additionally, Haroche~\cite[p.~127]{Haroche2006} argues that the high-frequency terms do not conserve energy as two particles are created or destroyed in one process.
However, such processes are, in general, possible in the nonlinear regime~\cite{QuesadaMejia2015}.
Inserting the definition of the positive and negative frequency operators, we find
\begin{equation}
	\hat{H}_\text{int}(t)
	=
	-
	\int_{\mathbb{R}^3}
	\frac{\dd[3]{p}}{(2\pi)^3\sqrt{2\omega(\vb{p})}}
	\int_{\mathbb{R}^3}
	\frac{\dd[3]{p}}{(2\pi)^3\sqrt{2\omega(\vb{q})}}
	\hat{a}(\vb{p})
	e^{-i\omega(\vb{p})t}
	g(t,\vb{p}-\vb{q})
	\hat{a}^\dagger(\vb{q})
	e^{+i\omega(\vb{q})t}
	+
	\text{h.c.}
\end{equation}
wherein we used the spatial Fourier transform of the coupling
\begin{equation}
	g(t,\vb{p}-\vb{q})
	=
	\int_{\mathbb{R}^3}\dd[3]{x}
	g(t,\vb{x})
	e^{-i(\vb{p}-\vb{q})\vdot\vb{x}}
	.
\end{equation}
Assuming the mode coupler to be linear and passive, we make the ansatz
\begin{equation}
	g(t,\vb{p}-\vb{q})
	=
	(2\pi)
	\delta^{(1)}\left(\omega(\vb{q})-\omega(\vb{p})\right)
	h\left(t,\omega(\vb{p}),\vu{e}_p,\vu{e}_q\right)
\end{equation}
and reduce the interaction Hamiltonian to
\begin{equation}
	\hat{H}_{ab}
	=
	-
	\int_{\mathbb{R}^3}
	\frac{\dd[3]{p}}{(2\pi)^3\sqrt{2\omega(\vb{p})}}
	\int_{\mathbb{R}^3}
	\frac{\dd[3]{p}}{(2\pi)^3\sqrt{2\omega(\vb{q})}}
	\hat{a}(\vb{p})
	g(t,\vb{p}-\vb{q})
	\hat{a}^\dagger(\vb{q})
	+
	\text{h.c.}
	.
\end{equation}
The time-evolution operator is dominated by the first term of the Magnus expansion
\begin{equation}
	\Omega^{(1)}(t,t_0)
	=
	i
	\int_{\mathbb{R}^3}
	\frac{\dd[3]{p}}{(2\pi)^3\sqrt{2\omega(\vb{p})}}
	\int_{\mathbb{R}^3}
	\frac{\dd[3]{q}}{(2\pi)^3\sqrt{2\omega(\vb{p})}}
	\hat{a}(\vb{p})
	\hat{a}^\dagger(\vb{q})
	\int_{t_0}^t
	\dd{t^\prime}
	g(t^\prime,\vb{p}-\vb{q})
	-
	\text{h.c.}
\end{equation}
and the main unitary action of turns out to be
\begin{equation}
	\hat{U}(t,t_0)
	=
	\exp\left\{
		i
		\int_{\mathbb{R}^3}
		\frac{\dd[3]{p}}{(2\pi)^3\sqrt{2\omega(\vb{p})}}
		\hat{a}(\vb{p})
		\int_{\mathbb{R}^3}
		\frac{\dd[3]{q}}{(2\pi)^3\sqrt{2\omega(\vb{q})}}
		\hat{a}^\dagger(\vb{q})
		\int_{t_0}^t
		\dd{t^\prime}
		g(t^\prime,\vb{p}-\vb{q})
		-
		\text{h.c.}
	\right\}
\end{equation}
which can be considered the field-theoretic time-dependent generalization of the beam splitter action reported in Ref.~\cite{Leonhardt2003,Haroche2006}.
% TODO: discuss second term in the Magnus expansion

\subsection{Beam splitter}

For a cubic beam splitter, we expect an unitary action of the form
\begin{equation}
	\hat{U}(t,t_0)
	=
	\exp\left\{
		i\theta
		\int_{\mathbb{R}}
		\frac{\dd{p}}{(2\pi)\sqrt{2p}}
		\int_{\mathbb{R}}
		\frac{\dd{q}}{(2\pi)\sqrt{2q}}
		g(p)
		g(q)
		\hat{a}(p,q,0)
		\hat{a}^\dagger(q,p,0)
		-
		\text{h.c.}
	\right\}
\end{equation}
where $g(p)$ denotes the response function of the coupler and $\theta$ quantifies the coupling strength during the interaction time from $t_0$ to $t$.

\subsection{Optical filter}

\subsection{Phase-shifter}
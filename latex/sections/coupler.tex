\section{Generalized mode coupler}

The mode coupler is a generalization of the beam splitter and is described by the Hamiltonian
\begin{equation}
	\hat{H}
	=
	\hat{H}_0
	+
	\hat{H}_\text{int}(t)
\end{equation}
where we have the Hamiltonian of the scalar Maxwell field
\begin{equation}
	\hat{H}_0
	=
	\int_{\mathbb{R}^3}\frac{\dd[3]{p}}{(2\pi)^3}
	\omega(\vb{p})
	\hat{a}^\dagger(\vb{p})
	\hat{a}(\vb{p})
\end{equation}
and the interaction Hamiltonian
\begin{equation}
	\hat{H}_\text{int}(t)
	=
	-
	\int\dd[3]{x}
	g(t,\vb{x})
	\hat{A}(t,\vb{x})
	\hat{A}(t,\vb{x})
	.
\end{equation}
We expand the fields into their positive and negative frequency parts and only keep the mixed terms known as the rotating wave approximation~\cite[p.~158]{Gardiner2000}
\begin{equation}
	\hat{H}_\text{int}(t)
	=
	-
	\int\dd[3]{x}
	g(t,\vb{x})
	\left[
		\hat{A}^{(-)}(t,\vb{x})
		\hat{A}^{(+)}(t,\vb{x})
		+
		\hat{A}^{(+)}(t,\vb{x})
		\hat{A}^{(-)}(t,\vb{x})
	\right]
	.
\end{equation}
The rotating wave approximation is justified when for long time scales relative to the highly oscillating terms.
Additionally, Haroche~\cite[p.~127]{Haroche2006} argues that the high-frequency terms do not conserve energy as two particles are created or destroyed in one process.
However, such processes are, in general, possible in the nonlinear regime~\cite{QuesadaMejia2015}.
Inserting the definition of the positive and negative frequency operators, we find
\begin{equation}
	\hat{H}_\text{int}(t)
	=
	-
	\int_{\mathbb{R}^3}
	\frac{\dd[3]{p}}{(2\pi)^3\sqrt{2\omega(\vb{p})}}
	\int_{\mathbb{R}^3}
	\frac{\dd[3]{p}}{(2\pi)^3\sqrt{2\omega(\vb{q})}}
	\hat{a}(\vb{p})
	e^{-i\omega(\vb{p})t}
	g(t,\vb{p}-\vb{q})
	\hat{a}^\dagger(\vb{q})
	e^{+i\omega(\vb{q})t}
	+
	\text{h.c.}
\end{equation}
wherein we used the spatial Fourier transform of the coupling
\begin{equation}
	g(t,\vb{p}-\vb{q})
	=
	\int_{\mathbb{R}^3}\dd[3]{x}
	g(t,\vb{x})
	e^{-i(\vb{p}-\vb{q})\vdot\vb{x}}
	.
\end{equation}
Assuming the mode coupler to be linear and passive, we make the ansatz
\begin{equation}
	g(t,\vb{p}-\vb{q})
	=
	(2\pi)
	\delta^{(1)}\left(\omega(\vb{q})-\omega(\vb{p})\right)
	h\left(t,\omega(\vb{p}),\vu{e}_p,\vu{e}_q\right)
\end{equation}
and reduce the interaction Hamiltonian to
\begin{equation}
	\hat{H}_{ab}
	=
	-
	\int_{\mathbb{R}^3}
	\frac{\dd[3]{p}}{(2\pi)^3\sqrt{2\omega(\vb{p})}}
	\int_{\mathbb{R}^3}
	\frac{\dd[3]{p}}{(2\pi)^3\sqrt{2\omega(\vb{q})}}
	\hat{a}(\vb{p})
	g(t,\vb{p}-\vb{q})
	\hat{a}^\dagger(\vb{q})
	+
	\text{h.c.}
	.
\end{equation}
The time-evolution operator is dominated by the first term of the Magnus expansion
\begin{equation}
	\Omega^{(1)}(t,t_0)
	=
	i
	\int_{\mathbb{R}^3}
	\frac{\dd[3]{p}}{(2\pi)^3\sqrt{2\omega(\vb{p})}}
	\int_{\mathbb{R}^3}
	\frac{\dd[3]{q}}{(2\pi)^3\sqrt{2\omega(\vb{p})}}
	\hat{a}(\vb{p})
	\hat{a}^\dagger(\vb{q})
	\int_{t_0}^t
	\dd{t^\prime}
	g(t^\prime,\vb{p}-\vb{q})
	-
	\text{h.c.}
\end{equation}
and the main unitary action of turns out to be
\begin{equation}
	\hat{U}(t,t_0)
	=
	\exp\left\{
		i
		\int_{\mathbb{R}^3}
		\frac{\dd[3]{p}}{(2\pi)^3\sqrt{2\omega(\vb{p})}}
		\hat{a}(\vb{p})
		\int_{\mathbb{R}^3}
		\frac{\dd[3]{q}}{(2\pi)^3\sqrt{2\omega(\vb{q})}}
		\hat{a}^\dagger(\vb{q})
		\int_{t_0}^t
		\dd{t^\prime}
		g(t^\prime,\vb{p}-\vb{q})
		-
		\text{h.c.}
	\right\}
\end{equation}
which can be considered the field-theoretic time-dependent generalization of the beam splitter action reported in Ref.~\cite{Leonhardt2003,Haroche2006}.
% TODO: discuss second term in the Magnus expansion

\subsection{Beam splitter}

For a cubic beam splitter, we expect an unitary action of the form
\begin{equation}
	\hat{U}(t,t_0)
	=
	\exp\left\{
		i\theta
		\int_{\mathbb{R}}
		\frac{\dd{p}}{(2\pi)\sqrt{2p}}
		\int_{\mathbb{R}}
		\frac{\dd{q}}{(2\pi)\sqrt{2q}}
		g(p)
		g(q)
		\hat{a}(p,q,0)
		\hat{a}^\dagger(q,p,0)
		-
		\text{h.c.}
	\right\}
\end{equation}
where $g(p)$ denotes the response function of the coupler and $\theta$ quantifies the coupling strength during the interaction time from $t_0$ to $t$.

\subsection{Optical filter}

\subsection{Phase-shifter}

\subsection{Four-point interaction}

\begin{equation}
	\hat{H}_\text{int}(t)
	=
	-
	\int\dd[3]{x_1}\dd[3]{x_2}\dd[3]{y_1}\dd[3]{y_2}
	g(t,\vb{x}_1,\vb{x}_2,\vb{y}_1,\vb{y}_2)
	\hat{A}(t,\vb{x}_1)
	\hat{A}(t,\vb{x}_2)
	\hat{A}(t,\vb{y}_1)
	\hat{A}(t,\vb{y}_2)
\end{equation}
Expanding into positive and negative frequency parts while performing the rotating wave approximation yields
\begin{equation}
	\begin{split}
		\hat{H}_\text{int}(t)
		=
		-
		\int &
		\dd[3]{x_1}\dd[3]{x_2}\dd[3]{y_1}\dd[3]{y_2}
		g(t,\vb{x}_1,\vb{x}_2,\vb{y}_1,\vb{y}_2)
		\\
		\biggl\{
			&
			\hat{A}^{(-)}(t,\vb{x}_1)
			\hat{A}^{(+)}(t,\vb{x}_2)
			\hat{A}^{(-)}(t,\vb{y}_1)
			\hat{A}^{(+)}(t,\vb{y}_2)
			\\
			+&
			\hat{A}^{(-)}(t,\vb{x}_1)
			\hat{A}^{(+)}(t,\vb{x}_2)
			\hat{A}^{(+)}(t,\vb{y}_1)
			\hat{A}^{(-)}(t,\vb{y}_2)
			\\
			+&
			\hat{A}^{(+)}(t,\vb{x}_1)
			\hat{A}^{(-)}(t,\vb{x}_2)
			\hat{A}^{(-)}(t,\vb{y}_1)
			\hat{A}^{(+)}(t,\vb{y}_2)
			\\
			+&
			\hat{A}^{(+)}(t,\vb{x}_1)
			\hat{A}^{(-)}(t,\vb{x}_2)
			\hat{A}^{(+)}(t,\vb{y}_1)
			\hat{A}^{(-)}(t,\vb{y}_2)
		\biggr\}
	\end{split}
\end{equation}
We insert the definition of the positive and negative frequency operators in terms of annihilation and creation operators.
For the first term this yields
\begin{equation}
	\begin{split}
		&
		\int\dd[3]{x_1}\dd[3]{x_2}\dd[3]{y_1}\dd[3]{y_2}
		g(t,\vb{x}_1,\vb{x}_2,\vb{y}_1,\vb{y}_2)
		\hat{A}^{(-)}(t,\vb{x}_1)
		\hat{A}^{(+)}(t,\vb{x}_2)
		\hat{A}^{(-)}(t,\vb{y}_1)
		\hat{A}^{(+)}(t,\vb{y}_2)
		\\
		=&
		\int
		\frac{\dd[3]{p_1}}{(2\pi)^3\sqrt{2\omega(\vb{p}_1)}}
		\dots
		\frac{\dd[3]{q_2}}{(2\pi)^3\sqrt{2\omega(\vb{q}_2)}}
		g(t,-\vb{p}_1,+\vb{p}_2,-\vb{q}_1,+\vb{q}_2)
		\hat{a}(\vb{p}_1)
		\hat{a}^\dagger(\vb{p}_2)
		\hat{a}(\vb{q}_1)
		\hat{a}^\dagger(\vb{q}_2)
		\\
		\times&
		e^{-i\omega(\vb{p}_1)t}
		e^{+i\omega(\vb{p}_2)t}
		e^{-i\omega(\vb{q}_1)t}
		e^{+i\omega(\vb{q}_2)t}
	\end{split}
\end{equation}
and for the second term we find
\begin{equation}
	\begin{split}
		&
		\int\dd[3]{x_1}\dd[3]{x_2}\dd[3]{y_1}\dd[3]{y_2}
		g(t,\vb{x}_1,\vb{x}_2,\vb{y}_1,\vb{y}_2)
		\hat{A}^{(-)}(t,\vb{x}_1)
		\hat{A}^{(+)}(t,\vb{x}_2)
		\hat{A}^{(+)}(t,\vb{y}_1)
		\hat{A}^{(-)}(t,\vb{y}_2)
		\\
		=&
		\int
		\frac{\dd[3]{p_1}}{(2\pi)^3\sqrt{2\omega(\vb{p}_1)}}
		\dots
		\frac{\dd[3]{q_2}}{(2\pi)^3\sqrt{2\omega(\vb{q}_2)}}
		g(t,-\vb{p}_1,+\vb{p}_2,+\vb{q}_1,-\vb{q}_2)
		\hat{a}(\vb{p}_1)
		\hat{a}^\dagger(\vb{p}_2)
		\hat{a}^\dagger(\vb{q}_1)
		\hat{a}(\vb{q}_2)
		\\
		\times&
		e^{-i\omega(\vb{p}_1)t}
		e^{+i\omega(\vb{p}_2)t}
		e^{+i\omega(\vb{q}_1)t}
		e^{-i\omega(\vb{q}_2)t}
	\end{split}
\end{equation}
wherein
\begin{equation}
	g(t,\vb{p}_1,\vb{p}_2,\vb{q}_1,\vb{q}_2)
	=
	\int\dd[3]{x_1}\dd[3]{x_2}\dd[3]{y_1}\dd[3]{y_2}
	g(t,\vb{x}_1,\vb{x}_2,\vb{y}_1,\vb{y}_2)
	e^{+i\vb{p}_1\vdot\vb{x}_1}
	e^{+i\vb{p}_2\vdot\vb{x}_2}
	e^{+i\vb{q}_1\vdot\vb{y}_1}
	e^{+i\vb{q}_2\vdot\vb{y}_2}
\end{equation}
Now, we need to insert add some assumptions. For one, we expect the total momentum to be conserved, i.e.,
\begin{equation}
	g(t,\vb{p}_1,\vb{p}_2,\vb{q}_1,\vb{q}_2)
	\propto
	(2\pi)^3
	\delta^{(3)}(\vb{p}_1+\vb{p}_2-\vb{q}_1-\vb{q}_2)
\end{equation}
and the first term reduces to
\begin{equation}
	\begin{split}
		\int
		&
		\frac{\dd[3]{p_1}}{(2\pi)^3\sqrt{2\omega(\vb{p}_1)}}
		\frac{\dd[3]{p_2}}{(2\pi)^3\sqrt{2\omega(\vb{p}_2)}}
		\frac{\dd[3]{q}}{(2\pi)^3\sqrt{2\omega(\vb{q})}}
		\frac{1}{\sqrt{2\omega(\vb{p}_1+\vb{p}_2-\vb{q})}}
		g(t,-\vb{p}_1,+\vb{p}_2,-\vb{q}_1)
		\\
		&\times
		\hat{a}(\vb{p}_1)
		\hat{a}^\dagger(\vb{p}_2)
		\hat{a}(\vb{q})
		\hat{a}^\dagger(\vb{p}_1+\vb{p}_2-\vb{q})
		e^{-i\omega(\vb{p}_1)t}
		e^{+i\omega(\vb{p}_2)t}
		e^{-i\omega(\vb{q}_1)t}
		e^{+i\omega(\vb{p}_1+\vb{p}_2-\vb{q})t}
	\end{split}
\end{equation}
in addition, we know that the coupling is linear in the energy
\begin{equation}
	\begin{split}
		\int
		&
		\frac{\dd[3]{p_1}}{(2\pi)^3}
		\frac{\dd[3]{p_2}}{(2\pi)^3}
		\frac{\dd[3]{q}}{(2\pi)^3}
		\left(
			\frac{1}{2\omega(\vb{p}_1)}
		\right)^2
		g(t,-\vb{p}_1,+\vb{p}_2,-\vb{q}_1)
		\hat{a}(\vb{p}_1)
		\hat{a}^\dagger(\vb{p}_2)
		\hat{a}(\vb{q})
		\hat{a}^\dagger(\vb{p}_1+\vb{p}_2-\vb{q})
	\end{split}
\end{equation}
Furthermore, assuming the incoming and outgoing momenta being orthogonal $\vb{q}_1\parallel\vb{p}_1\perp\vb{p}_2 \parallel\vb{q}_2$, we reduce the problem to one dimension
\begin{equation}
	\begin{split}
		\int
		&
		\frac{\dd{p_1}}{(2\pi)}
		\frac{\dd{p_2}}{(2\pi)}
		\frac{\dd{q}}{(2\pi)}
		\left(
			\frac{1}{2p_1}
		\right)^2
		g(t,-p_1,+p_2,-q_1)
		\hat{a}(p_1,0,0)
		\hat{a}^\dagger(0,p_2,0)
		\hat{a}(q,0,0)
		\hat{a}^\dagger(0,p_1-p_2-q,0)
	\end{split}
\end{equation}
but still different momenta couple, we therefore reduce the problem further to
\begin{equation}
	\begin{split}
		\int
		&
		\frac{\dd{p}}{(2\pi)}
		\left(
			\frac{1}{2p}
		\right)^2
		g(t,p)
		\hat{a}(p,0,0)
		\hat{a}^\dagger(0,p,0)
		\hat{a}(p,0,0)
		\hat{a}^\dagger(0,p,0)
	\end{split}
\end{equation}
but this would be a two-quantum coupler.
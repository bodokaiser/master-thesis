\section{Quantization of the Maxwell field in the Coulomb gauge}

% TODO: motivate approach and outcome of the section
% TODO: introduce four-potential (Maxwell field) from vector and scalar potential

\subsection{Maxwell Lagrangian and relativistic wave equation}

% TODO: motivate the Maxwell Lagrangian from fundamental principles
The Lagrangian of the Maxwell field $A^\mu(t,\vb{x})$~\cite[p.~339]{Srednicki2007}
\begin{equation}
	\mathcal{L}
	=
	\frac{1}{2}
	(\partial_\mu A_\nu)
	\left(
		\partial^\nu A^\mu
		-
		\partial^\mu A^\nu
	\right)
\end{equation}
describes a manifest Lorentz-covariant massless vector field.
From the covariant generalization of the Euler-Lagrange equations
\begin{equation}
	0
	=
	\partial_\mu
	\pdv{\mathcal{L}}{(\partial_\mu A_\nu)}
	-
	\pdv{\mathcal{L}}{A_\nu}
	=
	\partial_\mu\partial^\mu A^\nu
	-
	\partial^\nu\partial_\mu A^\mu
\end{equation}
we find the free equations of motion which govern the dynamics of the free Maxwell field.
Ignoring static charges $A_0=0$ and employing the Coulomb gauge $\partial_iA^i=0$ in which the Maxwell field is transverse, the free equations of motion reduce to
\begin{equation}
	0
	=
	\partial_\mu\partial^\mu
	\vb{A}
	=
	\left(
		\partial_t^2
		-
		\laplacian
	\right)
	\vb{A}
\end{equation}
which is the relativistic wave equation of a transverse vector field.
% TODO: explain what is meant with transverse (maybe figure)?

\subsection{Relativistic energy-momentum relation and plane-wave expansion}

The momentum space representation of the transverse Maxwell field $\vb{A}$ is
\begin{equation}
	\vb{A}(t,\vb{x})
	=
	\int_{\mathbb{R}^3}
	\frac{\dd[3]{p}}{(2\pi)^3}
	\vb{A}(t,\vb{p})
	e^{+i\vb{p}\vdot\vb{x}}
	=
	\int_{\mathbb{R}^4}
	\frac{\dd[4]{p}}{(2\pi)^4}
	\vb{A}(p_0,\vb{p})
	e^{-ip_\mu x^\mu}
\end{equation}
where we introduced the four-position $x^\mu=(t,\vb{x})$ and -momentum $p^\mu=(p_0,\vb{p})$ with inner Minkowski product $p_\mu x^\mu=p_0t-\vb{x}\vdot\vb{p}$.
Construct a transverse and orthonormal polarization basis $\left\{\vu{e}_1(\vb{p}),\vu{e}_2(\vb{p})\right\}$
\begin{align}
	\vb{p}\vdot\vu{e}_\lambda(\vb{p})
	=
	0
	&&
	\vu{e}_\lambda(\vb{p})
	\vdot
	\vu{e}_{\lambda^\prime}(\vb{p})
	=
	\delta_{\lambda,\lambda^\prime}
\end{align}
which is complete with respect to the transverse projector $P_\perp(\vb{p})$, see Ref.~\cite[p.~341]{Srednicki2007}
\begin{equation}
	\sum_{\lambda=1,2}
	\vu{e}_\lambda^i(\vb{p})
	\vu{e}_{\lambda^\prime}^j(\vb{p})
	=
	\delta^{ij}
	-
	\frac{p^ip^j}{\vb{p}^2}
	=
	P_\perp^{ij}(\vb{p})
\end{equation}
we find the plane-wave expansion of the transverse Maxwell field to be~\cite[p.~154]{Greiner2013}
\begin{equation}
	\vb{A}(t,\vb{x})
	=
	\sum_{\lambda=1,2}
	\int_{\mathbb{R}^4}\frac{\dd[4]{p}}{(2\pi)^4}
	A_\lambda(p_0,\vb{p})
	e^{+ip_\mu x^\mu}
	\vu{e}_\lambda(\vb{p})
	.
\end{equation}
Inserting the mode decomposition into the relativistic wave equation, we find that the modes need to satisfy $0=p_0^2-\vb{p}^2$ implying the relativistic energy-momentum relation for massless particles
\begin{equation}
	E(\vb{p})
	=
	\omega(\vb{p})
	=
	\vb{p}
	.
\end{equation}
Amending the integration domain of the mode decomposition to
\begin{equation}
	\left\{
		(p_0,\vb{p})\in\mathbb{R}^4
		\colon
		p_0^2
		=
		\omega(\vb{p})^2
	\right\}
\end{equation}
or equivalent, adding a factor
\begin{equation}
	(2\pi)
	\delta^{(1)}\left(p_0^2-\omega(\vb{p})^2\right)
	=
	(2\pi)
	\frac{
		\delta^{(1)}\left(p_0-\omega(\vb{p})\right)
		-
		\delta^{(1)}\left(p_0+\omega(\vb{p})\right)
	}{\sqrt{2\omega(\vb{p})}}
\end{equation}
to the integrand, we ensure the plane-wave expansion to satisfy equation of motion.
Finally, the plane-wave expansion of the transverse Maxwell field is brought into manifest Lorentz-covariant form
\begin{equation}
	\begin{split}
		\vb{A}(t,\vb{x})
		&=
		\sum_{\lambda=1,2}
		\int_{\mathbb{R}^3}\frac{\dd[3]{p}}{(2\pi)^3\sqrt{2\omega(\vb{p})}}
		\biggl\{
			a_\lambda(\vb{p})
			\vu{e}_\lambda(\vb{p})
			e^{-ip_0t+i\vb{p}\vdot\vb{x}}
			+
			a_\lambda(\vb{p})
			\vu{e}_\lambda(\vb{p})
			e^{+ip_0t+i\vb{p}\vdot\vb{x}}
		\biggr\}_{p_0=\omega(\vb{p})}
		\\
		&=
		\sum_{\lambda=1,2}
		\int_{\mathbb{R}^3}\frac{\dd[3]{p}}{(2\pi)^3\sqrt{2\omega(\vb{p})}}
		\biggl\{
			a_\lambda(\vb{p})
			\vu{e}_\lambda(\vb{p})
			e^{-ip_\mu x^\mu}
			+
			a_\lambda(\vb{p})^*
			\vu{e}_\lambda(\vb{p})^*
			e^{+ip_\mu x^\mu}
		\biggr\}_{p_0=\omega(\vb{p})}
	\end{split}
\end{equation}
where we evaluated the Dirac distributions in the first, and performed the substitution $\vb{p}\to-\vb{p}$ in the second line, and we defined the plane-wave amplitude
\begin{equation}
	a_\lambda(\vb{p})
	=
	A_\lambda\left(\omega(\vb{p}),\vb{p}\right)
\end{equation}
which together with the plane-wave basis vectors has conjugate symmetry
\begin{equation}
	a_\lambda(-\vb{p})
	=
	a_\lambda(\vb{p})^*	
\end{equation}
because the Maxwell field is real-valued.

\subsection{Electromagnetic field components and Hamiltonian density}

% TODO: cite definition of field-strength tensor and components
It is convenient to introduce the antisymmetric and manifest Lorentz-covariant field-strength tensor
\begin{equation}
	F^{\mu\nu}
	=
	\partial^\mu A^\nu
	-
	\partial^\nu A^\mu
\end{equation}
which encodes the electromagnetic field components via
\begin{align}
	F^{0i}
	=
	E^i
	&&
	F^{ij}
	=
	\varepsilon^{ijk}
	B_k
	.
\end{align}
Solving for the electromagnetic field components and inserting the plane-wave expansion of the transverse Maxwell field, we find the plane-wave expansion of the electromagnetic field components to be
\begin{align}
	\vb{E}(t,\vb{x})
	&=
	\sum_{\lambda=1,2}
	\int_{\mathbb{R}^3}
	\frac{\dd[3]{p}}{(2\pi)^3\sqrt{2\omega(\vb{p})}}
	\left(-i\omega(\vb{p})\right)
	\biggl\{
		a_\lambda(\vb{p})
		\vu{e}_\lambda(\vb{p})
		e^{-ip_\mu x^\mu}
		-
		\text{c.c.}
	\biggr\}_{p_0=\omega(\vb{p})}
	\\
	\vb{B}(t,\vb{x})
	&=
	\sum_{\lambda=1,2}
	\int_{\mathbb{R}^3}
	\frac{\dd[3]{p}}{(2\pi)^3\sqrt{2\omega(\vb{p})}}
	\left(+i\vb{p}\times\right)
	\biggl\{
		a_\lambda(\vb{p})
		\vu{e}_\lambda(\vb{p})
		e^{-ip_\mu x^\mu}
		-
		\text{c.c.}
	\biggr\}_{p_0=\omega(\vb{p})}
\end{align}
The Maxwell Lagrangian takes a simple form in terms of the field-strength tensor and electromagnetic components 
\begin{equation}
	\mathcal{L}
	=
	-\frac{1}{4}
	F^{\mu\nu}
	F_{\mu\nu}
	=
	\frac{1}{2}
	\left(
		\vb{E}^2
		-
		\vb{B}^2
	\right)
	.
\end{equation}
The canonical momentum density is
\begin{equation}
	\boldsymbol{\pi}(t,\vb{x})
	=
	\partial_t
	\pdv{\mathcal{L}}{(\partial_t\vb{A})}
	=
	-\vb{E}(t,\vb{x})
\end{equation}
and we can perform the Legendre transform of the Lagrangian density
\begin{equation}
	\mathcal{H}
	=
	\frac{1}{2}
	\boldsymbol{\pi}^2
	-
	\mathcal{L}
	=
	\frac{1}{2}
	\left(
		\vb{E}^2
		+
		\vb{B}^2
	\right)
\end{equation}
to obtain the Hamiltonian density.
Finally, the total energy and Hamiltonian is
\begin{equation}
	H
	=
	\sum_{\lambda=1,2}
	\int\frac{\dd[3]{p}}{(2\pi)^3}
	\omega(\vb{p})
	\abs{a_\lambda(\vb{p})}^2
	.
\end{equation}

\subsection{Canonical quantization in the Coulomb gauge}

To preserve the transversality of the fields, we cannot impose the usual canonical commutation relations in the Coulomb gauge but must use the the transverse delta distribution, see Ref.~\cite[p.~197]{Greiner2013},
\begin{equation}
	\delta_{\perp,i,j}^{(3)}(\vb{x}-\vb{y})
	=
	P^{ij}_\perp
	\delta^{(3)}(\vb{x}-\vb{y})
	=
	\int_{\mathbb{R}^3}
	\frac{\dd[3]{p}}{(2\pi)^3}
	\left(
		\delta_{ij}
		-
		\frac{p_ip_j}{\vb{p}^2}
	\right)
	e^{i\vb{p}\vdot(\vb{x}-\vb{y})}
\end{equation}
and the non-zero commutation relation between the conjugate field variables reads
\begin{equation}
	\comm{\hat{A}_i(t,\vb{x})}{\hat{E}_j(t,\vb{y})}
	=
	-i
	\delta_{\perp,ij}^{(3)}(\vb{x}-\vb{y})
	=
	\comm{\hat{A}_i(t,\vb{x})}{-\hat\pi_j(t,\vb{y})}
	.
\end{equation}
Promoting the dynamical field variables in the plane-wave expansion to operators
\begin{align}
	\hat{\vb{A}}(t,\vb{x})
	&=
	\sum_{\lambda=1,2}
	\int_{\mathbb{R}^3}
	\frac{\dd[3]{p}}{(2\pi)^3}
	\frac{1}{\sqrt{2\omega(\vb{p})}}
	\biggl\{
		\hat{a}_{\lambda}(\vb{p})
		e^{-ip_\mu x^\mu}
		\vu{e}_\lambda(\vb{p})
		+
		\text{h.c.}
	\biggr\}_{p_0=\omega(\vb{p})}
	\\
	\hat{\vb{E}}(t,\vb{x})
	&=
	\sum_{\lambda=1,2}
	\int_{\mathbb{R}^3}
	\frac{\dd[3]{p}}{(2\pi)^3}
	\left(
		-i
		\sqrt{\frac{\omega(\vb{p})}{2}}
	\right)
	\biggl\{
		\hat{a}_{\lambda}(\vb{p})
		e^{-ip_\mu x^\mu}
		\vu{e}_\lambda(\vb{p})
		-
		\text{h.c.}
	\biggr\}_{p_0=\omega(\vb{p})}
\end{align}
and inserting these operators into the canonical commutation relations, we deduce
\begin{align}
	\comm{\hat{a}_\lambda(\vb{p})}{\hat{a}_{\lambda^\prime}^\dagger(\vb{q})}
	&=
	(2\pi)^3
	\delta^{(3)}(\vb{q}-\vb{p})
	\delta_{\lambda\lambda^\prime}
	\\
	\comm{\hat{a}_\lambda^\dagger(\vb{p})}{\hat{a}_{\lambda^\prime}^\dagger(\vb{q})}
	&=
	\comm{\hat{a}_\lambda(\vb{p})}{\hat{a}_{\lambda^\prime}(\vb{q})}
	=
	0
	.
\end{align}
We define the Hamiltonian and number operator in normal-order, i.e.,
\begin{align}
	\hat{H}
	=
	\sum_{\lambda=1,2}
	\int\frac{\dd[3]{p}}{(2\pi)^3}
	\omega(\vb{p})
	\hat{a}_\lambda^\dagger(\vb{p})
	\hat{a}_\lambda(\vb{p})
	&&
	\hat{N}
	=
	\sum_{\lambda=1,2}
	\int\frac{\dd[3]{p}}{(2\pi)^3}
	\hat{a}_\lambda^\dagger(\vb{p})
	\hat{a}_\lambda(\vb{p})
	.
\end{align}
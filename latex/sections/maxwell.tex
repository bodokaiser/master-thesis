\section{Quantization of the Maxwell field in the Coulomb gauge}

\subsection{Relativistic field theory}

The Lagrangian of the Maxwell field $A^\mu(t,\vb{x})$ reads~\cite[p.~339]{Srednicki2007}
\begin{equation}
	\mathcal{L}
	=
	\frac{1}{2}
	(\partial_\mu A_\nu)
	\left(
		\partial^\nu A^\mu
		-
		\partial^\mu A^\nu
	\right)
\end{equation}
and the covariant generalization of the Euler-Lagrange equations
\begin{equation}
	0
	=
	\partial_\mu
	\pdv{\mathcal{L}}{(\partial_\mu A_\nu)}
	-
	\pdv{\mathcal{L}}{A_\nu}
	=
	\partial_\mu\partial^\mu A^\nu
	-
	\partial^\nu\partial_\mu A^\mu
\end{equation}
leads to the free equations of motion.
We ignore static charges $A_0=0$ and employ the Coulomb gauge $\partial_iA^i=0$ in which the Maxwell field is transverse.
The equations of motion simplify to relativistic wave equation
\begin{equation}
	0
	=
	\partial_\mu\partial^\mu\vb{A}
	=
	\partial_t^2\vb{A}
	-
	\laplacian\vb{A}
	.
\end{equation}

\subsection{Mode decomposition}

In momentum space the transverse field $\vb{A}$ reads
\begin{equation}
	\vb{A}(t,\vb{x})
	=
	\int_{\mathbb{R}^4}\frac{\dd[4]{p}}{(2\pi)^4}
	\vb{A}(p_0,\vb{p})
	e^{ip_0t-i\vb{p}\vdot\vb{x}}
	.
\end{equation}
In momentum space, we can construct a polarization basis $\vu{e}_1(\vb{p}),\vu{e}_2(\vb{p})$ being transverse
\begin{equation}
	\vb{p}\vdot\vu{e}_\lambda(\vb{p})
	=
	0
	,
\end{equation}
orthonormal
\begin{equation}
	\vu{e}_\lambda(\vb{p})
	\vdot
	\vu{e}_{\lambda^\prime}(\vb{p})
	=
	\delta_{\lambda,\lambda^\prime}
\end{equation}
and complete
\begin{equation}
	\sum_{\lambda=1,2}
	\vu{e}_\lambda^i(\vb{p})
	\vu{e}_{\lambda^\prime}^j(\vb{p})
	=
	\delta^{ij}
	-
	\frac{p^ip^j}{\vb{p}^2}
	=
	P_\perp^{ij}(\vb{p})
\end{equation}
with $P_\perp(\vb{p})$ being the transverse projector.
Expressing $\vb{A}(p_0,\vb{p})$ in the polarization basis, we find
\begin{equation}
	\vb{A}(p_0,\vb{p})
	=
	\sum_{\lambda=1,2}
	A_\lambda(p_0,\vb{p})
	\vu{e}_\lambda(\vb{p})
\end{equation}
and the mode decomposition reads
\begin{equation}
	\vb{A}(t,\vb{x})
	=
	\sum_{\lambda=1,2}
	\int_{\mathbb{R}^4}\frac{\dd[4]{p}}{(2\pi)^4}
	A_\lambda(p_0,\vb{p})
	e^{ip_0t-i\vb{p}\vdot\vb{x}}
	\vu{e}_\lambda(\vb{p})
	.
\end{equation}
Inserting the mode decomposition into the relativistic wave equation, we recover the relativistic energy-momentum relation for massless particles
\begin{equation}
	E(\vb{p})
	=
	\omega(\vb{p})
	=
	\vb{p}
	.
\end{equation}
Hence, if the Fourier modes $a_\lambda(p_0,\vb{p})$ satisfy the relativistic energy-momentum relation, $\vb{A}(t,\vb{x})$ satisfies the relativistic wave equation.
We enforce the mode decomposition to satisfy the energy-momentum relation by constraining the integration domain to
\begin{equation}
	V_p
	=
	\left\{
		(p_0,\vb{p})\in\mathbb{R}^4
		\colon
		p_0^2
		=
		\omega(\vb{p})^2
	\right\}
\end{equation}
or equivalent, adding a factor
\begin{equation}
	(2\pi)
	\delta^{(1)}\left(p_0^2-\omega(\vb{p})^2\right)
	=
	(2\pi)
	\frac{
		\delta^{(1)}\left(p_0-\omega(\vb{p})\right)
		-
		\delta^{(1)}\left(p_0+\omega(\vb{p})\right)
	}{\sqrt{2\omega(\vb{p})}}
\end{equation}
to the integrand.
Finally, we arrive at
\begin{equation}
	\vb{A}(t,\vb{x})
	=
	\sum_{\lambda=1,2}
	\int_{\mathbb{R}^3}\frac{\dd[3]{p}}{(2\pi)^3\sqrt{2\omega(\vb{p})}}
	\left\{
		a_\lambda(\vb{p})
		e^{i\omega(\vb{p})t-i\vb{p}\vdot\vb{x}}
		\vu{e}_\lambda(\vb{p})
		+
		\text{c.c.}
	\right\}
\end{equation}
where we defined
\begin{equation}
	a_\lambda(\vb{p})
	=
	A_\lambda\left(\omega(\vb{p}),\vb{p}\right)
\end{equation}
and used the conjugate symmetry of the Fourier amplitudes $a_\lambda(-\vb{p})=a_\lambda(\vb{p})^*$.

\subsection{Canonical quantization}
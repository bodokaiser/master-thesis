The present chapter describes a coherent state communication system from a signal-processing point of view.
We discuss concepts common to signal-processing, extend the concepts to quantum signals whenever necessary, and give references to the implementation details of the previous chapter.
In the next chapter, the coherent state communication system is extended to a \gls{cvqkd} communication system by choosing a random sequence of complex symbols, adding classical post-processing, and discussing security aspects.

\begin{figure}[htb]
	\centering
	\includestandalone[mode=buildnew]{figures/tikz/communication-system}
	\caption{Flow diagram of an abstract communication system delivering information encoded in a sequence of symbols from the transmitter to the receiver over a physical channel.}\label{fig:communication_system}
\end{figure}
\Cref{fig:communication_system} depicts the flow diagram of an abstract communication system.
A sequence of complex symbols $\alpha_1,\alpha_2,\dots,\alpha_n\in\mathbb{C}$ is  streamed to a transmitter.
The transmitter converts the symbol stream to an analog signal $\alpha(t)$ which is modulated onto an optical carrier, i.e., $\alpha(t)e^{-i\omega_0t}$, which is supplied to an optical fiber, the channel.
The signal leaving the channel and entering the receiver $\alpha^\prime(t)e^{-i\omega_0t}$ is related to the signal entering the channel $\alpha(t)e^{-i\omega_0t}$ by attenuation, dispersion, and noise.
The receiver removes the optical carrier and recovers the decoded symbol sequence from $\alpha^\prime(t)$.

\section{Transmitter}

\begin{figure}[htb]
	\centering
	\includestandalone[mode=buildnew]{figures/tikz/transmitter}
	\caption{Flow diagram of the transmitter converting a sequence of complex symbols to a coherent state signal.}\label{fig:transmitter}
\end{figure}

\subsection{Reconstruction}

\begin{figure}[htb]
	\centering
	\includestandalone[mode=buildnew]{figures/tikz/reconstruction}
	\caption{Flow diagram of the reconstruction converting a sequence of samples $x_1,\dots,x_n$ to a continuous-time signal $x(t)$}\label{fig:transmitter}
\end{figure}

% TODO: analog <-> digital (mathematical)
% TODO: samples -> upsampling -> pulse-shaping

Let $\alpha_1,\alpha_2,\dots,\alpha_n\in\mathbb{C}$ be a sequence of complex symbols.
We define the real symbols
\begin{align}
	x_1=\Re{\alpha_1},
	\dots,
	x_n=\Re{\alpha_n}
	\\
	p_1=\Im{\alpha_1},
	\dots,
	p_n=\Im{\alpha_n}
\end{align}
through the real and imaginary part.

\subsection{Up-conversion}

Let $x(t),p(t)$ be two real-valued baseband signals, then their complex baseband representation reads
\begin{equation}
	\alpha(t)
	=
	x(t)
	+
	ip(t)
	.
\end{equation}
While any physical signal is real-valued, the real-part of the complex baseband representation $\alpha(t)$ multiplied by $e^{-i\omega_0t}$ contains both $x(t),p(t)$~\cite[p.~25]{Madhow2008}
\begin{equation}
	\Re\left\{
		\alpha(t)
		e^{-i\omega_0t}
	\right\}
	=
	x(t)
	\cos(\omega_0t)
	+
	p(t)
	\sin(\omega_0t)
	,
\end{equation}
and one refers to $x(t)$ as the in-phase and $p(t)$ as the quadrature component of $\alpha(t)$.
\begin{figure}[htb]
	\centering
	\includestandalone[mode=buildnew]{figures/tikz/iq-modulator}
	\caption{Flow diagram of an in-phase and quadrature modulator.}
\end{figure}
\begin{figure}[htb]
	\centering
	\includestandalone[mode=buildnew]{figures/tikz/up-converter}
	\caption{Flow diagram of a complex up-converter.}
\end{figure}

\subsection{Attenuation}

\section{Channel}

% TODO: physical effects of a channel
% TODO: original signal, signal with noise, signal with noise and attenuation
% TODO: signal-to-noise ratios (?)
% TODO: quantum analog of noise (?)

\begin{figure}[htb]
	\centering
	\includestandalone[mode=buildnew]{figures/tikz/channel}
	\caption{Flow diagram of a communication system.}
\end{figure}
\begin{equation}
	y(t)
	=
	h(t)x(t)
	+
	n(t)
\end{equation}

\section{Receiver}

% TODO: spectrum diagram for down-conversion

\subsection{Down-conversion}

\begin{figure}[htb]
	\centering
	\includestandalone[mode=buildnew]{figures/tikz/iq-demodulator}
	\caption{Flow diagram of an in-phase and quadrature demodulator.}
\end{figure}
\begin{figure}[htb]
	\centering
	\includestandalone[mode=buildnew]{figures/tikz/down-converter}
	\caption{Flow diagram of a complex down-converter.}
\end{figure}
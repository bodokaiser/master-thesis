\chapter*{Introduction}
\addcontentsline{toc}{chapter}{Introduction}

Optical communication enables humanity worldwide to share information in a split second, with companies like Huawei undergoing tremendous efforts to advance the frontiers.
In addition to incremental innovation increasing the performance and decreasing the cost of optical communication technology, we observe intensified activities towards disruptive innovations that challenge our present understanding of communication.
One such branch of activity is quantum-optical communication, incorporating quantum aspects of light into classical communication and leading to novel communication technology like \gls{qkd}, which enables practical and secure key generation.
As a still young discipline, which emerged from two highly advanced fields, communication engineering and quantum physics, quantum communication lacks a unified description to which both communication engineers and quantum physicists agree.
The present thesis aims to resolve the seeming discrepancies between communication engineering and quantum physics by reviewing a practical implementation of a quantum communication system implementing a \gls{cvqkd} protocol.
In the process, we intend to develop a theoretical framework for quantum-optical communication, incorporating quantum effects into classical communication, which has applicability beyond \gls{cvqkd}.

\subsection*{Problem statement}

To raise awareness of the challenges ahead, we review the best-known quantum theory of light, single-mode quantum optics, along with central ideas from classical communication and outline where these pictures conflict.

In single-mode quantum optics, we model monochromatic light with frequency $\omega_0$ as a quantum harmonic oscillator with unit mass, $m=1$, and Hamiltonian~\cite{Gerry2005,Fox2006}
\begin{equation}
	\hat{H}
	=
	\omega_0
	\hat{a}^\dagger
	\hat{a}
	,
\end{equation}
wherein $\hat{a}$ and $\hat{a}^\dagger$ are the quantum annihilation and creation operators, destroying or creating an excitation or "mode" of frequency $\omega_0$.
The electric field operator,
\begin{equation}
	\hat{E}(t,x)
	=
	\mathcal{E}_0
	\left(
		\hat{a}
		+
		\hat{a}^\dagger
	\right)
	\sin(\omega_0x)
	,
\end{equation}
wherein $\mathcal{E}_0$ has the interpretation of an electric field density, establishes the connection between the quantum harmonic oscillator and electromagnetic radiation, including light~\cite[p.~12]{Gerry2005}.
Two of the most important quantum states are the number and the coherent state,
\begin{align}
	\ket{n}
	&=
	\frac{1}{\sqrt{n!}}
	\left(\hat{a}^\dagger\right)^n
	\ket{0}
	&
	&\text{and}
	&
	\ket{\alpha}
	&=
	\exp\left(-\frac{1}{2}\abs{\alpha}^2\right)
	\sum_{n=0}^\infty
	\frac{\alpha^n}{\sqrt{n!}}
	\ket{n}
	.
\end{align}
The number state is parametrized by a natural number $n\in\mathbb{N}_0$ counting the number of excitations.
The coherent state is parametrized by a complex number $\alpha\in\mathbb{C}$ encoding amplitude and phase.
The expectation value of the electric field operator with respect to a coherent state,
\begin{equation}
	\bra{\alpha}
	\hat{E}(t)
	\ket{\alpha}
	=
	\sqrt{2}
	\abs{\alpha}
	\mathcal{E}_0
	\sin\left(\omega_0t-\theta\right)
	,
\end{equation}
equals a classical monochromatic wave with amplitude proportional to $\abs{\alpha}$ and phase $\theta$~\cite[p.~45]{Gerry2005}.

The central concept in communication engineering is that of a signal, not specific to a particular physical realization, e.g., mechanical or electromagnetic waves.
In that sense, light is primarily an implementation detail for carrying information-bearing signals.
In the following, we assume the information-bearing signals to be narrowband, i.e., have a well-defined narrow bandwidth $B_d$ in the power spectrum.\footnote{In general, there is no requirement for the information-bearing signal to be narrowband, see, for instance, \gls{ofdm} and spread-spectrum techniques.}
Complementary, the carrying signal, ideally, only comprises a single well-defined frequency component, tailored to the physical transmission channel, which typically is many magnitudes higher than the frequency components of the information-bearing signal.
Superimposing carrier signals with frequencies sufficiently spaced apart allows multiplexing of information-bearing signals on a common transmission medium.
\begin{figure}[ht]
	\centering
	\includegraphics{figures/tikz/spectrum-upconversion}
	\caption{Power spectrum comprising an information-bearing signal with bandwidth $B_s$ at zero frequency, $\omega=0$, a carrier signal at a carrier frequency much greater than the bandwidth, $\omega_c\gg B$. The upconverted information-bearing spectrum is indicated by dashed lines.}\label{fig:spectrum_upconversion}
\end{figure}
Affiliating an information-bearing signal with a carrier is implemented by modulating the information-bearing signal onto the carrier signal.
On a more abstract level, the power spectrum of the information-bearing signal is shifted by the carrier frequency, as illustrated in \Cref{fig:spectrum_upconversion}.
The asymmetry of the information-bearing spectrum around zero frequency, $\omega=0$, in \Cref{fig:spectrum_upconversion} implies that the information-bearing signal is complex-valued.
At the same time, the spectrum of the information-bearing signal modulated onto the carrier, denoted by the dashed spectrum around $\omega_c$, has complex conjugate symmetry, i.e., is thus real-valued, as we would expect from a physical signal.
\begin{figure}[ht]
	\centering
	\includegraphics{figures/tikz/spectrum-downconversion}
	\caption{Power spectrum comprising a modulated carrier signal at carrier frequency $\omega_c$ with bandwidth $B_d$, a \gls{lo} signal with frequency slightly above the carrier frequency, $\omega_l>\omega_c$. The downconverted modulated-carrier spectra are indicated by dashed lines, lower band at $\omega_l-\omega_c>0$, and dotted lines, upper band at $\omega_l+\omega_c$. The detector bandwidth is indicated by $B_d>\omega_l-\omega_c$.}\label{fig:spectrum_downconversion}
\end{figure}
As it is technical unfeasible to measure the modulated carrier directly at the carrier frequency, we demodulate or downconvert the information-bearing signal from the carrier by mixing the received signal with a \gls{lo} at frequency $\omega_l$, producing a low- and high-frequency signal at $\omega_l-\omega_c$ respectively $\omega_l+\omega_c$, as depicted in \Cref{fig:spectrum_downconversion}.
For a useful measurement, the \gls{lo} frequency $\omega_l$ should be chosen such that the detector bandwidth $B_d$ covers the complete low-frequency signal, i.e., $\omega_l-\omega_c+B_s/2<B_d$.
The relative dependence of the measured spectrum from the \gls{lo} frequency $\omega_l$ and the fact that the modulated carrier signal and the information-bearing signal contain the same information suggests introducing the concept of base- and passband representation~\cite{Madhow2008}.
The passband representation corresponds to the physical reality where the information-bearing signal is modulated onto the carrier.
The passband signal is real-valued, and the spectrum has complex conjugate symmetry.
However, when we measure the spectrum, we do so with a relative frequency and obtain an asymmetric spectrum centered at zero frequency, the baseband representation.
\begin{figure}[ht]
	\centering
	\includegraphics{figures/pgfplots/signal-spectrum}
	\caption{Power spectrum comprising showing the received signal as baseband. At \SI{+100}{\mega\hertz}, the spectrum has a pilot tone. Centered at \SI{-25}{\mega\hertz}, the spectrum shows a first passband signal with \SI{12.5}{\mega\hertz} bandwidth. Centered at \SI{-168.75}{\mega\hertz}, the spectrum has a second passband signal with \SI{12.5}{\mega\hertz} bandwidth.}\label{fig:signal_spectrum}
\end{figure}
\Cref{fig:signal_spectrum} shows the power spectrum of a received signal in baseband representation comprising two signal bands and a pilot tone.

To sum up, single-mode quantum optics provides precise physical meaning to light, including quantum effects, although limited to monochromatic light.
On the other side, communication engineering provides a framework for efficiently encoding, transmitting, receiving, and decoding information but attempts no statements about the underlying physics.
For quantum-optical communication, it is inevitable to welcome and incorporate both views.
For instance, people with a background in quantum optics but foreign to communication engineering often advocate the concept of "one state, one universe", where each quantum transmission is completely independent.
However, if we include practical considerations, like assuming a single transmission line, the picture of "one state, one universe" is plagued by several ambiguities.
For example, a single-mode quantum state with well-defined frequency $\omega_0$ represents a perfect sinusoidal wave with infinite duration, making information transmission absurd.
The typical counter-argument is that single-mode quantum optics implicitly assumes pulses with $\omega_0$ being the center frequency of the pulse.
While the counter-argument is technically valid, we must admit that it only raises new questions, such as bandwidth-limitations on the pulse parameters, all properly addressed in communication engineering.

The multi-mode quantum-optics mentioned in popular quantum-optics books~\cite{Gerry2005,Fox2006} are insufficient to represent continuous-time signals, and performing a continuum limit might not be correct if we consider the huge differences between linear algebra and functional analysis.
The advanced quantum-optics literature~\cite{Vogel2006,Mandel1995} does sometimes use a continuous-mode formalism but does not explicitly investigate its properties.
We are only aware of two quantum-optics books~\cite{Loudon2000,Barnett2002} that explicitly discuss a continuous-mode theory of light but again open up new questions regarding the fundamental assumptions and justification thereof.
If we are willing to go one step deeper, we find answers in the quantum field-theory literature~\cite{Peskin1995,Srednicki2007,Greiner2013,Itzykson2012}, but it is up to us to transfer these insights from particle physics to quantum-optics applications.
We even have to go a bit deeper and look into mathematical quantum field-theory~\cite{Streater2016,Bogoliubov1982,Bogolubov1989} to answer some questions.
Finally, we want to understand and upgrade quantum models of (electro-)optical components in the literature~\cite{Vogel2006,Leonhardt2003,Haroche2006,Mandel1995} to a mode continuum for comparison with the results from the optical-communication community~\cite{Shapiro2009,Kikuchi2016}.
Regarding communication engineering, we retain the well-established signal-processing fundamentals, as presented, for example, in Ref.~\cite{Rauscher2011,Nossek2015,Madhow2008,Proakis2007,Gallager2008}.

\subsection*{Thesis outline}

Our work is divided into four chapters.
In \Cref{ch:qkd}, we present an introduction to \gls{qkd}, emphasizing the similarities between the plethora of seemingly different protocols and attempting to argue why \gls{cvqkd} resembles a coherent-state communication system.
In the following three chapters, we construct our theoretical framework for quantum-optical communication towards \gls{cvqkd}.
Starting from a general quantum theory of light in \Cref{ch:light}, applying the quantum theory to describe the building blocks of coherent communication systems in \Cref{ch:components}, to an abstract description of a coherent-state transmission system's signal processing in \Cref{ch:system}.
While the thesis chapter structure supports a bottom-up approach, it is equally possible to read the thesis from the back to the front, revealing more and more details.
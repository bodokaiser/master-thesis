\chapter*{Introduction}
\addcontentsline{toc}{chapter}{Introduction}

Optical communication enables humanity worldwide to share information in a split second, with companies like Huawei undergoing tremendous efforts to advance the frontiers.
In addition to incremental innovation increasing the performance and decreasing the cost of optical communication technology, we observe intensified activities towards disruptive innovations that challenge our present understanding of communication.
One such branch of activity is quantum optical communication, incorporating quantum aspects of light into classical communication and leading to novel communication technology like \gls{qkd}, which enables practical and secure key generation.
As a still young discipline, which emerged from two highly advanced fields, communication engineering and quantum physics, quantum communication lacks a unified description to which both communication engineers and quantum physicists agree.
The present thesis aims to resolve the seeming discrepancies between communication engineering and quantum physics by reviewing a practical implementation of a quantum communication system implementing a \gls{qkd} protocol.
In the process, we hope to develop a theoretical framework for quantum optical communication, incorporating quantum effects into classical communication, which has applicability beyond \gls{qkd}.

\section*{Problem statement}

To raise awareness of the challenges ahead, we review the best-known quantum theory of light, single-mode quantum optics, along with central ideas from classical communication and outline where these pictures conflict.

In single-mode quantum optics, we model monochromatic light with frequency $\omega_0$ as a quantum harmonic oscillator with unit mass, $m=1$, and Hamiltonian~\cite{Gerry2005,Fox2006}
\begin{equation}
	\hat{H}
	=
	\omega_0
	\hat{a}^\dagger
	\hat{a}
	,
\end{equation}
wherein $\hat{a}$ and $\hat{a}^\dagger$ are the quantum annihilation and creation operators, destroying or creating an excitation or "mode" of frequency $\omega_0$.
The electric field operator,
\begin{equation}
	\hat{E}(t,x)
	=
	\mathcal{E}_0
	\left(
		\hat{a}
		+
		\hat{a}^\dagger
	\right)
	\sin(\omega_0x)
	,
\end{equation}
wherein $\mathcal{E}_0$ has the interpretation of an electric field density, establishes the connection between the quantum harmonic oscillator and electromagnetic radiation, including light~\cite[p.~12]{Gerry2005}.
Two of the most important quantum states are the number and the coherent state,
\begin{align}
	\ket{n}
	&=
	\frac{1}{\sqrt{n!}}
	\left(\hat{a}^\dagger\right)^n
	\ket{0}
	&
	\ket{\alpha}
	&=
	\exp\left(-\frac{1}{2}\abs{\alpha}^2\right)
	\sum_{n=0}^\infty
	\frac{\alpha^n}{\sqrt{n!}}
	\ket{n}
	.
\end{align}
The number state is parametrized by a natural number $n\in\mathbb{N}_0$ counting the number excitations.
The coherent state is parametrized by a complex number $\alpha\in\mathbb{C}$ encoding amplitude and phase.
The expectation value of the electric field operator with respect to a coherent state,
\begin{equation}
	\bra{\alpha}
	\hat{E}(t)
	\ket{\alpha}
	=
	\sqrt{2}
	\abs{\alpha}
	\mathcal{E}_0
	\sin\left(\omega_0t-\theta\right)
	,
\end{equation}
equals a classical monochromatic wave with amplitude proportional to $\abs{\alpha}$ and phase $\theta$~\cite[p.~45]{Gerry2005}.

In communication engineering, light is primarily a means of transmitting signals that appear as frequency bands centered around an optical carrier frequency $\omega_c$, as illustrated in \Cref{fig:signal_spectrum}.
\begin{figure}[ht]
	\centering
	\includegraphics{figures/pgfplots/signal-spectrum}
	\caption{Receiver spectrum comprising multiple signal bands relative to a carrier frequency at $\omega=0$. At \SI{+100}{\mega\hertz}, the spectrum has a pilot tone broadened by phase noise. Centered at \SI{-25}{\mega\hertz}, the spectrum contains a first signal band with \SI{12.5}{\mega\hertz} bandwidth. Centered at \SI{-168.75}{\mega\hertz}, the spectrum contains a second signal band with \SI{12.5}{\mega\hertz} bandwidth. The remaining segments of the spectrum include mirror bands or disturbances.}\label{fig:signal_spectrum}
\end{figure}
The concept of frequency bands is extremely powerful as it allows the transmission of multiple independent signals around one carrier frequency.
More formally, we need to distinguish between base- and passband signals.
For a baseband signal $x_b(t)$, the signal power outside of the signal's bandwidth $B$ is negligible~\cite[p.~15]{Madhow2008}, i.e.,
\begin{align}
	\abs{x_b(\omega)}^2
	\approx
	0
	&&
	\abs{\omega}
	>
	B/2
	.
\end{align}
For a passband signal $x_p(t)$, the signal power outside the signal's band around a carrier frequency $\omega_c$ is negligible~\cite[p.~16]{Madhow2008}, i.e.,
\begin{align}
	\abs{x_p(\omega)}^2
	\approx
	0
	&&
	\abs{\omega\pm\omega_c}
	>
	B/2
	.	
\end{align}
\begin{figure}[ht]
	\centering
	\includegraphics{figures/tikz/up-conversion}
	\caption{Power spectrum illustrating up-conversion of a real-valued passband signal with bandwidth $B$ centered at $\omega_0$. Up-conversion by $\omega_c$ shifts the passband to $\omega_c+\omega_0$ and creates a mirror band at $\omega_c-\omega_0$.}\label{fig:up_conversion}
\end{figure}
A baseband signal can be up-converted to a passband signal at carrier frequency $\omega_c$ by shifting the spectrum by $\omega_c$ is known as up-conversion, see \Cref{fig:up_conversion}, and implemented by modulation.
Similar, a passband signal at carrier frequency $\omega_c$ is down-converted to a baseband signal by demodulation~\cite[p.~26]{Madhow2008}.


At first glance, it seems that the two descriptions could not have been more different.
At a second glance, we notice that the monochromatic assumption limits the presented quantum optics theory but extending the quantum theory of light to a continuous frequency spectrum should make it possible to transfer the concepts from communication engineering.


% literature review

\section*{Thesis outline}

Our work is divided into four chapters.
In \Cref{ch:qkd}, we present an introduction to \gls{qkd}, emphasizing the similarities between the plethora of seemingly different protocols and attempting to argue why practical \gls{qkd} based on weak coherent states is effectively a coherent state communication system.
In the following three chapters, we construct our theoretical framework for quantum optical communication towards practical \gls{qkd}, starting from a general quantum theory of light, \Cref{ch:light}, over applying the quantum theory to describe the building blocks of coherent communication systems, \Cref{ch:components}, to an abstract description of a coherent state transmission system's signal-processing, \Cref{ch:system}.
While the thesis chapter structure supports a bottom-up approach, it is equally possible to read the thesis from the back to the front, revealing more and more details.
Likewise, it is possible to skip certain chapters and pare down to the chapter summary at the end of each chapter.
\chapter*{Introduction}
\addcontentsline{toc}{chapter}{Introduction}

Optical communication enables humanity worldwide to share information in a split second, and companies like Huawei undergo a tremendous effort to overcome existing boundaries.
Besides incremental innovation aiming to increase capacity and decrease the cost of optical communication technology, we observe intensified activities towards disruptive innovations that challenge our present understanding of communication.
One such branch of activity, quantum optical communication, attempts to incorporate quantum properties of light into classical communication to open up entirely new avenues.
One such avenue considers the creation of long-range entanglement of quantum systems for distributed quantum computing and sensors, quantum repeaters.
A more established application concerns the practical and secure key generation for secure communication, \gls{qkd}, which we pursue here.
Whenever two highly advanced disciplines like optical communication and quantum mechanics interface, we are confronted with two different schools of thought, which at first may appear in conflict.
However, we also face the opportunity to challenge our existing knowledge and broaden our understanding of the world.
The present thesis aims to resolve the seeming discrepancies by reviewing a practical implementation of a \gls{qkd} system through a theoretical framework under which communication engineers and quantum physicists agree and recognize the key concepts of their domain in a unified language.

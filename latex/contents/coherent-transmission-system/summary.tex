\section*{Summary}
\addcontentsline{toc}{section}{Summary}

We presented a software-defined coherent-state transmission system for en- and decoding of a complex symbol sequence onto and from an optical coherent state by following the signal processing across the quantum-optical, analog and digital domains.
For simplicity, we assumed a perfect noiseless quantum channel without attenuation and perfect synchronization between the transmitter and receiver clocks.\footnote{\Cref{app:synchronization} summarizes some important techniques to compensate for synchronization error of the optical \glspl{lo} and the \gls{adc} clock.}
Ref.~\cite{Nielsen2010} considers realistic quantum channels, like the quantum analog of the classical additive Gaussian white-noise channel.
To extend the detection to nonclassical states, methods of quantum tomography, as discussed in, for example, Ref.~\cite{Vogel2006}, need to be considered.

\Cref{fig:transmitter_signal_processing} summarizes the transmitter's signal processing in a block diagram.
First, the complex digital signal corresponding to the symbol sequence $\alpha[n]$ are encoded in two real-valued bandwidth-optimized digital baseband signals.
The digital baseband signals are transferred to the analog domain, where we upconvert them to a single passband signal.
\begin{figure}[htb]
	\centering
	\includegraphics{figures/circuitikz/transmitter}
	\caption{Block diagram of the transmitter's signal processing. The real and imaginary part of a complex digital signal $\alpha[n]$ is upsampled, pulse-shaped and converted to anti-aliased analog signals, $x(t)$ and $p(t)$. The analog signals are individually mixed with a phase-shifted \gls{lo} with carrier frequency $\omega_c$ and then added to yield a complex signal $\alpha(t)$.}\label{fig:transmitter_signal_processing}
\end{figure}
For the receiver we discussed the differences between a single, dual homo- and heterodyne receiver.
With homodyning, the optical signal is directly downconverted without intermediate frequency, $\omega_i=0$, while with heterodyning we have an intermediate frequency, which we remove in the digital domain.
Although, homodyning directly reveals the quadrature information, it requires dual-quadrature downconversion in the electro-optical domain, increasing the hardware complexity.
Using heterodyning both quadratures can be digitally restored given sufficient detector bandwidth.
\begin{figure}[htb]
	\centering
	\includegraphics{figures/circuitikz/receiver}
	\caption{Block diagram of the heterodying receiver's signal processing. The signal $z(t)$ is mixed with $\cos(\omega_ct+\vartheta)$. The mixed analog signal is bandwidth-limited through a \gls{lp} filter and converted to the digital domain where it is digitally downconverted through multiplication with $\exp(i\omega_it)$. The digitally downconverted signal is downsampled a first time for compatibility with the matched \gls{rrc} filter and then downsampled a second time to recover the symbols.}\label{fig:receiver_signal_processing}
\end{figure}
\Cref{fig:receiver_signal_processing} summarizes our heterodyning receiver's signal processing in a block diagram.
First, the received passband signal $z(t)$ is downconverted to an intermediate (radio) frequency and converted to a digital signal.
The digital signal is multiplied by a complex exponential removing the intermediate frequency and recovering dual-quadrature information.
Finally, the pulse shaping is completed and the digital signal is downsampled to recover the digital signal $\beta[n]$ corresponding to the symbol sequence.

We found the coherent-state transformations for the \gls{iqm} and balanced detector, derived in the previous chapter, to exactly implement the electro-optical up- and downconversion of the transmitter respectively receiver.
\begin{table}[htb]
  \centering
  \begin{tabular}{lccc}
    \toprule
    Signal-processing & Optical implementation & Input signal & Output signal \\
    \midrule
    Upconversion & Modulation & $\alpha(t)$ & $\ket{\alpha(t)e^{-i\omega_c t}}$ \\
    Downconversion & Balanced detection & $\ket{\alpha(t)e^{-i\omega_c t}}$ & $\alpha(t)e^{-i\omega_it}$ \\
    \bottomrule
  \end{tabular}
  \caption{Electro-optical components implementing signal-processing operations with complex baseband signal $\alpha(t)$, optical carrier frequency $\omega_c$ and electrical intermediate frequency $\omega_i$.}\label{tab:electro_optical_signal_processing}
\end{table}
\Cref{tab:electro_optical_signal_processing} summarizes the relation between input and output state and signals for the electro-optical components implementing the up- and downconversion.
We have thus successfully bridged the gap between signal processing and quantum-optical communication.
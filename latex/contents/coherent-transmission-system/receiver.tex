\section{Receiver}
\FloatBarrier

We introduced the receiver as a component decoding a sequence of complex symbols,
\begin{equation}
	\left\{
		\beta_n
		\in
		\mathbb{C}
		\colon
		n\in I
	\right\}
	,
\end{equation}
from a coherent state $\ket{\beta(t)}$.
Just like the transmitter, we want to keep the receiver software-defined.
\begin{figure}[htb]
	\centering
	\includegraphics{figures/tikz/software-defined-receiver}
	\caption{Block diagram of the receiver's signal processing domains. The analog electrical signals $u(t)$, and optional $v(t)$, are demodulated from the quadratures of the optical coherent state $\ket{\beta(t)}$, and then converted to the digital signals $u[m]$, and optional $v[m]$, from which the \gls{dsp} decodes the symbol sequence $\{\beta_n\in\mathbb{C}\colon n\in I\}$.}\label{fig:software_defined_receiver}
\end{figure}
\Cref{fig:software_defined_receiver} shows the signal processing of a possible software-defined receiver.
The coherent state is transferred from the optical via the analog to the digital.

\FloatBarrier
\subsection{Downconversion}

At the transmitter, we upconverted two real baseband signals to a real passband signal.
For the receiver, we discuss options involving one and two real baseband signals.
We first consider the simpler case of direct downconversion as depicted in \Cref{fig:downconversion_single}.
\begin{figure}[htb]
	\centering
	\includegraphics{figures/circuitikz/downconversion-single}
	\caption{Block diagram of single-quadrature downconversion. The signal $z(t)$ is mixed with the \gls{lo} signal $\cos(\omega_lt+\vartheta)$. The downconverted signal $u(t)$ is obtained by \gls{lp} filtering the output signal of the mixing.}\label{fig:downconversion_single}
\end{figure}
In direct downconversion, we mix a real-valued signal,
\begin{equation}
	\begin{split}
		z(t)
		=
		\int_{-\infty}^{+\infty}\frac{\dd{\omega}}{2\pi}
		z(\omega)
		e^{+i\omega t}
		&=
		\int_0^{+\infty}\frac{\dd{\omega}}{2\pi}
		z(\omega)
		e^{+i\omega t}
		+
		\int_{-\infty}^0\frac{\dd{\omega}}{2\pi}
		z(\omega)
		e^{+i\omega t}
		\\
		&=
		\int_0^{+\infty}\frac{\dd{\omega}}{2\pi}
		z(\omega)
		e^{+i\omega t}
		-
		\int_{+\infty}^0\frac{\dd{\omega}}{2\pi}
		z(-\omega)
		e^{-i\omega t}
		\\
		&=
		\int_0^{+\infty}\frac{\dd{\omega}}{2\pi}
		\left[
			z(\omega)
			e^{+i\omega t}
			+
			z(\omega)^*
			e^{-i\omega t}
		\right]
		\\
		&=
		\int_0^{+\infty}\frac{\dd{\omega}}{2\pi}
		2\Re\left[
			z(\omega)
			e^{+i\omega t}
		\right]
		,
	\end{split}
\end{equation}
where we used the conjugate symmetry, $z(-\omega)=z(\omega)^*$, of the Fourier transform of a real-valued function $z(t)$.
Multiplication with the \gls{lo} signal $\cos(\omega_lt+\vartheta)$, the mixing produces a high- and low-frequency band
\begin{equation}
	\begin{split}
		z(t)
		\cos(\omega_lt+\vartheta)
		&=
		2\Re
		\int_0^\infty\frac{\dd{\omega}}{2\pi}
		z(\omega)
		e^{+i\omega t}
		\cos(\omega_lt+\vartheta)
		\\
		&=
		\Re
		\int_0^\infty\frac{\dd{\omega}}{2\pi}
		z(\omega)
		e^{+i\omega t}
		\left[
			e^{+i(\omega_lt+\vartheta)}
			+
			e^{-i(\omega_lt+\vartheta)}
		\right]
		\\
		&=
		\Re
		\int_0^\infty\frac{\dd{\omega}}{2\pi}
		z(\omega)
		\left[
			e^{+i(\omega+\omega_l)t+i\vartheta}
			+
			e^{+i(\omega-\omega_l)t-i\vartheta}
		\right]
		\\
		&=
		\Re
		\int_{+\omega_l}^\infty\frac{\dd{\omega}}{2\pi}
		z(\omega-\omega_l)
		e^{+i\omega t+i\vartheta}
		\\
		&\qquad+
		\Re
		\int_{-\omega_l}^\infty\frac{\dd{\omega}}{2\pi}
		z(\omega+\omega_l)
		e^{+i\omega t-i\vartheta}
		.
	\end{split}
\end{equation}
However, we suppress the high-frequency band using an ideal \gls{lp} filter with bandwidth $B$,
\begin{equation}
	\begin{split}
		u(t)
		&=
		\Re
		\int_{-B/2}^{+B/2}\frac{\dd{\omega}}{2\pi}
		z(\omega+\omega_l)
		e^{+i\omega t-i\vartheta}
		\\
		&=
		\Re
		\int_{0}^{+B/2}\frac{\dd{\omega}}{2\pi}
		z(\omega+\omega_l)
		e^{+i\omega t-i\vartheta}
		+
		\Re
		\int_{-B/2}^{0}\frac{\dd{\omega}}{2\pi}
		z(\omega+\omega_l)
		e^{+i\omega t-i\vartheta}
		\\
		&=
		\Re
		\int_{0}^{+B/2}\frac{\dd{\omega}}{2\pi}
		z(\omega+\omega_l)
		e^{+i\omega t-i\vartheta}
		-
		\Re
		\int_{B/2}^{0}\frac{\dd{\omega}}{2\pi}
		z(\omega-\omega_l)^*
		e^{-i\omega t-i\vartheta}
		\\
		&=
		\Re
		\int_{0}^{+B/2}\frac{\dd{\omega}}{2\pi}
		\left[
			z(\omega+\omega_l)
			e^{+i\omega t}
			+
			z(\omega-\omega_l)^*
			e^{-i\omega t}
		\right]
		e^{-i\vartheta}
		,
	\end{split}
	\label{eq:downconversion_real}
\end{equation}
where we assumed $\omega_l\gg B/2$.
For $\vartheta=0$, the downconverted signal $v(t)$ is equal to projecting the real part of the complex input spectrum $z(\omega)$, losing the imaginary part's information.
Furthermore, when rewriting $u(t)$ as an integral over positive frequencies, i.e., frequencies we can measure, we find a second term mirroring the first term.
\begin{figure}[htb]
	\centering
	\includegraphics{figures/tikz/spectrum-downconversion-single}
	\caption{Power spectrum illustrating downconversion of a passband signal (solid spectrum) mixed with a \gls{lo} signal $\omega_l$ to the intermediate frequency $\omega_i$ (dashed spectrum) and bandwidth-limited measurement (dotted spectrum).}\label{fig:spectrum_downconversion_single}
\end{figure}
\Cref{fig:spectrum_downconversion_single} shows the downconversion of the signal $z(t)$ around the \gls{lo} at $\omega_l$ to the intermediate frequency $\omega_i$.
The actual measurement involved only positive frequencies up to the detector bandwidth $B/2$ causing the actual signal to be imposed with the mirrored spectrum.

Single-quadrature downconversion only reveals a real projection of the complex spectrum.
To conserve both quadratures, we need to split the input signal into two branches and perform single-quadrature downconversion with two orthogonal phase references of the \gls{lo}, see \Cref{fig:downconversion_dual}.
\begin{figure}[htb]
	\centering
	\includegraphics{figures/circuitikz/downconversion-dual}
	\caption{Block diagram of dual-quadrature downconversion. The signal $z(t)$ is divided equally into an upper and a lower branch. The upper branch is mixed with the phase shifted $\gls{lo}$ signal $\cos(\omega_ct+\vartheta)$. The lower branch is mixed with \gls{lo} signal $\sin(\omega_ct+\vartheta)$. The mixer outputs are filtered separately by a \gls{lp} yielding the down-converted signals $u(t)$ and $v(t)$.}\label{fig:downconversion_dual}
\end{figure}
The signal of the upper branch $u(t)$ is equal to our result for the single-quadrature downconversion, \cref{eq:downconversion_real}.
The signal of the lower branch,
\begin{equation}
	\begin{split}
		v(t)
		&=
		\Im
		\int_{-B/2}^{+B/2}\frac{\dd{\omega}}{2\pi}
		z(\omega-\omega_l)
		e^{+i(\omega t+\vartheta)}
		\\
		&=
		\Im
		\int_{0}^{+B/2}\frac{\dd{\omega}}{2\pi}
		\left[
			z(\omega-\omega_l)
			e^{+i(\omega t+\vartheta)}
			+
			z(\omega+\omega_l)^*
			e^{-i(\omega t+\vartheta)}
		\right]
		,
	\end{split}
	\label{eq:downconversion_imag}	
\end{equation}
is simply obtained from \cref{eq:downconversion_real} by shifting the \gls{lo} phase reference by \SI{90}{\degree}, i.e., $\vartheta\to\vartheta+\pi/2$.
Regardless of the particular value of the \gls{lo} phase reference $\vartheta$, dual-quadrature downconversion recovers the complete information, the real and imaginary part, of the input signal spectrum $z(\omega)$.

We presented the electro-optical receiver setups implementing single- and dual-quadrature downconversion in \Cref{fig:coherent_receiver_active} and \Cref{fig:coherent_receiver_passive} in \Cref{ch:qkd}.
Essentially, the electro-optical setups combine the optical signal and \gls{lo} in an optical coupler and perform balanced detection on the coupler outputs for which derived the quantum theory in \Cref{sec:photodetectors}.
From a quantum viewpoint, balanced detection corresponds to a quadrature measurement at a particular frequency represented by the generalized quadrature operator, \cref{eq:quadrature_operator_generalized}.

\subsection{Homo- and heterodyning}

So far, we have not assumed any particular signal for the downconversion but treated the receiver as a spectrum analyzer.
If we now assume the input signal to be from the coherent-state transmitter $\ket{\beta(t)}$, the downconverted signals read
\begin{align}
	u(t)
	&=
	\Re
	\int_{-B/2}^{+B/2}\frac{\dd{\omega}}{2\pi}
	\beta(\omega-\omega_i)
	e^{+i(\omega t+\vartheta)}
	\label{eq:receiver_demod_real}
	\\
	v(t)
	&=
	\Im
	\int_{-B/2}^{+B/2}\frac{\dd{\omega}}{2\pi}
	\beta(\omega-\omega_i)
	e^{+i(\omega t+\vartheta)}
	\label{eq:receiver_demod_imag}
	,
\end{align}
where we defined the intermediate frequency as the difference between the transmitter- and receiver-laser frequency
\begin{equation}
	\omega_i
	=
	\omega_c-\omega_l
	<
	B/2
	.
\end{equation}
If the intermediate frequency is zero, $\omega_i=0$, we perform homodyning, otherwise, heterodyning.
Furthermore, we distinguish between single- and dual-quadrature homodyning, depending if we perform single- or dual- quadrature downconversion.
If the detector bandwidth is large enough to cover the baseband signal at the intermediate frequency, we can resolve both quadratures of the incoming signal with heterodyning and single-quadrature downconversion.
\begin{table}[htb]
  \centering
  \begin{tabular}{lccccc}
    \toprule
    Scheme & Homodyne (single) & Homodyne (dual) & Heterodyne \\
    \midrule
    Balanced detectors & \num{1} & \num{2} & \num{1} \\
    Quadratures & \num{1} & \num{2} & \num{2} \\
    Intermediate frequency & \multicolumn{2}{c}{$\omega_i=0$} & $\omega_i\neq 0$ \\
    Optical complexity & Low & High & Low \\
    Signal bandwidth & High & High & Low \\
    \gls{lo} synchronization & Frequency and phase & Frequency & Bandwidth \\
    \bottomrule
  \end{tabular}
  \caption{Comparison of receiver schemes according to Ref.~\cite{Brunner2017}. The single-quadrature homodyne detection offers low optical complexity and high bandwidth but only resolves one of two quadratures and required frequency and phase synchronization of the \gls{lo}. The dual-quadrature homodyne-detection resolves both quadratures with high bandwidth but requires two balanced detectors increasing the optical complexity and phase synchronization of the \gls{lo}. The heterodyne detection schemes resolves both quadratures with low complexity and no requirements on \gls{lo} synchronization at the cost of signal bandwidth.}\label{tab:receivers}
\end{table}
\Cref{tab:receivers} summarizes the characteristics between the single and dual homodyne and the heterodyne receiver schemes.
A strong advantage of the heterodyne receiver design is that both quadratures can be resolved with a single balanced detector, keeping the optical complexity low.
\begin{figure}[htb]
	\centering
	\includegraphics{figures/tikz/spectrum-heterodyning}
	\caption{Foo}\label{fig:spectrum_heterodyning}
\end{figure}

\FloatBarrier
\subsection{Symbol decoding}

We continue our receiver description, starting from the single-quadrature downconversion and assuming the more general heterodyning, which for $\omega_i=0$ reduces to single-quadrature homodyning.
\begin{figure}[htb]
	\centering
	\includegraphics{figures/circuitikz/symbol-decoding}
	\caption{Block diagram of the signal processing for the symbol decoding. The analog signal $u(t)$ is converted to the digital signal $u[m/(kl)]$. The real digital signal $u[m/(kl)]$ is multiplied with the complex exponential $\exp(i\omega_it)$, yielding the complex digital signal $\sigma[m/(kl)]$. $\sigma[m/(kl)]$ is downsampled by $l$ to yield the complex digital signal $\mu[m/k]$. $\mu[m/k]$ is pulse-shaped with the matched \gls{rrc} filter to yield the complex digital signal $\kappa[m/k]$. $\kappa[m/k]$ is downsampled to the complex digital signal $\beta[m]$ corresponding to the decoded symbol sequence.}\label{fig:symbol_decoding_blocks}
\end{figure}
\Cref{fig:symbol_decoding_blocks} summarizes the relevant signal processing for the symbol decoding.
The downconverted signal $u(t)$ corresponding to the real part of the received coherent-state spectrum $\beta(\omega)$, \cref{eq:receiver_demod_real}, is sampled by an \gls{adc}, yielding the digital signal $u[m/(kl)]$.
We remove the intermediate frequency in $u[m/(kl)]$ by multiplication with a complex exponential, i.e,
\begin{equation}
	\sigma\left[\frac{m}{kl}\right]
	=
	u\left[\frac{m}{kl}\right]
	e^{+2\pi i (m/kl) T_s}
	,
\end{equation}
making the signal complex-valued.
It follows a downsampling by $l$ of the signal such that we can apply the same \gls{rrc} filter, the matched filter, which we used in the symbol encoding to maximize \gls{snr}.
Finally, we downsample by $k$ to restore a digital signal corresponding to the symbol sequence.
\begin{figure}[htb]
	\centering
	\includegraphics{figures/pgfplots/symbol-decoding-frequency}
	\caption{Power spectrum of the symbol-decoding steps for a random \gls{qpsk} symbol sequence. The demodulated signal is a real-valued passband signal centered at the intermediate frequency (first row). After digital downconversion we have a complex-valued baseband signal, centered at zero frequency (second row). Applying the matched \gls{rrc} filter completes the pulse-shaping (third row). Downsampling recovers the initial symbol band (last row).}\label{fig:symbol_decoding_frequency}
\end{figure}
\Cref{fig:symbol_decoding_frequency} illustrates how the symbol decoding is carried out in the frequency domain.
The demodulated signal spectrum is a passband signal at the intermediate frequency and downconversion reduces the passband to a baseband signal.
Completing the pulse-shaping with the matched filter increases the steepness of the flanks which are collapsed with aliasing by the final downsampling step.
\begin{figure}[htb]
	\centering
	\includegraphics{figures/pgfplots/symbol-decoding-time-qpsk}
	\caption{Signal amplitude of the symbol decoding steps for a random \gls{qpsk} symbol sequence. The real-valued demodulated signal oscillates at the intermediate frequency (first row). Digital downconversion removed the intermediate frequency, yielding a complex signal (second row). Completing the pulse-shaping and downsampling by applying a matched \gls{rrc} filter (third row). Downsampling recovers the complex symbol sequence equal to the transmitted sequence (fourth and last row).}\label{fig:symbol_decoding_time}
\end{figure}
\Cref{fig:symbol_decoding_time}) shows the symbol decoding in the time domain.
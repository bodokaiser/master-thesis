\section{Receiver}

% what is the task of the receiver

% practical received spectrum, receiver as spectrum analyzer
\begin{figure}[htb]
	\centering
	\includegraphics{figures/pgfplots/rx-psd}
	\caption{Power spectral density of a practical transmitter's baseband signal. At \SI{100}{\mega\hertz} a strong }\label{fig:transmit_spectrum}
\end{figure}


\begin{table}[htb]
  \centering
  \begin{tabular}{lccccc}
    \toprule
    Receiver design & Homodyne (single) & Homodyne (dual) & Heterodyne \\
    \midrule
    Balanced detectors & \num{1} & \num{2} & \num{1} \\
    Quadratures & \num{1} & \num{2} & \num{2} \\
    Optical complexity & Low & High & Low \\
    Signal bandwidth & High & High & Low \\
    \gls{lo} synchronization & Frequency and phase & Frequency & Bandwidth \\
    \bottomrule
  \end{tabular}
  \caption{Comparison of receiver implementations according to Ref.~\cite{Brunner2017}: The single quadrature homodyne detection offers low optical complexity and high bandwidth but only resolves one of two quadratures and required frequency and phase synchronization of the \gls{lo}. The dual quadrature homodyne detection resolves both quadratures with high bandwidth but requires two balanced detectors increasing the optical complexity and phase synchronization of the \gls{lo}. The heterodyne detection schemes resolves both quadratures with low complexity and no requirements on \gls{lo} synchronization at the cost of signal bandwidth.}\label{tab:receivers}
\end{table}

\subsection{Single quadrature down-conversion}

\subsection{Dual quadrature down-conversion}

% figure down-conversion

% down-conversion equivalent to balanced detection

% equivalence of balanced detection signal with expectation value of quadrature operator
Assuming the \gls{lo} field to be perfectly monochromatic, i.e.,
\begin{equation}
	\alpha_l(t)
	=
	\abs{\alpha_l}
	e^{+i\omega_lt+i\vartheta}
	,
\end{equation}
we can write the detected signal
\begin{equation}
	\alpha_s(t)
	\alpha_l^*(t)
	+
	\alpha_s^*(t)
	\alpha_l(t)
	=
	\abs{\alpha_l}
	\alpha_s(t)
	e^{+i\omega_l t}
	+
	\alpha_s^*(t)
	\alpha_l(t)	
\end{equation}

\FloatBarrier
\subsection{Symbol decoding}

\begin{figure}[htb]
	\centering
	\includegraphics{figures/circuitikz/receiver-dsp}
	\caption{foo}
\end{figure}

\begin{figure}[htb]
	\centering
	\includegraphics{figures/pgfplots/rx-frequency}
	\caption{foo}
\end{figure}

\begin{figure}[htb]
	\centering
	\includegraphics{figures/pgfplots/rx-rand-time}
	\caption{foo}
\end{figure}
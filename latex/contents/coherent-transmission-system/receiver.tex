\section{Receiver}
\FloatBarrier

We introduced the receiver as a component decoding a sequence of complex symbols, $\left\{\beta_n\in\mathbb{C}\colon n\in I\right\}$, from a coherent state $\ket{\beta(t)}$.
As with the transmitter, we want our receiver to be software-defined, meaning that the predominant signal processing is in the digital domain and we need to down-convert the optical signal.
\begin{figure}[htb]
	\centering
	\includegraphics{figures/tikz/software-defined-receiver}
	\caption{Block diagram of the receiver's signal processing domains. The analog electrical signals $v(t)$ and $w(t)$ are demodulated from the quadratures of the optical coherent state $\ket{\beta(t)}$, and then converted to the digital signals $v[m]$ and $w[m]$ from which the \gls{dsp} decodes the symbol sequence $\{\beta_n\in\mathbb{C}\colon n\in I\}$.}\label{fig:software_defined_receiver}
\end{figure}
\Cref{fig:software_defined_receiver} illustrates the signal flow across the different domains inside the receiver.
Unlike the transmitter up-conversion, the receiver's down-conversion leaves us some design freedom.
In particular, it is possible to reduce the hardware complexity significantly by introducing an intermediate frequency, which carries over to the digital domain.

\FloatBarrier
\subsection{Single quadrature down-conversion}

\begin{figure}[htb]
	\centering
	\includegraphics{figures/circuitikz/down-conversion-single}
	\caption{Block diagram of single quadrature down-conversion.}\label{fig:down_conversion_single}
\end{figure}

\FloatBarrier
\subsection{Dual quadrature down-conversion}

\begin{figure}[htb]
	\centering
	\includegraphics{figures/circuitikz/down-conversion-dual}
	\caption{Block diagram of dual quadrature down-conversion.}\label{fig:down_conversion_dual}
\end{figure}

% figure down-conversion

% down-conversion equivalent to balanced detection

% equivalence of balanced detection signal with expectation value of quadrature operator
Assuming the \gls{lo} field to be perfectly monochromatic, i.e.,
\begin{equation}
	\alpha_l(t)
	=
	\abs{\alpha_l}
	e^{+i\omega_lt+i\vartheta}
	,
\end{equation}
we can write the detected signal
\begin{equation}
	\alpha_s(t)
	\alpha_l^*(t)
	+
	\alpha_s^*(t)
	\alpha_l(t)
	=
	\abs{\alpha_l}
	\alpha_s(t)
	e^{+i\omega_l t}
	+
	\alpha_s^*(t)
	\alpha_l(t)	
\end{equation}

\FloatBarrier
\subsection{Symbol decoding}

\begin{table}[htb]
  \centering
  \begin{tabular}{lccccc}
    \toprule
    Receiver design & Homodyne (single) & Homodyne (dual) & Heterodyne \\
    \midrule
    Balanced detectors & \num{1} & \num{2} & \num{1} \\
    Quadratures & \num{1} & \num{2} & \num{2} \\
    Optical complexity & Low & High & Low \\
    Signal bandwidth & High & High & Low \\
    \gls{lo} synchronization & Frequency and phase & Frequency & Bandwidth \\
    \bottomrule
  \end{tabular}
  \caption{Comparison of receiver implementations according to Ref.~\cite{Brunner2017}: The single quadrature homodyne detection offers low optical complexity and high bandwidth but only resolves one of two quadratures and required frequency and phase synchronization of the \gls{lo}. The dual quadrature homodyne detection resolves both quadratures with high bandwidth but requires two balanced detectors increasing the optical complexity and phase synchronization of the \gls{lo}. The heterodyne detection schemes resolves both quadratures with low complexity and no requirements on \gls{lo} synchronization at the cost of signal bandwidth.}\label{tab:receivers}
\end{table}

\begin{figure}[htb]
	\centering
	\includegraphics{figures/circuitikz/symbol-decoding}
	\caption{foo}
\end{figure}

\begin{figure}[htb]
	\centering
	\includegraphics{figures/pgfplots/rx-frequency}
	\caption{foo}
\end{figure}

\begin{figure}[htb]
	\centering
	\includegraphics{figures/pgfplots/rx-rand-time}
	\caption{foo}
\end{figure}
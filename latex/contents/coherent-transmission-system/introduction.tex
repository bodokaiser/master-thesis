\chapter{Coherent state transmission system}

%The following chapter describes a coherent state transmission system from a signal-processing point of view, bridging the gap between the quantum theory of optical components and the quantum information theory of the \gls{cvqkd} protocol. In particular, we extend the standard description of a digital transmission system~\cite{Gallager2008,Nossek2015,Oppenheim1989,Proakis2007} to coherent light states following the setup described in Ref.~\cite{Brunner2017}.


A transmission system attempts to reproduce information from a spacetime event $x$, the information source, at some space-like separated event $y$, the information destination.\footnote{Contemporary to the literature, we distinguish between a transmission and a communication system. The former allows only unidirectional, the later bidirectional transport of information.}
\begin{figure}[htb]
	\centering
	\includegraphics{figures/tikz/transmission-system}
	\caption{Block diagram of a general transmission system according to Ref.~\cite{Shannon1948}. An information source produces information that a transmitter encodes into a signal and transmits it through a channel to the receiver. The channel adds noise from a noise source to the transmitted signal. The receiver recovers the information from the received signal and passes it to the intended information destination.}\label{fig:transmission_system}
\end{figure}
To transport information from the information source and the information destination, the transmission system uses a transmitter-receiver pair connected by a physical channel (\Cref{fig:transmission_system}).
The task of the transmitter is to encode the information in a physical signal for efficient transmission over the channel.
In its most general form, the channel is a mapping from the transmitted to the received signal.
The received signal may contain additional information, not of interest to the information destination, which we refer to as noise.
The receiver listens to the channel and attempts to recover the transmitted from the received signal for the information destination.

Based on the type of information source and the channel, we distinguish between different transmission systems.
For instance, the information source may produce information in a continuous or discrete value range.
It is impossible to transmit continuous-valued information without error, so communication typically considers information in the form of a finite set of symbols, the alphabet.
Regarding the channel, one typically assumes the channel to be a linear mapping with additive noise.

% logical vs. physical channel?

% discrete noiseless systems (continuous systems are impossible to transmit exactly, noiseless for simplification)
% atomic elements of communication (symbol, alphabet, bit, ...)



% how does a classical communication system relate to a coherent transmission system? (concept of noise)

\section{Concepts}

\subsection{Communication theory}

A transmission system attempts to reproduce information from a spacetime event $x$, the information source, at some space-like separated event $y$, the information destination.\footnote{Contemporary to the literature, we distinguish between a transmission and a communication system. The former allows only unidirectional, the later bidirectional transport of information.}
\begin{figure}[htb]
	\centering
	\includegraphics{figures/tikz/transmission-system}
	\caption{Block diagram of a general transmission system according to Ref.~\cite{Shannon1948}. An information source produces information that a transmitter encodes into a signal and transmits it through a channel to the receiver. The channel adds noise from a noise source to the transmitted signal. The receiver recovers the information from the received signal and passes it to the intended information destination.}\label{fig:transmission_system}
\end{figure}
\Cref{fig:transmission_system} presents a block diagram of a general transmission system as proposed in Ref.~\cite{Shannon1948}.
The information source produces information (messages) unknown to but intended for the information destination.
The transmitter encodes the messages onto a signal and transmits it through a channel.
The channel maps the transmitted signal onto a received signal.
The received signal contains unwanted information (noise).
The receiver listens to the channel and attempts to recover the transmitted message from the received signal at the information destination.


% discrete noiseless systems (continuous systems are impossible to transmit exactly, noiseless for simplification)
% atomic elements of communication (symbol, alphabet, bit, ...)



% how does a classical communication system relate to a coherent transmission system? (concept of noise)

\subsection{Signal-processing}

% purpose of signal-processing
% electric (analog, digital) domain
% optical (classical, quantum) domain

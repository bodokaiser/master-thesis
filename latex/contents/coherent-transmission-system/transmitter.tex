\section{Transmitter}                                                                                                                                                                                                                                                                                                                                                                                                                                                                                                                                                                                                                                                                                                                     

We introduced the transmitter as a component encoding a sequence of complex symbols, $\left\{\alpha_n\in\mathbb{C}\colon n\in I\right\}$, onto a coherent state $\ket{\alpha(t)}$.
Efficient transmission through the channel and effective receiver detection impose additional constraints on the space of useful coherent states.
Together with practical considerations, these constraints lead to the particular design embodiment of the transmitter we will discuss.

First and foremost, the channel and receiver limit the spectrum of useful coherent states.
For instance, the receiver has limited bandwidth to detect the signal with signal power outside that bandwidth being lost.
Apart from that, the physical channel only shows favorable transmission properties over a certain frequency range, outside the signal is strongly suppressed and distorted.\footnote{For instance, the C-band, spanning wavelengths from \SI{1530}{\nano\meter} to \SI{1565}{\nano\meter}, is widely deployed for optical telecommunication.}
Additionally, different users may jointly use the same physical channel, and using the available bandwidth efficiently while reducing interference between users, requires the signal bandwidth to be minimal.
\begin{figure}[htb]
	\centering
	\includegraphics{figures/tikz/baseband-passband}
	\caption{Power spectrum showing the relationship between a real-valued baseband and passband signal. Both base- and passband signals have bandwidth $B$. The baseband signal is centered at $\omega=0$. The passband is shifted by $\omega_0$ and has a conjugate symmetric counterpart at $-\omega_0$.}\label{fig:baseband_passband}
\end{figure}
To illustrate the signal and channel bandwidth, we introduce the concept of baseband and passband signals\cite[p.~26]{Madhow2008}.
\Cref{fig:baseband_passband} depicts the power spectrum of a baseband and passband signal each with bandwidth $B$.
The spectrum of the baseband signal is centered at zero frequency, $\omega=0$, while the spectrum of the passband signal is located at $\pm\omega_0$.
For efficient use of the limited channel and receiver bandwidth, we want to minimize $B$ while keeping the \gls{snr} high.
In addition, we need to shift the baseband spectrum to an optical frequency $\omega_0$ for which the channel shows desirable transmission characteristics.
So from a signal processing point of view, we want the transmitter to
\begin{enumerate}
	\item first create a baseband signal with minimum bandwidth $B$ and
	\item then transfer it to a passband signal in the optical domain.
\end{enumerate}
In the following, we term the first step signal synthesis and the second step up-conversion.

Nowadays, the signal is almost exclusively constructed in the digital domain, and the analog part is limited to the digital-to-analog conversion.
Constructing the signal digitally allows for greater flexibility in the development process as the synthesis is mostly software-defined.
For up-conversion of the synthesized signal to the optical domain, we modulate the electric signal onto an optical carrier.
\begin{figure}[htb]
	\centering
	\includegraphics{figures/tikz/software-defined-transmitter}
	\caption{Block diagram of the transmitter's signal processing domains. The \gls{dsp} transforms a complex symbol sequence $\{\alpha_n\}_{n\in I}$ into two digital signals, $x^\prime[m]$ and $p^\prime[m]$, corresponding to the real and imaginary part. The \gls{dac} converts the digital signals to analog signals, $x(t)$ and $p(t)$ we modulate onto a coherent state $\ket{\alpha(t)}$.}\label{fig:software_defined_transmitter}
\end{figure}
\Cref{fig:software_defined_transmitter} illustrates how such a software-defined transmitter architecture applies to our coherent state transmission system.
The software-defined \gls{dsp} constructs the bandwidth-optimized digital signals $x[m]$ and $p[m]$, encoding the real and imaginary parts of the complex symbols.
The \gls{dac} stage converts the digital signals, $x[m]$ and $p[m]$, to bandwidth-limited analog signals $x(t)$ and $p(t)$.
Finally, the analog signals are modulated onto an optical carrier yielding a coherent state $\ket{\alpha(t)}$ which meets the bandwidth requirements of the channel.

We now take a closer look at how first to encode a symbol sequence in a bandwidth-optimized signal and then transfer it to an optical coherent state.

\subsection{Signal synthesis}

In the signal synthesis stage, we encode a symbol sequence into a bandwidth-limited (baseband) signal.

The very first step is to introduce a notion of time to write the complex symbol sequence, $\left\{\alpha_n\in\mathbb{C}\colon n\in I\right\}$, as a digital (time-discrete) signal,
\begin{equation}
	\begin{split}
		\alpha(t)
		\colon
		\mathbb{R}
		&
		\to
		\mathbb{C}
		\\
		t
		&\mapsto
		\sum_{m\in I}
		\alpha_m
		\delta^{(1)}(t-mT)
		,
	\end{split}
	\label{eq:complex_digital_signal}
\end{equation}
by introducing the symbol period $T$, denoting the temporal distance between two consecutive symbols.
The digital signal in \cref{eq:complex_digital_signal} requires an infinite bandwidth and is therefore not suitable for direct transmission.


% why can't we directly use the digital signal?
% concept of samples


\begin{figure}[htb]
	\centering
	\includegraphics{figures/circuitikz/signal-synthesis-pipeline}
	\caption{Block diagram of the signal synthesis pipeline. The digital signal $x[km]$ is upsampled by a factor $k$ to $x[m]$ and transformed by a \gls{rrc} filter to $x^\prime[m]$. A \gls{dac} converts the digital signal $x^\prime[m]$ to the analog signal $x^\prime(t)$ which after a \gls{lp} filter resolves to $x(t)$.}
\end{figure}


\begin{figure}[htb]
	\centering
	\includegraphics{figures/pgfplots/tx-unit-time}
	\caption{Pulse-shaping steps in the time domain for a single unit symbol: A unit symbol sequence $\Re\alpha[n]$ (first row) is upsampled by two (second row) to yield the samples $\Re\alpha^\prime[m]$. The digital \gls{rrc} filter is applied to the samples $x^\prime[m]$ determining the pulse-shape (third row). We can observe the diminishing ripple for the unit response of the filter to reduce the bandwidth. Finally, the pulse-shape $x^\prime[m]$ is converted to an analog signal and filtered by a \gls{lp} filter for smoothing and anti-aliasing.}\label{fig:pulse_shaping_unit_time}
\end{figure}

\begin{equation}
	x[m]
	=
	x(t_m)
	=
	x(mT)
	,
\end{equation}

\begin{figure}[htb]
	\centering
	\includegraphics{figures/pgfplots/tx-rand-time}
	\caption{Pulse-shaping steps in the time domain for random symbols from a complex uniform distribution over the interval $[-1,+1]$. The first row shows the real (orange) and imaginary part (blue) of the complex symbols $\alpha[n]$ at their corresponding symbol index $n$. The second row shows the symbols after upsampling to $\alpha^\prime[m]$. The third row shows the samples after applying the \gls{rrc} filter $x^\prime[m]+ip^\prime[m]$. The fourth row shows the anti-aliased analog signal $x(t)+ip(t)$.}\label{fig:pulse_shaping_rand_time}
\end{figure}

\begin{equation}
	\alpha^\prime[m]
	=
	\begin{cases}
		\alpha[m/2] & \text{if}\ m\mod2=0 \\
		0 & \text{otherwise}
	\end{cases}
\end{equation}

\begin{equation}
	\abs{h_\text{rc}\left(\frac{f}{f_s}\right)}
	=
	\begin{cases}
		1 & \abs*{\frac{f}{f_s}}\leq(1-\alpha) \\
		\cos\left[\frac{\pi}{4\alpha}\left(\abs*{\frac{f}{f_s}}-1+\alpha\right)\right] & 1-\alpha\leq\abs*{\frac{f}{f_s}}\leq1+\alpha \\
		0 & \text{otherwise}
	\end{cases}
	.
\end{equation}

\begin{figure}[htb]
	\centering
	\includegraphics{figures/pgfplots/tx-frequency}
	\caption{Pulse-shaping steps in the frequency domain (showing the relative \gls{psd}) for unit symbol (orange) and random symbols from a complex uniform distribution over the interval $[-1,+1]$ (blue). The unit response symbols have a perfectly flat power spectrum (first row). The random symbols have an approximately flat power spectrum (first row). Both symbol spectra (first row) occupy the Nyquist bandwidth. Upsampling doubles the Nyquist bandwidth by aliasing the spectrum (second row). Pulse-shaping acts as a bandpass by strongly suppressing the frequency components outside the original Nyquist bandwidth. Conversion to an analog signal (fourth row) is equivalent to infinite upsampling or adding infinitely many aliases to occupy the complete frequency spectrum. Finally, filtering with a \gls{lp} removes aliases.}\label{fig:pulse_shaping_freq}
\end{figure}

\subsection{Up-conversion}

\begin{equation}
	\alpha(t)
	=
	x(t)
	+
	ip(t)
	.
\end{equation}

\begin{equation}
	x(t)
	\cos(\omega_ct)
	+
	p(t)
	\sin(\omega_ct)
	=
	\Re\left\{
		\alpha^\prime(t)
		e^{-i\omega_ct}
	\right\}
	\label{eq:passband_signal}.
\end{equation}

\begin{figure}[htb]
	\centering
	\caption{Transmit spectrum.}\label{fig:transmit_spectrum}
\end{figure}
\section{Transmitter}                                                                                                                                                                                                                                                                                                                                                                                                                                                                                                                                                                                                                                                                                                                     

We introduced the transmitter as a component encoding a sequence of complex symbols, $\left\{\alpha_n\in\mathbb{C}\colon n\in I\right\}$, onto a time-continuous coherent state $\ket{\alpha(t)}$.
% what else does a transmitter need to do? (further requirements regarding the transmitter)
% - modulation bandwidth in modulation
% - up-conversion for physical channel (fiber)

The symbol sequence alone has no notion of time.
Time enters through the symbol period $T$ denoting the time between two consecutive symbols.


\begin{figure}[htb]
	\centering
	\includegraphics{figures/circuitikz/transmitter-signal-processing}
	\caption{Block diagram of the transmitter's signal-processing. The real and imaginary part of a complex digital signal $\alpha[n]$ is upsampled, pulse-shaped and converted to anti-aliased analog signals, $x(t)$ and $p(t)$. The analog signals are individually mixed with a phase-shifted \gls{lo} with carrier frequency $\omega_c$ and then added to yield a complex signal $\alpha(t)$.}\label{fig:transmitter_signal_processing}
\end{figure}

% implementation concerns
% - software-defined
\begin{figure}[htb]
	\centering
	\includegraphics{figures/tikz/transmitter-signal-processing}
	\caption{Block diagram of the the transmitter's signal processing domains. The \gls{dsp} transforms a symbol sequence $\{\alpha_n\}_{n\in I}$ into digital signals (samples), $x^\prime[n]$ and $p^\prime[n]$, which the \gls{dac} converts to analog signals, $x^\prime(t)$ and $p^\prime(t)$.}\label{fig:transmitter_signal_processing_domains}
\end{figure}

\subsection{Signal synthesis}

\begin{figure}[htb]
	\centering
	\includegraphics{figures/pgfplots/tx-unit-time}
	\caption{Pulse-shaping steps in the time domain for a single unit symbol: A unit symbol sequence $\Re\alpha[n]$ (first row) is upsampled by two (second row) to yield the samples $\Re\alpha^\prime[m]$. The digital \gls{rrc} filter is applied to the samples $x^\prime[m]$ determining the pulse-shape (third row). We can observe the diminishing ripple for the unit response of the filter to reduce the bandwidth. Finally, the pulse-shape $x^\prime[m]$ is converted to an analog signal and filtered by a \gls{lp} filter for smoothing and anti-aliasing.}\label{fig:pulse_shaping_unit_time}
\end{figure}

\begin{equation}
	x[m]
	=
	x(t_m)
	=
	x(mT)
	,
\end{equation}

\begin{figure}[htb]
	\centering
	\includegraphics{figures/pgfplots/tx-rand-time}
	\caption{Pulse-shaping steps in the time domain for random symbols from a complex uniform distribution over the interval $[-1,+1]$. The first row shows the real (orange) and imaginary part (blue) of the complex symbols $\alpha[n]$ at their corresponding symbol index $n$. The second row shows the symbols after upsampling to $\alpha^\prime[m]$. The third row shows the samples after applying the \gls{rrc} filter $x^\prime[m]+ip^\prime[m]$. The fourth row shows the anti-aliased analog signal $x(t)+ip(t)$.}\label{fig:pulse_shaping_rand_time}
\end{figure}

\begin{equation}
	\alpha^\prime[m]
	=
	\begin{cases}
		\alpha[m/2] & \text{if}\ m\mod2=0 \\
		0 & \text{otherwise}
	\end{cases}
\end{equation}

\begin{equation}
	\abs{h_\text{rc}\left(\frac{f}{f_s}\right)}
	=
	\begin{cases}
		1 & \abs*{\frac{f}{f_s}}\leq(1-\alpha) \\
		\cos\left[\frac{\pi}{4\alpha}\left(\abs*{\frac{f}{f_s}}-1+\alpha\right)\right] & 1-\alpha\leq\abs*{\frac{f}{f_s}}\leq1+\alpha \\
		0 & \text{otherwise}
	\end{cases}
	.
\end{equation}

\begin{figure}[htb]
	\centering
	\includegraphics{figures/pgfplots/tx-frequency}
	\caption{Pulse-shaping steps in the frequency domain (showing the relative \gls{psd}) for unit symbol (orange) and random symbols from a complex uniform distribution over the interval $[-1,+1]$ (blue). The unit response symbols have a perfectly flat power spectrum (first row). The random symbols have an approximately flat power spectrum (first row). Both symbol spectra (first row) occupy the Nyquist bandwidth. Upsampling doubles the Nyquist bandwidth by aliasing the spectrum (second row). Pulse-shaping acts as a bandpass by strongly suppressing the frequency components outside the original Nyquist bandwidth. Conversion to an analog signal (fourth row) is equivalent to infinite upsampling or adding infinitely many aliases to occupy the complete frequency spectrum. Finally, filtering with a \gls{lp} removes aliases.}\label{fig:pulse_shaping_freq}
\end{figure}

\subsection{Up-conversion}

\begin{equation}
	\alpha(t)
	=
	x(t)
	+
	ip(t)
	.
\end{equation}

\begin{equation}
	x(t)
	\cos(\omega_ct)
	+
	p(t)
	\sin(\omega_ct)
	=
	\Re\left\{
		\alpha^\prime(t)
		e^{-i\omega_ct}
	\right\}
	\label{eq:passband_signal}.
\end{equation}

\begin{figure}[htb]
	\centering
	\caption{Transmit spectrum.}\label{fig:transmit_spectrum}
\end{figure}
\section{Transmitter}                                                                                                                                                                                                                                                                                                                                                                                                                                                                                                                                                                                                                                                                                                                     

We introduced the transmitter as a component encoding a sequence of complex symbols,
\begin{equation}
	\left\{
		\alpha_n\in\mathbb{C}
		\colon
		n\in I
	\right\}
	,
\end{equation}
onto a coherent state $\ket{\alpha(t)}$.
Efficient transmission through the channel and effective receiver detection impose additional constraints on the space of useful coherent states.
Together with practical considerations, these constraints lead to the particular design embodiment of the transmitter we will discuss.

First and foremost, the channel and receiver limit the spectrum of useful coherent states.
For instance, the receiver has limited bandwidth to detect the signal with signal power outside that bandwidth being lost.
Apart from that, the physical channel only shows favorable transmission properties over a certain frequency range, outside the signal is strongly suppressed and distorted.\footnote{For instance, the C-band, spanning wavelengths from \SI{1530}{\nano\meter} to \SI{1565}{\nano\meter}, is widely deployed for optical telecommunication.}
Additionally, different users may jointly use the same physical channel, and using the available bandwidth efficiently while reducing interference between users, requires the signal bandwidth to be well-defined.

While the symbol rate effectively defines the bandwidth, insufficient usage of the bandwidth degrades the \gls{snr}.
In addition, we need to shift the baseband spectrum to an optical frequency $\omega_0$ for which the channel shows desirable transmission characteristics.
So from a signal-processing point of view, we want the transmitter to
\begin{enumerate}
	\item first create a baseband signal with well-defined spectrum,\footnote{We consider a spectrum well-defined if bandwidth-limited and compatible with the Nyquist criterion, which requires the signal bandwidth $B$ to equal at least half the symbol rate $2B>f_s$.}
	\item then transfer it to a passband signal in the optical domain.
\end{enumerate}
In the following, we term the first step symbol encoding and the second step upconversion.

In the present setup, the signal is almost exclusively constructed in the digital domain, and the analog part is limited to the digital-to-analog conversion.
Constructing the signal digitally allows for greater flexibility in the development process is mostly software-defined.
For upconversion of the analog signal to the optical domain, we modulate the electric signal onto an optical carrier.
\begin{figure}[htb]
	\centering
	\includegraphics{figures/tikz/software-defined-transmitter}
	\caption{Block diagram of the transmitter's signal-processing domains. The \gls{dsp} transforms a complex symbol sequence $\{\alpha_n\}_{n\in I}$ into two digital signals, $x^\prime[m]$ and $p^\prime[m]$, corresponding to the real and imaginary part. The \gls{dac} converts the digital signals to analog signals, $x(t)$ and $p(t)$ we modulate onto a coherent state $\ket{\alpha(t)}$.}\label{fig:software_defined_transmitter}
\end{figure}
\Cref{fig:software_defined_transmitter} illustrates how such a software-defined transmitter architecture applies to our coherent-state transmission system.
The software-defined \gls{dsp} constructs the bandwidth-optimized digital signals $x[m]$ and $p[m]$, encoding the real and imaginary parts of the complex symbols.
The \gls{dac} stage converts the digital signals, $x[m]$ and $p[m]$, to bandwidth-limited analog signals $x(t)$ and $p(t)$.
Finally, the analog signals are modulated onto an optical carrier yielding a coherent state $\ket{\alpha(t)}$ with well-defined spectrum.

\subsection{Symbol encoding}

To construct a bandwidth-optimized baseband signal, encoding the complex symbol sequence $\{\alpha_n\in\mathbb{C}\colon n\in I\}$, we first remark that the symbol sequence itself has no notion of time.
In contrast, a digital (time-discrete) signal, consisting of discrete samples, includes a time reference, the sample period $T_s$, denoting the temporal distance between two consecutive samples.
By defining the digital signal with samples equal to the symbols, $\alpha[n]=\alpha_n$, and introducing the symbol period $T_s$ as sample period, we find ourselves with the complete \gls{dsp} toolbox at our disposal.\footnote{Even if the \gls{dsp} itself does not work explicitly with the time reference $T_s$, we need time to give a meaningful interpretation of the signal between the steps.}
\begin{figure}[htb]
	\centering
	\includegraphics{figures/circuitikz/symbol-encoding}
	\caption{Block diagram of the signal processing for the symbol encoding. The digital signal $x[km]$ is upsampled by a factor $k$ to $r[m]$ and pulse-shaped by a \gls{rrc} filter to yield $q[m]$. A \gls{dac} converts the pulse-shaped signal $q[m]$ to the analog signal $q(t)$. Finally, the analog signal $q(t)$ is \gls{lp} filtered to yield the analog anti-aliased signal $x(t)$.}\label{fig:symbol_encoding}
\end{figure}
\Cref{fig:symbol_encoding} summarizes the essential \gls{dsp} steps including analog conversion of a real-valued digital signal $x[km]$ to construct a bandwidth-optimized analog baseband signal $x(t)$.\footnote{The baseband construction generalizes to a complex digital signal by applying the real-valued baseband construction separately to the real and imaginary part.}
The digital signal $x[km]$, containing the symbols, is first upsampled by an upsampling factor of $k$, adding $k$ zero-valued samples in between the original samples.
The pulse-shaping of the \gls{rrc} filter interpolates between the non-zero samples, the symbols, to precisely define the signal bandwidth.
Finally, an ideal \gls{dac} converts the digital signal $q[m]$ to the analog signal $q(t)$, equivalent to infinite upsampling.
The analog signal $q(t)$ contains infinite aliases through the upsampling, which we remove by filtering $q(t)$ with a \gls{lp}, yielding the anti-aliased analog signal $x(t)$.
\begin{figure}[htb]
	\centering
	\includegraphics{figures/pgfplots/symbol-encoding-time-unit}
	\caption{Symbol encoding for a single unit symbol in the time domain. The symbol sequence $\{x_n\in\mathbb{R}\colon n\in I\}$ is represented by the digital signal $x[km]$ with sample period $T_s$ (first row). The digital signal $x[km]$ is upsampled to $r[m]$ by an upsampling factor of $k=2$ (second row). The upsampled signal $r[m]$ is pulse-shaped with a \gls{rrc} filter to yield $q[m]$ (third row). The pulse-shaped digital signal $q[m]$ is converted to the anti-aliased analog signal $x(t)$.}\label{fig:baseband_construction_unit_time}
\end{figure}
\Cref{fig:baseband_construction_unit_time} illustrates the time domain signals for each signal-processing step for a symbol sequence which contains only a single non-zero symbol with unit value.
We see very well how the upsampling increases the resolution of the digital signal in the time domain and how the pulse-shaping filter interpolates between the samples.
We also see that the \gls{rrc} pulse-shaping filter corresponds to a $\sinc$-like impulse response.
The similarity of the analog signal with a $\sinc$ pulse is not surprising since the \gls{rrc} is the square-root of the raised-cosine filter.
The raised-cosine filter has frequency response
\begin{equation}
	\abs{h_\text{rc}\left(f/f_s\right)}
	=
	\begin{cases}
		1 & \abs*{f/f_s}\leq(1-\alpha) \\
		\cos\left[\frac{\pi}{4\alpha}\left(\abs*{f/f_s}-1+\alpha\right)\right] & 1-\alpha\leq\abs*{f/f_s}\leq1+\alpha \\
		0 & \text{otherwise}
	\end{cases}
	,
\end{equation}
wherein $f_s=1/T_s$ is the symbol rate and $\alpha$ determines the roll-off and satisfies the Nyquist criterion for optimal bandwidth~\cite[p.~51]{Madhow2008}.
Taking the square-root of the raised-cosine pulse-shaping filter and applying it once on the transmitter-side and once on the receiver-side, where it is named the matched filter, also satisfies the Nyquist criteria but suppresses out-of-band noise.\footnote{From a mathematical point of view, the matched filter is only asymptotically ideal for close-to-zero \gls{snr}. For sufficiently bad \gls{snr} the matched filter is almost as good as a Wiener filter.}
\begin{figure}[htb]
	\centering
	\includegraphics{figures/pgfplots/symbol-encoding-time-qpsk}
	\caption{Symbol encoding for a random \gls{qpsk} symbol-sequence in the time domain with real (orange) and imaginary (green) part . The complex symbol sequence $\{\alpha_n\in\mathbb{C}\colon n\in I\}$ is represented by the digital signal $\alpha[km]$ with sample period $T_s$ (first row). The digital signal $\alpha[km]$ is upsampled to $\rho[m]$ by an upsampling factor of $k=2$ (second row). The upsampled signal $\rho[m]$ is pulse-shaped with a \gls{rrc} filter to yield $\gamma[m]$ (third row). The pulse-shaped digital signal $\gamma[m]$ is converted to the anti-aliased analog signal $\alpha(t)$.}\label{fig:baseband_construction_rand_time}
\end{figure}
\Cref{fig:baseband_construction_rand_time} illustrates the time domain signals for each signal-processing step for a random \gls{qpsk} symbol sequence.
\begin{figure}[htb]
	\centering
	\includegraphics{figures/pgfplots/symbol-encoding-frequency}
	\caption{Power spectrum of the symbol encoding steps for a random \gls{qpsk} symbol sequence (green) and the unit symbol sequence (orange). The initial spectrum spans from $-1/2$ to $+1/2$ the normalized sampling frequency $f/f_s$ (first row). Upsampling by $k=2$ adds aliases left and right to the initial spectrum (second row). Pulse-shaping suppresses the left and ride flanks of the spectrum to precisely define the spectrum (third row). Analog conversion corresponds to infinite upsampling, adding infinite aliases left and right of the spectrum. (fourth row). Applying a \gls{lp} filter strongly suppresses the aliases with relaxed requirements on the filter spectrum (last row).}\label{fig:baseband_construction_freq}
\end{figure}
\Cref{fig:baseband_construction_freq} provides further inside into the signal-processing steps by presenting the power spectrum of the unit and \gls{qpsk} symbol sequences.
In the frequency domain, it is very clear to see how upsampling widens the spectrum without adding additional information.
We also see how the pulse-shaping filter shapes the upsampled spectrum, and the \gls{lp} filter suppresses aliases.

\FloatBarrier
\subsection{Upconversion}

We previously constructed the pulse-shaped baseband signals $x(t)$ and $p(t)$ encoding the real respective imaginary part of a complex symbol sequence $\{\alpha_n\in\mathbb{C}\colon n\in I\}$.
Modulating the signals onto the optical carrier frequency $\omega_c$ resembles an upconversion by the frequency $\omega_c$, corresponding to the multiplication with an \gls{lo} signal, as illustrated in \Cref{fig:upconversion_single}.
\begin{figure}[htb]
	\centering
	\includegraphics{figures/circuitikz/upconversion-single}
	\caption{Block diagram of single-quadrature upconversion. The signal $x(t)$ is mixed with the \gls{lo} signal $\cos(\omega_lt+\vartheta)$ and the output is filtered by a bandpass to remove harmonics.}\label{fig:upconversion_single}
\end{figure}
In the frequency representation, we find that the multiplication of the signal $x(t)$ with the \gls{lo} signal $\cos(\omega_ct+\vartheta)$ creates two copies of the spectrum $x(\omega)$, at the positive and negative upconversion-frequency $\pm\omega_c$, as illustrated in \Cref{fig:spectrum_upconversion_single}.
\begin{figure}[htb]
	\centering
	\includegraphics{figures/tikz/spectrum-upconversion-single}
	\caption{Power spectrum illustrating upconversion of the real-valued signal $x(t)$ centered at zero frequency $\omega=0$ (solid spectrum) to the carrier frequencies $\pm\omega_c$ (dashed spectra).}\label{fig:spectrum_upconversion_single}
\end{figure}
Because \Cref{fig:spectrum_upconversion_single} shows the power spectrum, it fails to convey that the spectrum at the negative frequencies is complex conjugate, $x(-\omega)=x(\omega)^*$, with respect to the positive frequencies, as required for real-valued signals, i.e.,
\begin{equation}
	\begin{split}
		x(t)
		\cos(\omega_ct+\vartheta)
		&=
		\int_{-B_s/2}^{+B_s/2}
		\frac{\dd{\omega}}{2\pi}
		x(\omega)
		e^{+i\omega t}
		\cos(\omega_ct+\vartheta)
		\\
		&=
		\frac{1}{2}
		\int_{-B_s/2}^{+B_s/2}
		\frac{\dd{\omega}}{2\pi}
		x(\omega)
		\left[
			e^{+i(\omega+\omega_c)t+i\vartheta}
			+
			e^{+i(\omega-\omega_c)t-i\vartheta}
		\right]
		\\
		&=
		\frac{1}{2}
		\int_{+\omega_c-B_s/2}^{+\omega_c+B_s/2}
		\frac{\dd{\omega}}{2\pi}
		x(\omega-\omega_c)
		e^{+i\omega t+i\vartheta}
		+
		\frac{1}{2}
		\int_{-\omega_c-B_s/2}^{-\omega_c+B_s/2}
		\frac{\dd{\omega}}{2\pi}
		x(\omega+\omega_c)
		e^{+i\omega t-i\vartheta}
		\\
		&=
		\frac{1}{2}
		\int_{+\omega_c-B_s/2}^{+\omega_c+B_s/2}
		\frac{\dd{\omega}}{2\pi}
		x(\omega-\omega_c)
		e^{+i\omega t+i\vartheta}
		-
		\frac{1}{2}
		\int_{+\omega_c+B_s/2}^{+\omega_c-B_s/2}
		\frac{\dd{\omega}}{2\pi}
		x(-\omega+\omega_c)
		e^{-i\omega t-i\vartheta}
		\\
		&=
		\frac{1}{2}
		\int_{+\omega_c-B_s/2}^{+\omega_c+B_s/2}
		\frac{\dd{\omega}}{2\pi}
		x(\omega-\omega_c)
		e^{+i\omega t+i\vartheta}
		+
		\frac{1}{2}
		\int_{+\omega_c-B_s/2}^{+\omega_c+B_s/2}
		\frac{\dd{\omega}}{2\pi}
		x(\omega-\omega_c)^*
		e^{-i\omega t-i\vartheta}
		\\
		&=
		\Re\int_{+\omega_c-B_s/2}^{+\omega_c+B_s/2}
		\frac{\dd{\omega}}{2\pi}
		x(\omega-\omega_c)
		e^{+i\omega t+i\vartheta}
		,
	\end{split}
	\label{eq:upconversion_real}
\end{equation}
wherein we used the conjugate symmetry of the spectrum in the last step.

\Cref{fig:upconversion_dual} shows how to extend the previously-discussed single-quadrature upconversion to dual-quadrature upconversion by splitting a single \gls{lo} with relative phase shift of $\pi/2$.
\begin{figure}[htb]
	\centering
	\includegraphics{figures/circuitikz/upconversion-dual}
	\caption{Block diagram illustrating upconversion of two real-valued baseband signals, $x(t)$ and $p(t)$, to a real-valued passband signal $s(t)$. The \gls{lo} signal $\sqrt{2}\cos(\omega_ct+\vartheta)$ is split into two branches with a relative phase shift between the branches of $\pi/2$. One branch is mixed with the baseband signal $x(t)$, the other is mixed with $p(t)$. The output of each mixer is filtered to remove harmonics and then added to yield the upconverted signal $s(t)$.}\label{fig:upconversion_dual}
\end{figure}
The filtered output of the lower branch mixer is obtained by replacing $\vartheta\to\vartheta+\pi/2$ and $x(t)\to p(t)$ in \cref{eq:upconversion_real}, i.e.,
\begin{equation}
	p(t)
	\sin(\omega_ct+\vartheta)
	=
	\Im
	\int_{+\omega_c-B_s/2}^{+\omega_c+B_s/2}
	\frac{\dd{\omega}}{2\pi}
	p(\omega-\omega_c)
	e^{+i\omega t+i\vartheta}
	.
	\label{eq:upconversion_real}
\end{equation}
The sum of both branches in \Cref{fig:upconversion_dual} equals
\begin{equation}
	\begin{split}
		s(t)
		&=
		x(t)
		\cos(\omega_ct+\vartheta)
		-
		p(t)
		\sin(\omega_ct+\vartheta)
		\\
		&=
		\Re
		\int_{+\omega_c-B_s/2}^{+\omega_c+B_s/2}
		\frac{\dd{\omega}}{2\pi}
		x(\omega-\omega_c)
		e^{+i\omega t+i\vartheta}
		-
		\Im
		\int_{+\omega_c-B_s/2}^{+\omega_c+B_s/2}
		\frac{\dd{\omega}}{2\pi}
		p(\omega-\omega_c)
		e^{+i\omega t+i\vartheta}
		\\
		&=
		\Re
		\int_{+\omega_c-B_s/2}^{+\omega_c+B_s/2}
		\frac{\dd{\omega}}{2\pi}
		\left[
			x(\omega-\omega_c)
			+
			ip(\omega-\omega_c)
		\right]
		e^{+i\omega t+i\vartheta}
		,
	\end{split}
	\label{eq:upconversion_dual}
\end{equation}
where we used $\Im(z)=-\Re(iz)$ in the last step.
\Cref{eq:upconversion_dual} suggests defining the complex-valued baseband signal
\begin{equation}
	\alpha(t)
	=
	x(t)
	+
	ip(t)
	\label{eq:upconversion_baseband}
\end{equation}
for which we can show that the dual-quadrature upconversion equals the multiplication with a complex exponential, i.e.,
\begin{equation}
	s(t)
	=
	\Re
	\left[
		\alpha(t)
		e^{+i(\omega_c t+\vartheta)}
	\right]
	,
\end{equation}
demonstrating equivalence between the baseband and passband representations.
\begin{figure}[htb]
	\centering
	\includegraphics{figures/tikz/spectrum-upconversion-dual}
	\caption{Power spectrum illustrating dual-quadrature upconversion of the complex-valued signal $\alpha(t)$ centered at zero frequency $\omega=0$ (solid spectrum) to the carrier frequencies $\pm\omega_c$ (dashed spectra).}\label{fig:spectrum_upconversion_dual}
\end{figure}
Compared to the power spectrum illustrating the single-quadrature upconversion in \Cref{fig:upconversion_dual}, the power spectras concentrated are now asymmetric around their respective carrier frequencies, $\omega=0,\pm\omega_c$.
However, the upconverted spectrum including the positive and negative frequencies remains to have complex conjugate symmetry, as depicted in \Cref{fig:spectrum_upconversion_dual}.

Using the complex baseband representation, \cref{eq:upconversion_baseband}, it is simple to link the dual-quadrature upconversion to our result of the electro-optical \gls{iqm}.
In particular, if we assume a narrow-linewidth \gls{lo} at frequency $\omega_c$, represented by the coherent state
\begin{equation}
	\ket{g(t)e^{+i\omega_ct}}
\end{equation}
with $g(t)$ encoding the laser profile, the unitary evolution operator associated with the \gls{iqm} acting on the \gls{lo} coherent state produces,
\begin{equation}
	\hat{U}_\text{IQM}[\alpha(t)]
	\ket{g(t)e^{+i\omega_ct}}
	=
	\ket{(g\alpha)(t)e^{+i\omega_ct}}
	,
\end{equation}
the upconversion from the electrical to the quantum-optical domain.
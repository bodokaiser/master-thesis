\section{Transmitter}                                                                                                                                                                                                                                                                                                                                                                                                                                                                                                                                                                                                                                                                                                                     

We introduced the transmitter as a component encoding a sequence of complex symbols, $\left\{\alpha_n\in\mathbb{C}\colon n\in I\right\}$, onto a time-continuous coherent state $\ket{\alpha(t)}$, which we submit through the channel to the receiver.
Usually, a physical channel is shard by multiple users and only shows favorable transmission properties over a certain frequency range, outside the signal is strongly suppressed and distorted.\footnote{For instance, telecommunications widely deploys the C-band, spanning wavelengths from \SI{1530}{\nano\meter} to \SI{1565}{\nano\meter}, for optical data transmission.}
To use the available bandwidth efficiently and reduce interference between users, one should minimize the signal bandwidth.
Typically, the transmitter meets these two goals by
\begin{enumerate}
	\item encoding the symbols in a time-continuous signal $\alpha(t)$ with minimal bandwidth (signal synthesis), and
	\item modulating the signal $\alpha(t)$ onto a carrier in the frequency range where the channel demonstrates favorable transmission characteristics (up-conversion).
\end{enumerate}
Nowadays, the signal is almost exclusively constructed in the digital domain, and the analog part is limited to the digital-to-analog conversion.
Constructing the signal digitally allows for greater flexibility in the development process as the synthesis is mostly software-defined.
For up-conversion of the synthesized signal to the optical domain, we modulate the electric signal onto an optical carrier.
\begin{figure}[htb]
	\centering
	\includegraphics{figures/tikz/transmitter-signal-processing}
	\caption{Block diagram of the the transmitter's signal processing domains. The \gls{dsp} transforms a complex symbol sequence $\{\alpha_n\}_{n\in I}$ into two digital signals, $x^\prime[m]$ and $p^\prime[m]$, corresponding to the real and imaginary part. The \gls{dac} converts the digital signals to analog signals, $x(t)$ and $p(t)$ we modulate onto a coherent state $\ket{\alpha(t)}$.}\label{fig:transmitter_signal_processing_domains}
\end{figure}
\Cref{fig:transmitter_signal_processing_domains} illustrates how such a software-defined transmitter architecture applies to our coherent state transmission system.
The software-defined \gls{dsp} constructs the bandwidth-optimized digital signals $x[m]$ and $p[m]$, encoding the real and imaginary parts of the complex symbols.
The \gls{dac} stage converts the digital signals, $x[m]$ and $p[m]$, to bandwidth-limited analog signals $x(t)$ and $p(t)$.
Finally, the analog signals are modulated onto an optical carrier yielding a coherent state $\ket{\alpha(t)}$ which meets the bandwidth requirements of the channel.

We now take a closer look at how first to encode a symbol sequence in a bandwidth-optimized signal and then transfer it to an optical coherent state.

\subsection{Signal synthesis}

The symbol sequence alone has no notion of time.
Time enters through the symbol period $T$ denoting the time between two consecutive symbols.

\begin{figure}[htb]
	\centering
	\includegraphics{figures/pgfplots/tx-unit-time}
	\caption{Pulse-shaping steps in the time domain for a single unit symbol: A unit symbol sequence $\Re\alpha[n]$ (first row) is upsampled by two (second row) to yield the samples $\Re\alpha^\prime[m]$. The digital \gls{rrc} filter is applied to the samples $x^\prime[m]$ determining the pulse-shape (third row). We can observe the diminishing ripple for the unit response of the filter to reduce the bandwidth. Finally, the pulse-shape $x^\prime[m]$ is converted to an analog signal and filtered by a \gls{lp} filter for smoothing and anti-aliasing.}\label{fig:pulse_shaping_unit_time}
\end{figure}

\begin{equation}
	x[m]
	=
	x(t_m)
	=
	x(mT)
	,
\end{equation}

\begin{figure}[htb]
	\centering
	\includegraphics{figures/pgfplots/tx-rand-time}
	\caption{Pulse-shaping steps in the time domain for random symbols from a complex uniform distribution over the interval $[-1,+1]$. The first row shows the real (orange) and imaginary part (blue) of the complex symbols $\alpha[n]$ at their corresponding symbol index $n$. The second row shows the symbols after upsampling to $\alpha^\prime[m]$. The third row shows the samples after applying the \gls{rrc} filter $x^\prime[m]+ip^\prime[m]$. The fourth row shows the anti-aliased analog signal $x(t)+ip(t)$.}\label{fig:pulse_shaping_rand_time}
\end{figure}

\begin{equation}
	\alpha^\prime[m]
	=
	\begin{cases}
		\alpha[m/2] & \text{if}\ m\mod2=0 \\
		0 & \text{otherwise}
	\end{cases}
\end{equation}

\begin{equation}
	\abs{h_\text{rc}\left(\frac{f}{f_s}\right)}
	=
	\begin{cases}
		1 & \abs*{\frac{f}{f_s}}\leq(1-\alpha) \\
		\cos\left[\frac{\pi}{4\alpha}\left(\abs*{\frac{f}{f_s}}-1+\alpha\right)\right] & 1-\alpha\leq\abs*{\frac{f}{f_s}}\leq1+\alpha \\
		0 & \text{otherwise}
	\end{cases}
	.
\end{equation}

\begin{figure}[htb]
	\centering
	\includegraphics{figures/pgfplots/tx-frequency}
	\caption{Pulse-shaping steps in the frequency domain (showing the relative \gls{psd}) for unit symbol (orange) and random symbols from a complex uniform distribution over the interval $[-1,+1]$ (blue). The unit response symbols have a perfectly flat power spectrum (first row). The random symbols have an approximately flat power spectrum (first row). Both symbol spectra (first row) occupy the Nyquist bandwidth. Upsampling doubles the Nyquist bandwidth by aliasing the spectrum (second row). Pulse-shaping acts as a bandpass by strongly suppressing the frequency components outside the original Nyquist bandwidth. Conversion to an analog signal (fourth row) is equivalent to infinite upsampling or adding infinitely many aliases to occupy the complete frequency spectrum. Finally, filtering with a \gls{lp} removes aliases.}\label{fig:pulse_shaping_freq}
\end{figure}

\subsection{Up-conversion}

\begin{equation}
	\alpha(t)
	=
	x(t)
	+
	ip(t)
	.
\end{equation}

\begin{equation}
	x(t)
	\cos(\omega_ct)
	+
	p(t)
	\sin(\omega_ct)
	=
	\Re\left\{
		\alpha^\prime(t)
		e^{-i\omega_ct}
	\right\}
	\label{eq:passband_signal}.
\end{equation}

\begin{figure}[htb]
	\centering
	\caption{Transmit spectrum.}\label{fig:transmit_spectrum}
\end{figure}
\section{Classical and quantum transmission system}

A classical transmission system attempts to reproduce information from a spacetime event $x$, the information source, at a spacetime event $y$, the information destination, wherein $y$ is in the forward light cone of $x$.\footnote{Contemporary to the literature, we distinguish between a transmission and a communication system. The former allows only unidirectional, the later bidirectional transport of information.}
\begin{figure}[htb]
	\centering
	\includegraphics{figures/tikz/transmission-system}
	\caption{Block diagram of a general transmission system according to Ref.~\cite{Shannon1948}. An information source produces information that a transmitter encodes into a signal and transmits it through a channel to the receiver. The channel adds noise from a noise source to the transmitted signal. The receiver recovers the information from the received signal and passes it to the intended information destination.}\label{fig:transmission_system}
\end{figure}
To transport information from the information source and the information destination, the transmission system uses a transmitter-receiver pair connected by a physical channel (\Cref{fig:transmission_system}).
The task of the transmitter is to encode the information in a physical signal for efficient transmission over the channel.
In its most general form, the channel is a mapping from the transmitted to the received signal.
The received signal may contain additional information, not of interest to the information destination, which we refer to as noise.
The receiver listens to the channel and attempts to recover the transmitted from the received signal for the information destination.

How to generalize the classical transmission system to a quantum transmission system?
First of all, it appears natural to define the channel as the distinguishing feature between a classical and quantum transmission system: While a classical channel maps a classical signal, we expect a quantum channel to map a quantum state.
Secondly, we need to address the inherent uncertainty associated with quantum measurements.
In principle, we could remove the quantum uncertainty by demanding the transmitter and receiver use the same basis and the quantum channel preserving orthogonality.
However, such a requirement would limit the capabilities of the quantum transmission system to the classical analog and be an unrealistic demand for realistic channels.
Finally, we are left with the information source, which can be either classical or quantum, depending on if we want to encode classical information onto a quantum state or if we want to transmit quantum states themselves for further processing, e.g., computation or sensing.
To conclude, we propose to define a quantum transmission system as a system that attempts to correlate a classical or quantum information destination at a spacetime event $y$ with an information source at a spacetime event $x$, wherein $y$ is in the forward light cone of $x$.
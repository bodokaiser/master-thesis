\section{Channel}

Based on the type of information source and the channel, we distinguish between different transmission systems.
For instance, the information source may produce information in a continuous or discrete value range.
It is impossible to transmit continuous-valued information without error~\cite[p.~47]{Shannon1948}, so communication typically considers information in the form of a finite set of symbols, the alphabet.
Regarding the channel, one typically assumes the channel to be a linear map $T$ with additive noise $n$ relating the transmitted signal $x$ with the received signal $y$ via
\begin{equation}
	y
	=
	Tx
	+
	n
	\label{eq:classic_channel}
	.
\end{equation}
Because of the linearity of the channel, the noise term $n$ can summarize any kind of (independent) noise, including transmitter noise or electronic noise from the receiver.
For a quantum channel, i.e., a channel mapping quantum states, a simple map as in \cref{eq:channel_linear}

% noise part of the quantum state -> assume noiseless channel
% logical and physical channel for coherent states (symbols are complex numbers)
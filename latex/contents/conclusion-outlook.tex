\chapter*{Conclusion and outlook}
\addcontentsline{toc}{chapter}{Conclusion and outlook}

% answer to the research question
This thesis aimed to outline the incorporation of quantum aspects into optical communication.
By crafting a continuous-mode theory of light and applying it to describe electro-optical components essential for optical communication, we proposed a coherent state transmission system as an extension of a classical optical transmission system to the quantum regime.
The suggested coherent state transmission system offers a platform to explore and understand novel communication protocols inspired by classical communication and quantum mechanics, especially practical \gls{qkd} protocols based on weak coherent states.

% reflection on the methodology
Away from the practical results of our work, we illuminate quantum aspects of light that have physical relevance but otherwise fall short, as our result on the generalized quadrature measurement shows.
On a meta-level, we can draw two key lessons from our work.
On the one hand, we need to be more careful in employing reductionism.
Some concepts are notoriously difficult to grasp, and there exists no shortcut to understanding them truly.
On the other hand, we need to be more encouraged to search for answers outside our traditional domain.
In deriving a continuous-mode quantum theory of light from quantum field theory, we consciously opted against extending single-mode quantum optics, which turned out to be a key factor in our research's success as there is more clarity in simplifying a complete theory as opposed to extending a simplified theory.
Of course, this does not mean disavowing established methods.
Rather it proves useful to switch perspectives from time to time.
For example, towards the end of our research, it turned out to be favorable to move away from a constructive bottom-up approach and work our way backward from the classical results.

% contribution to the literature
Our work is difficult to place into the existing literature, largely since our problem statement emerges from an applied industry-related setting.
The closest to our research is Shapiro's work regarding a quantum theory of optical communication~\cite{Shapiro2009}.
However, besides the photodetection, Shapiro approaches the topic from a quantum information perspective and does neither consider a practical implementations or a transmission setup.
Concerning the quantum theory of light, we are the first to our knowledge to present a continuous-mode theory rooted in quantum field theory and emphasizing the communication aspects.
The few existing references~\cite{Barnett2002,Loudon2000} do not attempt to justify their results but apply their formalism towards quantum optics.
For the quantum theory of electro-optical components, we have summarized and unified the existing literature.
Most importantly to mention here are the beam splitter~\cite{Haroche2006,Leonhardt2003,Vogel2006}, the photodetector~\cite{Vogel2006,Mandel1995,Shapiro2009}, and the phase modulator~\cite{Horoshko2018,QuesadaMejia2015}.

% limitation of our work
Some topics, e.g., quantum information and non-classical quantum states, require additional attention to complete a theoretical framework for quantum optical communication not limited to practical \gls{qkd} using weak coherent states.
In addition, our description of the coherent state transmission system misses a discussion of the quantum statistics and effects of a non-ideal quantum channel.

% outlook and recommendations
As many open questions our work leaves, as many opportunities it provides, the most obvious being the transfer of classical communication protocols to \gls{qkd} like, for example, \gls{ofdm}~\cite{Bahrani2015}, which requires a continuous-mode theory of light.
Another potential research direction concerns security proofs for practical \gls{qkd} protocols based on weak coherent states.
In particular, it would be interesting to follow up on the concept of a logical quantum channel investigating the equivalence of a tensor product coherent states transmission system with our time-continuous coherent state transmission system.
Such an equivalence, if confirmed, could be a useful tool to simplify existing and future security proofs.
Last but not least, it would be interesting to extend our theoretical framework to the transmission of non-classical quantum states and discuss if and how one can make sense of a communication system replicating quantum information.
One promising direction, which would benefit from our framework, is the transmission of frequency-entangled squeezed states, also known as broadband squeezed states~\cite{Vogel2006,Mandel1995}.
\chapter*{Conclusion and outlook}
\addcontentsline{toc}{chapter}{Conclusion and outlook}

% answer to the research question
This thesis aimed to outline the incorporation of quantum aspects into optical communication.
By crafting a continuous-mode theory of light and applying it to describe electro-optical components essential for optical communication, we proposed a coherent state transmission system as an extension of a classical optical transmission system to the quantum regime.
The suggested coherent state transmission system offers a platform to explore and understand novel communication protocols inspired by classical communication and quantum mechanics, especially practical \gls{qkd} protocols based on weak coherent states.

% reflection on the methodology
Away from the practical results of our work, we illuminate quantum aspects of light that have physical relevance but otherwise fall short, as our result on the generalized quadrature measurement shows.
On a meta-level, we can draw two key lessons from our work.
On the one hand, we need to be more careful in employing reductionism.
Some concepts are notoriously difficult to grasp, and there exists no shortcut to understanding them truly.
On the other hand, we need to be more encouraged to search for answers outside our traditional domain.
In deriving a continuous-mode quantum theory of light from quantum field theory, we consciously opted against extending single-mode quantum optics, which turned out to be a key factor in our research's success as there is less uncertainty in simplifying a complete theory than extending a simplified theory.
We had similar experiences in following a bottom-up as opposed to a top-down approach in tackling our research problem.
Of course, we should not banish established approaches, but rather, we should consider from time to time a change of perspective.
For example, switching from a bottom-up to a top-down perspective was essential to overcome difficulties in the final stages of our research.

% contribution to the literature
Our thesis fills multiple gaps in the previous literature.
First and foremost, we present a continuous-mode quantum theory of light, which, unlike the few existing references~\cite{Barnett2002,Loudon2000}, roots in quantum field theory.
Second, we summarize the existing quantum theory of electro-optical components, most importantly of the photodetector~\cite{Vogel2006,Mandel1995,Shapiro2009} and the phase modulator~\cite{Horoshko2018}, and present it in the context of a coherent state transmission system encoding classical information.
Third, we provide a compact but abstract introduction to \gls{qkd} towards communication engineering.

% limitation of our work
There are a lot of open questions and topics, which we have not properly addressed, to claim a complete theoretical framework for quantum optical communication, the strongest limitation being our focus on the most classical quantum states, the coherent states, and only considering the transmission of classical communication.
Concerning our description of the coherent state transmission system, we did not sufficiently discuss the statistical properties of the photodetection signals and used the strong assumption of an ideal quantum channel.
From a theoretical point of view, one might challenge if the decision to assume most of the signal-processing in the electrical domain.

% outlook and recommendations
% - security proofs with continuous fields (equivalence of tensor product of (single-mode) coherent states with time-continuous coherent state)
% - new protocols inspired from telecommunications (odfm)
% - frequency-entangled states (broadband squeezing)
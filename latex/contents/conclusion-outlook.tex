\chapter*{Conclusion and outlook}
\addcontentsline{toc}{chapter}{Conclusion and outlook}

This thesis aimed to outline how to incorporate quantum aspects into optical communication.
By crafting a continuous-mode theory of light and applying it to describe electro-optical components used in optical communication, we propose a quantum communication system based on the transmission of coherent states as an extension of a classical optical communication system to the quantum regime.
The proposed coherent state transmission offers a platform to discover novel communication protocols inspired by advanced classical communication and quantum mechanics.

% methodology

By deriving a continuous-mode quantum theory of light, applying it to describe electro-optical components, we extend optical communication to the quantum regime.

% contribution

While our quantum theory of light is largely compatible with continuous-mode quantum optics, it is the first to be explicitly derived from quantum field theory towards quantum optical communication.
Along the same lines, we are the first to consider existing quantum models of electro-optical components directly in the broader context of a communication system.

% outlook
% - security proofs with continuous fields (equivalence of tensor product of (single-mode) coherent states with time-continuous coherent state)
% - new protocols inspired from telecommunications (odfm)
% - frequency-entangled states (broadband squeezing)
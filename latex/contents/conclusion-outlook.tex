\chapter*{Conclusion and outlook}
\addcontentsline{toc}{chapter}{Conclusion and outlook}

% result
This thesis aimed to outline how to incorporate quantum aspects into optical communication.
By crafting a continuous-mode theory of light and applying it to describe electro-optical components used in optical communication, we propose a quantum communication system based on the transmission of coherent states as an extension of a classical optical communication system to the quantum regime.

% significance
The proposed coherent state transmission offers a platform to discover novel communication protocols inspired by classical communication and quantum mechanics.

% methodology
In deriving a continuous-mode quantum theory of light from quantum field theory, we consciously opted against extending single-mode quantum optics towards a theoretical framework for quantum optical communication.
The driving hypothesis for going with the bottom-up as opposed to the top-down approach was that there is less uncertainty in simplifying a complete theory than speculating how to extend a simple theory.
In the course of the work, the bottom-up approach turned out to be largely correct and advantageous.
At the same time, changing perspective and considering an issue top-down helped us overcome difficulties in the final parts of our thesis.

% contribution
Our thesis fills multiple gaps in the previous literature.
First and foremost, we present a continuous-mode quantum theory of light, which, unlike the few existing references~\cite{Barnett2002,Loudon2000}, roots in quantum field theory.
Second, we summarize the existing quantum theory of electro-optical components, most importantly of the photodetector~\cite{Vogel2006,Mandel1995,Shapiro2009} and the phase modulator~\cite{Horoshko2018}, and present it in the context of a coherent state transmission system encoding classical information.
Third, we provide a compact but abstract introduction to \gls{qkd} towards communication engineering.

% limitation
There are a lot of open questions and topics, which we have not properly addressed, to claim a complete theoretical framework for quantum optical communication, the strongest limitation being our focus on the most classical quantum states, the coherent states, and only considering the transmission of classical communication.
Concerning our description of the coherent state transmission system, we did not sufficiently discuss the statistical properties of the photodetection signals and used the strong assumption of an ideal quantum channel.
From a theoretical point of view, one might challenge if the decision to assume most of the signal-processing in the electrical domain.

% outlook
% - security proofs with continuous fields (equivalence of tensor product of (single-mode) coherent states with time-continuous coherent state)
% - new protocols inspired from telecommunications (odfm)
% - frequency-entangled states (broadband squeezing)
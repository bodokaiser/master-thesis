\section*{Summary}
\addcontentsline{toc}{section}{Summary}

First, we derived the quantum mode expansion of the Maxwell and electric field operators in the Coulomb gauge, \cref{eq:maxwell_operator} and \cref{eq:electric_operator}, from fundamental field-theoretic arguments.
Second, we defined a vacuum state from which we axiomatically constructed generalized number and coherent quantum states.
Our results for the field operators and quantum states are compatible with the results reported in the literature on quantum field theory~\cite{Itzykson2012,Greiner2013,Srednicki2007,Peskin1995}.
The field-theoretic description naturally contains a momentum distribution related to the wave function and generalizes the frequency spectrum, essential for optical communication.
Contrary to the number states in single-mode quantum optics, the generalized number states are not eigenstates of the energy and momentum operator.
On the other hand, the coherent states behave as one would expect from a generalization of the single-mode description.

Our field-theoretic description coincides with the continuous-mode description of quantum optics~\cite{Barnett2002,Loudon2000,Vogel2006,Shapiro2009} when we
\begin{enumerate}
	\item approximate the vector field $\vu{A}(t,\vb{x})$ as a scalar field $\hat{A}(t,\vb{x})$,
	\item ignore the transverse profile of the wave function,
	\item rescale the field by the mean-value of the Lorentz factor $\sqrt{2\omega_0}$.
\end{enumerate}
The first approximation is straightforward and not a strong limitation as it is easy to restore the polarization \gls{dof} through a tensor product of two scalar fields.
For optical communication, and quantum optics, the transverse momentum profile transversal to the propagation axis carries no information and it is reasonable to ignore the transversal \gls{dof}s and reduce the field to one-dimension~\cite[p.~53]{Cohen2019}
\begin{equation}
	\int_{\mathbb{R}^3}
	\frac{\dd[3]{p}}{(2\pi)^3\sqrt{2\omega(\vb{p})}}
	\to
	\int_{-\infty}^{\infty}
	\frac{\dd{p}}{(2\pi)\sqrt{2p}}
	.
\end{equation}
The approximation can be made even stronger when neglecting back-scattering and reflection effects and restricting the momentum distribution to forward momentum, i.e.,
\begin{equation}
	\int_{-\infty}^{\infty}
	\frac{\dd{p}}{(2\pi)\sqrt{2p}}
	\approx	
	\int_{0}^{\infty}
	\frac{\dd{p}}{(2\pi)\sqrt{2p}}	
\end{equation}
where the identification of the forward momentum with the frequency, $p=\omega$ is often made.
To justify the third approximation, we need to some kind of bandwidth limitation, then by the mean-value theorem for definite integrals
\begin{equation}
	\int_{0}^{\infty}
	\frac{\dd{\omega}}{(2\pi)\sqrt{2\omega}}
	\approx
	\frac{1}{\sqrt{2\omega_0}}
	\int_{0}^{\infty}
	\frac{\dd{\omega}}{2\pi}
	,
\end{equation}
we can absorb the prefactor into the field.
We conclude that (continuous-mode) quantum optics is the scalar, one-dimensional forward, optical bandwidth approximation of a Maxwell quantum field theory in the Coulomb gauge.
The Coulomb gauge restricts our predictions to a stationary reference frame.
While a stationary reference frame appears reasonable for terrestrial communication, it must be questioned for space communication.

An important observable we have not yet introduced is the generalized quadrature operator.
Ref.~\cite[p.~79]{Barnett2002} defines the generalized quadrature operator as\footnote{We adapted the quadrature operator to our convention of the Fourier transform, dividing the integration measure by $2\pi$, and used $p=\omega$.}
\begin{equation}
	\hat{X}(\vartheta)
	=
	\int\frac{\dd{p}}{2\pi}
	\frac{1}{\sqrt{2}}
	\left\{
		\hat{a}(p)
		e^{-i\vartheta}
		+
		\hat{a}^\dagger(p)
		e^{+i\vartheta}
	\right\}
	,
	\label{eq:quadrature_operator_barnett}
\end{equation}
which for $\vartheta=0,\pi/2$ reduces to the continuous-mode generalization of the position and momentum operator of the harmonic oscillator.
In Maxwell field theory, the field operator $\hat{A}(t,\vb{x})$ generalizes the position operator of the harmonic oscillator, and indeed, we find
\begin{equation}
	\hat{A}(0,\vb{0})
	=
	\int\frac{\dd[3]{p}}{(2\pi)^3}
	\frac{1}{\sqrt{2\omega(\vb{p})}}
	\left\{
		\hat{a}_\lambda(\vb{p})
		+
		\hat{a}_\lambda^\dagger(\vb{p})
	\right\}
\end{equation}
to generalize Barnett's quadrature operator, \cref{eq:quadrature_operator_barnett}, to three dimensions including the Lorentz factor.
Application of the phase-rotation operator~\cite[p.~38]{Leonhardt2010} yields the phase-rotated quadrature operator
\begin{equation}
	e^{-i\vartheta\hat{N}}
	\hat{A}(0,\vb{0})
	e^{+i\vartheta\hat{N}}
	=
	\int\frac{\dd[3]{p}}{(2\pi)^3}
	\frac{1}{\sqrt{2\omega(\vb{p})}}
	\left\{
		\hat{a}(\vb{p})
		e^{-i\vartheta}
		+
		\hat{a}^\dagger(\vb{p})
		e^{+i\vartheta}
	\right\}
\end{equation}
and suggests defining the phase-rotated Maxwell field as the generalized quadrature operator
\begin{equation}
	\hat{X}(t,\vb{x};\vartheta)
	=
	e^{-i\vartheta\hat{N}}
	\hat{A}(t,\vb{x})
	e^{+i\vartheta\hat{N}}
	.
	\label{eq:quadrature_operator}
\end{equation}
How can the Maxwell field be an observable when it is not gauge invariant?
While it is true that the Maxwell field shows a gauge symmetry, the gauge symmetry is uniquely fixed through the Coulomb gauge, which is required to remove unphysical polarization \gls{dof}s.\footnote{See Ref.~\cite{stackexchange676622} for a discussion on the physicality of the Maxwell field and the gauge freedom.}

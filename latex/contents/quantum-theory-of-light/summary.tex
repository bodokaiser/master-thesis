\summary{Summary}
\addcontentsline{toc}{section}{Summary}

The present chapter consists of two stages.
In the first stage, we derived the plane-mode expansions of the quantum Maxwell and electric field operators in the Coulomb gauge, \cref{eq:maxwell_operator} and \cref{eq:electric_operator}, from field-theoretic arguments.
In the second second, we axiomatically motivated a generalization of the number and coherent states to a momentum distribution.
The momentum distribution is related to the wave function and generalizes the frequency spectrum, essential for communication, to three dimensions.
While the properties of our generalized quantum states are qualitatively compatible with the simplified states from single-mode quantum optics, we presented them in a consistent framework justified by established results from quantum field theory.

Our field-theoretic description reduces to continuous-mode quantum optics~\cite{Barnett2002,Loudon2000} if we ignore
\begin{enumerate}
	\item the polarization \glspl{dof} of the Maxwell vector field,
	\item the transverse momentum distribution, and
	\item the Lorentz factor $1/\sqrt{2\omega}$.
\end{enumerate}
The first approximation is straightforward and not a strong limitation as it is easy to restore the polarization \glspl{dof} through a tensor product of two scalar fields.
For optical communication, and quantum optics, the transverse momentum profile transversal to the propagation axis carries no information and it is reasonable to ignore the transversal \glspl{dof}~\cite[p.~53]{Cohen2019}
\begin{equation}
	\int_{\mathbb{R}^3}
	\frac{\dd[3]{p}}{(2\pi)^3\sqrt{2\omega(\vb{p})}}
	\to
	\int_{-\infty}^{\infty}
	\frac{\dd{p}}{(2\pi)\sqrt{2p}}
	.
\end{equation}
The approximation can be made even stronger when neglecting back-scattering and reflection effects and restricting the momentum distribution to forward propagation, i.e.,
\begin{equation}
	\int_{-\infty}^{\infty}
	\frac{\dd{p}}{(2\pi)\sqrt{2p}}
	\approx	
	\int_{0}^{\infty}
	\frac{\dd{p}}{(2\pi)\sqrt{2p}}	
\end{equation}
where the identification of the forward momentum with the frequency, $p=\omega$ is often made.
The third approximation requires some additional investigation.
For one, every physical measurement is bandwidth-limited such that by the mean-value theorem for definite integrals
\begin{equation}
	\int_{0}^{\infty}
	\frac{\dd{\omega}}{(2\pi)\sqrt{2\omega}}
	\approx
	\frac{1}{\sqrt{2\omega_0}}
	\int_{0}^{\infty}
	\frac{\dd{\omega}}{2\pi}
	,
\end{equation}
the Lorentz factor is effectively constant.
However, in principle, it should be possible to measure the Lorentz factor.
That said, the Lorentz factor breaks our Fourier transform, making some results difficult for interpretation.\footnote{More precisely, we need to distinguish between the four-dimensional Fourier transform and the Fourier transform implementing the energy-momentum relation.}
For instance, if we ignore the Lorentz factor and ignore the transverse momentum distribution, we find that the eigenvalue of the coherent state with regard to the Maxwell field operator, \cref{eq:coherent_state_eigenstate_maxwell}, simplifies to
\begin{equation}
	\hat{A}^{(-)}(t,x)
	\ket{\alpha}
	=
	\int\frac{\dd{p}}{2\pi}
	\alpha(p)
	e^{-ip(t-x)}
	\ket{\alpha}
	=
	\alpha(t-x)^*
	\ket{\alpha}
	,
\end{equation}
where $\alpha(t-x)$ is the time-domain signal.\footnote{We identify the negative-frequency Maxwell operator $\hat{A}^{(-)}$ with the Fourier transform of the annihilation operator used by Loudon~\cite{Loudon2000}.}
Another assumption or restriction, which we have not mentioned explicitly yet, is the use of the Coulomb gauge.
The Coulomb gauge restricts our predictions to a stationary reference frame.
While a stationary reference frame appears reasonable for terrestrial communication, it must be questioned for space communication.

Another important operator, subject to controversies, which we have not mentioned so far, is the quadrature operator.
Ref.~\cite[p.~79]{Barnett2002} defines the continuous-mode quadrature operator as\footnote{We adapted the quadrature operator to our convention of the Fourier transform, dividing the integration measure by $2\pi$, and added time evolution.}
\begin{equation}
	\hat{X}(\vartheta)
	=
	\frac{1}{\sqrt{2}}
	\int\frac{\dd{p}}{2\pi}
	\left\{
		\hat{a}(p)
		e^{-i(pt+\vartheta)}
		+
		\hat{a}^\dagger(p)
		e^{+i(pt+\vartheta)}
	\right\}
	.
	\label{eq:quadrature_operator_barnett}
\end{equation}
Comparing \cref{eq:quadrature_operator_barnett} with the one-dimensional Maxwell operator,
\begin{equation}
	\hat{A}(t,x)
	=
	\int\frac{\dd{p}}{2\pi}
	\frac{1}{\sqrt{2p}}
	\left\{
		\hat{a}_\lambda(p)
		e^{-ip(t-x)}
		+
		\hat{a}_\lambda^\dagger(p)
		e^{+ip(t-x)}
	\right\}
	,
\end{equation}
we note that these are equal up to the Lorentz factor, a phase, and the spatial dependency $x$.
For a measurement at a stationary location, the spatial dependency reduces to a phase, which can be further modified using the phase-rotation operator~\cite[p.~38]{Leonhardt2010}, i.e.,
\begin{equation}
	e^{-i\vartheta\hat{N}}
	\hat{A}(t)
	e^{+i\vartheta\hat{N}}
	=
	\int\frac{\dd{p}}{2\pi}
	\frac{1}{\sqrt{2p}}
	\left\{
		\hat{a}_\lambda(p)
		e^{-ip(t+\vartheta)}
		+
		\hat{a}_\lambda^\dagger(p)
		e^{+ip(t+\vartheta)}
	\right\}
	,
\end{equation}
where $\hat{N}$ denotes the number operator defined in \cref{eq:maxwell_number_operator}.
Using that practical measurements are bandwidth-limited, we can invoke the mean-value theorem for definite integrals and pull out the Lorentz factor $1/\sqrt{2p}$ from the integrand and thereby fully recovering the quadrature operator proposed by in Ref.~\cite{Barnett2002}, \cref{eq:quadrature_operator_barnett}.
How can the Maxwell field be an observable when it is not gauge invariant?
While it is true that the Maxwell field shows a gauge symmetry, the gauge symmetry is uniquely fixed through the Coulomb gauge, which is required to remove unphysical polarization \glspl{dof}.\footnote{See Ref.~\cite{stackexchange676622} for a discussion on the physicality of the Maxwell field and the gauge freedom.}
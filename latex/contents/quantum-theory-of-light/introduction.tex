\chapter{Quantum theory of light}\label{ch:light}

As stated in the introduction, a continuous-mode quantum theory of light is essential to describe time-continuous information-bearing signals.
However, single-mode quantum optics~\cite{Fox2006,Gerry2005,Haroche2006,Meystre2007}, the most established physical theory for describing quantum aspects of light, strictly assumes monochromatic, i.e., time-independent, light.
The few references~\cite{Barnett2002,Loudon2000} presenting a continuous-mode quantum theory of light are sparse on the justification or origin of their results, making the results overall difficult to assess.
On the other hand, in quantum field theory, we find a description of light with a continuous momentum distribution but towards scattering experiments.

In the following chapter, we derive a continuous-mode quantum theory of light rooted in quantum-field theory, which acts as the foundation of our theoretical framework.
In the first part, we recap the classical description of the Maxwell field, which we quantize canonically in the last step to derive the relevant quantum field operators.
In the second part, we axiomatically motivate the more general quantum states of the Maxwell field and discuss some of their properties.
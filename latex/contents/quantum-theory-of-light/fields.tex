\section{Maxwell theory}

The Maxwell field is the (quantum) field theory whose excitations correspond to photons, the mediator of the electromagnetic force.

Our presentation of the Maxwell field heavily exploits the modern field-theoretic arguments that have shown great success in developing the standard model.
We first motivate the Lagrangian describing the Maxwell field, the Maxwell Lagrangian.
Second, we relate the Maxwell field to the electromagnetic field components and discuss Maxwell theory in the context of classical electromagnetism.
Third, we discuss and interpret the physical implications of the local gauge symmetry of the Maxwell field.
In the last two sections, we perform a plane-wave (mode) expansion of the Maxwell field and observables, followed by canonical quantization.

\subsection{Lagrangian from first principles}

From a theoretical physics viewpoint, Maxwell theory is a relativistic vector field theory with local gauge symmetry.
That alone is sufficient to "guess" the Maxwell Lagrangian from first principles~\cite[p.~149]{Greiner2013}.
Having established the Maxwell Lagrangian, we can fully infer the physics of electromagnetism inside the field theoretical framework.

As a relativistic vector field, we expect the Maxwell field $A^\mu$ to have four components, one time and three spatial components, and to transform as a Lorentz vector~\cite[p.~37]{Peskin1995}
\begin{equation}
	A^\mu(x)
	\to
	A^{\prime\mu}(x^\prime)
	=
	\Lambda^\mu_\nu
	A^\nu(\Lambda^{-1}x)
	.
\end{equation}
Contracting Lorentz vectors, or in general Lorentz tensors, to a scalar yields a Lorentz-invariant quantity.
Exclusively using Lorentz invariant quantities when constructing our theory ensures the compatibility of our theory with special relativity.

As a starting point for our theory's the Lagrangian (density), we propose
\begin{equation}
	\mathcal{L}
	=
	c_1
	\left(
		\partial_\mu
		A^\nu
	\right)
	\left(
		\partial^\mu
		A_\nu
	\right)
	+
	c_2
	\left(
		\partial_\mu
		A^\mu
	\right)
	\left(
		\partial_\nu
		A^\nu
	\right)
	+
	c_3
	\left(
		\partial_\mu
		A^\nu
	\right)
	\left(
		\partial_\nu
		A^\mu
	\right)
	+
	c_4
	m^2
	A_\mu A^\mu
	\label{eq:proposed_maxwell_lagrangian_proposal},
\end{equation}
wherein we restricted ourselves to terms which allow for dimensionless coefficients $c1,c2,c3,c4\in\mathbb{R}$, a mass term, and no (self-)interactions.\footnote{The restriction to dimensionless coefficients can be further motivated by renormalization arguments.}
For the action integral
\begin{equation}
	S
	=
	\int\dd{t}
	L
	=
	\int\dd[4]{x}
	\mathcal{L}
\end{equation}
to exist, we need $A^\mu$ to be square-integrable, i.e., vanish at the integration boundaries.
Using partial integration, we can show the redundancy of two of the first three terms in \cref{eq:proposed_maxwell_lagrangian}, see Ref.~\cite{deRham2014}.
We remove the redundancy by setting the third coefficient to zero, $c_3=0$.
Demanding invariance under local gauge transformations requires removing the mass term and determining the Lagrangian up to an overall constant.
The overall constant does not affect the dynamics, and its absolute value is equal to $1/2$ by convention.
We finally arrive at the well-known Lagrangian density of the Maxwell field,
\begin{equation}
	\mathcal{L}
	=
	-
	\frac{1}{2}
	\left(
		\partial_\mu
		A^\nu
	\right)
	\left(
		\partial^\mu
		A_\nu
	\right)
	+
	\frac{1}{2}
	\left(
		\partial_\mu
		A^\mu
	\right)
	\left(
		\partial_\nu
		A^\nu
	\right)
	\label{eq:maxwell_lagrangian_field}
	.
\end{equation}
While we can experimentally verify the vector nature of the Maxwell field by polarization experiments, the requirement of the Maxwell field having local gauge symmetry appears artificial.
If we accept the Dirac Lagrangian, describing charged fermions, then Noether's theorem links local gauge symmetry to charge conservation.
However, the Dirac Lagrangian itself cannot be made gauge-invariant without coupling to a gauge-invariant vector field, which turns out to be the Maxwell field.
The complete gauge-invariant Lagrangian describing charged fermions coupled to the Maxwell field lays down the foundation of \gls{qed}.

\subsection{Classical electromagnetism}

We have introduced the Maxwell field as the fundamental mediator of the electromagnetic force.
However, so far, it is unclear how the electromagnetic field components relate to the Maxwell field.
The (manifest) covariant formulation makes it particularly difficult to identify the non-covariant electromagnetic vector fields.
To shed some light, we first derive the covariant Lorentz force law and then compare it to the non-covariant vector formulation to identify the electromagnetic field components.

To derive the covariant Lorentz force law, let us consider the action of a point particle with mass $m$ and charge $q$ coupled to the Maxwell field $A^\mu$~\cite[p.~244]{Zee2013}
\begin{equation}
	S
	=
	-
	m\int\dd{\tau}
	\sqrt{-g_{\mu\nu}\odv{x^\mu}{\tau}\odv{x^\nu}{\tau}}
	+
	q\int\dd{\tau}
	A_\mu\left(x(\tau)\right)
	\odv{x^\mu}{\tau}
	\label{eq:lorentz_force_action}
\end{equation}
wherein $g_{\mu\nu}$ is the Minkowski metric.
The first term is the generalization of a line integral, parametrized by tau, to Minkowski spacetime.
The second term describes the coordinate-dependent coupling of the Maxwell field with the charge of the particle.
Invoking the variational calculus on the action, we recover the covariant formulation of Lorentz force law
\begin{equation}
	m
	\odv[order={2}]{x^\mu}{\tau}
	=
	qF^\mu_\nu(x)\odv{x^\nu}{\tau}
	\label{eq:covariant_lorentz_force}
\end{equation}
where we introduced the asymmetric field-strength tensor
\begin{equation}
	F_{\mu\nu}
	=
	\partial_\mu
	A_\nu
	-
	\partial_\nu
	A_\mu
	\label{eq:field_strength_tensor}
	.
\end{equation}
The field-strength tensor covariantly encodes the electromagnetic field components.

Using the field-strength tensor $F_{\mu\nu}$ and including the interaction of the Maxwell field with an external classical current $j^\mu$, we can rewrite the Maxwell Lagrangian of \cref{eq:maxwell_lagrangian_field} as
\begin{equation}
	\mathcal{L}
	=
	-
	\frac{1}{4}
	F_{\mu\nu}
	F^{\mu\nu}
	+
	A_\mu j^\mu
	\label{eq:maxwell_lagrangian_field_strength}
	.
\end{equation}
We note that the action of the interaction term in \cref{eq:maxwell_lagrangian_field_strength} reduces to the second action term in \cref{eq:lorentz_force_action} when using the current of a point particle with charge $q$~\cite[p.~177]{Peskin1995}
\begin{equation}
	j^\mu(x)
	=
	q
	\int\dd{\tau}
	\odv{y^\mu(\tau)}{\tau}
	\delta^{(4)}\left(x-y(\tau)\right)
	.
\end{equation}

Comparison of the covariant Lorentz force law, \cref{eq:covariant_lorentz_force}, with the non-covariant version
\begin{equation}
	m
	\odv[order={2}]{\vb{x}}{t}
	=
	q\left(
		\vb{E}
		+
		\odv{\vb{x}}{t}
		\vcross
		\vb{B}
	\right)
	\label{eq:lorentz_force}
	,
\end{equation}
we can relate the components of the field-strength tensor and the electromagnetic field~\cite[p.~245]{Zee2013}
\begin{align}
	E^1
	&=
	F^{01}
	=
	-
	F_{01}
	&
	E^2
	&=
	F^{02}
	=
	-
	F_{02}
	&
	E^3
	&=
	F^{03}
	=
	-
	F_{03}
	\\
	B^1
	&=
	F^{23}
	=
	-
	F_{23}
	&
	B^2
	&=
	F^{31}
	=
	-
	F_{31}
	&
	B^3
	&=
	F^{12}
	=
	-
	F_{12}
	,
\end{align}
which we can compactly summarize.~\cite[p.~336]{Srednicki2007}
\begin{align}
	F^{0i}
	&=
	E^i
	&
	F^{ij}
	&=
	\varepsilon^{ijk}
	B_k
	\label{eq:field_strength_components}
	.
\end{align}
We have successfully related the rather abstract Maxwell field with the observable electromagnetic field components using the covariant field-strength tensor.

In a final step, we would like to express the Maxwell Lagrangian in terms of the electromagnetic fields to complete the bridge to classical electrodynamics.
Using \cref{eq:field_strength_components}, we can derive the identity~\cite[p.~142]{Greiner2013}
\begin{equation}
	F_{\mu\nu}
	F^{\mu\nu}
	=
	-2
	\left(
		\vb{E}^2
		-
		\vb{B}^2
	\right)
	\label{eq:field_strength_scalar}
	.
\end{equation}
Inserting \cref{eq:field_strength_scalar} into \cref{eq:maxwell_lagrangian_field_strength} we find the Maxwell Lagrangian as reported in many books on classical electrodynamics
\begin{equation}
	\mathcal{L}
	=
	\frac{1}{2}
	\left(
		\vb{E}^2
		-
		\vb{B}^2
	\right)
	+
	A_0j^0
	-
	\vb{A}
	\vdot
	\vb{j}
\end{equation}
where we expanded the Minkowski inner product $A_\mu j^\mu$ in terms of the time and spatial components.
We identify $j^0$ and $\vb{j}$ as the charge and $\vb{j}$ as the current density as well as $A_0$ as the scalar and $\vb{A}$ as the vector potential of electromagnetism.

The manifest Lorentz-covariant Maxwell equations are~\cite[p.~336]{Srednicki2007}
\begin{align}
	\partial_\mu
	\tilde{F}^{\mu\nu}
	&=
	0
	\label{eq:covariant_homogeneous_maxwell_equations}
	\\
	\partial_\mu
	F^{\mu\nu}
	&=
	j^\mu
	\label{eq:covariant_inhomogeneous_maxwell_equations}
\end{align}
where we defined the dual field-strength tensor~\cite[p.~142]{Greiner2013}
\begin{equation}
	\tilde{F}^{\mu\nu}
	=
	\frac{1}{2}
	\varepsilon^{\mu\nu\rho\sigma}
	F_{\rho\sigma}
	\label{eq:dual_field_strength}
\end{equation}
with $\varepsilon^{\mu\nu\rho\sigma}$ being the completely antisymmetric unit tensor.
\Cref{eq:covariant_homogeneous_maxwell_equations} summarizes the homogeneous Maxwell equations and can be derived from the Bianchi identity.
\Cref{eq:covariant_inhomogeneous_maxwell_equations} summarizes the inhomogeneous Maxwell equations and follows from the variational calculus of the Maxwell Lagrangian, i.e., represents the Maxwell field's \gls{eom}s.
Evaluating the non-zero components of the Lorentz tensor equations and inserting the relation of the field-strength to the electromagnetic field, \cref{eq:field_strength_components}, we arrive at the microscopic Maxwell equations
\begin{align}
	\div\vb{E}
	&=
	j_0
	&
	\div\vb{B}
	&=
	0
	\\
	\curl\vb{E}
	&=
	-\partial_t\vb{B}
	&
	\curl\vb{B}
	&=
	\vb{j}
	+
	\partial_t\vb{E}
	.
\end{align}
We derived Maxwell equations by guessing the Maxwell Lagrangian from fundamental principles and symmetries, whereas historically, Maxwell equations summarized decades of experiments studying the electromagnetic field.

\subsection{Lorenz, Coulomb and temporal gauge conditions}

As a four-dimensional vector field, we would expect the Maxwell field $A^\mu$ to have four \gls{dof}s, one temporal $A^0$, one longitudinal $A_\parallel$, and two transverse $\vb{A}_\perp$.
At the same time, freely propagating electromagnetic waves have only two \gls{dof}s, the polarization.\footnote{Alternatively, we can take the particle picture and say that the photon has two helicities, $\pm1$.}
The zero mass of the photon requires the photon to travel at light speed, restricting the photon's temporal and longitudinal \gls{dof}.
Local gauge symmetry reflects the non-physicality, more precisely, the mathematical redundancy of two of the four \gls{dof}s.
To remove the unphysical \gls{dof}s, we impose a gauge condition, i.e., we choose a specific gauge field $\chi$ and perform a local gauge transformation
\begin{equation}
	A_\mu
	\to
	A_\mu^\prime
	=
	A_\mu
	+
	\partial_\mu\chi
\end{equation}
to remove some components of the Maxwell field.
Different gauge conditions exist, see Ref.~\cite[p.~144]{Greiner2013} and Ref.~\cite[p.~339]{Srednicki2007}, and some gauges may be more convenient than others depending on the problem under consideration.
For instance, the Lorenz gauge
\begin{equation}
	\partial_\mu
	A^\mu
	=
	0	
\end{equation}
has the advantage of being manifestly Lorentz covariant at the cost of having the different \gls{dof}s intertwined to be independent of a particular reference frame.
The Coulomb gauge
\begin{equation}
	\partial_i
	A^i
	=
	\div\vb{A}
	=
	0
	\label{eq:coulomb_gauge}
\end{equation}
corresponds to selecting a stationary reference frame in which the electromagnetic radiation is purely transverse~\cite[p.~40]{Bogoliubov1982}.
Imposing the Coulomb gauge only removes one \gls{dof} from the Maxwell field.
We use the remaining residual gauge freedom to impose the temporal gauge
\begin{equation}
	A_0
	=
	0
	\label{eq:temporal_gauge}
	.
\end{equation}
The temporal gauge is valid when there is no charge distribution, $j_0=0$.
Static charge distributions add a Coulomb interaction that does not involve the exchange of physical photons and is not subject to quantization.
However, static charges add constant energy to the system and a longitudinal component to the electric field.
In most cases, it is sufficient to discuss the effects of the Coulomb interaction separately from the radiation, justifying the temporal gauge.
See Ref.~\cite[p.~145,187,200]{Greiner2013} for more detail on the Coulomb interaction and the Maxwell field's longitudinal \gls{dof}.

\subsection{Plane-wave expansion in the Coulomb gauge}

The next step is to find a general solution to the free \gls{eom}, \cref{eq:covariant_inhomogeneous_maxwell_equations},
which in the Coulomb and temporal gauge reduces to the relativistic wave equation
\begin{equation}
	\partial_t^2
	\vb{A}_\perp
	=
	\laplace
	\vb{A}_\perp
	\label{eq:maxwell_eom}
\end{equation}
wherein $\vb{A}_\perp$ denotes the transverse Maxwell field.
From now on, we drop the subscript and assume the Maxwell field to be transverse, i.e., to satisfy the Coulomb gauge condition, \cref{eq:coulomb_gauge}.
The existence of the Maxwell action requires the Maxwell field to be square-integrable which implies the existence of the Fourier expansion
\begin{equation}
	\vb{A}(t,\vb{x})
	=
	\int_{\mathbb{R}^3}\frac{\dd[3]{p}}{(2\pi)^3}
	\vb{A}(t,\vb{p})
	e^{+i\vb{p}\vdot\vb{x}}
	=
	\int_{\mathbb{R}^4}\frac{\dd[4]{p}}{(2\pi)^4}
	\vb{A}(p_0,\vb{p})
	e^{-ip_0t+i\vb{p}\vdot\vb{x}}
	\label{eq:maxwell_fourier_expansion}
\end{equation}
with $\vb{A}(p_0,\vb{p})$ being the complex-valued Fourier vector amplitudes.
Inserting the Maxwell field's Fourier expansion, \cref{eq:maxwell_fourier_expansion}, into the relativistic wave equation, \cref{eq:maxwell_eom}, reduces the \gls{eom} in momentum space to
\begin{equation}
	0
	=
	p_0^2
	-
	\vb{p}^2
	=
	\left(
		p_0
		-
		\norm{\vb{p}}
	\right)
	\left(
		p_0
		+
		\norm{\vb{p}}
	\right)
	\label{eq:maxwell_eom_momentum}
	.
\end{equation}
Requiring the energy to be positive, $p_0>0$, we arrive at the relativistic energy-momentum relation
\begin{equation}
	p_0
	=
	\omega(\vb{p})
	=
	\norm{\vb{p}}
	\label{eq:energy_momentum_relation}
\end{equation}
for massless particles.
Fourier amplitudes satisfying the relativistic-energy momentum relation are plane-wave solutions to the Maxwell field's free \gls{eom}.
We define the plane-wave expansion as the Fourier expansion, \cref{eq:maxwell_fourier_expansion}, where we constrain the integration domain to
\begin{equation*}
	\Sigma
	=
	\left\{
		\left(p_0,\vb{p}\right)
		\in
		\mathbb{R}^4
		\colon
		p_0^2
		=
		\omega(\vb{p})^2
	\right\}
	.
\end{equation*}
Each plane-wave is a solution to the free \gls{eom} and the plane-wave expansion denotes a general solution.
We can extend the integration domain of the plane-wave expansion back to the $\mathbb{R}^4$ by having the delta distribution ensure the relativistic energy-momentum relation
\begin{equation}
	\begin{split}
		\vb{A}(t,\vb{x})
		&=
		\int_\Sigma\frac{\dd[4]{p}}{(2\pi)^4}
		\vb{A}(p_0,\vb{p})
		e^{-ip_\mu x^\mu}
		\\
		&=
		\int_{\mathbb{R}^4}\frac{\dd[4]{p}}{(2\pi)^3}
		\delta^{(1)}
		\left(
			p_0^2
			-
			\omega(\vb{p})^2
		\right)
		\vb{A}(p_0,\vb{p})
		e^{-ip_\mu x^\mu}
		.
	\end{split}
\end{equation}
Exploiting the composition property of the delta distribution,
\begin{equation}
	\delta^{(1)}
	\left(
		p_0^2
		-
		\omega(\vb{p})^2
	\right)
	=
	\frac{
		\delta^{(1)}
		\left(p_0-\omega(\vb{p})\right)
		+
		\delta^{(1)}
		\left(p_0+\omega(\vb{p})\right)
	}{\sqrt{2\omega(\vb{p})}}
	,
\end{equation}
we further decompose the plane-wave expansion into a positive and negative frequency part
\begin{equation}
	\begin{split}
		\vb{A}(t,\vb{x})
		&=
		\int_{\mathbb{R}^4}\frac{\dd[4]{p}}{(2\pi)^3\sqrt{2\omega(\vb{p})}}
		\delta^{(1)}
		\left(
			p_0
			-
			\omega(\vb{p})
		\right)
		\vb{A}(p_0,\vb{p})
		e^{-ip_0t+i\vb{p}\vdot\vb{x}}
		\\
		&+
		\int_{\mathbb{R}^4}\frac{\dd[4]{p}}{(2\pi)^3\sqrt{2\omega(\vb{p})}}
		\delta^{(1)}
		\left(
			p_0
			+
			\omega(\vb{p})
		\right)
		\vb{A}(p_0,\vb{p})
		e^{-ip_0t+i\vb{p}\vdot\vb{x}}
		\\
		&=
		\int_{\mathbb{R}^3}\frac{\dd[3]{p}}{(2\pi)^3\sqrt{2\omega(\vb{p})}}
		\vb{A}\left(\omega(\vb{p}),\vb{p}\right)
		e^{-i\omega(\vb{p})t+i\vb{p}\vdot\vb{x}}
		\\
		&+
		\int_{\mathbb{R}^3}\frac{\dd[3]{p}}{(2\pi)^3\sqrt{2\omega(\vb{p})}}
		\vb{A}\left(-\omega(\vb{p}),\vb{p}\right)
		e^{+i\omega(\vb{p})t+i\vb{p}\vdot\vb{x}}
		\\
		&=
		\int_{\mathbb{R}^3}\frac{\dd[3]{p}}{(2\pi)^3\sqrt{2\omega(\vb{p})}}
		\vb{A}\left(\omega(\vb{p}),\vb{p}\right)
		e^{-i\omega(\vb{p})t+i\vb{p}\vdot\vb{x}}
		\\
		&+
		\int_{\mathbb{R}^3}\frac{\dd[3]{p}}{(2\pi)^3\sqrt{2\omega(\vb{p})}}
		\vb{A}\left(\omega(\vb{p}),\vb{p}\right)^*
		e^{+i\omega(\vb{p})t-i\vb{p}\vdot\vb{x}}
		.
	\end{split}
	\label{eq:maxwell_plane_wave_expansion}
\end{equation}
To arrive at \cref{eq:maxwell_plane_wave_expansion}, we evaluated the delta distributions, performed the variable substitution $\vb{p}\to-\vb{p}$ in the second term and used the conjugate symmetry of the Maxwell field in momentum space, $A(-p_0,-\vb{p})=A(p_0,\vb{p})^*$.
The Maxwell field being transverse implies the momentum vector $\vb{p}$ being orthogonal to the Fourier amplitude,
\begin{equation}
	\vb{p}
	\vdot
	\vb{A}\left(\omega(\vb{p}),\vb{p}\right)
	=
	0
	.
\end{equation}
We construct a orthonormal basis $\left\{\vu{e}_\lambda\right\}_{\lambda=1,2}$ for each momentum vector $\vb{p}$, the polarization basis to momentum $\vb{p}$, which is orthogonal to the momentum $\vb{p}$ and complete~\cite[p.~341]{Srednicki2007}, i.e.,
\begin{align}
	\vb{p}
	\vdot
	\vu{e}_\lambda(\vb{p})
	&=
	0
	\\
	\vu{e}_\lambda(\vb{p})
	\vdot
	\vu{e}_{\lambda^\prime}(\vb{p})
	&=
	\delta_{\lambda,\lambda^\prime}
	\\
	\sum_{\lambda=1,2}
	\vu{e}_\lambda(\vb{p})^i
	\vu{e}_\lambda(\vb{p})^j
	&=
	\delta^{ij}
	-
	\frac{p^ip^j}{\vb{p}^2}
	.
\end{align}
Writing the Fourier vector amplitude in terms of the polarization basis
\begin{equation}
	\vb{A}\left(\omega(\vb{p}),\vb{p}\right)
	=
	\sum_{\lambda=1,2}
	a_\lambda(\vb{p})
	\vu{e}_\lambda(\vb{p})
\end{equation}
and inserting it back into the plane-wave expansion, \cref{eq:maxwell_plane_wave_expansion}, we arrive at our final result
\begin{equation}
	\vb{A}(t,\vb{x})
	=
	\sum_{\lambda=1,2}
	\int_{\mathbb{R}^3}\frac{\dd[3]{p}}{(2\pi)^3\sqrt{2\omega(\vb{p})}}
	\left\{
		a_\lambda(\vb{p})
		\vu{e}_\lambda(\vb{p})
		e^{-i\omega(\vb{p})t+i\vb{p}\vdot\vb{x}}
		+
		\text{c.c.}
	\right\}
	\label{eq:maxwell_expansion}
\end{equation}
The plane-wave expansion looks already similar to the quantum mode expansion of the Maxwell field but relies only on classical arguments.\footnote{The similarity is not surprising if we consider $A(t,\vb{x})$ as the expectation value of the Maxwell field operator given a coherent state, $\bra{\alpha}\vu{A}(t,\vb{x})\ket{\alpha}$.}
In principle, we could stop here, replace the Fourier modes with the annihilation operator satisfying the canonical commutation relation, and we have quantized our field.\footnote{The equivalence of Fourier and quantum modes has been experimentally established, see Ref.~\cite{Hulet1985}.}
However, for completeness, we review the standard canonical quantization of the Maxwell field in the next section.

\subsection{Canonical quantization in the Coulomb gauge}

In canonical quantization, we work in the Hamiltonian picture, promoting the conjugate variables with operators that satisfy the equal-time commutation relations.
In the canonical quantization of gauge theories, like the Maxwell field, we have the additional technical complication of handling the unphysical \gls{dof}s.
In the Coulomb gauge, we can take care of the transverse gauge condition by amending the equal-time commutation relations to be transverse, more to that later.
In the Lorenz gauge, the Gupta-Bleuler method includes the unphysical \gls{dof} in the quantization process but removes them later by constraining the state space, see Ref.~\cite[p.~180]{Greiner2013}.

Before performing the canonical quantization, we need to find the conjugate field variables and Hamiltonian (density).
The conjugate momentum to the Maxwell field $A_i$ is~\cite[p.~342]{Srednicki2007}
\begin{equation}
	\Pi_i
	=
	\pdv{\mathcal{L}}{(\partial_t A_i)}
	=
	-
	\partial_t A^i
	=
	\partial_t A_i
	\label{eq:maxwell_conjugate_momentum}
\end{equation}
where the sign change occurs due to the spatial components of the Minkowski metric when lowering a spatial index, $A^i=g^{ij}A_j=-\delta^{ij}A_j=-A_j$.
In the Coulomb gauge, the electric field components are equal to the negative conjugate momentum components, i.e.,
\begin{equation}
	E^i
	=
	F^{0i}
	=
	\partial_t A^i
	=
	-
	\Pi^i
\end{equation}
where we used \cref{eq:field_strength_components} and the temporal gauge.
In the following, we replace the conjugate momentum with the electric field components, $\Pi^i=-E^i$.
The Hamiltonian density of the Maxwell field is equal to
\begin{equation}
	\mathcal{H}
	=
	\frac{1}{2}
	E_i E^i
	+
	\frac{1}{2}
	\partial_i A_j
	\partial^j A^i
	\label{eq:maxwell_hamiltonian}
\end{equation}
and can be found by Legendre transform of the Lagrangian density~\cite[p.~342]{Srednicki2007} or using the energy density component from the energy-momentum tensor~\cite[p.~148]{Greiner2013}.

We now perform the canonical quantization by replacing the dynamical field variables $A_i,-E_i$ with the field operators $\hat{A}_i,-\hat{E}_i$ satisfying the equal-time commutation relations~\cite[p.~197]{Greiner2013}
\begin{align}
	\comm{\hat{A}_i(t,\vb{x})}{\hat{E}_j(t,\vb{y})}
	&=
	-i
	\delta^{(3)}_{\perp ij}(\vb{x}-\vb{y})
	\label{eq:equal_time_commutation_relation_nonzero}
	\\
	\comm{\hat{A}_i(t,\vb{x})}{\hat{A}_j(t,\vb{y})}
	&=
	\comm{\hat{E}_i(t,\vb{x})}{\hat{E}_j(t,\vb{y})}
	=
	0
	\label{eq:equal_time_commutation_relation_zero}
\end{align}
where we adapted the transverse delta distribution~\cite[p.~198]{Greiner2013}
\begin{equation}
	\delta^{(3)}_{\perp ij}(\vb{x})
	=
	\left(
		\delta_{ij}
		-
		\frac{\partial_i\partial_j}{\partial^2}
	\right)
	\delta^{(3)}(\vb{x})
	=
	\int_{\mathbb{R}^3}\frac{\dd[3]{p}}{(2\pi)^3}
	\left(
		\delta_{ij}
		-
		\frac{p_ip_j}{\vb{p}^2}
	\right)
	e^{i\vb{p}\vdot\vb{x}}
	\label{eq:transverse_delta_distribution}
\end{equation}
which implements the Coulomb gauge on the operator level.

Replacing the Fourier amplitudes, $a_\lambda(\vb{p})$ and $a_\lambda(\vb{p})^*$ in the plane-wave expansion of the Maxwell field with the annihilation operators $\hat{a}_\lambda(\vb{p})$ and the creation operator $\hat{a}_\lambda^\dagger(\vb{p})$, we find the Maxwell field operator to be
\begin{equation}
	\vu{A}(t,\vb{x})
	=
	\vu{A}^{(-)}(t,\vb{x})
	+
	\vu{A}^{(+)}(t,\vb{x})
	\label{eq:maxwell_operator}
\end{equation}
where we defined the positive and negative frequency parts to be
\begin{align}
	\vu{A}^{(-)}(t,\vb{x})
	&=
	\sum_{\lambda=1,2}
	\int_{\mathbb{R}^3}\frac{\dd[3]{p}}{(2\pi)^3\sqrt{2\omega(\vb{p})}}
	\hat{a}_\lambda(\vb{p})
	\vu{e}_\lambda(\vb{p})
	e^{-i\omega(\vb{p})t+i\vb{p}\vdot\vb{x}}
	\label{eq:maxwell_negative_operator}
	\\
	\vu{A}^{(+)}(t,\vb{x})
	&=
	\sum_{\lambda=1,2}
	\int_{\mathbb{R}^3}\frac{\dd[3]{p}}{(2\pi)^3\sqrt{2\omega(\vb{p})}}
	\hat{a}_\lambda^\dagger(\vb{p})
	\vu{e}_\lambda(\vb{p})^*
	e^{+i\omega(\vb{p})t-i\vb{p}\vdot\vb{x}}
	\label{eq:maxwell_positive_operator}
	.
\end{align}
In the literature, we find different conventions concerning the integration measure, the sign of the complex exponentials, and the complex conjugation of the of polarization basis vectors in \cref{eq:maxwell_positive_operator,eq:maxwell_negative_operator} originating from different conventions for the Fourier transform.
For example, Refs.~\cite{Peskin1995,Greiner2013} use $\hat{a}_\lambda(\vb{p})e^{-i\omega(\vb{p})t}$ while Refs.~\cite{Srednicki2007,Weinberg1995} use $\hat{a}_\lambda(\vb{p})e^{+i\omega(\vb{p})t}$.
The dimensionless limit of the Maxwell field operator, $\omega(\vb{p})\to1$, yields the quadrature operator
\begin{equation}
	\hat{X}(t,\vb{x})
	=
	\sum_{\lambda=1,2}
	\int_{\mathbb{R}^3}\frac{\dd[3]{p}}{(2\pi)^3}
	\frac{1}{\sqrt{2}}
	\left\{
		\hat{a}_\lambda(\vb{p})
		\vu{e}_\lambda(\vb{p})
		e^{-i\omega(\vb{p})t+i\vb{p}\vdot\vb{x}}
		+
		\hat{a}_\lambda^\dagger(\vb{p})
		\vu{e}_\lambda(\vb{p})^*
		e^{+i\omega(\vb{p})t-i\vb{p}\vdot\vb{x}}
	\right\}
	.
\end{equation}
Application of the phase-rotation operator~\cite{Leonhardt2010,Vogel2006} yields the generalized quadrature operator
\begin{equation}
	\hat{X}(t,\vb{x};\vartheta)
	=
	\sum_{\lambda=1,2}
	\int_{\mathbb{R}^3}\frac{\dd[3]{p}}{(2\pi)^3}
	\frac{1}{\sqrt{2}}
	\left\{
		\hat{a}_\lambda(\vb{p})
		\vu{e}_\lambda(\vb{p})
		e^{-i\omega(\vb{p})t+i\vb{p}\vdot\vb{x}-\vartheta}
		+
		\hat{a}_\lambda^\dagger(\vb{p})
		\vu{e}_\lambda(\vb{p})^*
		e^{+i\omega(\vb{p})t-i\vb{p}\vdot\vb{x}+\vartheta}
	\right\}
	.
\end{equation}
Analog to the Maxwell field operator, we decompose the electric field operator into a positive and negative frequency part
\begin{equation}
	\vu{E}(t,\vb{x})
	=
	\vu{E}^{(-)}(t,\vb{x})
	+
	\vu{E}^{(+)}(t,\vb{x})
	\label{eq:electric_operator}
\end{equation}
where we find the electric field operator components via $\hat{E}^i=\partial_t\hat{A}^i$ leading to
\begin{align}
	\vu{E}^{(-)}(t,\vb{x})
	&=
	-i
	\sum_{\lambda=1,2}
	\int_{\mathbb{R}^3}\frac{\dd[3]{p}}{(2\pi)^3\sqrt{2\omega(\vb{p})}}
	\omega(\vb{p})
	\hat{a}_\lambda(\vb{p})
	\vu{e}_\lambda(\vb{p})
	e^{-i\omega(\vb{p})t+i\vb{p}\vdot\vb{x}}
	\label{eq:electric_negative_operator}
	\\
	\vu{E}^{(+)}(t,\vb{x})
	&=
	+i
	\sum_{\lambda=1,2}
	\int_{\mathbb{R}^3}\frac{\dd[3]{p}}{(2\pi)^3\sqrt{2\omega(\vb{p})}}
	\omega(\vb{p})
	\hat{a}_\lambda^\dagger(\vb{p})
	\vu{e}_\lambda(\vb{p})^*
	e^{+i\omega(\vb{p})t-i\vb{p}\vdot\vb{x}}
	\label{eq:electric_positive_operator}
	.
\end{align}
The 

Inserting the Maxwell and electric field operators into the equal-time commutation relation, we derive the \gls{ccr}
\begin{align}
	\comm{\hat{a}_\lambda(\vb{p})}{\hat{a}_{\lambda^\prime}^\dagger(\vb{q})}
	&=
	(2\pi)^3
	\delta_{\lambda\lambda^\prime}
	\delta^{(3)}\left(\vb{p}-\vb{q}\right)
	\label{eq:canonical_commutation_relation_nonzero}
	\\
	\comm{\hat{a}_\lambda(\vb{p})}{\hat{a}_{\lambda^\prime}(\vb{q})}
	&=
	\comm{\hat{a}_\lambda^\dagger(\vb{p})}{\hat{a}_{\lambda^\prime}^\dagger(\vb{q})}
	=
	0
	\label{eq:canonical_commutation_relation_zero}
	.
\end{align}
The Maxwell field's Hamilton and momentum operator are~\cite[p.~199]{Greiner2013}
\begin{align}
	\hat{H}
	&=
	\sum_{\lambda=1,2}
	\int_{\mathbb{R}^3}\frac{\dd[3]{p}}{(2\pi)^3}
	\omega(\vb{p})
	\hat{a}_\lambda^\dagger(\vb{p})
	\hat{a}_\lambda(\vb{p})
	\label{eq:maxwell_hamilton_operator}
	\\
	\vu{P}
	&=
	\sum_{\lambda=1,2}
	\int_{\mathbb{R}^3}\frac{\dd[3]{p}}{(2\pi)^3}
	\vb{p}
	\hat{a}_\lambda^\dagger(\vb{p})
	\hat{a}_\lambda(\vb{p})
	\label{eq:maxwell_momentum_operator}
\end{align}
and can be found by inserting the Maxwell field operator into the classical definitions of energy and momentum and performing normal ordering.\footnote{Normal ordering removes infinite terms like the "vacuum energy".}
Reading the combination of annihilation and creation operator as particle number density
\begin{equation}
	\hat{n}_\lambda(\vb{p})
	=
	\hat{a}_\lambda^\dagger(\vb{p})
	\hat{a}_\lambda(\vb{p})
	\label{eq:maxwell_number_density_operator}
	,
\end{equation}
the energy and momentum operators then read as the total energy $\omega(\vb{p})$ and momentum $\vb{p}$ weighted by the number density.
\Cref{eq:maxwell_number_density_operator} suggests the total particle number operator to be
\begin{equation}
	\hat{N}
	=
	\int_{\mathbb{R}^3}\frac{\dd[3]{p}}{(2\pi)^3}
	\hat{a}_\lambda^\dagger(\vb{p})
	\hat{a}_\lambda(\vb{p})
	\label{eq:maxwell_number_operator}
	.
\end{equation}
Unsurprisingly, are total particle number and total momentum conserved quantities
\begin{align}
	\comm{\hat{H}}{\vu{P}}
	&=
	\vb{0}
	&
	\comm{\hat{H}}{\vu{N}}
	&=
	0
	.
\end{align}
In the next section, we provide a more rigorous discussion of the particle aspects by constructing the quantum states.
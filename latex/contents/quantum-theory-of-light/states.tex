\section{Quantum states}

In the previous section, we derived the relevant field operators for the Maxwell field encoding electromagnetism.
In the present section, we construct quantum states from these operators in a rather axiomatic approach as done in Ref.~\cite[p.~506]{Cohen2019} for the quantum harmonic oscillator, or in axiomatic field theory~\cite{Streater2016,Haag2012,Bogolubov1989}.

To keep the arguments and notation concise, we restrict the quantum state construction to one polarization mode of the Maxwell field.
The Maxwell field then effectively becomes a Klein-Gordon field with field operator
\begin{equation}
	\hat{A}(t,\vb{x})
	=
	\int_{\mathbb{R}^3}\frac{\dd[3]{p}}{(2\pi)^3\sqrt{2\omega(\vb{p})}}
	\left\{
		\hat{a}(\vb{p})
		e^{-i\omega(\vb{p})t+i\vb{p}\vdot\vb{x}}
		+
		\hat{a}^\dagger(\vb{p})
		e^{+i\omega(\vb{p})t-i\vb{p}\vdot\vb{x}}
	\right\}
	\label{eq:klein_gordon_operator}
\end{equation}
wherein the annihilation and creation operator, $\hat{a}(\vb{p}),\hat{a}^\dagger(\vb{p})$, satisfy the canonical commutation relations
\begin{align}
	\comm{\hat{a}(\vb{p})}{\hat{a}^\dagger(\vb{q})}
	&=
	(2\pi)^3\delta^{(3)}
	\left(\vb{p}-\vb{q}\right)
	\\
	\comm{\hat{a}(\vb{p})}{\hat{a}(\vb{q})}
	&=
	\comm{\hat{a}^\dagger(\vb{p})}{\hat{a}^\dagger(\vb{q})}
	=
	0
	\label{eq:klein_gordon_canonical_commutation_relation}.
\end{align}
To extend the results back to two polarization modes, we can construct a two-dimensional tensor product space from the single polarization mode.

\subsection{Vacuum state}

The fundamental assumption our state construction relies upon is the existence of a unique, up to a constant phase factor, vacuum state $\ket{0}$ invariant under the unitary Poincaré transformation~\cite[p.~97]{Streater2016}
\begin{equation}
	\hat{U}(a,\Lambda)
	\ket{0}
	=
	\ket{0}
	\label{eq:vacuum_invariance}
\end{equation}
where $\Lambda$ denotes a Lorentz transformation and $a$ a spacetime translation.
The vacuum state is an element of a one-dimensional complex Hilbert space, $\mathcal{H}^{(0)}=\mathcal{H}(\mathbb{C})$, the zero-particle state space.
The generator of the unitary spacetime translation is the four-momentum operator $\hat{P}^\mu=\left(\hat{H},\vu{P}\right)$~\cite[p.~28]{Haag2012}
\begin{equation}
	\hat{U}(a)
	=
	\hat{U}(a,\mathbb{1})
	=
	e^{i\hat{P}_\mu a^\mu}
	\label{eq:spacetime_translation}
	.
\end{equation}
The invariance of the vacuum under spacetime translations, \cref{eq:spacetime_translation}, implies that the vacuum is a zero eigenstate to the Hamilton and momentum operator
\begin{align}
	\hat{H}
	\ket{0}
	&=
	0
	&
	\vu{P}
	\ket{0}
	&=
	\vb{0}
	\label{eq:vacuum_energy_momentum}
	.	
\end{align}
From the mode expansion of the Hamilton operator, \cref{eq:maxwell_hamilton_operator}, and the vacuum state being a zero eigenstate to the Hamilton operator, we conclude that the vacuum state is also a zero eigenstate to the annihilation operator
\begin{equation}
	\hat{a}(\vb{p})
	\ket{0}
	=
	0
	\label{eq:vacuum_annihilation}
	.
\end{equation}
In the next paragraph, we motivate why we understand the annihilation and creation operators as adding or removing a particle excitation to and from the field.
Under these circumstances, we can read \cref{eq:vacuum_annihilation} as destroying the vacuum state, ensuring no negative energy or negative particle number states.

\subsection{Particle states}

The motivation of why the annihilation and creation operators add or remove a particle with energy and momentum to and from the field follows from applying the commutators of the number and momentum operator with the creation operator
\begin{align}
	\comm{\hat{N}}{\hat{a}^\dagger(\vb{p})}
	&=
	\hat{a}^\dagger(\vb{p})
	&
	\comm{\vu{P}}{\hat{a}^\dagger(\vb{p})}
	&=
	\vb{p}
	\hat{a}^\dagger(\vb{p})
	.
\end{align}
Applying the vacuum state $\ket{0}$ to the right of the commutator equations, yields eigenvalue equations for the number and momentum operator
\begin{align}
	\hat{N}
	\hat{a}^\dagger(\vb{p})
	\ket{0}
	&=
	1
	\hat{a}^\dagger(\vb{p})
	\ket{0}
	&
	\vu{P}
	\hat{a}^\dagger(\vb{p})
	\ket{0}
	&=
	\vb{p}
	\hat{a}^\dagger(\vb{p})
	\ket{0}
	.
\end{align}
The eigenvalue equations suggest
\begin{equation}
	\ket{\vb{p}}
	=
	\hat{a}^\dagger(\vb{p})
	\ket{0}
	\label{eq:momentum_state}
\end{equation}
to be a single-particle state with momentum $\vb{p}$ and energy $\omega(\vb{p})$, a momentum state.~\cite[p.~23]{Peskin1995}.
Unfortunately, the inner product between two momentum states does not yield a complex number $\mathbb{C}$ but, a distribution,
\begin{equation}
	\braket{\vb{p}}{\vb{q}}
	=
	\bra{0}
	\comm{\hat{a}(\vb{p})}{\hat{a}^\dagger(\vb{q})}
	\ket{0}
	=
	(2\pi)^3
	\delta^{(3)}(\vb{p}-\vb{q})
\end{equation}
suggesting that something essential is missing in our description.

It makes sense to take a step back and recap some mathematical context regarding distributions.\footnote{See Ref.~\cite[p.~590]{Zeidler2016} and Ref.~\cite[p.~193]{Mukhanov2007} for a mathematical discussion of distributions in a physical context.}
One approach considers distributions as functionals, i.e., maps from a function space, e.g., the space of real-valued square-integrable functions $L^2(\mathbb{R})$, to real numbers $\mathbb{R}$.
Implicitly, we already used functionals when we considered the action integral
\begin{equation}
	\hat{S}\left[x(t)\right]
	=
	\int_{t_0}^{t_1}\dd{t}
	L\left(x(t),\dot{x}(t)\right)
	,
\end{equation}
wherein $L$ is some classical Lagrangian, evaluated for some finite time interval $[t_0,t_1]$ maps the trajectory $x(t)$ of a point particle to a real number $\mathbb{R}$.
Often a functional $A$ acting on a function $f$ is written
\begin{equation}
	A[f]
	=
	\int\dd{x}
	f(x)A(x)
	\label{eq:functional}
\end{equation}
wherein $A(x)$ is denoted the integration kernel representing the functional, which may be an ordinary function or a distribution.
For example, the delta distribution $\delta(x-y)$ is the integration kernel of the functional $\delta_y$
\begin{equation}
	\delta_{y}[f]
	=
	\int\dd{x}
	f(x)
	\delta(x-y)
	=
	f(y)
	.
\end{equation}
Fourier transforms are another class of functionals we already used frequently.
Linear functionals share many convenient properties with ordinary functions and physicists often skip the distinction.
In axiomatic quantum field theory, the quantum field operators are precisely defined as operator-valued tempered distributions mapping from the space smearing or test functions $\mathcal{S}(\mathbb{R}^4)$ to the the set a set of operators defined on the corresponding Hilbert space $\mathcal{O}(\mathcal{H})$~\cite[p.~56]{Haag2012}.
A typical functional space for smearing functions is the Schwartz space, a subset of the space of square-integrable functions $L^2$, which rapidly fall off at infinitely, a property which we exploit for partial integration with vanishing boundary terms of the action integral.\footnote{Typical Schwartz functions are Gaussian functions multiplied with a monomial, e.g., $x^ne^{-a\norm{x}^2}$ where $n\in\mathbb{N}_0$ and $a\in\mathbb{R}_+$.}

Let us reinterpret the (positive frequency) field operator with this mathematical background by considering its action on a smearing function, i.e.,
\begin{equation}
	\hat{A}^{(+)}[f]
	=
	\int\dd[4]{x}
	f(t,\vb{x})
	\hat{A}^{(+)}(t,\vb{x})
	=
	\int\frac{\dd[3]{p}}{(2\pi)^3\sqrt{2\omega(\vb{p})}}
	f\left(\omega(\vb{p}),\vb{p}\right)
	\hat{a}^\dagger(\vb{p})
\end{equation}
where we inserted the plane-wave expansion for the field and the spacetime Fourier transform for the smearing function and used the orthogonality of the Fourier modes in the second equation.
Applying the smeared positive frequency field operator to the vacuum state,
\begin{equation}
	\begin{split}
		\ket{1_f}
		=
		\hat{A}^{(+)}[f]
		\ket{0}
		,
	\end{split}
	\label{eq:single_particle_state}
\end{equation}
and comparing the result to the momentum state, \cref{eq:momentum_state}, we find the function $f$ to smear the function in momentum and spacetime space~\cite[p.~35]{Srednicki2007}.\footnote{Interestingly though, only momentum components satisfying the energy-momentum relation, \cref{eq:energy_momentum_relation}, contribute to momentum space.}
The inner product of two such smeared states yields a complex number $\mathbb{C}$
\begin{equation}
	\braket{1_f}{1_g}
	=
	\int\frac{\dd[3]{p}}{(2\pi)^32\omega(\vb{q})}
	f(\vb{p})^*
	g(\vb{p})
\end{equation}
implying the smeared state $\ket{1_f}$ being normalizable if we require the smearing function to satisfy
\begin{equation}
	\braket{1_f}{1_f}
	=
	\int\frac{\dd[3]{p}}{(2\pi)^32\omega(\vb{q})}
	\abs*{f(\vb{p})}^2
	=
	1
	\label{eq:spectrum_normalization}
	.
\end{equation}
With the normalization condition imposed, the smeared state $\ket{1_f}$ is an eigenstate of the number operator to eigenvalue one
\begin{equation}
	\hat{N}
	\ket{1_f}
	=
	1
	\ket{1_f}
\end{equation}
suggesting the smeared state $\ket{1_f}$ to the physical single-particle state, we were looking for.
The smeared state $\ket{1_f}$ is not an eigenstate of the energy and momentum operator anymore but has expectation values
\begin{align}
	\bra{1_f}
	\hat{H}
	\ket{1_f}
	&=
	\int\frac{\dd[3]{p}}{(2\pi)^3}
	\omega(\vb{p})
	\abs*{f\left(\omega(\vb{p}),\vb{p}\right)}^2
	\\
	\bra{1_f}
	\vu{P}
	\ket{1_f}
	&=
	\int\frac{\dd[3]{p}}{(2\pi)^3}
	\vb{p}
	\abs*{f\left(\omega(\vb{p}),\vb{p}\right)}^2
\end{align}
suggesting that the smearing function has the physical interpretation of a frequency spectrum.

Let $\ket{1_f}$ be a smeared particle state, then the wave function $\Psi(t,\vb{x})$ is the probability amplitude of finding the particle at a spacetime coordinate $(t,\vb{x})$~\cite[p.~24]{Peskin1995}, i.e.,
\begin{equation}
	\Psi(t,\vb{x})
	=
	\bra{0}
	\hat{A}(t,\vb{x})
	\ket{1_f}
	=
	\int\dd{t^\prime}\dd[3]{x^\prime}
	D(t-t^\prime,\vb{x}-\vb{x}^\prime)
	f(t^\prime,\vb{x}^\prime)
\end{equation}
wherein $f(t,\vb{x})$ is the spacetime representation of the (initial) smearing function and $D(t,\vb{x})$ is the propagator as defined in Ref.~\cite[p.~27]{Peskin1995}.
Given the wave function, the relativistic probability current
\begin{equation}
	j_\mu(t,\vb{x})
	=
	2\Im\left\{
		\Psi(t,\vb{x})^*
		\partial_\mu
		\Psi(t,\vb{x})
	\right\}
\end{equation}
allows us to estimate the center-of-mass position and velocity of the particle, i.e.,
\begin{align}
	\expval{\vb{x}(t)}
	&=
	\int\dd[3]{x}
	\vb{x}
	\rho(t,\vb{x})
	&
	\expval{\vb{v}(t)}
	&=
	\int\dd[3]{x}
	\vb{j}(t,\vb{x})
\end{align}
wherein $\rho(t,\vb{x})=j_0(t,\vb{x})$ is the relativistic probability density.
For more details on the properties of relativistic wave packets, e.g., dispersion, see Ref.~\cite{Naumov2013} and Ref.~\cite{Naumov2009}.

To summarize our findings, we first motivated momentum eigenstates from the commutation algebra.
However, the momentum eigenstates are prone to mathematical inconsistencies, following that the momentum states are strictly speaking distributions, not functions.
Physically, the momentum states correspond to unphysical plane-waves.
A mathematical consistent single-particle state requires a momentum spectrum.
The momentum spectrum encodes many important physical properties like the localization and velocity of the particle.

\subsection{Fock space}

The single-particle state defined in \cref{eq:single_particle_state} is an element of the one-particle Hilbert space of square-integrable functions defined on three-dimensional space $\mathcal{H}^{(1)}=\mathcal{H}\left(L^2(\mathbb{R}^3)\right)$.
The generalization of the one-particle Hilbert space $\mathcal{H}^{(1)}$ to an $n$-particle Hilbert space $\mathcal{H}^{(n)}$ is the tensor product of one-particle Hilbert spaces
\begin{equation}
	\mathcal{H}^{(n)}
	=
	\bigotimes^n_{i=1}
	\mathcal{H}^{(1)}
	.
\end{equation}
Now, it is possible to have a superposition of, e.g., the vacuum state and a particle state
\begin{equation}
	\ket{\psi}
	=
	c_1
	\ket{0}
	+
	c_2
	\ket{1_f}
\end{equation}
with $c_1,c_2\in\mathbb{C}$ which means that we need to combine orthonormal $n$-particle states.
We first construct a tensor algebra over the Hilbert space $\mathcal{H}^{(1)}$ as the direct sum~\cite[p.~290]{Bogolubov1989}
\begin{equation}
	\bigoplus^\infty_{n=0}
	S_+
	\mathcal{H}^{(n)}
\end{equation}
wherein $S_+$ symmetrizes the Hilbert space for bosons.
Equipping the tensor algebra with an inner product and using the completeness of the $n$-particle Hilbert spaces, we obtain again a Hilbert space, named the symmetric Fock space $\mathcal{F}_+$~\cite[p.~35]{Haag2012}.

\subsection{Number states}

Applying the creation operator $\hat{A}^{(+)}[f]$ wherein $f$ is a smearing function or momentum spectrum satisfying the normalization condition, \cref{eq:spectrum_normalization}, suggests defining
\begin{equation}
	\ket{n_f}
	=
	\frac{1}{\sqrt{n!}}
	\hat{A}^{(+)}[f]^n
	\ket{0}
	\label{eq:number_state}
\end{equation}
as number state with spectrum $f$.\footnote{The factorial is required for normalization because bosons are indistinguishable.}
The positive and negative frequency field operators $\hat{A}^{(\pm)}(t,\vb{x})$ generalize the quantum harmonic annihilation and creation operators by adding or removing a particle with spectrum $f$ from the field
\begin{align}
	\hat{A}^{(+)}[f]
	\ket{n_f}
	&=
	\sqrt{n+1}
	\ket{{n+1}_f}
	\\
	\hat{A}^{(-)}[f]
	\ket{n_f}
	&=
	\sqrt{n}
	\ket{{n-1}_f}
	.
\end{align}
While the generalized number state $\ket{n_f}$ is still an eigenstate of the number operator to eigenvalue $n_f$,
\begin{equation}
	\hat{N}
	\ket{n_f}
	=
	n_f
	\ket{n_f}
	,
\end{equation}
it is not anymore an eigenstate of the energy and momentum operator but has energy and momentum expectation values
\begin{align}
	\bra{n_f}
	\hat{H}
	\ket{n_f}
	&=
	n
	\int\frac{\dd[3]{p}}{(2\pi)^3}
	\omega(\vb{p})
	\abs*{\frac{f\left(\omega(\vb{q}),\vb{q}\right)}{\sqrt{2\omega(\vb{p})}}}^2
	\\
	\bra{n_f}
	\vu{P}
	\ket{n_f}
	&=
	n
	\int\frac{\dd[3]{p}}{(2\pi)^3}
	\vb{p}
	\abs*{\frac{f\left(\omega(\vb{q}),\vb{q}\right)}{\sqrt{2\omega(\vb{p})}}}^2
	.
\end{align}
The expectation value and variance of the electric field operator are
\begin{align}
	\bra{n_f}
	\hat{E}(t,\vb{x})
	\ket{n_f}
	&=
	0
	\\
	\bra{n_f}
	\left(
		\Delta
		\hat{E}(t,\vb{x})
	\right)^2
	\ket{n_f}
	&=
	\frac{1}{2}
	\int\frac{\dd[3]{p}}{(2\pi)^3}
	\omega(\vb{p})
	+
	\frac{1}{2}
	\abs*{\Psi(t,\vb{x})}^2
	.
\end{align}
The electric field vanishes for our number states as known in quantum optics, see, for instance, Ref.~\cite{Gerry2005}, but the variance contains an additional term to the "vacuum fluctuations" from the momentum spectrum.
The vacuum fluctuations are in principle infinite, however, our detector is only able to detect a limited bandwidth which makes the vacuum fluctuations in practical applications finite again.

\subsection{Coherent states}

The interaction of a classical current $\vb{j}(t,\vb{x})$ with the Maxwell field operator in the Coulomb gauge $\vu{A}(t,\vb{x})$ is given by the interaction Hamiltonian
\begin{equation}
	\hat{H}_\text{int}(t)
	=
	-
	\int\dd[3]{x}
	\vb{j}(t,\vb{x})
	\vdot
	\hat{\vb{A}}(t,\vb{x})
	.
\end{equation}
Inserting the spatial Fourier transform of the current $\vb{j}(t,\vb{p})$ and the quantum mode mode expansion, \cref{eq:maxwell_positive_operator,eq:maxwell_negative_operator}, the interaction Hamiltonian becomes
\begin{equation}
	\hat{H}_\text{int}(t)
	=
	-
	\sum_{\lambda=1,2}
	\int\frac{\dd[3]{p}}{(2\pi)^3\sqrt{2\omega(\vb{p})}}
	\left\{
		\left(
			\vb{j}(t,\vb{p})^*
			\vdot
			\vu{e}_\lambda(\vb{p})
		\right)
		\hat{a}_\lambda(\vb{p})
		e^{-i\omega(\vb{p})t}
		+
		\text{h.c.}
	\right\}
	\label{eq:classical_current_interaction}
	.
\end{equation}
where we used the conjugate symmetry $\vb{j}(t,\vb{p})^*=\vb{j}(t,-\vb{p})$.

The effect of an interaction acting on a quantum state from time $t_0$ to $t$ is encoded in the time-evolution operator\footnote{See Ref.~\cite[p.~215]{Greiner2013} for an introduction into the time-evolution operator and interactions.}
\begin{equation}
	\hat{U}(t_0,t)
	=
	\mathcal{T}_+
	\exp\left\{
		-i
		\int_{t_0}^t\dd{t^\prime}
		\hat{H}_\text{int}(t^\prime)
	\right\}
	\label{eq:time_evolution_operator}
\end{equation}
wherein $\mathcal{T}_+$ denotes the time-ordering symbol.
The Magnus expansion presents a systematic approach in finding an explicit form of the time-evolution operator\footnote{See Ref.~\cite[p.~42]{QuesadaMejia2015}, for an introduction to the Magnus expansion with application to nonlinear processes.}, it is given by
\begin{equation}
	\hat{U}(t_0,t)
	=
	\exp\left\{
		\sum_{n=1}
		\Omega^{(n)}(t_0,t)
	\right\}
\end{equation}
wherein the first two terms are given by
\begin{align}
	\hat{\Omega}^{(1)}(t_0,t)
	&=
	-i
	\int_{t_0}^t\dd{t^\prime}
	\hat{H}_\text{int}(t^\prime)
	\\
	\hat{\Omega}^{(2)}(t_0,t)
	&=
	\frac{(-i)^2}{2!}
	\int_{t_0}^t\dd{t^\prime}
	\int_{t_0}^{t^\prime}\dd{t^{\prime\prime}}
	\comm{\hat{H}_\text{int}(t^\prime)}{\hat{H}_\text{int}(t^{\prime\prime})}
	.
\end{align}
For some interactions there exists no exact solution and we can truncate the expansion up to some finite term.
Compared to other expansions, e.g. the Neumann expansion, the truncated Magnus expansion is still unitary.

Let us apply the Magnus expansion to find the time-evolution operator corresponding to the interaction Hamiltonian of \cref{eq:classical_current_interaction}.
The first term of the Magnus expansion turns out to be
\begin{equation}
	\hat{\Omega}^{(1)}(t_0,t)
	=
	i
	\sum_{\lambda=1,2}
	\int_{\mathbb{R}^3}\frac{\dd[3]{p}}{(2\pi)^3\sqrt{2\omega(\vb{p})}}
	\left\{
		J_\lambda(t,t_0;\vb{p})
		\hat{a}_\lambda(\vb{p})
		+
		\text{h.c.}
	\right\}
\end{equation}
where we defined the time-integrated current for polarization $\lambda$
\begin{equation}
	J_\lambda(t_0,t;\vb{p})
	=
	\int_{t_0}^t\dd{t^\prime}
	\left(
		\vb{j}(t,\vb{p})^*
		\vdot
		\vu{e}_\lambda(\vb{p})
	\right)
	e^{-i\omega(\vb{p})t^\prime}
	.
\end{equation}
The second term in the Magnus expansion turns out to be complex
\begin{equation}
	\hat{\Omega}^{(2)}(t_0,t)
	=
	i\sum_{\lambda=1,2}
	\int_{\mathbb{R}^3}\frac{\dd[3]{p}}{(2\pi)^3\omega(\vb{p})}
	\Im\left\{
		J_\lambda(t_0,t^\prime;\vb{p})
		J_\lambda(t_0,t^{\prime\prime};\vb{p})^*
	\right\}
\end{equation}
which only contributes a phase to the time-evolution operator.
As the second commutator is complex-valued, higher order commutators vanish and the Magnus expansion is exact with the first two terms.
As long as we consider a single current source, no interference of phases can occur and we can ignore the complex phase originating from the second Magnus coefficient.
The time-evolution operator of the Maxwell field interacting with a classical source current therefore is~\cite[p.~168]{Itzykson2012}
\begin{equation}
	\hat{U}(t_0,t)
	=
	\exp\left\{
		i\sum_{\lambda=1,2}
		\int_{\mathbb{R}^3}\frac{\dd[3]{p}}{(2\pi) ^3\sqrt{2\omega(\vb{p})}}
		\left\{
			J_\lambda(t,t_0;\vb{p})
			\hat{a}_\lambda(\vb{p})
			+
			\text{h.c.}
		\right\}
	\right\}
	.
\end{equation}
Neglecting the polarization 
\begin{equation}
	\hat{U}(t_0,t)
	=
	\exp\left\{
		\int_{\mathbb{R}^3}\frac{\dd[3]{p}}{(2\pi) ^3\sqrt{2\omega(\vb{p})}}
		\left\{
			i
			J(t,t_0;\vb{p})
			\hat{a}(\vb{p})
			-
			\text{h.c.}
		\right\}
	\right\}
\end{equation}
we identify the time-evolution operator with the generalization of the displacement operator from quantum optics~\cite[p.~xx]{Barnett2002}
\begin{equation}
	\lim_{t\to\infty}\hat{U}(-t,+t)
	=
	\hat{D}\left[-iJ(\vb{p})\right]
\end{equation}
where we take the generalized displacement operator to be
\begin{equation}
	\begin{split}
		\hat{D}[\alpha]
		&=
		\exp\left\{
			\hat{A}^{(+)}[\alpha]
			-
			\hat{A}^{(-)}[\alpha^*]
		\right\}
		\\
		&=
		\exp\left\{
			\int_{\mathbb{R}^3}\frac{\dd[3]{p}}{(2\pi) ^3\sqrt{2\omega(\vb{p})}}
			\left\{
				\alpha(\vb{p})
				\hat{a}^\dagger(\vb{p})
				-
				\alpha(\vb{p})^*
				\hat{a}(\vb{p})
			\right\}
		\right\}
	\end{split}
	\label{eq:displacement_operator}
\end{equation}
where we identified the generalized field creation and annihilation operators, $\hat{A}^{(+)}[\alpha]$ and $\hat{A}^{(-)}[\alpha^*]$.
Noting that
\begin{equation}
	\comm{\hat{A}^{(+)}[\alpha]}{\hat{A}^{(-)}[\alpha^*]}
	=
	\int\frac{\dd[3]{p}}{(2\pi)^32\omega(\vb{p})}
	\abs*{\alpha(\vb{p})}
\end{equation}
we can employ the \gls{bch} formula as in Ref.~\cite[p.~xx]{Barnett2002} to write the displacement operator in normal-order
\begin{equation}
	\hat{D}[\alpha]
	=
	\exp\left\{
		-
		\frac{1}{2}
		\comm{\hat{A}^{(+)}[\alpha]}{\hat{A}^{(-)}[\alpha^*]}
	\right\}
	\exp\left\{
		+
		\hat{A}^{(+)}[\alpha]
	\right\}
	\exp\left\{
		-
		\hat{A}^{(-)}[\alpha]
	\right\}
	.
\end{equation}

Now, let us discuss some properties of the displacement operator.
The product of two different displacements is equal to the sum of the displacement times a suppression factor depending on the overlap for the displacement, i.e.,
\begin{equation}
	\begin{split}
		\hat{D}[\alpha]
		\hat{D}[\beta]
		&=
		\hat{D}[\alpha+\beta]
		\exp\left\{
			-
			\frac{1}{2}
			\comm{\hat{A}^{(+)}[\alpha]}{\hat{A}^{(-)}[\beta^*]}
			+
			\frac{1}{2}
			\comm{\hat{A}^{(+)}[\beta]}{\hat{A}^{(-)}[\alpha^*]}
		\right\}
		\\
		&=
		\hat{D}[\alpha+\beta]
		\exp\left\{
			-
			\frac{1}{2}
			\int\frac{\dd{p}}{(2\pi)^32\omega(\vb{p})}
			\left\{
				\alpha(\vb{p})
				\beta(\vb{p})^*
				-
				\alpha(\vb{p})^*
				\beta(\vb{p})
			\right\}
		\right\}
		.
	\end{split}
\end{equation}
Using the product formula, we can quickly show that the displacement operator is unitary
\begin{equation}
	\hat{D}[\alpha]
	\hat{D}[\alpha]^\dagger
	=
	\hat{D}[\alpha]
	\hat{D}[-\alpha]
	=
	\mathbb{1}
\end{equation}
which is not too surprising given the time-evolution is unitary.

The radiation emitted by a classical current is coherent, suggesting to identify the quantum state produced by a classical current as the coherent state
\begin{equation}
	\ket{\alpha}
	=
	\hat{D}[\alpha]
	\ket{0}
	=
	\exp\left\{
		-
		\frac{1}{2}
		\comm{\hat{A}^{(+)}[\alpha]}{\hat{A}^{(-)}[\alpha^*]}
	\right\}
	\exp\left\{
		\hat{A}^{(+)}[\alpha]
	\right\}
	\ket{0}
	\label{eq:coherent_state}
\end{equation}
where we used that the exponential of the generalized annihilation operator acting on the vacuum state produces the vacuum state.
Coherent states are non-orthogonal, i.e.,
\begin{equation}
	\begin{split}
		\braket{\alpha}{\beta}
		&=
		\exp\left\{
			-
			\frac{1}{2}
			\comm{\hat{A}^{(+)}[\alpha]}{\hat{A}^{(-)}[\alpha^*]}
			-
			\frac{1}{2}
			\comm{\hat{A}^{(+)}[\beta]}{\hat{A}^{(-)}[\beta^*]}
			+
			\comm{\hat{A}^{(+)}[\beta]}{\hat{A}^{(-)}[\alpha^*]}
		\right\}
		\\
		&=
		\exp\left\{
			-
			\frac{1}{2}
			\int\frac{\dd{p}}{(2\pi)^32\omega(\vb{p})}
			\left\{
				\abs*{\alpha(\vb{p})}^2
				+
				\abs*{\beta(\vb{p})}^2
				-
				2\alpha(\vb{p})^*\beta(\vb{p})
			\right\}
		\right\}
		\label{eq:coherent_state_inner}
		.
	\end{split}
\end{equation}
The coherent state is an eigenstate to the annihilation operator
\begin{equation}
	\hat{a}(\vb{p})
	\ket{\alpha}
	=
	\frac{\alpha(\vb{p})}{\sqrt{2\omega(\vb{p})}}
	\ket{\alpha}
	\label{eq:coherent_state_annihilation}
\end{equation}
which makes it simple to drive the expectation value of the Hamiltonian operator
\begin{equation}
	\bra{\alpha}
	\hat{H}
	\ket{\alpha}
	=
	\int\frac{\dd{p}}{(2\pi)^3}
	\omega(\vb{p})
	\abs*{\frac{\alpha(\vb{p})}{\sqrt{2\omega(\vb{p})}}}^2
	\label{eq:coherent_state_hamilton_expval}
\end{equation}
and its variance
\begin{equation}
	\bra{\alpha}
	\left(
		\Delta
		\hat{H}
	\right)^2
	\ket{\alpha}
	=
	\int\frac{\dd{p}}{(2\pi)^3}
	\omega(\vb{p})^2
	\abs*{\frac{\alpha(\vb{p})}{\sqrt{2\omega(\vb{p})}}}^2
	\label{eq:coherent_state_hamilton_variance}
	.
\end{equation}
For the number operator we find the mean to equal the variance
\begin{align}
	\bra{\alpha}
	\hat{N}
	\ket{\alpha}
	&=
	\int\frac{\dd{p}}{(2\pi)^3}
	\abs*{\frac{\alpha(\vb{p})}{\sqrt{2\omega(\vb{p})}}}^2
	=
	\overline{n}
	\label{eq:coherent_state_number_expval}
	\\
	\bra{\alpha}
	\left(
		\Delta
		\hat{N}
	\right)^2
	\ket{\alpha}
	&=
	\overline{n}^2
	\label{eq:coherent_state_number_variance}
	,
\end{align}
i.e., the photon number to be Poisson distributed, which follows simply by setting $\omega(\vb{p})=1$ in the results obtained for the Hamilton operator.
The expectation value of the electric field operator reads
\begin{equation}
	\begin{split}
		\bra{\alpha}
		\hat{E}(t,\vb{x})
		\ket{\alpha}
		&=
		\int\frac{\dd[3]{p}}{(2\pi)^3}
		\sqrt{2\omega(\vb{p})}
		\Im\left\{
			\alpha(\vb{p})
			e^{-i\omega(\vb{p})t+i\vb{p}\vdot\vb{x}}
		\right\}
		\\
		&=
		\int\frac{\dd[3]{p}}{(2\pi)^3}
		\sqrt{2\omega(\vb{p})}
		\abs*{\alpha(\vb{p})}
		\sin\left(
			\vb{p}\vdot\vb{x}
			-
			\omega(\vb{p})t
			+
			\varphi
		\right)
	\end{split}
	\label{eq:coherent_state_electric_expval}
\end{equation}
where we used the polar representation $\alpha(\vb{p})=\abs*{\alpha(\vb{p})}e^{i\varphi}$.
The inner product of a coherent state with a number state yields
\begin{equation}
	\braket{n_f}{\alpha}
	=
	\frac{1}{\sqrt{n!}}
	e^{-\overline{n}/2}
	\left(
		\int\frac{\dd{p}}{(2\pi)^32\omega(\vb{p})}
		f\left(\omega(\vb{p}),\vb{p}\right)^*
		\alpha(\vb{p})
	\right)^n
	\label{eq:coherent_state_inner_number}
	.
\end{equation}
We have derived the generalized coherent state from the interaction of the Maxwell field with a classical current where found the time-evolution operator to yield the displacement operator.
The generalized coherent state and displacement operators share the same properties as their single-mode quantum optics counterparts which is not too surprising given that the modes are independent of another.
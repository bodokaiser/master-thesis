\section{Coupler}

Optical couplers, including optical splitters, redistribute two optical inputs among two outputs and are an essential passive component for almost every setup.\footnote{The optical splitter is a special case of the optical coupler where one of the two optical inputs is zero, or, more precisely, the vacuum state.}

A plethora of approaches towards the (quantum) beam splitter exists~\cite{Leonhardt2010,Gerry2005,Loudon2000} but are vague on assumptions, which, if not discussed, lead to misconception and confusion, for instance, regarding the phase and energy conservation.
We would therefore approach the optical coupler, or beam splitter, from two directions:
First, we approach the free-ray beam splitter from an experimentalist's perspective, considering the reflection and transmission properties of a beam splitter.
Second, we discuss the fiber coupler from a theoretical mode-coupling perspective leading to an interaction Hamiltonian.
Of course, both paths lead to the same unitary transformations, which we present in matrix and operator form.
Finally, we discuss the input-output relations for coherent states and the interpretation of a frequency-dependent beam splitter as an optical filter.

\subsection{Free-ray beam splitter}

The most commonly employed (free-ray) designs of the beam splitter are the cubic, plate, and pellicle beam splitters, see \Cref{fig:beam_splitter_types}.
\begin{figure}[htb]
    \centering
    \includegraphics{figures/tikz/beam-splitter-types}
    \caption{Different types of free-ray beam splitters: (a) Cubic beam splitter made of two triangular prisms glued at their base (grey). (b) Plate beam splitter made of a dielectric plate. (c) Pellicle beam splitter made from a thin membrane.}\label{fig:beam_splitter_types}
\end{figure}
The cubic beam splitter is made of two triangular prisms.
The interface between the two prisms is finished with a dielectric coating.
The outward-facing surface of the prisms is grafted with an \gls{ar} coating.\footnote{The incident angle of the electric field is perpendicular to the surface of the cubic beam splitter. As the reflection angle is equal to the incidence angle, we have back-reflection of the input fields.}
he pellicle beam splitter consists of a few micrometer thin membrane, optionally with a one-sided coating.
The plate beam splitter is like a thick pellicle beam splitter made of glass.

To deduce the relation between the in- and output fields, we sequentially couple a laser pulse into each input while monitoring both outputs with a spectrum analyzer, see \Cref{fig:beam_splitter_inputs_outputs}.
Assuming the beam splitter to be an \gls{lti} system, knowing the spectral shape of the laser pulse lets us infer the frequency responses of the beam splitter.
\begin{figure}[htb]
    \centering
    \includegraphics{figures/tikz/beam-splitter-cubic-plate}
    \caption{Cubic (left) and plate beam splitter (right) with the two input fields, $\hat{E}_1(\omega)$ and $\hat{E}_2(\omega)$, and two output fields, $\hat{E}_1^\prime(\omega)$ and $\hat{E}_2^\prime(\omega)$, labelled by the momentum representation of the electric field operators.}\label{fig:beam_splitter_inputs_outputs}
\end{figure}
Invoking the superposition principle for electromagnetic waves, we find the frequency responses of the beam splitter to relate the electric fields by
\begin{equation}
    \begin{pmatrix}
        \expval{\hat{E}_1^\prime(\omega)} \\
        \expval{\hat{E}_2^\prime(\omega)}
    \end{pmatrix}
    =
    \begin{pmatrix}
        t(\omega) & r^\prime(\omega)
        \\
        r(\omega) & t^\prime(\omega)
    \end{pmatrix}
    \begin{pmatrix}
		\expval{\hat{E}_1(\omega)} \\
        \expval{\hat{E}_2(\omega)}
    \end{pmatrix}
    \label{eq:beam_splitter_expval}
\end{equation}
wherein $r(\omega),r^\prime(\omega)$ and $t(\omega),t^\prime(\omega)$ are the complex reflection respective transmission coefficients.
The absolute values of the transmission, $\abs{t(\omega)}$ and $\abs{t^\prime(\omega)}$, and reflection coefficients, $\abs{r(\omega)}$ and $\abs{r^\prime(\omega)}$, determine the splitting ratio of the input power among the outputs.
The complex phase factor of the reflection and transmission coefficients characterizes the phase shifts the output fields concerning the input fields.
The beam splitter is a passive device implying the output energy to be bound by the input energy
\begin{equation}
    \abs{\expval{\hat{E}_1^\prime(\omega)}}^2
    +
    \abs{\expval{\hat{E}_2^\prime(\omega)}}^2
    \leq
    \abs{\expval{\hat{E}_1(\omega)}}^2
    +
    \abs{\expval{\hat{E}_2(\omega)}}^2
    \label{eq:beam_splitter_passive}
    ,
\end{equation}
or equivalently, constraining the reflection and transmission coefficients by
\begin{align}
    \abs{r(\omega)}^2+\abs{t(\omega)}^2
    &\leq
    1,
    &
    \abs{r^\prime(\omega)}^2+\abs{t^\prime(\omega)}^2
    &\leq
    1
    \label{eq:beam_splitter_coefficients_constraint}
    .
\end{align}
The equality of these inequalities is only true for lossless devices for which there is no back-scattering.\footnote{Using an optical circulator it is in principle possible to measure all \num{16} scattering parameters.}
Sometimes, one finds the claim~\cite[p.~129]{Haroche2006} that the matrix transformation in \cref{eq:beam_splitter_expval} is required to be symmetric (or reciprocal) due to Maxwell's equations.
However, only optical systems with a single dielectric layer are reciprocal~\cite{Potton2004}, but most physical beam splitters comprise multiple dielectric layers.\footnote{For example, cubic beam splitters typically have a coating followed by optical cement between the prisms breaking reciprocal symmetry of the system.}
It is possible to derive exact expressions of the complex reflection, $r(\omega),r^\prime(\omega)$, and transmission coefficients, $t(\omega),t^\prime(\omega)$ using classical wave optics and perfect knowledge of the dimensions and material properties.
For example, Hénault~\cite{Henault2015} derived an exact expression for the reflected and transmitted amplitudes of a plate beam splitter with one input and a single dielectric layer.
Likewise, Hamilton~\cite{Hamilton2000} discusses the cubic beam splitter with two inputs and different coatings.
In general, the complex reflection and transmission coefficients need to account for multiple reflections at different dielectric layers inside the beam splitter.

Inserting the mode expansion of the electric field operators, \cref{eq:electric_operator,eq:electric_negative_operator,eq:electric_positive_operator}, and using the linearity of the device and the expectation value, we recover the transformation for the annihilation operators, sometimes referred to as quantum modes,
\begin{equation}
    \begin{pmatrix}
        \hat{a}_1^\prime(\omega) \\
        \hat{a}_2^\prime(\omega)
    \end{pmatrix}
    =
    \begin{pmatrix}
        t(\omega) & r^\prime(\omega)
        \\
        r(\omega) & t^\prime(\omega)
    \end{pmatrix}
    \begin{pmatrix}
        \hat{a}_1(\omega) \\
        \hat{a}_2(\omega)
    \end{pmatrix}
    \label{eq:beam_splitter_annihilation}
\end{equation}
in agreement with Refs.~\cite{Leonhardt2010,Gerry2005}.

\subsection{Fiber and waveguide coupler}

Contrary to the direct coupling in free-ray beam splitters, a fiber or waveguide coupler uses indirect coupling through the evanescent field.
The evanescent field of an electromagnetic field does not propagate but decays exponentially.
We often observe evanescent fields at the boundary of waveguiding structures.
One must bring the waveguides in proximity for the evanescent fields of two waveguided modes to couple efficiently.
The range where the waveguides are close is the interaction length $l$, see \Cref{fig:waveguide_coupler}.
Over the interaction length, the two energy of the field modes oscillates back and forth between the two waveguides.
\begin{figure}[htb]
    \centering
    \includegraphics{figures/tikz/waveguide-coupler}
    \caption{Waveguide coupler with input quantum modes $\hat{a}_1(\omega)$ and $\hat{a}_2(\omega)$ coupled evanescent over an interaction length $l$ yielding the output quantum modes $\hat{a}_1^\prime(\omega)$ and $\hat{a}_2^\prime(\omega)$.}\label{fig:waveguide_coupler}
\end{figure}
The weak coupling through evanescent fields is conceptionally similar to weakly coupled harmonic oscillators.
Haroché~\cite[p.~131]{Haroche2006} successfully exploits the analogy to derive the quantum beam splitter transform from interaction theory.
We generalize his approach for the mode continuum derived in the previous chapter.

Let $\hat{a}_1(\omega)$ and $\hat{a}_2(\omega)$ be the annihilation operators of the first and second waveguide modes.
The interaction Hamiltonian
\begin{equation}
	\hat{H}_\text{int}
	=
	-
	\int\frac{\dd{\omega}}{2\pi}
	\left\{
		g(\omega)
		\hat{a}_1(\omega)
		\hat{a}_2^\dagger(\omega)
		+
		g^*(\omega)
		\hat{a}_1^\dagger(\omega)
		\hat{a}_2(\omega)
	\right\}
	,
\end{equation}
wherein $g(\omega)$ is a complex-valued coupling parameter encoding the material and geometry of the coupler, describes the transitions of excitations between the first and the second mode.
As the interaction Hamiltonian is time-independent, all but the first term in the Magnus expansion vanish, and the time evolution operator is
\begin{equation}
	\hat{U}_\text{int}
	=
	\exp\left\{
		i
		\int\dd{t^\prime}
		\int\frac{\dd{\omega}}{2\pi}
		\left\{
			g(\omega)
			\hat{a}_1(\omega)
			\hat{a}_2^\dagger(\omega)
			+
			g^*(\omega)
			\hat{a}_1^\dagger(\omega)
			\hat{a}_2(\omega)
		\right\}
	\right\}
\end{equation}
wherein the time integration is over the duration of the interaction.
Assuming the interaction to be limited to the interaction length $l$, the interaction duration $T$ is approximately equal to the interaction length $l$ divided by the group velocity $v_g(\omega)$.
The group velocity depends on the materials of the coupler, suggesting redefining the coupling parameter to include the different interaction durations, i.e.,
\begin{equation}
	\hat{U}_\text{int}
	=
	\exp\left\{
		i
		\int\frac{\dd{\omega}}{2\pi}
		\theta(\omega)
		\left\{
			\hat{a}_1(\omega)
			\hat{a}_2^\dagger(\omega)
			e^{-i\varphi(\omega)}
			+
			\hat{a}_1^\dagger(\omega)
			\hat{a}_2(\omega)
			e^{+i\varphi(\omega)}
		\right\}
	\right\}
\end{equation}
where the real-valued couplings $\theta(\omega)$ and $\varphi(\omega)$ implicitly depend on the materials and geometry of the waveguide coupler and the interaction length $l$.
We define the generator
\begin{equation}
	\hat{G}
	=
	-i
	\int\frac{\dd{\omega}}{2\pi}
	\theta(\omega)
	\left\{
		\hat{a}_1(\omega)
		\hat{a}_2^\dagger(\omega)
		e^{-i\varphi(\omega)}
		+
		\hat{a}_1^\dagger(\omega)
		\hat{a}_2(\omega)
		e^{+i\varphi(\omega)}
	\right\}
\end{equation}
and calculate the commutator of the generator with the annihilation operators
\begin{align}
	\comm{\hat{G}}{\hat{a}_1(\omega)}
	&=
	i
	\theta(\omega)
	\hat{a}_2(\omega)
	e^{+i\varphi(\omega)}
	\\
	\comm{\hat{G}}{\hat{a}_2(\omega)}
	&=
	i
	\theta(\omega)
	\hat{a}_1(\omega)
	e^{-i\varphi(\omega)}
	.
\end{align}
The transformed annihilation operators turn out to be\footnote{Strictly speaking, the annihilation operators in the interaction picture have an additional factor $e^{-i\omega t}$.},
\begin{equation}
	\begin{split}
		\hat{a}_1^\prime(\omega)
		&=
		\hat{U}_\text{int}^\dagger
		\hat{a}_1(\omega)
		\hat{U}_\text{int}
		\\
		&=
		\hat{a}_1
		+
		\comm{\hat{G}}{\hat{a}_1}
		+
		\frac{1}{2!}
		\comm{\hat{G}}{\comm{\hat{G}}{\hat{a}_1}}
		+
		\frac{1}{3!}
		\comm{\hat{G}}{\comm{\hat{G}}{\comm{\hat{G}}{\hat{a}_1}}}
		+
		\dots
		\\
		&=
		\hat{a}_1(\omega)
		+
		i\theta(\omega)
		\hat{a}_2(\omega)
		e^{+i\varphi(\omega)}
		+
		\frac{1}{2!}
		\left(i\theta(\omega)\right)^2
		\hat{a}_1(\omega)
		+
		\frac{1}{3!}
		\left(i\theta(\omega)\right)^3
		\hat{a}_2(\omega)
		e^{+i\varphi(\omega)}
		+
		\dots
		\\
		&=
		\cos\theta(\omega)
		\hat{a}_1(\omega)
		+
		i\sin\theta(\omega)
		\hat{a}_2(\omega)
		e^{+i\varphi(\omega)}
	\end{split}
\end{equation}
and
\begin{equation}
	\begin{split}
		\hat{a}_2^\prime(\omega)
		&=
		\hat{U}_\text{int}^\dagger
		\hat{a}_2(\omega)
		\hat{U}_\text{int}
		\\
		&=
		\hat{a}_2
		+
		\comm{\hat{G}}{\hat{a}_2}
		+
		\frac{1}{2!}
		\comm{\hat{G}}{\comm{\hat{G}}{\hat{a}_2}}
		+
		\frac{1}{3!}
		\comm{\hat{G}}{\comm{\hat{G}}{\comm{\hat{G}}{\hat{a}_2}}}
		+
		\dots
		\\
		&=
		\hat{a}_2(\omega)
		+
		i\theta(\omega)
		\hat{a}_1(\omega)
		e^{-i\varphi(\omega)}
		+
		\frac{1}{2!}
		\left(i\theta(\omega)\right)^2
		\hat{a}_2(\omega)
		+
		\frac{1}{3!}
		\left(i\theta(\omega)\right)^3
		\hat{a}_1(\omega)
		e^{-i\varphi(\omega)}
		+
		\dots
		\\
		&=
		\cos\theta(\omega)
		\hat{a}_2(\omega)
		+
		i\sin\theta(\omega)
		\hat{a}_1(\omega)
		e^{-i\varphi(\omega)}
		,
	\end{split}
\end{equation}
where we used a kind of \gls{bch} formula, in agreement with Ref.~\cite[p.~131]{Haroche2006}.
In matrix notation, the transformation of the annihilation operators reads
\begin{equation}
	\begin{pmatrix}
        \hat{a}_1^\prime(\omega) \\
        \hat{a}_2^\prime(\omega)
    \end{pmatrix}
    =
    \begin{pmatrix}
        \cos\theta(\omega) & i\sin\theta(\omega)e^{+i\varphi} 
        \\
        i\sin\theta(\omega)e^{-i\varphi} & \cos\theta(\omega)
    \end{pmatrix}
    \begin{pmatrix}
        \hat{a}_1(\omega) \\
        \hat{a}_2(\omega)
    \end{pmatrix}
    \label{eq:waveguide_coupler_transformation}
    .
\end{equation}
Comparison of the annihilation operator transformation for the waveguide coupler, \cref{eq:waveguide_coupler_transformation}, and the beam splitter, \cref{eq:beam_splitter_annihilation}, our waveguide result implies lossless coupling.
Lossless coupling is essential for the transformed annihilation operators to satisfy the \gls{ccr}~\cite[p.~38]{Gerry2005}.
Modeling an absorbing coupler requires four quantum modes, two annihilation operators for the field, and two for a bosonic reservoir, see Ref.~\cite[p.~210]{Vogel2006} for details.

\subsection{Unitary matrix and operator transform}

The transform of the annihilation operators of the free-ray beam splitter and the fiber or waveguide coupler, \cref{eq:waveguide_coupler_transformation,eq:beam_splitter_annihilation}, have in common that they are two-dimensional unitary matrices.
The optical coupler transform being linear and unitary is not surprising since the coupler is a linear passive device, which we further assumed to be lossless.
Going a step further, we can take the linear unitary transform as the defining property of an ideal coupler.
Pursuing the idea further, we start with the unitary matrix decomposition~\cite[p.~95]{Leonhardt2010}
\begin{equation}
	\begin{split}
		U
		&=
		e^{i\Lambda/2}
		\begin{pmatrix}
			\cos(\Theta/2)e^{i(+\Phi+\Psi)/2} & \sin(\Theta/2)e^{i(+\Phi-\Psi)/2} \\
			-\sin(\Theta/2)e^{+(-\Phi+\Psi)/2} & \cos(\Theta/2)e^{i(-\Phi-\Psi)/2}
		\end{pmatrix}
		\\
		&=
		e^{i\Lambda/2}
		\begin{pmatrix}
			e^{+i\Phi/2} & 0 \\
			0 & e^{-i\Phi/2}
		\end{pmatrix}
		\begin{pmatrix}
			\cos(\Theta/2) & \sin(\Theta/2) \\
			-\sin(\Theta/2) & \cos(\Theta/2)
		\end{pmatrix}
		\begin{pmatrix}
			e^{+i\Psi/2} & 0 \\
			0 & e^{-i\Psi/2}
		\end{pmatrix}
	\end{split}
	\label{eq:unitary_matrix}
\end{equation}
wherein we suppress frequency-dependence of the real parameters, $\Lambda(\omega),\Theta(\omega),\Psi(\omega),\Phi(\omega)$, for clarity.
The matrix decomposition, \cref{eq:unitary_matrix}, has an equivalent decomposition in terms of operators, the Jordan-Schwinger operators.
The Jordan-Schwinger operators are defined~\cite[p.~97]{Leonhardt2010}
\begin{align}
	\hat{L}_t
	&=
	\frac{1}{2}
	\begin{pmatrix}
		\hat{a}_1^\dagger \\
		\hat{a}_2^\dagger
	\end{pmatrix}
	\mathbb{1}_2
	\begin{pmatrix}
		\hat{a}_1 \\
		\hat{a}_2
	\end{pmatrix}
	=
	\frac{1}{2}
	\left(
		\hat{a}_1^\dagger
		\hat{a}_1
		+
		\hat{a}_2^\dagger
		\hat{a}_2
	\right)
	\\
	\hat{L}_x
	&=
	\frac{1}{2}
	\begin{pmatrix}
		\hat{a}_1^\dagger \\
		\hat{a}_2^\dagger
	\end{pmatrix}
	\sigma_x
	\begin{pmatrix}
		\hat{a}_1 \\
		\hat{a}_2
	\end{pmatrix}
	=
	\frac{1}{2}
	\left(
		\hat{a}_1^\dagger
		\hat{a}_2
		+
		\hat{a}_2^\dagger
		\hat{a}_1
	\right)
	\\
	\hat{L}_y
	&=
	\frac{1}{2}
	\begin{pmatrix}
		\hat{a}_1^\dagger \\
		\hat{a}_2^\dagger
	\end{pmatrix}
	\sigma_y
	\begin{pmatrix}
		\hat{a}_1 \\
		\hat{a}_2
	\end{pmatrix}
	=
	\frac{i}{2}
	\left(
		\hat{a}_2^\dagger
		\hat{a}_1
		-
		\hat{a}_1^\dagger
		\hat{a}_2
	\right)
	\\
	\hat{L}_z
	&=
	\frac{1}{2}
	\begin{pmatrix}
		\hat{a}_1^\dagger \\
		\hat{a}_2^\dagger
	\end{pmatrix}
	\sigma_z
	\begin{pmatrix}
		\hat{a}_1 \\
		\hat{a}_2
	\end{pmatrix}
	=
	\frac{1}{2}
	\left(
		\hat{a}_1^\dagger
		\hat{a}_1
		-
		\hat{a}_2^\dagger
		\hat{a}_2
	\right)
\end{align}
where $\sigma_1,\sigma_2,\sigma_3$ denote the two-dimensional Pauli matrices.
The Jordan-Schwinger operators satisfy the angular-momentum commutation algebra~\cite[p.~97]{Leonhardt2010}
\begin{align}
	\comm{\hat{L}_i}{\hat{L}_j}
	&=
	i\varepsilon_{ijk}\hat{L}^k
	&
	\comm{\hat{L}_t}{\hat{L}_i}
	&=
	0
	.
\end{align}
The unitary operator
\begin{equation}
	\hat{U}
	=
	e^{i\Lambda\hat{L}_t}
	e^{i\Phi\hat{L}_z}
	e^{i\Theta\hat{L}_y}
	e^{i\Psi\hat{L}_z}
	\label{eq:unitary_operator}
\end{equation}
relates the annihilation operators to the matrix transform, \cref{eq:unitary_matrix}, via~\cite[p.~99]{Leonhardt2010}
\begin{equation}
	U
	\begin{pmatrix}
		\hat{a}_1 \\
		\hat{a}_2
	\end{pmatrix}
	=
	\begin{pmatrix}
		\hat{a}_1^\prime \\
		\hat{a}_2^\prime
	\end{pmatrix}
	=
	\begin{pmatrix}
		\hat{U}^\dagger\hat{a}_1\hat{U} \\
		\hat{U}^\dagger\hat{a}_2\hat{U}
	\end{pmatrix}
	=
	\hat{U}^\dagger
	\begin{pmatrix}
		\hat{a}_1 \\
		\hat{a}_2
	\end{pmatrix}
	\hat{U}
	.
\end{equation}
The inverse of the unitary operator, \cref{eq:unitary_matrix}, which is convenient when transforming the creation operators, can be rewritten
\begin{equation}
	\begin{split}
		\hat{U}(\Lambda,\Phi,\Theta,\Psi)^\dagger
		&=
		e^{-i\Psi\hat{L}_z}
		e^{-i\Theta\hat{L}_y}
		e^{-i\Phi\hat{L}_z}
		e^{-i\Lambda\hat{L}_t}
		\\
		&=
		e^{-i\Lambda\hat{L}_t}
		e^{-i\Psi\hat{L}_z}
		e^{-i\Theta\hat{L}_y}
		e^{-i\Phi\hat{L}_z}
		\\
		&=
		\hat{U}(-\Lambda,-\Psi,-\Theta,-\Phi)
		,
	\end{split}
	\label{eq:unitary_operator_inverse}
\end{equation}
where we used that $\hat{L}_t$ commutes with the other Jordan-Schwinger operators.

\subsection{Coherent input states}

The \gls{mzm} produces highly entangled output states for (displaced) number states, see, for instance, Ref.~\cite{Windhager2011}.
In the transmitter setting, we are exclusively interested in coherent input states, contrary to the receiver, where an attacker could have sent non-coherent states.

Let us consider the input state
\begin{align}
	\ket{\vb{\alpha}}
	&=
	\ket{\alpha_1,\alpha_2}
	&
	\vb{\alpha}
	=
	\begin{pmatrix}
		\alpha_1,
		\alpha_2
	\end{pmatrix}
	\in
	\mathbb{C}^{2\times1}
\end{align}
to a \gls{mzm}\footnote{We can always set $alpha=0$ to set the second input to vacuum.}, i.e.,
\begin{equation}
	\hat{U}_\text{MZM}
	\ket{\vb{\alpha}}
	=
	\hat{U}_\text{MZM}
	\hat{D}(\vb{\alpha})
	\hat{U}_\text{MZM}^\dagger
	\hat{U}_\text{MZM}
	\ket{0,0}
	=
	\hat{U}_\text{MZM}
	\hat{D}(\vb{\alpha})
	\hat{U}_\text{MZM}^\dagger
	\ket{0,0}
\end{equation}
where we used the invariance of the vacuum state $\ket{0,0}$ for the second equality.
The displacement operator transforms as
\begin{equation}
	\begin{split}
		\hat{U}
		\hat{D}(\vb{\alpha})
		\hat{U}^\dagger
		&=
		\hat{U}
		\exp\left\{
			\vb{\alpha}^\trans
			\hat{\vb{a}}^\dagger
			-
			\hat{\vb{a}}
			\vb{\alpha}^*
		\right\}
		\hat{U}^\dagger
		\\
		&=
		\exp\left\{
			\vb{\alpha}^\trans
			\hat{U}
			\hat{\vb{a}}^\dagger
			\hat{U}^\dagger
			-
			\hat{U}
			\hat{\vb{a}}
			\hat{U}^\dagger
			\vb{\alpha}^*
		\right\}
		\\
		&=
		\exp\left\{
			\vb{\alpha}^\trans
			\left(
				\hat{U}
				\hat{\vb{a}}
				\hat{U}^\dagger
			\right)^\dagger
			-
			\hat{U}
			\hat{\vb{a}}
			\hat{U}^\dagger
			\vb{\alpha}^*
		\right\}
	\end{split}
\end{equation}
where we used the operator identity
\begin{equation}
	\hat{U}
	e^{\hat{A}}
	\hat{U}^\dagger
	=
	\sum_{n=0}^\infty
	\frac{1}{n!}
	\hat{U}
	\hat{A}^n
	\hat{U}^\dagger
	=
	\sum_{n=0}^\infty
	\frac{1}{n!}
	\hat{U}
	\hat{A}
	\hat{U}^\dagger
	\cdots
	\hat{U}
	\hat{A}
	\hat{U}^\dagger
	=
	\sum_{n=0}^\infty
	\frac{1}{n!}
	\left(
		\hat{U}
		\hat{A}
		\hat{U}^\dagger
	\right)^n
	=
	e^{\hat{U}\hat{A}\hat{U}^\dagger}
\end{equation}
which follows from inserting $\hat{U}\hat{U}^\dagger=\mathbb{1}$ between the $\hat{A}$s.
For the unitary operator of the \gls{mzm}, we note that
\begin{equation}
	\begin{split}
		\hat{U}_\text{MZM}(\Lambda,\Phi,\Psi,\Theta)^\dagger
		&=
		e^{-i\Psi\hat{L}_z}
		e^{-i\Theta\hat{L}_y}
		e^{-i\Phi\hat{L}_z}
		e^{-i\Lambda\hat{L}_t}
		\\
		&=
		e^{-i\Lambda\hat{L}_t}
		e^{-i\Psi\hat{L}_z}
		e^{-i\Theta\hat{L}_y}
		e^{-i\Phi\hat{L}_z}
		\\
		&=
		\hat{U}_\text{MZM}(-\Lambda,-\Psi,-\Phi,-\Theta)
	\end{split}
\end{equation}
where we used in the second line that $\hat{L}_t$ commutes with the other Jordan-Schwinger operators $\hat{L}_y,\hat{L}_z$.
The transformed annihilation operators turn out to be
\begin{equation}
	\begin{split}
		\hat{\vb{a}}^\prime
		&=
		\hat{U}_\text{MZM}(\Lambda,\Phi,\Psi,\Theta)
		\hat{\vb{a}}
		\hat{U}_\text{MZM}(\Lambda,\Phi,\Psi,\Theta)^\dagger
		\\
		&=
		\hat{U}_\text{MZM}(-\Lambda,-\Psi,-\Phi,-\Theta)^\dagger
		\hat{\vb{a}}
		\hat{U}_\text{MZM}(-\Lambda,-\Psi,-\Phi,-\Theta)
		\\
		&=
		U_\text{MZM}(-\Lambda,-\Psi,-\Phi,-\Theta)
		\hat{\vb{a}}
		\\
		&=
		U_\text{MZM}(\Lambda,\Phi,\Psi,\Theta)^\dagger
		\hat{\vb{a}}
		.
	\end{split}
\end{equation}

Using the transformed operators, we find the transformed displacement operator to be~\cite[p.~210]{Vogel2006}
\begin{equation}
	\hat{D}^\prime(\vb{\alpha})
	=
	\exp\left\{
		\vb{\alpha}^\trans
		\left(\hat{\vb{a}}^\prime\right)^\dagger
		-
		\hat{\vb{a}}^\prime
		\vb{\alpha}^*
	\right\}
	=
	\exp\left\{
		\left(\vb{\alpha}^\prime\right)^\trans
		\hat{\vb{a}}^\dagger
		-
		\hat{\vb{a}}
		\left(\vb{\alpha}^\prime\right)^*
	\right\}
	.
\end{equation}
We therefore find the \gls{mzm} to transform the coherent input states according to
\begin{align}
	\hat{U}_\text{MZM}
	\ket{\vb{\alpha}}
	&=
	\ket{\vb{\alpha}^\prime}
	&
	\vb{\alpha}^\prime
	&=
	U_\text{MZM}
	\vb{\alpha}
	.
\end{align}
Using the explicit matrix elements in \cref{eq:mzm_matrix} and setting the second input to zero $\alpha_2=0$ and identifying the first as $\alpha_1=\alpha$, we find
\begin{equation}
	\hat{U}_\text{MZM}
	\ket{\alpha,0}
	=
	\ket{\alpha\cos(\Theta/2)e^{+i(\Psi+\Phi)/2},-\alpha\sin(\Theta/2)e^{-i(\Psi-\Phi)/2}}
	.
\end{equation}
We are only interested in the first output mode.
We can remove the second output via partial trace, i.e.,
\begin{equation}
	\trace_2\left\{
		\ketbra{\alpha,\beta}
	\right\}
	=
	\trace_2\left\{
		\ketbra{\alpha}
		\otimes
		\ketbra{\beta}
	\right\}
	=
	\ketbra{\alpha}
	\otimes
	\trace_2\left\{
		\ketbra{\beta}
	\right\}
\end{equation}
where
\begin{equation}
	\trace_2\left\{
		\ketbra{\beta}
	\right\}
	=
	\sum_{n=0}^\infty
	\braket{n}{\beta}
	\braket{\beta}{n}
	=
	\sum_{n=0}^\infty
	\abs{\braket{n}{\beta}}^2
	=
	1
\end{equation}
and thus
\begin{equation}
	\trace_2\left\{
		\ketbra{\alpha,\beta}
	\right\}
	=
	\ketbra{\alpha}
	=
	\hat{P}_1
	\ketbra{\alpha,\beta}
	\hat{P}_1
\end{equation}
i.e., the partial trace over a tensor product of coherent states is equivalent to the projection of the non-traced out states.
This is a special property of the coherent states.
In general, the partial trace over a tensor product of (entangled) quantum states returns a mixed state.

\subsection{Application as spectral filter}
\section{Modulators}

The (electro-optical) modulators allow us to encode an electrical signal onto an optical carrier, and in that sense, are the interface between the electrical and optical domains.
The domain crossover occurs at the electro-optical phase modulator, which we attempt to describe as a (quantum) nonlinear mixing process mediated by the dielectric, in a first part.
In a second part, we construct, in the optical domain, a complex, in-phase and quadrature, amplitude modulator using \gls{mzi}s driven by electro-optical phase modulators.

\subsection{Phase modulator using linear electro-optic effect}

In an electro-optical phase modulator, an electrical signal changes the refractive index of an optical transmission medium, causing a phase shift.
The linear electro-optic effect, also known as the Pockels effect, characterizes a linear refractive index change proportional to the amplitude of an external electric field~\cite[Ch.~18]{Saleh2007}.
It is present in noncentrosymmetric crystals~\cite[p.~2]{Boyd2020}, e.g., lithium niobate~\cite[p.~237]{Yariv1984}, commonly used in photonic integration.
\begin{figure}[htb]
    \centering
    \includegraphics{figures/tikz/phase-modulator}
    \caption{Traveling-wave electro-optical phase modulator comprising two electrodes of length $L$ (black): The electrodes are driven by a sinusoidal voltage signal creating an \gls{rf} field with frequency $\Omega_0$ between the electrodes. An optical field with frequency $\omega$ reaches the phase modulator from the left, and an optical field with multiple frequency components, $\omega,\omega\pm\Omega_0$, exits the phase modulator to the right.}\label{fig:phase_modulator}
\end{figure}
\Cref{fig:phase_modulator} depicts a traveling-wave electro-optical phase modulator.
The phase modulation signal is applied to the electrodes creating a traveling-wave \gls{rf} field between the two electrodes.
The linear-electro optical effect couples the different \gls{rf} and optical frequency components.
The output field contains sidebands, $\omega\pm\Omega_0$, at the modulation frequency, $\Omega_0$.

The modulation frequency, $\Omega$, is typically many magnitudes smaller than the optical frequency, $\omega$, and the phase-shift due to the refractive index change appears static from the optical domain.
In this case, we expect the complex amplitude of the optical signal leaving the phase modulator to be
\begin{equation}
	\alpha^\prime(t)
	=
	\alpha(t)
	e^{-i\varphi(t)}
	\label{eq:phase_modulation}
	,
\end{equation}
wherein $\alpha(t)$ is the initial amplitude and $\varphi(t)$ is some time-dependent phase.
To a good approximation, the phase modulator is an \gls{lti} system for which the time-dependent phase signal is a convolution
\begin{equation}
	\varphi(t)
	=
	\left(h*x\right)(t)
	=
	\int\dd{t^\prime}
	h(t^\prime)
	x(t-t^\prime)
\end{equation}
with $h(t)$ being the linear time-response of the phase modulator system and $x(t)$ being the voltage signal driving the modulator.
As a rough estimate of when the phase is effectively static concerning the optical field, we can compare the transmission time of the optical signal through the modulator with the period of the optical signal.\footnote{The transmission time $T$ is equal to the group velocity, $v_g(\omega)$, divided by the length of the modulator, $L$.}


While the former classical approach is perfectly sufficient, it does not fit well into our quantum mechanical framework regarding interactions and unitary operators.
In the following, we present the essence of Ref.~\cite{QuesadaMejia2015} and Ref.~\cite{Horoshko2018} to get an insight into quantum optical frequency conversion.
To simplify the discussion, we approximate the electric field (averages) inside the dielectric with free electric field operators at the cost of correctness relating, e.g., to phase-matching.
One particular challenge regarding time-dependent phase modulation is that there exists no simple representation in the frequency domain for arbitrary modulation signals. 
The best we can do is assume a sinusoidal modulation, e.g.,
\begin{equation}
	\varphi(t)
	=
	\beta_0
	\sin(\Omega_0t+\phi)
	\label{eq:sinusoidal_phase_signal}
	,
\end{equation}
for which we can use the Jacobi-Anger expansion~\cite[eq.~23]{Horoshko2018}
\begin{equation}
	e^{-i\beta_0\sin(\Omega_0t+\phi)}
	=
	\sum_{m\in\mathbb{Z}}
	J_m(\beta_0)
	e^{-im(\Omega_0t+\phi)}
	\label{eq:jacobi_anger_expansion}
\end{equation}
wherein $J_m(\beta_0)$ is the $m$th Bessel function of the first kind.
Inserting \cref{eq:sinusoidal_phase_signal} into \cref{eq:phase_modulation} and performing the Jacobi-Anger expansion, we identify the Fourier transform with
\begin{equation}
	\alpha^\prime(\omega)
	=
	\sum_{m\in\mathbb{Z}}
	J_m(\beta_0)
	e^{-im\phi}
	\alpha(\omega+m\Omega_0)
	\label{eq:sinusoidal_phase_signal_fourier}
\end{equation}
after transforming the integration variable.
Sinusoidal phase modulation with frequency $\Omega_0$ creates infinitely many sidebands around the carrier frequency $\omega$.
As we can write an arbitrary time-dependent signal as a sum of sinusoidals of different frequencies, we have to deal with an integral of infinite sums, making arbitrary phase-modulation untractable.

Creating new frequency components requires a nonlinear interaction, not unlike classical signal-processing, where a diode is used as a nonlinear element for frequency conversion.
In the appendix, particular \Cref{sec:frequency_convserion}, we discuss nonlinear processes mediated by the electric susceptibility of a nonabsorptive dielectric and find the
interaction Hamiltonian corresponding to frequency conversion to be
\begin{equation}
	\hat{H}_\text{int}
	=
	\int\frac{\dd{\omega}}{2\pi}
	\int\frac{\dd{\Omega}}{2\pi}
	g(\omega,\Omega)
	\hat{a}^\dagger(\omega)
	\hat{a}(\Omega)
	\hat{a}(\omega-\Omega)
	+
	\text{h.c.}
	\label{eq:frequency_conversion_interaction}
	,
\end{equation}
wherein $g(\omega,\Omega)$ encodes the geometry and properties of the dielectric material.
The coupling parameter $g(\omega,\Omega)$ is only non-zero for $\omega$ being an optical and $\Omega$ being an electrical frequency.
The operators in \cref{eq:frequency_conversion_interaction} describe the simultaneous absorption of an "electric" photon with \gls{rf} $\Omega$, $\hat{a}(\Omega)$, and an "optical" photon with frequency $\omega-\Omega$, $\hat{a}(\omega-\Omega)$, to create an "optical" photon at a higher frequency $\omega$, $\hat{a}^\dagger(\omega)$.
We are not interested in the quantum properties of the \gls{rf} field and assume it to be in a coherent state with complex amplitude $\beta(t)$
\begin{equation}
	\hat{H}_\text{int}
	=
	\int\frac{\dd{\omega}}{2\pi}
	\int\frac{\dd{\Omega}}{2\pi}
	g(\omega,\Omega)
	\beta(\Omega)
	\hat{a}^\dagger(\omega)
	\hat{a}(\omega-\Omega)
	+
	\text{h.c.}
	,
	\label{eq:frequency_conversion_interaction_approx}
\end{equation}
wherein $\beta(\Omega)$ is the Fourier amplitude corresponding to $\beta(t)$.
A sinusoidal modulation signal has a single frequency component,
\begin{align}
	\beta(t)
	&=
	\beta_0
	e^{i\Omega_0t}
	&
	\beta(\Omega)
	&=
	\beta_0
	(2\pi)
	\delta^{(1)}(\Omega-\Omega_0)
	,
\end{align}
Inserting the frequency representation of the sinusoidal modulation into the interaction Hamiltonian, \cref{eq:frequency_conversion_interaction_approx}, we find
\begin{equation}
	\hat{H}_\text{int}
	=
	\beta_0
	\int\frac{\dd{\omega}}{2\pi}
	g(\omega,\Omega_0)
	\hat{a}^\dagger(\omega)
	\hat{a}(\omega+\Omega_0)
	+
	\text{h.c.}
	\label{eq:frequency_conversion_interaction_sinusoidal}
	.
\end{equation}
Let the optical signal be limited to the bandwidth $B$, then
according to the mean-value theorem for integrals, there exists an $\omega_0\in B$ such that we can write
\begin{equation}
	\hat{H}_\text{int}
	=
	\beta_0
	g(\omega_0,\Omega_0)
	\int\frac{\dd{\omega}}{2\pi}
	\hat{a}^\dagger(\omega)
	\hat{a}(\omega+\Omega_0)
	+
	\text{h.c.}
	\label{eq:frequency_conversion_interaction_meanvalue}
	.
\end{equation}
The time-evolution operator corresponding to the interaction Hamiltonian, \cref{eq:frequency_conversion_interaction_meanvalue}, is equal to
\begin{align}
	\hat{U}(\theta,\varphi)
	&=
	e^{-i\hat{G}(\varphi)\theta/2}
	&
	\hat{G}(\varphi)
	&=
	-
	\left(
		\hat{T}_+(\Omega_0)
		e^{-i\varphi}
		+
		\hat{T}_-(\Omega_0)
		e^{+i\varphi}
	\right)
	\label{eq:frequency_conversion_evolution}
\end{align}
wherein we introduced the real-valued coupling constants $\theta,\varphi$ satisfying
\begin{equation}
	\beta_0
	g(\omega_0,\Omega_0)
	T
	=
	\frac{1}{2}
	\theta
	e^{+i\varphi}
\end{equation}
with $T$ being the interaction time approximately equal to the transmit time of the light through the phase modulator.\footnote{Strictly speaking, the transmit time is frequency-dependent requiring a frequency response filter instead of coupling constant.}
The operators in \cref{eq:frequency_conversion_evolution} are defined as
\begin{align}
	\hat{T}_\pm(\Omega)
	&=
	\int\frac{\dd{\omega}}{2\pi}
	\hat{a}^\dagger(\omega\pm\Omega)
	\hat{a}(\omega)
	&
	\hat{T}_+(\Omega)
	&=
	\hat{T}_-(\Omega)^\dagger
\end{align}
and satisfy the commutation relations
\begin{align}
	\comm{\hat{T}_\pm(\Omega)}{\hat{a}(\omega)}
	&=
	-
	\hat{a}(\omega\mp\Omega)
	&
	\comm{\hat{T}_\pm(\Omega)}{\hat{a}^\dagger(\omega)}
	&=
	+
	\hat{a}^\dagger(\omega\pm\Omega)
\end{align}
suggesting to identify $\hat{T}_\pm(\Omega)$ as up- and down-conversion operators.


% Ref.~\cite{Horoshko2018}, which we briefly summarize.
% which reduces to the phase-rotation operator proposed in Ref.~\cite[p.~38]{Leonhardt2010} and Ref.~\cite[p.~103]{Vogel2006} in the single-mode limit and time-independent $\varphi$.
% The energy of the electromagnetic field inside a dielectric medium is symbolically~\cite[p.~124]{Jackson2007}
%where $v_\text{gr}$ is the group velocity in the quasi-monochromatic approximation~\cite[p.~211]{Jackson2007}.

%\footnote{An introduction to the phase-rotation operator is found in Ref.~\cite[p.~38]{Leonhardt2010} and Ref.~\cite[p.~103]{Vogel2006}.}
% which captures the essence of more carefully derived results in Ref.~\cite[eq.~35]{Horoshko2018} and Ref.~\cite[p.~89]{QuesadaMejia2015}.
% We define the time-dependent phase-rotation operator to be\footnote{See Ref.~\cite[p.~103]{Vogel2006} for the time-independent single-mode phase-rotation operator.}
% The annihilation operator transforms as in time-independent case~\cite[p.~38]{Leonhardt2010}

In this particular case, it makes sense to first start from the classical result and move into the quantum direction.
Classically, phase modulation adds a time-dependent phase.
For a coherent state, we therefore expect the existence of a unitary operator, which is a functional of the phase signal, acting on a coherent state to add such time-dependent phase factor, i.e.,
\begin{equation}
	\hat{U}\left[\phi(t)\right]
	\ket*{\alpha(t)}
	=
	\ket*{\alpha(t)e^{-i\phi(t)}}
	.
\end{equation}
However, classically, we already have the problem that there exists no closed Fourier representation of a multiplication with a time-dependent factor.
The best we can do is for single-mode modulation at frequency $\Omega_0$ the Jacobi-Anger expansion for which we can then rewrite the phase-modulated amplitude

and we can read off the transformed Fourier amplitude\footnote{In Ref.~\cite{Horoshko2018}, we find $\omega-m\Omega_0$ because the authors use a different sign convention for the Fourier transform.}
\begin{equation}
	\alpha^\prime(\omega)
	=
	\sum_{m\in\mathbb{Z}}
	J_m(d)
	e^{-im\varphi}
	\alpha(\omega+m\Omega_0)
	.
\end{equation}
For the transformed displacement operator this means
\begin{align}
	\begin{split}
		\hat{D}\left[\alpha^\prime(t)\right]
		&=
		\exp\left\{
			\int\frac{\dd{\omega}}{2\pi}
			\left\{
				\alpha^\prime(\omega)
				\hat{a}^\dagger(\omega)
				-
				\text{h.c.}
			\right\}
		\right\}
		\\
		&=
		\exp\left\{
			\int\frac{\dd{\omega}}{2\pi}
			\left\{
				\alpha(\omega)
				\sum_{m\in\mathbb{Z}}
				J_m(d)
				e^{-im\varphi}
				\hat{a}^\dagger(\omega-m\Omega_0)
				-
				\text{h.c.}
			\right\}
		\right\}
		\\
		&=
		\exp\left\{
			\int\frac{\dd{\omega}}{2\pi}
			\left\{
				\alpha(\omega)
				\hat{U}\left[d\sin(\Omega_0t+\varphi)\right]
				\hat{a}^\dagger(\omega)
				\hat{U}\left[d\sin(\Omega_0t+\varphi)\right]^\dagger
				-
				\text{h.c.}
			\right\}
		\right\}
	\end{split}	
\end{align}
and we reverse-engineered the relation down to
\begin{equation}
	\sum_{m\in\mathbb{Z}}
	J_m(d)
	e^{-im\varphi}
	\hat{a}^\dagger(\omega-m\Omega_0)
	=
	\hat{U}\left[d\sin(\Omega_0t+\varphi)\right]
	\hat{a}^\dagger(\omega)
	\hat{U}\left[d\sin(\Omega_0t+\varphi)\right]^\dagger
	.
\end{equation}

We define
satisfy the commutation relations


We then transform the annihilation operator
\begin{equation}
	\hat{U}(\theta,\varphi)^\dagger
	\hat{a}(\omega)
	\hat{U}(\theta,\varphi)
	=
	\sum_{m\in\mathbb{Z}}
	J_m(\theta)
	e^{im(\theta-\pi/2)}
	\hat{a}(\omega+m\Omega_0)
\end{equation}
in agreement with Ref.~\cite[eq.~40]{Horoshko2018} which can find using the commutators
\begin{align}
	\comm{\hat{G}}{\hat{a}(\omega)}
	&=
	\hat{a}(\omega-\Omega_0)
	e^{-i\varphi}
	+
	\hat{a}(\omega+\Omega_0)
	e^{+i\varphi}
	\\
	\comm{\hat{G}}{\comm{\hat{G}}{\hat{a}(\omega)}}
	&=
	\hat{a}(\omega-2\Omega_0)
	e^{-2i\varphi}
	+
	2\hat{a}(\omega)
	+
	\hat{a}(\omega+2\Omega_0)
	e^{+2i\varphi}
	\\
	\comm{\hat{G}}{\comm{\hat{G}}{\comm{\hat{G}}{\hat{a}(\omega)}}}
	&=
	\hat{a}(\omega-3\Omega_0)
	e^{-3i\varphi}
	+
	2^2
	\hat{a}(\omega-\Omega_0)
	e^{-i\varphi}
	\\
	&+
	\hat{a}(\omega+3\Omega_0)
	e^{+3i\varphi}
	+
	2^2
	\hat{a}(\omega+\Omega_0)
	e^{+i\varphi}
\end{align}

Then to transform the creation operator for the displacement operator, we use the usual trick
\begin{equation}
	\begin{split}
		\hat{U}(\theta,\varphi)^\dagger
		\hat{a}(\omega)
		\hat{U}(\theta,\varphi)
		&=
		\left[
			\hat{U}(\theta,\varphi)^\dagger
			\hat{a}(\omega)
			\hat{U}(\theta,\varphi)	
		\right]^\dagger
		\\
		&=
		\left[
			\hat{U}(-\theta,\varphi)
			\hat{a}(\omega)
			\hat{U}(-\theta,\varphi)^\dagger	
		\right]^\dagger
		\\
		&=
		\left[
			\sum_{m\in\mathbb{Z}}
			J_m(-\theta)
			e^{-im(\theta+\pi/2)}
			\hat{a}(\omega+m\Omega_0)
		\right]^\dagger
		\\
		&=
		\sum_{m\in\mathbb{Z}}
		J_m(-\theta)
		e^{+im(\theta+\pi/2)}
		\hat{a}^\dagger(\omega+m\Omega_0)
		\\
		&=
		\sum_{m\in\mathbb{Z}}
		J_m(\theta)
		e^{+im(\theta-\pi/2)}
		\hat{a}^\dagger(\omega+m\Omega_0)
	\end{split}
\end{equation}
where we used
\begin{equation}
	J_m(-\theta)
	e^{+im(\theta+\pi/2)}
	=
	J_m(\theta)
	(-1)^m
	e^{+im(\theta+\pi/2)}
	=
	J_m(\theta)
	e^{+im(\theta-\pi/2)}
	.
\end{equation}

\subsection{Amplitude modulator using Mach-Zehnder interferometer}

The \gls{mzm} uses two phase modulators to perform amplitude modulation through interference.
The \gls{mzm} enables electrically-driven amplitude modulation if the phase modulators are driven electrically, for instance, using the Pockels effect as discussed previously.
\begin{figure}[htb]
	\centering
	\includestandalone{figures/pstricks/mzi-symmetric}
	\caption{Symmetric \gls{mzm} using free-space optics comprising two balanced \gls{bs}, BS1 and BS2, two mirrors, M1 and M2, and two phase shifters, PS1 and PS2. The input Fourier amplitudes, $\alpha_1(\omega)$ and $\alpha_2(\omega)$, enter BS1 and are split into an upper and lower path. The upper path receives a phase shift of $\phi_1(\omega)+\pi$ from PS1 and M1 before entering BS2 from the top. The lower path receives a phase shift of $\phi_2(\omega)+\pi$ from M2 and PS2 before entering BS2 from the left. BS2 recombines the phase-shifted upper and lower path into the output Fourier amplitudes $\alpha_1^\prime(\omega)$ and $\alpha_2^\prime(\omega)$.}\label{fig:mzi_symmetric}
\end{figure}
\Cref{fig:mzi_symmetric} shows a free-space optics setup of a symmetric \gls{mzi}\footnote{The \gls{mzi} is a static, i.e., time-independent, \gls{mzm}.} with one signal input; the other input being in the vacuum state.
The most crucial components of the \gls{mzi} are a splitter, a coupler, and two independent phase modulators.
The splitter divides the input light into two branches.
Each branch adds a relative phase shift, $\phi_1(\omega)$ and $\phi_2(\omega)$, using an independently  driven phase modulator, PS1 and PS2.
The coupler recombines both branches into two outputs.
Two cubic beam splitters implement the splitter (BS1) and the coupler (BS2) for our free-space setup.
For additional beam alignment, our free-space setup utilizes two mirrors (M1 and M2).

To find the effect of the \gls{mzm} on a coherent input state, we combine the actions of its individual optical components.
Approximating each of the optical components an ideal optical coupler, for which we showed that the Fourier amplitudes transform according to a unitary matrix, lets us write the output Fourier amplitudes of the \gls{mzm} as the matrix product
\begin{equation}
	\vb{\alpha}^\prime(\omega)
	=
	U_\text{MZM}(\omega)
	\vb{\alpha}(\omega)
	=
	U_\text{BS2}(\omega)
	U_\text{PS}(\omega)
	U_\text{BS1}(\omega)
	\vb{\alpha}(\omega)
	\label{eq:mzm_fourier}
	,
\end{equation}
i.e., the matrix transform of the symmetric \gls{mzm}, $U_\text{MZM}$, is equal to the matrix product of the second beam splitter's, the phase shifts', and the first beam splitter's matrix transform, $U_\text{BS2}U_\text{PS}U_\text{BS1}$.
Ignoring the relative phases between the individual components, we use the beam splitter transforms
\begin{align}
	U_\text{BS1}
	&=
	\frac{1}{\sqrt{2}}
	\begin{pmatrix}
		1 & i \\
		i & 1
	\end{pmatrix}
	&
	U_\text{BS2}
	&=
	\frac{1}{\sqrt{2}}
	\begin{pmatrix}
		i & 1 \\
		1 & i
	\end{pmatrix}
	,
\end{align}
corresponding to a perfect cubic beam splitter with a single dielectric layer~\cite[p.~139]{Gerry2005}, where we exchanged the rows for consistency with the input labels.
The matrix encoding the phase shifts from the phase modulation $\phi_1(\omega),\phi_2(\omega)$ and the reflection at the mirrors M1 and M2, $\pi$ is
\begin{equation}
	U_\text{PS}(\omega)
	=
	\begin{pmatrix}
		ie^{i\phi_1(\omega)} & 0 \\
		0 & ie^{i\phi_2(\omega)}
	\end{pmatrix}
\end{equation}
Performing the matrix multiplication and writing the exponentials as trigonometric functions, we find the matrix transform of the symmetric \gls{mzi} to be
\begin{equation}
	U_\text{MZM}(\omega)
	=
	-
	\begin{pmatrix}
		\cos\left(\frac{\phi_2(\omega)-\phi_1(\omega)}{2}\right) & \sin\left(\frac{\phi_2(\omega)-\phi_1(\omega)}{2}\right) \\
		-\sin\left(\frac{\phi_2(\omega)-\phi_1(\omega)}{2}\right) & \cos\left(\frac{\phi_2(\omega)-\phi_1(\omega)}{2}\right)
	\end{pmatrix}
	e^{i\frac{\phi_1(\omega)+\phi_2(\omega)}{2}}
	\label{eq:mzm_matrix1}
	.
\end{equation}
Comparing \cref{eq:mzm_matrix1} with the unitary matrix decomposition \cref{eq:unitary_matrix} suggests that accounting for relative phase would not change the main characteristics significantly or could be compensated by offsetting $\phi_1(\omega)$ or $\phi_2(\omega)$.
It appears useful to define the common-mode and differential-mode phases
\begin{align}
	\phi_+(\omega)
	&=
	\frac{\phi_2(\omega)+\phi_1(\omega)}{2}
	&
	\phi_-(\omega)
	&=
	\frac{\phi_2(\omega)-\phi_1(\omega)}{2}
\end{align}
for which the matrix transform simplifies to
\begin{equation}
	U_\text{MZM}(\omega)
	=
	-
	\begin{pmatrix}
		\cos\phi_-(\omega) & \sin\phi_-(\omega) \\
		-\sin\phi_-(\omega) & \cos\phi_-(\omega)
	\end{pmatrix}
	e^{i\phi_+(\omega)}
	\label{eq:mzm_matrix2}
\end{equation}
and we note that the common-mode phase $\phi_+(\omega)$ adds a global phase shift of $\phi_+(\omega)$ while the differential-mode phase $\phi_-(\omega)$ changes the splitting ratios at the output.

Our result is therefore analog to our result for the spectral filter.
\textcolor{red}{Is this actually correct? Who says that $\phi(\omega)$ and not that $\phi(t)$?}

\begin{figure}[htb]
    \centering
    \includegraphics{figures/tikz/iqm}
    \caption{Integrated \gls{iqm} using three \gls{mzm} arms: A coherent input amplitude, $\alpha(t)$, is split into an upper and lower branch. The upper and lower branches comprise an integrated \gls{mzm} that performs amplitude modulation with the in-phase and quadrature signal, $I(t)$ respectively $Q(t)$. The integrated \gls{mzm} consists of a hexagonal-shaped waveguide with an inside signal electrode and two outer grounds. The outputs of the in-phase and quadrature modulated form a third \gls{mzm} used to set a relative phase shift of $\pi$ between the in-phase and quadrature signals, yielding the coherent output amplitude $\alpha^\prime(t)$.}\label{fig:iqm}
\end{figure}
\section{Modulators}

The (electro-optical) modulators allow us to encode an electrical signal onto an optical carrier, and in that sense, are the interface between the electrical and optical domains.
The domain crossover occurs at the electro-optical phase modulator, which we attempt to describe as a (quantum) nonlinear mixing process mediated by the dielectric, in a first part.
In a second part, we construct, in the optical domain, a complex, in-phase and quadrature, amplitude modulator using \gls{mzi}s driven by electro-optical phase modulators.

\subsection{Phase modulator}

In an electro-optical phase modulator, an electrical signal changes the refractive index of an optical transmission medium, causing a phase shift.
The linear electro-optic effect, also known as the Pockels effect, characterizes a linear refractive index change proportional to the amplitude of an external electric field~\cite[Ch.~18]{Saleh2007}.
It is present in noncentrosymmetric crystals~\cite[p.~2]{Boyd2020}, e.g., lithium niobate~\cite[p.~237]{Yariv1984}, commonly used in photonic integration.
\begin{figure}[htb]
    \centering
    \includegraphics{figures/tikz/phase-modulator}
    \caption{Traveling-wave electro-optical phase modulator comprising two electrodes of length $L$ (black): The electrodes are driven by a sinusoidal voltage signal creating an \gls{rf} field with frequency $\Omega_0$ between the electrodes. An optical field with frequency $\omega$ reaches the phase modulator from the left, and an optical field with multiple frequency components, $\omega,\omega\pm\Omega_0$, exits the phase modulator to the right.}\label{fig:phase_modulator}
\end{figure}
\Cref{fig:phase_modulator} depicts a traveling-wave electro-optical phase modulator.
The phase modulation signal is applied to the electrodes creating a traveling-wave \gls{rf} field between the two electrodes.
The linear-electro optical effect couples the different \gls{rf} and optical frequency components.
The output field contains sidebands, $\omega\pm\Omega_0$, at the modulation frequency, $\Omega_0$.

The modulation frequency, $\Omega$, is typically many magnitudes smaller than the optical frequency, $\omega$, and the phase-shift due to the refractive index change appears static from the optical domain.
In this case, we expect the complex amplitude of the optical signal leaving the phase modulator to be
\begin{equation}
	\alpha^\prime(t)
	=
	\alpha(t)
	e^{-i\varphi(t)}
	\label{eq:phase_modulation}
	,
\end{equation}
wherein $\alpha(t)$ is the initial amplitude and $\varphi(t)$ is some time-dependent phase.
To a good approximation, the phase modulator is an \gls{lti} system for which the time-dependent phase signal is a convolution
\begin{equation}
	\varphi(t)
	=
	\left(h\conv x\right)(t)
	=
	\int\dd{t^\prime}
	h(t^\prime)
	x(t-t^\prime)
\end{equation}
with $h(t)$ being the linear time-response of the phase modulator system and $x(t)$ being the voltage signal driving the modulator.
As a rough estimate of when the phase is effectively static concerning the optical field, we can compare the transmission time of the optical signal through the modulator with the period of the optical signal.\footnote{The transmission time $T$ is equal to the group velocity, $v_g(\omega)$, divided by the length of the modulator, $L$.}


While the former classical approach is perfectly sufficient, it does not fit well into our quantum mechanical framework regarding interactions and unitary operators.
In the following, we present the essence of Ref.~\cite{QuesadaMejia2015} and Ref.~\cite{Horoshko2018} to get an insight into quantum optical frequency conversion.
To simplify the discussion, we approximate the electric field (averages) inside the dielectric with free electric field operators at the cost of correctness relating, e.g., to phase-matching.
One particular challenge regarding time-dependent phase modulation is that there exists no simple representation in the frequency domain for arbitrary modulation signals. 
The best we can do is assume a sinusoidal modulation, e.g.,
\begin{equation}
	\varphi(t)
	=
	\beta_0
	\sin(\Omega_0t+\phi)
	\label{eq:sinusoidal_phase_signal}
	,
\end{equation}
for which we can use the Jacobi-Anger expansion~\cite[eq.~23]{Horoshko2018}
\begin{equation}
	e^{-i\beta_0\sin(\Omega_0t+\phi)}
	=
	\sum_{m\in\mathbb{Z}}
	J_m(\beta_0)
	e^{-im(\Omega_0t+\phi)}
	\label{eq:jacobi_anger_expansion}
\end{equation}
wherein $J_m(\beta_0)$ is the $m$th Bessel function of the first kind.
Inserting \cref{eq:sinusoidal_phase_signal} into \cref{eq:phase_modulation} and performing the Jacobi-Anger expansion, we identify the Fourier transform with
\begin{equation}
	\begin{split}
		\alpha^\prime(\omega)
		&=
		\sum_{m\in\mathbb{Z}}
		J_m(\beta_0)
		e^{-im\phi}
		\int\dd{t}
		\alpha(t)
		e^{-i(\omega+m\Omega_0)t}
		\\
		&=
		\sum_{m\in\mathbb{Z}}
		J_m(\beta_0)
		e^{-im\phi}
		\alpha(\omega+m\Omega_0)
		.
	\end{split}
	\label{eq:sinusoidal_phase_signal_fourier}
\end{equation}
Sinusoidal phase modulation with frequency $\Omega_0$ creates infinitely many sidebands around the carrier frequency $\omega$.
As we can write an arbitrary time-dependent signal as a sum of sinusoidals of different frequencies, we have to deal with an integral of infinite sums, making arbitrary phase-modulation untractable.

Creating new frequency components requires a nonlinear interaction, not unlike classical signal-processing, where a diode is used as a nonlinear element for frequency conversion.
In the appendix, particular \Cref{sec:frequency_convserion}, we discuss nonlinear processes mediated by the electric susceptibility of a nonabsorptive dielectric and find the
interaction Hamiltonian corresponding to frequency conversion to be
\begin{equation}
	\hat{H}_\text{int}
	=
	\int\frac{\dd{\omega}}{2\pi}
	\int\frac{\dd{\Omega}}{2\pi}
	g(\omega,\Omega)
	\hat{a}^\dagger(\omega)
	\hat{a}(\Omega)
	\hat{a}(\omega-\Omega)
	+
	\text{h.c.}
	\label{eq:frequency_conversion_interaction}
	,
\end{equation}
wherein $g(\omega,\Omega)$ encodes the geometry and properties of the dielectric material.
The coupling parameter $g(\omega,\Omega)$ is only non-zero for $\omega$ being an optical and $\Omega$ being an electrical frequency.
The operators in \cref{eq:frequency_conversion_interaction} describe the simultaneous absorption of an "electric" photon with \gls{rf} $\Omega$, $\hat{a}(\Omega)$, and an "optical" photon with frequency $\omega-\Omega$, $\hat{a}(\omega-\Omega)$, to create an "optical" photon at a higher frequency $\omega$, $\hat{a}^\dagger(\omega)$.
We are not interested in the quantum properties of the \gls{rf} field and assume it to be in a coherent state with complex amplitude $\beta(t)$
\begin{equation}
	\hat{H}_\text{int}
	=
	\int\frac{\dd{\omega}}{2\pi}
	\int\frac{\dd{\Omega}}{2\pi}
	g(\omega,\Omega)
	\beta(\Omega)
	\hat{a}^\dagger(\omega)
	\hat{a}(\omega-\Omega)
	+
	\text{h.c.}
	,
	\label{eq:frequency_conversion_interaction_approx}
\end{equation}
wherein $\beta(\Omega)$ is the Fourier amplitude corresponding to $\beta(t)$.
A sinusoidal modulation signal has a single frequency component,
\begin{align}
	\beta(t)
	&=
	\beta_0
	e^{i\Omega_0t}
	&
	\beta(\Omega)
	&=
	\beta_0
	(2\pi)
	\delta^{(1)}(\Omega-\Omega_0)
	,
\end{align}
Inserting the frequency representation of the sinusoidal modulation into the interaction Hamiltonian, \cref{eq:frequency_conversion_interaction_approx}, we find
\begin{equation}
	\hat{H}_\text{int}
	=
	\beta_0
	\int\frac{\dd{\omega}}{2\pi}
	g(\omega,\Omega_0)
	\hat{a}^\dagger(\omega)
	\hat{a}(\omega+\Omega_0)
	+
	\text{h.c.}
	\label{eq:frequency_conversion_interaction_sinusoidal}
	.
\end{equation}
Let the optical signal be limited to the bandwidth $B$, then
according to the mean-value theorem for integrals, there exists an $\omega_0\in B$ such that we can write
\begin{equation}
	\begin{split}
		\hat{H}_\text{int}
		&=
		\beta_0
		g(\omega_0,\Omega_0)
		\int\frac{\dd{\omega}}{2\pi}
		\hat{a}^\dagger(\omega)
		\hat{a}(\omega+\Omega_0)
		+
		\beta_0^*
		g(\omega_0,\Omega_0)^*
		\int\frac{\dd{\omega}}{2\pi}
		\hat{a}^\dagger(\omega)
		\hat{a}(\omega-\Omega_0)
		\\
		&=
		\beta_0
		g(\omega_0,\Omega_0)
		\int\frac{\dd{\omega}}{2\pi}
		\hat{a}^\dagger(\omega-\Omega_0)
		\hat{a}(\omega)
		+
		\beta_0^*
		g(\omega_0,\Omega_0)^*
		\int\frac{\dd{\omega}}{2\pi}
		\hat{a}^\dagger(\omega+\Omega_0)
		\hat{a}(\omega)
	\end{split}
	\label{eq:frequency_conversion_interaction_meanvalue}
\end{equation}
where we performed an integration variable substitution in the second step.

We define the up- and down-conversion operators for the frequency $\Omega_0$ as
\begin{equation}
	\hat{T}_\pm(\Omega_0)
	=
	\int\frac{\dd{\omega}}{2\pi}
	\hat{a}^\dagger(\omega\pm\Omega_0)
	\hat{a}(\omega)
	\label{eq:frequency_conversion_operator}
\end{equation}
which are related via the hermitian conjugate, $\hat{T}_+(\Omega_0)=\hat{T}_-(\Omega_0)^\dagger$, and satisfy the commutation relations
\begin{align}
	\comm{\hat{T}_\pm(\Omega_0)}{\hat{a}(\omega)}
	&=
	-
	\hat{a}(\omega\mp\Omega_0)
	&
	\comm{\hat{T}_\pm(\Omega_0)}{\hat{a}^\dagger(\omega)}
	&=
	+
	\hat{a}^\dagger(\omega\pm\Omega_0)
	,
\end{align}
which can be used to up- or donw-convert the frequency of a photon by $\Omega_0$.

The time-evolution operator corresponding to the interaction Hamiltonian in \cref{eq:frequency_conversion_interaction_meanvalue} turns out to be
\begin{equation}
	\begin{split}
		\hat{U}(\theta,\Omega_0,\varphi)
		=
		\exp\left\{
			-iT
			\hat{H}_\text{int}
		\right\}
		=
		\exp\left\{
			-i
			\frac{1}{2}
			\theta
			\left[
				\hat{T}_+(\Omega_0)
				e^{-i\varphi}
				+
				\hat{T}_-(\Omega_0)
				e^{+i\varphi}
			\right]
		\right\}
	\end{split}
	\label{eq:frequency_conversion_evolution}
\end{equation}
where we new parameters $\Theta,\varphi$ relate to the old coupling parameters via
\begin{equation}
	\beta_0^*
	g(\omega_0,\Omega_0)^*
	T
	=
	\frac{1}{2}
	\theta
	e^{+i\varphi}
\end{equation}
with $T$ being the interaction time approximately equal to the transmit time of the light through the phase modulator.\footnote{Strictly speaking, the transmit time is frequency-dependent requiring a frequency response filter instead of coupling constant.}
Before, we transform the annihilation operator, we define the generator
\begin{equation}
	\hat{G}(\varphi,\Omega_0)
	=
	\hat{T}_+(\Omega_0)
	e^{-i\varphi}
	+
	\hat{T}_-(\Omega_0)
	e^{+i\varphi}
	\label{eq:frequency_conversion_generator}
\end{equation}
which relates to the evolution operator via
\begin{equation}
	\hat{U}(\theta,\Omega_0,\varphi)
	=
	\exp\left\{
		-i\frac{1}{2}
		\theta
		\hat{G}(\varphi,\Omega_0)
	\right\}
	.
\end{equation}
The iterated commutators of the generator with the annihilation operator are
\begin{align}
	\comm{\hat{G}}{\hat{a}(\omega)}
	&=
	(-1)
	\left[
		\hat{a}(\omega-\Omega_0)
		e^{-i\varphi}
		+
		\hat{a}(\omega+\Omega_0)
		e^{+i\varphi}
	\right]
	\\
	\comm{\hat{G}}{\comm{\hat{G}}{\hat{a}(\omega)}}
	&=
	(-1)^2
	\left[
		\hat{a}(\omega-2\Omega_0)
		e^{-2i\varphi}
		+
		2\hat{a}(\omega)
		+
		\hat{a}(\omega+2\Omega_0)
		e^{+2i\varphi}
	\right]
	\\
	\comm{\hat{G}}{\comm{\hat{G}}{\comm{\hat{G}}{\hat{a}(\omega)}}}
	&=
	(-1)^3
	\bigl[
	\hat{a}(\omega-3\Omega_0)
	e^{-3i\varphi}
	+
	2^2
	\hat{a}(\omega-\Omega_0)
	e^{-i\varphi}
	\\
	&+
	\hat{a}(\omega+3\Omega_0)
	e^{+3i\varphi}
	+
	2^2
	\hat{a}(\omega+\Omega_0)
	e^{+i\varphi}
	\bigr]
\end{align}
and we can invoke the \gls{bch} formula, to transform the annihilation operator
\begin{equation}
	\begin{split}
		\hat{a}^\prime(\omega)
		&=
		\hat{U}(\theta,\Omega_0,\varphi)^\dagger
		\hat{a}(\omega)
		\hat{U}(\theta,\Omega_0,\varphi)
		\\
		&=
		\hat{a}(\omega)
		+
		\frac{i\theta}{2}
		\comm{\hat{G}}{\hat{a}(\omega)}
		+
		\frac{i^2\theta^2}{2^22!}
		\comm{\hat{G}}{\comm{\hat{G}}{\hat{a}(\omega)}}
		+
		\frac{i^3\theta^3}{2^33!}
		\comm{\hat{G}}{\comm{\hat{G}}{\dots}}
		+
		\dots
		\\
		&=
		\sum_{m\in\mathbb{Z}}
		J_m(\theta)
		e^{im(\varphi-\pi/2)}
		\hat{a}(\omega+m\Omega_0)
	\end{split}
	\label{eq:frequency_conversion_annihilation}
\end{equation}
in agreement with Ref.~\cite[eq.~40]{Horoshko2018}.

To transform the displacement operator, we need to evaluate
\begin{equation}
	\begin{split}
		\hat{U}(\theta,\Omega_0,\varphi)
		\hat{a}^\dagger(\omega)
		\hat{U}(\theta,\Omega_0,\varphi)^\dagger
		&=
		\left[
			\hat{U}(\theta,\Omega_0,\varphi)
			\hat{a}(\omega)
			\hat{U}(\theta,\Omega_0,\varphi)^\dagger	
		\right]^\dagger
		\\
		&=
		\left[
			\hat{U}(-\theta,\Omega_0,\varphi)^\dagger
			\hat{a}(\omega)
			\hat{U}(-\theta,\Omega_0,\varphi)
		\right]^\dagger
		\\
		&=
		\left[
			\sum_{m\in\mathbb{Z}}
			J_m(-\theta)
			e^{+im(\varphi-\pi/2)}
			\hat{a}(\omega+m\Omega_0)
		\right]^\dagger
		\\
		&=
		\sum_{m\in\mathbb{Z}}
		J_m(-\theta)
		e^{-im(\varphi-\pi/2)}
		\hat{a}^\dagger(\omega+m\Omega_0)
		\\
		&=
		\sum_{m\in\mathbb{Z}}
		J_m(\theta)
		e^{+im(\varphi+\pi/2)}
		\hat{a}^\dagger(\omega+m\Omega_0)
	\end{split}
\end{equation}
where we used
\begin{equation}
	J_m(-\theta)
	e^{-im(\varphi-\pi/2)}
	=
	(-1)^m
	J_m(\theta)
	e^{+im(\varphi-\pi/2)}
	=
	J_m(\theta)
	e^{+im(\varphi+\pi/2)}
\end{equation}
in the last step.
The transformed displacement operator then reads
\begin{align}
	\begin{split}
		\hat{D}^\prime\left[\alpha(t)\right]
		&=
		\hat{U}(\theta,\Omega_0,\varphi)
		\hat{D}\left[\alpha(t)\right]
		\hat{U}(\theta,\Omega_0,\varphi)^\dagger
		\\
		&=
		\exp\left\{
			\int\frac{\dd{\omega}}{2\pi}
			\left[
				\alpha(\omega)
				\hat{U}(\theta,\Omega_0,\varphi)
				\hat{a}^\dagger(\omega)
				\hat{U}(\theta,\Omega_0,\varphi)^\dagger
				-
				\text{h.c.}
			\right]
		\right\}
		\\
		&=
		\exp\left\{
			\int\frac{\dd{\omega}}{2\pi}
			\left[
				\sum_{m\in\mathbb{Z}}
				\alpha(\omega)
				J_m(\theta)
				e^{+im(\varphi+\pi/2)}
				\hat{a}^\dagger(\omega+m\Omega_0)
				-
				\text{h.c.}
			\right]
		\right\}
		\\
		&=
		\exp\left\{
			\int\frac{\dd{\omega}}{2\pi}
			\left[
				\sum_{m\in\mathbb{Z}}
				J_m(\theta)
				\alpha(\omega-m\Omega_0)
				e^{im(\varphi+\pi/2)}
				\hat{a}^\dagger(\omega)
				-
				\text{h.c.}
			\right]
		\right\}
		\\
		&=
		\exp\left\{
			\int\frac{\dd{\omega}}{2\pi}
			\left[
				\alpha^\prime(\omega)
				\hat{a}^\dagger(\omega)
				-
				\text{h.c.}
			\right]
		\right\}
	\end{split}	
\end{align}
where we identified the transformed Fourier amplitude with
\begin{equation}
	\alpha^\prime(\omega)
	=
	\sum_{m\in\mathbb{Z}}
	J_m(\theta)
	e^{im(\varphi+\pi/2)}
	\alpha(\omega-m\Omega_0)
	\label{eq:frequency_conversion_frequency_amplitude}
	.
\end{equation}
Written as a convolution, the transformed Fourier amplitude reads
\begin{equation}
	\alpha^\prime(\omega)
	=
	\int\frac{\dd{\omega^\prime}}{2\pi}
	h(\omega^\prime)
	\alpha(\omega-\omega^\prime)
	,
\end{equation}
where the convolution kernel is the Dirac train
\begin{equation}
	\sum_{m\in\mathbb{Z}}
	J_m(\theta)
	e^{im(\varphi+\pi/2)}
	(2\pi)
	\delta^{(1)}(\omega^\prime-m\Omega_0)
	.
\end{equation}
As the transformed Fourier amplitude is a convolution in frequency space, we expect a product in time space, i.e.,
\begin{equation}
	\begin{split}
		\alpha^\prime(t)
		=
		\int\frac{\dd{\omega}}{2\pi}
		\alpha^\prime(\omega)
		e^{-i\omega t}
		&=
		\int\frac{\dd{\omega}}{2\pi}
		\alpha(\omega-m\Omega_0)
		e^{-i\omega t}
		\sum_{m\in\mathbb{Z}}
		J_m(\theta)
		e^{im(\varphi+\pi/2)}
		\\
		&=
		\int\frac{\dd{\omega}}{2\pi}
		\alpha(\omega)
		e^{-i\omega t}
		\sum_{m\in\mathbb{Z}}
		J_m(\theta)
		e^{-im(\Omega_0t-\varphi-\pi/2)}
		\\
		&=
		\alpha(t)
		e^{-i\theta\sin(\Omega_0t-\varphi-\pi/2)}
		=
		\alpha(t)
		e^{+i\theta\cos(\Omega_0t-\varphi)}
	\end{split}
\end{equation}
where we again used the Jacobi-Anger expansion, \cref{eq:jacobi_anger_expansion}.
We conclude that a coherent state transform under sinusoidal phase-modulation as
\begin{equation}
	\hat{U}(\theta,\Omega_0,\varphi)
	\ket*{\alpha(t)}
	=
	\ket*{\alpha(t)e^{+i\theta\cos(\Omega_0t-\varphi)}}
	.
\end{equation}
We extend our result to signals of finite duration by noting that we can write such a signal as a sum of harmonics, i.e.,
\begin{equation}
	\varphi(t)
	=
	\sum_{k=0}^N
	\theta_k
	\cos(\Omega_kt-\varphi_k)
	,
\end{equation}
and apply a product of sinusoidal phase modulation operators, \cref{eq:frequency_conversion_evolution},
\begin{equation}
	\left(
		\prod^N_{k=0}
		\hat{U}(\theta_k,\Omega_k,\varphi_k)
	\right)
	\ket*{\alpha(t)}
	=
	\ket*{\alpha(t)e^{+i\varphi(t)}}
	.
\end{equation}
Although we cannot write down a closed solution for an arbitrary time-dependent signal in frequency space, we expect such a solution to exist and to be a convolution.

\subsection{Amplitude modulator}

The \gls{mzm} uses two phase modulators to perform amplitude modulation through interference.
\begin{figure}[htb]
	\centering
	\includegraphics{figures/pstricks/mzi-symmetric}
	\caption{Symmetric \gls{mzi} using free-space optics comprising two balanced \gls{bs}, BS1 and BS2, two mirrors, M1 and M2, and two phase modulators, PM1 and PM2. The input amplitudes, $\alpha_1(t)$ and $\alpha_2(t)$, enter BS1 and are split into an upper and lower path. The upper path receives a phase shift of $\phi_1(t)+\pi$ from PM1 and M1 before entering BS2 from the top. The lower path receives a phase shift of $\phi_2(t)+\pi$ from M2 and PM2 before entering BS2 from the left. BS2 recombines the phase-shifted upper and lower path into the output Fourier amplitudes $\alpha_1^\prime(t)$ and $\alpha_2^\prime(t)$.}\label{fig:mzi_symmetric}
\end{figure}
\Cref{fig:mzi_symmetric} shows a symmetric \gls{mzi}\footnote{We distinguish between \gls{mzi} and \gls{mzm}, whether it is an optical (\gls{mzi}) or integrated (\gls{mzm}) embodiment. However, there is no difference in the theoretical treatment.} using free-space optics with one signal input; the other input being in the vacuum state.
The most crucial components of the \gls{mzi} are a splitter, a coupler, and two independent phase modulators.
The splitter divides the input light into two branches, which are phase modulated with $\phi_1(t)$ and $\phi_2(t)$ by PM1 respectively PM2.
The coupler recombines both branches into two outputs.
Two cubic beam splitters implement the splitter (BS1) and the coupler (BS2) for our free-space setup.
For additional beam alignment, our free-space setup utilizes two mirrors (M1 and M2).

To find the effect of the \gls{mzm} on a coherent input state, we study the cumulative effect of the individual optical components.
One particular challenge is that the transformation of passive components is a convolution in the time domain.
In contrast, the transformation of active components is a convolution in the frequency domain.
A possible way forward is to invoke the narrow-bandwidth approximation and assume that the passive components are free of dispersion over the relevant optical bandwidth.
Under this simplifying assumption, the passive components are described by the ideal transformations
\begin{align}
	U_\text{BS1}
	&=
	\frac{1}{\sqrt{2}}
	\begin{pmatrix}
		1 & i \\
		i & 1
	\end{pmatrix}
	&
	U_\text{BS2}
	&=
	\frac{1}{\sqrt{2}}
	\begin{pmatrix}
		i & 1 \\
		1 & i
	\end{pmatrix}
	,
\end{align}
which are valid in both time and frequency space.\footnote{The particular choice corresponds to a perfect cubic beam splitter with a single dielectric layer~\cite[p.~139]{Gerry2005}, where we exchanged the rows of the second \gls{bs} for consistency with the input labels.}
The transformation of the active phase modulation is
\begin{equation}
	U_\text{PM}(t)
	=
	\begin{pmatrix}
		e^{i\phi_1(\omega)} & 0 \\
		0 & e^{i\phi_2(\omega)}
	\end{pmatrix}
	,
\end{equation}
where we ignored static phase shifts.\footnote{For practical applications, one calibrates the phase modulators with a static bias voltage to compensate for undesired static phase shifts.}
The composition of these transformations yields the transformation of the \gls{mzm}
\begin{equation}
	U_\text{MZM}(t)
	=
	U_\text{BS2}
	U_\text{PM}(t)
	U_\text{BS1}
	=
	\begin{pmatrix}
		\cos\phi_-(t) & +\sin\phi_-(t) \\
		-\sin\phi_-(t) & \cos\phi_-(t)
	\end{pmatrix}
	ie^{i\varphi_+(t)}
	\label{eq:mzm_matrix}
	,
\end{equation}
where we introduced the common-mode and differential-mode phase signals
\begin{align}
	\phi_+(t)
	&=
	\frac{\phi_2(t)+\phi_1(t)}{2}
	&
	\phi_-(t)
	&=
	\frac{\phi_2(t)-\phi_1(t)}{2}
	.
\end{align}
For the input state being a tensor product of a coherent and a vacuum state
\begin{equation}
	\ket{\vb{\alpha}(t)}
	=
	\ket{\alpha(t),0}
	,
\end{equation}
the matrix transformation of the \gls{mzm}, \cref{eq:mzm_matrix}, predicts the output amplitudes to be
\begin{equation}
	\alpha^\prime(t)
	=
	U_\text{MZM}(t)
	\alpha(t)
	=
	\alpha(t)
	\begin{pmatrix}
		\cos\phi_-(t) \\
		-\sin\phi_-(t)
	\end{pmatrix}
	ie^{i\varphi_+(t)}
	\label{eq:mzm_amplitude_time}
\end{equation}
where the common-mode phase signal $\varphi_+(t)$ changes the global phase of the output signal and the differential-mode signal $\varphi_-(t)$ changes the power splitting ratio of the outputs.
In the previous sections, we derived the unitary evolution operator corresponding to the unitary matrix transforms.
We therefore claim the existence of an unitary operator, $\hat{U}_\text{MZM}$, corresponding to the unitary matrix transform of \cref{eq:mzm_matrix}, where the action of such operator on a coherent and vacuum input state is
\begin{equation}
	\hat{U}_\text{MZM}(t)
	\ket*{\alpha(t),0}
	=
	\ket*{\alpha(t)\cos\phi_-(t)ie^{i\varphi_+(t)},\alpha(t)\sin\phi_-(t)ie^{i\varphi_+(t)}}
	.
\end{equation}
Usually, one output is monitored for bias control of the phase modulators, and the other output is used for further processing.
In this case, we can remove the other other output using a projection operator, $\hat{P}$, and we find
\begin{equation}
	\hat{P}
	\hat{U}_\text{MZM}(t)
	\ket*{\alpha(t),0}
	=
	\ket*{\alpha(t)\beta_\text{MZM}(t)}
	\label{eq:mzm_state}
	,
\end{equation}
where we defined the complex-valued amplitude modulation signal $\beta(t)$ with $\abs{\beta(t)}\leq 1$.

\Cref{eq:mzm_state} suggests that a \gls{mzm} with two independent phase modulators allows for complex amplitude modulation, i.e.,  modulation of in-phase and quadrature components of the input signal $\alpha(t)$.
\begin{figure}[htb]
    \centering
    \includegraphics{figures/tikz/iqm}
    \caption{Integrated \gls{iqm} using three \gls{mzm} arms: A coherent input amplitude, $\alpha(t)$, is split into an upper and lower branch. The upper and lower branches comprise an integrated \gls{mzm} that performs amplitude modulation with the in-phase and quadrature signal, $I(t)$ respectively $Q(t)$. The integrated \gls{mzm} consists of a hexagonal-shaped waveguide with an inside signal electrode and two outer grounds. The outputs of the in-phase and quadrature modulated form a third \gls{mzm} used to set a relative phase of $\Lambda$ between the in-phase and quadrature signals, yielding the coherent output amplitude $\alpha^\prime(t)$.}\label{fig:iqm}
\end{figure}
In practice, however, we the \gls{iqm} to be implemented using three \gls{mzm}s as depicted in \Cref{fig:iqm}.
The phases of the \gls{mzm} in an integrated \gls{iqm} are only driven differentially.
The upper branch modulates the in-phase component, $I(t)$, while the lower branch modulates the quadrature component, $Q(t)$.
A third \gls{mzm} adds a static relative phase $\Lambda$ between the in-phase and quadrature branch such that these branches are recombined at $\pi/2$.
For the coherent output state, we find
\begin{equation}
	\ket*{\alpha^\prime(t)}
	=
	\hat{U}_\text{IQM}(t)
	\ket*{\alpha(t)}
	=
	\ket*{\alpha(t)\beta_\text{IQM}(t)}
\end{equation}
wherein the complex amplitude modulation signal is now
\begin{equation}
	\beta_\text{IQM}(t)
	=
	I(t)
	+
	iQ(t)
	\label{eq:iqm_modulation}
	.
\end{equation}
While the \gls{iqm} is equivalent to a \gls{mzm} with two independent phase modulators for time-independent modulation, the \gls{mzm} cannot, in general, perform continuously phase modulation.
The reason being that the complex phase of the \gls{mzm}, \cref{eq:mzm_amplitude_time}, cannot be increased arbitrary for practical electro-optical phase modulators but must be brought back to some working point, which would require instantaneous jumps.
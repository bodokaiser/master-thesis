\section{Modulators}

The (electro-optical) modulators allow us to encode an electrical signal onto an optical carrier, and in that sense, are the interface between the electrical and optical domains.
The domain crossover occurs at the electro-optical phase modulator, which we attempt to describe in a (quantum) nonlinear mixing process mediated by the dielectric.
In the remaining parts, we construct, in the optical domain, real and complex amplitude modulators using constructive and destructive interference driven by electro-optical phase modulators.

\subsection{Phase modulator based on the Pockels effect}

The electro-optical phase modulator relates an electric signal to a phase shift of an optical signal by changing the refractive index of its transmission medium.
There are many ways to change the refractive index, such as applying an electric field, the charge carrier density, the temperature, or the mechanical strain of the optical transmission medium.
Changing the refractive index directly through an electric field is, for many applications, the most convenient.
The two most relevant electro-optical effects are the Pockels and Kerr effects~\cite[Ch.~18]{Saleh2007}.
The Pockels effect labels a linear change, while the Kerr effect characterizes a quadratic refractive index change in terms of an external electric field.
If a material exhibits an electro-optical effect depends strongly on the crystal symmetry~\cite[p.~237]{Yariv1984}.
In the following, we focus on the linear electro-optic effect, the Pockels effect, which is present in noncentrosymmetric crystals~\cite[p.~2]{Boyd2020}, e.g., lithium niobate, and commonly used in integrated telecommunication modulators.

The Pockels cell is a Pockels medium embedded inside a plate capacitor, see \Cref{fig:pockels_cell}, and constitutes a primitive embodiment of a phase modulator.
\begin{figure}[htb]
    \centering
    \includegraphics{figures/tikz/pockels-cell}
    \caption{Pockels cell of length $l$ and diameter $d$: The quantum optical input field, $\hat{t}_1(t)$, enters the cell from the left. Inside the Pockels cell, an electric quantum field, $\hat{E}_2(t)$, created by applying a voltage to the capacitor plates (black rectangles), couples with the input field and produces an quantum optical output field, $\hat{E}_1^\prime(t)$.}\label{fig:pockels_cell}
\end{figure}
Applying a voltage signal $V(t)$ to the capacitor plates creates an electric field inside the cell, perpendicular to the plates, denoted by $\hat{E}_2(t)$.\footnote{The mean expectation value for the electric field operator inside the cell is $\expval{\hat{E}_2(t)}=V(t)/d$.}
An optical input field, $\hat{E}_1(\omega)$, approaches from the left couples to $\hat{E}_1(t)$ over the length of the Pockels cell, $l$.
The optical output field, $\hat{E}_1^\prime(t)$, exiting the cell to the right contains frequency components from both the optical input field, $\hat{E}_1(t)$, and the electric field inside the cell, $\hat{E}_2(t)$.

An ideal Pockels medium is a lossless time-invariant system for which the dielectric permittivity tensor can be approximated\footnote{See, for instance, Ref.~\cite{Murti2014} and Ref.~\cite[p.~1070]{Mandel1995}.}
\begin{equation}
	\varepsilon_{ij}(\omega,\Omega)
	\approx
	\delta_{ij}
	+
	\chi_{ij}^{(1)}(\omega)
	+
	\chi_{ijk}^{(2)}(\omega)
	E^k(\Omega)
\end{equation}
wherein $\chi^{(n)}$ the denotes the $n$th order of the Pockels dielectric's electric susceptibility tensor and $\Omega_0\ll\omega$ is the center frequency of the electric signal.
The refractive index tensor of the Pockels dielectric takes the form
\begin{equation}
	n_{ij}(\omega)
	=
	\sqrt{\varepsilon_{ij}}
	\approx
	n_{ij}^{(0)}
	+
	n_{ijk}^{(1)}(\omega)
	E^k(\Omega_0)
\end{equation}
wherein the refractive index coefficients relate to the dielectric susceptibility tensor via~\cite{Rerat2020}
\begin{align}
	n^{(0)}_{ij}(\omega)
	&=
	\sqrt{1+\chi^{(1)}_{ij}(\omega)}
	&
	n^{(1)}_{ijk}(\omega)
	&=
	\frac{\chi^{(2)}_{ijk}(\omega)
	E^k(\Omega_0)}{2\sqrt{1+\chi^{(1)}_{ij}(\omega)}}
	.
\end{align}
Many elements of the refractive index tensor vanish because of crystal symmetries~\cite[p.~237]{Yariv1984} and we can express the refractive index relevant for the optical field
\begin{equation}
	n_{zz}(\omega,\Omega_0)
	\approx
	n^{(0)}_{zz}(\omega)
	+
	n^{(1)}_{zzx}(\omega)
	E^x(\Omega_0)
\end{equation}
which is linear in the electric \gls{rf} field inside the Pockels cell.
If we now imagine an optical field with frequency $\omega_0$ traveling through a Pockels cell it will accumulate a phase of
\begin{equation}
	\phi(E_x)
	=
	2\pi\frac{n_{zz}(\omega_0,E^x)}{\lambda_0}L
\end{equation} 
and the phase difference compared to a Pockels cell without electric field is proportional to the electric field, i.e.,
\begin{equation}
	\Delta\phi(E_x)
	=
	2\pi\frac{n_{zz}(\omega_0,E^x)-n_{zz}(\omega_0,0)}{\lambda_0}L
	=
	2\pi\frac{n_{zzx}^{(1)}(\omega_0)E^x}{\lambda_0}L
	.
\end{equation}

Up to now, the analysis was entirely classical.
A complete quantum description of the Pockels modulator is given in Ref.~\cite{Horoshko2018}, which we briefly summarize.
The Hamiltonian density of an electromagnetic field in a Pockels dielectric is, see Ref.~\cite[p.~21]{Vogel2006} together with Ref.~\cite[p.~124]{Jackson2007},
\begin{equation}
	H
	=
	\frac{1}{2}
	\int\dd[3]{x}
	\left\{
		\vb{E}(\vb{x})
		\vdot
		\vb{D}(\vb{x})
		+
		\vb{B}(\vb{x})^2
	\right\}
	\label{eq:dielectric_em_hamiltonian}
\end{equation}
wherein $\vb{E}$ is the electric field, $\vb{B}$ is the magnetic field and $\vb{D}$ is the displacement field with components
\begin{equation}
	D_i(\vb{p})
	=
	\varepsilon_{ij}(\vb{p})
	E^j(\vb{p})
	\approx
	E_i(\vb{p})
	+
	\int\frac{\dd[3]{q}}{(2\pi)^3}
	\chi_{ijk}^{(2)}(\vb{q},\vb{p})
	E^j(\vb{p})
	E^k(\vb{q})
\end{equation}
is the electric displacement field where we inserted the nonlinear expansion of the dielectric permittivity tensor and neglected the first-order susceptibility $\chi^{(1)}$.\footnote{The first-order susceptibility can be treated as an additional interaction.}
Inserting the displacement field into the Hamiltonian density, \cref{eq:dielectric_em_hamiltonian}, we recover the free Hamiltonian plus an interaction term
\begin{equation}
	H
	=
	H_0
	+
	H_\text{int}
\end{equation}
where the interaction term is given by
\begin{equation}
	H_\text{int}
	=	
	\frac{1}{2}
	\int\frac{\dd[3]{p}}{(2\pi)^3}
	\int\frac{\dd[3]{q}}{(2\pi)^3}
	\chi_{ijk}^{(2)}(\vb{q},\vb{p})
	E^i(\vb{p})
	E^j(\vb{p})
	E^k(\vb{q})	
\end{equation}
describing any three-point interactions including the linear electro-optical (Pockels) effect but also second harmonic generation, parametric mixing and amplification as well as optical rectification~\cite[p.~14]{Murti2014}.
Adapting the geometry of \Cref{fig:pm} the interaction Hamiltonian becomes
\begin{equation}
	H_\text{int}
	=	
	\frac{1}{2}
	\int\frac{\dd[3]{p}}{(2\pi)^3}
	\int\frac{\dd[3]{q}}{(2\pi)^3}
	\chi_{zzx}^{(2)}(\vb{q},\vb{p})
	E^z(\vb{p}-\vb{q})
	E^z(\vb{p}+\vb{q})
	E^x(\vb{q})	
\end{equation}
describing energy transfer from the electric field to the outgoing optical field.
Comparison of the classic interaction with the quantum nonlinear interactions in Ref.~\cite[p.~33]{QuesadaMejia2015} suggests the quantum interaction Hamiltonian
\begin{equation}
	\hat{H}_\text{int}
	=
	\frac{1}{2}
	\int\frac{\dd[3]{p}}{(2\pi)^3}
	\int\frac{\dd[3]{q}}{(2\pi)^3}
	\Upsilon(\vb{p},\vb{q})
	\hat{a}_x(\vb{q})
	\hat{a}_z^\dagger(\vb{p}-\vb{q})
	\hat{a}_z(\vb{p}+\vb{q})
	+
	\text{h.c.}
\end{equation}
wherein the spatial and response properties of the Pockels cell as well as the phase and momentum matching are encoded into $\Upsilon$.
Assuming only two perpendicular momenta $p_x,p_z$ and taking the electric \gls{rf} as classic, the interaction Hamiltonian further simplifies to
\begin{equation}
	\hat{H}_\text{int}
	=
	\frac{1}{2}
	\int\frac{\dd{p}}{2\pi}
	\int\frac{\dd{q}}{2\pi}
	\Upsilon(p,q)
	\hat{a}_x(q)
	\hat{a}_z^\dagger(p-q)
	\hat{a}_z(p+q)
	+
	\text{h.c.}
	.
\end{equation}
Let $\ket{\beta(t)}$ be the coherent state describing the electric \gls{rf} field, then the interaction Hamiltonian becomes
\begin{equation}
	\hat{H}_\text{int}(t)
	=
	\frac{1}{2}
	\int\frac{\dd{p}}{2\pi}
	\int\frac{\dd{q}}{2\pi}
	\Upsilon(p,q)
	\beta_x(q)^*
	e^{-ipt}
	\hat{a}_z^\dagger(p-q)
	\hat{a}_z(p+q)
	+
	\text{h.c.}
\end{equation}
and when taking the momentum of the electric \gls{rf} field to be small compared to the momentum of the optical field, $p\ll q$, such that
\begin{equation}
	\hat{a}_z^\dagger(p-q)
	\hat{a}_z(p+q)
	\approx
	\hat{a}_z^\dagger(p)
	\hat{a}_z(p)
\end{equation}
we obtain the approximate interaction Hamiltonian
\begin{equation}
	\hat{H}_\text{int}(t)
	\approx
	g(t)
	\int\frac{\dd{q}}{2\pi}
	\hat{a}^\dagger(q)
	\hat{a}(q)
\end{equation}
for which all but the first term of the Magnus expansion vanish\footnote{The operator part is time-independent and therefore the interaction Hamiltonian at different times commutes.} and the exact time-evolution operator turns out to be
\begin{align}
	\hat{U}(t_0,t)
	&=
	\exp\left\{
		-i\varphi(t_0,t)
		\int\frac{\dd{p}}{2\pi}
		\hat{a}^\dagger(p)
		\hat{a}(p)
	\right\}
	&
	\varphi(t_0,t)
	=
	\int_{t_0}^{t}\dd{t^\prime}
	g(t^\prime)
\end{align}
which reduces to the phase-rotation operator proposed in Ref.~\cite[p.~38]{Leonhardt2010} and Ref.~\cite[p.~103]{Vogel2006} in the single-mode limit and time-independent $\varphi$.

We define the action of the phase modulator
\begin{equation}
	\hat{U}_\text{PM}\left[\varphi(t,T)\right]
	=
	\exp\left\{
		-i\varphi(t,T)
		\int\frac{\dd{p}}{2\pi}
		\hat{a}^\dagger(p)
		\hat{a}(p)
	\right\}
\end{equation}
wherein $T$ is the transmit time of the light through the phase modulator.
Assuming a phase modulator of length $L$ and refractive index $n$, the transmit time is approximately
\begin{align}
	T
	&=
	\frac{Ln}{v_\text{gr}}
	&
	v_\text{gr}
	\approx
	\left[
		n
		+
		\omega
		\pdv{n}{\omega}
	\right]_{\omega=\omega_0}^{-1}
\end{align}
where $v_\text{gr}$ is the group velocity in the quasi-monochromatic approximation~\cite[p.~211]{Jackson2007}.
We employ the mean-value theorem for definite integrals
\begin{align}
	\varphi(t,T)
	&=
	\int_{t}^{t+T}\dd{t^\prime}
	g(t^\prime)
	=
	g(\tau)
	T
	&
	\tau\in[t,t+T]
\end{align}
and note that typically the transmit time $T$ is many magnitudes smaller than the modulation time scale $t$, i.e., $T\ll t$, and we can use
\begin{equation}
	\varphi(t,T)
	\approx
	g(t)
	T
	=
	\varphi(t)
	.
\end{equation}

The phase modulator transforms the annihilation operator according to~\cite[p.~38]{Leonhardt2010}
\begin{equation}
	\hat{U}_\text{PM}^\dagger
	\hat{a}(p,t)
	\hat{U}_\text{PM}
	=
	\hat{a}(p,t)
	e^{-i\varphi(t)}
	\label{eq:pm_annihilation}
	.
\end{equation}
The continuous-mode displacement operator transforms according to
\begin{equation}
	\begin{split}
		\hat{U}_\text{PM}\left[\varphi(t)\right]
		\hat{D}\left[\alpha(t)\right]
		\hat{U}_\text{PM}\left[\varphi(t)\right]^\dagger
		&=
		\exp\left\{
			\int_0^\infty\frac{\dd{p}}{2\pi}
			\alpha(p)
			\hat{U}_\text{PM}\left[\varphi(t)\right]
			\hat{a}^\dagger(p,t)
			\hat{U}_\text{PM}\left[\varphi(t)\right]^\dagger
			-
			\text{h.c.}
		\right\}
		\\
		&=
		\exp\left\{
			\int_0^\infty\frac{\dd{p}}{2\pi}
			\alpha(p)
			\left(
				\hat{U}_\text{PM}\left[\varphi(t)\right]
				\hat{a}(p,t)
				\hat{U}_\text{PM}\left[\varphi(t)\right]^\dagger
			\right)^\dagger
			-
			\text{h.c.}
		\right\}
		\\
		&=
		\exp\left\{
			\int_0^\infty\frac{\dd{p}}{2\pi}
			\alpha(p)
			\left(
				\hat{U}_\text{PM}\left[-\varphi(t)\right]^\dagger
				\hat{a}(p,t)
				\hat{U}_\text{PM}\left[-\varphi(t)\right]
			\right)^\dagger
			-
			\text{h.c.}
		\right\}
		\\
		&=
		\exp\left\{
			\int_0^\infty\frac{\dd{p}}{2\pi}
			\alpha(p)
			e^{-i\varphi(t)}
			\hat{a}^\dagger(p,t)
			-
			\text{h.c.}
		\right\}
		\\
		&=
		\hat{D}\left[
			\alpha(t)
			e^{-i\varphi(t)}
		\right]
	\end{split}
\end{equation}
where we used \cref{eq:pm_annihilation} in the fourth line.
The phase-modulated amplitude
\begin{equation}
	\alpha(t)e^{-i\varphi(t)}
\end{equation}
is a multiplication in the time domain.
We expect therefore a convolution in the frequency domain, i.e.,
\begin{equation}
	\int_0^\infty\frac{\dd{p}}{2\pi}
	\alpha(p)
	e^{+ipt}
	e^{-i\varphi(t)}
	\hat{a}^\dagger(p,t)
	=
	\int_0^\infty\frac{\dd{p}}{2\pi}
	\left(u*\alpha\right)(p)
	\hat{a}^\dagger(p,t)
\end{equation}
wherein we used the convolution
\begin{equation}
	\left(u*\alpha\right)(p)
	=
	\int\frac{\dd{p^\prime}}{2\pi}
	u(p^\prime)
	\alpha(p-p^\prime)
	.
\end{equation}
It is difficult to give an explicit form of $u(p)$ for signals $\varphi(t)$ with a continuous frequency spectrum $u(p)$.
However, if we take a simple single-frequency modulation, e.g.,
\begin{equation}
	\varphi(t)
	=
	-
	\varphi_0
	\sin(\omega_0t)
\end{equation}
we can use the Jacobi-Anger expansion to write
\begin{equation}
	e^{i\varphi_0\sin(\omega_0t)}
	=
	\sum_{n\in\mathbb{Z}}
	J_n(\varphi_0)
	e^{in\omega_0 t}
\end{equation}
wherein $J_n$ denotes the $n$th Bessel function of the first kind.
Assuming second order harmonics to be outside the bandwidth of our setup, it is sufficient to only keep the first terms of the Jacobi-Anger expansion
\begin{equation}
	e^{i\varphi_0\sin(\omega_0t)}
	=
	J_{-1}(\varphi_0)
	e^{-i\omega_0t}
	+
	J_0(\varphi_0)
	+
	J_{+1}(\varphi_0)
	e^{+i\omega_0t}
\end{equation}
which corresponds to the convolution kernel
\begin{equation}
	u(p)
	=
	J_{-1}(\varphi_0)
	\delta^{(1)}(p-\omega_0)
	+
	J_0(\varphi_0)
	\delta^{(1)}(p)
	+
	J_{+1}(\varphi_0)
	\delta^{(1)}(p+\omega_0)
	.
\end{equation}

\subsection{Amplitude (Mach-Zehnder)}

The \gls{mzm} uses two phase modulators to perform amplitude modulation through interference.
The \gls{mzm} enables electrically-driven amplitude modulation if the phase modulators are driven electrically, for instance, using the Pockels effect as discussed previously.
\begin{figure}[htb]
	\centering
	\includestandalone{figures/pstricks/mzi-symmetric}
	\caption{Symmetric \gls{mzm} using free-space optics comprising two balanced \gls{bs}, BS1 and BS2, two mirrors, M1 and M2, and two phase shifters, PS1 and PS2. The input Fourier amplitudes, $\alpha_1(\omega)$ and $\alpha_2(\omega)$, enter BS1 and are split into an upper and lower path. The upper path receives a phase shift of $\phi_1(\omega)+\pi$ from PS1 and M1 before entering BS2 from the top. The lower path receives a phase shift of $\phi_2(\omega)+\pi$ from M2 and PS2 before entering BS2 from the left. BS2 recombines the phase-shifted upper and lower path into the output Fourier amplitudes $\alpha_1^\prime(\omega)$ and $\alpha_2^\prime(\omega)$.}\label{fig:mzi_symmetric}
\end{figure}
\Cref{fig:mzi_symmetric} shows a free-space optics setup of a symmetric \gls{mzi}\footnote{The \gls{mzi} is a static, i.e., time-independent, \gls{mzm}.} with one signal input; the other input being in the vacuum state.
The most crucial components of the \gls{mzi} are a splitter, a coupler, and two independent phase modulators.
The splitter divides the input light into two branches.
Each branch adds a relative phase shift, $\phi_1(\omega)$ and $\phi_2(\omega)$, using an independently  driven phase modulator, PS1 and PS2.
The coupler recombines both branches into two outputs.
Two cubic beam splitters implement the splitter (BS1) and the coupler (BS2) for our free-space setup.
For additional beam alignment, our free-space setup utilizes two mirrors (M1 and M2).

To find the effect of the \gls{mzm} on a coherent input state, we combine the actions of its individual optical components.
Approximating each of the optical components an ideal optical coupler, for which we showed that the Fourier amplitudes transform according to a unitary matrix, lets us write the output Fourier amplitudes of the \gls{mzm} as the matrix product
\begin{equation}
	\vb{\alpha}^\prime(\omega)
	=
	U_\text{MZM}(\omega)
	\vb{\alpha}(\omega)
	=
	U_\text{BS2}(\omega)
	U_\text{PS}(\omega)
	U_\text{BS1}(\omega)
	\vb{\alpha}(\omega)
	\label{eq:mzm_fourier}
	,
\end{equation}
i.e., the matrix transform of the symmetric \gls{mzm}, $U_\text{MZM}$, is equal to the matrix product of the second beam splitter's, the phase shifts', and the first beam splitter's matrix transform, $U_\text{BS2}U_\text{PS}U_\text{BS1}$.
Ignoring the relative phases between the individual components, we use the beam splitter transforms
\begin{align}
	U_\text{BS1}
	&=
	\frac{1}{\sqrt{2}}
	\begin{pmatrix}
		1 & i \\
		i & 1
	\end{pmatrix}
	&
	U_\text{BS2}
	&=
	\frac{1}{\sqrt{2}}
	\begin{pmatrix}
		i & 1 \\
		1 & i
	\end{pmatrix}
	,
\end{align}
corresponding to a perfect cubic beam splitter with a single dielectric layer~\cite[p.~139]{Gerry2005}, where we exchanged the rows for consistency with the input labels.
The matrix encoding the phase shifts from the phase modulation $\phi_1(\omega),\phi_2(\omega)$ and the reflection at the mirrors M1 and M2, $\pi$ is
\begin{equation}
	U_\text{PS}(\omega)
	=
	\begin{pmatrix}
		ie^{i\phi_1(\omega)} & 0 \\
		0 & ie^{i\phi_2(\omega)}
	\end{pmatrix}
\end{equation}
Performing the matrix multiplication and writing the exponentials as trigonometric functions, we find the matrix transform of the symmetric \gls{mzi} to be
\begin{equation}
	U_\text{MZM}(\omega)
	=
	-
	\begin{pmatrix}
		\cos\left(\frac{\phi_2(\omega)-\phi_1(\omega)}{2}\right) & \sin\left(\frac{\phi_2(\omega)-\phi_1(\omega)}{2}\right) \\
		-\sin\left(\frac{\phi_2(\omega)-\phi_1(\omega)}{2}\right) & \cos\left(\frac{\phi_2(\omega)-\phi_1(\omega)}{2}\right)
	\end{pmatrix}
	e^{i\frac{\phi_1(\omega)+\phi_2(\omega)}{2}}
	\label{eq:mzm_matrix1}
	.
\end{equation}
Comparing \cref{eq:mzm_matrix1} with the unitary matrix decomposition \cref{eq:unitary_matrix} suggests that accounting for relative phase would not change the main characteristics significantly or could be compensated by offsetting $\phi_1(\omega)$ or $\phi_2(\omega)$.
It appears useful to define the common-mode and differential-mode phases
\begin{align}
	\phi_+(\omega)
	&=
	\frac{\phi_2(\omega)+\phi_1(\omega)}{2}
	&
	\phi_-(\omega)
	&=
	\frac{\phi_2(\omega)-\phi_1(\omega)}{2}
\end{align}
for which the matrix transform simplifies to
\begin{equation}
	U_\text{MZM}(\omega)
	=
	-
	\begin{pmatrix}
		\cos\phi_-(\omega) & \sin\phi_-(\omega) \\
		-\sin\phi_-(\omega) & \cos\phi_-(\omega)
	\end{pmatrix}
	e^{i\phi_+(\omega)}
	\label{eq:mzm_matrix2}
\end{equation}
and we note that the common-mode phase $\phi_+(\omega)$ adds a global phase shift of $\phi_+(\omega)$ while the differential-mode phase $\phi_-(\omega)$ changes the splitting ratios at the output.

Our result is therefore analog to our result for the spectral filter.
\textcolor{red}{Is this actually correct? Who says that $\phi(\omega)$ and not that $\phi(t)$?}

\subsection{In-phase and quadrature}



\begin{figure}[htb]
    \centering
    \includegraphics{figures/tikz/iqm}
    \caption{Integrated \gls{iqm} using three \gls{mzm} arms: A coherent input amplitude, $\alpha(t)$, is split into an upper and lower branch. The upper and lower branches comprise an integrated \gls{mzm} that performs amplitude modulation with the in-phase and quadrature signal, $I(t)$ respectively $Q(t)$. The integrated \gls{mzm} consists of a hexagonal-shaped waveguide with an inside signal electrode and two outer grounds. The outputs of the in-phase and quadrature modulated form a third \gls{mzm} used to set a relative phase shift of $\pi$ between the in-phase and quadrature signals, yielding the coherent output amplitude $\alpha^\prime(t)$.}\label{fig:iqm}
\end{figure}
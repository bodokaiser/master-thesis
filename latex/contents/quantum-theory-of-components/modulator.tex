\section{Modulators}

The (electro-optical) modulators allow us to encode an electrical signal onto an optical carrier, and in that sense, are the interface between the electrical and optical domains.
The domain crossover occurs at the electro-optical phase modulator, which we attempt to describe as a (quantum) nonlinear mixing process mediated by the dielectric, in a first part.
In a second part, we construct, in the optical domain, a complex, in-phase and quadrature, amplitude modulator using \gls{mzi}s driven by electro-optical phase modulators.

\subsection{Phase modulator using nonlinear frequency conversion}

The electro-optical phase modulator relates an electric signal to a phase shift of an optical signal by changing the refractive index of its transmission medium.
There are many ways to change the refractive index, such as applying an electric field, the charge carrier density, the temperature, or the mechanical strain of the optical transmission medium.
Changing the refractive index directly through an electric field is, for many applications, the most convenient.
The two most relevant electro-optical effects are the Pockels and Kerr effects~\cite[Ch.~18]{Saleh2007}.
The Pockels effect labels a linear change, while the Kerr effect characterizes a quadratic refractive index change in terms of an external electric field.
If a material exhibits an electro-optical effect depends strongly on the crystal symmetry~\cite[p.~237]{Yariv1984}.
In the following, we focus on the linear electro-optic effect, the Pockels effect, which is present in noncentrosymmetric crystals~\cite[p.~2]{Boyd2020}, e.g., lithium niobate, and commonly used in integrated telecommunication modulators.

The Pockels cell is a Pockels medium embedded inside a plate capacitor (\Cref{fig:pockels_cell}) and constitutes a primitive embodiment of a phase modulator.
\begin{figure}[htb]
    \centering
    \includegraphics{figures/tikz/pockels-cell}
    \caption{Pockels cell of length $L$ modulating sidebands onto an optical carrier: The unmodulated optical input field $\hat{E}(\omega)$ approaches the Pockels cell from the left. Inside the Pockels cell, a modulation field $\hat{E}(\Omega)$, polarized along the optical input field, adds two sidebands, $\hat{E}(\omega),\hat{E}(\omega\pm\Omega)$, to the optical carrier.}\label{fig:pockels_cell}
\end{figure}
Applying a voltage signal $V(t)$ to the capacitor plates creates an electric field inside the cell, depicted by the black, low-frequency sine wave in \Cref{fig:pockels_cell}.
An optical input field, $\hat{E}_1(\omega)$, approaches from the left couples to $\hat{E}_1(t)$ over the length of the Pockels cell, $l$.
The optical output field, $\hat{E}_3(t)$, exiting the cell to the right contains frequency components from both the optical input field, $\hat{E}_1(t)$, and the electric field inside the cell, $\hat{E}_2(t)$.

% usual description via refractive index not possible!
% Ref.~\cite{Horoshko2018}, which we briefly summarize.
% which reduces to the phase-rotation operator proposed in Ref.~\cite[p.~38]{Leonhardt2010} and Ref.~\cite[p.~103]{Vogel2006} in the single-mode limit and time-independent $\varphi$.
% The energy of the electromagnetic field inside a dielectric medium is symbolically~\cite[p.~124]{Jackson2007}
%where $v_\text{gr}$ is the group velocity in the quasi-monochromatic approximation~\cite[p.~211]{Jackson2007}.

%\footnote{An introduction to the phase-rotation operator is found in Ref.~\cite[p.~38]{Leonhardt2010} and Ref.~\cite[p.~103]{Vogel2006}.}

In \cref{sec:frequency_convserion} we motivated the interaction Hamiltonian corresponding to frequency conversion
\begin{equation}
	\hat{H}_\text{int}(t)
	=
	\int\frac{\dd{\omega}}{2\pi}
	\int\frac{\dd{\omega^\prime}}{2\pi}
	g(\omega,\omega^\prime)
	\hat{a}(\omega-\omega^\prime)
	\hat{a}(\omega^\prime)
	\hat{a}^\dagger(\omega)
	+
	\text{h.c.}
\end{equation}
capturing the main characteristics of more carefully derived results in Ref.~\cite[eq.~35]{Horoshko2018} and Ref.~\cite[p.~89]{QuesadaMejia2015} when replacing the first annihilation operator with the classical amplitude.

\subsection{Amplitude modulator using Mach-Zehnder interferometer}

The \gls{mzm} uses two phase modulators to perform amplitude modulation through interference.
The \gls{mzm} enables electrically-driven amplitude modulation if the phase modulators are driven electrically, for instance, using the Pockels effect as discussed previously.
\begin{figure}[htb]
	\centering
	\includestandalone{figures/pstricks/mzi-symmetric}
	\caption{Symmetric \gls{mzm} using free-space optics comprising two balanced \gls{bs}, BS1 and BS2, two mirrors, M1 and M2, and two phase shifters, PS1 and PS2. The input Fourier amplitudes, $\alpha_1(\omega)$ and $\alpha_2(\omega)$, enter BS1 and are split into an upper and lower path. The upper path receives a phase shift of $\phi_1(\omega)+\pi$ from PS1 and M1 before entering BS2 from the top. The lower path receives a phase shift of $\phi_2(\omega)+\pi$ from M2 and PS2 before entering BS2 from the left. BS2 recombines the phase-shifted upper and lower path into the output Fourier amplitudes $\alpha_1^\prime(\omega)$ and $\alpha_2^\prime(\omega)$.}\label{fig:mzi_symmetric}
\end{figure}
\Cref{fig:mzi_symmetric} shows a free-space optics setup of a symmetric \gls{mzi}\footnote{The \gls{mzi} is a static, i.e., time-independent, \gls{mzm}.} with one signal input; the other input being in the vacuum state.
The most crucial components of the \gls{mzi} are a splitter, a coupler, and two independent phase modulators.
The splitter divides the input light into two branches.
Each branch adds a relative phase shift, $\phi_1(\omega)$ and $\phi_2(\omega)$, using an independently  driven phase modulator, PS1 and PS2.
The coupler recombines both branches into two outputs.
Two cubic beam splitters implement the splitter (BS1) and the coupler (BS2) for our free-space setup.
For additional beam alignment, our free-space setup utilizes two mirrors (M1 and M2).

To find the effect of the \gls{mzm} on a coherent input state, we combine the actions of its individual optical components.
Approximating each of the optical components an ideal optical coupler, for which we showed that the Fourier amplitudes transform according to a unitary matrix, lets us write the output Fourier amplitudes of the \gls{mzm} as the matrix product
\begin{equation}
	\vb{\alpha}^\prime(\omega)
	=
	U_\text{MZM}(\omega)
	\vb{\alpha}(\omega)
	=
	U_\text{BS2}(\omega)
	U_\text{PS}(\omega)
	U_\text{BS1}(\omega)
	\vb{\alpha}(\omega)
	\label{eq:mzm_fourier}
	,
\end{equation}
i.e., the matrix transform of the symmetric \gls{mzm}, $U_\text{MZM}$, is equal to the matrix product of the second beam splitter's, the phase shifts', and the first beam splitter's matrix transform, $U_\text{BS2}U_\text{PS}U_\text{BS1}$.
Ignoring the relative phases between the individual components, we use the beam splitter transforms
\begin{align}
	U_\text{BS1}
	&=
	\frac{1}{\sqrt{2}}
	\begin{pmatrix}
		1 & i \\
		i & 1
	\end{pmatrix}
	&
	U_\text{BS2}
	&=
	\frac{1}{\sqrt{2}}
	\begin{pmatrix}
		i & 1 \\
		1 & i
	\end{pmatrix}
	,
\end{align}
corresponding to a perfect cubic beam splitter with a single dielectric layer~\cite[p.~139]{Gerry2005}, where we exchanged the rows for consistency with the input labels.
The matrix encoding the phase shifts from the phase modulation $\phi_1(\omega),\phi_2(\omega)$ and the reflection at the mirrors M1 and M2, $\pi$ is
\begin{equation}
	U_\text{PS}(\omega)
	=
	\begin{pmatrix}
		ie^{i\phi_1(\omega)} & 0 \\
		0 & ie^{i\phi_2(\omega)}
	\end{pmatrix}
\end{equation}
Performing the matrix multiplication and writing the exponentials as trigonometric functions, we find the matrix transform of the symmetric \gls{mzi} to be
\begin{equation}
	U_\text{MZM}(\omega)
	=
	-
	\begin{pmatrix}
		\cos\left(\frac{\phi_2(\omega)-\phi_1(\omega)}{2}\right) & \sin\left(\frac{\phi_2(\omega)-\phi_1(\omega)}{2}\right) \\
		-\sin\left(\frac{\phi_2(\omega)-\phi_1(\omega)}{2}\right) & \cos\left(\frac{\phi_2(\omega)-\phi_1(\omega)}{2}\right)
	\end{pmatrix}
	e^{i\frac{\phi_1(\omega)+\phi_2(\omega)}{2}}
	\label{eq:mzm_matrix1}
	.
\end{equation}
Comparing \cref{eq:mzm_matrix1} with the unitary matrix decomposition \cref{eq:unitary_matrix} suggests that accounting for relative phase would not change the main characteristics significantly or could be compensated by offsetting $\phi_1(\omega)$ or $\phi_2(\omega)$.
It appears useful to define the common-mode and differential-mode phases
\begin{align}
	\phi_+(\omega)
	&=
	\frac{\phi_2(\omega)+\phi_1(\omega)}{2}
	&
	\phi_-(\omega)
	&=
	\frac{\phi_2(\omega)-\phi_1(\omega)}{2}
\end{align}
for which the matrix transform simplifies to
\begin{equation}
	U_\text{MZM}(\omega)
	=
	-
	\begin{pmatrix}
		\cos\phi_-(\omega) & \sin\phi_-(\omega) \\
		-\sin\phi_-(\omega) & \cos\phi_-(\omega)
	\end{pmatrix}
	e^{i\phi_+(\omega)}
	\label{eq:mzm_matrix2}
\end{equation}
and we note that the common-mode phase $\phi_+(\omega)$ adds a global phase shift of $\phi_+(\omega)$ while the differential-mode phase $\phi_-(\omega)$ changes the splitting ratios at the output.

Our result is therefore analog to our result for the spectral filter.
\textcolor{red}{Is this actually correct? Who says that $\phi(\omega)$ and not that $\phi(t)$?}

\begin{figure}[htb]
    \centering
    \includegraphics{figures/tikz/iqm}
    \caption{Integrated \gls{iqm} using three \gls{mzm} arms: A coherent input amplitude, $\alpha(t)$, is split into an upper and lower branch. The upper and lower branches comprise an integrated \gls{mzm} that performs amplitude modulation with the in-phase and quadrature signal, $I(t)$ respectively $Q(t)$. The integrated \gls{mzm} consists of a hexagonal-shaped waveguide with an inside signal electrode and two outer grounds. The outputs of the in-phase and quadrature modulated form a third \gls{mzm} used to set a relative phase shift of $\pi$ between the in-phase and quadrature signals, yielding the coherent output amplitude $\alpha^\prime(t)$.}\label{fig:iqm}
\end{figure}
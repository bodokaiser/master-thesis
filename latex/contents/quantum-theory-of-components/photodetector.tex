\section{Photodetectors}

% We follow mostly Mandel & Wolf supplemented by Knight and Vogel.

\subsection{Photoelectric effect in phototube and photodiode}

The photoelectric effect describes the emission of electrons from an illuminated material.
Historically, it provided strong evidence for the existence of a light quantum, the photon, as the kinetic energy of the emitted electrons,
\begin{equation}
	E_k
	=
	E_\gamma
	-
	E_w
	\label{eq:photoelectric_effect}
	,
\end{equation}
does not depend on the intensity but on the frequency, $E_\gamma=\omega$, of the incident light minus some work energy, $E_w$.
The photoelectric effect relates the momentum spectrum of electrons with the frequency spectrum of photons and provides a mechanism for photodetection.
We present two kinds of photodetectors exploiting the photoelectric effect: the phototube and the photodiode.

A phototube (\Cref{fig:phototube}) comprises a metallic cathode and a biased anode parallel to the cathode.
Whenever photons with energy $\omega>E_w$, wherein $E_w$ is the work energy of the cathode, hit eject an electrode, the anode accelerates the emitted electrons away from the cathode.
The cathode is then in excess of positive charge carriers, creating a positive current, $I$.
\begin{figure}[htb]
    \centering
    \includegraphics{figures/tikz/phototube}
    \caption{Schematic of a phototube, a vacuum tube utilizing the photoelectric effect for photodetection. Photons, $\gamma$, hit a metallic cathode and eject electrons, $e_-$, through the photoelectric effect. The anode is biased with a positive voltage, $V>0$, to attract the emitted electrons. The cathode is in excess of positive charge carriers, creating a positive electric current, $I>0$.}\label{fig:phototube}
\end{figure}
A photodiode is typically made of a PN-junction (\Cref{fig:photodiode}), a junction of a positively- (P) with a negatively-doped (N) semiconductor, supplemented by a contact on the P- and a contact on the N-doped semiconductor.
The contact at the P-doped semiconductor is negatively charged, creating a depletion layer between the PN-junction with no free charge carriers.
The P layer absorbs incident photons, $\gamma$, exciting electrons, which accelerate through the depletion layer towards the cathode, creating the photocurrent, $I$.
\begin{figure}[htb]
    \centering
    \includegraphics{figures/tikz/photodiode}
    \caption{Schematic of a photodiode, a PN-junction. The P layer absorbs incident photons, $\gamma$, exciting electrons, $e_-$, and holes, $e_+$. The electrons are accelerated towards the anode (not shown), and the holes are accelerated towards the cathode, creating a photocurrent $I$.}\label{fig:photodiode}
\end{figure}
Contrary to the phototube, the excited electrons are not truly free but rather excited to an energy band, where they move with higher mobility through the semiconductor.
That said, the central idea of exciting photoelectrons through photon absorption to some higher energy, applies to both kinds of photodetectors~\cite[p.~128]{Cohen1992}.
We will oversee the subtile differences and assume a single photoelectron bound to a single atom in a unique ground state, $\ket{g}$, which is excited through photon absorption to some excited state continuum, $\ket{e}$.\footnote{Quantum models for photon absorption specific to semiconductors are found in Ref.~\cite{Hoyer2002} and Ref.~\cite{Rossi2002}.}

The Hamiltonian of an electron is
\begin{equation}
	\hat{H}_e
	=
	\frac{\vu{p}^2}{2m_e}
	+
	V(\vu{x})
	,
\end{equation}
wherein $V(\vu{x})$ denotes the binding potential.
We expect the ground state, $\ket{g}$, to be an energy eigenstate with eigenvalue, $E_g$.
In the ground state, $\ket{g}$, the electron is bound and dominated by the potential term in the Hamiltonian.
In an excited state, $\ket{e}$, the electron is approximately free and dominated by the kinetic term in the Hamiltonian.
\begin{figure}[htb]
    \centering
    \includegraphics{figures/tikz/photoelectron}
    \caption{Energy level diagram of a photoelectron excited from a bound ground state, $\ket{g}$, with energy $E_g$ to a free excited state, $\ket{e}$, with energy $E_e$ by absorbing a photon with energy $E_\gamma$. To excite the electron from the ground state the photon must have at least the energy of the bandgap, $\min_eE_e-E_g$, wherein $\min_eE_e$ is the minimum energy of the excitation energy band.}\label{fig:photoelectron}
\end{figure}
\Cref{fig:photoelectron} shows the corresponding energy level diagram.
The ground state, $\ket{g}$, is unique and has the single energy $E_g$.
For the excited state, $\ket{e}$, an energy continuum exists with energies ranging from $\min_eE_e$ to $\max_eE_e$.
The difference of the lowest excited energy, $\min_eE_e$, and the ground state energy, $E_g$, is equivalent to the bandgap energy inside a semiconductor.
The transition from the ground to an excited state, $\ket{g}\to\ket{e}$, through photon absorption requires the photon energy, $\omega$, to be at least the bandgap energy, $\omega\geq\min_eE_e-E_g$.

\subsection{Single photoelectron excitation probability}

The transition of a bound electron to a free photoelectron is a probabilistic process.
Let $\ket{i}$ and $\ket{f}$ denote the initial and final light states, and let $\ket{g}$ and $\ket{e}$ be the electron ground and excited states.
The probability for the transition, $\ket{g,i}\to\ket{e,f}$, from time $t$ to $t+\Delta t$ is equal to
\begin{equation}
	\abs{
		\bra{e,f}
		\hat{U}_\text{int}(t,t+\Delta t)
		\ket{g,i}
	}^2
	\label{eq:photoelectric_transition_prob}
	,
\end{equation}
wherein $\hat{U}_\text{int}$ is the time-evolution operator of the photo-atom interaction in the dipole approximation~\cite[p.~689]{Mandel1995},
\begin{align}
	\hat{U}_\text{int}(t,t+\Delta t)
	&=
	\mathcal{T}_+
	\exp\left\{
		-i
		\int_t^{t+\Delta t}
		\dd{t^\prime}
		\hat{H}_\text{int}(t^\prime)
	\right\}
	&
	\hat{H}_\text{int}(t)
	&=
	-
	\hat{\vb{p}}(t)
	\vdot
	\hat{\vb{A}}(t)
	\label{eq:photoelectric_time_evolution_operator}
\end{align}
with $\vu{p}$ being the electron's momentum operator, $\vu{A}(t)=\vu{A}(t,\vb{x}_0)$ being the Maxwell field in the Coulomb gauge approximated at the atom \gls{com}, and $\mathcal{T}_+$ denoting (forward) time-ordering.
In the more general density operator formailism \cref{eq:photoelectric_transition_prob} reads~\cite[p.~686]{Mandel1995}
\begin{equation}
	\begin{split}
		\abs{
			\bra{e,f}
			\hat{U}_\text{int}(t,t+\Delta t)
			\ket{g,i}
		}^2
		&=
		\bra{e,f}
		\hat{U}_\text{int}(t,t+\Delta t)
		\ketbra{g,i}
		\hat{U}_\text{int}(t,t+\Delta t)^\dagger
		\ket{e,f}
		\\
		&=
		\trace\biggl\{
			\bra{e,f}
			\hat{U}_\text{int}(t,t+\Delta t)
			\ketbra{g,i}
			\hat{U}_\text{int}(t,t+\Delta t)^\dagger
			\ket{e,f}
		\biggr\}
		\\
		&=
		\trace\biggl\{
			\ketbra{e,f}
			\hat{U}_\text{int}(t,t+\Delta t)
			\ketbra{g,i}
			\hat{U}_\text{int}(t,t+\Delta t)^\dagger
		\biggr\}
		\\
		&=
		\trace\biggl\{
			\hat\varrho_{e,f}
			\hat{U}_\text{int}(t,t+\Delta t)
			\hat\rho(t)
			\hat{U}_\text{int}(t,t+\Delta t)^\dagger
		\biggr\}
		\\
		&=
		\trace\biggl\{
			\hat\varrho_{e,f}
			\hat\rho(t+\Delta t)
		\biggr\}
		,
	\end{split}
	\label{eq:photoelectric_transition_prob_density}
\end{equation}
wherein we used that the trace of a scalar is the scalar in the second line and the cyclic property of the trace in the third line.
Performing the Magnus expansion of the time-evolution operator, \cref{eq:photoelectric_time_evolution_operator}, up to the first term,
\begin{equation}
	\hat{U}_\text{int}(t,t+\Delta t)
	\approx
	\exp\left\{
		-i
		\int_t^{t+\Delta t}\dd{t^\prime}
		\hat{H}_\text{int}(t^\prime)
	\right\}
	,
\end{equation}
we use it to evolve the state in \cref{eq:photoelectric_transition_prob_density},
\begin{equation}
	\begin{split}
		\hat\rho(t+\Delta t)
		&=
		\hat{U}_\text{int}(t,t+\Delta t)
		\hat\rho(t)
		\hat{U}_\text{int}(t,t+\Delta t)^\dagger
		\\
		&=
		\hat\rho(t)
		+
		(-i)
		\int_t^{t+\Delta t}\dd{t_1}
		\comm{\hat{H}_\text{int}(t_1)}{\hat\rho(t_0)}
		+
		\frac{(-i)^2}{2!}
		\int_{t}^{t+\Delta t}\dd{t_1}
		\int_{t}^{t_1}\dd{t_2}
		\comm{\hat{H}_\text{int}(t_1)}{\comm{\hat{H}_\text{int}(t_2)}{\hat\rho(t)}}
		+
		\dots
	\end{split}
\end{equation}
where the second equation follows from the \gls{bch} formula.
Inserting the expansion into the photoemission probability, \cref{eq:photoelectric_transition_prob_density}, the first two term vanish due to orthogonality with $\hat\varrho_{e,f}$~\cite[p.~686]{Mandel1995}, leaving us with
\begin{equation}
	\begin{split}
		\trace\left\{
			\hat\varrho_{e,f}
			\hat\rho(t_0+\Delta t)
		\right\}
		&=
		\int_t^{t+\Delta t}\dd{t_1}
		\int_t^{t_1}
		\dd{t_2}
		\trace\left\{
			\hat\varrho_{e,f}
			\hat{H}_\text{int}(t_1)
			\hat\rho(t)
			\hat{H}_\text{int}(t_2)
		\right\}
		+
		\text{c.c.}
		\\
		&=
		\int_t^{t+\Delta t}\dd{t_1}
		\int_t^{t_1}
		\dd{t_2}
		\bra{e,f}
			\hat{H}_\text{int}(t_1)
			\hat\rho(t)
			\hat{H}_\text{int}(t_2)
		\ket{e,f}
		+
		\text{c.c.}
		.
	\end{split}
\end{equation}
Inserting the interaction Hamiltonian, \cref{eq:photoelectric_time_evolution_operator}, into our previous result, we find~\cite[p.~693]{Mandel1995}
\begin{equation}
	\begin{split}
		\trace\left\{
			\hat\varrho_{e,f}
			\hat\rho(t_0+\Delta t)
		\right\}
		&=
		\bra{e}\hat{p}^i\ket{g}
		\bra{g}\hat{p}^j\ket{e}
		\int_t^{t+\Delta t}\dd{t_1}
		\int_t^{t_1}
		\dd{t_2}
		\\
		&\times
		\bra{f}
			\hat{A}_i(t_1)
			\ketbra{i}
			\hat{A}_j(t_2)
		\ket{f}
		e^{i(E_e-E_g)(t_1-t_2)}
		+
		\text{c.c.}
		,
	\end{split}	
\end{equation}
wherein we used the energy eigenvalues of the electron's ground and excited state, $\ket{g},\ket{e}$.
The final states of the light field are of no interest to use and can be marginalized~\cite[p.~694]{Mandel1995}, i.e.,
\begin{equation}
	\begin{split}
		\sum_f
		\trace\left\{
			\hat\varrho_{e,f}
			\hat\rho(t_0+\Delta t)
		\right\}
		&=
		\bra{e}\hat{p}^i\ket{g}
		\bra{g}\hat{p}^j\ket{e}
		\int_t^{t+\Delta t}\dd{t_1}
		\int_t^{t_1}
		\dd{t_2}
		\\
		&\times
		\bra{i}
			\hat{A}_i(t_1)
			\hat{A}_j(t_2)
		\ket{i}
		e^{i(E_e-E_g)(t_1-t_2)}
		+
		\text{c.c.}
	\end{split}	
\end{equation}
and we conclude the probability for a single photoelectron to be emitted between time $t$ and $t+\Delta t$ to be
\begin{equation}
	p(t,\Delta t)
	=
	\int_t^{t+\Delta t}\dd{t_1}
	\int_t^{t_1}
	\dd{t_2}
	k^{ij}(t_1-t_2)
	\expval{
		\hat{A}_i(t_1)
		\hat{A}_j(t_2)
	}
	+
	\text{c.c.}
	,
\end{equation}
wherein $k_{ij}(t)$ is the effective response function of the detector atom\footnote{The effective response function depends on the electron's density of states and dipole transition moments, see Ref.~\cite[p.~694]{Mandel1995}.}, and the expectation value of the Maxwell field two-point correlation function is with respect to the initial light state.
It is possible to show that only the normal-ordered Maxwell field expectation value contributes to the photoemission probability~\cite[p.~696]{Mandel1995}
\begin{equation}
	p(t,\Delta t)
	=
	\int_t^{t+\Delta t}\dd{t_1}
	\int_t^{t_1}
	\dd{t_2}
	k^{ij}(t_1-t_2)
	\expval{
		\norder{
			\hat{A}_i(t_1)
			\hat{A}_j(t_2)
		}
	}
	+
	\text{c.c.}
	.	
\end{equation}
We select a coordinate system in which the Maxwell field propagates along the $z$ direction, then one can show~\cite{Kimble1984} that
\begin{equation}
	\begin{split}
		p(t,\Delta t)
		&=
		\int_t^{t+\Delta t}\dd{t_1}
		\int_t^{t_1}
		\dd{t_2}
		k(t_1-t_2)
		\sum_{\lambda=1,2}
		\expval{
			\norder{
				\hat{A}_\lambda(t_1)
				\hat{A}_\lambda(t_2)
			}
		}
		\\
		&=
		\int_t^{t+\Delta t}\dd{t_1}
		\int_t^{t_1}
		\dd{t_2}
		k(t_1-t_2)
		\expval{
			\norder{
				\hat{A}(t_1)
				\hat{A}(t_2)
			}
		}
		+
		\text{c.c.}
		,
	\end{split}
\end{equation}
wherein we defined the scalar polarization-averaged Maxwell field and have the effective response function
\begin{equation}
	k(t)
	=
	\int_0^\infty\frac{\dd{E}}{2\pi}
	K(E)
	e^{-i(E-E_g)t}
\end{equation}
with $K(E)$ being a function of the electron wave function along the $xy$ plane.

Expanding the normal-ordered two-point correlation function of the Maxwell field into positive and negative frequency parts
\begin{equation}
	\begin{split}
		\expval{
			\norder{
				\hat{A}_i(t_1)
				\hat{A}_j(t_2)
			}
		}
		&=
		\expval{
			\norder{
				\left[
					\hat{A}_i^{(-)}(t_1)
					+
					\hat{A}_i^{(+)}(t_1)
				\right]
				\left[
					\hat{A}_j^{(-)}(t_2)
					+
					\hat{A}_j^{(+)}(t_2)
				\right]
			}
		}
		\\
		&=
		\expval{
			\hat{A}_i^{(+)}(t_1)
			\hat{A}_j^{(-)}(t_2)
		}
		+
		\expval{
			\hat{A}_i^{(+)}(t_2)
			\hat{A}_j^{(-)}(t_1)
		}
		,
	\end{split}
\end{equation}
where the non-mixed frequency terms vanish because they contain an unequal number of annihilation and creation operators~\cite[p.~134]{Cohen1992}.
Inserting the mixed frequency terms into the photoemission probability and expanding the effective detector response function in the frequency domain, we drop the highly oscillatory terms and find\footnote{See Ref.~\cite[p.~697]{Mandel1995} and Ref.~\cite[p.~136]{Cohen1992} for an exact argument.}
\begin{equation}
	p(t,\Delta t)
	=
	\int_t^{t+\Delta t}\dd{t_1}
	\int_t^{t_1}
	\dd{t_2}
	k^{ij}(t_1-t_2)
	\expval{
		\hat{A}_i^{(+)}(t_1)
		\hat{A}_j^{(-)}(t_2)
	}
	+
	\text{c.c.}
	.
\end{equation}



The differential probability for photoelectron emission of a single detector atom is~\cite{Kimble1984}
\begin{equation}
	p(t,\Delta t)
	\approx
	K(\omega_0+E_g)
	\expval{
		\hat{A}^{(+)}(t)
		\hat{A}^{(-)}(t)
	}
	\Delta t
	\label{eq:differential_photoemission_probability}
\end{equation}
wherein $\omega_0$ is an optical center frequency.

\subsection{Two-point time correlation of Maxwell and electric field}

We previously derived that the differential single photoelectron excitation probability is proportional to the two-point time correlation function of the Maxwell field,
\begin{equation}
	\hat{A}^{(+)}(t_1)
	\hat{A}^{(-)}(t_2)
	=
	\left(
		\int\frac{\dd{p}}{2\pi}
		\frac{1}{\sqrt{2p}}
		\hat{a}^\dagger(p)
		e^{+ipt_1}
	\right)
	\left(
		\int\frac{\dd{q}}{2\pi}
		\frac{1}{\sqrt{2q}}
		\hat{a}(q)
		e^{-iqt_2}
	\right)
	,
	\label{eq:maxwell_field_two_point_correlation}
\end{equation}
where we inserted the one-dimensional reduction of the positive and negative frequency Maxwell field operators, \cref{eq:maxwell_positive_operator,eq:maxwell_negative_operator}.
It is convenient to relate the two-point time correlation of the Maxwell field, \cref{eq:maxwell_field_two_point_correlation}, with the two-point time correlation of the electric field,
\begin{equation}
	\hat{E}^{(+)}(t_1)
	\hat{E}^{(-)}(t_2)
	=
	\left(
		\int\frac{\dd{p}}{2\pi}
		\sqrt{\frac{p}{2}}
		\hat{a}^\dagger(p)
		e^{+ipt_1}
	\right)
	\left(
		\int\frac{\dd{q}}{2\pi}
		\sqrt{\frac{q}{2}}
		\hat{a}(q)
		e^{-iqt_2}
	\right)
	,
	\label{eq:electric_field_two_point_correlation}
\end{equation}
where we used \cref{eq:electric_positive_operator,eq:electric_negative_operator}, because for a coherent state, $\ket{\alpha}$, we can express the expectation value in terms of the signal amplitude, $\alpha(t)$, i.e.,
\begin{equation}
	\begin{split}
		\bra{\alpha(t)}
		\hat{E}^{(+)}(t_1)
		\hat{E}^{(-)}(t_2)
		\ket{\alpha(t)}
		&=
		\left(
			\int\frac{\dd{p}}{2\pi}
			\sqrt{\frac{p}{2}}
			\frac{\alpha(p)^*}{\sqrt{2p}}
			e^{+ipt_1}
		\right)
		\left(
			\int\frac{\dd{q}}{2\pi}
			\sqrt{\frac{q}{2}}
			\frac{\alpha(q)}{\sqrt{2q}}
			e^{-iqt_2}
		\right)
		\\
		&=
		\frac{1}{4}
		\left(
			\int\frac{\dd{p}}{2\pi}
			\alpha(p)
			e^{-ipt_1}
		\right)^*
		\left(
			\int\frac{\dd{q}}{2\pi}
			\alpha(q)
			e^{-iqt_2}
		\right)
		\\
		&=
		\frac{1}{4}
		\alpha(t_1)^*
		\alpha(t_2)
		.
	\end{split}
\end{equation}
Ref.~\cite{Kimble1984} claims that for quasimonochromatic light with center frequency $\omega_0$, we can connect both time-correlation functions by
\begin{equation}
	\hat{E}^{(+)}(t_1)
	\hat{E}^{(-)}(t_2)
	=
	\omega_0^2
	\hat{A}^{(+)}(t_1)
	\hat{A}^{(-)}(t_2)
	.
\end{equation}
Quasimonochromatic light is bandwidth limited, hence, we can invoke the mean-value theorem for definite integrals to write
\begin{equation}
	\omega_0
	\hat{A}^{(+)}(t)
	\approx
	\int\frac{\dd{p}}{2\pi}
	\frac{p}{\sqrt{2p}}
	\hat{a}^\dagger(p)
	e^{+ipt_1}
	=
	\hat{E}^{(+)}(t)
	,
\end{equation}
assuming $\omega_0$ represents the \gls{com} in the frequency distribution.
The argument becomes problematic if we consider a time-dependent signal, e.g., a time-dependent coherent state, $\ket{\alpha(t)}$.

\textcolor{red}{How to resolve this conflict? Absorb frequency dependence into linear filter?}

\subsection{Photoelectrons counting probability}

In Ref.~\cite[p.~725]{Mandel1995} and Ref.~\cite[p.~180]{Vogel2006}, the differential probability for a single photoelectron excitation is generalized to the integral probability of counting $m$ photoelectrons from infinitely many detector atoms from time $t$ to $t+T$
\begin{equation}
	p_m(t,T)
	=
	\frac{1}{m!}
	\expval{
		\mathcal{T}_+
		\norder{
			\hat{I}(t,T)^m
			\exp\left\{
				-\hat{I}(t,T)
			\right\}
		}
	}
	\label{eq:photoelectron_counting_prob}
	,	
\end{equation}
wherein we defined the bandwidth-limited intensity operator\footnote{In the literature the detector efficiency, $\eta$, is assumed constant. From a signal-processing viewpoint, a linear filter function is a better fit.}
\begin{align}
	\hat{I}(t,T)
	=
	\int_t^{t+T}\dd{t^\prime}
	\eta(t^\prime)
	\expval{
		\hat{E}^{(+)}(t-t^\prime)
		\hat{E}^{(-)}(t-t^\prime)
	}
\end{align}
with detector efficiency response function $\eta(t)$.

\textcolor{red}{How to derive this?}

% independent atoms in ensemble: Rocca 1971, Arnedo and Rocca 1974
% effective detector response including spatiality in z?
% mean and variance for photoelectrons
% renormalization of inefficiency when considering spatial detector

\subsection{Direct (intensity) detector}

% photocurrent for coherent state
% transimpedance amplifier
% statistics

\subsection{Balanced (quadrature) detector}

% literature review
% \cite{Kikuchi2016}
% \cite{Shapiro2009}
% \cite{Loudon2000}

% \cite[p.~206]{Vogel2006}
\begin{figure}[htb]
    \centering
    \includestandalone{figures/tikz/balanced-detector}
    \caption{Optical part of the balanced detector comprising a beam splitter and two photodetectors: A first signal mode $\hat{a}_s=\hat{a}_1$ enters the beam splitter from the left. A second \gls{lo} mode $\hat{a}_l=\hat{a}_2$ enters the beam splitter from the top. A first output mode $\hat{a}_1^\prime$ exits the beam splitter to the right where a first photodetector with photocurrent $i_1$ is placed. A second output mode $\hat{a}_2^\prime$ exits the beam splitter to the bottom where a second photodetector with photocurrent $i_2$ is placed.}\label{fig:balanced_detector_optics}
\end{figure}

We write the frequency-dependent beam splitter \cite[p.~207]{Vogel2006}
\begin{equation}
	\begin{pmatrix}
		\hat{a}_1^\prime(\omega) \\
		\hat{a}_2^\prime(\omega)
	\end{pmatrix}
	=
	\begin{pmatrix}
		t(\omega) & r(\omega) \\
		-r^*(\omega) & t^*(\omega)
	\end{pmatrix}
	\begin{pmatrix}
		\hat{a}_s(\omega) \\
		\hat{a}_l(\omega)
	\end{pmatrix}
	=
	\begin{pmatrix}
		t(\omega)\hat{a}_s(\omega) + r(\omega)\hat{a}_l(\omega) \\
		t^*(\omega)\hat{a}_l(\omega) - r^*(\omega)\hat{a}_s(\omega)
	\end{pmatrix}
\end{equation}
and find the transformed number operators of the output modes to be
\begin{align}
	\hat{n}_1^\prime(\omega)
	&=
	\abs{t(\omega)}^2
	\hat{n}_s(\omega)
	+
	\abs{r(\omega)}^2
	\hat{n}_l(\omega)
	+
	\left[
		t(\omega)^*
		r(\omega)
		\hat{a}_s^\dagger(\omega)
		\hat{a}_l(\omega)
		+
		\text{h.c.}
	\right]
	\\
	\hat{n}_2^\prime(\omega)
	&=
	\abs{r(\omega)}^2
	\hat{n}_s(\omega)
	+
	\abs{t(\omega)}^2
	\hat{n}_l(\omega)
	-
	\left[
		t(\omega)^*
		r(\omega)
		\hat{a}_s^\dagger(\omega)
		\hat{a}_l(\omega)
		+
		\text{h.c.}
	\right]
\end{align}
Assuming the frequency-dependent beam splitter to be sufficiently balanced
\begin{equation}
	\abs{t(\omega)}
	\approx
	\abs{r(\omega)}
\end{equation}
within the frequencies of the detector bandwidth $B$, the photon number difference between the detector is
\begin{equation}
	\begin{split}
		\hat{n}^\prime_\Delta(\omega)
		=
		\hat{n}_1^\prime(\omega)
		-
		\hat{n}_2^\prime(\omega)
		&\approx
		2\left[
			t(\omega)^*
			r(\omega)
			\hat{a}_s^\dagger(\omega)
			\hat{a}_l(\omega)
			+
			\text{h.c.}
		\right]
		\\
		&=
		2\abs{t(\omega)r(\omega)}
		\left[
			\hat{a}_s^\dagger(\omega)
			\hat{a}_l(\omega)
			e^{i(\phi_r-\phi_t)}
			+
			\text{h.c.}
		\right]
	\end{split}
\end{equation}
Assuming the quantum state~\cite[p.~213]{Vogel2006}
\begin{equation}
	\hat\rho(t)
	=
	\hat\rho_s(t)
	\otimes
	\hat\rho_l(t)
	=
	\hat\rho_s(t)
	\otimes
	\ketbra{\alpha_l(t)}
\end{equation}
wherein the \gls{lo} coherent state is
\begin{equation}
	\ket{\alpha_l(t)}
	=
	e^{-\overline{n}_l(t)/2}
	\exp\left\{
		\int_0^\infty\dd{\omega}
		\alpha_l(\omega,t)
		\hat{a}_l^\dagger(\omega)
	\right\}
	\ket{0}
	.
\end{equation}
We then find
\begin{equation}
	\begin{split}
		\overline{n}^\prime_\Delta(t)
		&=
		\int_0^\infty\frac{\dd{\omega}}{2\pi}
		\trace\left\{
			\hat\rho(t)
			\hat{n}_\Delta^\prime(\omega)
		\right\}
		\\
		&=
		2
		\int_0^\infty\frac{\dd{\omega}}{2\pi}
		\abs{t(\omega)r(\omega)}
		\trace_s\left\{
			\hat\rho_s(t)
			\expval{
				\hat{a}_s^\dagger(\omega)
				\hat{a}_l(\omega)
				e^{+i(\phi_r-\phi_t)}
				+
				\text{h.c.}
			}{\alpha_l(t)}
		\right\}
		\\
		&=
		2
		\int_0^\infty\frac{\dd{\omega}}{2\pi}
		\abs{t(\omega)r(\omega)}
		\trace_s\left\{
			\hat\rho_s(t)
			\expval{
				\hat{a}_s^\dagger(\omega)
				\alpha_l(\omega,t)
				e^{+i(\phi_r-\phi_t)}
				+
				\text{h.c.}
			}{\alpha_l(t)}
		\right\}
		\\
		&=
		2
		\int_0^\infty\frac{\dd{\omega}}{2\pi}
		\abs{t(\omega)r(\omega)\alpha_l(\omega)}
		\trace_s\left\{
			\hat\rho_s(t)
			\left[
				\hat{a}_s^\dagger(\omega)
				e^{-i\vartheta(t)}
				+
				\text{h.c.}
			\right]
		\right\}
		\\
		&=
		2\sqrt{2}
		\int_0^\infty\frac{\dd{\omega}}{2\pi}
		\abs{t(\omega)r(\omega)\alpha_l(\omega)}
		\trace_s\left\{
			\hat\rho_s(t)
			\hat{X}\left(\omega,\vartheta(t)\right)
		\right\}
		\\
		&=
		2\sqrt{2}
		\int_0^\infty\frac{\dd{\omega}}{2\pi}
		\abs{t(\omega)r(\omega)\alpha_l(\omega)}
		\expval{\hat{X}\left(\omega,\vartheta(\omega,t)\right)}_s
	\end{split}
\end{equation}
with
\begin{equation}
	\vartheta(\omega,t)
	=
	\phi_t-\phi_r-\varphi_l(\omega,t)
\end{equation}
and
\begin{equation}
	\hat{X}\left(\omega,\vartheta(t)\right)
	=
	\frac{1}{\sqrt{2}}
	\left(
		\hat{a}_s(\omega)
		e^{+i\vartheta(t)}
		+
		\hat{a}_s^\dagger(\omega)
		e^{-i\vartheta(t)}
	\right)
	.
\end{equation}
We can further write
\begin{equation}
	\overline{n}^\prime_\Delta(t)
	=
	\left(h*\overline{X}\right)(t)
\end{equation}
where we defined the filter
\begin{equation}
	h(\omega)
	=
	2\sqrt{2}
	\abs{t(\omega)r(\omega)\alpha_l(\omega)}
	.
\end{equation}

We are left to relate the photon number difference operator with the analog signals.
\Cref{fig:balanced_detector_electronics} shows the schematic of the balanced detector and \gls{tia} circuitry.
The photodiodes are both biased with bias voltage $\pm V_b$ to improve frequency response.
At the node between the anode of $\text{PD}_1$ and the cathode of $\text{PD}_2$, the two photocurrents $i_1$ and $i_2$ are directly subtracted.
\begin{figure}[htb]
    \centering
    \includestandalone{figures/circuitikz/balanced-detector}
    \caption{Electronic part of the balanced detector comprising two biassed photodiodes in balanced configuration with a \gls{tia} frontend: The cathode of a first photodiode $\text{PD}_1$ is biased with a positive bias voltage $+V_b$. The anode of a second photodiode $\text{PD}_2$ is biased with a negative bias voltage $-V_b$. The photocurrent difference $i_1-i_2$ runs through the line connecting the anode of $\text{PD}_1$ and the cathode of $\text{PD}_2$. A \gls{tia} with complex feedback impedance $Z_f$ converts and amplifies the photocurrent difference $i_1-i_2$ to an output voltage $V_o$.}\label{fig:balanced_detector_electronics}
\end{figure}
The mean of the photocurrent difference is equal to
\begin{equation}
	\overline{\Delta i}(t)
	=
	\overline{i}_1(t)
	-
	\overline{i}_2(t)
	=
	\int\dd{t^\prime}
	\eta(t^\prime)
	\overline{n}_\Delta^\prime(t-t^\prime)
\end{equation}
wherein $\eta$ is the \gls{qe} or frequency response of the photodiodes.
The mean photocurrent difference is amplified and converted to a voltage signal $V_0$ by the \gls{tia} frontend.
In particular,
\begin{equation}
	V_o(\omega)
	=
	-
	Z_f(\omega)
	\overline{\Delta i}(\omega)
\end{equation}
where $Z_F$ is the complex feedback impedance.
The mean voltage signal is equal to
\begin{equation}
	V_o(t)
	=
	-
	\left(Z_f*\overline{\Delta i}\right)(t)
	=
	-
	\left(Z_f*h*\overline{X}\right)(t)
	.
\end{equation}

The variance of a photodetector with efficiency $\eta$ is given by~\cite[p.~194]{Vogel2006}
\begin{equation}
	\begin{split}
		\overline{(\Delta n^\prime)^2}(\omega)
		&=
		\eta(\omega)
		\expval{\hat{n}^\prime(\omega)}
		+
		\eta(\omega)^2
		\expval{\colon\left(\Delta\hat{n}^\prime(\omega)\right)^2\colon}
		\\
		&=
		\eta(\omega)\left(1-\eta(\omega)\right)
		\expval{\hat{n}^\prime(\omega)}
		+
		\eta(\omega)^2
		\expval{\left(\Delta\hat{n}^\prime(\omega)\right)^2}
	\end{split}
\end{equation}
\begin{equation}
	\begin{split}
		\expval{\colon\hat{n}_\Delta^\prime(\omega)^2\colon}
		&=
		4\abs{t(\omega)r(\omega)}^2
		\expval{
			\colon
			\left[
				\hat{a}_s^\dagger(\omega)
				\hat{a}_l(\omega)
				e^{+i(\phi_r-\phi_l)}
				+
				\hat{a}_s(\omega)
				\hat{a}_l^\dagger(\omega)
				e^{-i(\phi_r-\phi_l)}
			\right]^2
			\colon
		}
		\\
		&=
		4\abs{t(\omega)r(\omega)}^2
		\expval{
			\hat{a}_s^\dagger(\omega)^2
			\hat{a}_l(\omega)^2
			e^{+2i(\phi_r-\phi_l)}
			+
			\hat{a}_l^\dagger(\omega)^2
			\hat{a}_s(\omega)^2
			e^{-2i(\phi_r-\phi_l)}
			+
			2
			\hat{a}_s^\dagger(\omega)
			\hat{a}_l^\dagger(\omega)
			\hat{a}_l(\omega)
			\hat{a}_s(\omega)
		}
		\\
		&=
		4\abs{t(\omega)r(\omega)}^2
		\left[
			2
			\overline{n}_s(t)
			\overline{n}_l(t)
			+
			\dots
		\right]
	\end{split}
\end{equation}
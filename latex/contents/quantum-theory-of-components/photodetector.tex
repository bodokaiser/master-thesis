\section{Photodetectors}

% We follow mostly Mandel & Wolf supplemented by Knight and Vogel.

\subsection{Photoelectric effect in phototube and photodiode}

The photoelectric effect describes the emission of electrons from an illuminated material.
Historically, it provided strong evidence for the existence of a light quantum, the photon, as the kinetic energy of the emitted electrons,
\begin{equation}
	E_k
	=
	E_\gamma
	-
	E_w
	\label{eq:photoelectric_effect}
	,
\end{equation}
does not depend on the intensity but on the frequency, $E_\gamma=\omega$, of the incident light minus some work energy, $E_w$.
The photoelectric effect relates the momentum spectrum of electrons with the frequency spectrum of photons and provides a mechanism for photodetection.
We present two kinds of photodetectors exploiting the photoelectric effect: the phototube and the photodiode.

A phototube (\Cref{fig:phototube}) comprises a metallic cathode and a biased anode parallel to the cathode.
Whenever photons with energy $\omega>E_w$, wherein $E_w$ is the work energy of the cathode, hit eject an electrode, the anode accelerates the emitted electrons away from the cathode.
The cathode is then in excess of positive charge carriers, creating a positive current, $I$.
\begin{figure}[htb]
    \centering
    \includegraphics{figures/tikz/phototube}
    \caption{Schematic of a phototube, a vacuum tube utilizing the photoelectric effect for photodetection. Photons, $\gamma$, hit a metallic cathode and eject electrons, $e_-$, through the photoelectric effect. The anode is biased with a positive voltage, $V>0$, to attract the emitted electrons. The cathode is in excess of positive charge carriers, creating a positive electric current, $I>0$.}\label{fig:phototube}
\end{figure}
A photodiode is typically made of a PN-junction (\Cref{fig:photodiode}), a junction of a positively- (P) with a negatively-doped (N) semiconductor, supplemented by a contact on the P- and a contact on the N-doped semiconductor.
The contact at the P-doped semiconductor is negatively charged, creating a depletion layer between the PN-junction with no free charge carriers.
The P layer absorbs incident photons, $\gamma$, exciting electrons, which accelerate through the depletion layer towards the cathode, creating the photocurrent, $I$.
\begin{figure}[htb]
    \centering
    \includegraphics{figures/tikz/photodiode}
    \caption{Schematic of a photodiode, a PN-junction. The P layer absorbs incident photons, $\gamma$, exciting electrons, $e_-$, and holes, $e_+$. The electrons are accelerated towards the anode (not shown), and the holes are accelerated towards the cathode, creating a photocurrent $I$.}\label{fig:photodiode}
\end{figure}
Contrary to the phototube, the excited electrons are not truly free but rather excited to an energy band, where they move with higher mobility through the semiconductor.
That said, the central idea of exciting photoelectrons through photon absorption to some higher energy, applies to both kinds of photodetectors~\cite[p.~128]{Cohen1992}.
We will oversee the subtile differences and assume a single photoelectron bound to a single atom in a unique ground state, $\ket{g}$, which is excited through photon absorption to some excited state continuum, $\ket{e}$.\footnote{Quantum models for photon absorption specific to semiconductors are found in Ref.~\cite{Hoyer2002} and Ref.~\cite{Rossi2002}.}

The Hamiltonian of an electron is
\begin{equation}
	\hat{H}_e
	=
	\frac{\vu{p}^2}{2m_e}
	+
	V(\vu{x})
	,
\end{equation}
wherein $V(\vu{x})$ denotes the binding potential.
We expect the ground state, $\ket{g}$, to be an energy eigenstate with eigenvalue, $E_g$.
In the ground state, $\ket{g}$, the electron is bound and dominated by the potential term in the Hamiltonian.
In an excited state, $\ket{e}$, the electron is approximately free and dominated by the kinetic term in the Hamiltonian.
\begin{figure}[htb]
    \centering
    \includegraphics{figures/tikz/photoelectron}
    \caption{Energy level diagram of a photoelectron excited from a bound ground state, $\ket{g}$, with energy $E_g$ to a free excited state, $\ket{e}$, with energy $E_e$ by absorbing a photon with energy $E_\gamma$. To excite the electron from the ground state the photon must have at least the energy of the bandgap, $\min_eE_e-E_g$, wherein $\min_eE_e$ is the minimum energy of the excitation energy band.}\label{fig:photoelectron}
\end{figure}
\Cref{fig:photoelectron} shows the corresponding energy level diagram.
The ground state, $\ket{g}$, is unique and has the single energy $E_g$.
For the excited state, $\ket{e}$, an energy continuum exists with energies ranging from $\min_eE_e$ to $\max_eE_e$.
The difference of the lowest excited energy, $\min_eE_e$, and the ground state energy, $E_g$, is equivalent to the bandgap energy inside a semiconductor.
The transition from the ground to an excited state, $\ket{g}\to\ket{e}$, through photon absorption requires the photon energy, $\omega$, to be at least the bandgap energy, $\omega\geq\min_eE_e-E_g$.

\subsection{Photoelectron counting probability}

% independent atoms in ensemble: Rocca 1971, Arnedo and Rocca 1974
% effective detector response including spatiality in z?
% renormalization of inefficiency when considering spatial detector

Ref.~\cite[p.~725]{Mandel1995} generalizes the differential probability for a single photoelectron excitation to the \gls{povm} of counting $m$ photoelectrons from infinitely many independent detector atoms from time $t$ to $t+T$ as\footnote{If we assume no sources present inside the detector, the fields are approximately free at the detector, and we can neglect the time-ordering in \cref{eq:photoelectron_counting_prob}~\cite[p.~183]{Vogel2006}.}
\begin{equation}
	\hat{P}_m(t,T)
	=
	\mathcal{T}_+
	\norder{
		\frac{1}{m!}
		\hat{I}(t,T)^m
		\exp\left\{
			-\hat{I}(t,T)
		\right\}
	}
	\label{eq:photoelectron_povm}
	,	
\end{equation}
wherein the intensity operator is equal to the integrated equal-time correlation function of the electric field, 
\begin{align}
	\hat{I}(t,T)
	=
	\eta
	\int_t^{t+T}\dd{t^\prime}
	\expval{
		\hat{E}^{(+)}(t^\prime)
		\hat{E}^{(-)}(t^\prime)
	}
	\label{eq:intensity_operator_integrated}
\end{align}
with detector efficiency constant $\eta$.
From a signal-processing perspective, it is more natural to define a detector response function, $\eta(t)$, which generalizes the measurement period $T$ and is experimentally accessible.
The convolved intensity operator,
\begin{align}
	\hat{I}(t)
	=
	\eta
	\int\dd{t^\prime}
	\eta(t^\prime)
	\expval{
		\hat{E}^{(+)}(t-t^\prime)
		\hat{E}^{(-)}(t-t^\prime)
	}
	\label{eq:intensity_operator_convolved}
	,
\end{align}
reduces to the integrated intensity operator, \cref{eq:intensity_operator_integrated}, when using a constant step response for the detector.
From here on, we use the \cref{eq:photoelectron_povm} with the convolved intensity operator, \cref{eq:intensity_operator_convolved}.

Using the generating function of the photoelectron \gls{povm}, \cref{eq:photoelectron_povm}, we find the average number of photoelectrons emitted at $t$ to be~\cite[p.~183]{Vogel2006}
\begin{equation}
	\overline{n(t)}
	=
	\sum_{m\in\mathbb{N}_0}
	m
	\expval{\hat{P}_m(t)}
	=
	\expval{\hat{I}(t)}
	\label{eq:photoelectron_povm_mean}
	.
\end{equation}
For the variance, we find~\cite[p.~736]{Mandel1995}
\begin{equation}
	\overline{\left(\Delta n(t)\right)^2}
	=
	\overline{n(t)}
	+
	\expval{\left(\Delta\hat{I}(t)\right)^2}
	\label{eq:photoelectron_povm_var}
	,
\end{equation}
wherein $\expval{\left(\Delta\hat{I}(t)\right)^2}$ denotes the variance of the bandwidth-limited intensity operator, \cref{eq:intensity_operator_convolved}, which can become negative indicating sub-Poissionian statistics for certain quantum states.

\FloatBarrier
\subsection{Direct (intensity) detector}

\Cref{fig:detector_direct} presents the electronic schematic of a direct detector circuit.
A photodiode, PD, is reverse biased with voltage $-V_b<0$ to reduce the response time of the photodiode.
Under optical illumination, PD produces a photocurrent, $i_p$, which in natural units is equal to the mean photoelectron number, \cref{eq:photoelectron_povm_mean}.
\begin{figure}[htb]
    \centering
    \includestandalone{figures/circuitikz/detector-direct}
    \caption{Electronic schematic of a direct (intensity) detector comprising a photodiode and an operational amplifier in \gls{tia} configuration. The anode of the photodiode is reverse biased with voltage $-V_b<0$. The cathode of the photodiode emits the photocurrent $i_p$ and is connected to the inverting input of the operational amplifier. The non-inverting input of the operational amplifier is connected with ground, while the inverting input is coupled with feedback impedance $Z_f$ to the operational amplifier output.}\label{fig:detector_direct}
\end{figure}
For a coherent state, $\ket{\alpha(t)}$, we find the mean photocurrent to be equal to the power of the signal, $\abs{\alpha(t)}^2$, convolved with the detector response function,
\begin{equation}
	i_p(t)
	=
	\int\dd{t^\prime}
	\eta(t^\prime)
	\abs{\alpha(t-t^\prime)}^2
	=
	\left(\eta*\abs{\alpha}^2\right)(t)
	\label{eq:detector_direct_photocurrent}
	.
\end{equation}

\FloatBarrier
\subsection{Balanced (quadrature) detector}

% literature review
% \cite{Kikuchi2016}
% \cite{Shapiro2009}
% \cite{Loudon2000}
% \cite[p.~206]{Vogel2006}

\begin{figure}[htb]
    \centering
    \includegraphics{figures/tikz/detector-balanced}
    \caption{Optical part of the balanced detector comprising a beam splitter and two photodetectors: A first signal mode $\hat{a}_s=\hat{a}_1$ enters the beam splitter from the left. A second \gls{lo} mode $\hat{a}_l=\hat{a}_2$ enters the beam splitter from the top. A first output mode $\hat{a}_1^\prime$ exits the beam splitter to the right where a first photodetector with photocurrent $i_1$ is placed. A second output mode $\hat{a}_2^\prime$ exits the beam splitter to the bottom where a second photodetector with photocurrent $i_2$ is placed.}\label{fig:detector_balanced_optics}
\end{figure}

\begin{figure}[htb]
    \centering
    \includegraphics{figures/circuitikz/detector-balanced}
    \caption{Electronic schematic of a direct (intensity) detector comprising a photodiode and an operational amplifier in \gls{tia} configuration. The anode of the photodiode is reverse biased with voltage $-V_b<0$. The cathode of the photodiode emits the photocurrent $i_p$ and is connected to the inverting input of the operational amplifier. The non-inverting input of the operational amplifier is connected with ground, while the inverting input is coupled with feedback impedance $Z_f$ to the operational amplifier output.}\label{fig:detector_balanced_electronics}
\end{figure}

We write the frequency-dependent beam splitter \cite[p.~207]{Vogel2006}
\begin{equation}
	\begin{pmatrix}
		\hat{a}_1^\prime(\omega) \\
		\hat{a}_2^\prime(\omega)
	\end{pmatrix}
	=
	\begin{pmatrix}
		t(\omega) & r(\omega) \\
		-r^*(\omega) & t^*(\omega)
	\end{pmatrix}
	\begin{pmatrix}
		\hat{a}_s(\omega) \\
		\hat{a}_l(\omega)
	\end{pmatrix}
	=
	\begin{pmatrix}
		t(\omega)\hat{a}_s(\omega) + r(\omega)\hat{a}_l(\omega) \\
		t^*(\omega)\hat{a}_l(\omega) - r^*(\omega)\hat{a}_s(\omega)
	\end{pmatrix}
\end{equation}
and find the transformed number operators of the output modes to be
\begin{align}
	\hat{n}_1^\prime(\omega)
	&=
	\abs{t(\omega)}^2
	\hat{n}_s(\omega)
	+
	\abs{r(\omega)}^2
	\hat{n}_l(\omega)
	+
	\left[
		t(\omega)^*
		r(\omega)
		\hat{a}_s^\dagger(\omega)
		\hat{a}_l(\omega)
		+
		\text{h.c.}
	\right]
	\\
	\hat{n}_2^\prime(\omega)
	&=
	\abs{r(\omega)}^2
	\hat{n}_s(\omega)
	+
	\abs{t(\omega)}^2
	\hat{n}_l(\omega)
	-
	\left[
		t(\omega)^*
		r(\omega)
		\hat{a}_s^\dagger(\omega)
		\hat{a}_l(\omega)
		+
		\text{h.c.}
	\right]
\end{align}
Assuming the frequency-dependent beam splitter to be sufficiently balanced
\begin{equation}
	\abs{t(\omega)}
	\approx
	\abs{r(\omega)}
\end{equation}
within the frequencies of the detector bandwidth $B$, the photon number difference between the detector is
\begin{equation}
	\begin{split}
		\hat{n}^\prime_\Delta(\omega)
		=
		\hat{n}_1^\prime(\omega)
		-
		\hat{n}_2^\prime(\omega)
		&\approx
		2\left[
			t(\omega)^*
			r(\omega)
			\hat{a}_s^\dagger(\omega)
			\hat{a}_l(\omega)
			+
			\text{h.c.}
		\right]
		\\
		&=
		2\abs{t(\omega)r(\omega)}
		\left[
			\hat{a}_s^\dagger(\omega)
			\hat{a}_l(\omega)
			e^{i(\phi_r-\phi_t)}
			+
			\text{h.c.}
		\right]
	\end{split}
\end{equation}
Assuming the quantum state~\cite[p.~213]{Vogel2006}
\begin{equation}
	\hat\rho(t)
	=
	\hat\rho_s(t)
	\otimes
	\hat\rho_l(t)
	=
	\hat\rho_s(t)
	\otimes
	\ketbra{\alpha_l(t)}
\end{equation}
wherein the \gls{lo} coherent state is
\begin{equation}
	\ket{\alpha_l(t)}
	=
	e^{-\overline{n}_l(t)/2}
	\exp\left\{
		\int_0^\infty\dd{\omega}
		\alpha_l(\omega,t)
		\hat{a}_l^\dagger(\omega)
	\right\}
	\ket{0}
	.
\end{equation}
We then find
\begin{equation}
	\begin{split}
		\overline{n}^\prime_\Delta(t)
		&=
		\int_0^\infty\frac{\dd{\omega}}{2\pi}
		\trace\left\{
			\hat\rho(t)
			\hat{n}_\Delta^\prime(\omega)
		\right\}
		\\
		&=
		2
		\int_0^\infty\frac{\dd{\omega}}{2\pi}
		\abs{t(\omega)r(\omega)}
		\trace_s\left\{
			\hat\rho_s(t)
			\expval{
				\hat{a}_s^\dagger(\omega)
				\hat{a}_l(\omega)
				e^{+i(\phi_r-\phi_t)}
				+
				\text{h.c.}
			}{\alpha_l(t)}
		\right\}
		\\
		&=
		2
		\int_0^\infty\frac{\dd{\omega}}{2\pi}
		\abs{t(\omega)r(\omega)}
		\trace_s\left\{
			\hat\rho_s(t)
			\expval{
				\hat{a}_s^\dagger(\omega)
				\alpha_l(\omega,t)
				e^{+i(\phi_r-\phi_t)}
				+
				\text{h.c.}
			}{\alpha_l(t)}
		\right\}
		\\
		&=
		2
		\int_0^\infty\frac{\dd{\omega}}{2\pi}
		\abs{t(\omega)r(\omega)\alpha_l(\omega)}
		\trace_s\left\{
			\hat\rho_s(t)
			\left[
				\hat{a}_s^\dagger(\omega)
				e^{-i\vartheta(t)}
				+
				\text{h.c.}
			\right]
		\right\}
		\\
		&=
		2\sqrt{2}
		\int_0^\infty\frac{\dd{\omega}}{2\pi}
		\abs{t(\omega)r(\omega)\alpha_l(\omega)}
		\trace_s\left\{
			\hat\rho_s(t)
			\hat{X}\left(\omega,\vartheta(t)\right)
		\right\}
		\\
		&=
		2\sqrt{2}
		\int_0^\infty\frac{\dd{\omega}}{2\pi}
		\abs{t(\omega)r(\omega)\alpha_l(\omega)}
		\expval{\hat{X}\left(\omega,\vartheta(\omega,t)\right)}_s
	\end{split}
\end{equation}
with
\begin{equation}
	\vartheta(\omega,t)
	=
	\phi_t-\phi_r-\varphi_l(\omega,t)
\end{equation}
and
\begin{equation}
	\hat{X}\left(\omega,\vartheta(t)\right)
	=
	\frac{1}{\sqrt{2}}
	\left(
		\hat{a}_s(\omega)
		e^{+i\vartheta(t)}
		+
		\hat{a}_s^\dagger(\omega)
		e^{-i\vartheta(t)}
	\right)
	.
\end{equation}
We can further write
\begin{equation}
	\overline{n}^\prime_\Delta(t)
	=
	\left(h*\overline{X}\right)(t)
\end{equation}
where we defined the filter
\begin{equation}
	h(\omega)
	=
	2\sqrt{2}
	\abs{t(\omega)r(\omega)\alpha_l(\omega)}
	.
\end{equation}

We are left to relate the photon number difference operator with the analog signals.
\Cref{fig:balanced_detector_electronics} shows the schematic of the balanced detector and \gls{tia} circuitry.
The photodiodes are both biased with bias voltage $\pm V_b$ to improve frequency response.
At the node between the anode of $\text{PD}_1$ and the cathode of $\text{PD}_2$, the two photocurrents $i_1$ and $i_2$ are directly subtracted.
\begin{figure}[htb]
    \centering
    \includegraphics{figures/circuitikz/detector-balanced}
    \caption{Electronic part of the balanced detector comprising two biassed photodiodes in balanced configuration with a \gls{tia} frontend: The cathode of a first photodiode $\text{PD}_1$ is biased with a positive bias voltage $+V_b$. The anode of a second photodiode $\text{PD}_2$ is biased with a negative bias voltage $-V_b$. The photocurrent difference $i_1-i_2$ runs through the line connecting the anode of $\text{PD}_1$ and the cathode of $\text{PD}_2$. A \gls{tia} with complex feedback impedance $Z_f$ converts and amplifies the photocurrent difference $i_1-i_2$ to an output voltage $V_o$.}\label{fig:balanced_detector_electronics}
\end{figure}
The mean of the photocurrent difference is equal to
\begin{equation}
	\overline{\Delta i}(t)
	=
	\overline{i}_1(t)
	-
	\overline{i}_2(t)
	=
	\int\dd{t^\prime}
	\eta(t^\prime)
	\overline{n}_\Delta^\prime(t-t^\prime)
\end{equation}
wherein $\eta$ is the \gls{qe} or frequency response of the photodiodes.
The mean photocurrent difference is amplified and converted to a voltage signal $V_0$ by the \gls{tia} frontend.
In particular,
\begin{equation}
	V_o(\omega)
	=
	-
	Z_f(\omega)
	\overline{\Delta i}(\omega)
\end{equation}
where $Z_F$ is the complex feedback impedance.
The mean voltage signal is equal to
\begin{equation}
	V_o(t)
	=
	-
	\left(Z_f*\overline{\Delta i}\right)(t)
	=
	-
	\left(Z_f*h*\overline{X}\right)(t)
	.
\end{equation}

The variance of a photodetector with efficiency $\eta$ is given by~\cite[p.~194]{Vogel2006}
\begin{equation}
	\begin{split}
		\overline{(\Delta n^\prime)^2}(\omega)
		&=
		\eta(\omega)
		\expval{\hat{n}^\prime(\omega)}
		+
		\eta(\omega)^2
		\expval{\colon\left(\Delta\hat{n}^\prime(\omega)\right)^2\colon}
		\\
		&=
		\eta(\omega)\left(1-\eta(\omega)\right)
		\expval{\hat{n}^\prime(\omega)}
		+
		\eta(\omega)^2
		\expval{\left(\Delta\hat{n}^\prime(\omega)\right)^2}
	\end{split}
\end{equation}
\begin{equation}
	\begin{split}
		\expval{\colon\hat{n}_\Delta^\prime(\omega)^2\colon}
		&=
		4\abs{t(\omega)r(\omega)}^2
		\expval{
			\colon
			\left[
				\hat{a}_s^\dagger(\omega)
				\hat{a}_l(\omega)
				e^{+i(\phi_r-\phi_l)}
				+
				\hat{a}_s(\omega)
				\hat{a}_l^\dagger(\omega)
				e^{-i(\phi_r-\phi_l)}
			\right]^2
			\colon
		}
		\\
		&=
		4\abs{t(\omega)r(\omega)}^2
		\expval{
			\hat{a}_s^\dagger(\omega)^2
			\hat{a}_l(\omega)^2
			e^{+2i(\phi_r-\phi_l)}
			+
			\hat{a}_l^\dagger(\omega)^2
			\hat{a}_s(\omega)^2
			e^{-2i(\phi_r-\phi_l)}
			+
			2
			\hat{a}_s^\dagger(\omega)
			\hat{a}_l^\dagger(\omega)
			\hat{a}_l(\omega)
			\hat{a}_s(\omega)
		}
		\\
		&=
		4\abs{t(\omega)r(\omega)}^2
		\left[
			2
			\overline{n}_s(t)
			\overline{n}_l(t)
			+
			\dots
		\right]
	\end{split}
\end{equation}
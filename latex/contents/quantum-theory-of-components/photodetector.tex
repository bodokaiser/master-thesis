\section{Photodetector}

et us consider the composite system of an atom with a single electron and a radiation field.
\textcolor{red}{Figure where we see an electron in a potential well and how radiation can excite it...}

In the ground state $\ket{g}$, the electron is bound and satisfies
\begin{equation}
	\hat{H}_a
	\ket{g}
	=
	E_g
	\ket{g}
	.
\end{equation}
For energies $E>0$, the electron is in one of many free excited state $\ket{e}$.
\textcolor{red}{photoelectric effect. why can this be used for photodiodes where there is no ionization happening. See Cohen-Tannoudji for explanation.}

Whenever, we ionize the atom, we can collect the free electron indicating a photo detection.
The probability that such an event occurs in the time interval $[t_0,t_0+\Delta t]$ is
\begin{equation}
	p_{e,f}(t_0,t_0+\Delta t)
	=
	\abs{
		\bra{e,f}
		\hat{U}(t_0,t_0+\Delta t)
		\ket{g,i}
	}^2
	.
\end{equation}
wherein $\hat{U}$ is the time-evolution operator.
We can recast the probability in the more general density operator formalism
\begin{equation}
	\begin{split}
		p_{e,f}(t_0,t_0+\Delta t)
		&=
		\bra{e,f}
		\hat{U}(t_0,t_0+\Delta t)
		\ketbra{g,i}
		\hat{U}^\dagger(t_0,t_0+\Delta t)
		\ket{e,f}
		\\
		&=
		\trace\left\{
			\bra{e,f}
			\hat{U}(t_0,t_0+\Delta t)
			\ketbra{g,i}
			\hat{U}^\dagger(t_0,t_0+\Delta t)
			\ket{e,f}
		\right\}
		\\
		&=
		\trace\left\{
			\ketbra{e,f}
			\hat{U}(t_0,t_0+\Delta t)
			\ketbra{g,i}
			\hat{U}^\dagger(t_0,t_0+\Delta t)
		\right\}
		\\
		&=
		\trace\left\{
			\hat\varrho
			\hat{U}(t_0,t_0+\Delta t)
			\hat\rho(t_0)
			\hat{U}^\dagger(t_0,t_0+\Delta t)
		\right\}
		\\
		&=
		\trace\left\{
			\hat\varrho
			\hat\rho(t_0+\Delta t)
		\right\}
	\end{split}
\end{equation}
where we used that the trace of a scalar is the scalar in the second line and the cyclic property of the trace in the third line.
The time-evolution operator is
\begin{equation}
	\hat{U}(t_0,t)
	=
	\mathcal{T}_+
	\exp\left\{
		-i
		\int_{t_0}^t\dd{t^\prime}
		\hat{H}_\text{int}(t^\prime)
	\right\}
\end{equation}
with the interaction Hamiltonian in the electric-dipole and rotating wave approximation being
\begin{equation}
	\hat{H}_\text{int}(t)
	=
	-
	\hat{\vb{p}}(t)
	\vdot
	\hat{\vb{A}}(\vb{x}_0,t)
	.
\end{equation}
Instead of the time-ordered exponential, we can use the Magnus expansion
\begin{align}
	\hat{U}(t_0,t)
	&=
	e^{\Omega(t_0,t)}
	&
	\Omega(t_0,t)
	&=
	\sum_{n=1}\Omega^{(n)}(t_0,t)
\end{align}
The time evolved quantum state is then
\begin{equation}
	\begin{split}
		\hat\rho(t_0+\Delta t)
		&=
		\hat{U}(t_0,t_0+\Delta t)
		\hat\rho(t_0)
		\hat{U}^\dagger(t_0,t_0+\Delta t)
		\\
		&=
		\hat\rho(t_0)
		+
		\comm{\Omega(t_0,t)}{\hat\rho(t_0)}
		+
		\frac{1}{2!}
		\comm{\Omega(t_0,t)}{\comm{\Omega(t_0,t)}{\hat\rho(t_0)}}
		+
		\dots
	\end{split}
\end{equation}
where we used the \gls{bch} formula.
We perform Magnus expansion up to the first term and find the perturbative solution
\begin{equation}
	\begin{split}
		\hat\rho(t_0+\Delta t)
		\approx
		\hat\rho(t_0)
		&+
		(-i)
		\int_{t_0}^{t_0+\Delta t}\dd{t_1}
		\comm{\hat{H}_\text{int}(t_1)}{\hat\rho(t_0)}
		\\
		&+
		\frac{(-i)^2}{2!}
		\int_{t_0}^{t_0+\Delta t}\dd{t_1}
		\int_{t_0}^{t_1}\dd{t_2}
		\comm{\hat{H}_\text{int}(t_1)}{\comm{\hat{H}_\text{int}(t_2)}{\hat\rho(t_0)}}
	\end{split}	
\end{equation}
and insert the expansion into the photoemission probability
\begin{equation}
	\begin{split}
		p_{e,f}(t_0,t_0+\Delta t)
		&=
		\trace\left\{
			\hat\varrho
			\hat\rho(t_0+\Delta t)
		\right\}
		\\
		&=
		\int_{t_0}^{t_0+\Delta t}\dd{t_1}
		\int_{t_0}^{t_1}\dd{t_2}
		\trace\left\{
			\hat{H}_\text{int}(t_1)
			\hat\rho(t_0)
			\hat{H}_\text{int}(t_2)
		\right\}
		+
		\text{c.c.}
		\\
		&=
		\bra{e}\hat{p}_i\ket{g}
		\bra{g}\hat{p}_j\ket{e}
		\int_{t_0}^{t_0+\Delta t}\dd{t_1}
		\int_{t_0}^{t_1}\dd{t_2}
		e^{i(E_e-E_g)(t_1-t_2)}
		\\
		&\times
		\expval{
			\hat{A}_j(\vb{x}_0,t_2)
			\hat\rho(t_0)
			\hat{A}_i(\vb{x}_0,t_1)
		}{i}
		+
		\text{c.c.}
	\end{split}
\end{equation}
\textcolor{red}{steps to second line missing!}
We are not interested in the final states and can integrate this degree of freedom.
Furthermore, we can assume an ensemble of independent detector atoms, so the probability for a photoemission is
\begin{equation}
	p(t_0,t_0+\Delta t)
	=
	\int_{t_0}^{t_0+\Delta t}\dd{t_1}
	\int_{t_0}^{t_1}\dd{t_2}
	k_{ij}(t_1-t_2)
	\expval{
		\hat{A}_j(\vb{x}_0,t_2)
		\hat{A}_i(\vb{x}_0,t_1)
	}
	+
	\text{c.c.}
\end{equation}
wherein $k_{ij}$ encodes the microscopic properties of the detector atom ensemble, see Ref.~\cite[p.~694]{Mandel1995}.
Expansion of the Maxwell field operator into positive and negative frequency parts as well as discarding high-frequency terms, we find~\cite[p.~698]{Mandel1995}.
\begin{equation}
	p(t_0,t_0+\Delta t)
	\approx
	\int_0^{\Delta t}\dd{t^\prime}
	\int_0^{t^\prime}\dd{t^{\prime\prime}}
	k_{ij}(t^{\prime}-t^{\prime\prime})
	\expval{
		\hat{A}_j^{(+)}(\vb{x}_0,t_0+t^{\prime\prime})
		\hat{A}_i^{(-)}(\vb{x}_0,t_0+t^{\prime})
	}
	+
	\text{c.c.}
	.
\end{equation}
Furthermore, performing the quasi-monochromatic approximation
\begin{align}
	\hat{A}_j^{(+)}(\vb{x}_0,t_0+t^{\prime\prime})
	&\approx
	\hat{A}_j^{(+)}(\vb{x}_0,t_0)
	e^{+i\omega_0t^{\prime\prime}}
	\\
	\hat{A}_j^{(-)}(\vb{x}_0,t_0+t^{\prime})
	&\approx
	\hat{A}_j^{(-)}(\vb{x}_0,t_0)
	e^{-i\omega_0t^{\prime}}
\end{align}
we find
\begin{equation}
	\begin{split}
		p(t_0,t_0+\Delta t)
		&\approx
		\expval{
			\hat{A}_j^{(+)}(\vb{x}_0,t_0)
			\hat{A}_i^{(-)}(\vb{x}_0,t_0)
		}
		\int_0^{\Delta t}\dd{t^\prime}
		\int_0^{t^\prime}\dd{t^{\prime\prime}}
		k_{ij}(t^{\prime}-t^{\prime\prime})
		e^{-i\omega_0(t^\prime-t^{\prime\prime})}
		+
		\text{c.c.}
		\\
		&=
		\expval{
			\hat{A}_j^{(+)}(\vb{x}_0,t_0)
			\hat{A}_i^{(-)}(\vb{x}_0,t_0)
		}
		\int_0^{\Delta t}\dd{t^\prime}
		\int_0^{t^\prime}\dd{\tau}
		k_{ij}(\tau)
		e^{-i\omega_0\tau}
		+
		\text{c.c.}
		\\
		&=
		\expval{
			\hat{A}_j^{(+)}(\vb{x}_0,t_0)
			\hat{A}_i^{(-)}(\vb{x}_0,t_0)
		}
		\int_0^{\Delta t}\dd{t^\prime}
		\int_{-t^\prime}^{t^\prime}\dd{\tau}
		k_{ij}(\tau)
		e^{-i\omega_0\tau}
		\\
		&\approx
		\expval{
			\hat{A}_j^{(+)}(\vb{x}_0,t_0)
			\hat{A}_i^{(-)}(\vb{x}_0,t_0)
		}
		k_{ij}(\omega_0)
		\Delta t
	\end{split}
\end{equation}
as in Ref.~\cite[p.~699]{Mandel1995} where $k_{ij}(\omega_0)$ is the frequency response of the detector atom ensemble.
We assume a detector equally sensitive to both polarizations and finally find
\begin{equation}
	p(t_0,t_0+\Delta t)
	\approx
	\expval{\hat{N}}
	k(\omega_0)
	\Delta t
\end{equation}
we can even relax the quasi-monochromatic approximation a bit by writing
\begin{equation}
	p(t_0,t_0+\Delta t)
	\approx
	\int\dd{\omega}
	k(\omega)
	\expval{\hat{n}(\omega)}
	\Delta t
	.
\end{equation}
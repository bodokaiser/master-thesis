\section*{Summary}
\addcontentsline{toc}{section}{Summary}

In this chapter, we presented quantum models of the optical coupler, electro-optical modulator, and detector.
We left out a quantum theory of a coherent-state source with narrow line width in the optical range, typically implemented by a laser.
Proper quantum-treatment of the laser requires analysis of nonlinear quantum-stochastic equations and is beyond the scope of this thesis.
We refer the interested reader to Ref.~\cite[p.~900]{Mandel1995} and Ref.~\cite{Haken2012,Gardiner2000}.
For an intuitive argument why lasers emit coherent states based on decoherence, see Ref.~\cite{Gea1998}.

For the lossless and \gls{lti} optical coupler, we found the most intuitive a characterization in terms of a unitary scattering matrix comprising reflection and transmission coefficients~\cite{Vogel2006}.
That said, we also investigated theoretical methods including evolution operators~\cite{Haroche2006} and the Jordan-Schwinger operator algebra~\cite{Leonhardt2003}.
\begin{figure}[htb]
    \centering
    \includegraphics{figures/tikz/components-coupler}
    \caption{Quantum-optical coupler with the input coherent-states, $\ket{\alpha(t)}$ and $\ket{\beta^\prime(t)}$, on the left side, and the output coherent-states, $\ket{\alpha^\prime(t)}$ and $\ket{\beta^\prime(t)}$, on the right side.}\label{fig:components_two_port}
\end{figure}
Concerning coherent states, the scattering matrix connects the amplitudes of the coherent in- and output states in \Cref{fig:components_two_port} via
\begin{equation}
	\begin{pmatrix}
		\alpha^\prime(t)
		\\
		\beta^\prime(t)
	\end{pmatrix}
	=
	\begin{pmatrix}
		r(t) & t^\prime(t) \\
		t(t) & r^\prime(t)
	\end{pmatrix}
	\conv
	\begin{pmatrix}
		\alpha(t)
		\\
		\beta(t)
	\end{pmatrix}	
\end{equation}
in the time and
\begin{equation}
	\begin{pmatrix}
		\alpha^\prime(\omega)
		\\
		\beta^\prime(\omega)
	\end{pmatrix}
	=
	\begin{pmatrix}
		r(\omega) & t^\prime(\omega) \\
		t(\omega) & r^\prime(\omega)
	\end{pmatrix}
	\begin{pmatrix}
		\alpha(\omega)
		\\
		\beta(\omega)
	\end{pmatrix}	
\end{equation}
in the frequency domain.
The complex reflection and transmission coefficients, the scattering-matrix parameters, are required to satisfy
\begin{equation}
	\abs{r(\omega)}
	+
	\abs{t^\prime(\omega)}
	=
	1
	=
	\abs{t(\omega)}
	+
	\abs{r^\prime(\omega)}
\end{equation}
for conservation of energy flow and the \gls{ccr}, making the scattering matrix unitary.
If we assume the input coherent-states to be narrow-bandwidth and the optical coupler to have a constant frequency response over the optical bandwidth, the convolution in the time domain reduces to a multiplication.
Certain embodiments of the optical coupler, such as the plate beam-splitter, arbitrarily superimpose two input coherent-states.
That said, most implementations of optical couplers exhibit backscattering, which accounting for requires an optical four-port.
As the scattering matrix of an optical four-port requires \num{16} complex parameters, most calculations become impractical.
\begin{figure}[htb]
    \centering
    \includegraphics{figures/tikz/components-filter}
    \caption{Quantum optical filter with input coherent-state, $\ket{\alpha(t)}$, and output coherent-state, $\ket{\alpha^\prime(t)}$.}\label{fig:components_filter}
\end{figure}
An optical coupler with a vacuum state as a second input and tracing out one output effectively implements a linear filter.
The output coherent-state in \Cref{fig:components_filter} relates to the input coherent-state via
\begin{align*}
	\alpha^\prime(t)
	&=
	\left(h\conv\alpha\right)(t)
	&
	\alpha^\prime(\omega)
	&=
	h(\omega)
	\alpha(\omega)
\end{align*}
in the time and frequency domain, wherein the frequency-response function of the filter is required to satisfy $\abs{h(\omega)}\leq1$.

Using the linear electro-optical effect and nonlinear frequency-conversion with the help of Ref.~\cite{Horoshko2018,QuesadaMejia2015}, we derived a unitary evolution operator corresponding to phase modulation with a sinusoidal signal.
Neglecting second-order interactions between the modulation sidebands, we argued that phase modulation with an arbitrary signal is thinkable.
\begin{figure}[htb]
    \centering
    \includegraphics{figures/tikz/components-modulator-phase}
    \caption{Quantum-optical phase modulator with electric phase signal $\varphi(t)$, input coherent-state $\ket{\alpha(t)}$, and output coherent-state, $\ket{\alpha^\prime(t)}$.}\label{fig:components_modulator_phase}
\end{figure}
The amplitude of the output coherent-state in \Cref{fig:components_modulator_phase} is
\begin{align}
	\alpha^\prime(t)
	&=
	\alpha(t)
	e^{i\varphi(t)}
	&
	\alpha^\prime(\omega)
	&=
	\left(g\conv\alpha\right)(\omega)
\end{align}
in the time and frequency domain, where we can only give an explicit expression for the kernel $g(\omega)$ in the case of a sinusoidal modulation, \cref{eq:phase_modulation_sinusoidal_kernel}.
Arranging two electro-optical phase modulators in an \gls{mzm} and driving the modulators with a differential voltage enables electro-optical amplitude modulation.
Arranging two electrically-driven \glspl{mzm} with a relative phase shift of $\pi/2$ generalizes amplitude modulation with a real- to a complex-valued signal.
\begin{figure}[htb]
    \centering
    \includegraphics{figures/tikz/components-modulator-amplitude}
    \caption{Quantum-optical amplitude modulator with input voltage-signals, $x(t)$ and $p(t)$, input coherent-state $\ket{\alpha(t)}$, and output coherent-state $\ket{\alpha^\prime(t)}$.}\label{fig:components_modulator_phase}
\end{figure}
The amplitude of the output coherent-state in \Cref{fig:components_modulator_phase} is
\begin{equation}
	\alpha^\prime(t)
	=
	\left[
		x(t)
		+
		ip(t)
	\right]
	\alpha(t)
	\qquad
	\abs{x(t)},
	\abs{p(t)}
	\leq
	1
	,
\end{equation}
wherein $x(t)$ and $p(t)$ are proportional to the differential voltages driving the \glspl{mzm} and the constraint follows from the constructive interference used for \gls{am}.
As with the phase modulator, no closed form exists for the convolution kernel in the frequency domain.

Regarding the detectors, we studied photodetection theory from Ref.~\cite{Mandel1995,Kimble1984,Vogel2006} in the appendix, \Cref{app:photodetection_theory}, and started employing the photocurrent operator to predict the mean photocurrent.
Additionally, we presented an electric circuit of a \gls{tia} converting and amplifying the photocurrent to a voltage signal.
\begin{figure}[htb]
    \centering
    \includegraphics{figures/tikz/components-detector-direct}
    \caption{Quantum-optical direct detector with input coherent-state $\ket{\alpha(t)}$, and output voltage-signal $y(t)$.}\label{fig:components_detector_direct}
\end{figure}
The output voltage-signal $y(t)$ in \Cref{fig:components_detector_direct} is equal to the photon flux $\abs{\alpha(t)}^2$ convolved with the combined response-function of the detector and the \gls{tia} $\eta(\omega)$, i.e.,
\begin{equation}
	y(t)
	=
	\left(\eta\conv\abs{\alpha}^2\right)(t)
	=
	\int_{-\infty}^{+\infty}\frac{\dd{\omega}}{2\pi}
	\eta(\omega)
	\abs{\alpha(\omega)}^2
	,
\end{equation}
wherein $\abs{\alpha(\omega)}^2$ is the signal power.
The direct detector is a square-law detector where the photocurrent is proportional to the signal power.
To resolve the quadratures of the signal, we combine the input with a \gls{lo} signal in an optical coupler and monitor the outputs with two direct detectors in a balanced configuration.
\begin{figure}[htb]
    \centering
    \includegraphics{figures/tikz/components-detector-balanced}
    \caption{Quantum-optical balanced detector with input coherent-state $\ket{\alpha(t)}$, \gls{lo} coherent-state $\ket{e^{i(\omega_0t+\vartheta)}}$, and output voltage-signal $z(t)$.}\label{fig:components_detector_balanced}
\end{figure}
The output voltage-signal $z(t)$ in \Cref{fig:components_detector_balanced} is equal to the projection of the downconverted signal $\alpha(t)$ onto a real axis under the \gls{lo} angle $\vartheta$, i.e.,
\begin{equation}
	z(t)
	=
	2\Re
	\int_{-\infty}^{+\infty}\frac{\dd{\omega}}{2\pi}
	\eta(\omega+\omega_l)
	\alpha(\omega+\omega_l)
	e^{+i(\omega t+\vartheta)}
	,
\end{equation}
wherein $\omega_l$ and $\vartheta$ are the \gls{lo} frequency and phase, and the \gls{lo} lineshape is absorbed into the signal $\alpha(\omega)$.
Comparison of the balanced detector's mean voltage signal with the expectation value of the quadrature operator leads us to motivate the generalized quadrature operator
\begin{equation}
	\hat{X}(t;\omega_l,B_d)
	=
	\int_{-B_d/2}^{+B_d/2}\frac{\dd{\omega}}{2\pi}
	\left[
		\hat{a}(\omega+\omega_l)
		e^{-i(\omega t+\vartheta)}
		+
		\text{H.c.}
	\right]
	,
\end{equation}
accounting for an effective detector-bandwidth $B_d$ and downconversion frequency and phase, by combining the frequency-conversion operator with a spectral filter.
Our results are compatible with Ref.~\cite{Gardiner2000,Shapiro2009,Loudon2000,Vogel2006,Kikuchi2016} on homo- and heterodyne detection but are more general in that we account for continuous-time signals, downconversion frequency and phase, and the detector bandwidth.
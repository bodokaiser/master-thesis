\section*{Summary}
\addcontentsline{toc}{section}{Summary}

In this chapter, we presented quantum models of the optical coupler, electro-optical modulator, and detector.
What we are missing is a quantum theory of a coherent state source with narrow linewidth in the optical range, typically implemented by a laser.
A full laser quantum treatment of the laser requires analysis of nonlinear quantum stochastic equations and is out of the scope of this thesis but can be found in Ref.~\cite[p.~900]{Mandel1995} and Ref.~\cite{Haken2012,Gardiner2000}.
For an intuitive argument why lasers emit coherent states based on decoherence, see Ref.~\cite{Gea1998}.

One of the most important components is the optical two-port characterized by the complex transmission and reflection coefficients of the two inputs.
\begin{figure}[htb]
    \centering
    \includegraphics{figures/tikz/components-two-port}
    \caption{Quantum optical two-port with the incident and reflected coherent states, $\ket{\alpha(t)}$ and $\ket{\alpha^\prime(t)}$, on the left side, and the incident and reflected coherent states, $\ket{\beta(t)}$ and $\ket{\beta^\prime(t)}$, on the right side.}\label{fig:components_two_port}
\end{figure}
For a lossless \gls{lti} two-port, the amplitudes of the in- and output coherent states in \Cref{fig:components_two_port} are related by
\begin{align}
	\begin{pmatrix}
		\alpha^\prime(t)
		\\
		\beta^\prime(t)
	\end{pmatrix}
	&=
	\begin{pmatrix}
		r(t) & t^\prime(t) \\
		t(t) & r^\prime(t)
	\end{pmatrix}
	\conv
	\begin{pmatrix}
		\alpha(t)
		\\
		\beta(t)
	\end{pmatrix}
	\\
	\begin{pmatrix}
		\alpha^\prime(\omega)
		\\
		\beta^\prime(\omega)
	\end{pmatrix}
	&=
	\begin{pmatrix}
		r(\omega) & t^\prime(\omega) \\
		t(\omega) & r^\prime(\omega)
	\end{pmatrix}
	\begin{pmatrix}
		\alpha(\omega)
		\\
		\beta(\omega)
	\end{pmatrix}
\end{align}
in the time and frequency domain, wherein the complex reflection and transmission coefficients, the scattering paramters, are required to satisfy
\begin{equation}
	\abs{r(\omega)}
	+
	\abs{t^\prime(\omega)}
	=
	1
	=
	\abs{t(\omega)}
	+
	\abs{r^\prime(\omega)}
\end{equation}
for conservation of energy flow and the \gls{ccr}.
If we assume the coherent input states to be of narrow-bandwidth and the optical two-port to have constant frequency response over the optical bandwidth, the convolution in the time domain reduces to a multiplication.
Certain embodiments of the optical two-port, such as the plate beam splitter, implement an optical coupler, which arbitrarily superimposes two coherent input states.
Most implementations of optical couplers exhibit backscattering, which accounting for requires an optical four-port.
However, an optical four-port requires \num{16} complex parameters for characterization, which makes most calculations impractical and one therefore assumes an ideal two-port for coupling.
\begin{figure}[htb]
    \centering
    \includegraphics{figures/tikz/components-filter}
    \caption{Quantum optical filter with input coherent state, $\ket{\alpha(t)}$, and output coherent state, $\ket{\alpha^\prime(t)}$.}\label{fig:components_filter}
\end{figure}
An optical two-port with a vacuum state as a second input and tracing out one output effectively implements a linear filter.
The coherent output state in \Cref{fig:components_filter} relates to the coherent input state via
\begin{align*}
	\alpha^\prime(t)
	&=
	\left(h\conv\alpha\right)(t)
	&
	\alpha^\prime(\omega)
	&=
	h(\omega)
	\alpha(\omega)
\end{align*}
in the time and frequency domain, wherein the frequency response function of the filter is required to satisfy $\abs{h(\omega)}\leq1$.

Using the linear electro-optical effect and nonlinear frequency conversion with the help of Ref.~\cite{Horoshko2018,QuesadaMejia2015}, we derived a unitary evolution corresponding to phase modulation with a sinusoidal signal.
Neglecting second-order interactions between the modulation sidebands, we argued that phase modulation with an arbitrary signal is thinkable.
\begin{figure}[htb]
    \centering
    \includegraphics{figures/tikz/components-modulator-phase}
    \caption{Quantum optical phase modulator with electric phase signal $\varphi(t)$, coherent input state, $\ket{\alpha(t)}$, and coherent output state, $\ket{\alpha^\prime(t)}$.}\label{fig:components_modulator_phase}
\end{figure}
The amplitude of the coherent output state in \Cref{fig:components_modulator_phase} is
\begin{align}
	\alpha^\prime(t)
	&=
	\alpha(t)
	e^{i\varphi(t)}
	&
	\alpha^\prime(\omega)
	&=
	\left(g\conv\alpha\right)(\omega)
\end{align}
in the time and frequency domain, wherein we can only give an explicit expression for the kernel $g(\omega)$ in the case of a sinusoidal modulation, \cref{eq:phase_modulation_sinusoidal_kernel}.
Arranging two electro-optical phase modulators in an \gls{mzm} and driving the modulators with a differential voltage enables electro-optical amplitude modulation.
Arranging two electrically driven \gls{mzm}s with a relative phase shift of $\pi/2$ generalizes amplitude modulation with a real- to a complex-valued signal.
\begin{figure}[htb]
    \centering
    \includegraphics{figures/tikz/components-modulator-amplitude}
    \caption{Quantum optical amplitude modulator with input voltage signals, $x(t)$ and $p(t)$, coherent input state, $\ket{\alpha(t)}$, and coherent output state $\ket{\alpha^\prime(t)}$.}\label{fig:components_modulator_phase}
\end{figure}
The amplitude of the coherent output state in \Cref{fig:components_modulator_phase} is
\begin{equation}
	\alpha^\prime(t)
	=
	\left[
		x(t)
		+
		ip(t)
	\right]
	\alpha(t)
	,
\end{equation}
wherein $x(t)$ and $p(t)$ are proportional to the differential voltages driving the \gls{mzm}s.
As with the phase modulator, no closed form exists for the convolution kernel in the frequency domain.
Furthermore, one should keep in mind that constructive interference in the \gls{mzm}s constraints the components of the complex amplitude modulation, $x(t)+ip(t)$, by $\abs{x(t)},\abs{p(t)}\leq1$.

%Comparison of results with literature
% literature review
% \cite{Kikuchi2016}
% \cite{Shapiro2009}
% \cite{Loudon2000}
% \cite[p.~206]{Vogel2006}
% quantum noise p. 265


\begin{figure}[htb]
    \centering
    \includegraphics{figures/tikz/components-detector-direct}
    \caption{Quantum optical direct detector with coherent input state, $\ket{\alpha(t)}$, and voltage output signal, $y(t)$.}\label{fig:components_detector_direct}
\end{figure}

\begin{figure}[htb]
    \centering
    \includegraphics{figures/tikz/components-detector-balanced}
    \caption{Quantum optical balanced detector with coherent input signal state, $\ket{\alpha(t)}$, coherent input \gls{lo} state, $\ket{e^{i(\omega_0t+\vartheta)}}$, and voltage output signal, $z(t)$}\label{fig:components_detector_balanced}
\end{figure}

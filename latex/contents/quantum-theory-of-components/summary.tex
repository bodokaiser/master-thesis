\section*{Summary}
\addcontentsline{toc}{section}{Summary}

The optical coupler, or beam splitter, transforms the coherent input state
\begin{equation}
	\ket{\alpha(\omega),\beta(\omega)}
	\to
	\ket{
		t(\omega)\alpha(\omega)+r(\omega)\beta(\omega),
		t(\omega)^*\beta(\omega)-r(\omega)^*\alpha(\omega)
	}
\end{equation}
wherein the coefficients satisfy
\begin{equation}
	\abs{t(\omega)}^2
	+
	\abs{r(\omega)}^2
	=
	1
	.
\end{equation}
With $\beta(\omega)=0$ and dumping the second output mode, we find the optical filter
\begin{equation}
	\ket{\alpha(\omega)}
	\to
	\ket{t(\omega)\alpha(\omega)}
\end{equation}
wherein $\abs{t(\omega)}\leq1$.

For the different modulators, we find
\begin{table}[htb]
	\centering	
	\begin{tabular}{lcc}
		\toprule
			Modulator & Output state & Constraint \\
		\midrule
			Phase & $\ket{\alpha(t)e^{i\varphi(t)}}$ & $0\leq\varphi(t)<2\pi$ \\
			Amplitude & $\ket{\alpha(t)\beta(t)}$ & $\abs{\beta(t)}\leq1$ \\
		\bottomrule
	\end{tabular}
	\caption{Quantum input-output relations for modulators with input state $\ket{\alpha(t)}$.}
\end{table}

The measurement operator for 
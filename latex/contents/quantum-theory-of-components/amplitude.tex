\section{Amplitude modulator}

\subsection{Mach-Zehnder modulator}

The \gls{mzm} arranges two phase modulators to perform amplitude and phase modulation on an optical input field.
Using two electrical-driven phase modulators, the \gls{mzm} performs amplitude and phase modulation on an optical input field.
To begin with, we derive the quantum operator matrix transform from a specific implementation of the \gls{mzm}, the symmetric free-space \gls{mzi}.
Then, we link the quantum operator matrix transform to more general unitary operators.
Finally, we derive the quantum state transform for a coherent input state to the \gls{mzm}.

\begin{figure}[htb]
	\centering
	\includestandalone{figures/pstricks/mzi-symmetric}
	\caption{Free-space setup of a symmetric \gls{mzm}: The input light mode $\hat{a}_1$ enters a first beam splitter BS1 from the left. A vacuum light mode $\hat{a}_2$ enters BS2 from the top. The transformed mode $\hat{a}_1^\prime$ and $\hat{a}_2^\prime$ exit BS1 to the right and the bottom. A first phase shifter and first mirror M1, right to BS1, add a relative phase of $\varphi_1+\pi$ from mode $\hat{a}_1$ to $\hat{a}_1^{\prime\prime}$. Below BS1, a second mirror M2 directs the light to a right second phase shifter, both adding a relative phase of $\varphi_2+\pi$ from mode $\hat{a}_2^\prime$ to $\hat{a}_2^{\prime\prime}$. A second beam splitter BS2 transforms the input modes $\hat{a}_1^{\prime\prime}$ and $\hat{a}_2^{\prime\prime}$ to the output modes $\hat{a}_1^{\prime\prime\prime}$ and $\hat{a}_2^{\prime\prime\prime}$.}\label{fig:mzi_symmetric}
\end{figure}
\Cref{fig:mzi_symmetric} shows a free-space optics setup of a symmetric \gls{mzi} with one signal input; the other input being in the vacuum state.
The most crucial components of the \gls{mzi} are a splitter, a coupler, and two independent phase modulators.
The splitter divides the input light into two branches.
Each branch adds a relative phase shift using an independent phase modulator, i.e., $\phi_1$ and $\phi_2$.
The coupler recombines both branches into two outputs.
For our free-space setup, two cubic beam splitters implement the splitter (BS1) and the coupler (BS2).
For additional beam alignment, our free-space setup utilizes two mirrors (M1 and M2).

Finding the transformation of the annihilation operators at each stage of the \gls{mzi} is sufficient to find the quantum input-output relations.
Idealizing the symmetric \gls{mzi}'s passive components as lossless allows relating the annihilation operators by two-dimensional unitary matrices.
We label the input annihilation operators of the \gls{mzi} $\hat{a}_1,\hat{a}_2$ and the output annihilation operators $\hat{a}_1^{\prime\prime\prime},\hat{a}_2^{\prime\prime\prime}$.
The annihilation operators after splitting and before (after) phase shifting are denoted by two (three) primes, i.e., $\hat{a}_1^{\prime},\hat{a}_2^{\prime}$ and $\hat{a}_1^{\prime\prime},\hat{a}_2^{\prime\prime}$.
Going backwards through the transformations from the output to the input annihilation operators
\begin{equation}
	\vb{\hat{a}}^{\prime\prime\prime}
	=
	U_\text{BS2}
	\hat{\vb{a}}^{\prime\prime}
	=
	U_\text{BS2}
	U_\text{PS}
	\hat{\vb{a}}^{\prime}
	=
	U_\text{BS2}
	U_\text{PS}
	U_\text{BS1}
	\hat{\vb{a}}
	=
	U_\text{MZI}
	\vb{\hat{a}}
\end{equation}
we find the symmetric \gls{mzi}'s unitary matrix transform $U_\text{MZM}$ to be equal to the matrix product of the second beam splitter's, the phase shifts', and the first beam splitter's unitary matrix transform  $U_\text{BS2}U_\text{PS}U_\text{BS1}$.

An ideal cubic beam splitter with a single dielectric layer has the unitary matrix transform~\cite[p.~139]{Gerry2005}
\begin{equation}
	U_\text{BS1}
	=
	\frac{1}{\sqrt{2}}
	\begin{pmatrix}
		1 & i \\
		i & 1
	\end{pmatrix}
\end{equation}
where the off-diagonal elements of $U_\text{BS1}$, $i/\sqrt{2}$, account for the phase-shift due to the reflection at the diagonal of the cubic beam splitter.
The matrix encoding the phase shifts from the phase modulation $\phi_1,\phi_2$ and the reflection at the mirrors M1 and M2, $\pi$ is
\begin{equation}
	U_\text{PS}
	=
	\begin{pmatrix}
		ie^{i\phi_1} & 0 \\
		0 & ie^{i\phi_2}
	\end{pmatrix}
\end{equation}
For the second beam splitter, BS2, we again assume an ideal cubic beam splitter with a single dielectric layer.
The corresponding matrix transform is
\begin{equation}
	U_\text{BS2}
	=
	\frac{1}{\sqrt{2}}
	\begin{pmatrix}
		i & 1 \\
		1 & i
	\end{pmatrix}	
\end{equation}
where we exchanged the rows for consistency with the input labels.

Performing the matrix multiplication and writing the exponentials as trigonometric functions, we find the matrix transform of the symmetric \gls{mzi} to be
\begin{equation}
	U_\text{MZI}
	=
	-
	\begin{pmatrix}
		\cos\left(\frac{\phi_2-\phi_1}{2}\right) & \sin\left(\frac{\phi_2-\phi_1}{2}\right) \\
		-\sin\left(\frac{\phi_2-\phi_1}{2}\right) & \cos\left(\frac{\phi_2-\phi_1}{2}\right)
	\end{pmatrix}
	e^{i\frac{\phi_1+\phi_2}{2}}
	.
\end{equation}
It appears useful to define the common-mode and differential-mode phases
\begin{align}
	\phi_+
	&=
	\phi_2
	+
	\phi_1
	&
	\phi_-
	&=
	\phi_2-\phi_1
\end{align}
for which the matrix transform simplifies to
\begin{equation}
	U_\text{MZI}
	=
	-
	\begin{pmatrix}
		\cos(\phi_-/2) & \sin(\phi_-/2) \\
		-\sin(\phi_-/2) & \cos(\phi_-/2)
	\end{pmatrix}
	e^{i\phi_+/2}
\end{equation}
and we note that the common-mode phase $\phi_+$ adds a global phase shift of $\phi_+/2$ while the differential-mode phase $\phi_-$ changes the splitting ratios at the output.

We conclude that the the action of a static \gls{mzm} on a coherent input state $\ket{\alpha}$ is a amplitude and phase modulation, i.e.,
\begin{equation}
	\ket{\alpha}
	\to
	\ket{\alpha^\prime}
	=
	\hat{P}_1
	\hat{U}_\text{MZM}
	\ket{\alpha,0}
	=
	\ket{\alpha\cos(\Theta/2)e^{i\Lambda/2}}
	\label{eq:mzm_transform}
\end{equation}
where $\Theta$ is the differential- and $\Lambda$ is the common-mode phase applied to the Pockels modulator of the \gls{mzm}.
We have effectively multiplied the amplitude of the initial coherent state with a real value $\cos(\Theta/2)$.

Our result holds true for time-dependent modulation.
From the discussion of the Pockels modulator we know that phase modulation can be written as a convolution in momentum space which is a linear operation and therefore compatible with our derivation of the \gls{mzm}.

\subsection{I/Q modulator}
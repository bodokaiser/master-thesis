\section{Mach-Zehnder}


The \gls{mzm} arranges two phase modulators to perform amplitude and phase modulation on an optical input field.
Using two electrical-driven phase modulators, the \gls{mzm} performs amplitude and phase modulation on an optical input field.
To begin with, we derive the quantum operator matrix transform from a specific implementation of the \gls{mzm}, the symmetric free-space \gls{mzi}.
Then, we link the quantum operator matrix transform to more general unitary operators.
Finally, we derive the quantum state transform for a coherent input state to the \gls{mzm}.

\begin{figure}[htb]
	\centering
	\includestandalone{figures/pstricks/mzi-symmetric}
	\caption{Free-space setup of a symmetric \gls{mzm}: The input light mode $\hat{a}_1$ enters a first beam splitter BS1 from the left. A vacuum light mode $\hat{a}_2$ enters BS2 from the top. The transformed mode $\hat{a}_1^\prime$ and $\hat{a}_2^\prime$ exit BS1 to the right and the bottom. A first phase shifter and first mirror M1, right to BS1, add a relative phase of $\varphi_1+\pi$ from mode $\hat{a}_1$ to $\hat{a}_1^{\prime\prime}$. Below BS1, a second mirror M2 directs the light to a right second phase shifter, both adding a relative phase of $\varphi_2+\pi$ from mode $\hat{a}_2^\prime$ to $\hat{a}_2^{\prime\prime}$. A second beam splitter BS2 transforms the input modes $\hat{a}_1^{\prime\prime}$ and $\hat{a}_2^{\prime\prime}$ to the output modes $\hat{a}_1^{\prime\prime\prime}$ and $\hat{a}_2^{\prime\prime\prime}$.}\label{fig:mzi_symmetric}
\end{figure}
\Cref{fig:mzi_symmetric} shows a free-space optics setup of a symmetric \gls{mzi} with one signal input; the other input being in the vacuum state.
The most crucial components of the \gls{mzi} are a splitter, a coupler, and two independent phase modulators.
The splitter divides the input light into two branches.
Each branch adds a relative phase shift using an independent phase modulator, i.e., $\phi_1$ and $\phi_2$.
The coupler recombines both branches into two outputs.
For our free-space setup, two cubic beam splitters implement the splitter (BS1) and the coupler (BS2).
For additional beam alignment, our free-space setup utilizes two mirrors (M1 and M2).

Finding the transformation of the annihilation operators at each stage of the \gls{mzi} is sufficient to find the quantum input-output relations.
Idealizing the symmetric \gls{mzi}'s passive components as lossless allows relating the annihilation operators by two-dimensional unitary matrices.
We label the input annihilation operators of the \gls{mzi} $\hat{a}_1,\hat{a}_2$ and the output annihilation operators $\hat{a}_1^{\prime\prime\prime},\hat{a}_2^{\prime\prime\prime}$.
The annihilation operators after splitting and before (after) phase shifting are denoted by two (three) primes, i.e., $\hat{a}_1^{\prime},\hat{a}_2^{\prime}$ and $\hat{a}_1^{\prime\prime},\hat{a}_2^{\prime\prime}$.
Going backwards through the transformations from the output to the input annihilation operators
\begin{equation}
	\vb{\hat{a}}^{\prime\prime\prime}
	=
	U_\text{BS2}
	\hat{\vb{a}}^{\prime\prime}
	=
	U_\text{BS2}
	U_\text{PS}
	\hat{\vb{a}}^{\prime}
	=
	U_\text{BS2}
	U_\text{PS}
	U_\text{BS1}
	\hat{\vb{a}}
	=
	U_\text{MZI}
	\vb{\hat{a}}
\end{equation}
we find the symmetric \gls{mzi}'s unitary matrix transform $U_\text{MZM}$ to be equal to the matrix product of the second beam splitter's, the phase shifts', and the first beam splitter's unitary matrix transform  $U_\text{BS2}U_\text{PS}U_\text{BS1}$.

An ideal cubic beam splitter with a single dielectric layer has the unitary matrix transform~\cite[p.~139]{Gerry2005}
\begin{equation}
	U_\text{BS1}
	=
	\frac{1}{\sqrt{2}}
	\begin{pmatrix}
		1 & i \\
		i & 1
	\end{pmatrix}
\end{equation}
where the off-diagonal elements of $U_\text{BS1}$, $i/\sqrt{2}$, account for the phase-shift due to the reflection at the diagonal of the cubic beam splitter.
The matrix encoding the phase shifts from the phase modulation $\phi_1,\phi_2$ and the reflection at the mirrors M1 and M2, $\pi$ is
\begin{equation}
	U_\text{PS}
	=
	\begin{pmatrix}
		ie^{i\phi_1} & 0 \\
		0 & ie^{i\phi_2}
	\end{pmatrix}
\end{equation}
For the second beam splitter, BS2, we again assume an ideal cubic beam splitter with a single dielectric layer.
The corresponding matrix transform is
\begin{equation}
	U_\text{BS2}
	=
	\frac{1}{\sqrt{2}}
	\begin{pmatrix}
		i & 1 \\
		1 & i
	\end{pmatrix}	
\end{equation}
where we exchanged the rows for consistency with the input labels.

Performing the matrix multiplication and writing the exponentials as trigonometric functions, we find the matrix transform of the symmetric \gls{mzi} to be
\begin{equation}
	U_\text{MZI}
	=
	-
	\begin{pmatrix}
		\cos\left(\frac{\phi_2-\phi_1}{2}\right) & \sin\left(\frac{\phi_2-\phi_1}{2}\right) \\
		-\sin\left(\frac{\phi_2-\phi_1}{2}\right) & \cos\left(\frac{\phi_2-\phi_1}{2}\right)
	\end{pmatrix}
	e^{i\frac{\phi_1+\phi_2}{2}}
	.
\end{equation}
It appears useful to define the common-mode and differential-mode phases
\begin{align}
	\phi_+
	&=
	\phi_2
	+
	\phi_1
	&
	\phi_-
	&=
	\phi_2-\phi_1
\end{align}
for which the matrix transform simplifies to
\begin{equation}
	U_\text{MZI}
	=
	-
	\begin{pmatrix}
		\cos(\phi_-/2) & \sin(\phi_-/2) \\
		-\sin(\phi_-/2) & \cos(\phi_-/2)
	\end{pmatrix}
	e^{i\phi_+/2}
\end{equation}
and we note that the common-mode phase $\phi_+$ adds a global phase shift of $\phi_+/2$ while the differential-mode phase $\phi_-$ changes the splitting ratios at the output.

For a general lossless \gls{mzm}, we propose the generic unitary matrix transform~\cite[p.~95]{Leonhardt2010}
\begin{equation}
	\begin{split}
		U_\text{MZM}
		&=
		e^{i\Lambda/2}
		\begin{pmatrix}
			\cos(\Theta/2)e^{i(+\Phi+\Psi)/2} & \sin(\Theta/2)e^{i(+\Phi-\Psi)/2} \\
			-\sin(\Theta/2)e^{+(-\Phi+\Psi)/2} & \cos(\Theta/2)e^{i(-\Phi-\Psi)/2}
		\end{pmatrix}
		\\
		&=
		e^{i\Lambda/2}
		\begin{pmatrix}
			e^{+i\Phi/2} & 0 \\
			0 & e^{-i\Phi/2}
		\end{pmatrix}
		\begin{pmatrix}
			\cos(\Theta/2) & \sin(\Theta/2) \\
			-\sin(\Theta/2) & \cos(\Theta/2)
		\end{pmatrix}
		\begin{pmatrix}
			e^{+i\Psi/2} & 0 \\
			0 & e^{-i\Psi/2}
		\end{pmatrix}
	\end{split}
	\label{eq:mzm_matrix}
\end{equation}
wherein the global phase $\Theta$ and the rotation angle $\Lambda$ are time-dependent but $\Phi,\Psi$ are constants.
We obtain the matrix transform of the symmetric free-space $U_\text{MZI}$ after identification of $\Theta$ with the differential-mode phase, $\Lambda+\pi$ with the common-mode phase, and choosing $\Psi=0=\Phi$.
One advantage of the proposed decomposition in \cref{eq:mzm_matrix} is the one-to-one correspondence between the unitary operator~\cite[p.~99]{Leonhardt2010}
\begin{equation}
	\hat{U}_\text{MZM}(\Lambda,\Phi,\Psi,\Theta)
	=
	e^{i\Lambda\hat{L}_t}
	e^{i\Phi\hat{L}_z}
	e^{i\Theta\hat{L}_y}
	e^{i\Psi\hat{L}_z}
	\label{eq:mzm_operator}
\end{equation}
wherein $\hat{L}_j$ denote the Jordan-Schwinger operators~\cite[p.~97]{Leonhardt2010}
\begin{equation}
	\hat{L}_j
	=
	\frac{1}{2}
	\begin{pmatrix}
		\hat{a}_1^\dagger, \hat{a}_2^\dagger
	\end{pmatrix}
	\sigma_j
	\begin{pmatrix}
		\hat{a}_1 \\
		\hat{a}_2
	\end{pmatrix}
	=
	\frac{1}{2}
	\hat{\vb{a}}^\dagger
	\sigma_j
	\hat{\vb{a}}
\end{equation}
with $\sigma_j$ being the two-dimensional Pauli matrices and $\sigma_0$ being the identity matrix, and the unitary matrix transform
\begin{equation}
	U_\text{MZM}
	\hat{\vb{a}}
	=
	\hat{U}_\text{MZM}^\dagger
	\hat{\vb{a}}
	\hat{U}_\text{MZM}
	.
\end{equation}

The \gls{mzm} produces highly entangled output states for (displaced) number states, see, for instance, Ref.~\cite{Windhager2011}.
In the transmitter setting, we are exclusively interested in coherent input states, contrary to the receiver, where an attacker could have sent non-coherent states.

Let us consider the input state
\begin{align}
	\ket{\vb{\alpha}}
	&=
	\ket{\alpha_1,\alpha_2}
	&
	\vb{\alpha}
	=
	\begin{pmatrix}
		\alpha_1,
		\alpha_2
	\end{pmatrix}
	\in
	\mathbb{C}^{2\times1}
\end{align}
to a \gls{mzm}\footnote{We can always set $alpha=0$ to set the second input to vacuum.}, i.e.,
\begin{equation}
	\hat{U}_\text{MZM}
	\ket{\vb{\alpha}}
	=
	\hat{U}_\text{MZM}
	\hat{D}(\vb{\alpha})
	\hat{U}_\text{MZM}^\dagger
	\hat{U}_\text{MZM}
	\ket{0,0}
	=
	\hat{U}_\text{MZM}
	\hat{D}(\vb{\alpha})
	\hat{U}_\text{MZM}^\dagger
	\ket{0,0}
\end{equation}
where we used the invariance of the vacuum state $\ket{0,0}$ for the second equality.
The displacement operator transforms as
\begin{equation}
	\begin{split}
		\hat{U}
		\hat{D}(\vb{\alpha})
		\hat{U}^\dagger
		&=
		\hat{U}
		\exp\left\{
			\vb{\alpha}^\trans
			\hat{\vb{a}}^\dagger
			-
			\hat{\vb{a}}
			\vb{\alpha}^*
		\right\}
		\hat{U}^\dagger
		\\
		&=
		\exp\left\{
			\vb{\alpha}^\trans
			\hat{U}
			\hat{\vb{a}}^\dagger
			\hat{U}^\dagger
			-
			\hat{U}
			\hat{\vb{a}}
			\hat{U}^\dagger
			\vb{\alpha}^*
		\right\}
		\\
		&=
		\exp\left\{
			\vb{\alpha}^\trans
			\left(
				\hat{U}
				\hat{\vb{a}}
				\hat{U}^\dagger
			\right)^\dagger
			-
			\hat{U}
			\hat{\vb{a}}
			\hat{U}^\dagger
			\vb{\alpha}^*
		\right\}
	\end{split}
\end{equation}
where we used the operator identity
\begin{equation}
	\hat{U}
	e^{\hat{A}}
	\hat{U}^\dagger
	=
	\sum_{n=0}^\infty
	\frac{1}{n!}
	\hat{U}
	\hat{A}^n
	\hat{U}^\dagger
	=
	\sum_{n=0}^\infty
	\frac{1}{n!}
	\hat{U}
	\hat{A}
	\hat{U}^\dagger
	\cdots
	\hat{U}
	\hat{A}
	\hat{U}^\dagger
	=
	\sum_{n=0}^\infty
	\frac{1}{n!}
	\left(
		\hat{U}
		\hat{A}
		\hat{U}^\dagger
	\right)^n
	=
	e^{\hat{U}\hat{A}\hat{U}^\dagger}
\end{equation}
which follows from inserting $\hat{U}\hat{U}^\dagger=\mathbb{1}$ between the $\hat{A}$s.
For the unitary operator of the \gls{mzm}, we note that
\begin{equation}
	\begin{split}
		\hat{U}_\text{MZM}(\Lambda,\Phi,\Psi,\Theta)^\dagger
		&=
		e^{-i\Psi\hat{L}_z}
		e^{-i\Theta\hat{L}_y}
		e^{-i\Phi\hat{L}_z}
		e^{-i\Lambda\hat{L}_t}
		\\
		&=
		e^{-i\Lambda\hat{L}_t}
		e^{-i\Psi\hat{L}_z}
		e^{-i\Theta\hat{L}_y}
		e^{-i\Phi\hat{L}_z}
		\\
		&=
		\hat{U}_\text{MZM}(-\Lambda,-\Psi,-\Phi,-\Theta)
	\end{split}
\end{equation}
where we used in the second line that $\hat{L}_t$ commutes with the other Jordan-Schwinger operators $\hat{L}_y,\hat{L}_z$.
The transformed annihilation operators turn out to be
\begin{equation}
	\begin{split}
		\hat{\vb{a}}^\prime
		&=
		\hat{U}_\text{MZM}(\Lambda,\Phi,\Psi,\Theta)
		\hat{\vb{a}}
		\hat{U}_\text{MZM}(\Lambda,\Phi,\Psi,\Theta)^\dagger
		\\
		&=
		\hat{U}_\text{MZM}(-\Lambda,-\Psi,-\Phi,-\Theta)^\dagger
		\hat{\vb{a}}
		\hat{U}_\text{MZM}(-\Lambda,-\Psi,-\Phi,-\Theta)
		\\
		&=
		U_\text{MZM}(-\Lambda,-\Psi,-\Phi,-\Theta)
		\hat{\vb{a}}
		\\
		&=
		U_\text{MZM}(\Lambda,\Phi,\Psi,\Theta)^\dagger
		\hat{\vb{a}}
		.
	\end{split}
\end{equation}
Using the transformed operators, we find the transformed displacement operator to be~\cite[p.~210]{Vogel2006}
\begin{equation}
	\hat{D}^\prime(\vb{\alpha})
	=
	\exp\left\{
		\vb{\alpha}^\trans
		\left(\hat{\vb{a}}^\prime\right)^\dagger
		-
		\hat{\vb{a}}^\prime
		\vb{\alpha}^*
	\right\}
	=
	\exp\left\{
		\left(\vb{\alpha}^\prime\right)^\trans
		\hat{\vb{a}}^\dagger
		-
		\hat{\vb{a}}
		\left(\vb{\alpha}^\prime\right)^*
	\right\}
	.
\end{equation}
We therefore find the \gls{mzm} to transform the coherent input states according to
\begin{align}
	\hat{U}_\text{MZM}
	\ket{\vb{\alpha}}
	&=
	\ket{\vb{\alpha}^\prime}
	&
	\vb{\alpha}^\prime
	&=
	U_\text{MZM}
	\vb{\alpha}
	.
\end{align}
Using the explicit matrix elements in \cref{eq:mzm_matrix} and setting the second input to zero $\alpha_2=0$ and identifying the first as $\alpha_1=\alpha$, we find
\begin{equation}
	\hat{U}_\text{MZM}
	\ket{\alpha,0}
	=
	\ket{\alpha\cos(\Theta/2)e^{+i(\Psi+\Phi)/2},-\alpha\sin(\Theta/2)e^{-i(\Psi-\Phi)/2}}
	.
\end{equation}
We are only interested in the first output mode.
We can remove the second output via partial trace, i.e.,
\begin{equation}
	\trace_2\left\{
		\ketbra{\alpha,\beta}
	\right\}
	=
	\trace_2\left\{
		\ketbra{\alpha}
		\otimes
		\ketbra{\beta}
	\right\}
	=
	\ketbra{\alpha}
	\otimes
	\trace_2\left\{
		\ketbra{\beta}
	\right\}
\end{equation}
where
\begin{equation}
	\trace_2\left\{
		\ketbra{\beta}
	\right\}
	=
	\sum_{n=0}^\infty
	\braket{n}{\beta}
	\braket{\beta}{n}
	=
	\sum_{n=0}^\infty
	\abs{\braket{n}{\beta}}^2
	=
	1
\end{equation}
and thus
\begin{equation}
	\trace_2\left\{
		\ketbra{\alpha,\beta}
	\right\}
	=
	\ketbra{\alpha}
	=
	\hat{P}_1
	\ketbra{\alpha,\beta}
	\hat{P}_1
\end{equation}
i.e., the partial trace over a tensor product of coherent states is equivalent to the projection of the non-traced out states.
This is a special property of the coherent states.
In general, the partial trace over a tensor product of (entangled) quantum states returns a mixed state.

We conclude that the the action of a static \gls{mzm} on a coherent input state $\ket{\alpha}$ is a amplitude and phase modulation, i.e.,
\begin{equation}
	\ket{\alpha}
	\to
	\ket{\alpha^\prime}
	=
	\hat{P}_1
	\hat{U}_\text{MZM}
	\ket{\alpha,0}
	=
	\ket{\alpha\cos(\Theta/2)e^{i\Lambda/2}}
	\label{eq:mzm_transform}
\end{equation}
where $\Theta$ is the differential- and $\Lambda$ is the common-mode phase applied to the Pockels modulator of the \gls{mzm}.
We have effectively multiplied the amplitude of the initial coherent state with a real value $\cos(\Theta/2)$.

Our result holds true for time-dependent modulation.
From the discussion of the Pockels modulator we know that phase modulation can be written as a convolution in momentum space which is a linear operation and therefore compatible with our derivation of the \gls{mzm}.

\chapter{Quantum theory of (electro-)optical components}\label{ch:components}

In the last chapter, we derived a general quantum theory of light for optical communication.
Now it is time to apply it to describe the electro-optical building blocks from which we can assemble a transmission system.
In particular, we present quantum models for the optical coupler, electro-optical modulators, and detectors, which we use to derive input-output relations for coherent states.

First, we introduce the optical coupler as a generalization of the beam splitter and the waveguide coupler.
Second, we apply nonlinear quantum-optics to model electro-optical phase modulation~\cite{Horoshko2018} as nonlinear frequency-conversion~\cite{QuesadaMejia2015} and employ it to perform amplitude modulation through electrically-driven interference.
Third, we briefly review the photoelectric effect~\cite{Mandel1995,Vogel2006} and derive direct and balanced detectors and their interpretation in terms of quantum measurements.

For the first two parts, we aim to motivate an evolution operator from some interaction Hamiltonian. 
At some point, we cannot give an explicit evolution operator but only argue why such an operator may, in principle, exist.
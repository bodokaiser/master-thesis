\chapter{Supplementary quantum theory}

The supplemental material captures auxiliary calculations which would otherwise disrupt the reading.

\section{Quantum theory of light}

\subsection{Commutator between momentum and spacetime annihilation and creation operators}

\begin{lemma}\label{th:annihilation_field_commutators}
	Let $\hat{a}(\vb{k})$ and $\hat{a}^\dagger(\vb{k})$ be the annihilation and creation operator satisfying the \gls{ccr} and $\hat{A}^{(\pm)}$ be the positive and negative frequency field operator, then the commutator of the annihilation and creation operator with the positive and negative smeared field operators yields
	\begin{align}
		\comm{\hat{a}(\vb{p})}{\hat{A}^{(+)}[f]}
		&=
		+\frac{f\left(\omega(\vb{p}),\vb{p}\right)}{\sqrt{2\omega(\vb{p})}}
		\\
		\comm{\hat{a}^\dagger(\vb{p})}{\hat{A}^{(-)}[f]}
		&=
		-\frac{f\left(\omega(\vb{p}),\vb{p}\right)^*}{\sqrt{2\omega(\vb{p})}}
		.
	\end{align}
\end{lemma}
\begin{proof}
	Inserting the smeared field operator in momentum space and using the \gls{ccr} to evaluate the integral yields the first identity
	\begin{equation}
		\comm{\hat{a}(\vb{p})}{\hat{A}^{(+)}[f]}
		=
		\int\frac{\dd[3]{q}}{(2\pi)^3\sqrt{2\omega(\vb{q})}}
		f\left(\omega(\vb{q}),\vb{q}\right)
		\comm{\hat{a}(\vb{p})}{\hat{a}^\dagger(\vb{q})}
		=
		+\frac{f\left(\omega(\vb{p}),\vb{p}\right)}{\sqrt{2\omega(\vb{p})}}
		.
	\end{equation}
	The second identity follows analog
	\begin{equation}
		\comm{\hat{a}^\dagger(\vb{p})}{\hat{A}^{(-)}[f]}
		=
		\int\frac{\dd[3]{q}}{(2\pi)^3\sqrt{2\omega(\vb{q})}}
		f\left(\omega(\vb{q}),\vb{q}\right)^*
		\comm{\hat{a}^\dagger(\vb{p})}{\hat{a}(\vb{q})}
		=
		-\frac{f\left(\omega(\vb{p}),\vb{p}\right)^*}{\sqrt{2\omega(\vb{p})}}
		.
	\end{equation}
\end{proof}

\begin{lemma}\label{th:number_annihilation}
	Let $\hat{a}(\vb{k})$ be the annihilation operator and $\hat{A}^{(-)}$ be the negative frequency field operator, then their commutator yields
	\begin{equation}
		\hat{a}(\vb{p})
		\ket{n_f}
		=
		\sqrt{n}
		\frac{f\left(\omega(\vb{p}),\vb{p}\right)}{\sqrt{2\omega(\vb{p})}}
		\ket{{n-1}_f}
		\label{eq:number_annihilation}
		.
	\end{equation}
\end{lemma}
\begin{proof}
	First, we note the recursive relation
	\begin{equation}
		\hat{a}(\vb{p})
		\ket{n_f}
		=
		\frac{1}{\sqrt{n}}
		\left[
			\frac{f\left(\omega(\vb{p}),\vb{p}\right)}{\sqrt{2\omega(\vb{p})}}
			\ket{{n-1}_f}
			+
			\hat{A}^{(+)}[f]
			\hat{a}(\vb{p})
			\ket{{n-1}_f}
		\right]
		,
	\end{equation}
	where we used the commutator from \cref{th:annihilation_field_commutators}.
	The induction start, $n=0$, follows from the action of the annihilation operator on the vacuum state, \cref{eq:vacuum_annihilation}.
	The induction step, $n\to n+1$, goes
	\begin{equation}
		\begin{split}
			\hat{a}(\vb{p})
			\ket{{n+1}_f}
			&=
			\frac{1}{\sqrt{n+1}}
			\hat{a}(\vb{p})
			\hat{A}^{(+)}[f]
			\ket{n_f}
			\\
			&=
			\frac{1}{\sqrt{n+1}}
			\left[
				\frac{f\left(\omega(\vb{p}),\vb{p}\right)}{\sqrt{2\omega(\vb{p})}}
				+
				\hat{A}^{(+)}[f]
				\hat{a}(\vb{p})
			\right]
			\ket{n_f}
			\\
			&=
			\frac{1}{\sqrt{n+1}}
			\left[
				\frac{f\left(\omega(\vb{p}),\vb{p}\right)}{\sqrt{2\omega(\vb{p})}}
				\ket{n_f}
				+
				\hat{A}^{(+)}[f]
				\sqrt{n}
				\frac{f\left(\omega(\vb{p}),\vb{p}\right)}{\sqrt{2\omega(\vb{p})}}
				\ket{{n-1}_f}
			\right]
			\\
			&=
			\frac{1}{\sqrt{n+1}}
			\frac{f\left(\omega(\vb{p}),\vb{p}\right)}{\sqrt{2\omega(\vb{p})}}
			\left[
				\ket{n_f}
				+
				n
				\ket{n_f}
			\right]
			\\
			&=
			\sqrt{n+1}
			\frac{f\left(\omega(\vb{p}),\vb{p}\right)}{\sqrt{2\omega(\vb{p})}}
			\ket{n_f}
			,
		\end{split}
	\end{equation}
	where we used the recursive relation.
\end{proof}

\subsection{Expectation values of number and coherent states}

\begin{lemma}
	Let $\ket{n_f}$ be a number state and $\vu{P}$ be the momentum operator, \cref{eq:maxwell_momentum_operator}, then number state has the expectation value
	\begin{equation}
		\bra{n_f}
		\vu{P}
		\ket{n_f}
		=
		n
		\int\frac{\dd[3]{p}}{(2\pi)^3}
		\vb{p}
		\abs*{\frac{f\left(\omega(\vb{q}),\vb{q}\right)}{\sqrt{2\omega(\vb{p})}}}^2
	\end{equation}
	of the momentum operator.
\end{lemma}
\begin{proof}
	We insert the definition of the momentum operator and use the action of the annihilation operator on the number state, \cref{th:number_annihilation},
	\begin{equation}
		\begin{split}
			\bra{n_f}
			\vu{P}
			\ket{n_f}
			&=
			\int\frac{\dd[3]{p}}{(2\pi)^3}
			\vb{p}
			\bra{n_f}
			\hat{a}^\dagger(\vb{p})
			\hat{a}(\vb{p})
			\ket{n_f}
			\\
			&=
			n
			\int\frac{\dd[3]{p}}{(2\pi)^3}
			\vb{p}
			\abs*{\frac{f\left(\omega(\vb{q}),\vb{q}\right)}{\sqrt{2\omega(\vb{p})}}}^2
			\braket{{n-1}_f}{{n-1}_f}
		\end{split}
	\end{equation}
	and we used took the hermitian conjugate of \cref{eq:number_annihilation}.
\end{proof}

\begin{lemma}\label{th:electric_operator_mixed_frequency}
	Let $\hat{E}^{(-)}(t,\vb{x})$ be the negative frequency and $\hat{E}^{(+)}(t,\vb{x})$ be the positive frequency part of the electric field operator, then
	\begin{equation}
		\hat{E}^{(-)}(t,\vb{x})
		\hat{E}^{(+)}(t,\vb{x})
		=
		\frac{1}{2}
		\int\frac{\dd[3]{p}}{(2\pi)^3}
		\omega(\vb{p})
		+
		\hat{E}^{(+)}(-t,-\vb{x})
		\hat{E}^{(-)}(-t,-\vb{x})
		.
	\end{equation}
\end{lemma}
\begin{proof}
	The identity follows from the definitions of the electric and positive frequency electric field operator, \cref{eq:electric_positive_operator,eq:electric_negative_operator},
	and using the \gls{ccr} of the annihilation and creation operator, i.e.,
	\begin{equation}
		\begin{split}
			\hat{E}^{(-)}(t,\vb{x})
			\hat{E}^{(+)}(t,\vb{x})
			&=
			\int\frac{\dd[3]{p}}{(2\pi)^3\sqrt{2\omega(\vb{p})}}
			\omega(\vb{p})
			e^{+i\omega(\vb{p})t-i\vb{p}\vdot\vb{x}}
			\\
			&\times
			\int\frac{\dd[3]{q}}{(2\pi)^3\sqrt{2\omega(\vb{q})}}
			\omega(\vb{q})
			e^{-i\omega(\vb{q})t+i\vb{q}\vdot\vb{x}}
			\\
			&\times
			\left\{
				\comm{\hat{a}(\vb{p})}{\hat{a}^\dagger(\vb{q})}
				+
				\hat{a}^\dagger(\vb{q})
				\hat{a}(\vb{p})
			\right\}
			\\
			&=
			\frac{1}{2}
			\int\frac{\dd[3]{p}}{(2\pi)^3}
			\omega(\vb{p})
			+
			\hat{E}^{(+)}(-t,-\vb{x})
			\hat{E}^{(-)}(-t,-\vb{x})
			.
		\end{split}
	\end{equation}
\end{proof}

\begin{theorem}
	Let $\ket{n_f}$ be a number state and $\hat{E}(t,\vb{x})$ be the electric field operator,
	then we have the expectation values
	\begin{align}
		\bra{n_f}
		\hat{E}(t,\vb{x})
		\ket{n_f}
		&=
		0
		\\
		\bra{n_f}
		\left(
			\Delta
			\hat{E}(t,\vb{x})
		\right)^2
		\ket{n_f}
		&=
		\frac{1}{2}
		\int\frac{\dd[3]{p}}{(2\pi)^3}
		\omega(\vb{p})
		+
		\frac{1}{2}
		\abs*{\Psi(t,\vb{x})}^2
	\end{align}
\end{theorem}
\begin{proof}
	The expectation values of an unequal number of annihilation and creation operators is always zero.
	Expanding the electric field operator into positive and negative frequency parts
	\begin{equation}
		\bra{n_f}
		\hat{E}(t,\vb{x})
		\ket{n_f}
		=
		\bra{n_f}
		\hat{E}^{(-)}(t,\vb{x})
		\ket{n_f}
		+
		\bra{n_f}
		\hat{E}^{(+)}(t,\vb{x})
		\ket{n_f}
		,
	\end{equation}
	we note that the first term comprises $n+1$ annihilation and $n$ creation operators, and the second term comprises $n$ annihilation and $n+1$ creation operators, i.e., an unequal number of annihilation and creation operators, and we conclude that the expectation value of the electric field operator given a number state is zero.
	The variance is equal to the second moment as the first moment is zero
	\begin{equation}
		\bra{n_f}
		\left(
			\Delta
			\hat{E}(t,\vb{x})
		\right)^2
		\ket{n_f}
		=
		\bra{n_f}
		\hat{E}(t,\vb{x})^2
		\ket{n_f}
		=
		\bra{n_f}
		\hat{E}^{(+)}(t,\vb{x})
		\hat{E}^{(-)}(t,\vb{x})
		\ket{n_f}
		+
		\text{h.c.}
	\end{equation}
	where we expanded the square of the electric field operator in its positive and negative frequency parts and used that only mixed terms survive in the second equation.
	Using \cref{th:number_annihilation}, we find
	\begin{equation}
		\begin{split}
			\bra{n_f}
			\hat{E}^{(+)}(t,\vb{x})
			\hat{E}^{(-)}(t,\vb{x})
			\ket{n_f}
			&=
			\int\frac{\dd[3]{p}}{(2\pi)^3\sqrt{2\omega(\vb{p})}}
			\omega(\vb{p})
			e^{+i\omega(\vb{p})t-i\vb{p}\vdot\vb{x}}
			\\
			&\times
			\int\frac{\dd[3]{q}}{(2\pi)^3\sqrt{2\omega(\vb{q})}}
			\omega(\vb{q})
			e^{-i\omega(\vb{q})t+i\vb{q}\vdot\vb{x}}
			\\
			&\times
			\bra{n_f}
			\hat{a}^\dagger(\vb{p})
			\hat{a}(\vb{q})
			\ket{n_f}
			\\
			&=
			\abs*{
				\frac{1}{2}
				\int\frac{\dd[3]{p}}{(2\pi)^3}
				f\left(\omega(\vb{p}),\vb{p}\right)
				e^{+i\omega(\vb{p})t-i\vb{p}\vdot\vb{x}}
			}^2
			.
		\end{split}
	\end{equation}
	The second mixed term can be written in terms of the first by using \cref{th:electric_operator_mixed_frequency}
	\begin{equation}
		\begin{split}
			\bra{n_f}
			\hat{E}^{(-)}(t,\vb{x})
			\hat{E}^{(+)}(t,\vb{x})
			\ket{n_f}
			&=
			\frac{1}{2}
			\int\frac{\dd[3]{p}}{(2\pi)^3}
			\omega(\vb{p})
			+
			\bra{n_f}
			\hat{E}^{(+)}(-t,-\vb{x})
			\hat{E}^{(-)}(-t,-\vb{x})
			\ket{n_f}
			\\
			&=
			\frac{1}{2}
			\int\frac{\dd[3]{p}}{(2\pi)^3}
			\omega(\vb{p})
			+
			\abs*{
				\frac{1}{2}
				\int\frac{\dd[3]{p}}{(2\pi)^3}
				f\left(\omega(\vb{p}),\vb{p}\right)
				e^{-i\omega(\vb{p})t+i\vb{p}\vdot\vb{x}}
			}^2
			.
		\end{split}
	\end{equation}
	Adding both terms together and identifying the restricted Fourier transform of the momentum spectrum as the wave function, $\Psi(t,\vb{x})$, we find the electric field variance to be
	\begin{equation}
		\bra{n_f}
		\left(
			\Delta
			\hat{E}(t,\vb{x})
		\right)^2
		\ket{n_f}
		=
		\frac{1}{2}
		\int\frac{\dd[3]{p}}{(2\pi)^3}
		\omega(\vb{p})
		+
		\frac{1}{4}
		\abs*{\Psi(t,\vb{x})}^2
		+
		\frac{1}{4}
		\abs*{\Psi(-t,-\vb{x})}^2
	\end{equation}
	where the first term denotes the "vacuum fluctuations".
\end{proof}

\begin{theorem}
	Let $\ket{\alpha}$ be a coherent state and $\hat{E}(t,\vb{x})$ be the electric field operator, then its expectation value is
	\begin{equation}
		\bra{\alpha}
		\hat{E}(t,\vb{x})
		\ket{\alpha}
		=
		\frac{i}{2}
		\int\frac{\dd[3]{p}}{(2\pi)^3}
		\left\{
			\alpha(\vb{p})
			e^{+i\omega(\vb{p})t-i\vb{p}\vdot\vb{x}}
			-
			\alpha(\vb{p})^*
			e^{-i\omega(\vb{p})t+i\vb{p}\vdot\vb{x}}
		\right\}
	\end{equation}
\end{theorem}
\begin{proof}
	The expectation value follows from using the the definition of the electric field operator, \cref{eq:electric_operator}, and using that the coherent state is an eigenstate of the annihilation operator, \cref{eq:coherent_state_annihilation}.
	Expanding the squared electric field operator into its positive and negative frequency parts
	\begin{equation}
		\begin{split}
			\hat{E}(t,\vb{x})^2
			&=
			\hat{E}^{(+)}(t,\vb{x})^2
			+
			\hat{E}^{(+)}(t,\vb{x})
			\hat{E}^{(-)}(t,\vb{x})
			+
			\hat{E}^{(-)}(t,\vb{x})
			\hat{E}^{(+)}(t,\vb{x})
			+
			\hat{E}^{(-)}(t,\vb{x})^2
			\\
			&=
			\frac{1}{2}
			\int\frac{\dd[3]{p}}{(2\pi)^3}
			\omega(\vb{p})
			+
			\hat{E}^{(+)}(t,\vb{x})^2
			+
			\hat{E}^{(-)}(t,\vb{x})^2
			\\
			&+
			\hat{E}^{(+)}(t,\vb{x})
			\hat{E}^{(-)}(t,\vb{x})
			+
			\hat{E}^{(+)}(-t,-\vb{x})
			\hat{E}^{(-)}(-t,-\vb{x})
		\end{split}
	\end{equation}
\end{proof}

\section{Macroscopic electrodynamics}

\subsection{Polarization density and electric susceptibility tensor}\label{sec:polarization_density_susceptibility}

In the time domain, the first-order polarization density is~\cite[p.~17]{Murti2014}
\begin{equation}
	P_i^{(1)}(t)
	=
	\int\dd{t_1}
	\chi^{(1)}_{ij}(t_1)
	E^j(t-t_1)
	\label{eq:polarization_density_fo_time}
\end{equation}
wherein $\chi^{(1)}_{ijk}(t_1)$ is the first-order time response tensor of the dielectric media.
Inserting the Fourier transform of the electric field components, we find
\begin{equation}
	P_i^{(1)}(t)
	=
	\int\frac{\dd{\omega_1}}{2\pi}
	\chi^{(1)}_{ij}(\omega_1)
	E^j(\omega_1)
	e^{-i\omega_1t}
	\label{eq:polarization_density_fo_time_fourier}
	,
\end{equation}
wherein the second-order frequency response tensor is the Fourier transform of the time response tensor, i.e.,
\begin{equation}
	\chi^{(1)}_{ij}(\omega_1)
	=
	\int\dd{t_1}
	\chi^{(1)}_{ij}(t_1)
	e^{+i\omega_1t}
	.
\end{equation}
From \cref{eq:polarization_density_fo_time_fourier} we can directly read of the first-order polarization density in the frequency domain
\begin{equation}
	P_i^{(1)}(\omega)
	=
	\chi^{(1)}_{ij}(\omega)
	E^j(\omega)
	.
	\label{eq:polarization_density_fo_freq}
\end{equation}

In the time domain, the second-order polarization density is~\cite[p.~55]{Boyd2020}
\begin{equation}
	P_i^{(2)}(t)
	=
	\int\dd{t_1}
	\int\dd{t_2}
	\chi^{(2)}_{ijk}(t_1,t_2)
	E^j(t-t_1)
	E^k(t-t_2)
	\label{eq:polarization_density_so_time}
\end{equation}
wherein $\chi^{(2)}_{ijk}(t_1,t_2)$ is the second-order time response tensor of the dielectric media.
Inserting the Fourier transform of the electric field components, we find
\begin{equation}
	P_i^{(2)}(t)
	=
	\int\frac{\dd{\omega_1}}{2\pi}
	\int\frac{\dd{\omega_2}}{2\pi}
	\chi^{(2)}_{ijk}(\omega_1,\omega_2)
	E^j(\omega_1)
	E^k(\omega_2)
	e^{-i(\omega_1+\omega_2)t}
	\label{eq:polarization_density_so_time_fourier}
\end{equation}
wherein the second-order frequency response tensor is the Fourier transform of the time response tensor, i.e.,
\begin{equation}
	\chi^{(2)}_{ijk}(\omega_1,\omega_2)
	=
	\int\dd{t_1}
	\int\dd{t_2}
	\chi^{(2)}_{ijk}(t_1,t_2)
	e^{+i\omega_1t}
	e^{+i\omega_2t}
	.
\end{equation}
Contrary to \cref{eq:polarization_density_fo_time_fourier}, we cannot directly read of the polarization density in the frequency domain from \cref{eq:polarization_density_so_time_fourier}.
Instead, we insert \cref{eq:polarization_density_so_time_fourier} into the Fourier transform of the second-order polarization density
\begin{equation}
	\begin{split}
		P_i^{(2)}(\omega)
		&=
		\int\dd{t}
		P_i^{(2)}(t)
		e^{+i\omega t}
		\\
		&=
		\int\frac{\dd{\omega_1}}{2\pi}
		\int\frac{\dd{\omega_2}}{2\pi}
		\chi^{(2)}_{ijk}(\omega_1,\omega_2)
		E^j(\omega_1)
		E^k(\omega_2)
		\int\dd{t}
		e^{-i(\omega_1+\omega_2-\omega)t}
		\\
		&=
		\int\frac{\dd{\omega_1}}{2\pi}
		\int\frac{\dd{\omega_2}}{2\pi}
		\chi^{(2)}_{ijk}(\omega_1,\omega_2)
		E^j(\omega_1)
		E^k(\omega_2)
		(2\pi)
		\delta^{(1)}(\omega_2-\omega+\omega_1)
		\\
		&=
		\int\frac{\dd{\omega^\prime}}{2\pi}
		\chi^{(2)}_{ijk}(\omega^\prime,\omega-\omega^\prime)
		E^j(\omega^\prime)
		E^k(\omega-\omega^\prime)
		.
	\end{split}
	\label{eq:polarization_density_so_freq}
\end{equation}

The presence of the integral in \cref{eq:polarization_density_so_freq} breaks the linearity of the second-order polarization density in the electric field $E^k$.
To restore linearity, we use the mean value theorem for integrals at the radio center frequency $\Omega_0$,
\begin{equation}
	\begin{split}
		P_i^{(2)}(\omega)
		&\approx
		\chi^{(2)}_{ijk}(0,\omega)
		E^j(0)
		E^k(\omega)
		\\
		&+
		\chi^{(2)}_{ijk}(-\Omega_0,\omega+\Omega_0)
		E^j(-\Omega_0)
		E^k(\omega)
		\\
		&+
		\chi^{(2)}_{ijk}(+\Omega_0,\omega-\Omega_0)
		E^j(+\Omega_0)
		E^k(\omega)
		,
	\end{split}
\end{equation}
and find one zero-frequency component and two sidebands at $\pm\Omega_0$.
Usually, the second-order polarization density of the Pockels effect is only given with the zero-frequency component, corresponding to a static (time-independent) electric field~\cite[p.~495]{Boyd2020}.

\subsection{Refractive index of static Pockels effect}\label{sec:static_pockels}

The second-order polarization density of the static Pockels effect is~\cite[p.~495]{Boyd2020}
\begin{equation}
	P_i^{(2)}(\omega)
	=
	\chi^{(2)}_{ijk}(0,\omega)
	E^j(0)
	E^k(\omega)
	.
	\label{eq:polarization_density_so_freq_pockels}
\end{equation}
The electric displacement field in the frequency domain with polarization density expanded to second-order is~\cite[p.~1070]{Mandel1995}
\begin{equation}
	D_i(\omega)
	=
	E_i(\omega)
	+
	P_i^{(1)}(\omega)
	+
	P_i^{(2)}(\omega)
	.
	\label{eq:electric_displacement_field_expanded}
\end{equation}
Inserting \cref{eq:polarization_density_fo_freq} and \cref{eq:polarization_density_so_freq_pockels} into \cref{eq:electric_displacement_field_expanded}, we find
\begin{equation}
	D_i(\omega)
	=
	E_i(\omega)
	+
	\chi^{(1)}_{ij}(\omega)
	E^j(\omega)
	+
	\chi^{(2)}_{ijk}(0,\omega)
	E^j(0)
	E^k(\omega)
	.
\end{equation}
Rewriting the expansion of the electric displacement field as
\begin{equation}
	D_i(\omega)
	=
	\left[
		\delta_{ik}
		+
		\chi^{(1)}_{ik}(\omega)
		+
		\chi^{(2)}_{ijk}(0,\omega)
		E^j(0)
	\right]
	E^k(\omega)
	,
\end{equation}
we can read off the electric susceptibility tensor
\begin{equation}
	\varepsilon_{ik}(\omega)
	=
	\delta_{ik}
	+
	\chi^{(1)}_{ik}(\omega)
	+
	\chi^{(2)}_{ijk}(0,\omega)
	E^j(0)
	.
\end{equation}
The electric susceptibility tensor is equal to the squared refractive index tensor, $\varepsilon_{ik}(\omega)=n_{ik}(\omega)^2$, or equivalently~\cite[p.~3]{Brooker2003}
\begin{equation}
	n_{ij}(\omega)
	=
	\sqrt{\varepsilon_{ij}}
	\approx
	n_{ik}^{(0)}
	+
	n_{ijk}^{(1)}(\omega)
	E^j(0)
\end{equation}
wherein the refractive index coefficients relate to the dielectric susceptibility tensor via~\cite{Rerat2020}
\begin{align}
	n^{(0)}_{ij}(\omega)
	&=
	\sqrt{1+\chi^{(1)}_{ij}(\omega)}
	&
	n^{(1)}_{ijk}(\omega)
	&=
	\frac{\chi^{(2)}_{ijk}(\omega)
	E^k(0)}{2\sqrt{1+\chi^{(1)}_{ij}(\omega)}}
	.
\end{align}

\subsection{Nonlinear interaction Hamiltonians}

The Hamiltonian of the electromagnetic field in a dielectric medium is symbolically~\cite[p.~124]{Jackson2007}
\begin{equation}
	\hat{H}
	=
	\frac{1}{2}
	\int\dd[3]{x}
	\left\{
		\hat{E}^i(t,\vb{x})
		\hat{D}_i(t,\vb{x})
		+
		\hat{B}^i(t,\vb{x})
		\hat{B}_i(t,\vb{x})
	\right\}
	.
\end{equation}
We expand the electric displacement field up to the second-order polarization density
\begin{equation}
	\hat{D}_i(t,\vb{x})
	=
	\hat{E}_i(t,\vb{x})
	+
	\hat{P}_i^{(1)}(t,\vb{x})
	+
	\hat{P}_i^{(2)}(t,\vb{x})
\end{equation}
and insert it back into the Hamiltonian of the electromagnetic field
\begin{equation}
	\hat{H}
	=
	\frac{1}{2}
	\int\dd[3]{x}
	\left\{
		\hat{E}^i(t,\vb{x})
		\hat{E}_i(t,\vb{x})
		+
		\hat{B}^i(t,\vb{x})
		\hat{B}_i(t,\vb{x})
	\right\}
	+
	\hat{H}_\text{int}^{(1)}(t)
	+
	\hat{H}_\text{int}^{(2)}(t)
\end{equation}
and identify the free field Hamiltonian and two interaction terms with
\begin{align}
	\hat{H}_\text{int}^{(n)}(t)
	=	
	\frac{1}{2}
	\int\dd[3]{x}
	\hat{E}^i(t,\vb{x})
	\hat{P}_i^{(n)}(t,\vb{x})
	.
\end{align}

Assuming the dielectric to be absorption-free and time-invariant, we can write the first-order polarization density as
\begin{equation}
	\hat{P}_i^{(1)}(t,\vb{x})
	=
	\int\dd{t_1}
	\chi_{ij}^{(1)}(t_1,\vb{x})
	\hat{E}_i^{(i)}(t-t_1,\vb{x})
\end{equation}
and insert it back into the interaction Hamiltonian
\begin{equation}
	\hat{H}_\text{int}^{(1)}(t)
	=	
	\frac{1}{2}
	\int\dd{t_1}
	\chi_{ij}^{(1)}(t_1,\vb{x})
	\int\dd[3]{x}
	\hat{E}^i(t,\vb{x})
	\hat{E}^j(t-t_1,\vb{x})
	.
\end{equation}
We assume the electric fields to be polarized along the same axis and neglect the transverse mode profile
\begin{equation}
	\hat{H}_\text{int}^{(1)}(t)
	\approx
	\frac{1}{2}
	\int\dd{t_1}
	\chi^{(1)}(t_1,x)
	\int\dd{x}
	\hat{E}(t,x)
	\hat{E}(t-t_1,x)
	+
	\text{h.c.}
	.
\end{equation}
We expand the electric field into positive and negative frequency parts and perform the rotating-wave approximation
\begin{equation}
	\hat{H}_\text{int}^{(1)}(t)
	\approx
	\frac{1}{2}
	\int\dd{t_1}
	\chi^{(1)}(t_1,x)
	\int\dd{x}
	\hat{E}^{(-)}(t,x)
	\hat{E}^{(+)}(t-t_1,x)
	+
	\text{h.c.}
\end{equation}
by only keeping mixed frequency parts.
Inserting the mode expansion, \cref{eq:electric_positive_operator,eq:electric_positive_operator}, we find
\begin{equation}
	\begin{split}
		\hat{H}_\text{int}^{(1)}(t)
		&=
		\int\frac{\dd{p}}{2\pi\sqrt{2\omega(p)}}
		\omega(p)
		\hat{a}(p)
		e^{-i\omega(p)t}
		\int\frac{\dd{q}}{2\pi\sqrt{2\omega(q)}}
		\omega(q)
		\hat{a}^\dagger(q)
		e^{+i\omega(p)(t-t_1)}
		\\
		&\times
		\frac{1}{2}
		\int\dd{t_1}
		\int\dd{x}
		\chi^{(1)}(t_1,x)
		e^{i(p-q)x}
		+
		\text{h.c.}
		.
	\end{split}
\end{equation}
Furthermore, assuming the interaction strength to be uniform inside the dielectric, we can evaluate the first integral
\begin{equation}
	\int\dd{x}
	\chi^{(1)}(t_1,x)
	e^{i(p-q)x}
	=
	\chi^{(1)}(t_1)
	\int_{-L/2}^{+L/2}\dd{x}
	e^{i(p-q)x}
	=
	\chi^{(1)}(t_1)
	\sinc\left(\frac{p-q}{2/L}\right)
	L
\end{equation}
which is known as the phase-matching function in non-linear optics~\cite[p.~33]{QuesadaMejia2015}.
We assume perfect phase-matching
\begin{equation}
	\chi^{(1)}(t_1)
	\sinc\left(\frac{p-q}{2/L}\right)
	L
	\approx
	\chi^{(1)}(t_1)L
	(2\pi)
	\delta^{(1)}(p-q)
\end{equation}
and insert the result into the interaction Hamiltonian
\begin{equation}
	\begin{split}
		\hat{H}_\text{int}^{(1)}(t)
		&=
		\frac{L}{2}
		\int\frac{\dd{p}}{2\pi}
		\omega(p)
		\hat{a}(p)
		\hat{a}^\dagger(p)
		\int\dd{t_1}
		\chi^{(1)}(t_1)
		e^{-ipt_1}
		+
		\text{h.c.}
		\\
		&=
		\frac{L}{2}
		\int\frac{\dd{p}}{2\pi}
		\chi^{(1)}(p)
		\omega(p)
		\hat{a}(p)
		\hat{a}^\dagger(p)
		+
		\text{h.c.}
	\end{split}
\end{equation}
which can be solved exactly with the Magnus expansion.
The interaction Hamiltonian corresponds to a constant energy shift due to the polarization inside the dielectric.

In general, we follow the same approach as for the first-order susceptibility.
However, now we have to deal with nonlinearities.

The second-order polarization density operator for a lossless and time-invariant dielectric is
\begin{equation}
	\hat{P}_i^{(2)}(t,\vb{x})
	=
	\int\dd{t_1}
	\int\dd{t_2}
	\chi^{(2)}_{ijk}(t_1,t_2,\vb{x})
	\hat{E}^j(t-t_1,\vb{x})
	\hat{E}^k(t-t_2,\vb{x})
\end{equation}
which inserted into the second-order interaction Hamiltonian yields
\begin{equation}
	\hat{H}_\text{int}^{(2)}(t)
	=
	\frac{1}{2}
	\int\dd[3]{x}
	\int\dd{t_1}
	\int\dd{t_2}
	\chi^{(2)}_{ijk}(t_1,t_2,\vb{x})
	\hat{E}^i(t,\vb{x})
	\hat{E}^j(t-t_1,\vb{x})
	\hat{E}^k(t-t_2,\vb{x})
	.
\end{equation}
We assume the electric fields to be polarized along the same axis
\begin{equation}
	\hat{H}_\text{int}^{(2)}(t)
	\approx
	\frac{1}{2}
	\int\dd[3]{x}
	\int\dd{t_1}
	\int\dd{t_2}
	\chi^{(2)}(t_1,t_2,\vb{x})
	\hat{E}(t,\vb{x})
	\hat{E}(t-t_1,\vb{x})
	\hat{E}(t-t_2,\vb{x})
	.
\end{equation}
Furthermore, we expand the electric fields into positive and negative frequency parts and only keep the terms corresponding to frequency conversion
\begin{equation}
	\begin{split}
		\hat{H}_\text{int}^\text{FC}(t)
		&\approx
		\frac{1}{2}
		\int\dd[3]{x}
		\int\dd{t_1}
		\int\dd{t_2}
		\chi^{(2)}(t_1,t_2,\vb{x})
		\\
		&\times
		\hat{E}^{(+)}(t,\vb{x})
		\hat{E}^{(-)}(t-t_1,\vb{x})
		\hat{E}^{(-)}(t-t_2,\vb{x})
		+
		\text{h.c.}
	\end{split}
\end{equation}
dropping all other terms.
Performing the mode expansion, \cref{eq:electric_positive_operator,eq:electric_negative_operator},
\begin{equation}
	\begin{split}
		\hat{H}_\text{int}^\text{FC}(t)
		&=
		\frac{1}{2}
		\int\dd[3]{x}
		\int\dd{t_1}
		\int\dd{t_2}
		\chi^{(2)}(t_1,t_2,\vb{x})
		\\
		&\times
		\int\frac{\dd[3]{p}}{(2\pi)^3\sqrt{2\omega(\vb{p})}}
		\omega(\vb{p})
		\hat{a}^\dagger(\vb{p})
		e^{+i\omega(\vb{p})t-i\vb{p}\vdot\vb{x}}
		\\
		&\times
		\int\frac{\dd[3]{q_1}}{(2\pi)^3\sqrt{2\omega(\vb{q}_1)}}
		\omega(\vb{q}_1)
		\hat{a}(\vb{q}_1)
		e^{-i\omega(\vb{q}_1)(t-t_1)+i\vb{q}_1\vdot\vb{x}}
		\\
		&\times
		\int\frac{\dd[3]{q_2}}{(2\pi)^3\sqrt{2\omega(\vb{q}_2)}}
		\omega(\vb{q}_2)
		\hat{a}(\vb{q}_2)
		e^{-i\omega(\vb{q}_2)(t-t_2)+i\vb{q}_2\vdot\vb{x}}
		+
		\text{h.c.}
	\end{split}
\end{equation}
We neglect the transverse dynamics and reduce the Hamiltonian to one dimension along the propagation direction and assume the susceptibility to be uniform along the dielectric medium
\begin{equation}
	\begin{split}
		\hat{H}_\text{int}^\text{FC}(t)
		&=
		\frac{1}{4}
		\int\frac{\dd{p}}{2\pi}
		\int\frac{\dd{q_1}}{2\pi}
		\int\frac{\dd{q_2}}{2\pi}
		\sqrt{\frac{pq_1q_2}{2}}
		\int\dd{t_1}
		\int\dd{t_2}
		\chi^{(2)}(t_1,t_2)
		e^{+iq_1t_1}
		e^{+iq_2t_2}
		\\
		&\times
		\int_{-L/2}^{+L/2}\dd{x}
		e^{i(p-q_1-q_2)x}
		\hat{a}^\dagger(p)
		\hat{a}(q_1)
		\hat{a}(q_2)
		e^{+ipt}
		e^{-iq_1t}
		e^{-iq_2t}
		+
		\text{h.c.}
	\end{split}
\end{equation}
and we identify the Fourier transform of the susceptibility
\begin{equation}
	\begin{split}
		\hat{H}_\text{int}^\text{FC}(t)
		&=
		\frac{1}{4}
		\int\frac{\dd{p}}{2\pi}
		\int\frac{\dd{q_1}}{2\pi}
		\int\frac{\dd{q_2}}{2\pi}
		\sqrt{\frac{pq_1q_2}{2}}
		\chi^{(2)}(q_1,q_2)L
		\sinc\left(\frac{p-q_1-q_2}{L/2}\right)
		\\
		&\times
		\hat{a}^\dagger(p)
		\hat{a}(q_1)
		\hat{a}(q_2)
		e^{+ipt}
		e^{-iq_1t}
		e^{-iq_2t}
		+
		\text{h.c.}
	\end{split}
\end{equation}
We approximate the phase-matching function
\begin{equation}
	\sinc\left(\frac{p-q_1-q_2}{L/2}\right)
	\approx
	(2\pi)
	\delta^{(1)}(p-q_1-q_2)
\end{equation}
and simplify the interaction energy to
\begin{equation}
	\hat{H}_\text{int}^\text{FC}(t)
	=
	\int\frac{\dd{p}}{2\pi}
	\int\frac{\dd{q}}{2\pi}
	g(p,q)
	\hat{a}^\dagger(p)
	\hat{a}(q)
	\hat{a}(p-q)
	+
	\text{h.c.}
	.
\end{equation}
which agrees from the main characteristics with the result reported in Ref.~\cite[eq.~35]{Horoshko2018}.

\section{Photodetection theory}

\subsection{Single photoelectron excitation probability}

The transition of a bound electron to a free photoelectron is a probabilistic process.
Let $\ket{i}$ and $\ket{f}$ denote the initial and final light states, and let $\ket{g}$ and $\ket{e}$ be the electron ground and excited states.
The probability for the transition, $\ket{g,i}\to\ket{e,f}$, from time $t$ to $t+\Delta t$ is equal to
\begin{equation}
	\abs{
		\bra{e,f}
		\hat{U}_\text{int}(t,t+\Delta t)
		\ket{g,i}
	}^2
	\label{eq:photoelectric_transition_prob}
	,
\end{equation}
wherein $\hat{U}_\text{int}$ is the time-evolution operator of the photo-atom interaction in the dipole approximation~\cite[p.~689]{Mandel1995},
\begin{align}
	\hat{U}_\text{int}(t,t+\Delta t)
	&=
	\mathcal{T}_+
	\exp\left\{
		-i
		\int_t^{t+\Delta t}
		\dd{t^\prime}
		\hat{H}_\text{int}(t^\prime)
	\right\}
	&
	\hat{H}_\text{int}(t)
	&=
	-
	\hat{\vb{p}}(t)
	\vdot
	\hat{\vb{A}}(t)
	\label{eq:photoelectric_time_evolution_operator}
\end{align}
with $\vu{p}$ being the electron's momentum operator, $\vu{A}(t)=\vu{A}(t,\vb{x}_0)$ being the Maxwell field in the Coulomb gauge approximated at the atom \gls{com}, and $\mathcal{T}_+$ denoting (forward) time-ordering.
In the more general density operator formailism \cref{eq:photoelectric_transition_prob} reads~\cite[p.~686]{Mandel1995}
\begin{equation}
	\begin{split}
		\abs{
			\bra{e,f}
			\hat{U}_\text{int}(t,t+\Delta t)
			\ket{g,i}
		}^2
		&=
		\bra{e,f}
		\hat{U}_\text{int}(t,t+\Delta t)
		\ketbra{g,i}
		\hat{U}_\text{int}(t,t+\Delta t)^\dagger
		\ket{e,f}
		\\
		&=
		\trace\biggl\{
			\bra{e,f}
			\hat{U}_\text{int}(t,t+\Delta t)
			\ketbra{g,i}
			\hat{U}_\text{int}(t,t+\Delta t)^\dagger
			\ket{e,f}
		\biggr\}
		\\
		&=
		\trace\biggl\{
			\ketbra{e,f}
			\hat{U}_\text{int}(t,t+\Delta t)
			\ketbra{g,i}
			\hat{U}_\text{int}(t,t+\Delta t)^\dagger
		\biggr\}
		\\
		&=
		\trace\biggl\{
			\hat\varrho_{e,f}
			\hat{U}_\text{int}(t,t+\Delta t)
			\hat\rho(t)
			\hat{U}_\text{int}(t,t+\Delta t)^\dagger
		\biggr\}
		\\
		&=
		\trace\biggl\{
			\hat\varrho_{e,f}
			\hat\rho(t+\Delta t)
		\biggr\}
		,
	\end{split}
	\label{eq:photoelectric_transition_prob_density}
\end{equation}
wherein we used that the trace of a scalar is the scalar in the second line and the cyclic property of the trace in the third line.
Performing the Magnus expansion of the time-evolution operator, \cref{eq:photoelectric_time_evolution_operator}, up to the first term,
\begin{equation}
	\hat{U}_\text{int}(t,t+\Delta t)
	\approx
	\exp\left\{
		-i
		\int_t^{t+\Delta t}\dd{t^\prime}
		\hat{H}_\text{int}(t^\prime)
	\right\}
	,
\end{equation}
we use it to evolve the state in \cref{eq:photoelectric_transition_prob_density},
\begin{equation}
	\begin{split}
		\hat\rho(t+\Delta t)
		&=
		\hat{U}_\text{int}(t,t+\Delta t)
		\hat\rho(t)
		\hat{U}_\text{int}(t,t+\Delta t)^\dagger
		\\
		&=
		\hat\rho(t)
		+
		(-i)
		\int_t^{t+\Delta t}\dd{t_1}
		\comm{\hat{H}_\text{int}(t_1)}{\hat\rho(t_0)}
		+
		\frac{(-i)^2}{2!}
		\int_{t}^{t+\Delta t}\dd{t_1}
		\int_{t}^{t_1}\dd{t_2}
		\comm{\hat{H}_\text{int}(t_1)}{\comm{\hat{H}_\text{int}(t_2)}{\hat\rho(t)}}
		+
		\dots
	\end{split}
\end{equation}
where the second equation follows from the \gls{bch} formula.
Inserting the expansion into the photoemission probability, \cref{eq:photoelectric_transition_prob_density}, the first two term vanish due to orthogonality with $\hat\varrho_{e,f}$~\cite[p.~686]{Mandel1995}, leaving us with
\begin{equation}
	\begin{split}
		\trace\left\{
			\hat\varrho_{e,f}
			\hat\rho(t_0+\Delta t)
		\right\}
		&=
		\int_t^{t+\Delta t}\dd{t_1}
		\int_t^{t_1}
		\dd{t_2}
		\trace\left\{
			\hat\varrho_{e,f}
			\hat{H}_\text{int}(t_1)
			\hat\rho(t)
			\hat{H}_\text{int}(t_2)
		\right\}
		+
		\text{c.c.}
		\\
		&=
		\int_t^{t+\Delta t}\dd{t_1}
		\int_t^{t_1}
		\dd{t_2}
		\bra{e,f}
			\hat{H}_\text{int}(t_1)
			\hat\rho(t)
			\hat{H}_\text{int}(t_2)
		\ket{e,f}
		+
		\text{c.c.}
		.
	\end{split}
\end{equation}
Inserting the interaction Hamiltonian, \cref{eq:photoelectric_time_evolution_operator}, into our previous result, we find~\cite[p.~693]{Mandel1995}
\begin{equation}
	\begin{split}
		\trace\left\{
			\hat\varrho_{e,f}
			\hat\rho(t_0+\Delta t)
		\right\}
		&=
		\bra{e}\hat{p}^i\ket{g}
		\bra{g}\hat{p}^j\ket{e}
		\int_t^{t+\Delta t}\dd{t_1}
		\int_t^{t_1}
		\dd{t_2}
		\\
		&\times
		\bra{f}
			\hat{A}_i(t_1)
			\ketbra{i}
			\hat{A}_j(t_2)
		\ket{f}
		e^{i(E_e-E_g)(t_1-t_2)}
		+
		\text{c.c.}
		,
	\end{split}	
\end{equation}
wherein we used the energy eigenvalues of the electron's ground and excited state, $\ket{g},\ket{e}$.
The final states of the light field are of no interest to use and can be marginalized~\cite[p.~694]{Mandel1995}, i.e.,
\begin{equation}
	\begin{split}
		\sum_f
		\trace\left\{
			\hat\varrho_{e,f}
			\hat\rho(t_0+\Delta t)
		\right\}
		&=
		\bra{e}\hat{p}^i\ket{g}
		\bra{g}\hat{p}^j\ket{e}
		\int_t^{t+\Delta t}\dd{t_1}
		\int_t^{t_1}
		\dd{t_2}
		\\
		&\times
		\bra{i}
			\hat{A}_i(t_1)
			\hat{A}_j(t_2)
		\ket{i}
		e^{i(E_e-E_g)(t_1-t_2)}
		+
		\text{c.c.}
	\end{split}	
\end{equation}
and we conclude the probability for a single photoelectron to be emitted between time $t$ and $t+\Delta t$ to be
\begin{equation}
	p(t,\Delta t)
	=
	\int_t^{t+\Delta t}\dd{t_1}
	\int_t^{t_1}
	\dd{t_2}
	k^{ij}(t_1-t_2)
	\expval{
		\hat{A}_i(t_1)
		\hat{A}_j(t_2)
	}
	+
	\text{c.c.}
	,
\end{equation}
wherein $k_{ij}(t)$ is the effective response function of the detector atom\footnote{The effective response function depends on the electron's density of states and dipole transition moments, see Ref.~\cite[p.~694]{Mandel1995}.}, and the expectation value of the Maxwell field two-point correlation function is with respect to the initial light state.
It is possible to show that only the normal-ordered Maxwell field expectation value contributes to the photoemission probability~\cite[p.~696]{Mandel1995}
\begin{equation}
	p(t,\Delta t)
	=
	\int_t^{t+\Delta t}\dd{t_1}
	\int_t^{t_1}
	\dd{t_2}
	k^{ij}(t_1-t_2)
	\expval{
		\norder{
			\hat{A}_i(t_1)
			\hat{A}_j(t_2)
		}
	}
	+
	\text{c.c.}
	.	
\end{equation}
We select a coordinate system in which the Maxwell field propagates along the $z$ direction, then one can show~\cite{Kimble1984} that
\begin{equation}
	\begin{split}
		p(t,\Delta t)
		&=
		\int_t^{t+\Delta t}\dd{t_1}
		\int_t^{t_1}
		\dd{t_2}
		k(t_1-t_2)
		\sum_{\lambda=1,2}
		\expval{
			\norder{
				\hat{A}_\lambda(t_1)
				\hat{A}_\lambda(t_2)
			}
		}
		\\
		&=
		\int_t^{t+\Delta t}\dd{t_1}
		\int_t^{t_1}
		\dd{t_2}
		k(t_1-t_2)
		\expval{
			\norder{
				\hat{A}(t_1)
				\hat{A}(t_2)
			}
		}
		+
		\text{c.c.}
		,
	\end{split}
\end{equation}
wherein we defined the scalar polarization-averaged Maxwell field and have the effective response function
\begin{equation}
	k(t)
	=
	\int_0^\infty\frac{\dd{E}}{2\pi}
	K(E)
	e^{-i(E-E_g)t}
\end{equation}
with $K(E)$ being a function of the electron wave function along the $xy$ plane.

Expanding the normal-ordered two-point correlation function of the Maxwell field into positive and negative frequency parts
\begin{equation}
	\begin{split}
		\expval{
			\norder{
				\hat{A}_i(t_1)
				\hat{A}_j(t_2)
			}
		}
		&=
		\expval{
			\norder{
				\left[
					\hat{A}_i^{(-)}(t_1)
					+
					\hat{A}_i^{(+)}(t_1)
				\right]
				\left[
					\hat{A}_j^{(-)}(t_2)
					+
					\hat{A}_j^{(+)}(t_2)
				\right]
			}
		}
		\\
		&=
		\expval{
			\hat{A}_i^{(+)}(t_1)
			\hat{A}_j^{(-)}(t_2)
		}
		+
		\expval{
			\hat{A}_i^{(+)}(t_2)
			\hat{A}_j^{(-)}(t_1)
		}
		,
	\end{split}
\end{equation}
where the non-mixed frequency terms vanish because they contain an unequal number of annihilation and creation operators~\cite[p.~134]{Cohen1992}.
Inserting the mixed frequency terms into the photoemission probability and expanding the effective detector response function in the frequency domain, we drop the highly oscillatory terms and find\footnote{See Ref.~\cite[p.~697]{Mandel1995} and Ref.~\cite[p.~136]{Cohen1992} for an exact argument.}
\begin{equation}
	p(t,\Delta t)
	=
	\int_t^{t+\Delta t}\dd{t_1}
	\int_t^{t_1}
	\dd{t_2}
	k^{ij}(t_1-t_2)
	\expval{
		\hat{A}_i^{(+)}(t_1)
		\hat{A}_j^{(-)}(t_2)
	}
	+
	\text{c.c.}
	.
\end{equation}



The differential probability for photoelectron emission of a single detector atom is~\cite{Kimble1984}
\begin{equation}
	p(t,\Delta t)
	\approx
	K(\omega_0+E_g)
	\expval{
		\hat{A}^{(+)}(t)
		\hat{A}^{(-)}(t)
	}
	\Delta t
	\label{eq:differential_photoemission_probability}
\end{equation}
wherein $\omega_0$ is an optical center frequency.

\subsection{Instantaneous intensities of the Maxwell and electric field}

We previously derived that the differential single photoelectron excitation probability is proportional to the two-point time correlation function of the Maxwell field,
\begin{equation}
	\hat{A}^{(+)}(t_1)
	\hat{A}^{(-)}(t_2)
	=
	\left(
		\int\frac{\dd{p}}{2\pi}
		\frac{1}{\sqrt{2p}}
		\hat{a}^\dagger(p)
		e^{+ipt_1}
	\right)
	\left(
		\int\frac{\dd{q}}{2\pi}
		\frac{1}{\sqrt{2q}}
		\hat{a}(q)
		e^{-iqt_2}
	\right)
	,
	\label{eq:maxwell_field_two_point_correlation}
\end{equation}
where we inserted the one-dimensional reduction of the positive and negative frequency Maxwell field operators, \cref{eq:maxwell_positive_operator,eq:maxwell_negative_operator}.
It is convenient to relate the two-point time correlation of the Maxwell field, \cref{eq:maxwell_field_two_point_correlation}, with the two-point time correlation of the electric field,
\begin{equation}
	\hat{E}^{(+)}(t_1)
	\hat{E}^{(-)}(t_2)
	=
	\left(
		\int\frac{\dd{p}}{2\pi}
		\sqrt{\frac{p}{2}}
		\hat{a}^\dagger(p)
		e^{+ipt_1}
	\right)
	\left(
		\int\frac{\dd{q}}{2\pi}
		\sqrt{\frac{q}{2}}
		\hat{a}(q)
		e^{-iqt_2}
	\right)
	,
	\label{eq:electric_field_two_point_correlation}
\end{equation}
where we used \cref{eq:electric_positive_operator,eq:electric_negative_operator}, because for a coherent state, $\ket{\alpha}$, we can express the expectation value in terms of the signal amplitude, $\alpha(t)$, i.e.,
\begin{equation}
	\begin{split}
		\bra{\alpha(t)}
		\hat{E}^{(+)}(t_1)
		\hat{E}^{(-)}(t_2)
		\ket{\alpha(t)}
		&=
		\left(
			\int\frac{\dd{p}}{2\pi}
			\sqrt{\frac{p}{2}}
			\frac{\alpha(p)^*}{\sqrt{2p}}
			e^{+ipt_1}
		\right)
		\left(
			\int\frac{\dd{q}}{2\pi}
			\sqrt{\frac{q}{2}}
			\frac{\alpha(q)}{\sqrt{2q}}
			e^{-iqt_2}
		\right)
		\\
		&=
		\frac{1}{4}
		\left(
			\int\frac{\dd{p}}{2\pi}
			\alpha(p)
			e^{-ipt_1}
		\right)^*
		\left(
			\int\frac{\dd{q}}{2\pi}
			\alpha(q)
			e^{-iqt_2}
		\right)
		\\
		&=
		\frac{1}{4}
		\alpha(t_1)^*
		\alpha(t_2)
		.
	\end{split}
\end{equation}
Ref.~\cite{Kimble1984} claims that for quasimonochromatic light with center frequency $\omega_0$, we can connect both time-correlation functions by
\begin{equation}
	\hat{E}^{(+)}(t_1)
	\hat{E}^{(-)}(t_2)
	=
	\omega_0^2
	\hat{A}^{(+)}(t_1)
	\hat{A}^{(-)}(t_2)
	.
\end{equation}
Quasimonochromatic light is bandwidth limited, hence, we can invoke the mean-value theorem for definite integrals to write
\begin{equation}
	\omega_0
	\hat{A}^{(+)}(t)
	\approx
	\int\frac{\dd{p}}{2\pi}
	\frac{p}{\sqrt{2p}}
	\hat{a}^\dagger(p)
	e^{+ipt_1}
	=
	\hat{E}^{(+)}(t)
	,
\end{equation}
assuming $\omega_0$ represents the \gls{com} in the frequency distribution.
The argument becomes problematic if we consider a time-dependent signal, e.g., a time-dependent coherent state, $\ket{\alpha(t)}$.

\textcolor{red}{How to resolve this conflict? Absorb frequency dependence into linear filter?}
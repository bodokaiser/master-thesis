\chapter{Receiver synchronization}\label{app:synchronization}

Our presentation of the coherent-state transmission system assumes the receiver and transmitter use the same time reference, particularly regarding the \gls{adc}, \gls{dac}, and laser \glspl{lo}.
However, in practice, the receiver and transmitter run independent clocks requiring synchronization procedures.

As shown in \Cref{ch:system}, the transmitter \gls{lo} defines the upconversion frequency $\omega_c$, and the receiver \gls{lo} determines the downconversion frequency $\omega_l$.
Together the up- and downconversion frequencies defines the intermediate frequency $\omega_i=\omega_c-\omega_l$.
For drifting transmitter and receiver \glspl{lo}, the intermediate frequency shifts and, in the worst case, exceeds the detector bandwidth.
The transmitter adds a pilot tone, a strong narrow-linewidth signal to provide a frequency reference for the receiver.\footnote{Refs.~\cite{Huang2015,Qi2015,Soh2015} discuss the usage of a pilot tone for \gls{cvqkd}.}
\begin{figure}[ht]
	\centering
	\includegraphics{figures/pgfplots/laser-sync}
	\caption{Power spectrum of the transmitted and received pilot tone. The transmitted pilot tone (green) represents a perfect sinusoidal. The received pilot tone (orange) is broadened by phase noise and shifted by the offset between the transmitter and receiver \gls{lo}.}\label{fig:pilot_tones}
\end{figure}
Additionally to frequency offsets, the pilot tone allows to suppress phase noise by employing a Wiener filter~\cite{Chen2006} on the broadened lineprofile of the pilot tone, as illustrated in \Cref{fig:pilot_tones}.

The second set of oscillators that need synchronization are the clocks of the \gls{dac} and \gls{adc}.
Unlike the optical \gls{lo}s, the electronic clocks oscillate much slower, and we only need to compensate for timing offsets.
\begin{figure}[htb]
	\centering
	\includegraphics{figures/pgfplots/adc-sync}
	\caption{Timing offset between the \gls{dac} (first row) and \gls{adc} (second row) samples. The third row shows the linear increase in phase due to the accumulated delay.}\label{fig:adc_sync}
\end{figure}
\Cref{fig:adc_sync} illustrates the timing offset between the transmitted and received samples.
To correct for symbol-timing errors on the receiver, one can employ, for instance, the Goddard algorithm~\cite{Godard1978}, which exploits that a delay increases the phase of a signal.\footnote{See Ref.~\cite[p.~359]{Proakis2007} for more details on symbol-timing estimation techniques.}.

On a protocol level, the receiver needs to detect the beginning of a data frame, i.e., a related symbol sequence, and assign a frame number.
\begin{figure}[htb]
	\centering
	\includegraphics{figures/tikz/frame}
	\caption{Data layout of the transmission frames. A frame comprises a training and data sequence. The training sequence is the same for all frames.}\label{fig:frame}
\end{figure}
One possibility for the transmitter to inform the receiver about the beginning of a new frame is to interleave the data sequences with a fixed training sequence, see \Cref{fig:frame}, known to the receiver.
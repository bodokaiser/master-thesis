\chapter{Photodetection theory}\label{app:photodetection_theory}

This section aims to derive the differential photoelectron-emission probability from which one can derive the photocurrent operator.
Our derivation summarizes the steps in Refs.~\cite{Mandel1995,Vogel2006} but adds additional details on the truncated steps.

The transition of a bound electron to a free photoelectron is a probabilistic process.
Let $\ket{i}$ and $\ket{f}$ denote the initial and final light states, and let $\ket{g}$ and $\ket{e}$ be the electron ground and excited states.
The probability for the transition, $\ket{g,i}\to\ket{e,f}$, from time $t$ to $t+\Delta t$ is equal to
\begin{equation}
	\abs{
		\bra{e,f}
		\hat{U}_\text{int}(t,t+\Delta t)
		\ket{g,i}
	}^2
	\label{eq:photoelectric_transition_prob}
	,
\end{equation}
wherein $\hat{U}_\text{int}$ is the time-evolution operator of the photo-atom interaction in the dipole approximation~\cite[p.~689]{Mandel1995},
\begin{align}
	\hat{U}_\text{int}(t,t+\Delta t)
	&=
	\mathcal{T}_+
	\exp\left\{
		-i
		\int_t^{t+\Delta t}
		\dd{t^\prime}
		\hat{H}_\text{int}(t^\prime)
	\right\}
	&
	\hat{H}_\text{int}(t)
	&=
	-
	\hat{\vb{p}}(t)
	\vdot
	\hat{\vb{A}}(t)
	\label{eq:photoelectric_time_evolution_operator}
\end{align}
with $\vu{p}$ being the electron's momentum operator, $\vu{A}(t)=\vu{A}(t,\vb{x}_0)$ being the Maxwell field in the Coulomb gauge approximated at the atom \gls{com}, and $\mathcal{T}_+$ denoting (forward) time-ordering.
In the more general density operator formailism \cref{eq:photoelectric_transition_prob} reads~\cite[p.~686]{Mandel1995}
\begin{equation}
	\begin{split}
		\abs{
			\bra{e,f}
			\hat{U}_\text{int}(t,t+\Delta t)
			\ket{g,i}
		}^2
		&=
		\bra{e,f}
		\hat{U}_\text{int}(t,t+\Delta t)
		\ketbra{g,i}
		\hat{U}_\text{int}(t,t+\Delta t)^\dagger
		\ket{e,f}
		\\
		&=
		\trace\biggl\{
			\bra{e,f}
			\hat{U}_\text{int}(t,t+\Delta t)
			\ketbra{g,i}
			\hat{U}_\text{int}(t,t+\Delta t)^\dagger
			\ket{e,f}
		\biggr\}
		\\
		&=
		\trace\biggl\{
			\ketbra{e,f}
			\hat{U}_\text{int}(t,t+\Delta t)
			\ketbra{g,i}
			\hat{U}_\text{int}(t,t+\Delta t)^\dagger
		\biggr\}
		\\
		&=
		\trace\biggl\{
			\hat\varrho_{e,f}
			\hat{U}_\text{int}(t,t+\Delta t)
			\hat\rho(t)
			\hat{U}_\text{int}(t,t+\Delta t)^\dagger
		\biggr\}
		\\
		&=
		\trace\biggl\{
			\hat\varrho_{e,f}
			\hat\rho(t+\Delta t)
		\biggr\}
		,
	\end{split}
	\label{eq:photoelectric_transition_prob_density}
\end{equation}
wherein we used that the trace of a scalar is the scalar in the second line and the cyclic property of the trace in the third line.
Performing the Magnus expansion of the time-evolution operator, \cref{eq:photoelectric_time_evolution_operator}, up to the first term,
\begin{equation}
	\hat{U}_\text{int}(t,t+\Delta t)
	\approx
	\exp\left\{
		-i
		\int_t^{t+\Delta t}\dd{t^\prime}
		\hat{H}_\text{int}(t^\prime)
	\right\}
	,
\end{equation}
we use it to evolve the state in \cref{eq:photoelectric_transition_prob_density},
\begin{equation}
	\begin{split}
		\hat\rho(t+\Delta t)
		&=
		\hat{U}_\text{int}(t,t+\Delta t)
		\hat\rho(t)
		\hat{U}_\text{int}(t,t+\Delta t)^\dagger
		\\
		&=
		\hat\rho(t)
		+
		(-i)
		\int_t^{t+\Delta t}\dd{t_1}
		\comm{\hat{H}_\text{int}(t_1)}{\hat\rho(t_0)}
		\\
		&\qquad\times
		+
		\frac{(-i)^2}{2!}
		\int_{t}^{t+\Delta t}\dd{t_1}
		\int_{t}^{t_1}\dd{t_2}
		\comm{\hat{H}_\text{int}(t_1)}{\comm{\hat{H}_\text{int}(t_2)}{\hat\rho(t)}}
		+
		\dots
	\end{split}
\end{equation}
where the second equation follows from the \gls{bch} formula.
Inserting the expansion into the photoemission probability, \cref{eq:photoelectric_transition_prob_density}, the first two term vanish due to orthogonality with $\hat\varrho_{e,f}$~\cite[p.~686]{Mandel1995}, leaving us with
\begin{equation}
	\begin{split}
		\trace\left\{
			\hat\varrho_{e,f}
			\hat\rho(t_0+\Delta t)
		\right\}
		&=
		\int_t^{t+\Delta t}\dd{t_1}
		\int_t^{t_1}
		\dd{t_2}
		\trace\left\{
			\hat\varrho_{e,f}
			\hat{H}_\text{int}(t_1)
			\hat\rho(t)
			\hat{H}_\text{int}(t_2)
		\right\}
		+
		\text{c.c.}
		\\
		&=
		\int_t^{t+\Delta t}\dd{t_1}
		\int_t^{t_1}
		\dd{t_2}
		\bra{e,f}
			\hat{H}_\text{int}(t_1)
			\hat\rho(t)
			\hat{H}_\text{int}(t_2)
		\ket{e,f}
		+
		\text{c.c.}
		.
	\end{split}
\end{equation}
Inserting the interaction Hamiltonian, \cref{eq:photoelectric_time_evolution_operator}, into our previous result, we find~\cite[p.~693]{Mandel1995}
\begin{equation}
	\begin{split}
		\trace\left\{
			\hat\varrho_{e,f}
			\hat\rho(t_0+\Delta t)
		\right\}
		&=
		\bra{e}\hat{p}^i\ket{g}
		\bra{g}\hat{p}^j\ket{e}
		\int_t^{t+\Delta t}\dd{t_1}
		\int_t^{t_1}
		\dd{t_2}
		\\
		&\qquad\times
		\bra{f}
			\hat{A}_i(t_1)
			\ketbra{i}
			\hat{A}_j(t_2)
		\ket{f}
		e^{i(E_e-E_g)(t_1-t_2)}
		+
		\text{c.c.}
		,
	\end{split}	
\end{equation}
wherein we used the energy eigenvalues of the electron's ground and excited state, $\ket{g},\ket{e}$.
The final states of the light field are of no interest to use and can be marginalized~\cite[p.~694]{Mandel1995}, i.e.,
\begin{equation}
	\begin{split}
		\sum_f
		\trace\left\{
			\hat\varrho_{e,f}
			\hat\rho(t_0+\Delta t)
		\right\}
		&=
		\bra{e}\hat{p}^i\ket{g}
		\bra{g}\hat{p}^j\ket{e}
		\int_t^{t+\Delta t}\dd{t_1}
		\int_t^{t_1}
		\dd{t_2}
		\\
		&\qquad\times
		\bra{i}
			\hat{A}_i(t_1)
			\hat{A}_j(t_2)
		\ket{i}
		e^{i(E_e-E_g)(t_1-t_2)}
		+
		\text{c.c.}
	\end{split}	
\end{equation}
and we conclude the probability for a single photoelectron to be emitted between time $t$ and $t+\Delta t$ to be
\begin{equation}
	p(t,\Delta t)
	=
	\int_t^{t+\Delta t}\dd{t_1}
	\int_t^{t_1}
	\dd{t_2}
	k^{ij}(t_1-t_2)
	\expval{
		\hat{A}_i(t_1)
		\hat{A}_j(t_2)
	}
	+
	\text{c.c.}
	,
\end{equation}
wherein $k_{ij}(t)$ is the effective response function of the detector atom\footnote{The effective response function depends on the electron's density of states and dipole transition moments, see Ref.~\cite[p.~694]{Mandel1995}.}, and the expectation value of the Maxwell field two-point correlation function is with respect to the initial light state.
It is possible to show that only the normal-ordered Maxwell field expectation value contributes to the photoemission probability~\cite[p.~696]{Mandel1995}
\begin{equation}
	p(t,\Delta t)
	=
	\int_t^{t+\Delta t}\dd{t_1}
	\int_t^{t_1}
	\dd{t_2}
	k^{ij}(t_1-t_2)
	\expval{
		\norder{
			\hat{A}_i(t_1)
			\hat{A}_j(t_2)
		}
	}
	+
	\text{c.c.}
	.	
\end{equation}
We select a coordinate system in which the Maxwell field propagates along the $z$ direction, then one can show~\cite{Kimble1984} that
\begin{equation}
	\begin{split}
		p(t,\Delta t)
		&=
		\int_t^{t+\Delta t}\dd{t_1}
		\int_t^{t_1}
		\dd{t_2}
		k(t_1-t_2)
		\sum_{\lambda=1,2}
		\expval{
			\norder{
				\hat{A}_\lambda(t_1)
				\hat{A}_\lambda(t_2)
			}
		}
		\\
		&=
		\int_t^{t+\Delta t}\dd{t_1}
		\int_t^{t_1}
		\dd{t_2}
		k(t_1-t_2)
		\expval{
			\norder{
				\hat{A}(t_1)
				\hat{A}(t_2)
			}
		}
		+
		\text{c.c.}
		,
	\end{split}
\end{equation}
wherein we defined the scalar polarization-averaged Maxwell field and have the effective response function
\begin{equation}
	k(t)
	=
	\int_0^\infty\frac{\dd{E}}{2\pi}
	K(E)
	e^{-i(E-E_g)t}
\end{equation}
with $K(E)$ being a function of the electron wave function along the $xy$ plane.

Expanding the normal-ordered two-point correlation function of the Maxwell field into positive and negative frequency parts
\begin{equation}
	\begin{split}
		\expval{
			\norder{
				\hat{A}_i(t_1)
				\hat{A}_j(t_2)
			}
		}
		&=
		\expval{
			\norder{
				\left[
					\hat{A}_i^{(-)}(t_1)
					+
					\hat{A}_i^{(+)}(t_1)
				\right]
				\left[
					\hat{A}_j^{(-)}(t_2)
					+
					\hat{A}_j^{(+)}(t_2)
				\right]
			}
		}
		\\
		&=
		\expval{
			\hat{A}_i^{(+)}(t_1)
			\hat{A}_j^{(-)}(t_2)
		}
		+
		\expval{
			\hat{A}_i^{(+)}(t_2)
			\hat{A}_j^{(-)}(t_1)
		}
		,
	\end{split}
\end{equation}
where the non-mixed frequency terms vanish because they contain an unequal number of annihilation and creation operators~\cite[p.~134]{Cohen1992}.
Inserting the mixed frequency terms into the photoemission probability and expanding the effective detector response function in the frequency domain, we drop the highly oscillatory terms and find\footnote{See Ref.~\cite[p.~697]{Mandel1995} and Ref.~\cite[p.~136]{Cohen1992} for an exact argument.}
\begin{equation}
	p(t,\Delta t)
	=
	\int_t^{t+\Delta t}\dd{t_1}
	\int_t^{t_1}
	\dd{t_2}
	k^{ij}(t_1-t_2)
	\expval{
		\hat{A}_i^{(+)}(t_1)
		\hat{A}_j^{(-)}(t_2)
	}
	+
	\text{c.c.}
	.
\end{equation}

The differential probability for photoelectron emission of a single detector atom is~\cite{Kimble1984}
\begin{equation}
	p(t,\Delta t)
	\approx
	K(\omega_0+E_g)
	\expval{
		\hat{A}^{(+)}(t)
		\hat{A}^{(-)}(t)
	}
	\Delta t
	\label{eq:differential_photoemission_probability}
\end{equation}
wherein $\omega_0$ is an optical center frequency.
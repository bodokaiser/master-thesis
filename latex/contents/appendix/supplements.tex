\chapter{Supplements}

The supplemental material captures auxiliary calculations which would otherwise disrupt the reading.

\section{Theorems for quantum states}

\begin{lemma}\label{th:annihilation_field_commutators}
	Let $\hat{a}(\vb{k})$ and $\hat{a}^\dagger(\vb{k})$ be the annihilation and creation operator satisfying the \gls{ccr} and $\hat{A}^{(\pm)}$ be the positive and negative frequency field operator, then the commutator of the annihilation and creation operator with the positive and negative smeared field operators yields
	\begin{align}
		\comm{\hat{a}(\vb{p})}{\hat{A}^{(+)}[f]}
		&=
		+\frac{f\left(\omega(\vb{p}),\vb{p}\right)}{\sqrt{2\omega(\vb{p})}}
		\\
		\comm{\hat{a}^\dagger(\vb{p})}{\hat{A}^{(-)}[f]}
		&=
		-\frac{f\left(\omega(\vb{p}),\vb{p}\right)^*}{\sqrt{2\omega(\vb{p})}}
		.
	\end{align}
\end{lemma}
\begin{proof}
	Inserting the smeared field operator in momentum space and using the \gls{ccr} to evaluate the integral yields the first identity
	\begin{equation}
		\comm{\hat{a}(\vb{p})}{\hat{A}^{(+)}[f]}
		=
		\int\frac{\dd[3]{q}}{(2\pi)^3\sqrt{2\omega(\vb{q})}}
		f\left(\omega(\vb{q}),\vb{q}\right)
		\comm{\hat{a}(\vb{p})}{\hat{a}^\dagger(\vb{q})}
		=
		+\frac{f\left(\omega(\vb{p}),\vb{p}\right)}{\sqrt{2\omega(\vb{p})}}
		.
	\end{equation}
	The second identity follows analog
	\begin{equation}
		\comm{\hat{a}^\dagger(\vb{p})}{\hat{A}^{(-)}[f]}
		=
		\int\frac{\dd[3]{q}}{(2\pi)^3\sqrt{2\omega(\vb{q})}}
		f\left(\omega(\vb{q}),\vb{q}\right)^*
		\comm{\hat{a}^\dagger(\vb{p})}{\hat{a}(\vb{q})}
		=
		-\frac{f\left(\omega(\vb{p}),\vb{p}\right)^*}{\sqrt{2\omega(\vb{p})}}
		.
	\end{equation}
\end{proof}

\begin{lemma}\label{th:number_annihilation}
	Let $\hat{a}(\vb{k})$ be the annihilation operator and $\hat{A}^{(-)}$ be the negative frequency field operator, then their commutator yields
	\begin{equation}
		\hat{a}(\vb{p})
		\ket{n_f}
		=
		\sqrt{n}
		\frac{f\left(\omega(\vb{p}),\vb{p}\right)}{\sqrt{2\omega(\vb{p})}}
		\ket{{n-1}_f}
		\label{eq:number_annihilation}
		.
	\end{equation}
\end{lemma}
\begin{proof}
	First, we note the recursive relation
	\begin{equation}
		\hat{a}(\vb{p})
		\ket{n_f}
		=
		\frac{1}{\sqrt{n}}
		\left[
			\frac{f\left(\omega(\vb{p}),\vb{p}\right)}{\sqrt{2\omega(\vb{p})}}
			\ket{{n-1}_f}
			+
			\hat{A}^{(+)}[f]
			\hat{a}(\vb{p})
			\ket{{n-1}_f}
		\right]
		,
	\end{equation}
	where we used the commutator from \cref{th:annihilation_field_commutators}.
	The induction start, $n=0$, follows from the action of the annihilation operator on the vacuum state, \cref{eq:vacuum_annihilation}.
	The induction step, $n\to n+1$, goes
	\begin{equation}
		\begin{split}
			\hat{a}(\vb{p})
			\ket{{n+1}_f}
			&=
			\frac{1}{\sqrt{n+1}}
			\hat{a}(\vb{p})
			\hat{A}^{(+)}[f]
			\ket{n_f}
			\\
			&=
			\frac{1}{\sqrt{n+1}}
			\left[
				\frac{f\left(\omega(\vb{p}),\vb{p}\right)}{\sqrt{2\omega(\vb{p})}}
				+
				\hat{A}^{(+)}[f]
				\hat{a}(\vb{p})
			\right]
			\ket{n_f}
			\\
			&=
			\frac{1}{\sqrt{n+1}}
			\left[
				\frac{f\left(\omega(\vb{p}),\vb{p}\right)}{\sqrt{2\omega(\vb{p})}}
				\ket{n_f}
				+
				\hat{A}^{(+)}[f]
				\sqrt{n}
				\frac{f\left(\omega(\vb{p}),\vb{p}\right)}{\sqrt{2\omega(\vb{p})}}
				\ket{{n-1}_f}
			\right]
			\\
			&=
			\frac{1}{\sqrt{n+1}}
			\frac{f\left(\omega(\vb{p}),\vb{p}\right)}{\sqrt{2\omega(\vb{p})}}
			\left[
				\ket{n_f}
				+
				n
				\ket{n_f}
			\right]
			\\
			&=
			\sqrt{n+1}
			\frac{f\left(\omega(\vb{p}),\vb{p}\right)}{\sqrt{2\omega(\vb{p})}}
			\ket{n_f}
			,
		\end{split}
	\end{equation}
	where we used the recursive relation.
\end{proof}

\begin{lemma}
	Let $\ket{n_f}$ be a number state and $\vu{P}$ be the momentum operator, \cref{eq:maxwell_momentum_operator}, then number state has the expectation value
	\begin{equation}
		\bra{n_f}
		\vu{P}
		\ket{n_f}
		=
		n
		\int\frac{\dd[3]{p}}{(2\pi)^3}
		\vb{p}
		\abs*{\frac{f\left(\omega(\vb{q}),\vb{q}\right)}{\sqrt{2\omega(\vb{p})}}}^2
	\end{equation}
	of the momentum operator.
\end{lemma}
\begin{proof}
	We insert the definition of the momentum operator and use the action of the annihilation operator on the number state, \cref{th:number_annihilation},
	\begin{equation}
		\begin{split}
			\bra{n_f}
			\vu{P}
			\ket{n_f}
			&=
			\int\frac{\dd[3]{p}}{(2\pi)^3}
			\vb{p}
			\bra{n_f}
			\hat{a}^\dagger(\vb{p})
			\hat{a}(\vb{p})
			\ket{n_f}
			\\
			&=
			n
			\int\frac{\dd[3]{p}}{(2\pi)^3}
			\vb{p}
			\abs*{\frac{f\left(\omega(\vb{q}),\vb{q}\right)}{\sqrt{2\omega(\vb{p})}}}^2
			\braket{{n-1}_f}{{n-1}_f}
		\end{split}
	\end{equation}
	and we used took the hermitian conjugate of \cref{eq:number_annihilation}.
\end{proof}

\begin{lemma}\label{th:electric_operator_mixed_frequency}
	Let $\hat{E}^{(-)}(t,\vb{x})$ be the negative frequency and $\hat{E}^{(+)}(t,\vb{x})$ be the positive frequency part of the electric field operator, then
	\begin{equation}
		\hat{E}^{(-)}(t,\vb{x})
		\hat{E}^{(+)}(t,\vb{x})
		=
		\frac{1}{2}
		\int\frac{\dd[3]{p}}{(2\pi)^3}
		\omega(\vb{p})
		+
		\hat{E}^{(+)}(-t,-\vb{x})
		\hat{E}^{(-)}(-t,-\vb{x})
		.
	\end{equation}
\end{lemma}
\begin{proof}
	The identity follows from the definitions of the electric and positive frequency electric field operator, \cref{eq:electric_positive_operator,eq:electric_negative_operator},
	and using the \gls{ccr} of the annihilation and creation operator, i.e.,
	\begin{equation}
		\begin{split}
			\hat{E}^{(-)}(t,\vb{x})
			\hat{E}^{(+)}(t,\vb{x})
			&=
			\int\frac{\dd[3]{p}}{(2\pi)^3\sqrt{2\omega(\vb{p})}}
			\omega(\vb{p})
			e^{+i\omega(\vb{p})t-i\vb{p}\vdot\vb{x}}
			\\
			&\times
			\int\frac{\dd[3]{q}}{(2\pi)^3\sqrt{2\omega(\vb{q})}}
			\omega(\vb{q})
			e^{-i\omega(\vb{q})t+i\vb{q}\vdot\vb{x}}
			\\
			&\times
			\left\{
				\comm{\hat{a}(\vb{p})}{\hat{a}^\dagger(\vb{q})}
				+
				\hat{a}^\dagger(\vb{q})
				\hat{a}(\vb{p})
			\right\}
			\\
			&=
			\frac{1}{2}
			\int\frac{\dd[3]{p}}{(2\pi)^3}
			\omega(\vb{p})
			+
			\hat{E}^{(+)}(-t,-\vb{x})
			\hat{E}^{(-)}(-t,-\vb{x})
			.
		\end{split}
	\end{equation}
\end{proof}

\begin{theorem}
	Let $\ket{n_f}$ be a number state and $\hat{E}(t,\vb{x})$ be the electric field operator,
	then we have the expectation values
	\begin{align}
		\bra{n_f}
		\hat{E}(t,\vb{x})
		\ket{n_f}
		&=
		0
		\\
		\bra{n_f}
		\left(
			\Delta
			\hat{E}(t,\vb{x})
		\right)^2
		\ket{n_f}
		&=
		\frac{1}{2}
		\int\frac{\dd[3]{p}}{(2\pi)^3}
		\omega(\vb{p})
		+
		\frac{1}{2}
		\abs*{\Psi(t,\vb{x})}^2
	\end{align}
\end{theorem}
\begin{proof}
	The expectation values of an unequal number of annihilation and creation operators is always zero.
	Expanding the electric field operator into positive and negative frequency parts
	\begin{equation}
		\bra{n_f}
		\hat{E}(t,\vb{x})
		\ket{n_f}
		=
		\bra{n_f}
		\hat{E}^{(-)}(t,\vb{x})
		\ket{n_f}
		+
		\bra{n_f}
		\hat{E}^{(+)}(t,\vb{x})
		\ket{n_f}
		,
	\end{equation}
	we note that the first term comprises $n+1$ annihilation and $n$ creation operators, and the second term comprises $n$ annihilation and $n+1$ creation operators, i.e., an unequal number of annihilation and creation operators, and we conclude that the expectation value of the electric field operator given a number state is zero.
	The variance is equal to the second moment as the first moment is zero
	\begin{equation}
		\bra{n_f}
		\left(
			\Delta
			\hat{E}(t,\vb{x})
		\right)^2
		\ket{n_f}
		=
		\bra{n_f}
		\hat{E}(t,\vb{x})^2
		\ket{n_f}
		=
		\bra{n_f}
		\hat{E}^{(+)}(t,\vb{x})
		\hat{E}^{(-)}(t,\vb{x})
		\ket{n_f}
		+
		\text{h.c.}
	\end{equation}
	where we expanded the square of the electric field operator in its positive and negative frequency parts and used that only mixed terms survive in the second equation.
	Using \cref{th:number_annihilation}, we find
	\begin{equation}
		\begin{split}
			\bra{n_f}
			\hat{E}^{(+)}(t,\vb{x})
			\hat{E}^{(-)}(t,\vb{x})
			\ket{n_f}
			&=
			\int\frac{\dd[3]{p}}{(2\pi)^3\sqrt{2\omega(\vb{p})}}
			\omega(\vb{p})
			e^{+i\omega(\vb{p})t-i\vb{p}\vdot\vb{x}}
			\\
			&\times
			\int\frac{\dd[3]{q}}{(2\pi)^3\sqrt{2\omega(\vb{q})}}
			\omega(\vb{q})
			e^{-i\omega(\vb{q})t+i\vb{q}\vdot\vb{x}}
			\\
			&\times
			\bra{n_f}
			\hat{a}^\dagger(\vb{p})
			\hat{a}(\vb{q})
			\ket{n_f}
			\\
			&=
			\abs*{
				\frac{1}{2}
				\int\frac{\dd[3]{p}}{(2\pi)^3}
				f\left(\omega(\vb{p}),\vb{p}\right)
				e^{+i\omega(\vb{p})t-i\vb{p}\vdot\vb{x}}
			}^2
			.
		\end{split}
	\end{equation}
	The second mixed term can be written in terms of the first by using \cref{th:electric_operator_mixed_frequency}
	\begin{equation}
		\begin{split}
			\bra{n_f}
			\hat{E}^{(-)}(t,\vb{x})
			\hat{E}^{(+)}(t,\vb{x})
			\ket{n_f}
			&=
			\frac{1}{2}
			\int\frac{\dd[3]{p}}{(2\pi)^3}
			\omega(\vb{p})
			+
			\bra{n_f}
			\hat{E}^{(+)}(-t,-\vb{x})
			\hat{E}^{(-)}(-t,-\vb{x})
			\ket{n_f}
			\\
			&=
			\frac{1}{2}
			\int\frac{\dd[3]{p}}{(2\pi)^3}
			\omega(\vb{p})
			+
			\abs*{
				\frac{1}{2}
				\int\frac{\dd[3]{p}}{(2\pi)^3}
				f\left(\omega(\vb{p}),\vb{p}\right)
				e^{-i\omega(\vb{p})t+i\vb{p}\vdot\vb{x}}
			}^2
			.
		\end{split}
	\end{equation}
	Adding both terms together and identifying the restricted Fourier transform of the momentum spectrum as the wave function, $\Psi(t,\vb{x})$, we find the electric field variance to be
	\begin{equation}
		\bra{n_f}
		\left(
			\Delta
			\hat{E}(t,\vb{x})
		\right)^2
		\ket{n_f}
		=
		\frac{1}{2}
		\int\frac{\dd[3]{p}}{(2\pi)^3}
		\omega(\vb{p})
		+
		\frac{1}{4}
		\abs*{\Psi(t,\vb{x})}^2
		+
		\frac{1}{4}
		\abs*{\Psi(-t,-\vb{x})}^2
	\end{equation}
	where the first term denotes the "vacuum fluctuations".
\end{proof}

\begin{theorem}
	Let $\ket{\alpha}$ be a coherent state and $\hat{E}(t,\vb{x})$ be the electric field operator, then its expectation value is
	\begin{equation}
		\bra{\alpha}
		\hat{E}(t,\vb{x})
		\ket{\alpha}
		=
		\frac{i}{2}
		\int\frac{\dd[3]{p}}{(2\pi)^3}
		\left\{
			\alpha(\vb{p})
			e^{+i\omega(\vb{p})t-i\vb{p}\vdot\vb{x}}
			-
			\alpha(\vb{p})^*
			e^{-i\omega(\vb{p})t+i\vb{p}\vdot\vb{x}}
		\right\}
	\end{equation}
\end{theorem}
\begin{proof}
	The expectation value follows from using the the definition of the electric field operator, \cref{eq:electric_operator}, and using that the coherent state is an eigenstate of the annihilation operator, \cref{eq:coherent_state_annihilation}.
	Expanding the squared electric field operator into its positive and negative frequency parts
	\begin{equation}
		\begin{split}
			\hat{E}(t,\vb{x})^2
			&=
			\hat{E}^{(+)}(t,\vb{x})^2
			+
			\hat{E}^{(+)}(t,\vb{x})
			\hat{E}^{(-)}(t,\vb{x})
			+
			\hat{E}^{(-)}(t,\vb{x})
			\hat{E}^{(+)}(t,\vb{x})
			+
			\hat{E}^{(-)}(t,\vb{x})^2
			\\
			&=
			\frac{1}{2}
			\int\frac{\dd[3]{p}}{(2\pi)^3}
			\omega(\vb{p})
			+
			\hat{E}^{(+)}(t,\vb{x})^2
			+
			\hat{E}^{(-)}(t,\vb{x})^2
			\\
			&+
			\hat{E}^{(+)}(t,\vb{x})
			\hat{E}^{(-)}(t,\vb{x})
			+
			\hat{E}^{(+)}(-t,-\vb{x})
			\hat{E}^{(-)}(-t,-\vb{x})
		\end{split}
	\end{equation}
\end{proof}

\section{Second-order electric susceptibility tensor}

The susceptibility tensor relevant for the Pockels effect is usually given in a DC approximation without sidebands.

In the time domain, the second-order polarization density is~\cite[p.~55]{Boyd2020}
\begin{equation}
	P_i^{(2)}(t)
	=
	\int\dd{t_1}
	\int\dd{t_2}
	\chi^{(2)}_{ijk}(t_1,t_2)
	E^j(t-t_1)
	E^k(t-t_2)
	\label{eq:polarization_density_so_time}
\end{equation}
wherein $\chi^{(2)}_{ijk}(t_1,t_2)$ is the second-order time response tensor of the dielectric media.
Inserting the Fourier transform of the electric field components, we find
\begin{equation}
	P_i^{(2)}(t)
	=
	\int\frac{\dd{\omega_1}}{2\pi}
	\int\frac{\dd{\omega_2}}{2\pi}
	\chi^{(2)}_{ijk}(\omega_1,\omega_2)
	E^j(\omega_1)
	E^k(\omega_2)
	e^{-i(\omega_1+\omega_2)t}
	\label{eq:polarization_density_so_freq}
\end{equation}
wherein the second-order frequency response tensor is the Fourier transform of the time response tensor, i.e.,
\begin{equation}
	\chi^{(2)}_{ijk}(\omega_1,\omega_2)
	=
	\int\dd{t_1}
	\int\dd{t_2}
	\chi^{(2)}_{ijk}(t_1,t_2)
	e^{+i\omega_1t}
	e^{+i\omega_2t}
	.
\end{equation}
Inserting the \cref{eq:polarization_density_so_freq} into the Fourier transform of the second-order polarization density, we find
\begin{equation}
	\begin{split}
		P_i^{(2)}(\omega)
		&=
		\int\dd{t}
		P_i^{(2)}(t)
		e^{+i\omega t}
		\\
		&=
		\int\frac{\dd{\omega_1}}{2\pi}
		\int\frac{\dd{\omega_2}}{2\pi}
		\chi^{(2)}_{ijk}(\omega_1,\omega_2)
		E^j(\omega_1)
		E^k(\omega_2)
		\int\dd{t}
		e^{-i(\omega_1+\omega_2-\omega)t}
		\\
		&=
		\int\frac{\dd{\omega_1}}{2\pi}
		\int\frac{\dd{\omega_2}}{2\pi}
		\chi^{(2)}_{ijk}(\omega_1,\omega_2)
		E^j(\omega_1)
		E^k(\omega_2)
		(2\pi)
		\delta^{(1)}(\omega_2-\omega+\omega_1)
		\\
		&=
		\int\frac{\dd{\omega^\prime}}{2\pi}
		\chi^{(2)}_{ijk}(\omega^\prime,\omega-\omega^\prime)
		E^j(\omega^\prime)
		E^k(\omega-\omega^\prime)
		.
	\end{split}
\end{equation}
Assuming that only a static electric field contributes,
\begin{equation}
	\chi^{(2)}_{ijk}(\omega^\prime,\omega-\omega^\prime)
	\propto
	\delta^{(1)}(\omega^\prime)
	,
\end{equation}
we recover the polarization density for the static Pockels effect~\cite[p.~495]{Boyd2020}
\begin{equation}
	P_i^{(2)}(\omega)
	=
	\chi^{(2)}_{ijk}(0,\omega)
	E^j(0)
	E^k(\omega)
	=
	\varepsilon^{(2)}_{ij}(\omega)
	E^j(\omega)
\end{equation}
wherein $\varepsilon^{(2)}_{ij}(\omega)$ is the second-order dielectric susceptibility tensor in the frequency domain.
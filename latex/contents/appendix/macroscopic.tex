\chapter{Macroscopic electrodynamics}

\section{Polarization density and electric susceptibility tensor}\label{sec:polarization_density_susceptibility}

In the time domain, the first-order polarization density is~\cite[p.~17]{Murti2014}
\begin{equation}
	P_i^{(1)}(t)
	=
	\int\dd{t_1}
	\chi^{(1)}_{ij}(t_1)
	E^j(t-t_1)
	\label{eq:polarization_density_fo_time}
\end{equation}
wherein $\chi^{(1)}_{ijk}(t_1)$ is the first-order time response tensor of the dielectric media.
Inserting the Fourier transform of the electric field components, we find
\begin{equation}
	P_i^{(1)}(t)
	=
	\int\frac{\dd{\omega_1}}{2\pi}
	\chi^{(1)}_{ij}(\omega_1)
	E^j(\omega_1)
	e^{-i\omega_1t}
	\label{eq:polarization_density_fo_time_fourier}
	,
\end{equation}
wherein the second-order frequency response tensor is the Fourier transform of the time response tensor, i.e.,
\begin{equation}
	\chi^{(1)}_{ij}(\omega_1)
	=
	\int\dd{t_1}
	\chi^{(1)}_{ij}(t_1)
	e^{+i\omega_1t}
	.
\end{equation}
From \cref{eq:polarization_density_fo_time_fourier} we can directly read of the first-order polarization density in the frequency domain
\begin{equation}
	P_i^{(1)}(\omega)
	=
	\chi^{(1)}_{ij}(\omega)
	E^j(\omega)
	.
	\label{eq:polarization_density_fo_freq}
\end{equation}

In the time domain, the second-order polarization density is~\cite[p.~55]{Boyd2020}
\begin{equation}
	P_i^{(2)}(t)
	=
	\int\dd{t_1}
	\int\dd{t_2}
	\chi^{(2)}_{ijk}(t_1,t_2)
	E^j(t-t_1)
	E^k(t-t_2)
	\label{eq:polarization_density_so_time}
\end{equation}
wherein $\chi^{(2)}_{ijk}(t_1,t_2)$ is the second-order time response tensor of the dielectric media.
Inserting the Fourier transform of the electric field components, we find
\begin{equation}
	P_i^{(2)}(t)
	=
	\int\frac{\dd{\omega_1}}{2\pi}
	\int\frac{\dd{\omega_2}}{2\pi}
	\chi^{(2)}_{ijk}(\omega_1,\omega_2)
	E^j(\omega_1)
	E^k(\omega_2)
	e^{-i(\omega_1+\omega_2)t}
	\label{eq:polarization_density_so_time_fourier}
\end{equation}
wherein the second-order frequency response tensor is the Fourier transform of the time response tensor, i.e.,
\begin{equation}
	\chi^{(2)}_{ijk}(\omega_1,\omega_2)
	=
	\int\dd{t_1}
	\int\dd{t_2}
	\chi^{(2)}_{ijk}(t_1,t_2)
	e^{+i\omega_1t}
	e^{+i\omega_2t}
	.
\end{equation}
Contrary to \cref{eq:polarization_density_fo_time_fourier}, we cannot directly read of the polarization density in the frequency domain from \cref{eq:polarization_density_so_time_fourier}.
Instead, we insert \cref{eq:polarization_density_so_time_fourier} into the Fourier transform of the second-order polarization density
\begin{equation}
	\begin{split}
		P_i^{(2)}(\omega)
		&=
		\int\dd{t}
		P_i^{(2)}(t)
		e^{+i\omega t}
		\\
		&=
		\int\frac{\dd{\omega_1}}{2\pi}
		\int\frac{\dd{\omega_2}}{2\pi}
		\chi^{(2)}_{ijk}(\omega_1,\omega_2)
		E^j(\omega_1)
		E^k(\omega_2)
		\int\dd{t}
		e^{-i(\omega_1+\omega_2-\omega)t}
		\\
		&=
		\int\frac{\dd{\omega_1}}{2\pi}
		\int\frac{\dd{\omega_2}}{2\pi}
		\chi^{(2)}_{ijk}(\omega_1,\omega_2)
		E^j(\omega_1)
		E^k(\omega_2)
		(2\pi)
		\delta^{(1)}(\omega_2-\omega+\omega_1)
		\\
		&=
		\int\frac{\dd{\omega^\prime}}{2\pi}
		\chi^{(2)}_{ijk}(\omega^\prime,\omega-\omega^\prime)
		E^j(\omega^\prime)
		E^k(\omega-\omega^\prime)
		.
	\end{split}
	\label{eq:polarization_density_so_freq}
\end{equation}

The presence of the integral in \cref{eq:polarization_density_so_freq} breaks the linearity of the second-order polarization density in the electric field $E^k$.
To restore linearity, we use the mean value theorem for integrals at the radio center frequency $\Omega_0$,
\begin{equation}
	\begin{split}
		P_i^{(2)}(\omega)
		&\approx
		\chi^{(2)}_{ijk}(0,\omega)
		E^j(0)
		E^k(\omega)
		\\
		&+
		\chi^{(2)}_{ijk}(-\Omega_0,\omega+\Omega_0)
		E^j(-\Omega_0)
		E^k(\omega)
		\\
		&+
		\chi^{(2)}_{ijk}(+\Omega_0,\omega-\Omega_0)
		E^j(+\Omega_0)
		E^k(\omega)
		,
	\end{split}
\end{equation}
and find one zero-frequency component and two sidebands at $\pm\Omega_0$.
Usually, the second-order polarization density of the Pockels effect is only given with the zero-frequency component, corresponding to a static (time-independent) electric field~\cite[p.~495]{Boyd2020}.

\section{Refractive index of static Pockels effect}\label{sec:static_pockels}

The second-order polarization density of the static Pockels effect is~\cite[p.~495]{Boyd2020}
\begin{equation}
	P_i^{(2)}(\omega)
	=
	\chi^{(2)}_{ijk}(0,\omega)
	E^j(0)
	E^k(\omega)
	.
	\label{eq:polarization_density_so_freq_pockels}
\end{equation}
The electric displacement field in the frequency domain with polarization density expanded to second-order is~\cite[p.~1070]{Mandel1995}
\begin{equation}
	D_i(\omega)
	=
	E_i(\omega)
	+
	P_i^{(1)}(\omega)
	+
	P_i^{(2)}(\omega)
	.
	\label{eq:electric_displacement_field_expanded}
\end{equation}
Inserting \cref{eq:polarization_density_fo_freq} and \cref{eq:polarization_density_so_freq_pockels} into \cref{eq:electric_displacement_field_expanded}, we find
\begin{equation}
	D_i(\omega)
	=
	E_i(\omega)
	+
	\chi^{(1)}_{ij}(\omega)
	E^j(\omega)
	+
	\chi^{(2)}_{ijk}(0,\omega)
	E^j(0)
	E^k(\omega)
	.
\end{equation}
Rewriting the expansion of the electric displacement field as
\begin{equation}
	D_i(\omega)
	=
	\left[
		\delta_{ik}
		+
		\chi^{(1)}_{ik}(\omega)
		+
		\chi^{(2)}_{ijk}(0,\omega)
		E^j(0)
	\right]
	E^k(\omega)
	,
\end{equation}
we can read off the electric susceptibility tensor
\begin{equation}
	\varepsilon_{ik}(\omega)
	=
	\delta_{ik}
	+
	\chi^{(1)}_{ik}(\omega)
	+
	\chi^{(2)}_{ijk}(0,\omega)
	E^j(0)
	.
\end{equation}
The electric susceptibility tensor is equal to the squared refractive index tensor, $\varepsilon_{ik}(\omega)=n_{ik}(\omega)^2$, or equivalently~\cite[p.~3]{Brooker2003}
\begin{equation}
	n_{ij}(\omega)
	=
	\sqrt{\varepsilon_{ij}}
	\approx
	n_{ik}^{(0)}
	+
	n_{ijk}^{(1)}(\omega)
	E^j(0)
\end{equation}
wherein the refractive index coefficients relate to the dielectric susceptibility tensor via~\cite{Rerat2020}
\begin{align}
	n^{(0)}_{ij}(\omega)
	&=
	\sqrt{1+\chi^{(1)}_{ij}(\omega)}
	&
	n^{(1)}_{ijk}(\omega)
	&=
	\frac{\chi^{(2)}_{ijk}(\omega)
	E^k(0)}{2\sqrt{1+\chi^{(1)}_{ij}(\omega)}}
	.
\end{align}
\chapter{Macroscopic interactions}

\section{Nonlinear tensor expansion}\label{sec:polarization_density_susceptibility}

In the time domain, the first-order polarization density is~\cite[p.~17]{Murti2014}
\begin{equation}
	P_i^{(1)}(t)
	=
	\int\dd{t_1}
	\chi^{(1)}_{ij}(t_1)
	E^j(t-t_1)
	\label{eq:polarization_density_fo_time}
\end{equation}
wherein $\chi^{(1)}_{ijk}(t_1)$ is the first-order time response tensor of the dielectric media.
Inserting the Fourier transform of the electric field components, we find
\begin{equation}
	P_i^{(1)}(t)
	=
	\int\frac{\dd{\omega_1}}{2\pi}
	\chi^{(1)}_{ij}(\omega_1)
	E^j(\omega_1)
	e^{-i\omega_1t}
	\label{eq:polarization_density_fo_time_fourier}
	,
\end{equation}
wherein the second-order frequency response tensor is the Fourier transform of the time response tensor, i.e.,
\begin{equation}
	\chi^{(1)}_{ij}(\omega_1)
	=
	\int\dd{t_1}
	\chi^{(1)}_{ij}(t_1)
	e^{+i\omega_1t}
	.
\end{equation}
From \cref{eq:polarization_density_fo_time_fourier} we can directly read of the first-order polarization density in the frequency domain
\begin{equation}
	P_i^{(1)}(\omega)
	=
	\chi^{(1)}_{ij}(\omega)
	E^j(\omega)
	.
	\label{eq:polarization_density_fo_freq}
\end{equation}

In the time domain, the second-order polarization density is~\cite[p.~55]{Boyd2020}
\begin{equation}
	P_i^{(2)}(t)
	=
	\int\dd{t_1}
	\int\dd{t_2}
	\chi^{(2)}_{ijk}(t_1,t_2)
	E^j(t-t_1)
	E^k(t-t_2)
	\label{eq:polarization_density_so_time}
\end{equation}
wherein $\chi^{(2)}_{ijk}(t_1,t_2)$ is the second-order time response tensor of the dielectric media.
Inserting the Fourier transform of the electric field components, we find
\begin{equation}
	P_i^{(2)}(t)
	=
	\int\frac{\dd{\omega_1}}{2\pi}
	\int\frac{\dd{\omega_2}}{2\pi}
	\chi^{(2)}_{ijk}(\omega_1,\omega_2)
	E^j(\omega_1)
	E^k(\omega_2)
	e^{-i(\omega_1+\omega_2)t}
	\label{eq:polarization_density_so_time_fourier}
\end{equation}
wherein the second-order frequency response tensor is the Fourier transform of the time response tensor, i.e.,
\begin{equation}
	\chi^{(2)}_{ijk}(\omega_1,\omega_2)
	=
	\int\dd{t_1}
	\int\dd{t_2}
	\chi^{(2)}_{ijk}(t_1,t_2)
	e^{+i\omega_1t}
	e^{+i\omega_2t}
	.
\end{equation}
Contrary to \cref{eq:polarization_density_fo_time_fourier}, we cannot directly read of the polarization density in the frequency domain from \cref{eq:polarization_density_so_time_fourier}.
Instead, we insert \cref{eq:polarization_density_so_time_fourier} into the Fourier transform of the second-order polarization density
\begin{equation}
	\begin{split}
		P_i^{(2)}(\omega)
		&=
		\int\dd{t}
		P_i^{(2)}(t)
		e^{+i\omega t}
		\\
		&=
		\int\frac{\dd{\omega_1}}{2\pi}
		\int\frac{\dd{\omega_2}}{2\pi}
		\chi^{(2)}_{ijk}(\omega_1,\omega_2)
		E^j(\omega_1)
		E^k(\omega_2)
		\int\dd{t}
		e^{-i(\omega_1+\omega_2-\omega)t}
		\\
		&=
		\int\frac{\dd{\omega_1}}{2\pi}
		\int\frac{\dd{\omega_2}}{2\pi}
		\chi^{(2)}_{ijk}(\omega_1,\omega_2)
		E^j(\omega_1)
		E^k(\omega_2)
		(2\pi)
		\delta^{(1)}(\omega_2-\omega+\omega_1)
		\\
		&=
		\int\frac{\dd{\omega^\prime}}{2\pi}
		\chi^{(2)}_{ijk}(\omega^\prime,\omega-\omega^\prime)
		E^j(\omega^\prime)
		E^k(\omega-\omega^\prime)
		.
	\end{split}
	\label{eq:polarization_density_so_freq}
\end{equation}

The presence of the integral in \cref{eq:polarization_density_so_freq} breaks the linearity of the second-order polarization density in the electric field $E^k$.
To restore linearity, we use the mean value theorem for integrals at the radio center frequency $\Omega_0$,
\begin{equation}
	\begin{split}
		P_i^{(2)}(\omega)
		&\approx
		\chi^{(2)}_{ijk}(0,\omega)
		E^j(0)
		E^k(\omega)
		\\
		&+
		\chi^{(2)}_{ijk}(-\Omega_0,\omega+\Omega_0)
		E^j(-\Omega_0)
		E^k(\omega)
		\\
		&+
		\chi^{(2)}_{ijk}(+\Omega_0,\omega-\Omega_0)
		E^j(+\Omega_0)
		E^k(\omega)
		,
	\end{split}
\end{equation}
and find one zero-frequency component and two sidebands at $\pm\Omega_0$.
Usually, the second-order polarization density of the Pockels effect is only given with the zero-frequency component, corresponding to a static (time-independent) electric field~\cite[p.~495]{Boyd2020}.

\section{Static Pockels effect}\label{sec:static_pockels}

The second-order polarization density of the static Pockels effect is~\cite[p.~495]{Boyd2020}
\begin{equation}
	P_i^{(2)}(\omega)
	=
	\chi^{(2)}_{ijk}(0,\omega)
	E^j(0)
	E^k(\omega)
	.
	\label{eq:polarization_density_so_freq_pockels}
\end{equation}
The electric displacement field in the frequency domain with polarization density expanded to second-order is~\cite[p.~1070]{Mandel1995}
\begin{equation}
	D_i(\omega)
	=
	E_i(\omega)
	+
	P_i^{(1)}(\omega)
	+
	P_i^{(2)}(\omega)
	.
	\label{eq:electric_displacement_field_expanded}
\end{equation}
Inserting \cref{eq:polarization_density_fo_freq} and \cref{eq:polarization_density_so_freq_pockels} into \cref{eq:electric_displacement_field_expanded}, we find
\begin{equation}
	D_i(\omega)
	=
	E_i(\omega)
	+
	\chi^{(1)}_{ij}(\omega)
	E^j(\omega)
	+
	\chi^{(2)}_{ijk}(0,\omega)
	E^j(0)
	E^k(\omega)
	.
\end{equation}
Rewriting the expansion of the electric displacement field as
\begin{equation}
	D_i(\omega)
	=
	\left[
		\delta_{ik}
		+
		\chi^{(1)}_{ik}(\omega)
		+
		\chi^{(2)}_{ijk}(0,\omega)
		E^j(0)
	\right]
	E^k(\omega)
	,
\end{equation}
we can read off the electric susceptibility tensor
\begin{equation}
	\varepsilon_{ik}(\omega)
	=
	\delta_{ik}
	+
	\chi^{(1)}_{ik}(\omega)
	+
	\chi^{(2)}_{ijk}(0,\omega)
	E^j(0)
	.
\end{equation}
The electric susceptibility tensor is equal to the squared refractive index tensor, $\varepsilon_{ik}(\omega)=n_{ik}(\omega)^2$, or equivalently~\cite[p.~3]{Brooker2003}
\begin{equation}
	n_{ij}(\omega)
	=
	\sqrt{\varepsilon_{ij}}
	\approx
	n_{ik}^{(0)}
	+
	n_{ijk}^{(1)}(\omega)
	E^j(0)
\end{equation}
wherein the refractive index coefficients relate to the dielectric susceptibility tensor via~\cite{Rerat2020}
\begin{align}
	n^{(0)}_{ij}(\omega)
	&=
	\sqrt{1+\chi^{(1)}_{ij}(\omega)}
	&
	n^{(1)}_{ijk}(\omega)
	&=
	\frac{\chi^{(2)}_{ijk}(\omega)
	E^k(0)}{2\sqrt{1+\chi^{(1)}_{ij}(\omega)}}
	.
\end{align}

\section{Frequency conversion}

The Hamiltonian of the electromagnetic field in a dielectric medium is symbolically~\cite[p.~124]{Jackson2007}
\begin{equation}
	\hat{H}
	=
	\frac{1}{2}
	\int\dd[3]{x}
	\left\{
		\hat{E}^i(t,\vb{x})
		\hat{D}_i(t,\vb{x})
		+
		\hat{B}^i(t,\vb{x})
		\hat{B}_i(t,\vb{x})
	\right\}
	.
\end{equation}
We expand the electric displacement field up to the second-order polarization density
\begin{equation}
	\hat{D}_i(t,\vb{x})
	=
	\hat{E}_i(t,\vb{x})
	+
	\hat{P}_i^{(1)}(t,\vb{x})
	+
	\hat{P}_i^{(2)}(t,\vb{x})
\end{equation}
and insert it back into the Hamiltonian of the electromagnetic field
\begin{equation}
	\hat{H}
	=
	\frac{1}{2}
	\int\dd[3]{x}
	\left\{
		\hat{E}^i(t,\vb{x})
		\hat{E}_i(t,\vb{x})
		+
		\hat{B}^i(t,\vb{x})
		\hat{B}_i(t,\vb{x})
	\right\}
	+
	\hat{H}_\text{int}^{(1)}(t)
	+
	\hat{H}_\text{int}^{(2)}(t)
\end{equation}
and identify the free field Hamiltonian and two interaction terms with
\begin{align}
	\hat{H}_\text{int}^{(n)}(t)
	=	
	\frac{1}{2}
	\int\dd[3]{x}
	\hat{E}^i(t,\vb{x})
	\hat{P}_i^{(n)}(t,\vb{x})
	.
\end{align}

Assuming the dielectric to be absorption-free and time-invariant, we can write the first-order polarization density as
\begin{equation}
	\hat{P}_i^{(1)}(t,\vb{x})
	=
	\int\dd{t_1}
	\chi_{ij}^{(1)}(t_1,\vb{x})
	\hat{E}_i^{(i)}(t-t_1,\vb{x})
\end{equation}
and insert it back into the interaction Hamiltonian
\begin{equation}
	\hat{H}_\text{int}^{(1)}(t)
	=	
	\frac{1}{2}
	\int\dd{t_1}
	\chi_{ij}^{(1)}(t_1,\vb{x})
	\int\dd[3]{x}
	\hat{E}^i(t,\vb{x})
	\hat{E}^j(t-t_1,\vb{x})
	.
\end{equation}
We assume the electric fields to be polarized along the same axis and neglect the transverse mode profile
\begin{equation}
	\hat{H}_\text{int}^{(1)}(t)
	\approx
	\frac{1}{2}
	\int\dd{t_1}
	\chi^{(1)}(t_1,x)
	\int\dd{x}
	\hat{E}(t,x)
	\hat{E}(t-t_1,x)
	+
	\text{h.c.}
	.
\end{equation}
We expand the electric field into positive and negative frequency parts and perform the rotating-wave approximation
\begin{equation}
	\hat{H}_\text{int}^{(1)}(t)
	\approx
	\frac{1}{2}
	\int\dd{t_1}
	\chi^{(1)}(t_1,x)
	\int\dd{x}
	\hat{E}^{(-)}(t,x)
	\hat{E}^{(+)}(t-t_1,x)
	+
	\text{h.c.}
\end{equation}
by only keeping mixed frequency parts.
Inserting the mode expansion, \cref{eq:electric_positive_operator,eq:electric_positive_operator}, we find
\begin{equation}
	\begin{split}
		\hat{H}_\text{int}^{(1)}(t)
		&=
		\int\frac{\dd{p}}{2\pi\sqrt{2\omega(p)}}
		\omega(p)
		\hat{a}(p)
		e^{-i\omega(p)t}
		\int\frac{\dd{q}}{2\pi\sqrt{2\omega(q)}}
		\omega(q)
		\hat{a}^\dagger(q)
		e^{+i\omega(p)(t-t_1)}
		\\
		&\times
		\frac{1}{2}
		\int\dd{t_1}
		\int\dd{x}
		\chi^{(1)}(t_1,x)
		e^{i(p-q)x}
		+
		\text{h.c.}
		.
	\end{split}
\end{equation}
Furthermore, assuming the interaction strength to be uniform inside the dielectric, we can evaluate the first integral
\begin{equation}
	\int\dd{x}
	\chi^{(1)}(t_1,x)
	e^{i(p-q)x}
	=
	\chi^{(1)}(t_1)
	\int_{-L/2}^{+L/2}\dd{x}
	e^{i(p-q)x}
	=
	\chi^{(1)}(t_1)
	\sinc\left(\frac{p-q}{2/L}\right)
	L
\end{equation}
which is known as the phase-matching function in non-linear optics~\cite[p.~33]{QuesadaMejia2015}.
We assume perfect phase-matching
\begin{equation}
	\chi^{(1)}(t_1)
	\sinc\left(\frac{p-q}{2/L}\right)
	L
	\approx
	\chi^{(1)}(t_1)L
	(2\pi)
	\delta^{(1)}(p-q)
\end{equation}
and insert the result into the interaction Hamiltonian
\begin{equation}
	\begin{split}
		\hat{H}_\text{int}^{(1)}(t)
		&=
		\frac{L}{2}
		\int\frac{\dd{p}}{2\pi}
		\omega(p)
		\hat{a}(p)
		\hat{a}^\dagger(p)
		\int\dd{t_1}
		\chi^{(1)}(t_1)
		e^{-ipt_1}
		+
		\text{h.c.}
		\\
		&=
		\frac{L}{2}
		\int\frac{\dd{p}}{2\pi}
		\chi^{(1)}(p)
		\omega(p)
		\hat{a}(p)
		\hat{a}^\dagger(p)
		+
		\text{h.c.}
	\end{split}
\end{equation}
which can be solved exactly with the Magnus expansion.
The interaction Hamiltonian corresponds to a constant energy shift due to the polarization inside the dielectric.

In general, we follow the same approach as for the first-order susceptibility.
However, now we have to deal with nonlinearities.

The second-order polarization density operator for a lossless and time-invariant dielectric is
\begin{equation}
	\hat{P}_i^{(2)}(t,\vb{x})
	=
	\int\dd{t_1}
	\int\dd{t_2}
	\chi^{(2)}_{ijk}(t_1,t_2,\vb{x})
	\hat{E}^j(t-t_1,\vb{x})
	\hat{E}^k(t-t_2,\vb{x})
\end{equation}
which inserted into the second-order interaction Hamiltonian yields
\begin{equation}
	\hat{H}_\text{int}^{(2)}(t)
	=
	\frac{1}{2}
	\int\dd[3]{x}
	\int\dd{t_1}
	\int\dd{t_2}
	\chi^{(2)}_{ijk}(t_1,t_2,\vb{x})
	\hat{E}^i(t,\vb{x})
	\hat{E}^j(t-t_1,\vb{x})
	\hat{E}^k(t-t_2,\vb{x})
	.
\end{equation}
We assume the electric fields to be polarized along the same axis
\begin{equation}
	\hat{H}_\text{int}^{(2)}(t)
	\approx
	\frac{1}{2}
	\int\dd[3]{x}
	\int\dd{t_1}
	\int\dd{t_2}
	\chi^{(2)}(t_1,t_2,\vb{x})
	\hat{E}(t,\vb{x})
	\hat{E}(t-t_1,\vb{x})
	\hat{E}(t-t_2,\vb{x})
	.
\end{equation}
Furthermore, we expand the electric fields into positive and negative frequency parts and only keep the terms corresponding to frequency conversion
\begin{equation}
	\begin{split}
		\hat{H}_\text{int}^\text{FC}(t)
		&\approx
		\frac{1}{2}
		\int\dd[3]{x}
		\int\dd{t_1}
		\int\dd{t_2}
		\chi^{(2)}(t_1,t_2,\vb{x})
		\\
		&\times
		\hat{E}^{(+)}(t,\vb{x})
		\hat{E}^{(-)}(t-t_1,\vb{x})
		\hat{E}^{(-)}(t-t_2,\vb{x})
		+
		\text{h.c.}
	\end{split}
\end{equation}
dropping all other terms.
Performing the mode expansion, \cref{eq:electric_positive_operator,eq:electric_negative_operator},
\begin{equation}
	\begin{split}
		\hat{H}_\text{int}^\text{FC}(t)
		&=
		\frac{1}{2}
		\int\dd[3]{x}
		\int\dd{t_1}
		\int\dd{t_2}
		\chi^{(2)}(t_1,t_2,\vb{x})
		\\
		&\times
		\int\frac{\dd[3]{p}}{(2\pi)^3\sqrt{2\omega(\vb{p})}}
		\omega(\vb{p})
		\hat{a}^\dagger(\vb{p})
		e^{+i\omega(\vb{p})t-i\vb{p}\vdot\vb{x}}
		\\
		&\times
		\int\frac{\dd[3]{q_1}}{(2\pi)^3\sqrt{2\omega(\vb{q}_1)}}
		\omega(\vb{q}_1)
		\hat{a}(\vb{q}_1)
		e^{-i\omega(\vb{q}_1)(t-t_1)+i\vb{q}_1\vdot\vb{x}}
		\\
		&\times
		\int\frac{\dd[3]{q_2}}{(2\pi)^3\sqrt{2\omega(\vb{q}_2)}}
		\omega(\vb{q}_2)
		\hat{a}(\vb{q}_2)
		e^{-i\omega(\vb{q}_2)(t-t_2)+i\vb{q}_2\vdot\vb{x}}
		+
		\text{h.c.}
	\end{split}
\end{equation}
We neglect the transverse dynamics and reduce the Hamiltonian to one dimension along the propagation direction and assume the susceptibility to be uniform along the dielectric medium
\begin{equation}
	\begin{split}
		\hat{H}_\text{int}^\text{FC}(t)
		&=
		\frac{1}{4}
		\int\frac{\dd{p}}{2\pi}
		\int\frac{\dd{q_1}}{2\pi}
		\int\frac{\dd{q_2}}{2\pi}
		\sqrt{\frac{pq_1q_2}{2}}
		\int\dd{t_1}
		\int\dd{t_2}
		\chi^{(2)}(t_1,t_2)
		e^{+iq_1t_1}
		e^{+iq_2t_2}
		\\
		&\times
		\int_{-L/2}^{+L/2}\dd{x}
		e^{i(p-q_1-q_2)x}
		\hat{a}^\dagger(p)
		\hat{a}(q_1)
		\hat{a}(q_2)
		e^{+ipt}
		e^{-iq_1t}
		e^{-iq_2t}
		+
		\text{h.c.}
	\end{split}
\end{equation}
and we identify the Fourier transform of the susceptibility
\begin{equation}
	\begin{split}
		\hat{H}_\text{int}^\text{FC}(t)
		&=
		\frac{1}{4}
		\int\frac{\dd{p}}{2\pi}
		\int\frac{\dd{q_1}}{2\pi}
		\int\frac{\dd{q_2}}{2\pi}
		\sqrt{\frac{pq_1q_2}{2}}
		\chi^{(2)}(q_1,q_2)L
		\sinc\left(\frac{p-q_1-q_2}{L/2}\right)
		\\
		&\times
		\hat{a}^\dagger(p)
		\hat{a}(q_1)
		\hat{a}(q_2)
		e^{+ipt}
		e^{-iq_1t}
		e^{-iq_2t}
		+
		\text{h.c.}
	\end{split}
\end{equation}
We approximate the phase-matching function
\begin{equation}
	\sinc\left(\frac{p-q_1-q_2}{L/2}\right)
	\approx
	(2\pi)
	\delta^{(1)}(p-q_1-q_2)
\end{equation}
and simplify the interaction energy to
\begin{equation}
	\hat{H}_\text{int}^\text{FC}(t)
	=
	\int\frac{\dd{p}}{2\pi}
	\int\frac{\dd{q}}{2\pi}
	g(p,q)
	\hat{a}^\dagger(p)
	\hat{a}(q)
	\hat{a}(p-q)
	+
	\text{h.c.}
	.
\end{equation}
which agrees from the main characteristics with the result reported in Ref.~\cite[eq.~35]{Horoshko2018}.
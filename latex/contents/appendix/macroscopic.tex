\chapter{Nonlinear interaction theory}\label{app:macroscopic_interactions}

In the following, we summarize some results on nonlinear (quantum-)optics from Refs.~\cite{Murti2014,Boyd2020,QuesadaMejia2015,Mandel1995}, which are important for describing the linear electro-optic effect, essential for our investigation of the phase modulator.

We start by looking into the Pockels effect by investigating the interaction of the classical electric field with a dielectric exhibiting a nonlinear susceptibility.
The electric displacement field, in the frequency domain given by~\cite[p.~1070]{Mandel1995}
\begin{equation}
	D_i(\omega)
	=
	E_i(\omega)
	+
	P_i(\omega)
	,
	\label{eq:electric_displacement_field}
\end{equation}
accounts for macroscopic effects through the macroscopic polarization density $P_i$.
Although the Pockels effect causes a linear change of the refractive index by an external electric field, it is macroscopically described by the second-order macroscopic polarization density, in the time domain given by~\cite[p.~55]{Boyd2020}
\begin{equation}
	P_i^{(2)}(t)
	=
	\int\dd{t_1}
	\int\dd{t_2}
	\chi^{(2)}_{ijk}(t_1,t_2)
	E^j(t-t_1)
	E^k(t-t_2)
	\label{eq:polarization_density_so_time}
\end{equation}
wherein $\chi^{(2)}_{ijk}(t_1,t_2)$ is the second-order time-response tensor of the dielectric.
Inserting the Fourier transform of the electric field components, we find
\begin{equation}
	P_i^{(2)}(t)
	=
	\int\frac{\dd{\omega_1}}{2\pi}
	\int\frac{\dd{\omega_2}}{2\pi}
	\chi^{(2)}_{ijk}(\omega_1,\omega_2)
	E^j(\omega_1)
	E^k(\omega_2)
	e^{-i(\omega_1+\omega_2)t}
	\label{eq:polarization_density_so_time_fourier}
\end{equation}
wherein the second-order frequency-response tensor is the Fourier transform of the time-response tensor, i.e.,
\begin{equation}
	\chi^{(2)}_{ijk}(\omega_1,\omega_2)
	=
	\int\dd{t_1}
	\int\dd{t_2}
	\chi^{(2)}_{ijk}(t_1,t_2)
	e^{+i\omega_1t}
	e^{+i\omega_2t}
	.
\end{equation}
The Fourier transform of the second-order polarization density turns out to be
\begin{equation}
	\begin{split}
		P_i^{(2)}(\omega)
		&=
		\int\dd{t}
		P_i^{(2)}(t)
		e^{+i\omega t}
		\\
		&=
		\int\frac{\dd{\omega_1}}{2\pi}
		\int\frac{\dd{\omega_2}}{2\pi}
		\chi^{(2)}_{ijk}(\omega_1,\omega_2)
		E^j(\omega_1)
		E^k(\omega_2)
		\int\dd{t}
		e^{-i(\omega_1+\omega_2-\omega)t}
		\\
		&=
		\int\frac{\dd{\omega_1}}{2\pi}
		\int\frac{\dd{\omega_2}}{2\pi}
		\chi^{(2)}_{ijk}(\omega_1,\omega_2)
		E^j(\omega_1)
		E^k(\omega_2)
		(2\pi)
		\delta^{(1)}(\omega_2-\omega+\omega_1)
		\\
		&=
		\int\frac{\dd{\omega^\prime}}{2\pi}
		\chi^{(2)}_{ijk}(\omega^\prime,\omega-\omega^\prime)
		E^j(\omega^\prime)
		E^k(\omega-\omega^\prime)
		.
	\end{split}
	\label{eq:polarization_density_so_freq}
\end{equation}
The presence of the integral in \cref{eq:polarization_density_so_freq} breaks the linearity in the electric field $E^k$.
To restore linearity, the external electric field is assumed to be effectively static, simplifying the second-order polarization density to~\cite[p.~495]{Boyd2020}
\begin{equation}
	P_i^{(2)}(\omega)
	=
	\chi^{(2)}_{ijk}(0,\omega)
	E^j(0)
	E^k(\omega)
	.
	\label{eq:polarization_density_so_freq_pockels}
\end{equation}
Inserting the so linearized second-order polarization density and inserting it into \cref{eq:electric_displacement_field},
\begin{equation}
	\begin{split}
		D_i(\omega)
		&=
		E_i(\omega)
		+
		\chi^{(2)}_{ijk}(0,\omega)
		E^j(0)
		E^k(\omega)
		\\
		&=
		\left[
			\delta_{ik}
			+
			\chi^{(2)}_{ijk}(0,\omega)
			E^j(0)
		\right]
		E^k(\omega)
		,
	\end{split}
\end{equation}
we can read off the electric susceptibility tensor
\begin{equation}
	\varepsilon_{ik}(\omega)
	=
	\delta_{ik}
	+
	\chi^{(2)}_{ijk}(0,\omega)
	E^j(0)
	.
\end{equation}
The refractive-index tensor is defined as the square root of the electric susceptibility tensor~\cite[p.~3]{Brooker2003}
\begin{equation}
	n_{ij}(\omega)
	=
	\sqrt{\varepsilon_{ij}}
	\approx
	n_{ik}^{(0)}(\omega)
	+
	n_{ijk}^{(1)}(\omega)
	E^j(0)
	,
	\label{eq:refractive_index_expansion}
\end{equation}
where we performed a series expansion of the square root and introduced the refractive-index coefficients~\cite{Rerat2020}
\begin{align}
	n^{(0)}_{ik}(\omega)
	&=
	\sqrt{1+\chi_{ij}^{(1)}(\omega)}
	&
	n^{(1)}_{ijk}(\omega)
	&=
	\frac{\chi^{(2)}_{ijk}(\omega)}{2\sqrt{1+\chi^{(1)}_{ij}(\omega)}}
	.
\end{align}
With \cref{eq:refractive_index_expansion}, we have shown the linear change of the refractive index with an external electric field.
However, it was necessary to assume the external electric field to be effectively static in time to linearize the second-order macroscopic polarization density.

For a simple quantum treatment, we replace the classical fields with their respective free-field operator.
For instance, the quantum Hamiltonian of the electromagnetic field in a dielectric is~\cite[p.~124]{Jackson2007}
\begin{equation}
	\hat{H}
	=
	\frac{1}{2}
	\int\dd[3]{x}
	\left\{
		\hat{E}^i(t,\vb{x})
		\hat{D}_i(t,\vb{x})
		+
		\hat{B}^i(t,\vb{x})
		\hat{B}_i(t,\vb{x})
	\right\}
	,
\end{equation}
wherein $\hat{D}_i(t,\vb{x})$ is the electric displacement operator.
Expanding the electric displacement operator in terms of the macroscopic polarization-density operators, we find the interaction Hamiltonian to involve three electric fields, i.e.,
\begin{equation}
	\begin{split}
		\hat{H}_\text{int}^{(2)}(t)
		&=
		\frac{1}{2}
		\int\dd[3]{x}
		\hat{E}^i(t,\vb{x})
		\hat{P}_i^{(2)}(t,\vb{x})
		\\
		&=
		\frac{1}{2}
		\int\dd[3]{x}
		\int\dd{t_1}
		\int\dd{t_2}
		\chi^{(2)}_{ijk}(t_1,t_2,\vb{x})
		\hat{E}^i(t,\vb{x})
		\hat{E}^j(t-t_1,\vb{x})
		\hat{E}^k(t-t_2,\vb{x})
		\\
		&\approx
		\frac{1}{2}
		\int\dd{z}
		\int\dd{t_1}
		\int\dd{t_2}
		\chi^{(2)}(t_1,t_2,z)
		\hat{E}(t,z)
		\hat{E}(t-t_1,z)
		\hat{E}(t-t_2,z)
		,
	\end{split}
\end{equation}
where we assumed the electric fields to be polarized along the same axis in the last step and ignored the transverse mode profile.
Expanding the electric field operators into positive and negative frequency parts reveals the possible absorption and emission processes.
For frequency conversion, we drop all but the following terms
\begin{equation}
	\hat{H}_\text{int}^\text{FC}(t)
	\approx
	\frac{1}{2}
	\int\dd{z}
	\int\dd{t_1}
	\int\dd{t_2}
	\chi^{(2)}(t_1,t_2,z)
	\hat{E}^{(+)}(t,z)
	\hat{E}^{(-)}(t-t_1,z)
	\hat{E}^{(-)}(t-t_2,z)
	+
	\text{H.c.}
	.
\end{equation}
Inserting the plane-wave expansion of the positive- and negative-frequency electric-field operators, \cref{eq:electric_positive_operator,eq:electric_negative_operator},
\begin{equation}
	\begin{split}
		\hat{H}_\text{int}^\text{FC}(t)
		&=
		\frac{1}{2}
		\int\dd{z}
		\int\dd{t_1}
		\int\dd{t_2}
		\chi^{(2)}(t_1,t_2,z)
		\int\frac{\dd{\omega_1}}{2\pi}
		\int\frac{\dd{\omega_2}}{2\pi}
		\int\frac{\dd{\omega_3}}{2\pi}
		\omega_1
		\omega_2
		\omega_3
		\\
		&\qquad\times
		\hat{a}^\dagger(\omega_1)
		\hat{a}(\omega_2)
		\hat{a}(\omega_3)
		e^{+i\omega_1(t-z)}
		e^{-i\omega_2(t-t_1)+i\omega_2z}
		e^{-i\omega_3(t-t_2)+i\omega_3z}
		+
		\text{H.c.}
		\\
		&=
		\frac{1}{2}
		\int\frac{\dd{\omega_1}}{2\pi}
		\int\frac{\dd{\omega_2}}{2\pi}
		\int\frac{\dd{\omega_3}}{2\pi}
		\omega_1
		\omega_2
		\omega_3
		\int\dd{z}
		\chi^{(2)}(\omega_2,\omega_3,z)
		\\
		&\qquad\times
		\hat{a}^\dagger(\omega_1)
		\hat{a}(\omega_2)
		\hat{a}(\omega_3)
		e^{+i(\omega_1-\omega_2-\omega_3)t}
		e^{-i(\omega_1-\omega_2-\omega_3)z}
		+
		\text{H.c.}
		,
	\end{split}
\end{equation}
where we identified the Fourier transform of the second-order electric susceptibility
\begin{equation}
	\chi^{(2)}(\omega_2,\omega_3,z)
	=
	\int\dd{t_1}
	\int\dd{t_2}
	\chi^{(2)}(t_1,t_2,z)
	e^{+i\omega_2 t_1}
	e^{+i\omega_3 t_2}
	.
\end{equation}
We assume the second-order electric susceptibility to be constant over the interaction length $L$ and zero otherwise, then we can further simplify the frequency-conversion Hamiltonian to
\begin{equation}
	\hat{H}_\text{int}^\text{FC}(t)
	=
	\frac{1}{2}
	\int\frac{\dd{\omega_1}}{2\pi}
	\int\frac{\dd{\omega_2}}{2\pi}
	\int\frac{\dd{\omega_3}}{2\pi}
	g(\omega_1,\omega_2,\omega_3)
	\hat{a}^\dagger(\omega_1)
	\hat{a}(\omega_2)
	\hat{a}(\omega_3)
	e^{+i(\omega_1-\omega_2-\omega_3)t}
	+
	\text{H.c.}
	,
\end{equation}
where we introduced the phase-matching function
\begin{equation}
	g(\omega_1,\omega_2,\omega_3)
	=
	\omega_1
	\omega_2
	\omega_3
	\chi^{(2)}(\omega_2,\omega_3)
	\sinc\left(\frac{\omega_1-\omega_2-\omega_3}{L/2}\right)
	.
\end{equation}
Approximating the phase-matching function with a Delta distribution,
\begin{equation}
	g(\omega_1,\omega_2,\omega_3)
	\approx
	g(\omega_1,\omega_2)
	(2\pi))
	\delta^{(1)}(\omega_1-\omega_2-\omega_3)
	,
\end{equation}
the interaction Hamiltonian further simplifies to
\begin{equation}
	\hat{H}_\text{int}^\text{FC}
	=
	\frac{1}{2}
	\int\frac{\dd{\omega_1}}{2\pi}
	\int\frac{\dd{\omega_2}}{2\pi}
	g(\omega_1,\omega_2)
	\hat{a}^\dagger(\omega_1)
	\hat{a}(\omega_2)
	\hat{a}(\omega_1-\omega_2)
	+
	\text{H.c.}
	,
\end{equation}
which agrees in the most important characteristics with the result reported in Ref.~\cite[eq.~35]{Horoshko2018}.
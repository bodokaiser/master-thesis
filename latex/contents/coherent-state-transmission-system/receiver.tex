\section{Receiver}

The receiver extracts a symbol sequence $\beta[n]$ from the received coherent state $\ket{\beta(t)}$.

In contrast to the transmitter, the receiver is more complex:
On the one hand, the receiver must orient itself towards the sender.
In this sense, the sender has the freedom to define specific parameters that the recipient does not have.
In addition, the receiver needs to compensate for signal deterioration.
The increased complexity of the receiver leaves room for a variety of receiver designs.
For an overview of practical, coherent receiver designs, see Ref. 1~\cite{Kikuchi2016}.
Specific to \gls{cvqkd}, Ref.~\cite{Brunner2017} compares the single and dual quadrature intradyne and software-defined heterodyne receivers, see \Cref{tab:receivers}.
\begin{table}[htb]
  \centering
  \begin{tabular}{lccccc}
    \toprule
    Receiver design & Homodyne (single) & Homodyne (dual) & Heterodyne \\
    \midrule
    Balanced detectors & \num{1} & \num{2} & \num{1} \\
    Quadratures & \num{1} & \num{2} & \num{2} \\
    Optical complexity & Low & High & Low \\
    Signal bandwidth & High & High & Low \\
    \gls{lo} synchronization & Frequency and phase & Frequency & Bandwidth \\
    \bottomrule
  \end{tabular}
  \caption{Comparison of receiver implementations according to Ref.~\cite{Brunner2017}: The single quadrature homodyne detection offers low optical complexity and high bandwidth but only resolves one of two quadratures and required frequency and phase synchronization of the \gls{lo}. The dual quadrature homodyne detection resolves both quadratures with high bandwidth but requires two balanced detectors increasing the optical complexity and phase synchronization of the \gls{lo}. The heterodyne detection schemes resolves both quadratures with low complexity and no requirements on \gls{lo} synchronization at the cost of signal bandwidth.}\label{tab:receivers}
\end{table}
The single and dual quadrature homodyne receivers mix the received signal with the receiver \gls{lo} tuned to the carrier frequency whereas the heterodyne receivers tune the \gls{lo} to have an offset to the carrier frequency of $\omega_i$, the intermediate frequency.
The homodyne detection schemes are conceptional simpler as the detected signal directly corresponds to the transmitted baseband.
However, homodyne receivers require precise calibration of the \gls{lo} which turns out to be practically challenging.
In addition to the problem of \gls{lo} synchronization, is the optical simple homodyne design only able to resolve one quadrature.
The dual quadrature homodyne receiver requires twice the detectors to resolve both signal quadratures at the same time.
If we are willing to give up bandwidth - which is reasonable for \gls{cvqkd} - the heterodyne receiver moves the problem of synchronization to the digital domain and lets us resolve both quadratures while keeping the complexity low.
Furthermore, having most of the signal-processing to be software-defined makes protocol updates using the same hardware possible.
Finally, syncronization (of the \gls{lo}s) turns out to be more challenging using the homodyne receivers as these are highly sensitive to phase noise.
We therefore follow the conclusion of Ref.~\cite{Brunner2017} and limit the discussions of the receiver to that of a software-defined heterodyne kind.
\begin{figure}[htb]
	\centering
	\includestandalone{figures/tikz/receiver}
	\caption{Block diagram of a software-defined heterodyne receiver's signal flow: \gls{osp} is performed on a received coherent state $\ket{\beta(t)}$ yielding two coherent states $\ket{\beta_1(t)},\ket{\beta_2(t)}$ which are converted to two photocurrents $i_1(t),i_2(t)$ during photodetection. \gls{asp} transforms the photocurrents to the signal $\beta^\prime(t)$ which is sampled by a \gls{adc}. Finally, the symbols $\beta[n]$ are extracted from the samples $\beta^\prime[m]$ using \gls{dsp}.}\label{fig:receiver_signal_flow}
\end{figure}
\Cref{fig:receiver_signal_flow} illustrates the signal flow of a software-defined heterodyne receiver.
The \gls{osp} maps from the received coherent state $\ket{\beta(t)}$ to two coherent states $\ket{\beta_1(t)}$ and $\ket{\beta_2(t)}$ which represents the output of the mixing of $\ket{\beta(t)}$ with the \gls{lo}.
The signal encoded in the two coherent states $\ket{\beta_1(t)}$ and $\ket{\beta_2(t)}$ is passed onto two photocurrents $i_1(t),i_2(t)$ in the process of photodetection.
With the \gls{asp}, we convert the two photocurrents to a single analog signal $\beta^\prime(t)$ which we convert to a digital signal. $\beta^\prime[m]$ using a \gls{adc}.
Finally, we employ \gls{dsp} to extract the symbols $\beta[n]$ from the digital signal $\beta^\prime[m]$.

In the following, we divide the discussion of the software-defined heterodyne receiver into two stages:
The conversion of the optical to the digital signal and the symbol extraction.
For both these stages we will assume the receiver to be perfectly in sync with the transmitter to focus on the essential signal-processing.
In a last part, we give a rough overview of syncronization challenges and possible mitigations.

\subsection{Balanced detection and digital conversion}

The optical-to-digital conversion in the software-defined heterodyne receiver describes the conversion of a received coherent state $\ket{\beta(t)}$ to the digital samples $\beta^\prime[m]$.
In the optical domain, we superimpose the received optical signal with a \gls{lo} in an optical coupler or beam splitter and monitor the two outputs using two photodiodes as depicted in \Cref{fig:fiber_optical_heterodyne_detector}.
\begin{figure}[htb]
	\centering
	\includegraphics{figures/pstricks/balanced-detector}
	\caption{Fiber-optical balanced detector detector: First, the fiber output passes through an optical isolator to limit signal loss through back scattering. Second, the isolated optical signal is coupled with a \gls{lo} laser in a balanced optical coupler. Third, the coupler outputs are detected by  the balanced photodetector comprising the photodiodes PD1 and PD2.}\label{fig:fiber_optical_balanced_detector}
\end{figure}
The first photodiode PD1 outputs a photocurrent $i_1(t)$ while the second photodiode PD2 outputs a photocurrent $i_2(t)$.
The two photodiodes form together a balanced detector if one subtracts the photocurrents from another with the balanced detection signal being $\Delta i(t)=i_1(t)-i_2(t)$.
Using a \gls{tia}, the balanced current signal $\Delta i(t)$ is amplified and converted to a voltage signal which after \gls{lp} filtering we reference as $\beta^\prime(t)e^{-i\omega_i t}$.
We explicitly write out the intermediate frequency $\omega_i$ which is the difference between the transmitter's and receivers \gls{lo} such that $\beta^\prime(t)$ denotes the signal baseband.
The \gls{lp} is important to reduce noise from the \gls{tia} and to remove high-frequency components which cannot be resolved by the subsequent \gls{adc}.
Finally, the \gls{adc} outputs the samples $\beta^\prime(t)e^{-i\omega_it}$.
It is essential that the bandwidth of the \gls{adc} can fully resolve the intermediate frequency $\omega_i$ and the baseband bandwidth.
\begin{figure}[htb]
	\centering
	\includestandalone{figures/circuitikz/receiver-asp}
	\caption{Block diagram of the receiver's \gls{asp} and analog-to-digital conversion: The photodetectors output the photocurrents $i_1(t),i_2(t)$ and the balanced signal $\Delta i(t)=i_1(t)-i_2(t)$ is formed. A \gls{tia} converts the balanced current signal $\Delta i(t)$ a voltage signal which is filtered by a \gls{lp} filter to remove high-frequency noise from the \gls{tia} and reduce the bandwidth for the \gls{adc}. A \gls{adc} converts the filtered voltage signal $\beta^\prime(t)e^{-i\omega_it}$ to the digital samples $\beta^\prime[m]e^{-i\omega_it}$ completing the analog-to-digital conversion.}
\end{figure}
So far, we described the conversion from the baseband signal of the received coherent state from the optical over the analog to the digital domain for the software-defined heterodyne receiver.
By mixing the input signal with an \gls{lo} with offset $\omega_i>$ compared to the carrier frequency, we can move the quadrature information into the frequency domain while utilizing only a single balanced detector, keeping the electro-optical complexity low.

\FloatBarrier
\subsection{Digital signal processing}

Assuming perfect synchronization between the transmitter and the receiver, we are left to remove the intermediate frequency from the digital baseband, complete the pulse-shaping by applying the second \gls{rrc}, and perform downsampling to recover the symbols, see \Cref{fig:receiver_dsp}.
\begin{figure}[htb]
	\centering
	\includegraphics{figures/circuitikz/receiver-dsp}
	\caption{Block diagram of the receiver's \gls{dsp}: First, the intermediate frequency $\omega_i$ is removed from the samples $\beta^\prime[m]e^{-i\omega_it}$ by multiplying the time-domain samples with $e^{+i\omega_lt}$. Second, the samples $\beta^\prime[m]$ are downsampled to be compatible with the \gls{rrc} (matched) filter coefficients of the transmitter. Third, the output of the matched filter $\beta^\prime[l]$ is downsampled to reveal the symbols $\beta[n]$.}\label{fig:receiver_dsp}
\end{figure}
In the frequency domain, removing the intermediate frequency shifts recenters the spectrum, see the transition from the first to the second row in \Cref{fig:receiver_frequency}.
Applying the second \gls{rrc} filter increases the steepness at the band edges (third row) and downsampling recovers the original spectrum of the samples (last row) in \Cref{fig:receiver_frequency}.
\begin{figure}[htb]
	\centering
	\includestandalone[width=\textwidth]{figures/pgfplots/rx-frequency}
	\caption{Frequency domain of the receiver's \gls{dsp} for the complex random symbol sequence discussed for the transmitter: The spectrum of the \gls{adc} samples $\beta^\prime[m]e^{-i\omega_it}$ (first row) has a frequency offset equal to the intermediate frequency $\omega_i$ (first row). Down-conversion centers the spectrum (second row) by removing the intermediate frequency. Downsampling and applying the matched \gls{rrc} filter increases the steepness of the spectrum (third row). Finally, downsampling removes the steep band edges (last row) and reveals the symbols $\beta[n]$.}\label{fig:receiver_frequency}
\end{figure}
\Cref{fig:receiver_time} presents the same \gls{dsp} steps in the time domain applied to the samples.
\begin{figure}[htb]
	\centering
	\includestandalone[width=\textwidth]{figures/pgfplots/rx-rand-time}
	\caption{Time domain of the receiver's \gls{dsp} for the complex random symbol sequence discussed for the transmitter: The samples from the \gls{adc} $\beta^\prime[m]e^{-i\omega_it}$ (first row) are multiplied by the complex exponential $e^{+i\omega_it}$ to remove the intermediate frequency (second row). Then, the samples without the intermediate frequency $\beta^\prime[m]$ are downsampled for the matched \gls{rrc} filter (third row). Finally, the matched filtered samples $\beta[m]$ are downsampled to reveal the receiver's symbol sequence $\beta[n]$ (last row).}\label{fig:receiver_time}
\end{figure}

\FloatBarrier
\subsection{Syncronization}

% Citations:
\cite[p.~359]{Proakis2007} % Symbol timing estimation
\cite{Godard1978} % Godard algorithm for phase correction
\cite{Huang2015,Qi2015,Soh2015} % CV-QKD with pilot tone

So far we have assumed the transmitter and the receiver to be in sync to concentrate on the signal recover.
In practice, most of the receiver's \gls{dsp} is about compensating for synchronization errors.
On a hardware level, we need to account for the synchronization of two oscillators:
The lasers and the clocks of the \gls{dac} or \gls{adc}.
Usually, errors fall into the receiver's responsibility since the channel already introduces errors that have to be accounted for.

The transmitter \gls{lo} runs at the carrier frequency $\omega_t$ and the receivers \gls{lo} runs at $\omega_r$ with the difference between the two being the intermediate frequency $\omega_i=\abs{\omega_t-\omega_r}$.
Ideally, the frequency difference between the two \gls{lo}s is constant and there is no phase difference.
Practically, the laser frequency is subject to drifts due to thermal and electric effects.
Control loops can compensate for these effects to a certain extent without becoming too complex.
Concerning the software-defined heterodyne receiver, the frequency drifts should be much less than the target intermediate frequency, otherwise, the detector might not be able to capture the complete signal band.
The phase offset is a more challenging affair as it involves not only drifts but stochastic effects.
\begin{figure}[htb]
	\centering
	\includestandalone[width=\textwidth]{figures/pgfplots/laser-sync}
	\caption{Frequency spectrum of the transmitted and received pilot tone: The transmitted pilot tone (green) represents a perfect sinusoidal. The received pilot tone (orange) is broadened by phase variations and shifted by the offset between the transmitter and receiver \gls{lo}.}\label{fig:pilot_tones}
\end{figure}
To compensate for both effects, the transmitter adds a pilot tone to the transmitted signal which the receiver can use as a reference point.
\Cref{fig:pilot_tones} depicts the transmitted (green) and received (orange) pilot tone.
Comparing the received and transmitted pilot tones, we notice that the received pilot tone has a relative frequency offset due to the offset of the receiver's \gls{lo} and some significant broadening due to phase noise.
Using peak detection, the received pilot tone is used to determine the offset from the target intermediate frequency $\omega_i$.
To suppress the phase noise, a Wiener filter~\cite{Chen2006} is trained on the received phase noise and then applied to the signal spectrum.

The second set of oscillators which need synchronization are the clocks of the \gls{dac} and \gls{adc}.
Again, we assume the \gls{dac} to be the reference clock and the \gls{adc} to be the clock which needs compensation.
Because the \gls{adc} clock oscillates much slower as the optical laser, the relative frequency drifts can be neglected and we are mostly concerned with compensating for a timing offset.
The timing offset means that we sample the signal at times which do not intersect with the symbol times of the transmitter.
The situation is depicted in the first two rows of \Cref{fig:adc_sync} where the first row shows the transmitted signal and the second row the received signal.
The original symbol is located at the peak of the pulse-shape but for the received signal, we have two samples around the peak.
\begin{figure}[htb]
	\centering
	\includestandalone[width=\textwidth]{figures/pgfplots/adc-sync}
	\caption{Timing offset compensation of the \gls{adc} clock: The sampling at the receiver is delayed and not in sync with the symbols compared to the receiver.}\label{fig:adc_sync}
\end{figure}
From Fourier analysis, we know that a shift in time corresponds to an additional phase.
By performing a Fourier transform of the received samples, we can see how there is a linear increase of the phase with frequency, see third row in \Cref{fig:adc_sync}.
We can perform a linear fit and remove the phase shifts which corrects the sample offset.

Finally, on a protocol level, we need to synchronize the frames sent by the transmitter.
In particular, the receiver needs to detect the beginning of a frame and to assign a frame number.
The frame layout is depicted in \Cref{fig:frame}.
A single frame consists of a training and a data sequence.
The training sequence is the same for all frames and known by both the transmitter and the receiver such that one can correlate the known training sequence with the received sequence to find the start of a frame.
\begin{figure}[htb]
	\centering
	\includestandalone{figures/tikz/frame}
	\caption{Layout of the transmission frames: A frame comprises a training and data sequence. The training sequence is the same for all frames and used to detect the start of a frame and perform calibration for the data sequence which contains the actual transmission data.}\label{fig:frame}
\end{figure}
The additional identification of the frame number is performed using a classical service communication channel which is used to negotiate further calibration parameters between the receiver and the transmitter.

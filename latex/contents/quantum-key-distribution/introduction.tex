\chapter{Quantum-key distribution}\label{ch:qkd}

Before diving deep into the technical details of our quantum theory of light, we would like to introduce the reader to \gls{qkd} as an example of quantum optical communication.
In particular, we want to emphasize the many different layers, quantum and classical, involved in practical quantum optical communication.
Our introduction favors breadth over depth and ignores protocol-specific details to highlight the similarities, specifically between \gls{dvqkd} and \gls{cvqkd}.

The literature divides \gls{qkd} protocols among \gls{dvqkd} and \gls{cvqkd} and \gls{dpsqkd}, though this thesis will not address \gls{dpsqkd} further.
One of the few resources painting a comprehensive and decisive picture of \gls{qkd}, and providing much inspiration for the present chapter, is Ref.~\cite{Wolf2021}.
Other notable references are Ref.~\cite{Diamanti2016}, reviewing the practical aspects of \gls{qkd}, and Refs.~\cite{Duvsek2006,Gisin2002} for \gls{cvqkd}.
\gls{dvqkd} is often approached in the context of Gaussian quantum information theory, see Refs.~\cite{Weedbrook2012,Ferraro2005}.
More advanced resources highlighting the practical implementation of \gls{qkd} are found in Refs.~\cite{Scarani2009,Fung2010,Laudenbach2018}.
Compared to the existing literature, our introduction to \gls{qkd} attempts to weaken the distinction between \gls{dvqkd} and \gls{cvqkd} by framing it as an encoding detail.

The chapter organizes as follows. 
First, we motivate the challenge of secure and practical key distribution in the context of secure communication.
Second, we present the key concepts of protocols based on qubit and bosonic quantum information.
Third, we discuss the classical post-processing procedure required to distill a shared secret key.
Fourth, we provide rough ideas on how to perform a security analysis of \gls{qkd}.
Finally, we argue for the concept of a logical and encoding quantum layer for practical \gls{qkd}.
One of the most relevant conclusions to draw from this chapter is that a coherent-state transmission system resembles the encoding quantum layer for \gls{cvqkd} protocols.
\section{Secret-key distribution problem}

Two spatially distanced parties, Alice and Bob, share a communication channel that allows Alice to send messages to Bob.
How can Alice and Bob secure their communication against an opposing third party, Eve?
Alice needs to ensure that the message she transmits is confidential, i.e., only Bob can read it.
Bob needs to ensure that the message he receives is integer, i.e., only Alice can have sent it.
\begin{figure}[htb]
	\centering
	\includegraphics{figures/tikz/secure-communication}
	\caption{Secure communication system comprising a transmitter (Alice), a receiver (Bob) and a public communication channel. Alice encrypts a plaintext message yielding a ciphertext. By adding the hash to the ciphertext, Alice constructs a message she transmits through the channel to Bob. Bob removes the hash from his received message to resolve the ciphertext. Decrypting the ciphertext unveils the plaintext for Bob.}\label{fig:secure_communication}
\end{figure}
\Cref{fig:secure_communication} depicts a secure communication system implementing integrity and confidentiality between Alice and Bob.\footnote{For a discussion on the order of hashing and encryption, see Ref.~\cite{Kohno2003,Krawczyk2001,Bellare2000}.}
First, Alice encrypts a plaintext message, using symmetric encryption like the \gls{otp}~\cite{Shannon1949} or the more practical \gls{aes}~\cite{Daemen1999}, to ensure confidentiality.
Second, she adds a \gls{mac} using, e.g., universal hash functions~\cite{Carter1979}, to ensure integrity.
Bob receives the encrypted message with \gls{mac} from Alice.
He confirms the integrity of the message by checking the \gls{mac}, then he decrypts the message to access the plaintext.

Message authentication and cipher require Alice and Bob to possess a shared secret key.
If Alice and Bob use the \gls{otp} cipher and their secret key is truly random, the communication system is eternally secure~\cite{Shannon1949} --- provided that Alice and Bob do not re-use their secret key.
Unless Alice and Bob do not want to meet in person every time they initiate communication, Alice and Bob need a practical method to distribute a secret key.

\subsection{Public-key distribution}

The standard attempt to solve the key-distribution problem is to use an asymmetric cipher comprising a public key for encryption and a private key for decryption.
One of the parties, for example, Bob, first generates an asymmetric key pair of which he discloses the public key to Alice.
Alice generates a secret key from a true random generator, encrypts it with the public key, and sends it to Bob, see the left-hand side of \Cref{fig:pkd_system}.
Bob decrypts the message he received from Alice with his private key to obtain Alice's secret key, see the right-hand side of \Cref{fig:pkd_system}.
Assuming the asymmetric cipher to be secure, Alice and Bob now share a secret key.
\begin{figure}[htb]
	\centering
	\includegraphics{figures/tikz/pkd-system}
	\caption{Alices and Bob use an asymmetric cipher to distribute a secret key. Alice possesses the secret key and a public key, and Bob owns a private key corresponding to Alice's public key. Alice encrypts the secret key with the public key and transmits the message over a (public) channel to Bob. Bob decrypts the message with the private key to obtain the secret key.}\label{fig:pkd_system}
\end{figure}
Public-key distribution algorithms, for instance, Diffie-Hellman~\cite{Diffie1976} and variants thereof, e.g., \gls{ecdh}, are heavily employed on the internet because they are effortless to deploy.

The core principle behind asymmetric ciphers is the concept of one-way functions, functions easy to compute but difficult to invert.
Here, easy and difficult refer to the computational complexity, denoting the upper bound of the best (known) algorithm to solve the problem.
The time to break an asymmetric cipher depends on the computational resources and complexity.
In practice, one chooses a key length, such that attacks, possible with present computational resources, become impractical.
\begin{figure}[htb]
	\centering
	\includegraphics{figures/pgfplots/runtime}
	\caption{Computational runtimes for prime number factorization algorithms: The most efficient known classical algorithm is the \gls{gnfs}~\cite{Lenstra1993} (green) and Shor's algorithm~\cite{Shor1994} (orange). The runtime of the classical algorithm increases exponentially with the number of bits, while the quantum algorithm rises almost linearly.}
	\label{fig:prime_number_factorization_runtime}
\end{figure}
However, computational resources advance with time.\footnote{According to Moore's observation, the number of transistors in integrated circuits doubles every two years.}
A key length that was considered secure in the past might be rendered useless in the future.
It is imaginable to store communication from the present in the hope of breaking it in the future.
In addition to technological progress in computing, discovering new algorithms may decrease the computational complexity and make certain attacks practicable.
For example, Shor's quantum algorithm~\cite{Shor1994} provides an exponential speed-up in prime number factorization compared to the fastest (known) classical algorithm, \gls{gnfs}, see \Cref{fig:prime_number_factorization_runtime}.
As prime number factorization is used by, for instance, Diffie-Hellman key exchange, as a one-way function, a sufficiently-powerful quantum computer can break previously, assumed to be, secure communication.

As long as there exists no mathematical proof for a theoretical lower bound of the computational complexity of a particular class of one-way functions, PKD is not forward secure, i.e., might be broken in the future.\footnote{Post-quantum cryptography~\cite{Bernstein2017,Chen2016} attempts to mitigate the vulnerability of present asymmetric ciphers against quantum algorithms by choosing a different class of one-way functions suspected to be secure. But again, as long as there is no absolute lower bound for the computational complexity of a particular class of one-way functions, forward security of PKD, including post-quantum ciphers, remains contestable.}

\FloatBarrier
\subsection{Quantum-key distribution}

The potential of \gls{pkd} becoming a vulnerability inflames the demand for a practical and forward secure key distribution scheme.
Ideally, the key-distribution scheme should dynamically react to any third-party interaction by compensating for the information leak or aborting the protocol.
\gls{qkd} claims to be such a key-distribution scheme.
Based on the laws of quantum physics, \gls{qkd} exploits the inherent uncertainty of measuring non-orthogonal quantum states to generate random correlations between Alice and Bob.
Alice's and Bob's correlations provide insights about potential information loss to a third party.
Using classical techniques, Alice and Bob can then compensate for the potential information loss or abort the protocol if necessary.
\begin{figure}[htb]
	\centering
	\includegraphics{figures/tikz/qkd-system}
	\caption{Alice and Bob use a quantum and classical channel to generate a shared secret-key. Alice possesses a transmitter, and Bob owns a receiver. Transmitter and receiver both output a secret key and connect to a quantum and classical channel. A potential adversary, Eve, has read and write access to the quantum but only read access to the classical channel.}\label{fig:qkd_system}
\end{figure}
\Cref{fig:qkd_system} presents a communication system proficient for \gls{qkd}.
The classical and quantum channel connecting Bob's receiver with Alice's transmitter is usually the same physical medium.
For example, an optical fiber where the quantum and classical channels occupy different wavelengths or polarization.
If the \gls{qkd} protocol succeeds, Alice's transmitter and Bob's receiver output the same secret key to Alice, respectively Bob.
The adversary, Eve, has full access (read and write) to the quantum channel but has only read access to the classical channel.
The restriction that Eve has only read access over the classical channel is important to exclude man-in-the-middle attacks.
However, the restriction is not a practical limitation as we can ensure integrity by promoting the classical channel to an authenticated channel using \glspl{mac}.

The details on \gls{qkd} strongly depend on the particular implementation at hand, the \gls{qkd} protocol.
A huge zoo of \gls{qkd} protocols exists, and it is useful to overview common features for a systematic presentation.
The \gls{qkd} literature typically distinguishes between \gls{dvqkd}, \gls{cvqkd}, and \gls{dpsqkd}.
Leaving out \gls{dpsqkd}, it is unclear which features unambiguously differentiate between \gls{cvqkd} and \gls{dvqkd}.
For instance, most practical \gls{dvqkd} protocols use weak coherent states~\cite{Duvsek2006}, which are anything but discrete.
The accepted opinion considers a protocol discrete when using a single photon and continuous when using a coherent detector.
However, this view is challenged by discrete \gls{qkd} protocols, like BB84~\cite{Bennett1984}, using coherent detection~\cite{Qi2021}.
What other feature can we use if the detection method does not clearly distinguish between \gls{cvqkd} and \gls{dvqkd}?
\begin{figure}[htb]
	\centering
	\includegraphics[scale=0.8]{figures/tikz/qkd-classification}
	\caption{Common characteristics of \gls{qkd} protocols in a tree diagram. The protocol schema is either \gls{pandm} or \gls{eb}. The Measurement basis selection is either passive or active. The detection is usually coherent or based on single-photon clicks. The logical Hilbert space is either finite or countable. Among many, the physical encoding uses the field quadratures, polarization, or squeezing \glspl{dof} of light.}\label{fig:qkd_classification}
\end{figure}
\Cref{fig:qkd_classification} summarizes common features among \gls{qkd} protocols.
Features like the protocol schema, the measurement basis selection, and the detector are uniquely determined.
More opaque is the distinction between the physical encoding and the logical Hilbert space, a concept which we introduce here.
The physical encoding refers to the light \glspl{dof} used to encode the data, for example, the polarization of light or the particular quantum state.
More often, the technical facilities determine the physical encoding.
For example, coherent light sources and detectors are mature technologies.
At the same time, many \gls{qkd} protocols assume a more simple quantum system than that of light, which we refer to as the logical Hilbert space.
The logical Hilbert space is either finite or countable.
If it is finite, it is often a qubit or generalization thereof.
If it is countable, it is often bosonic.
Therefore, we propose not to partition \gls{qkd} protocols among \gls{cvqkd} or \gls{dvqkd} but by their logical Hilbert space being bosonic- or qubit-based\footnote{Such a distinction is also indicated in quantum information theory, see, for instance, Ref.~\cite[p.~2]{Weedbrook2012}.}, which we both present in the next two sections.
\section{Secret-key distribution problem}

Two spatially distanced parties, Alice and Bob, share a communication channel that allows Alice to send messages to Bob.
How can Alice and Bob secure their communication against an opposing third party, Eve?
Alice needs to ensure that the message she transmits is confidential, i.e., only Bob can read it.
Bob needs to ensure that the message he receives is integer, i.e., only Alice can have sent it.
\begin{figure}[htb]
	\centering
	\includegraphics{figures/tikz/secure-communication}
	\caption{Secure communication system comprising a transmitter (Alice), a receiver (Bob) and a public communication channel: Alice encrypts a plaintext message yielding a ciphertext. By adding the hash to the ciphertext, Alice constructs a message she transmits through the channel to Bob. Bob removes the hash from his received message to resolve the ciphertext. Decrypting the ciphertext unveils the plaintext for Bob.}\label{fig:secure_communication}
\end{figure}
\Cref{fig:secure_communication} depicts a secure communication system implementing integrity and confidentiality between Alice and Bob.\footnote{For a discussion on the order of hashing and encryption, see Ref.~\cite{Kohno2003,Krawczyk2001,Bellare2000}.}
First, Alice encrypts a plaintext message, using symmetric encryption like the \gls{otp}~\cite{Shannon1949} or the more practical \gls{aes}~\cite{Daemen1999}, to ensure confidentiality.
Second, she adds a \gls{mac} using, e.g., universal hash functions~\cite{Carter1979}, to ensure integrity.
Bob receives the encrypted message with \gls{mac} from Alice.
He confirms the integrity of the message by checking the \gls{mac}, then he decrypts the message to access the plaintext.

Message authentication and cipher require Alice and Bob to possess a shared secret key.
If Alice and Bob use the \gls{otp} cipher and their secret key is truly random, the communication system is eternally secure~\cite{Shannon1949} --- provided that Alice and Bob do not re-use their secret key.
Unless Alice and Bob do not want to meet in person every time they initiate communication, Alice and Bob need a practical method to distribute a secret key.

\subsection{Public-key distribution}

The standard attempt to solve the key distribution problem is to use an asymmetric cipher comprising a public key for encryption and a private key for decryption.
One of the parties, for example, Bob, first generates an asymmetric key pair of which he discloses the public key to Alice.
Alice generates a secret key from a true random generator, encrypts it with the public key, and sends it to Bob, see the left-hand side of \Cref{fig:pkd_system}.
Bob decrypts the message he received from Alice with his private key to obtain Alice's secret key, see the right-hand side of \Cref{fig:pkd_system}.
Assuming the asymmetric cipher to be secure, Alice and Bob now share a secret key.
\begin{figure}[htb]
	\centering
	\includegraphics{figures/tikz/pkd-system}
	\caption{\Gls{pkd} system used to create a shared key between between two spatially distanced parties, Alice and Bob.}\label{fig:pkd_system}
\end{figure}
Public-key distribution algorithms, for instance, Diffie-Hellman~\cite{Diffie1976} and variants thereof, e.g., \gls{ecdh}, are heavily employed on the internet because they are effortless to deploy.

The core principle behind asymmetric ciphers is the concept of one-way functions, functions easy to compute but difficult to invert.
Here, easy and difficult refer to the computational complexity, denoting the upper bound of the best (known) algorithm to solve the problem.
The time to break an asymmetric cipher depends on the computational resources and complexity.
In practice, Alice and Bob can increase the key length, such that any attacks by Eve become impractical.
\begin{figure}[htb]
	\centering
	\includegraphics{figures/pgfplots/runtime}
	\caption{Computational runtimes for prime number factorization algorithms: The most efficient known classical algorithm is the \gls{gnfs}~\cite{Lenstra1993} and Shor's algorithm~\cite{Shor1994}.}
	\label{fig:prime_number_factorization_runtime}
\end{figure}
However, computational resources increase with technological progress (Moore's law).
At the same time, computational complexity might decrease with the discovery of new algorithms.
A key length that in the past was considered secure might be rendered useless in the future.
A particular example of a reduction in computational complexity is Shor's algorithm~\cite{Shor1994}.
Compared to the fastest classical algorithm, \gls{gnfs}, Shor's algorithm provides an exponential speed-up, see \Cref{fig:prime_number_factorization_runtime}, in prime number factorization, a particular class of one-way functions used in, for instance, Diffie-Hellman key exchange.


Even though, post-quantum algorithms have been proposed~\cite{Bernstein2017,Chen2016} using a different class of one-way functions, the   where no efficient (quantum) algorithm is (yet) known.
As long as there is no proof for the computational complexity for a specific family of one-way functions, public-key distribution is only safe for key lengths that are not computationally feasible.
However, what is not computationally feasible today might be tomorrow, and it is conceivable that encrypted communication recorded today will be broken in the feature with specialized hardware.
Therefore, public-key distribution is not considered to be forward secure.

\FloatBarrier
\subsection{Quantum-key distribution}

The chance of public-key distribution becoming a critical vulnerability in secure communication infrastructure inflamed a demand for a practical and information-theoretical provable alternative key distribution.
\gls{qkd} claims to be such a practical and information-theoretical secure key distribution.
In summary, \gls{qkd} exploits the inherent uncertainty in measuring non-orthogonal quantum states for shared random number generation:
\begin{enumerate}
	\item Alice samples some random variables $\alpha_1,\dots,\alpha_n$ and encodes them onto a quantum state $\ket{\psi(t)}$.
	\item Bob performs measurements of the quantum state $\ket{\psi(t)}$ yielding $\beta_1,\dots,\beta_n$.
	\item Alice's $\alpha_1,\dots,\alpha_n$ and Bob's $\beta_1,\dots,\beta_n$ are correlated random variables from which a shared secret key can be distilled using classical algorithms (post-processing).
\end{enumerate}
On a more practical level, a typical \gls{qkd} system comprises a transmitter, a receiver, a quantum and classical communication channel as illustrated in \Cref{fig:qkd_system}.
\begin{figure}[htb]
	\centering
	\includestandalone{figures/tikz/qkd-system}
	\caption{\Gls{qkd} system used to create a shared key between between two spatially distanced parties, Alice and Bob, secret to a third party, Eve. Eve has full access to the quantum channel but can only read information from the classical authenticated channel.}\label{fig:qkd_system}
\end{figure}
A potential adverse eavesdropper, Eve, has full access to the quantum channel, i.e., can intercept and temper with quantum states sent by Alice, but has only read access to the classical channel.
More precisely, the classical channel is an authenticated channel to prevent man-in-the-middle attacks.
Authentication can be implemented classically using \gls{mac} as discussed for the secure communication system.
% assumptions to Eve and what kind of attacks she can do (coherent, ?)
% explain superoperator (effect of quantum channel)

% entanglement- vs. prepare-and-measure -> EB used for proofs but otherwise PM used practical
% comparison cv/dv on a quantum level (possible quantum states) and on a practical level (fiber, free-space) -> CV and DV depends on state space of the receiver
\begin{table}[htb]
	\centering
	\begin{tabular}{lll}
		\toprule
		Encoding variable & State space & Information support \\
		\midrule
		Polarization & Finite & Polarization state \\
		Photon number & Countable & Number state \\
		Quadratures & Uncountable & Coherent state \\
		Squeezing & Uncountable & Squeezed state \\
		Time-bin & Finite & Time of arrival \\
		Phase-bin & Finite & Phase \\
		\bottomrule
	\end{tabular}
	\caption{Encoding variables to use for \gls{qkd}: Variables with uncountable state space are considered for \gls{cvqkd} while variables with (possibly restricted) countable state space are considered for \gls{dvqkd}.}
\end{table}

Before diving into specific \gls{qkd} protocols, we introduce a taxonomy of \gls{qkd} protocols.
By identifying common protocol features, allows for a reductionist model of \gls{qkd}.

The literature typically distinguishes between \gls{dvqkd}, \gls{cvqkd}, and \gls{dpsqkd}.
Ignoring \gls{dpsqkd}, it is unclear what exact features unambiguously differentiate between \gls{cvqkd} and \gls{dvqkd}.
For instance, most practical \gls{dvqkd} protocols use weak coherent states~\cite{Duvsek2006}, which are anything but discrete.
Fr this reason, the accepted opinion considers a protocol discrete when using a single-photon and continuous when using a coherent detector.
This view has been recently challenged by proposing a BB84-like protocol using coherent detection~\cite{Qi2021}.
\begin{figure}[htb]
	\centering
	\includestandalone[scale=0.8]{figures/tikz/qkd-classification}
	\caption{Common features among \gls{qkd} protocols: Detection, physical encoding, logical state space, measurement basis selection and schema.}\label{fig:qkd_classification}
\end{figure}
We propose a more subtle classification based on definite protocol features.
\Cref{fig:qkd_classification} illustrates common features we identified among \gls{qkd} protocols that can be uniquely determined.
Many of these features are implementation details such as physical encoding, measurement basis selection, and detection.
In contrast, other features are more fundamental, like the logical Hilbert space or the schema.
We will discuss and exemplify these features in the next two sections.

In the next two sections, we present \gls{qkd} protocols that differ by their logical Hilbert space.
The logical Hilbert space defines the quantum system assumed on an abstract protocol level, e.g., a qubit or boson, and is a more precise concept than discrete and continuous~\cite[p.~2]{Weedbrook2012}.
The logical quantum system does not need to be equal to the quantum system physically encoding the quantum states, for example, a weak coherent state mimicking a single-photon.

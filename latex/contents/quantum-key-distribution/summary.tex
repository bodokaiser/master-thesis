\section*{Summary}
\addcontentsline{toc}{section}{Summary}

We introduced \gls{qkd} as an example of quantum-optical communication and a mechanism for practical and secure key-distribution, which, together with classical symmetric ciphers, enables secure communication, including means to estimate information leakage.
To keep track of the diversity of \gls{qkd} protocols, we restricted our treatment to \gls{cvqkd} and \gls{cvqkd} protocols.
In an attempt to formalize the notion of \gls{cvqkd} and \gls{cvqkd}, we proposed the concept of logical and encoding quantum system, which suggests itself, when comparing theoretical with practical \gls{qkd} transmission systems as illustrated in \Cref{fig:qkd_protocol}.
\begin{figure}[htb]
	\centering
	\includegraphics{figures/tikz/qkd-protocol}
	\caption{Block diagram comparing a theoretical with a practical \gls{qkd} transmission system. The theoretical \gls{qkd} transmission system comprises a classical information source, a prepared qubit or boson state, a quantum channel, a received qubit or boson state, and a classical information source. Compared to the theoretical \gls{qkd} transmission system, the practical \gls{qkd} transmission system prepares and receives a light instead of a qubit or boson state.}\label{fig:qkd_protocol}
\end{figure}
The idea here is that from an idealistic viewpoint, we encode classical information into a qubit or boson state, while for any practical implementation, we use light states for encoding.
\Cref{tab:qkd_comparison} summarizes the differences between qubit- and boson-based \gls{qkd} for which discussed particular encoding schemes like the time-phase-encoding BB84 for qubit-based \gls{qkd} or the coherent-encoding GG02 for boson-based \gls{qkd}.
\begin{table}[htb]
	\centering	
	\begin{tabular}{lcc}
		\toprule
			& Qubit-based & Boson-based \\
		\midrule
			Visualization & Bloch sphere & Phase space \\
			Hilbert space (dim) & Finite (two) & Countable (infinite) \\
			Measurement operator & $\vb{\hat{S}}(\vb{n})=\hat{S}_in^i$ & $\hat{X}(\vartheta)=\frac{1}{\sqrt{2}}\left(\hat{a}e^{-i\vartheta}+\hat{a}^\dagger e^{+i\vartheta}\right)$ \\
			Standard basis & $\left\{\ket{0},\ket{1}\right\}$ & $\left\{\ket{x},\ket{p}\colon x,p\in\mathbb{R}\right\}$ \\
		\bottomrule
	\end{tabular}
	\caption{Comparison of qubit- and boson-based \gls{qkd} protocols.}\label{tab:qkd_comparison}
\end{table}
Although security has been proven individually for theoretical and practical \gls{qkd} protocols, the larger Hilbert space of light makes a unitary mapping between qubit or boson states and light states problematic~\cite{Chefles2000}, requiring further investigations.
The technological maturity of coherent communication makes it highly practical to implement qubit- and boson-based using weak coherent states.
\begin{figure}[htb]
	\centering
	\includegraphics{figures/tikz/cvqkd-protocol}
	\caption{Block diagram comparing a theoretical with a practical coherent-state transmission system. The theoretical coherent-state transmission system comprises a classical information source, a prepared tensor-product coherent state, a quantum channel, a received tensor-product coherent state, and a classical information source. Compared to the theoretical coherent-state transmission system, the practical coherent-state transmission system continuously-modulates and -demodulates a (continuous-time) coherent state.}\label{fig:cvqkd_protocol}
\end{figure}
However, the concept of a coherent-state transmission system opens up a new gap between theory and practice.
While quantum information theory assumes the coherent states to be independent, i.e., the transmission sequence involves a tensor-product of (bosonic) coherent states, do we know from communication engineering that we need to consider continuously-modulated coherent states.
In any case, we have shown that practical \gls{qkd} implementations are very different from their original theoretical protocol, demanding an abstraction for these encoding details.

Back to our general treatment of \gls{qkd}, we compiled classical methods to distill a shared secret-key between the transmitter and receiver, known as classical post-processing, from the quantum-transmission data.
The classical post-processing maps the discrete or continuous data from the transmission sequence to binary symbols, corrects errors, discards failed data blocks, and removes information from the partially secret-key using privacy amplification.
Finally, we roughly outlined some ideas for the security analysis of \gls{qkd}.
\section*{Summary}
\addcontentsline{toc}{section}{Summary}

In the present chapter, we introduced \gls{qkd} as an example for quantum optical communication and as a mechanism for practical and secure key distribution, which, together with classical symmetric ciphers, enables secure communication with means to estimate information leakage.
We then analyzed the quantum transmission phase of qubit- and boson-based \gls{qkd} protocols generating correlated information between the receiver and transmitter.
\begin{table}[htb]
	\centering	
	\begin{tabular}{lcc}
		\toprule
			& Qubit-based & Boson-based \\
		\midrule
			Visualization & Bloch sphere & Phase space \\
			Hilbert space (dim) & Finite (two) & Countable (infinite) \\
			Measurement operator & $\vb{\hat{S}}(\vb{n})=\hat{S}_in^i$ & $\hat{X}(\vartheta)=\frac{1}{\sqrt{2}}\left(\hat{a}e^{-i\vartheta}+\hat{a}^\dagger e^{+i\vartheta}\right)$ \\
			Standard basis & $\left\{\ket{0},\ket{1}\right\}$ & $\left\{\ket{x},\ket{p}\colon x,p\in\mathbb{R}\right\}$ \\
		\bottomrule
	\end{tabular}
	\caption{Comparison of qubit- and boson-based \gls{qkd} protocols.}\label{tab:qkd_comparison}
\end{table}
By introducing the concept of qubit- and boson-based \gls{qkd} protocols, with their respective key properties summarized in \Cref{tab:qkd_comparison}, we formalized the concept of \gls{dvqkd} and \gls{cvqkd}.
Additionally, we formulated the concept of a logical and an encoding quantum system allowing us to encode qubits onto number or coherent states, which might share some similarity with the concept of symbols and pulse-shaping in classical signal processing.
Because the technology to prepare and measure coherent states is highly mature, most practical \gls{qkd} implementations, regardless of qubit- or boson-based, transmit weak coherent states.
\begin{figure}[htb]
	\centering
	\includegraphics{figures/tikz/qkd-protocol}
	\caption{An abstract \gls{qkd} protocol comprises a binary encoder, a logical quantum system, a binary decoder, and some post-processing. The binary encoder maps bits onto a quantum state of the logical quantum system. The binary decoder extracts the bits from the logical quantum system. The logical quantum system is a subspace of a larger physical quantum system. The state encoder and decoder map between the logical and physical quantum states.}\label{fig:qkd_protocol}
\end{figure}
Given the correlated data from the quantum transmission, we compiled classical methods to distill a shared secret key between the transmitter and receiver, known as classical post-processing. 
The classical post-processing maps the discrete or continuous data from the transmission sequence to binary symbols, corrects errors, discards failed data blocks, and removes information from the partially secret key using privacy amplification.
Finally, we roughly outlined some ideas for the security analysis of \gls{qkd}.

The concept of a logical and en encoding quantum layer in \gls{qkd} needs further investigation but might open up new more general protocols and simplify security proofs.
Concerning our thesis, our investigations suggest developing our theoretical framework for quantum optical communication towards a coherent-state transmission system.
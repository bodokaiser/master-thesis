\section*{Abstract}

Quantum optical communication exploits the unique quantum properties of light to challenge the frontier of established communication.

As a discipline that emerged from telecommunication and quantum optics, quantum optical communication lacks a unified theoretical framework incorporating the quantum properties into traditional signal processing theory.
For example, typical quantum optics has no notion of a frequency spectrum but uses a single-mode, a single frequency mode.
The present thesis attempts such a unified theoretical framework.

Since a theory without an experiment is of limited use, we develop our theory together with a quantum description of real-world \gls{cvqkd} prototypes.
\gls{cvqkd} utilizes both state-of-the-art signal-processing and quantum (information) theory, making it an ideal testbed for our proposal.

We approach our theoretical framework in three steps: First, we present and compare \acrshort{qkd} protocols, introducing the abstract concepts of qubit- and boson-based \acrshort{qkd} protocols.
Second, we recap the quantum theory of the light, including the momentum spectrum, which turns out to be the frequency spectrum essential for signal processing.
Third, we apply and test our quantum theory to describe a coherent state transmission system used for \gls{cvqkd}.

While our theory is mostly equivalent to a continuous-mode generalization of quantum optics, it is, to the best of our knowledge, the first attempt to justify and discuss the explicit form of such a generalization.
Applied to \gls{cvqkd}, we found a self-contained description that the respective domain experts can agree on, allowing for a future transfer of methods between quantum optics and telecommunication.

\section*{Abstract}

Quantum optical communication combines quantum aspects of light and communication engineering to bring forth novel approaches to communication.
One of these approaches is quantum-key distribution, enabling practical and secure key generation.
As a fledgling discipline, quantum optical communication lacks a unified theoretical framework to which both communication engineers and quantum physicists agree.
Single-mode quantum optics, the standard framework used to describe the quantum properties of light, lacks the notion of a continuous spectrum essential for communication theory.

The present thesis aims to deliver the missing theoretical framework for quantum optical communication by reviewing the quantum description of a coherent state transmission system implementing QKD in three steps:
First, we present and compare QKD protocols and argue why practical QKD based on weak coherent states effectively represents a coherent state transmission system.
Second, we derive a continuous-mode quantum theory of light rooted in quantum field theory and apply it to describe the building blocks of a coherent state transmission system.
Third, we assemble the previously derived building blocks to a coherent state transmission system and compare it to a software-defined implementation of a coherent transmission system while emphasizing the signal-processing aspects.

Our continuous-mode quantum theory of light is compatible with few references on continuous-mode quantum optics but is more transparent in the underlying assumptions.
Applied to quantum optical communication, we motivated a generalized quadrature operator, which accounts for measurements of particular frequency bands.
Furthermore, we showed that the electro-optical IQ modulator and balanced detector implement up- and down-conversion of classical signal-processing.

Applied to CV-QKD, we found a self-contained description that the respective domain experts can agree on, allowing for a future transfer of methods between quantum optics and telecommunication.
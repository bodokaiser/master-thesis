\section*{Notation}

We now give a brief summary of how the notation emphasizes the different mathematical objects used in this thesis.
We choose to adopt a notation commonly used in the quantum field theory literature~\cite{Weinberg1995,Peskin1995}.
On the one hand, we work with the same diverse set of mathematical objects, e.g., functionals, operators, tensors, as quantum field theory.
On the other hand, we can easily compare key results with results from quantum field theory.

\begin{align*}
	f(t)
	&=
	\int_{\mathbb{R}}\frac{\dd{\omega}}{2\pi}
	f(\omega)
	e^{+i\omega t}
	&
	f(\omega)
	&=
	\int_{\mathbb{R}}\dd{t}
	f(t)
	e^{-i\omega t}
\end{align*}
\begin{align}
	f(\vb{x})
	&=
	\int_{\mathbb{R}^3}\frac{\dd[3]{p}}{(2\pi)^3}
	f(\vb{p})
	e^{-i\vb{p}\vdot\vb{x}}
	&
	f(\vb{p})
	&=
	\int_{\mathbb{R}^3}\dd[3]{x}
	f(\vb{x})
	e^{+i\vb{p}\vdot\vb{x}}
\end{align}

% vectors (three vs four-dimensional)
% tensors
% unit vectors, oprators, vector-valued operators
% 

% Minkowski space, four vectors
% why we use p instead of omega for modes -> to distinguish between frequency and momentum


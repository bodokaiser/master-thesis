\section*{Notation}

We mostly adopt the mathematical notation from popular quantum field-theory books, e.g. Refs.~\cite{Weinberg1995,Peskin1995}.

Throughout the thesis, we exclusively use the natural unit system, where the natural constants, i.e., speed of light $c$, reduced Planck constant $\hbar$, electric charge $e$, electron mass $m_e$, dielectric constant $\varepsilon_0$, are set to one, significantly reducing notational clutter.
If required, one can restore the SI units by dimensional analysis.

If not explicitly stated, the integration domain covers the $n$-dimensional real numbers, $\mathbb{R}^n$.

For the temporal Fourier transform, we choose the convention
\begin{align*}
	f(t)
	&=
	\int\frac{\dd{\omega}}{2\pi}
	f(\omega)
	e^{+i\omega t}
	&
	f(\omega)
	&=
	\int\dd{t}
	f(t)
	e^{-i\omega t}
	.
\end{align*}
For the spatial Fourier transform, we choose the convention
\begin{align*}
	f(\vb{x})
	&=
	\int\frac{\dd[3]{p}}{(2\pi)^3}
	f(\vb{p})
	e^{-i\vb{p}\vdot\vb{x}}
	&
	f(\vb{p})
	&=
	\int\dd[3]{x}
	f(\vb{x})
	e^{+i\vb{p}\vdot\vb{x}}
	.
\end{align*}
The four-dimensional (spacetime), Fourier transform follows from the combined temporal and spatial Fourier transform
\begin{align*}
	f(t,\vb{x})
	&=
	\int\frac{\dd[4]{p}}{(2\pi)^4}
	f(p_0,\vb{p})
	e^{+ip_0 t-i\vb{p}\vdot\vb{x}}
	&
	f(p_0,\vb{p})
	&=
	\int\dd[4]{x}
	f(t,\vb{x})
	e^{-ip_0 t+i\vb{p}\vdot\vb{x}}
\end{align*}
where we identify $p_0$ with the energy $\omega$.
We denote the convolution operator with $\conv$, i.e,
\begin{equation*}
	\left(f\conv g\right)(t)
	=
	\int\dd{t^\prime}
	f(t^\prime)
	g(t-t^\prime)
	=
	\int\frac{\dd{\omega}}{2\pi}
	f(\omega)
	g(\omega)
	e^{+i\omega t}
\end{equation*}
in the frequency and
\begin{equation*}
	\left(f\conv g\right)(\omega)
	=
	\int\frac{\dd{\omega^\prime}}{2\pi}
	f(\omega^\prime)
	g(\omega-\omega^\prime)
	=
	\int\dd{t}
	f(t)
	g(t)
	e^{-i\omega t}
\end{equation*}
in the time domain.

To become proficient with the Minkowski metric and tensors, we recommend the study of Ref.~\cite{Carroll1997}.
Three-dimensional vectors are denoted by boldface, i.e.,
\begin{equation*}
	\vb{a}
	=
	\begin{pmatrix}
		a^1 \\
		a^2 \\
		a^3
	\end{pmatrix}
	.
\end{equation*}
Sometimes, we express vectors as linear combinations of unit vectors, e.g.,
\begin{equation*}
	\vb{a}
	=
	a^i
	\vu{e}_i
	=
	\sum_{i=1}^3
	a^i
	\vu{e}_i
\end{equation*}
wherein we used the Einstein summation convention, summing over a pair of lower and upper indices, named "contraction".
Three-dimensional vector components carry a latin index, e.g., $i,j,k,l$.
Four-dimensional vector components carry a greek index, e.g., $\mu,\nu,\rho$.
Four-dimensional vectors are denoted without boldface, i.e.,
\begin{equation*}
	a
	=
	a^\mu
	\vu{e}_\mu
	=
	\begin{pmatrix}
		a^0 \\
		\vb{a}
	\end{pmatrix}
	=
	\begin{pmatrix}
		a^0 \\
		a^1 \\
		a^2 \\
		a^3
	\end{pmatrix}
\end{equation*}
and we refer to the zeroth component $a^0$ as the time component and the other components $a^i$ as the spatial components.
It is common practice to refer to a vector by its component, i.e., $a^\mu$ refers to the four-dimensional vector $a$.

For the Minkowski metric $g^{\mu\nu}$ we adopt the "mostly minus" convention, i.e.,
\begin{equation*}
	g^{\mu\nu}
	=
	\begin{pmatrix}
		g^{00} & g^{01} & g^{02} & g^{03} \\
		g^{10} & g^{11} & g^{12} & g^{13} \\
		g^{20} & g^{21} & g^{22} & g^{23} \\
		g^{30} & g^{31} & g^{32} & g^{33}
	\end{pmatrix}
	=
	\begin{pmatrix}
		+1 & 0 & 0 & 0 \\
		0 & -1 & 0 & 0 \\
		0 & 0 & -1 & 0 \\
		0 & 0 & 0 & -1
	\end{pmatrix}
	.
\end{equation*}
The Minkowski product given two four-dimensional vectors, $a^\mu$ and $b^\mu$, is given by the contraction
\begin{equation*}
	a^\mu g_{\mu\nu} b^\nu
	=
	a^0 b^0
	-
	\vb{a}\vdot\vb{b}
	=
	a^0 b^0
	-
	a_i b^i
\end{equation*}
wherein $\vb{a}\vdot\vb{b}$ is the scalar product on Euclidean space.
The Minkowski metric can be used to raise and lower indices,
\begin{align*}
	a_\mu
	=
	g_{\mu\nu}
	a^\nu
	=
	\begin{pmatrix}
		a^0 \\
		-a^1 \\
		-a^2 \\
		-a^3 \\
	\end{pmatrix}
	,
\end{align*}
and a spatial component with a lower index is not in general equal to a spatial component with a raised index, $a_i\neq a^i$!

With regard to quantum mechanics, we use the standard bra-ket notation where we denote scalar-valued operators by a hat, e.g., $\hat{X}$, and vector-valued operators by boldface hat, e.g., $\vu{X}$.

Analog to the continuous-time signal $x(t)$, we define the discrete-time signal
\begin{equation*}
	x[n]
	=
	\int\dd{t}
	x(t)
	\delta^{(1)}(t-nT)
	,
\end{equation*}
wherein $T$ is the sampling period.
Sometimes we refer to $x[n]$ as samples or a sample similar as $x(t)$ denotes a function or the function evaluated at $t$.
To distinguish samples from symbols, which do not require a sampling period $T$, we use the index notation, i.e., $x_m$ denotes a symbol while $x[m]$ denotes a sample.

Regarding operator ordering, we note the normal-ordering symbol, moving the creation operators, e.g., $\hat{a}^\dagger$, to the left and the annihilation operators, e.g., $\hat{a}$, to the right, by
\begin{equation*}
	\norder{\hat{a}\hat{a}^\dagger\hat{b}}
	=
	\hat{a}^\dagger
	\hat{a}
	\hat{b}
	.
\end{equation*}
As time-ordering symbol we adopt $\mathcal{T}_+$, which evaluates an operator in forward time-order~\cite[p.~84]{Peskin1995},
\begin{equation*}
	\frac{1}{2}
	\hat{T}_+
	\int_{t_0}^t\dd{t_1}
	\int_{t_0}^t\dd{t_2}
	\hat{A}(t_1)
	\hat{B}(t_2)
	=
	\int_{t_0}^t\dd{t_1}
	\int_{t_0}^{t_1}\dd{t_2}
	\hat{A}(t_1)
	\hat{B}(t_2)
	.
\end{equation*}
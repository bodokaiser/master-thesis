\section{Encoder}

Alice prepares a tensor product of $N$ coherent states
\begin{equation}
	\bigotimes_{n=1}^N
	\ket{\alpha_n}_n
	.
\end{equation}
where $\alpha_1,\dots,\alpha_N\in\mathbb{C}$ are sampled from a complex normal distribution.

What unitary transformations does Alice need to apply to obtain the time-continuous coherent state?

First, she applies a local time-evolution operator $\hat{U}_n\left(t_0,t_0+nT\right)$ to each coherent state from some initial time $t_0$ to $t_0+t-nT$ where $n$ refers to the state index, i.e.,
\begin{equation}
	\begin{split}
		\bigotimes_{n=1}^N
		\ket{\alpha_n(t-nT)}_n
		&=
		\hat{U}_T(t_0;t_0+t-T,\dots,t_0+t-NT)
		\left(
			\bigotimes_{n=1}^N
			\ket{\alpha_n(t)}_n
		\right)
		\\
		&=
		\bigotimes_{n=1}^N
		\hat{U}_{T,n}(t_0,t_0+t-nT)
		\ket{\alpha_n(t_0)}_n
		.
	\end{split}
\end{equation}
We interleave the states using a symmetric beam splitter transform, see Ref.~\cite{Zukowski1997} and project the first output
\begin{equation}
	\begin{split}
		\ket{\frac{1}{\sqrt{N}}\sum_{n=1}^N\alpha_n(t-nT)}
		&=
		\hat{P}_1
		\hat{U}_{BS}
		\left(
			\bigotimes_{n=1}^N
			\ket{\alpha_n(t-nT)}_n
		\right)
		.
	\end{split}
\end{equation}
We apply a second beam splitter transform to perform optical filtering
\begin{equation}
	\begin{split}
		\hat{U}
		\ket{\frac{1}{\sqrt{N}}\sum_{n=1}^N\alpha_n(t-nT)}_a
		\otimes
		\ket{0}_B
	\end{split}
\end{equation}

\subsection{Pulse-shaping}

Pulse-shaping turns a sequence of (complex) symbols into a time-continuous signal.
For practical implementations, the pulse-shaping is performed in the digital domain.
Moving the pulse-shaping to the optical domain allows for a quantum mechanical description.

\begin{figure}[htb]
	\centering
    \includestandalone{figures/pstricks/quantum-pulse-shaping}
    \caption{Fiber-optical setup to perform pulse-shaping in the (quantum) optical domain for three symbols. Each of the three symbols is independently \gls{iq}-modulated on a pulsed laser with increasing time delay. The pulses are coupled together and optically filtered before being passed to a fiber.}\label{fig:quantum_pulse_shaping}
\end{figure}
\Cref{fig:quantum_pulse_shaping} presents a fiber-optical setup implementing pulse-shaping in the optical domain for three symbols.
The generalization to $n$ symbols is straightforward by adding more \gls{iq}-modulated pulsed-lasers with a time delay to the coupler and is discussed in the next paragraphs.
First, we need to produce independent pulses directly encoding the symbols, symbol pulses, by \gls{iq}-modulating pulsed lasers.
Second, we add a time delay $nT$ to each symbol pulse equal to the symbol period $T$ multiplied by the symbol index $n$.
Third, we superimpose the time-delayed symbol pulses through an optical coupler yielding a symbol pulse train.
Assuming the symbol pulses to be strongly peaked in the time domain, we can approximate the symbol pulse train as a sequence of Dirac pulses.
Pulse-shaping occurs when the symbol pulse train passes an optical filter.
We can model the optical filter as a frequency-dependent beam splitter with one vacuum input mode and one output mode dumped.

Let us present the quantum states and transformations relevant to \Cref{fig:quantum_pulse_shaping}.
The symbol pulse is a continuous-mode coherent state of the form
\begin{equation}
	\ket{\alpha(t)}
	=
	e^{-\overline{n}/2}
	\exp\left\{
		\int_0^\infty\dd{p}
		\alpha(p)
		e^{+ipt}
		\hat{a}^\dagger(p)
	\right\}
	\ket{0}
\end{equation}
wherein $\overline{n}$ is the mean photon number and $\alpha(p)$ is some pulse-shape.
For instance, a Gaussian pulse-shape centered at frequency $p_0$ with spread $\sigma_p$ would take the form
\begin{equation}
	\alpha(p)
	\propto
	\exp\left\{
		-\frac{1}{4}\left(\frac{p-p_0}{\sigma_p}\right)^2
	\right\}
\end{equation}
while for a Lorentzian pulse-shape takes the form
\begin{equation}
	\alpha(p)
	\propto
	\frac{1}{\gamma}
	\frac{1}{1+\left(\frac{p-p_0}{\gamma}\right)^2}
\end{equation}
wherein $\gamma$ is a scale parameter.
Many other pulse-shapes are possible and we do not want to limit ourselves to a particular pulse-shape.
However, we require the pulse-shape $\alpha(p)$ to approach a delta distribution in the limit of an infinitely narrow pulse-shape.

Adding a time delay $nT$ is equivalent to time-reversal or inverted time-evolution, i.e.,
\begin{equation}
	\hat{U}(t-nT,t)
	\ket{\alpha(t)}
	=
	\ket{\alpha(t-nT)}.
\end{equation}
The total quantum state is the tensor product of the individual symbol pulse states
\begin{equation}
	\ket{\alpha_1(t-T)}
	\otimes
	\ket{\alpha_2(t-2T)}
	\otimes
	\dots
	\otimes
	\ket{\alpha_N(t-NT)}
\end{equation}
where each symbol pulse is time delayed by the symbol period $T$ multiplied by the symbol index $n$.
According to Ref.~\cite{Zukowski1997}, a symmetric multiport beam splitter has matrix elements
\begin{equation}
	U_{mn}
	=
	\frac{1}{\sqrt{N}}
	\left(e^{i2\pi/N}\right)^{(n-1)(m-1)}
\end{equation}
and the first output mode of the multiport beam splitter provided the tensor product of symbol pulses is the symbol pulse train
\begin{equation}
	\ket{\alpha_1^\prime(t)}
	=
	\ket{\sum_{n=1}^NU_{1n}\alpha_n(t-nT)}
	=
	\ket{\frac{1}{\sqrt{N}}\sum_{n=1}^N\alpha_n(t-nT)}
	.
\end{equation}
We decompose the spectrum of the symbol pulse train into
\begin{equation}
	\alpha_1^\prime(p)
	=
	f(p)g(p)
	.
\end{equation}
The first factor $f(p)$ denotes the pulse-shape and the second factor $g(p)$ encodes the symbols
\begin{equation}
	g(p)
	=
	\frac{1}{\sqrt{N}}
	\sum_{n=1}^N
	\alpha_n
	e^{-ipnT}
	.
\end{equation}

Finally, optical filtering is equivalent to applying a unitary beam splitter transform with the second input mode in vacuum, i.e.,
\begin{equation}
	\hat{U}\ket{\alpha_1^\prime(t),0}
	=
	e^{-\overline{n}_1^\prime/2}
	\exp\left\{
		\int_0^\infty\dd{p}
		\alpha_1^\prime(p)
		e^{+ipt}
		\hat{U}
		\hat{a}^\dagger(p)
		\hat{U}^\dagger
	\right\}
	\ket{0,0}
\end{equation}
Taking only the first output mode, we find
\begin{equation}
	\ket{\alpha_1^{\prime\prime}}
	=
	e^{-\overline{n}_1^{\prime\prime}/2}
	\exp\left\{
		\int_0^\infty\dd{p}
		\alpha_1^{\prime\prime}(p)
		e^{+ipt}
		\hat{a}^\dagger(p)
	\right\}
	\ket{0}
\end{equation}
wherein
\begin{equation}
	\alpha_1^{\prime\prime}(p)
	=
	\cos\left(\theta(p)/2\right)
	\alpha_1^\prime(p)
	=
	\cos\left(\theta(p)/2\right)
	f(p)g(p)
	.
\end{equation}
We define
\begin{equation}
	h(p)
	=
	f(p)
	\cos\left(\theta(p)/2\right)
\end{equation}
and use the convolution theorem to write
\begin{equation}
	\ket{\alpha_1^{\prime\prime}}
	=
	\ket{\left(h*g\right)(t)}
	.
\end{equation}

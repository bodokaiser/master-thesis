\FloatBarrier
\section{Receiver}

The receiver extracts a symbol sequence $\beta[n]$ from the received coherent state $\ket{\beta(t)}$.

In contrast to the transmitter, the receiver is more complex:
On the one hand, the receiver must orient itself towards the sender.
In this sense, the sender has the freedom to define specific parameters that the recipient does not have.
In addition, the receiver needs to compensate for signal deterioration.
The increased complexity of the receiver leaves room for a variety of receiver designs.
For an overview of practical, coherent receiver designs, see Ref. 1~\cite{Kikuchi2016}.
Specific to \gls{cvqkd}, Ref.~\cite{Brunner2017} compares the single and dual quadrature intradyne and software-defined heterodyne receivers, see \Cref{tab:receivers}.
\begin{table}[htb]
  \centering
  \begin{tabular}{lcccc}
    \toprule
      Receiver & Detectors & Quadratures & Complexity & Bandwidth \\
    \midrule
      Single quadrature intradyne & \num{1} & \num{1} & Low & High \\
      Dual quadrature intradyne & \num{2} & \num{2} & High & High \\
      Software-defined heterodyne & \num{1} & \num{2} & Low & Low \\
    \bottomrule
  \end{tabular}
  \caption{Comparison of receiver implementations according to Ref.~\cite{Brunner2017}: The two intradyne detection schemes refer to specific cases of homodyne detection. While the single quadrature intradyne detection offers low (electro-optical) complexity and high bandwidth it only resolves one of two quadratures. The dual quadrature intradyne detection resolves both quadratures and has high bandwidth but requires two balanced detectors increasing the complexity. The software-defined heterodyne detection schemes resolves both quadratures with low complexity at the cost of bandwidth.}\label{tab:receivers}
\end{table}
The single and dual quadrature intradyne receivers use homodyne detection, i.e., mixing the received signal with a \gls{lo} such that the intermediate frequency is zero $\omega_i=0$ where the heterodyne receiver uses a non-zero intermedia frequency.
While the homodyne detection schemes are conceptional simpler in the sense that the detection result directly corresponds to the transmitted baseband, resolving both quadratures is only possible at the expense of increased electro-optical complexity.
If we are willing to give up bandwidth - which is reasonable for \gls{cvqkd} - a software-defined heterodyne receiver lets us resolve both quadratures while keeping the complexity low.
Furthermore, having most of the signal-processing to be software-defined makes protocol updates using the same hardware possible.
We therefore follow the conclusion of Ref.~\cite{Brunner2017} and limit the discussions of the receiver to that of a software-defined heterodyne kind.
\begin{figure}[htb]
	\centering
	\includestandalone{figures/tikz/receiver}
	\caption{Block diagram of a software-defined heterodyne receiver's signal flow: \gls{osp} is performed on a received coherent state $\ket{\beta(t)}$ yielding two coherent states $\ket{\beta_1(t)},\ket{\beta_2(t)}$ which are converted to two photocurrents $i_1(t),i_2(t)$ during photodetection. \gls{asp} transforms the photocurrents to the signal $\beta^\prime(t)$ which is sampled by a \gls{adc}. Finally, the symbols $\beta[n]$ are extracted from the samples $\beta^\prime[m]$ using \gls{dsp}.}\label{fig:receiver_signal_flow}
\end{figure}
\Cref{fig:receiver_signal_flow} illustrates the signal flow of a software-defined heterodyne receiver.
The \gls{osp} maps from the received coherent state $\ket{\beta(t)}$ to two coherent states $\ket{\beta_1(t)}$ and $\ket{\beta_2(t)}$ which represents the output of the mixing of $\ket{\beta(t)}$ with the \gls{lo}.
The signal encoded in the two coherent states $\ket{\beta_1(t)}$ and $\ket{\beta_2(t)}$ is passed onto two photocurrents $i_1(t),i_2(t)$ in the process of photodetection.
With the \gls{asp}, we convert the two photocurrents to a single analog signal $\beta^\prime(t)$ which we convert to a digital signal. $\beta^\prime[m]$ using a \gls{adc}.
Finally, we employ \gls{dsp} to extract the symbols $\beta[n]$ from the digital signal $\beta^\prime[m]$.

\subsection{Optical and analog-to-digital conversion}

\begin{figure}[htb]
	\centering
	\includegraphics{figures/pstricks/receiver}
	\caption{Fiber-optical heterodyne detector: First, the fiber output passes through an optical isolator to limit signal loss through back scattering. Second, the isolated optical signal is coupled with a \gls{lo} laser in a balanced optical coupler. Third, the coupler outputs are detected by  the balanced photodetector comprising the photodiodes PD1 and PD2.}
\end{figure}

\begin{figure}[htb]
	\centering
	\includestandalone{figures/circuitikz/receiver-asp}
	\caption{Block diagram of the receiver's \gls{asp} and analog-to-digital conversion: The photodetectors output the photocurrents $i_1(t),i_2(t)$ and the balanced signal $\Delta i(t)=i_1(t)-i_2(t)$ is formed. A \gls{tia} converts the balanced current signal $\Delta i(t)$ a voltage signal which is filtered by a \gls{lp} filter to remove high-frequency noise from the \gls{tia} and reduce the bandwidth for the \gls{adc}. A \gls{adc} converts the filtered voltage signal $\beta^\prime(t)e^{-i\omega_it}$ to the digital samples $\beta^\prime[m]e^{-i\omega_it}$ completing the analog-to-digital conversion.}
\end{figure}

\FloatBarrier
\subsection{Digital signal processing}

% what happens here?

\begin{figure}[htb]
	\centering
	\includegraphics{figures/circuitikz/receiver-dsp}
	\caption{Block diagram of the receiver's \gls{dsp}: First, the intermediate frequency $\omega_i$ is removed from the samples $\beta^\prime[m]e^{-i\omega_it}$ by multiplying the time-domain samples with $e^{+i\omega_lt}$. Second, the samples $\beta^\prime[m]$ are downsampled to be compatible with the \gls{rrc} (matched) filter coefficients of the transmitter. Third, the output of the matched filter $\beta^\prime[l]$ is downsampled to reveal the symbols $\beta[n]$.}
\end{figure}

\begin{figure}[htb]
	\centering
	\includestandalone[width=\textwidth]{figures/pgfplots/rx-frequency}
	\caption{Frequency domain of the receiver's \gls{dsp} for the complex random symbol sequence discussed for the transmitter: The spectrum of the \gls{adc} samples $\beta^\prime[m]e^{-i\omega_it}$ (first row) has a frequency offset equal to the intermediate frequency $\omega_i$ (first row). Down-conversion centers the spectrum (second row) by removing the intermediate frequency. Downsampling and applying the matched \gls{rrc} filter increases the steepness of the spectrum (third row). Finally, downsampling removes the steep band edges (last row) and reveals the symbols $\beta[n]$.}
\end{figure}

\begin{figure}[htb]
	\centering
	\includestandalone[width=\textwidth]{figures/pgfplots/rx-rand-time}
	\caption{Time domain of the receiver's \gls{dsp} for the complex random symbol sequence discussed for the transmitter: The samples from the \gls{adc} $\beta^\prime[m]e^{-i\omega_it}$ (first row) are multiplied by the complex exponential $e^{+i\omega_it}$ to remove the intermediate frequency (second row). Then, the samples without the intermediate frequency $\beta^\prime[m]$ are downsampled for the matched \gls{rrc} filter (third row). Finally, the matched filtered samples $\beta[m]$ are downsampled to reveal the receiver's symbol sequence $\beta[n]$ (last row).}
\end{figure}

\FloatBarrier
\subsection{Syncronization}

% Citations:
% Symbol timing estimation Proakis p. 359 \cite[p.~359]{Proakis2007}
% Precise Noise Calibration for CV-QKD (https://ieeexplore.ieee.org/abstract/document/9203405) \cite{Brunner2020}
% Derivation of RRC filter / proof that it is optimal for ISI and matched-filter for AWGN channel \cite{Cubukcu2012}
% Godard algorithm (D.N.Godard. Passband timing recovery in an all-digital modem receiver. 1978) \cite{Godard1978}
% Wiener filter \cite{Chen2006}

So far we have assumed perfect syncronization between the transmitter and the receiver.
In practice, however, the clocks in the transmitter and the receiver are independent and only accurate to finite precision.
In the following, we discuss different schemes employed to correct for syncronization errors.

The most important syncronization concerns the transmitter's and receiver's \gls{lo}:
\begin{figure}[htb]
	\centering
	\includestandalone[width=\textwidth]{figures/pgfplots/laser-sync}
	\caption{Frequency spectrum of the transmitted and received pilot tone: The transmitted pilot tone (green) represents a perfect sinusoidal. The received pilot tone (orange) is broadened by phase variations and shifted by the offset between the transmitter and receiver \gls{lo}.}
\end{figure}
\begin{figure}[htb]
	\centering
	\includestandalone[width=\textwidth]{figures/pgfplots/adc-sync}
	\caption{Timing offset compensation of the \gls{adc} clock: The sampling at the receiver is delayed and not in sync with the symbols compared to the receiver.}
\end{figure}

\begin{figure}[htb]
	\centering
	\includestandalone{figures/tikz/frame}
	\caption{Layout of the transmission frames: A frame comprises a training and data sequence. The training sequence is the same for all frames and used to detect the start of a frame and perform calibration for the data sequence which contains the actual transmission data.}
\end{figure}

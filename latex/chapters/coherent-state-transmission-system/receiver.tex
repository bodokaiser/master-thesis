\section{Receiver}

% compare receiver architectures and motivation for low-complexity receiver
% See Ref.~\cite{Kikuchi2016} for introduction to coherent architecture
% See Ref.~\cite{Brunner2017} for coherent receiver architectures for CV-QKD

\begin{figure}[htb]
	\centering
	\includestandalone{figures/tikz/receiver}
	\caption{Block diagram showing the signal flow across the receiver: }\label{fig:receiver}
\end{figure}

\subsection{Optical and analog-to-digital conversion}

\begin{figure}[htb]
	\centering
	\includegraphics{figures/pstricks/receiver}
	\caption{Fiber-optical coherent detector: The fiber output passes through an optical isolator and is coupled with a \gls{lo} laser in a balanced optical coupler. The coupler outputs are monitored by the photodiodes PD1 and PD2 in balanced photodetection configuration.}
\end{figure}

\begin{figure}[htb]
	\centering
	\includestandalone{figures/circuitikz/receiver-asp}
	\caption{Block diagram of the receiver's \gls{asp} and analog-to-digital conversion: The photodetectors output the photocurrents $i_1(t),i_2(t)$ and the difference $\Delta i(t)=i_1(t)-i_2(t)$ is created. A \gls{tia} converts the current signal to $\Delta i(t)$ a voltage signal which is filtered by a \gls{lp} filter to remove high-frequency noise from the \gls{tia} and reduce the bandwidth to half the sampling rate of the \gls{adc} to satisfy the Nyquist criteria. The signal $\beta^{\prime\prime}(t)$ is then sampled by the \gls{adc} and yields the samples $\beta^{\prime\prime}[m]$.}
\end{figure}

\FloatBarrier
\subsection{Digital signal processing}

% what happens here?

\begin{figure}[htb]
	\centering
	\includegraphics{figures/circuitikz/receiver-dsp}
	\caption{Block diagram of the receiver's \gls{dsp}: First, the intermediate frequency is removed from $\beta^{\prime\prime}[m]$ by multiplying the time-domain samples with the \gls{lo} frequency $e^{+i\omega_lt}$. Second, the samples $\beta^\prime[m]$ are downsampled to be compatible with the \gls{rrc} (matched) filter coefficients of the transmitter. Finally, the output of the matched filter $\beta^\prime[l]$ is downsampled to reveal the symbols $\beta[n]$.}
\end{figure}

\begin{figure}[htb]
	\centering
	\includestandalone[width=\textwidth]{figures/pgfplots/rx-rand-time}
	\caption{Time domain plot of the receiver's \gls{dsp} for the complex random symbol sequence discussed for the transmitter: The samples from the \gls{adc} (first row) are multiplied by the complex exponential $e^{+i\omega_it}$ to remove the intermediate frequency (second row). The samples without the intermediate frequency are downsampled to the same sampling rate of the transmitter and filtered by the. matched \gls{rrc} (third row). Finally, the matched filtered samples are downsampled to reveal the receiver's symbol sequence $\beta[n]$ (last row).}
\end{figure}

\begin{figure}[htb]
	\centering
	\includestandalone[width=\textwidth]{figures/pgfplots/rx-frequency}
	\caption{Power spectral density plot of the receiver's \gls{dsp} for the complex random symbol sequence discussed for the transmitter: The spectrum from the \gls{adc} samples is shifted by the intermediate frequency $\omega_i$ (first row). Down-conversion centers the spectrum (second row). Downsampling and applying the matched \gls{rrc} filter increases the steepness of the spectrum (third row). Downsampling to the symbols removes the steep band edges (last row).}
\end{figure}

\FloatBarrier
\subsection{Syncronization}

% Citations:
% Proakis p. 359
% Precise Noise Calibration for CV-QKD (https://ieeexplore.ieee.org/abstract/document/9203405)
% Derivation of RRC filter / proof that it is optimal for ISI and matched-filter for AWGN channel
% Godard algorithm (D.N.Godard. Passband timing recovery in an all-digital modem receiver. 1978)
% Wiener filter

So far we have assumed perfect syncronization between the transmitter and the receiver.
In practice, however, the clocks in the transmitter and the receiver are independent and only accurate to finite precision.
In the following, we discuss different schemes employed to correct for syncronization errors.

The most important syncronization concerns the transmitter's and receiver's \gls{lo}:
\begin{figure}[htb]
	\centering
	\includestandalone[width=\textwidth]{figures/pgfplots/laser-sync}
	\caption{Power spectral density of the transmitted and received pilot tone.}
\end{figure}

\begin{figure}[htb]
	\centering
	\includestandalone{figures/tikz/frame}
	\caption{Transmission frames.}
\end{figure}

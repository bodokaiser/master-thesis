\section{Classical transmitter}                                                                                                                                                                                                                                                                                                                                                                                                                                                                                                                                                                                                                                                                                                                     

We introduced the transmitter in the coherent state transmission system model as a device that encodes symbols $\alpha[n]$ onto a coherent state $\ket{\alpha(t)}$.
The far-reaching significance of electronic processing in our present state-of-the-art strongly suggests the symbols $\alpha[n]$ to be electrically encoded while the coherent state resides in the optical domain.
Therefore, any practical transmitter requires at least converting the symbols in the electrical domain to the coherent state in the optical domain.
Integration of the transmitter as a subsystem, for instance, as part of a \gls{cvqkd} system, highly suggests processing the symbols digitally as this is possible without loss of precision.
Additionally, digital signal processing can be defined on the software side, which allows for a more flexible and updatable design which is especially beneficial to update \gls{cvqkd} protocols.
In summary, the considerations made strongly advocate to start the signal processing of the transmitter in the software-defined digital domain and end the signal processing in the optical domain with analog signal processing mediating between these two ends.
\begin{figure}[htb]
	\centering
	\includestandalone{figures/tikz/transmitter}
	\caption{Block diagram showing the signal flow across the transmitter: Information is coded into complex-valued symbols $\alpha[n]$, the complex-valued symbols $\alpha[n]$ are converted to samples $\alpha^\prime[m]$ by upsampling and provided to the \gls{dsp}. The \gls{dsp} outputs the real-valued samples $x^\prime[m],p^\prime[m]$ which are converted to physical voltage signals $x^\prime(t),p^\prime(t)$ by digital-to-analog conversion and provided to the \gls{asp}. The \gls{asp} outputs the real-valued voltage signals $x(t),p(t)$ which are modulated onto a coherent state $\ket{\alpha^\prime(t)}$. Finally, \gls{osp} transforms $\ket{\alpha^\prime(t)}$ to $\ket{\alpha(t)}$.}\label{fig:transmitter}
\end{figure}
\Cref{fig:transmitter} depicts the proposed signal processing pipeline across the different domains.
Using software-defined \gls{dsp}, the digital symbols $\alpha[n]$ are encoded into the digital pulse shapes $x^\prime[m]$ and $p^\prime[m]$, which are converted to the analog signals $x^\prime(t)$ and $p^\prime(t)$.
An \gls{asp} stage prepares the analog signals $x(t),p(t)$ to drive the optical modulator.
The optical modulator encodes the analog signals $x(t), p(t)$ onto the coherent state $\ket{\alpha^\prime(t)}$, which might be further subject to a last \gls{osp} stage to output $\ket{\alpha(t)}$.
\begin{figure}[htb]
	\centering
	\includestandalone{figures/pgfplots/transmit-spectrum}
	\caption{Transmit spectrum.}\label{fig:transmit_spectrum}
\end{figure}
% TODO: explain components (and artifacts?) of the transmit spectrum and which of them we are going to discuss and why

\subsection{Digital signal-processing and analog conversion}

In the first half of the transmitter's signal processing, we aim to synthesize two analog baseband signals that encode the real respective imaginary part of the complex symbols later modulated onto the optical carrier.
The essential steps are digitally constructing a waveform encoding the complex symbols that satisfy our bandwidth requirements and converting the digital waveform to the analog domain.

\begin{figure}[htb]
	\centering
	\includestandalone{figures/circuitikz/transmitter-dsp}
	\caption{Block diagram of the digital and analog signal-processing of the pulse-shaping: First, the real and imaginary part of the symbols $\alpha[n]$ are upsampled to yield the samples $\alpha^\prime[m]$. Second, the samples $\alpha^\prime[m]$ are \gls{rrc} filtered yielding the samples $x^\prime[m]$ for real respective $p^\prime[m]$ for the imaginary part. Third, a \gls{dac} converts $x^\prime[m]$ and the $p^\prime[m]$ to the analog signals $x^\prime(t)$ and $p^\prime(t)$. Finally, an analog \gls{lp} filter removes aliases from the \gls{dac} output.}\label{fig:pulse_shaping_block}
\end{figure}

\Cref{fig:pulse_shaping_block} presents the digital and analog signal processing steps involved for the baseband signal synthesis:
The complex symbols $\alpha[n]$ are decomposed into their real $\Re\alpha[n]$ and imaginary part $\Im\alpha[n]$.
We limit the discussion to the real part of the complex symbols $\Re\alpha[n]$ but they are the same for the imaginary part $\Im\alpha[n]$.
For the real symbols $\Re\alpha[n]$, we first convert the symbols to samples $\Re\alpha^\prime[m]$ by upsampling.
Then, we apply \gls{rrc} filter to perform digital pulse-shaping of the samples to $x^\prime[m]$.
The pulse-shaped samples $x^\prime[m]$ are converted to an analog waveform $x^\prime(t)$ by a \gls{dac} followed by an \gls{lp} filter to remove aliases and smoothen the waveform.

\begin{figure}[htb]
	\centering
	\includestandalone[width=\textwidth]{figures/pgfplots/tx-unit-time}
	\caption{Pulse-shaping steps in the time domain for a single unit symbol: A unit symbol sequence $\Re\alpha[n]$ (first row) is upsampled by two (second row) to yield the samples $\Re\alpha^\prime[m]$. The digital \gls{rrc} filter is applied to the samples $x^\prime[m]$ determining the pulse-shape (third row). We can observe the diminishing ripple for the unit response of the filter to reduce the bandwidth. Finally, the pulse-shape $x^\prime[m]$ is converted to an analog signal and filtered by a \gls{lp} filter for smoothing and anti-aliasing.}\label{fig:pulse_shaping_unit_time}
\end{figure}

We start with a sequence of complex symbols $\alpha[1],\alpha[2],...$ which we shortly reference as $\alpha[n]$ with $n$ being the symbol index.
Depending on the context, $\alpha[n]$ refers to a single symbol or a sequence of symbols.
In practice, the symbol sequence is infinite and coded in blocks.
For instance, in \gls{cvqkd}, the symbols are sampled from a capped complex normal distribution.
In another embodiment, the symbols can encode real-time measurements from a sensor.
In both cases, there is no final symbol.

Symbols alone have no notion of time and therefore are not sufficient to represent a time-continuous signal.
Samples, in contrast, refer to a sequence of numbers together with a sampling frequency $f_s$ or a sample period $T_s=1/f_s$ are sufficient to represent a time-continuous signal.
The relation between the samples $x[m]$ and the time-continuous signal $x(t)$ is
\begin{equation}
	x[m]
	=
	x(t_m)
	=
	x(mT)
	,
\end{equation}
i.e., the samples $x[m]$ of a time-continuous signal $x(t)$ correspond to the values of $x(t)$ at some integer multiple of a time step $T_s$.
Obviously there is some information deficiency associated by representing a time-continuous signal by samples.
The Nyquist criteria states that the samples $x[m]$ together with a sampling frequency $f_s$ can at most describe signals where the highest frequency component is $f_s/2$.

\begin{figure}[htb]
	\centering
	\includestandalone[width=\textwidth]{figures/pgfplots/tx-rand-time}
	\caption{Pulse-shaping steps in the time domain for random symbols from a complex uniform distribution over the interval $[-1,+1]$. The first row shows the real (orange) and imaginary part (blue) of the complex symbols $\alpha[n]$ at their corresponding symbol index $n$. The second row shows the symbols after upsampling to $\alpha^\prime[m]$. The third row shows the samples after applying the \gls{rrc} filter $x^\prime[m]+ip^\prime[m]$. The fourth row shows the anti-aliased analog signal $x(t)+ip(t)$.}\label{fig:pulse_shaping_rand_time}
\end{figure}

In upsampling, we artificially increase the sampling frequency $f_s$ (or equivalently decrease the sample period $T_s$) by inserting zero-valued samples between the original samples, i.e.,
\begin{equation}
	\alpha^\prime[m]
	=
	\begin{cases}
		\alpha[m/2] & \text{if}\ m\mod2=0 \\
		0 & \text{otherwise}
	\end{cases}
\end{equation}
for an upsampling factor of two.
For every original sample we get as many samples as the upsampling factor with the first value being equal to the original sample and the other samples being equal to zero.
If there previously were $N$ samples (in a block) and after upsampling we have $M>N$ samples, then the upsampling factor is equal to the integer $M/N$ and the original sampling frequency $f_s$ increases by the upsampling factor to $f_sM/N$.
In the frequency spectrum, upsampling aliases the original spectrum to fill the increased signal bandwidth, we are going to discuss this later in detail.

In our present case, we assign a sampling frequency $f_s$ to the symbols $\Re\alpha[n],\Im\alpha[n]$ and upsample by a factor of two and find ourselves with samples $\Re\alpha^\prime[m],\Im\alpha^\prime[m]$ of twice the sampling frequency $2f_s$.

The zero-valued samples from the upsampling do not encode any information yet, however they allow for smoother interpolation between the non-zero samples.
In a naive attempt one could perform linear spline-based interpolation between the non-zero values.
While spline-based interpolation would smoothen our waveform in the time domain, we do so at the expense of bandwidth in the frequency domain.
The higher the bandwidth, the higher the performance requirements on the transmission system as nonlinear effects distort the received signal and cause intersymbol interference putting our naive attempt of spline-based interpolation at disfavor.
A better attempt uses a \gls{lp} filter to reduce the bandwidth requirements.
However, the optimal approach uses an \gls{rc} filter with absolute value of the transfer function given by~\cite[p.~33]{Nossek2015}
\begin{equation}
	\abs{h_\text{rc}\left(\frac{f}{f_s}\right)}
	=
	\begin{cases}
		1 & \abs*{\frac{f}{f_s}}\leq(1-\alpha) \\
		\cos\left[\frac{\pi}{4\alpha}\left(\abs*{\frac{f}{f_s}}-1+\alpha\right)\right] & 1-\alpha\leq\abs*{\frac{f}{f_s}}\leq1+\alpha \\
		0 & \text{otherwise}
	\end{cases}
	.
\end{equation}
In the presence of stochastic white noise, placing the square-root of the raised-cosine filter, the \gls{rrc} filter $h_\text{rrc}=\sqrt{h_\text{rc}}$, at both the transmitter and receiver side further improves the \gls{snr} which we adapt here~\cite{Cubukcu2012}.

At the boundary of the digital and analog domain, the pulse-shaped samples $x^\prime[m]$ and $p^\prime[m]$ are converted to an analog signal $x^\prime(t)$ respectively $p^\prime(t)$ by a \gls{dac}.
In the following, we assume an ideal \gls{dac} which performs digital-to-analog conversion by infinite upsampling instead of sample-and-hold (convolution with a rectangular response function).
The assumption is justified for a compensated \gls{dac} which is approximately linear in the passband of the \gls{lp} which removes aliases and smoothens the waveform.

We illustrated the pulse-shaping using a single unit symbol $\alpha[15]=1$ in \Cref{fig:pulse_shaping_unit_time} and a complex uniform random sequence in \Cref{fig:pulse_shaping_rand_time}.
Furthermore, \Cref{fig:pulse_shaping_freq} visualizes the pulse-shaping steps in the frequency domain.
\begin{figure}[htb]
	\centering
	\includestandalone[width=\textwidth]{figures/pgfplots/tx-frequency}
	\caption{Pulse-shaping steps in the frequency domain (showing the relative \gls{psd}) for unit symbol (orange) and random symbols from a complex uniform distribution over the interval $[-1,+1]$ (blue). The unit response symbols have a perfectly flat power spectrum (first row). The random symbols have an approximately flat power spectrum (first row). Both symbol spectra (first row) occupy the Nyquist bandwidth. Upsampling doubles the Nyquist bandwidth by aliasing the spectrum (second row). Pulse-shaping acts as a bandpass by strongly suppressing the frequency components outside the original Nyquist bandwidth. Conversion to an analog signal (fourth row) is equivalent to infinite upsampling or adding infinitely many aliases to occupy the complete frequency spectrum. Finally, filtering with a \gls{lp} removes aliases.}\label{fig:pulse_shaping_freq}
\end{figure}

\FloatBarrier
\subsection{Modulation and optical signal processing}

After digital-to-analog conversion and some additional \gls{asp}, we have the analog baseband signals $x(t)$ and $p(t)$ which we want to modulate onto an optical carrier with frequency $\omega_c$ for two reasons:
First, the transmission properties of the channel dictates the frequency bands, for instance, \SI{1550}{\nano\meter} in standard telecommunication optical fiber, for efficient transmission.
Second, the orthogonality of sine and cosine, concerning a fixed frequency $\omega_c$, allows transmission of two real-valued baseband signals $x(t)$ and $p(t)$.

From an abstract signal-processing perspective, we multiply the complex representation of the baseband signal~\cite[p.~25]{Madhow2008}, short complex baseband,
\begin{equation}
	\alpha(t)
	=
	x(t)
	+
	ip(t)
	.
\end{equation}
with the complex exponential oscillation factor $e^{-i\omega_ct}$ to get to the complex representation of the passband signal $\alpha(t)e^{-i\omega_ct}$.
The multiplication of the complex baseband with an oscillating exponential is also known as up-conversion.
\begin{figure}[htb]
	\centering
	\includestandalone{figures/circuitikz/iq-modulator}
	\caption{Block diagram of an electro-optical \gls{iq}-modulator: A sinusoidal oscillator with carrier frequency $\omega_c$ is equally split into two branches. The upper branch is mixed with the baseband signal $p(t)$ while the lower branch is phase-shifted by $\pi/2$ and mixed with the baseband signal $x(t)$. The output of both mixers are added and yield the \gls{iq} passband signal.}\label{fig:iqmod}
\end{figure}
The signal flow is depicted in a signal-processing block diagram in \Cref{fig:iqmod}.
A sinusoidal oscillator running at frequency $\omega_c$ is split into two branches with a relative phase shift of $\pi/2$ between the branches.
We define the upper branch as the phase reference.
The upper branch is mixed with the $p(t)$ signal yielding $p(t)\sin(\omega_ct)$.
The lower branch is mixed with the $x(t)$ signal yielding $x(t)\cos(\omega_ct)$.
Finally, upper and lower branch are added to yield the real passband signal
\begin{equation}
	x(t)
	\cos(\omega_ct)
	+
	p(t)
	\sin(\omega_ct)
	=
	\Re\left\{
		\alpha^\prime(t)
		e^{-i\omega_ct}
	\right\}
	\label{eq:passband_signal}.
\end{equation}
A particular implementation of the up-conversion into the optical domain is the \gls{iq}-modulator.
\Cref{fig:optical_transmitter} depicts the fiber-optical configuration such a \gls{iq}-modulator.
\begin{figure}[htb]
	\centering
	\includegraphics{figures/pstricks/transmitter}
	\caption{Optical \gls{iq}-modulator: The output of a coherent laser source with telecommunication wavelength \SI{1550}{\nano\meter} is equally split into two branches. The lower branch is optically phase-shifted by $\pi/2$ and passed to a \gls{mzm} MZM1 which is electro-optically modulated by the analog baseband signal $x(t)$. The upper branch is directly passed to MZM2 which is modulated by the analog baseband signal $p(t)$. The output of MZM1 is combined with MZM2 in an optical coupler. \SI{99}{\percent} of the output power is monitored by a photodiode PD. \SI{1}{\percent} of the output power is passed to a third \gls{mzm} MZM3 which is controlled by a power controller which uses the electrical signal of PD as input. Finally, the output of MZM3 is passed through an optical isolator.}\label{fig:optical_transmitter}
\end{figure}
An ultra-narrow-linewidth continuous-wave laser with center wavelength $\lambda_c=\SI{1550}{\nano\meter}$ is connected to a phase-symmetric and balanced 50:50 coupler which divides into an upper and lower branch.
The lower branch is phase-shifted by $\pi/2$ and modulated by a first \gls{mzm}, MZM1, which is driven by the baseband signal $x(t)$.
The upper branch is modulated by a second \gls{mzm}, MZM2, which is driven by a baseband signal $p(t)$.
The outputs of MZM1 and MZM2 are coupled with a second fiber coupler where the majority of the output power, \SI{99}{\percent}, is feed into a third \gls{mzm}, MZM3 for power control.
The remaining minority of the output power of the second fiber coupler is monitored by a power meter photodiode, PD.
The power controller operates on a timescale much larger than the modulation time scale and compensates for (thermal) power fluctuations of the laser while ensuring that the output power makes a description in terms of optical coherent state necessary.

% TODO: explain implementation of optical setup (in particular, show relevant quantum properties)

\section{Quantum transmitter}

The transmitter consists of a coherent light source, an \gls{iqm}, and a \gls{voa}.

Practically, a laser implements the coherent light source.
Advanced quantum stochastic models predicting laser characteristics (gain, stability, coherence, linewidth) are established and found in  Ref.~\cite[p.~900]{Mandel1995} and Ref.~\cite{Haken2012}.
For our purpose, it is sufficient to know that lasers emit coherent states with a frequency spectrum.\footnote{For an intuitive argument why laser emit coherent states based on decoherence, see Ref.~\cite{Gea1998}.}

The \gls{voa} attenuates the \gls{iq}-modulated signal such that the signal power becomes comparable with the quantum noise.
It is straightforward to model the \gls{voa}'s with a beam splitter transform, such that we will not draw further attention to it.

The \gls{iq} modulation is the most important step in the transmitter as it indicates the borderline between the classical and quantum signals.
It is also the most mysterious as we are unaware of any published quantum mechanical investigation.
The modulation occurs at three stages: the Pockels (phase) modulator, the \gls{mzm} (amplitude) modulator, and the \gls{iqm} modulator.
The transition from the classical signal to the quantum state occurs at the phase modulation, while the MZM and \gls{iqm} modulator operate on the quantum state.

\FloatBarrier
\subsection{Pockels modulator}

The Pockels modulator is a phase modulator where an electric field drives the phase of an optical field.
In the optical domain, we can use interference to perform amplitude modulation.
In that sense, the Pockels modulator is the interface between the electrical and optical domains.
For a complete quantum description of the transmitter, understanding the underlying dynamics of the Pockels modulator, particularly the implications of time-dependent modulation, is crucial.

\begin{figure}[htb]
    \centering
    \includegraphics{figures/tikz/pm}
    \caption{Pockels modulator comprising a Pockels cell embedded into a waveguide: The Pockels cell consists of two capacitor plates of length $L$ distanced at $d$. An optical field $\hat{E}_z(t)$ approaches the Pockels cell from the left and exits it to the right as $\hat{E}_z^\prime(t)$. An electric field $\hat{E}_x(t)$ perpendicular to the optical field spans the Pockels cell.}\label{fig:pm}
\end{figure}
\Cref{fig:pm} depicts a Pockels modulator comprising a Pockels cell embedded into a dielectric waveguide.
The Pockels cell is a dielectric plate capacitor with a specific dielectric that mediates the electric and optical field coupling.
The electric field inside the Pockels cell $E_x$ is proportional to the voltage $V$ applied to the Pockels cell.
Assuming the Pockels dielectric to be non-absorbing with \gls{lti} response, the dielectric permittivity tensor can be approximated, see Ref.~\cite{Murti2014} and Ref.~\cite[p.~1070]{Mandel1995},
\begin{equation}
	\varepsilon_{ij}(\omega)
	\approx
	\delta_{ij}
	+
	\chi_{ij}^{(1)}(\omega)
	+
	\chi_{ijk}^{(2)}(\omega)
	E^k(\Omega_0)
\end{equation}
wherein $\chi^{(n)}$ the denotes the $n$th order of the Pockels dielectric's electric susceptibility tensor and $\Omega_0\ll\omega$ is the center frequency of the electric signal.
The refractive index tensor of the Pockels dielectric takes the form
\begin{equation}
	n_{ij}(\omega)
	=
	\sqrt{\varepsilon_{ij}}
	\approx
	n_{ij}^{(0)}
	+
	n_{ijk}^{(1)}(\omega)
	E^k(\Omega_0)
\end{equation}
wherein the refractive index coefficients relate to the dielectric susceptibility tensor via~\cite{Rerat2020}
\begin{align}
	n^{(0)}_{ij}(\omega)
	&=
	\sqrt{1+\chi^{(1)}_{ij}(\omega)}
	&
	n^{(1)}_{ijk}(\omega)
	&=
	\frac{\chi^{(2)}_{ijk}(\omega)
	E^k(\Omega_0)}{2\sqrt{1+\chi^{(1)}_{ij}(\omega)}}
	.
\end{align}
Many elements of the refractive index tensor vanish because of crystal symmetries~\cite[p.~237]{Yariv1984} and we can express the refractive index relevant for the optical field
\begin{equation}
	n_{zz}(\omega,\Omega_0)
	\approx
	n^{(0)}_{zz}(\omega)
	+
	n^{(1)}_{zzx}(\omega)
	E^x(\Omega_0)
\end{equation}
which is linear in the electric \gls{rf} field inside the Pockels cell.
If we now imagine an optical field with frequency $\omega_0$ traveling through a Pockels cell it will accumulate a phase of
\begin{equation}
	\phi(E_x)
	=
	2\pi\frac{n_{zz}(\omega_0,E^x)}{\lambda_0}L
\end{equation} 
and the phase difference compared to a Pockels cell without electric field is proportional to the electric field, i.e.,
\begin{equation}
	\Delta\phi(E_x)
	=
	2\pi\frac{n_{zz}(\omega_0,E^x)-n_{zz}(\omega_0,0)}{\lambda_0}L
	=
	2\pi\frac{n_{zzx}^{(1)}(\omega_0)E^x}{\lambda_0}L
	.
\end{equation}

Up to now, the analysis was entirely classical.
A complete quantum description of the Pockels modulator is given in Ref.~\cite{Horoshko2018}, which we briefly summarize.
The Hamiltonian density of an electromagnetic field in a Pockels dielectric is, see Ref.~\cite[p.~21]{Vogel2006} together with Ref.~\cite[p.~124]{Jackson2007},
\begin{equation}
	H
	=
	\frac{1}{2}
	\int\dd[3]{x}
	\left\{
		\vb{E}(\vb{x})
		\vdot
		\vb{D}(\vb{x})
		+
		\vb{B}(\vb{x})^2
	\right\}
	\label{eq:dielectric_em_hamiltonian}
\end{equation}
wherein $\vb{E}$ is the electric field, $\vb{B}$ is the magnetic field and $\vb{D}$ is the displacement field with components
\begin{equation}
	D_i(\vb{p})
	=
	\varepsilon_{ij}(\vb{p})
	E^j(\vb{p})
	\approx
	E_i(\vb{p})
	+
	\int\frac{\dd[3]{q}}{(2\pi)^3}
	\chi_{ijk}^{(2)}(\vb{q},\vb{p})
	E^j(\vb{p})
	E^k(\vb{q})
\end{equation}
is the electric displacement field where we inserted the nonlinear expansion of the dielectric permittivity tensor and neglected the first-order susceptibility $\chi^{(1)}$.\footnote{The first-order susceptibility can be treated as an additional interaction.}
Inserting the displacement field into the Hamiltonian density, \cref{eq:dielectric_em_hamiltonian}, we recover the free Hamiltonian plus an interaction term
\begin{equation}
	H
	=
	H_0
	+
	H_\text{int}
\end{equation}
where the interaction term is given by
\begin{equation}
	H_\text{int}
	=	
	\frac{1}{2}
	\int\frac{\dd[3]{p}}{(2\pi)^3}
	\int\frac{\dd[3]{q}}{(2\pi)^3}
	\chi_{ijk}^{(2)}(\vb{q},\vb{p})
	E^i(\vb{p})
	E^j(\vb{p})
	E^k(\vb{q})	
\end{equation}
describing any three-point interactions including the linear electro-optical (Pockels) effect but also second harmonic generation, parametric mixing and amplification as well as optical rectification~\cite[p.~14]{Murti2014}.
Adapting the geometry of \Cref{fig:pm} the interaction Hamiltonian becomes
\begin{equation}
	H_\text{int}
	=	
	\frac{1}{2}
	\int\frac{\dd[3]{p}}{(2\pi)^3}
	\int\frac{\dd[3]{q}}{(2\pi)^3}
	\chi_{zzx}^{(2)}(\vb{q},\vb{p})
	E^z(\vb{p}-\vb{q})
	E^z(\vb{p}+\vb{q})
	E^x(\vb{q})	
\end{equation}
describing energy transfer from the electric field to the outgoing optical field.
Comparison of the classic interaction with the quantum nonlinear interactions in Ref.~\cite[p.~33]{QuesadaMejia2015} suggests the quantum interaction Hamiltonian
\begin{equation}
	\hat{H}_\text{int}
	=
	\frac{1}{2}
	\int\frac{\dd[3]{p}}{(2\pi)^3}
	\int\frac{\dd[3]{q}}{(2\pi)^3}
	\Upsilon(\vb{p},\vb{q})
	\hat{a}_x(\vb{q})
	\hat{a}_z^\dagger(\vb{p}-\vb{q})
	\hat{a}_z(\vb{p}+\vb{q})
	+
	\text{h.c.}
\end{equation}
wherein the spatial and response properties of the Pockels cell as well as the phase and momentum matching are encoded into $\Upsilon$.
Assuming only two perpendicular momenta $p_x,p_z$ and taking the electric \gls{rf} as classic, the interaction Hamiltonian further simplifies to
\begin{equation}
	\hat{H}_\text{int}
	=
	\frac{1}{2}
	\int\frac{\dd{p}}{2\pi}
	\int\frac{\dd{q}}{2\pi}
	\Upsilon(p,q)
	\hat{a}_x(q)
	\hat{a}_z^\dagger(p-q)
	\hat{a}_z(p+q)
	+
	\text{h.c.}
	.
\end{equation}
Let $\ket{\beta(t)}$ be the coherent state describing the electric \gls{rf} field, then the interaction Hamiltonian becomes
\begin{equation}
	\hat{H}_\text{int}(t)
	=
	\frac{1}{2}
	\int\frac{\dd{p}}{2\pi}
	\int\frac{\dd{q}}{2\pi}
	\Upsilon(p,q)
	\beta_x(q)^*
	e^{-ipt}
	\hat{a}_z^\dagger(p-q)
	\hat{a}_z(p+q)
	+
	\text{h.c.}
\end{equation}
and when taking the momentum of the electric \gls{rf} field to be small compared to the momentum of the optical field, $p\ll q$, such that
\begin{equation}
	\hat{a}_z^\dagger(p-q)
	\hat{a}_z(p+q)
	\approx
	\hat{a}_z^\dagger(p)
	\hat{a}_z(p)
\end{equation}
we obtain the approximate interaction Hamiltonian
\begin{equation}
	\hat{H}_\text{int}(t)
	\approx
	g(t)
	\int\frac{\dd{q}}{2\pi}
	\hat{a}^\dagger(q)
	\hat{a}(q)
\end{equation}
for which all but the first term of the Magnus expansion vanish\footnote{The operator part is time-independent and therefore the interaction Hamiltonian at different times commutes.} and the exact time-evolution operator turns out to be
\begin{align}
	\hat{U}(t_0,t)
	&=
	\exp\left\{
		-i\varphi(t_0,t)
		\int\frac{\dd{p}}{2\pi}
		\hat{a}^\dagger(p)
		\hat{a}(p)
	\right\}
	&
	\varphi(t_0,t)
	=
	\int_{t_0}^{t}\dd{t^\prime}
	g(t^\prime)
\end{align}
which reduces to the phase-rotation operator proposed in Ref.~\cite[p.~38]{Leonhardt2010} and Ref.~\cite[p.~103]{Vogel2006} in the single-mode limit and time-independent $\varphi$.

We define the action of the phase modulator
\begin{equation}
	\hat{U}_\text{PM}\left[\varphi(t,T)\right]
	=
	\exp\left\{
		-i\varphi(t,T)
		\int\frac{\dd{p}}{2\pi}
		\hat{a}^\dagger(p)
		\hat{a}(p)
	\right\}
\end{equation}
wherein $T$ is the transmit time of the light through the phase modulator.
Assuming a phase modulator of length $L$ and refractive index $n$, the transmit time is approximately
\begin{align}
	T
	&=
	\frac{Ln}{v_\text{gr}}
	&
	v_\text{gr}
	\approx
	\left[
		n
		+
		\omega
		\pdv{n}{\omega}
	\right]_{\omega=\omega_0}^{-1}
\end{align}
where $v_\text{gr}$ is the group velocity in the quasi-monochromatic approximation~\cite[p.~211]{Jackson2007}.
We employ the mean-value theorem for definite integrals
\begin{align}
	\varphi(t,T)
	&=
	\int_{t}^{t+T}\dd{t^\prime}
	g(t^\prime)
	=
	g(\tau)
	T
	&
	\tau\in[t,t+T]
\end{align}
and note that typically the transmit time $T$ is many magnitudes smaller than the modulation time scale $t$, i.e., $T\ll t$, and we can use
\begin{equation}
	\varphi(t,T)
	\approx
	g(t)
	T
	=
	\varphi(t)
	.
\end{equation}

The phase modulator transforms the annihilation operator according to~\cite[p.~38]{Leonhardt2010}
\begin{equation}
	\hat{U}_\text{PM}^\dagger
	\hat{a}(p,t)
	\hat{U}_\text{PM}
	=
	\hat{a}(p,t)
	e^{-i\varphi(t)}
	\label{eq:pm_annihilation}
	.
\end{equation}
The continuous-mode displacement operator transforms according to
\begin{equation}
	\begin{split}
		\hat{U}_\text{PM}\left[\varphi(t)\right]
		\hat{D}\left[\alpha(t)\right]
		\hat{U}_\text{PM}\left[\varphi(t)\right]^\dagger
		&=
		\exp\left\{
			\int_0^\infty\frac{\dd{p}}{2\pi}
			\alpha(p)
			\hat{U}_\text{PM}\left[\varphi(t)\right]
			\hat{a}^\dagger(p,t)
			\hat{U}_\text{PM}\left[\varphi(t)\right]^\dagger
			-
			\text{h.c.}
		\right\}
		\\
		&=
		\exp\left\{
			\int_0^\infty\frac{\dd{p}}{2\pi}
			\alpha(p)
			\left(
				\hat{U}_\text{PM}\left[\varphi(t)\right]
				\hat{a}(p,t)
				\hat{U}_\text{PM}\left[\varphi(t)\right]^\dagger
			\right)^\dagger
			-
			\text{h.c.}
		\right\}
		\\
		&=
		\exp\left\{
			\int_0^\infty\frac{\dd{p}}{2\pi}
			\alpha(p)
			\left(
				\hat{U}_\text{PM}\left[-\varphi(t)\right]^\dagger
				\hat{a}(p,t)
				\hat{U}_\text{PM}\left[-\varphi(t)\right]
			\right)^\dagger
			-
			\text{h.c.}
		\right\}
		\\
		&=
		\exp\left\{
			\int_0^\infty\frac{\dd{p}}{2\pi}
			\alpha(p)
			e^{-i\varphi(t)}
			\hat{a}^\dagger(p,t)
			-
			\text{h.c.}
		\right\}
		\\
		&=
		\hat{D}\left[
			\alpha(t)
			e^{-i\varphi(t)}
		\right]
	\end{split}
\end{equation}
where we used \cref{eq:pm_annihilation} in the fourth line.
The phase-modulated amplitude
\begin{equation}
	\alpha(t)e^{-i\varphi(t)}
\end{equation}
is a multiplication in the time domain.
We expect therefore a convolution in the frequency domain, i.e.,
\begin{equation}
	\int_0^\infty\frac{\dd{p}}{2\pi}
	\alpha(p)
	e^{+ipt}
	e^{-i\varphi(t)}
	\hat{a}^\dagger(p,t)
	=
	\int_0^\infty\frac{\dd{p}}{2\pi}
	\left(u*\alpha\right)(p)
	\hat{a}^\dagger(p,t)
\end{equation}
wherein we used the convolution
\begin{equation}
	\left(u*\alpha\right)(p)
	=
	\int\frac{\dd{p^\prime}}{2\pi}
	u(p^\prime)
	\alpha(p-p^\prime)
	.
\end{equation}
It is difficult to give an explicit form of $u(p)$ for signals $\varphi(t)$ with a continuous frequency spectrum $u(p)$.
However, if we take a simple single-frequency modulation, e.g.,
\begin{equation}
	\varphi(t)
	=
	-
	\varphi_0
	\sin(\omega_0t)
\end{equation}
we can use the Jacobi-Anger expansion to write
\begin{equation}
	e^{i\varphi_0\sin(\omega_0t)}
	=
	\sum_{n\in\mathbb{Z}}
	J_n(\varphi_0)
	e^{in\omega_0 t}
\end{equation}
wherein $J_n$ denotes the $n$th Bessel function of the first kind.
Assuming second order harmonics to be outside the bandwidth of our setup, it is sufficient to only keep the first terms of the Jacobi-Anger expansion
\begin{equation}
	e^{i\varphi_0\sin(\omega_0t)}
	=
	J_{-1}(\varphi_0)
	e^{-i\omega_0t}
	+
	J_0(\varphi_0)
	+
	J_{+1}(\varphi_0)
	e^{+i\omega_0t}
\end{equation}
which corresponds to the convolution kernel
\begin{equation}
	u(p)
	=
	J_{-1}(\varphi_0)
	\delta^{(1)}(p-\omega_0)
	+
	J_0(\varphi_0)
	\delta^{(1)}(p)
	+
	J_{+1}(\varphi_0)
	\delta^{(1)}(p+\omega_0)
	.
\end{equation}

\subsection{Mach-Zehnder modulator}

The \gls{mzm} arranges two phase modulators to perform amplitude and phase modulation on an optical input field.
Using two electrical-driven phase modulators, the \gls{mzm} performs amplitude and phase modulation on an optical input field.
To begin with, we derive the quantum operator matrix transform from a specific implementation of the \gls{mzm}, the symmetric free-space \gls{mzi}.
Then, we link the quantum operator matrix transform to more general unitary operators.
Finally, we derive the quantum state transform for a coherent input state to the \gls{mzm}.

\begin{figure}[htb]
	\centering
	\includestandalone{figures/pstricks/mzi-symmetric}
	\caption{Free-space setup of a symmetric \gls{mzm}: The input light mode $\hat{a}_1$ enters a first beam splitter BS1 from the left. A vacuum light mode $\hat{a}_2$ enters BS2 from the top. The transformed mode $\hat{a}_1^\prime$ and $\hat{a}_2^\prime$ exit BS1 to the right and the bottom. A first phase shifter and first mirror M1, right to BS1, add a relative phase of $\varphi_1+\pi$ from mode $\hat{a}_1$ to $\hat{a}_1^{\prime\prime}$. Below BS1, a second mirror M2 directs the light to a right second phase shifter, both adding a relative phase of $\varphi_2+\pi$ from mode $\hat{a}_2^\prime$ to $\hat{a}_2^{\prime\prime}$. A second beam splitter BS2 transforms the input modes $\hat{a}_1^{\prime\prime}$ and $\hat{a}_2^{\prime\prime}$ to the output modes $\hat{a}_1^{\prime\prime\prime}$ and $\hat{a}_2^{\prime\prime\prime}$.}\label{fig:mzi_symmetric}
\end{figure}
\Cref{fig:mzi_symmetric} shows a free-space optics setup of a symmetric \gls{mzi} with one signal input; the other input being in the vacuum state.
The most crucial components of the \gls{mzi} are a splitter, a coupler, and two independent phase modulators.
The splitter divides the input light into two branches.
Each branch adds a relative phase shift using an independent phase modulator, i.e., $\phi_1$ and $\phi_2$.
The coupler recombines both branches into two outputs.
For our free-space setup, two cubic beam splitters implement the splitter (BS1) and the coupler (BS2).
For additional beam alignment, our free-space setup utilizes two mirrors (M1 and M2).

Finding the transformation of the annihilation operators at each stage of the \gls{mzi} is sufficient to find the quantum input-output relations.
Idealizing the symmetric \gls{mzi}'s passive components as lossless allows relating the annihilation operators by two-dimensional unitary matrices.
We label the input annihilation operators of the \gls{mzi} $\hat{a}_1,\hat{a}_2$ and the output annihilation operators $\hat{a}_1^{\prime\prime\prime},\hat{a}_2^{\prime\prime\prime}$.
The annihilation operators after splitting and before (after) phase shifting are denoted by two (three) primes, i.e., $\hat{a}_1^{\prime},\hat{a}_2^{\prime}$ and $\hat{a}_1^{\prime\prime},\hat{a}_2^{\prime\prime}$.
Going backwards through the transformations from the output to the input annihilation operators
\begin{equation}
	\vb{\hat{a}}^{\prime\prime\prime}
	=
	U_\text{BS2}
	\hat{\vb{a}}^{\prime\prime}
	=
	U_\text{BS2}
	U_\text{PS}
	\hat{\vb{a}}^{\prime}
	=
	U_\text{BS2}
	U_\text{PS}
	U_\text{BS1}
	\hat{\vb{a}}
	=
	U_\text{MZI}
	\vb{\hat{a}}
\end{equation}
we find the symmetric \gls{mzi}'s unitary matrix transform $U_\text{MZM}$ to be equal to the matrix product of the second beam splitter's, the phase shifts', and the first beam splitter's unitary matrix transform  $U_\text{BS2}U_\text{PS}U_\text{BS1}$.

An ideal cubic beam splitter with a single dielectric layer has the unitary matrix transform~\cite[p.~139]{Gerry2005}
\begin{equation}
	U_\text{BS1}
	=
	\frac{1}{\sqrt{2}}
	\begin{pmatrix}
		1 & i \\
		i & 1
	\end{pmatrix}
\end{equation}
where the off-diagonal elements of $U_\text{BS1}$, $i/\sqrt{2}$, account for the phase-shift due to the reflection at the diagonal of the cubic beam splitter.
The matrix encoding the phase shifts from the phase modulation $\phi_1,\phi_2$ and the reflection at the mirrors M1 and M2, $\pi$ is
\begin{equation}
	U_\text{PS}
	=
	\begin{pmatrix}
		ie^{i\phi_1} & 0 \\
		0 & ie^{i\phi_2}
	\end{pmatrix}
\end{equation}
For the second beam splitter, BS2, we again assume an ideal cubic beam splitter with a single dielectric layer.
The corresponding matrix transform is
\begin{equation}
	U_\text{BS2}
	=
	\frac{1}{\sqrt{2}}
	\begin{pmatrix}
		i & 1 \\
		1 & i
	\end{pmatrix}	
\end{equation}
where we exchanged the rows for consistency with the input labels.

Performing the matrix multiplication and writing the exponentials as trigonometric functions, we find the matrix transform of the symmetric \gls{mzi} to be
\begin{equation}
	U_\text{MZI}
	=
	-
	\begin{pmatrix}
		\cos\left(\frac{\phi_2-\phi_1}{2}\right) & \sin\left(\frac{\phi_2-\phi_1}{2}\right) \\
		-\sin\left(\frac{\phi_2-\phi_1}{2}\right) & \cos\left(\frac{\phi_2-\phi_1}{2}\right)
	\end{pmatrix}
	e^{i\frac{\phi_1+\phi_2}{2}}
	.
\end{equation}
It appears useful to define the common-mode and differential-mode phases
\begin{align}
	\phi_+
	&=
	\phi_2
	+
	\phi_1
	&
	\phi_-
	&=
	\phi_2-\phi_1
\end{align}
for which the matrix transform simplifies to
\begin{equation}
	U_\text{MZI}
	=
	-
	\begin{pmatrix}
		\cos(\phi_-/2) & \sin(\phi_-/2) \\
		-\sin(\phi_-/2) & \cos(\phi_-/2)
	\end{pmatrix}
	e^{i\phi_+/2}
\end{equation}
and we note that the common-mode phase $\phi_+$ adds a global phase shift of $\phi_+/2$ while the differential-mode phase $\phi_-$ changes the splitting ratios at the output.

For a general lossless \gls{mzm}, we propose the generic unitary matrix transform~\cite[p.~95]{Leonhardt2010}
\begin{equation}
	\begin{split}
		U_\text{MZM}
		&=
		e^{i\Lambda/2}
		\begin{pmatrix}
			\cos(\Theta/2)e^{i(+\Phi+\Psi)/2} & \sin(\Theta/2)e^{i(+\Phi-\Psi)/2} \\
			-\sin(\Theta/2)e^{+(-\Phi+\Psi)/2} & \cos(\Theta/2)e^{i(-\Phi-\Psi)/2}
		\end{pmatrix}
		\\
		&=
		e^{i\Lambda/2}
		\begin{pmatrix}
			e^{+i\Phi/2} & 0 \\
			0 & e^{-i\Phi/2}
		\end{pmatrix}
		\begin{pmatrix}
			\cos(\Theta/2) & \sin(\Theta/2) \\
			-\sin(\Theta/2) & \cos(\Theta/2)
		\end{pmatrix}
		\begin{pmatrix}
			e^{+i\Psi/2} & 0 \\
			0 & e^{-i\Psi/2}
		\end{pmatrix}
	\end{split}
	\label{eq:mzm_matrix}
\end{equation}
wherein the global phase $\Theta$ and the rotation angle $\Lambda$ are time-dependent but $\Phi,\Psi$ are constants.
We obtain the matrix transform of the symmetric free-space $U_\text{MZI}$ after identification of $\Theta$ with the differential-mode phase, $\Lambda+\pi$ with the common-mode phase, and choosing $\Psi=0=\Phi$.
One advantage of the proposed decomposition in \cref{eq:mzm_matrix} is the one-to-one correspondence between the unitary operator~\cite[p.~99]{Leonhardt2010}
\begin{equation}
	\hat{U}_\text{MZM}(\Lambda,\Phi,\Psi,\Theta)
	=
	e^{i\Lambda\hat{L}_t}
	e^{i\Phi\hat{L}_z}
	e^{i\Theta\hat{L}_y}
	e^{i\Psi\hat{L}_z}
	\label{eq:mzm_operator}
\end{equation}
wherein $\hat{L}_j$ denote the Jordan-Schwinger operators~\cite[p.~97]{Leonhardt2010}
\begin{equation}
	\hat{L}_j
	=
	\frac{1}{2}
	\begin{pmatrix}
		\hat{a}_1^\dagger, \hat{a}_2^\dagger
	\end{pmatrix}
	\sigma_j
	\begin{pmatrix}
		\hat{a}_1 \\
		\hat{a}_2
	\end{pmatrix}
	=
	\frac{1}{2}
	\hat{\vb{a}}^\dagger
	\sigma_j
	\hat{\vb{a}}
\end{equation}
with $\sigma_j$ being the two-dimensional Pauli matrices and $\sigma_0$ being the identity matrix, and the unitary matrix transform
\begin{equation}
	U_\text{MZM}
	\hat{\vb{a}}
	=
	\hat{U}_\text{MZM}^\dagger
	\hat{\vb{a}}
	\hat{U}_\text{MZM}
	.
\end{equation}

The \gls{mzm} produces highly entangled output states for (displaced) number states, see, for instance, Ref.~\cite{Windhager2011}.
In the transmitter setting, we are exclusively interested in coherent input states, contrary to the receiver, where an attacker could have sent non-coherent states.

Let us consider the input state
\begin{align}
	\ket{\vb{\alpha}}
	&=
	\ket{\alpha_1,\alpha_2}
	&
	\vb{\alpha}
	=
	\begin{pmatrix}
		\alpha_1,
		\alpha_2
	\end{pmatrix}
	\in
	\mathbb{C}^{2\times1}
\end{align}
to a \gls{mzm}\footnote{We can always set $alpha=0$ to set the second input to vacuum.}, i.e.,
\begin{equation}
	\hat{U}_\text{MZM}
	\ket{\vb{\alpha}}
	=
	\hat{U}_\text{MZM}
	\hat{D}(\vb{\alpha})
	\hat{U}_\text{MZM}^\dagger
	\hat{U}_\text{MZM}
	\ket{0,0}
	=
	\hat{U}_\text{MZM}
	\hat{D}(\vb{\alpha})
	\hat{U}_\text{MZM}^\dagger
	\ket{0,0}
\end{equation}
where we used the invariance of the vacuum state $\ket{0,0}$ for the second equality.
The displacement operator transforms as
\begin{equation}
	\begin{split}
		\hat{U}
		\hat{D}(\vb{\alpha})
		\hat{U}^\dagger
		&=
		\hat{U}
		\exp\left\{
			\vb{\alpha}^\trans
			\hat{\vb{a}}^\dagger
			-
			\hat{\vb{a}}
			\vb{\alpha}^*
		\right\}
		\hat{U}^\dagger
		\\
		&=
		\exp\left\{
			\vb{\alpha}^\trans
			\hat{U}
			\hat{\vb{a}}^\dagger
			\hat{U}^\dagger
			-
			\hat{U}
			\hat{\vb{a}}
			\hat{U}^\dagger
			\vb{\alpha}^*
		\right\}
		\\
		&=
		\exp\left\{
			\vb{\alpha}^\trans
			\left(
				\hat{U}
				\hat{\vb{a}}
				\hat{U}^\dagger
			\right)^\dagger
			-
			\hat{U}
			\hat{\vb{a}}
			\hat{U}^\dagger
			\vb{\alpha}^*
		\right\}
	\end{split}
\end{equation}
where we used the operator identity
\begin{equation}
	\hat{U}
	e^{\hat{A}}
	\hat{U}^\dagger
	=
	\sum_{n=0}^\infty
	\frac{1}{n!}
	\hat{U}
	\hat{A}^n
	\hat{U}^\dagger
	=
	\sum_{n=0}^\infty
	\frac{1}{n!}
	\hat{U}
	\hat{A}
	\hat{U}^\dagger
	\cdots
	\hat{U}
	\hat{A}
	\hat{U}^\dagger
	=
	\sum_{n=0}^\infty
	\frac{1}{n!}
	\left(
		\hat{U}
		\hat{A}
		\hat{U}^\dagger
	\right)^n
	=
	e^{\hat{U}\hat{A}\hat{U}^\dagger}
\end{equation}
which follows from inserting $\hat{U}\hat{U}^\dagger=\mathbb{1}$ between the $\hat{A}$s.
For the unitary operator of the \gls{mzm}, we note that
\begin{equation}
	\begin{split}
		\hat{U}_\text{MZM}(\Lambda,\Phi,\Psi,\Theta)^\dagger
		&=
		e^{-i\Psi\hat{L}_z}
		e^{-i\Theta\hat{L}_y}
		e^{-i\Phi\hat{L}_z}
		e^{-i\Lambda\hat{L}_t}
		\\
		&=
		e^{-i\Lambda\hat{L}_t}
		e^{-i\Psi\hat{L}_z}
		e^{-i\Theta\hat{L}_y}
		e^{-i\Phi\hat{L}_z}
		\\
		&=
		\hat{U}_\text{MZM}(-\Lambda,-\Psi,-\Phi,-\Theta)
	\end{split}
\end{equation}
where we used in the second line that $\hat{L}_t$ commutes with the other Jordan-Schwinger operators $\hat{L}_y,\hat{L}_z$.
The transformed annihilation operators turn out to be
\begin{equation}
	\begin{split}
		\hat{\vb{a}}^\prime
		&=
		\hat{U}_\text{MZM}(\Lambda,\Phi,\Psi,\Theta)
		\hat{\vb{a}}
		\hat{U}_\text{MZM}(\Lambda,\Phi,\Psi,\Theta)^\dagger
		\\
		&=
		\hat{U}_\text{MZM}(-\Lambda,-\Psi,-\Phi,-\Theta)^\dagger
		\hat{\vb{a}}
		\hat{U}_\text{MZM}(-\Lambda,-\Psi,-\Phi,-\Theta)
		\\
		&=
		U_\text{MZM}(-\Lambda,-\Psi,-\Phi,-\Theta)
		\hat{\vb{a}}
		\\
		&=
		U_\text{MZM}(\Lambda,\Phi,\Psi,\Theta)^\dagger
		\hat{\vb{a}}
		.
	\end{split}
\end{equation}
Using the transformed operators, we find the transformed displacement operator to be~\cite[p.~210]{Vogel2006}
\begin{equation}
	\hat{D}^\prime(\vb{\alpha})
	=
	\exp\left\{
		\vb{\alpha}^\trans
		\left(\hat{\vb{a}}^\prime\right)^\dagger
		-
		\hat{\vb{a}}^\prime
		\vb{\alpha}^*
	\right\}
	=
	\exp\left\{
		\left(\vb{\alpha}^\prime\right)^\trans
		\hat{\vb{a}}^\dagger
		-
		\hat{\vb{a}}
		\left(\vb{\alpha}^\prime\right)^*
	\right\}
	.
\end{equation}
We therefore find the \gls{mzm} to transform the coherent input states according to
\begin{align}
	\hat{U}_\text{MZM}
	\ket{\vb{\alpha}}
	&=
	\ket{\vb{\alpha}^\prime}
	&
	\vb{\alpha}^\prime
	&=
	U_\text{MZM}
	\vb{\alpha}
	.
\end{align}
Using the explicit matrix elements in \cref{eq:mzm_matrix} and setting the second input to zero $\alpha_2=0$ and identifying the first as $\alpha_1=\alpha$, we find
\begin{equation}
	\hat{U}_\text{MZM}
	\ket{\alpha,0}
	=
	\ket{\alpha\cos(\Theta/2)e^{+i(\Psi+\Phi)/2},-\alpha\sin(\Theta/2)e^{-i(\Psi-\Phi)/2}}
	.
\end{equation}
We are only interested in the first output mode.
We can remove the second output via partial trace, i.e.,
\begin{equation}
	\trace_2\left\{
		\ketbra{\alpha,\beta}
	\right\}
	=
	\trace_2\left\{
		\ketbra{\alpha}
		\otimes
		\ketbra{\beta}
	\right\}
	=
	\ketbra{\alpha}
	\otimes
	\trace_2\left\{
		\ketbra{\beta}
	\right\}
\end{equation}
where
\begin{equation}
	\trace_2\left\{
		\ketbra{\beta}
	\right\}
	=
	\sum_{n=0}^\infty
	\braket{n}{\beta}
	\braket{\beta}{n}
	=
	\sum_{n=0}^\infty
	\abs{\braket{n}{\beta}}^2
	=
	1
\end{equation}
and thus
\begin{equation}
	\trace_2\left\{
		\ketbra{\alpha,\beta}
	\right\}
	=
	\ketbra{\alpha}
	=
	\hat{P}_1
	\ketbra{\alpha,\beta}
	\hat{P}_1
\end{equation}
i.e., the partial trace over a tensor product of coherent states is equivalent to the projection of the non-traced out states.
This is a special property of the coherent states.
In general, the partial trace over a tensor product of (entangled) quantum states returns a mixed state.

We conclude that the the action of a static \gls{mzm} on a coherent input state $\ket{\alpha}$ is a amplitude and phase modulation, i.e.,
\begin{equation}
	\ket{\alpha}
	\to
	\ket{\alpha^\prime}
	=
	\hat{P}_1
	\hat{U}_\text{MZM}
	\ket{\alpha,0}
	=
	\ket{\alpha\cos(\Theta/2)e^{i\Lambda/2}}
	\label{eq:mzm_transform}
\end{equation}
where $\Theta$ is the differential- and $\Lambda$ is the common-mode phase applied to the Pockels modulator of the \gls{mzm}.
We have effectively multiplied the amplitude of the initial coherent state with a real value $\cos(\Theta/2)$.

Our result holds true for time-dependent modulation.
From the discussion of the Pockels modulator we know that phase modulation can be written as a convolution in momentum space which is a linear operation and therefore compatible with our derivation of the \gls{mzm}.

\subsection{In-phase and quadrature modulator}

In the previous section we derived that the \gls{mzm} performs amplitude and phase modulation on a coherent state.
The amplitude modulation is equivalent to multiplying the initial amplitude of the coherent state by a real number.
However, the amplitude of a coherent state is a complex number
The \gls{iqm} arranges two \gls{mzm} such that both quadratures of the coherent light can be modulator, i.e., we multiply the amplitude of the coherent state by a complex number.
This can be done by first splitting the signal into two branches, where each branch has an \gls{mzm}, and then recombining the two branches.
The \gls{mzm} \gls{pm} must hold a total phase difference of $\pi/2$ between the branches.
Otherwise, the quadrature components become correlated.

\Cref{fig:iqm} presents a fiber-optical embodiment of an \gls{iqm} where we labeled the in- and output modes using the corresponding quantum annihilation operator.
We indicated non-signal in- and outputs by a dashed orange line.
The non-signal inputs must be in a vacuum state.
The non-signal outputs may be dumped or monitored.
Two voltage signals, not indicated in the figure, drive each of the \gls{mzm}.
\begin{figure}[htb]
	\centering
	\includestandalone{figures/pstricks/iqm}
	\caption{Possible fiber-optical setup of a \gls{iqm} comprising a splitter, a coupler, and two \gls{mzm}: An input coherent state $\ket{\alpha}$ enters the splitter, which distributes the input coherent state's amplitude among an upper and lower branch, then hosts coherent states $\ket{\alpha_1}$ and $\ket{\alpha_2}$. The first \gls{mzm}, MZM1, at the upper branch, transforms the coherent state $\ket{\alpha_1}$ to the coherent state $\ket{\alpha_1^\prime}$. The second \gls{mzm}, MZM2, at the lower branch, transforms the coherent state $\ket{\alpha_2}$ to the coherent state $\ket{\alpha_2^\prime}$. The coupler recombines the coherent states $\ket{\alpha_1^\prime}$ and $\ket{\alpha_2^\prime}$ into the output coherent state $\ket{\alpha^\prime}$, exiting the \gls{iqm}.}\label{fig:iqm}
\end{figure}

Let us consider the coherent input state $\ket{\alpha}$.
The splitter takes the coherent input state together with a vacuum state and performs a beam splitter transform.
For a balanced beam splitter, we find
\begin{equation}
	\ket{\alpha_1,\alpha_2}
	=
	\hat{U}_\text{BS1}(\varphi_1)
	\ket{\alpha,0}
	=
	\ket*{\frac{\alpha}{\sqrt{2}},\frac{\alpha}{\sqrt{2}}ie^{-i\varphi_1}}
\end{equation}
where the splitter added a relative phase shift of $\varphi$ between the two branches.
The \gls{mzm} transform the upper and lower branch coherent states according to
\begin{align}
	\ket{\alpha_1}
	&\to
	\ket{\alpha_1^\prime}
	=
	\ket*{\frac{\alpha}{\sqrt{2}}\cos(\Theta_1/2)e^{i\Lambda_1/2}}
	\\
	\ket{\alpha_2}
	&\to
	\ket{\alpha_2^\prime}
	=
	\ket*{\frac{\alpha}{\sqrt{2}}ie^{-i\varphi_1}\cos(\Theta_2/2)e^{i\Lambda_2/2}}
	.
\end{align}
For the second balanced beam splitter, we find for the output mode
\begin{equation}
	\ket{\alpha^\prime}
	=
	\hat{P}_1
	\hat{U}_\text{BS2}(\varphi_2)
	\ket{\alpha_1^\prime,\alpha_2^\prime}
	=
	\ket{\alpha z(\Theta_1,\Theta_2)}
\end{equation}
where we choose the first \gls{mzm} phase to be zero $\Lambda_1=0$ and the second phase $\Lambda_2$ such that
\begin{equation}
	z(\Theta_1,\Theta_2)
	=
	\frac{1}{2}
	\left[
		\cos(\Theta_1/2)
		+
		i\cos(\Theta_2/2)
	\right]
	.
\end{equation}
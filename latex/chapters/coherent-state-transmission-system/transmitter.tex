\section{Transmitter}                                                                                                                                                                                                                                                                                                                                                                                                                                                                                                                                                                                                                                                                                                                     

We introduced the transmitter in the coherent state transmission system model as a device that encodes symbols $\alpha[n]$ onto a coherent state $\ket{\alpha(t)}$.
The far-reaching significance of electronic processing in our present state-of-the-art strongly suggests the symbols $\alpha[n]$ to be electrically encoded while the coherent state resides in the optical domain.
Therefore, any practical transmitter requires at least converting the symbols in the electrical domain to the coherent state in the optical domain.
Integration of the transmitter as a subsystem, for instance, as part of a \gls{cvqkd} system, highly suggests processing the symbols digitally as this is possible without loss of precision.
Additionally, digital signal processing can be defined on the software side, which allows for a more flexible and updatable design which is especially beneficial to update \gls{cvqkd} protocols.
In summary, the considerations made strongly advocate to start the signal processing of the transmitter in the software-defined digital domain and end the signal processing in the optical domain with analog signal processing mediating between these two ends.
\begin{figure}[htb]
	\centering
	\includestandalone{figures/tikz/transmitter}
	\caption{Block diagram showing the signal flow across the transmitter: Information is coded into complex-valued symbols $\alpha[n]$, the complex-valued symbols $\alpha[n]$ are converted to samples $\alpha^\prime[m]$ by upsampling and provided to the \gls{dsp}. The \gls{dsp} outputs the real-valued samples $x^\prime[m],p^\prime[m]$ which are converted to physical voltage signals $x^\prime(t),p^\prime(t)$ by digital-to-analog conversion and provided to the \gls{asp}. The \gls{asp} outputs the real-valued voltage signals $x(t),p(t)$ which are modulated onto a coherent state $\ket{\alpha^\prime(t)}$. Finally, \gls{osp} transforms $\ket{\alpha^\prime(t)}$ to $\ket{\alpha(t)}$.}\label{fig:transmitter}
\end{figure}
\Cref{fig:transmitter} depicts the proposed signal processing pipeline across the different domains.
Using software-defined \gls{dsp}, the digital symbols $\alpha[n]$ are encoded into the digital pulse shapes $x^\prime[m]$ and $p^\prime[m]$, which are converted to the analog signals $x^\prime(t)$ and $p^\prime(t)$.
An \gls{asp} stage prepares the analog signals $x(t),p(t)$ to drive the optical modulator.
The optical modulator encodes the analog signals $x(t), p(t)$ onto the coherent state $\ket{\alpha^\prime(t)}$, which might be further subject to a last \gls{osp} stage to output $\ket{\alpha(t)}$.
\begin{figure}[htb]
	\centering
	\includestandalone{figures/pgfplots/transmit-spectrum}
	\caption{Transmit spectrum of the transmitter.}\label{fig:transmit_spectrum}
\end{figure}
% TODO: explain components (and artifacts?) of the transmit spectrum and which of them we are going to discuss and why

\subsection{Pulse-shaping}

\Cref{fig:pulse_shaping_block} depicts the analog and digital signal-processing of the pulse-shaping process.
\begin{figure}[htb]
	\centering
	\includestandalone{figures/circuitikz/pulse-shaping}
	\caption{Block diagram of the digital and analog signal-processing of the pulse-shaping: First, the real and imaginary part of the symbols $\alpha[n]$ are upsampled to yield the samples $\alpha^\prime[m]$. Second, the samples $\alpha^\prime[m]$ are \gls{rrc} filtered yielding the samples $x^\prime[m]$ for real respective $p^\prime[m]$ for the imaginary part. Third, a \gls{dac} converts $x^\prime[m]$ and the $p^\prime[m]$ to the analog signals $x^\prime(t)$ and $p^\prime(t)$. Finally, an analog \gls{lp} filter removes aliases and smoothens the output of the \gls{dac}.}\label{fig:pulse_shaping_block}
\end{figure}
The symbols $\alpha[n]$ are first upsampled by an upsampling factor of two by adding one zero-valued samples in between the symbols, i.e.,
\begin{equation}
	\alpha^\prime[m]
	=
	\begin{cases}
		\alpha[m/2] & \text{if}\ m\mod2=0 \\
		0 & \text{otherwise}
	\end{cases}
	.
\end{equation}
For higher upsampling rates as many zero-valued samples are inserted in-between the original samples until the number of output samples equals the number of input samples times the upsampling factor.
Applying a digital filter to the upsampled symbols $\alpha^\prime[n]$, we set the additional samples such that there is a smoother transition between the symbols.
The combined process of upsampling and digital filtering is also referred to as (digital) pulse-shaping.
Typically, a first \gls{rrc} filter with transfer function~\cite[p.~33]{Nossek2015}
\begin{equation}
	h_\text{rrc}\left(f/f_s\right)
	=
	\begin{cases}
		1 & \abs{f/f_s}\leq(1-\alpha) \\
		\cos\left[\frac{\pi}{4\alpha}\left(\abs{f/f_s}-1+\alpha\right)\right] & 1-\alpha\leq\abs{f/f_s}\leq1+\alpha \\
		0 & \text{otherwise}
	\end{cases}
\end{equation}
is used as pulse-shaping filter yielding the pulse-shape samples $u^\prime[n]$.
Combining the first \gls{rrc} for pulse-shaping with a second \gls{rrc} matched-filter at the receiver minimizes intersymbol interference.
At the boundary of the digital and analog domain, the pulse-shape samples $u^\prime[n]$ are converted to an analog signal $u^\prime$ by a \gls{dac}.
The \gls{dac} performs a sample-and-hold operation equivalent to the convolution of the pulse-shaped samples with a rectangular pulse.
In the following, we factorize the transfer function of the \gls{dac} together with the transfer function of the \gls{lp}.
Furthermore, we assume that \gls{dac} is compensated in that the combined transfer function is approximately linear.

To illustrate the pulse-shaping, we show first consider a single symbol $\alpha[15]=1$ as depicted in \Cref{fig:pulse_shaping_unit_time}.
\begin{figure}[htb]
	\centering
	\includestandalone[width=\textwidth]{figures/pgfplots/pulse-shaping-unit-time}
	\caption{Pulse-shaping steps in the time domain for a single unit symbol: A unit symbol sequence $\Re\alpha[n]$ (first row) is upsampled by two (second row) to yield the samples $\Re\alpha^\prime[m]$. The digital \gls{rrc} filter is applied to the samples $x^\prime[m]$ determining the pulse-shape (third row). We can observe the diminishing ripple for the unit response of the filter to reduce the bandwidth. Finally, the pulse-shape $x^\prime[m]$ is converted to an analog signal and filtered by a \gls{lp} filter for smoothing and anti-aliasing.}\label{fig:pulse_shaping_unit_time}
\end{figure}
For complex symbols \Cref{fig:pulse_shaping_rand_time} shows the pulse-shaping.
\begin{figure}[htb]
	\centering
	\includestandalone[width=\textwidth]{figures/pgfplots/pulse-shaping-rand-time}
	\caption{Pulse-shaping steps in the time domain for random symbols from a complex uniform distribution over the interval $[-1,+1]$. The first row shows the real (orange) and imaginary part (blue) of the complex symbols $\alpha[n]$ at their corresponding symbol index $n$. The second row shows the symbols after upsampling to $\alpha^\prime[m]$. The third row shows the samples after applying the \gls{rrc} filter $x^\prime[m]+ip^\prime[m]$. The fourth row shows the anti-aliased analog signal $x(t)+ip(t)$.}\label{fig:pulse_shaping_rand_time}
\end{figure}
Finally, it is of interest how the pulse-shaping affects the frequency domain.
\Cref{fig:pulse_shaping_freq} shows the logarithmic power spectral density of the different pulse-shaping steps for the unit symbol and random symbols.
\begin{figure}[htb]
	\centering
	\includestandalone[width=\textwidth]{figures/pgfplots/pulse-shaping-freq}
	\caption{Pulse-shaping steps in the frequency domain (showing the relative \gls{psd}) for unit symbol (orange) and random symbols from a complex uniform distribution over the interval $[-1,+1]$ (blue). The unit response symbols have a perfectly flat power spectrum (first row). The random symbols have an approximately flat power spectrum (first row). Both symbol spectra (first row) occupy the Nyquist bandwidth. Upsampling doubles the Nyquist bandwidth by aliasing the spectrum (second row). Pulse-shaping acts as a bandpass by strongly suppressing the frequency components outside the original Nyquist bandwidth. Conversion to an analog signal (fourth row) is equivalent to infinite upsampling or adding infinitely many aliases to occupy the complete frequency spectrum. Finally, filtering with a \gls{lp} removes aliases.}\label{fig:pulse_shaping_freq}
\end{figure}

\FloatBarrier
\subsection{Modulation}

We modulate the signals $x(t)$ and $p(t)$ onto an optical carrier with frequency $\omega_c$ for two reasons:
First, the transmission properties of the channel dictates the frequency bands, for instance, \SI{1550}{\nano\meter} in standard telecommunication optical fiber, for efficient transmission.
Second, the orthogonality of sine and cosine, concerning a fixed frequency $\omega_c$, allows transmitting two real signals $x(t)$ and $p(t)$.
\begin{figure}[htb]
	\centering
	\includestandalone{figures/circuitikz/iq-modulator}
	\caption{Block diagram of an electro-optical \gls{iq}-modulator: A sinusoidal oscillator with carrier frequency $\omega_c$ is equally split into two branches. The upper branch is mixed with the baseband signal $p(t)$ while the lower branch is phase-shifted by $\pi/2$ and mixed with the baseband signal $x(t)$. The output of both mixers are added and yield the \gls{iq} passband signal.}\label{fig:iqmod}
\end{figure}
\Cref{fig:iqmod} shows a block diagram of an \gls{iq}-modulator.
A sinusoidal oscillator running at frequency $\omega_c$ is split into two branches with a relative phase shift of $\pi/2$ between the branches.
We define the upper branch as the phase reference.
The upper branch is mixed with the $p(t)$ signal yielding $p(t)\sin(\omega_ct)$.
The lower branch is mixed with the $x(t)$ signal yielding $x(t)\cos(\omega_ct)$.
Finally, upper and lower branch are added to
\begin{equation}
	x(t)
	\cos(\omega_ct)
	+
	p(t)
	\sin(\omega_ct)
	=
	\Re\left\{
		\alpha^\prime(t)
		e^{-i\omega_ct}
	\right\}
	\label{eq:passband_signal}.
\end{equation}
In the signal-processing literature, e.g., Ref.~\cite[p.~25]{Madhow2008}, $x(t)$ and $p(t)$ are referred to as baseband signals while, for instance, $x(t)\cos(\omega_ct)$ is referred to as a passband signal because $x(t)$ would pass a lowpass filter but $x(t)\cos(\omega_ct)$ would only pass a bandpass filter (around $\omega_c$).
\Cref{eq:passband_signal} suggests the definition of the complex baseband signal
\begin{equation}
	\alpha^\prime(t)
	=
	x(t)
	+
	ip(t)
\end{equation}
and the complex passband signal $\alpha^\prime(t)e^{-i\omega_ct}$.
While the physical signal is always real-valued, i.e., \cref{eq:passband_signal}, it is often convenient to work with the complex representation.
In the complex representation, \gls{iq}-modulation is equivalent to the multiplication of the complex baseband $\alpha^\prime(t)$ with $e^{-i\omega_ct}$.
The multiplication of the complex representation of a signal with a complex exponential $e^{-i\omega t}$ is also called up-conversion.


\begin{figure}[htb]
	\centering
	\includegraphics{figures/pstricks/transmitter}
	\caption{Electro-optical transmitter configuration implementing the \gls{iq}-modulator: The output of a coherent laser source with telecommunication wavelength \SI{1550}{\nano\meter} is equally split into two branches. The lower branch is optically phase-shifted by $\pi/2$ and passed to a \gls{mzm} MZM1 which is electro-optically modulated by the analog baseband signal $x(t)$. The upper branch is directly passed to MZM2 which is modulated by the analog baseband signal $p(t)$. The output of MZM1 is combined with MZM2 in an optical coupler. \SI{99}{\percent} of the output power is monitored by a photodiode PD. \SI{1}{\percent} of the output power is passed to a third \gls{mzm} MZM3 which is controlled by a power controller which uses the electrical signal of PD as input. Finally, the output of MZM3 is passed through an optical isolator.}\label{fig:optical_transmitter}
\end{figure}
\Cref{fig:optical_transmitter} shows the configuration of the fiber-optical transmitter.
An ultra-narrow-linewidth continuous-wave laser with center wavelength $\lambda_c=\SI{1550}{\nano\meter}$ is connected to a phase-symmetric and balanced 50:50 coupler which separates into an upper and lower branch.
The upper branch is coupled with a first \gls{mzm}, MZM1.
The lower branch is coupled with a second \gls{mzm}, MZM2.
MZM1 operates in the linear regime and is driven by the baseband signal $x(t)$ while MZM2 is driven by $p(t)e^{i\pi/2}$.
The complex exponential factor of $e^{i\pi/2}$ accounts for the relative phase shift between the branches and can be added in the digital domain before the pulse-shaping.
The output of MZM1 and MZM2 is coupled to a second fiber coupler.
The majority of the output power, \SI{99}{\percent}, of the second fiber coupler is feed into a third \gls{mzm}, MZM3.
The remaining minority of the output power of the second fiber coupler is monitored by a power meter photodiode, PD.
The output of PD is used as the control input for a power controller which controls MZM3.
The power controller operates on a timescale much larger than the modulation time scale and compensates for (thermal) power fluctuations of the laser while ensuring that the output power makes a description in terms of optical coherent state necessary.

% TODO: explain implementation of optical setup (in particular, show relevant quantum properties)
\section{Transmitter}

The transmitter encodes the symbols $\alpha[n]$ onto the coherent state $\ket{\alpha(t)}$ by performing pulse-shaping, modulation and attenuation as illustrated in \Cref{fig:transmitter}.
In the process, the signal is transformed from the digital over the analog electrical to the quantum optical domain.
\begin{figure}[htb]
	\centering
	\includestandalone{figures/tikz/transmitter}
	\caption{Transmitter transforming symbols $\alpha[n]$ to a coherent state $\ket{\alpha(t)}$ by performing pulse-shaping, modulation and attenuation. The modulator marks the boundary between the electrical and optical signal spectrum while the pulse-shaper and attenuator contain the border between the digital and analog, respective analog (classical) and quantum domain.}\label{fig:transmitter}
\end{figure}
The pulse-shaping converts the symbols $\alpha[n]$ from the digital to the signal $u(t)$ in the analog domain.
At the modulator, the electrical signal $u(t)$ is modulated onto an optical carrier using a \gls{mzm}, yielding the classical optical signal $v(t)$.
Finally, a controlled attenuation moves the optical signal $v(t)$ to the quantum regime where an adequate description of the quantum signal is given in terms of the coherent state $\ket{\alpha(t)}$.

\subsection{Pulse-shaping}

\Cref{fig:pulse_shaping_block} depicts the analog and digital signal-processing of the pulse-shaping process.
\begin{figure}[htb]
	\centering
	\includestandalone{figures/circuitikz/pulse-shaping}
	\caption{Block diagram of the digital and analog signal-processing of the pulse-shaping: First, the symbols $\alpha[n]$ are upsampled with an upsampling factor of two yielding samples $\alpha^\prime[n]$. Second, a \gls{rrc} is applied to the samples $\alpha^\prime[n]$ yielding $u^\prime[n]$. Third, a \gls{dac} performs sample-and-hold to output the analog signal $u^\prime(t)$. Finally, an analog \gls{lp} filter smoothens the output of the \gls{dac}.}\label{fig:pulse_shaping_block}
\end{figure}
The symbols $\alpha[n]$ are first upsampled by an upsampling factor of two by adding two zero-valued samples in between the symbols, i.e.,
\begin{equation}
	\alpha^\prime[n]
	=
	\begin{cases}
		\alpha[n/2] & \text{if}\ n\mod2=0 \\
		0 & \text{otherwise}
	\end{cases}
	.
\end{equation}
Upsampling increases the number of samples by the upsampling factor.
Applying a digital filter to the upsampled symbols $\alpha^\prime[n]$, we set the additional samples such that there is a smoother transition between the symbols.
The combined process of upsampling and digital filtering is also referred to as (digital) pulse-shaping.
Typically, a first \gls{rrc} filter with transfer function~\cite[p.~33]{Nossek2015}
\begin{equation}
	h_\text{rrc}\left(f/f_s\right)
	=
	\begin{cases}
		1 & \abs{f/f_s}\leq(1-\alpha) \\
		\cos\left[\frac{\pi}{4\alpha}\left(\abs{f/f_s}-1+\alpha\right)\right] & 1-\alpha\leq\abs{f/f_s}\leq1+\alpha \\
		0 & \text{otherwise}
	\end{cases}
\end{equation}
is used as pulse-shaping filter yielding the pulse-shape samples $u^\prime[n]$.
Combining the first \gls{rrc} for pulse-shaping with a second \gls{rrc} matched-filter at the receiver minimizes intersymbol interference.
At the boundary of the digital and analog domain, the pulse-shape samples $u^\prime[n]$ are converted to an analog signal $u^\prime$ by a \gls{dac}.
The \gls{dac} performs a sample-and-hold operation equivalent to the convolution of the pulse-shaped samples with a rectangular pulse.
\textcolor{red}{Why can we neglect the transfer function of the DAC?}

To illustrate the pulse-shaping, we show first consider a single symbol $\alpha[15]=1$ as depicted in \Cref{fig:pulse_shaping_unit_time}.
\begin{figure}[htb]
	\centering
	\includestandalone[width=\textwidth]{figures/pgfplots/pulse-shaping-unit-time}
	\caption{Pulse-shaping steps in the time domain for a single unit symbol.}\label{fig:pulse_shaping_unit_time}
\end{figure}
For complex symbols \Cref{fig:pulse_shaping_rand_time} shows the pulse-shaping.
\begin{figure}[htb]
	\centering
	\includestandalone[width=\textwidth]{figures/pgfplots/pulse-shaping-rand-time}
	\caption{Pulse-shaping steps in the time domain for random symbols from a complex normal distribution.}\label{fig:pulse_shaping_rand_time}
\end{figure}
Finally, it is of interest how the pulse-shaping affects the frequency domain.
\Cref{fig:pulse_shaping_freq} shows the logarithmic power spectral density of the different pulse-shaping steps for the unit symbol and random symbols.
\begin{figure}[htb]
	\centering
	\includestandalone[width=\textwidth]{figures/pgfplots/pulse-shaping-freq}
	\caption{Pulse-shaping steps in the frequency domain for unit symbol (orange) and random symbols from a complex normal distribution (blue.}\label{fig:pulse_shaping_freq}
\end{figure}

\subsection{Up-conversion}

Let $x(t),p(t)$ be the analog signals reconstructed from the real respective imaginary parts of the complex symbol sequence, then up-conversion multiplies these signals such that the output signal is~\cite[p.~25]{Madhow2008}
\begin{equation}
	\Re\left\{
		\alpha(t)
		e^{-i\omega_0t}
	\right\}
	=
	x(t)
	\cos(\omega_0t)
	+
	p(t)
	\sin(\omega_0t)
\end{equation}
and one refers to $x(t)$ as the in-phase and $p(t)$ as the quadrature component of $\alpha(t)$.

\Cref{fig:iqmod} shows a diagram of an I/Q-modulator which can be used to obtain the previous equation.
\begin{figure}[htb]
	\centering
	\includestandalone{figures/circuitikz/iq-modulator}
	\caption{Diagram of an in-phase and quadrature modulator which can be used for signal up-conversion.}\label{fig:iqmod}
\end{figure}
The output of a sinusoidal oscillator with frequency $\omega_0$ is split into two paths.
The first path is mixed with the the $p(t)$ signal while the second path is phase-shifted by \SI{90}{\degree} and mixed with the $x(t)$ signal.
Finally, the outputs of both mixers are combined to obtain the real part of the complex baseband signal
\begin{equation}
	\alpha(t)
	=
	x(t)
	+
	ip(t)
	.
\end{equation}
One often works with the complex signal as up-conversion takes the simple form of a multiplication.
\begin{figure}[htb]
	\centering
	\includestandalone{figures/circuitikz/up-converter}
	\caption{For a complex signal $\alpha(t)$ up-conversion is simply the multiplication with $e^{-i\omega_0t}$.}
\end{figure}
However, one should keep in mind that the true physical signal is real-valued, thus only the real-part of the complex signal has physical relevance!

\subsection{Attenuation}

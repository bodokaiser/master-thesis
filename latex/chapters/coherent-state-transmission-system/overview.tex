\section{Overview}

In analogy with the classical communication system known from information theory, we propose a coherent state transmission system comprising a transmitter, a channel, and a receiver.
In contrast to the classical communication system, the channel is quantum in nature and maps a time-dependent coherent state from the transmitter to the receiver.
\begin{figure}[htb]
	\centering
	\includestandalone{figures/tikz/transmission-system}
	\caption{Coherent state transmission system comprising a transmitter, a quantum channel, and a receiver. The transmitter encodes the symbols $\alpha[n]$ onto a coherent state $\ket{\alpha(t)}$. The channel maps from the coherent state at the transmitter $\ket{\alpha(t)}$ to a coherent state at the receiver $\ket{\beta(t)}$. The receiver decodes symbols $\beta[n]$ from the coherent state $\ket{\beta(t)}$.}\label{fig:transmission_system}
\end{figure}
\Cref{fig:transmission_system} illustrates the information flow of the coherent state transmission system:
First, a sequence of complex numbers $\alpha[n]$, referred to as symbols, is supplied as input to the transmitter.
The transmitter encodes the symbols $\alpha[n]$ onto a coherent state $\ket{\alpha(t)}$.
The quantum channel takes the transmitter's coherent state $\ket{\alpha(t)}$ and forwards it as another coherent state $\ket{\beta(t)}$ to the receiver.
For an ideal lossless quantum channel, the coherent states are perfectly correlated, i.e., identical.
The receiver decodes the symbols $\beta[n]$ from the coherent state $\ket{\beta(t)}$.
Compared with the classical communication system, the transmitter and receiver symbols are never perfectly correlated, not even for a perfect quantum channel, as the measurement is probabilistic.
\begin{figure}[htb]
	\centering
	\includestandalone{figures/circuitikz/cv-qkd-system}
	\caption{\Gls{cvqkd} transmission system from a signal-processing perspective.}
\end{figure}
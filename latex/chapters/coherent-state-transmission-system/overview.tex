\section{Overview}

The principal objective of a digital transmission system is the transmission of discrete symbols $\left\{\alpha_n\right\}_{n\in\mathbb{N}}$ from a transmitter to a spatially distanced receiver through a physical transmission medium, the communication channel, see \Cref{fig:transmission_system}.
\begin{figure}[htb]
	\centering
	\includestandalone{figures/tikz/transmission-system}
	\caption{Transmission system delivering a symbol $\alpha_n$ from a transmitter to a spatially distanced receiver over a physical channel. The physical channel transforms a continuous signal $\alpha(t)$ at the transmitter to a continuous signal $\beta(t)$ at the receiver from which the receiver extracts the symbol $\beta_n$ correlated with $\alpha_n$.}\label{fig:transmission_system}
\end{figure}
Efficient transmission of information over the physical channel requires the transmitter to encode the discrete symbols $\left\{\alpha_n\right\}_{n\in\mathbb{N}}$ into a continuous signal $\alpha(t)$ tailored to the transmission properties of the channel.
The physical channel's output signal $\beta(t)$ relates to the input signal $\alpha(t)$ by accounting for physical effects like, for instance, dispersion, attenuation, and noise.
Finally, the receiver decodes the symbols $\left\{\beta_n\right\}_{n\in\mathbb{N}}$ from the received signal $\beta(t)$.
In the case of weak light signals, a classical description of the receiver and transmitter signals is insufficient, and a quantum description in terms of the coherent states $\ket{\alpha(t)},\ket{\beta(t)}$ becomes vital.
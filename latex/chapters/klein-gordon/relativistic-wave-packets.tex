\section{Relativistic wave packets}

\begin{definition}[Coordinate wave function]
	The Klein-Gordon states $\ket{\psi}$ coordinate wave function is
	\begin{equation}
		\psi(x^\mu)
		=
		\bra{0}
		\hat\phi(x^\mu)
		\ket{\psi}
		.
	\end{equation}
\end{definition}
\begin{definition}[Relativistic probability current]
	The relativistic probability current is
	\begin{equation}
		j_\mu(x^\mu)
		=
		2
		\Im\left\{
			\psi(x^\mu)^*
			\partial_\mu
			\psi(x^\mu)
		\right\}
		\label{eq:qkg_probability_current}
	\end{equation}
	where $\psi(t,\vb{x})$ is a coordinate wave function.
	The time component $j^0(t,\vb{x})$ is equal to the probability density $\rho(t,\vb{x})$ and the spatial components $j^i(t,\vb{x})$ are equal to the probability current.
\end{definition}
\begin{definition}[Localization]
	The center-of-mass position of the probability density
	\begin{equation}
		\expval{\vb{x}(t)}
		=
		\int\dd[3]{x}
		\vb{x}
		\rho(t,\vb{x})
	\end{equation}
	lets us localize a Klein-Gordon state.\footnote{There exists no position operator free of contradictions in quantum field theory!}
\end{definition}
\begin{definition}[Group velocity]
	The total probability current
	\begin{equation}
		\expval{\vb{v}(t)}
		=
		\int\dd[3]{x}
		\vb{j}(t,\vb{x})
		\label{eq:group_velocity}
	\end{equation}
	equals the group velocity of a field excitation.
\end{definition}
\begin{definition}[Spatial dispersion]
	The spatial dispersion is equal to the variance of the position weighted by the probability density
	\begin{equation}
		\sigma_x(t)^2
		=
		\expval{\vb{x}(t)^2}
		-
		\expval{\vb{x}(t)}^2
	\end{equation}
	and quantifies the spatial spread of a wave packet.
\end{definition}

\begin{definition}[Covariant Gaussian spectrum]
	The spectrum of a Gaussian number state is
	\begin{equation}
		f(\vb{p})
		\propto
		\exp\left\{
			\frac{(p_\mu-k_\mu)(p^\mu-k^\mu)}{4\sigma^2}
		\right\}_{\substack{p_0=\omega(\vb{p})\\k_0=\omega(\vb{k})}}
		\label{eq:covariant_gaussian_spectrum}
	\end{equation}
	where the spectrum has mean $k^\mu=(k_0,\vb{k})$ and variance $\sigma^2$.\footnote{See Ref.~\cite{Naumov2013,Naumov2009} for a in-depth discussion.}
\end{definition}
\begin{restatable}{lemma}{nonrelativisticgaussianmom}\label{thm:non_relativistic_gaussian_momentum}
	For massless particles and $\omega(\vb{p})\ll\sigma$, the covariant Gaussian spectrum can be approximated
	\begin{equation}
		f(\vb{p})
		\propto
		\exp\left\{
			-
			\frac{
				\vb{p}_\perp^2
			}{4\sigma^2}
		\right\}
	\end{equation}
	wherein $\vb{p}_\perp$ is the momentum transverse to $\vb{k}$, for instance, if $\vb{k}=k_0\vu{e}_z$ then $\vb{p}_\perp^2=p_x^2+p_y^2$.
\end{restatable}

\begin{example}
	The smearing function
	\begin{equation*}
		f(p_0,\vb{p})
		=
		\sqrt{2\omega(\vb{p}_0)}
		\left(\frac{2\pi}{\sigma^2}\right)^{\frac{3}{4}}
		\exp\left\{
			-
			\frac{(\vb{p}-\vb{p}_0)^2}{4\sigma^2}
		\right\}
	\end{equation*}
	is normalized
	\begin{equation*}
		\begin{split}
			\int\frac{\dd[3]{p}}{(2\pi)^32\omega(\vb{p})}
			\abs{f\left(\omega(\vb{p}),\vb{p}\right)}^2
			&=
			2\omega(\vb{p}_0)
			\left(\frac{2\pi}{\sigma^2}\right)^{\frac{3}{2}}
			\int\frac{\dd[3]{p}}{(2\pi)^32\omega(\vb{p})}
			\exp\left\{
				-
				\frac{(\vb{p}-\vb{p}_0)^2}{2\sigma^2}
			\right\}
			\\
			&=
			\left(\frac{2\pi}{\sigma^2}\right)^{\frac{3}{2}}
			\left(
				\int\frac{\dd[3]{p}}{2\pi}
				\exp\left\{
					-
					\frac{(p-p_0)^2}{2\sigma^2}
				\right\}
			\right)^3
			\\
			&=
			\left(\frac{2\pi}{\sigma^2}\right)^{\frac{3}{2}}
			\left(
				\frac{1}{2\pi}
				\sqrt{2\pi\sigma^2}
			\right)^3
			=
			1
		\end{split}
	\end{equation*}
	where we used the mean value theorem in the second line.
	The coordinate representation of the smearing function is
	\begin{equation*}
		\begin{split}
			f_1(t,\vb{x})
			&=
			\int\frac{\dd[4]{p}}{(2\pi)^4}
			f(p_0,\vb{p})
			e^{+ip_\mu x^\mu}
			\\
			&=
			\sqrt{2\omega(\vb{p}_0)}
			\left(\frac{2\pi}{\sigma^2}\right)^{\frac{3}{4}}
			\int\frac{\dd[4]{p}}{(2\pi)^4}
			\exp\left\{
				-
				\frac{(\vb{p}-\vb{p}_0)^2}{4\sigma^2}
			\right\}
			e^{+i(p_0t-\vb{p}\vdot\vb{x})}
			\\
			&=
			\sqrt{2\omega(\vb{p}_0)}
			\left(\frac{2\pi}{\sigma^2}\right)^{\frac{3}{4}}
			\int\frac{\dd{p_0}}{2\pi}
			e^{-ip_0t}
			\int\frac{\dd[3]{p}}{(2\pi)^3}
			\exp\left\{
				-
				\frac{(\vb{p}-\vb{p}_0)^2}{4\sigma^2}
				-
				i\vb{p}\vdot\vb{x}
			\right\}
			\\
			&=
			\sqrt{2\omega(\vb{p}_0)}
			\left(\frac{2\pi}{\sigma^2}\right)^{\frac{3}{4}}
			\delta(t)
			e^{-i\vb{p}_0\vdot\vb{x}}
			\left(
				\int\frac{\dd{q}}{2\pi}
				\exp\left\{
					-
					\frac{q^2+2q(2i\sigma^2x)+(2i\sigma^2x)^2-(2i\sigma^2x)^2}{4\sigma^2}
				\right\}
			\right)^3
			\\
			&=
			\sqrt{2\omega(\vb{p}_0)}
			\left(\frac{2\pi}{\sigma^2}\right)^{\frac{3}{4}}
			\delta(t)
			e^{-i\vb{p}_0\vdot\vb{x}}
			e^{-(\sigma\vb{x})^2}
			\left(
				\int\frac{\dd{q}}{2\pi}
				\exp\left\{
					-
					\frac{(q+i\sigma^2x)^2}{4\sigma^2}
				\right\}
			\right)^3
			\\
			&=
			\sqrt{2\omega(\vb{p}_0)}
			\left(\frac{2\pi}{\sigma^2}\right)^{\frac{3}{4}}
			\delta(t)
			e^{-i\vb{p}_0\vdot\vb{x}}
			e^{-(\sigma\vb{x})^2}
			\left(
				\frac{1}{2\pi}
				\sqrt{4\pi\sigma^2}
			\right)^3
			\\
			&=
			\sqrt{2\omega(\vb{p}_0)}
			\left(\frac{2\pi}{\sigma^2}\right)^{\frac{3}{4}}
			\delta(t)
			e^{-i\vb{p}_0\vdot\vb{x}}
			e^{-(\sigma\vb{x})^2}
			\left(
				\frac{\sigma^2}{\pi}
			\right)^\frac{3}{2}
			\\
			&=
			\sqrt{2\omega(\vb{p}_0)}
			\left(\frac{2\sigma^2}{\pi}\right)^{\frac{3}{4}}
			\delta(t)
			e^{-i\vb{p}_0\vdot\vb{x}}
			e^{-(\sigma\vb{x})^2}
			,
		\end{split}
	\end{equation*}
	hence, at time $t=0$, the wave packet is localized at $\vb{x}=0$ with spatial spread $\sigma$.

	We conclude that the smearing function represents the state at some initial time.
	The complete time-dependent coordinate representation is encoded in the wave function
	\begin{equation*}
		\begin{split}
			\psi_1(t,\vb{x})
			&=
			\int\frac{\dd[3]{p}}{(2\pi)^32p_0}
			\eval{
				f(p_0,\vb{p})^*
				e^{+ip_\mu x^\mu}
			}_{p_0=\omega(\vb{p})}
			\\
			&=
			2\omega(\vb{p}_0)
			\left(\frac{2\pi}{\sigma^2}\right)^{\frac{3}{2}}
			\int\frac{\dd[3]{p}}{(2\pi)^32\omega(\vb{p})}
			\exp\left\{
				-
				\frac{(\vb{p}-\vb{p}_0)^2}{4\sigma^2}
			\right\}
			e^{+i\omega(\vb{p})t-i\vb{p}\vdot\vb{x}}
			\\
			&=
			\left(\frac{2\pi}{\sigma^2}\right)^{\frac{3}{2}}
			\int\frac{\dd[3]{p}}{(2\pi)^3}
			\exp\left\{
				-
				\left(
					\frac{\vb{p}-\vb{p}_0}{2\sigma}
				\right)^2
				-
				i\vb{p}\vdot\vb{x}
				+
				i\omega(\vb{p})t
			\right\}
			\\
			&=
			\left(\frac{2}{\pi}\right)^{\frac{3}{2}}
			\int\dd[3]{q}
			\exp\left\{
				-
				\vb{q}^2
				-
				i\left(\vb{p}_0+2\sigma\vb{q}\right)\vdot\vb{x}
				+
				i\omega(\vb{p}_0+2\sigma\vb{q})t
			\right\}
			\\
			&=
			\left(\frac{2}{\pi}\right)^{\frac{3}{2}}
			e^{-i\vb{p}_0\vdot\vb{x}}
			\int\dd[3]{q}
			\exp\left\{
				-
				\vb{q}^2
				-
				2i\sigma
				\vb{q}\vdot\vb{x}
				+
				i\omega(\vb{p}_0+2\sigma\vb{q})t
			\right\}
		\end{split}
	\end{equation*}
	where we again used the mean-value theorem in the second line.
	To make any further progress, we need to expand the dispersion relation
	\begin{equation*}
		\begin{split}
			\omega(\vb{p}_0+2\sigma\vb{q})
			&=
			\omega(\vb{p}_0)
			+
			2\sigma\omega^\prime(\vb{p}_0)\vdot\vb{q}
			+
			4\sigma^2
			\vb{q}^\trans
			\omega^{\prime\prime}(\vb{p}_0)
			\vb{q}
			+
			\order{\vb{q}^3}
			\\
			&=
			\omega(\vb{p}_0)
			+
			2\sigma\vu{p}_0\vdot\vb{q}
			+
			\frac{4\sigma^2}{\omega(\vb{p}_0)}
			q^i
			\left(\delta_{ij}-\frac{p_{0i}p_{0j}}{\vb{p}_0^2}\right)
			q^j
			+
			\order{\vb{q}^3}
			\\
			&=
			\omega(\vb{p}_0)
			+
			2\sigma\vu{p}_0\vdot\vb{q}
			+
			\frac{4\sigma^2}{\omega(\vb{p}_0)}
			\left(
				\vb{q}^2
				-
				(\vu{p}_0\vdot\vb{q})^2
			\right)
			+
			\order{\vb{q}^3}
			\\
			&=
			\omega(\vb{p}_0)
			+
			\omega(2\sigma\vb{q}_\parallel)
			+
			\frac{\omega(2\sigma\vb{q}_\perp)^2}{\omega(\vb{p}_0)}
			+
			\order{\vb{q}^3}
		\end{split}
	\end{equation*}
	where we have used the longitudinal momentum $\vb{q}_\parallel=(\vb{q}\vdot\vu{p}_0)\vu{p}_0$ and the transverse momentum $\vb{q}_\perp^2=\vb{q}^2-\vb{q}_\parallel^2$
	\begin{equation*}
		\psi(x^\mu)
		\approx
		\left(\frac{2}{\pi}\right)^\frac{3}{2}
		e^{i\omega(\vb{p}_0)t-i\vb{p}_0\vdot\vb{x}}
		\int\dd[3]{q}
		\exp\left\{
			-
			\vb{q}^2
			-
			2i\sigma
			\vb{q}
			\vdot
			\vb{x}
			+
			i\omega(2\sigma\vb{q}_\parallel)t
			+
			i\frac{\omega(2\sigma\vb{q}_\perp)^2}{\omega(\vb{p}_0)}t
		\right\}
	\end{equation*}
	Switching to cylindrical coordinates and further evaluating the integral
	\begin{equation*}
		\begin{split}
			&\
			\int\dd{q_\perp}\dd{q_\parallel}\dd{\theta} q_\perp
			\exp\left\{
				-
				q_\perp^2
				-
				q_\parallel^2
				-
				2i\sigma
				\left(
					q_\perp
					\cos\theta
					x^1
					+
					q_\perp
					\sin\theta
					x^2
					+
					q_\parallel
					x^3
				\right)
				+
				i2\sigma q_\parallel t
				+
				i\frac{4\sigma^2 q_\perp^2}{\omega(\vb{p}_0)}t
			\right\}
			\\
			=&\
			\int\dd{q_\perp}\dd{\theta} q_\perp
			\exp\left\{
				-
				q_\perp^2
				-
				2i\sigma
				q_\perp
				\left(
					\cos\theta
					x^1
					+
					\sin\theta
					x^2
				\right)
				+
				i\frac{4\sigma^2 q_\perp^2}{\omega(\vb{p}_0)}t
			\right\}
			\int\dd{q_\parallel}
			\exp\left\{
				-
				q_\parallel^2
				-
				2i\sigma
				q_\parallel x^3
				+
				2i\sigma q_\parallel t
			\right\}
		\end{split}		
	\end{equation*}
	The second integral is
	\begin{equation*}
		\begin{split}
			&\
			\int\dd{q_\parallel}
			\exp\left\{
				-
				q_\parallel^2
				-
				2q_\parallel i\sigma (x^3-t)
				-
				\left(i\sigma (x^3-t)\right)^2
				+
				\left(i\sigma (x^3-t)\right)^2
			\right\}
			\\
			=&\
			e^{-\sigma(x^3-t)^2}
			\int\dd{q_\parallel}
			\exp\left\{
				-
				\left(
					q_\parallel^2
					+
					i\sigma (x^3-t)
				\right)^2
			\right\}
			\\
			=&\
			\sqrt{\pi}
			e^{-\sigma^2(x^3-t)^2}
		\end{split}
	\end{equation*}
	and the first integral becomes
	\begin{equation*}
		\begin{split}
			&\
			\int\dd{q_\perp}\dd{\theta} q_\perp
			\exp\left\{
				-
				q_\perp^2
				-
				2i\sigma
				q_\perp
				\left(
					\cos\theta
					x^1
					+
					\sin\theta
					x^2
				\right)
				+
				i\frac{4\sigma^2 q_\perp^2}{\omega(\vb{p}_0)}t
			\right\}
			\\
			=&\
			\int\dd{q_\perp}q_\perp
			\exp\left\{
				-
				\left(
					1
					-
					4i\frac{\sigma^2t}{\omega(\vb{p}_0)}
				\right)
				q_\perp^2
			\right\}
			\int\dd{\theta}
			\exp\left\{
				-
				2i\sigma
				q_\perp
				\left(
					\cos\theta
					x^1
					+
					\sin\theta
					x^2
				\right)
			\right\}
			\\
			=&\
			\int\dd{q_\perp}q_\perp
			\exp\left\{
				-
				\left(
					1
					-
					4i\frac{\sigma^2t}{\omega(\vb{p}_0)}
				\right)
				q_\perp^2
			\right\}
			2\pi
			I_0\left(2i\sigma q_\perp\rho\right)
			\\
			\approx &\
			\frac{2\pi}{2\left(1-4i\sigma^2 t/\omega(\vb{p}_0)\right)}
		\end{split}
	\end{equation*}
	Putting these things together, we have for $\rho\ll1$
	\begin{equation*}
		\begin{split}
			\psi(t,\rho,z)
			&\approx
			\left(\frac{2}{\pi}\right)^\frac{3}{2}
			e^{i\omega(\vb{p}_0)t-i\vb{p}_0\vdot\vb{x}}
			\frac{\pi}{\left(1-4i\sigma^2 t/\omega(\vb{p}_0)\right)}
			\sqrt{\pi}
			e^{-\sigma^2(z-t)^2}
			\\
			&=
			\frac{2^\frac{3}{2}e^{-\sigma^2(z-t)}}{1-4i\sigma^2 t/\omega(\vb{p}_0)}
			e^{i\omega(\vb{p}_0)t-i\vb{p}_0\vdot\vb{x}}
		\end{split}
	\end{equation*}
	comparing this to the smearing function, we find
	\begin{align*}
		f(t,\vb{x})
		&\approx
		\sqrt{2\omega(\vb{p}_0)}
		\left(\frac{2\sigma^2}{\pi}\right)^{\frac{3}{4}}
		\delta(t)
		e^{-i\vb{p}_0\vdot\vb{x}}
		e^{-(\sigma\vb{x})^2}
		\\
		\psi(t,\rho,z)
		&\approx
		\frac{2^\frac{3}{2}e^{-\sigma^2(z-t)}}{1-4i\sigma^2 t/\omega(\vb{p}_0)}
		e^{i\omega(\vb{p}_0)t-i\vb{p}_0\vdot\vb{x}}
	\end{align*}
	which for $t=0$ are equal up to a proportionality constant.
\end{example}
\section{Relativistic wave packets}

\begin{definition}[Coordinate wave function]
	The Klein-Gordon states $\ket{\psi}$ coordinate wave function is
	\begin{equation}
		\psi(x^\mu)
		=
		\bra{0}
		\hat\phi(x^\mu)
		\ket{\psi}
		.
	\end{equation}
\end{definition}
\begin{definition}[Relativistic probability current]
	The relativistic probability current is
	\begin{equation}
		j_\mu(x^\mu)
		=
		2
		\Im\left\{
			\psi(x^\mu)^*
			\partial_\mu
			\psi(x^\mu)
		\right\}
		\label{eq:qkg_probability_current}
	\end{equation}
	where $\psi(t,\vb{x})$ is a coordinate wave function.
	The time component $j^0(t,\vb{x})$ is equal to the probability density $\rho(t,\vb{x})$ and the spatial components $j^i(t,\vb{x})$ are equal to the probability current.
\end{definition}
\begin{definition}[Localization]
	The center-of-mass position of the probability density
	\begin{equation}
		\expval{\vb{x}(t)}
		=
		\int\dd[3]{x}
		\vb{x}
		\rho(t,\vb{x})
	\end{equation}
	lets us localize a Klein-Gordon state.\footnote{There exists no position operator free of contradictions in quantum field theory!}
\end{definition}
\begin{definition}[Group velocity]
	The total probability current
	\begin{equation}
		\expval{\vb{v}(t)}
		=
		\int\dd[3]{x}
		\vb{j}(t,\vb{x})
		\label{eq:group_velocity}
	\end{equation}
	equals the group velocity of a field excitation.
\end{definition}
\begin{definition}[Spatial dispersion]
	The spatial dispersion is equal to the variance of the position weighted by the probability density
	\begin{equation}
		\sigma_x(t)^2
		=
		\expval{\vb{x}(t)^2}
		-
		\expval{\vb{x}(t)}^2
	\end{equation}
	and quantifies the spatial spread of a wave packet.
\end{definition}

\begin{definition}[Covariant Gaussian spectrum]
	The spectrum of a Gaussian number state is
	\begin{equation}
		f(\vb{p})
		\propto
		\exp\left\{
			\frac{(p_\mu-k_\mu)(p^\mu-k^\mu)}{4\sigma^2}
		\right\}_{\substack{p_0=\omega(\vb{p})\\k_0=\omega(\vb{k})}}
		\label{eq:covariant_gaussian_spectrum}
	\end{equation}
	where the spectrum has mean $k^\mu=(k_0,\vb{k})$ and variance $\sigma^2$.\footnote{See Ref.~\cite{Naumov2013,Naumov2009} for a in-depth discussion.}
\end{definition}
\begin{restatable}{lemma}{nonrelativisticgaussianmom}\label{thm:non_relativistic_gaussian_momentum}
	For massless particles and $\omega(\vb{p})\ll\sigma$, the covariant Gaussian spectrum can be approximated
	\begin{equation}
		f(\vb{p})
		\propto
		\exp\left\{
			-
			\frac{
				\vb{p}_T^2
			}{4\sigma^2}
		\right\}
	\end{equation}
	wherein $\vb{p}_T$ is the momentum transverse to $\vb{k}$, for instance, if $\vb{k}=k_0\vu{e}_z$ then $\vb{p}_T^2=p_x^2+p_y^2$.
\end{restatable}
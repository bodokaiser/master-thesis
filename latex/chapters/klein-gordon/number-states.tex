\section{Number states}

\begin{definition}[Single-particle number state]\label{def:single_particle_number_state}
	The single-particle number state with Lorentz-invariant spectrum $f(\vb{p})=f(\omega(\vb{p}),\vb{p})\in\mathcal{S}(\mathbb{R}^4,\mathbb{C})$ is
	\begin{equation}
		\ket{1_f}
		=
		\int\frac{\dd[3]{p}}{(2\pi)^3\sqrt{2\omega(\vb{p})}}
		f(\vb{p})
		\hat{a}^\dagger(\vb{p})
		\ket{0}
		.
	\end{equation}
\end{definition}
\begin{lemma}\label{thm:single_particle_number_state_smeared_kg}
	The single-particle number state is equal to the smeared Klein-Gordon field operator acting on the vacuum state
	\begin{equation}
		\hat\phi[f]
		\ket{0}
		=
		\int\dd[4]{x}
		f(x)
		\hat\phi(x)
		\ket{0}
		=
		\ket{1_f}
	\end{equation}
	known from Wightman quantum field theory.\footnote{Wightman quantum field theory defines quantum field operators as operator-valued distributions acting on the Schwartz function space, see Ref.~\cite{Bogolubov1989} and Ref.~\cite{Streater2016}.}
\end{lemma}
\begin{lemma}\label{th:single_particle_number_states_inner_product}
	Let $\ket{1_f},\ket{1_g}$ be two single-particle number states then their inner product is equal to
	\begin{equation}
		\braket{1_g}{1_f}
		=
		\int\frac{\dd[3]{p}}{(2\pi)^32\omega(\vb{p})}
		f(\vb{p})g(\vb{p})^*
	\end{equation}
	indicating that the single-particle number states are overcomplete.
\end{lemma}
\begin{lemma}\label{thm:single_particle_number_state_normalization}
	The spectrum of a single-particle number state satisfies
	\begin{equation}
		\bra{1_f}\ket{1_f}
		=
		\int\frac{\dd[3]{p}}{(2\pi)^32\omega(\vb{p})}
		\abs{f(\vb{p})}^2
		=
		1
		.
	\end{equation}
\end{lemma}
Let us consider some examples for the single-particle number state spectrum $f(\vb{p})$.
\begin{example}
	The spectrum of a Gaussian single-particle number state is
	\begin{equation}
		f(\vb{p})
		\propto
		\exp\left\{
			\frac{(p_\mu-k_\mu)(p^\mu-k^\mu)}{4\sigma_P^2}
		\right\}
		\label{eq:covariant_gaussian_spectrum}
	\end{equation}
	where the spectrum has mean $k^\mu=(k_0,\vb{k})$ and variance $\sigma^2$.\footnote{See Ref.~\cite{Naumov2013,Naumov2009} for a in-depth discussion.}
\end{example}
\begin{theorem}\label{thm:single_particle_number_state_number_eigenstate}
	The single-particle number state $\ket{1_f}$ is an eigenstate of the number operator $\hat{N}$
	\begin{equation}
		\hat{N}
		\ket{1_f}
		=
		1
		\ket{1_f}
	\end{equation}
	with eigenvalue $1$.
\end{theorem}
\begin{lemma}\label{thm:single_particle_number_state_energy}
	The first energy moment of the single-particle number state $\ket{1_f}$ is
	\begin{equation}
		\expval{\hat{H}}{1_f}
		=
		\int\frac{\dd[3]{p}}{(2\pi)^3\sqrt{2\omega(\vb{p})}}
		\omega(\vb{p})
		\abs{f(\vb{p})}^2
	\end{equation}
	and the second moment is
	\begin{equation}
		\expval{\hat{H}^2}{1_f}
		=
		\int\frac{\dd[3]{p}}{(2\pi)^3\sqrt{2\omega(\vb{p})}}
		\omega(\vb{p})^2
		\abs{f(\vb{p})}^2
		.
	\end{equation}
\end{lemma}
\begin{corollary}
	The single-particle number state has non-zero energy fluctuations, i.e.,
	\begin{equation}
		\expval{\left(\Delta\hat{H}\right)^2}{1_f}
		>
		0
		.
	\end{equation}
\end{corollary}
\begin{lemma}\label{thm:single_particle_number_state_wave_function}
	The coordinate wave function of a single-particle number state is
	\begin{equation}
		\psi(t,\vb{x})
		=
		\int\frac{\dd[3]{p}}{(2\pi)^32\omega(\vb{p})}
		f(\vb{p})e^{-ip_\mu x^\mu}
		.
	\end{equation}
\end{lemma}
\begin{lemma}\label{thm:single_particle_number_state_group_velocity}
	The group velocity of a single-particle number state $\ket{1_f}$ is
	\begin{equation}
		\expval{\vb{v}}
		=
		\int\frac{\dd[3]{p}}{(2\pi)^32\omega(\vb{p})}
		\abs{f(\vb{p})}^2
		\frac{\vb{p}}{\omega(\vb{p})}
		\label{eq:single_particle_number_state_group_velocity}
		.
	\end{equation}
\end{lemma}
\begin{lemma}
	The single-particle number state is localized on a trajectory
	\begin{equation}
		\expval{\vb{x}(t)}
		=
		\expval{\vb{v}}t
	\end{equation}
	moving with the group velocity $\expval{\vb{v}}$.
\end{lemma}
Finally, we calculate the quantum statistics for the momentum density operator.
The momentum density operator will turn out to capture the physical field-strength.
\begin{lemma}\label{thm:single_particle_number_state_momentum_density_mean}
	The mean momentum density of the single-particle number state is zero, i.e.,
	\begin{equation}
		\expval{\hat\pi(t,\vb{x})}{1_f}
		=
		0
	\end{equation}
\end{lemma}
\begin{lemma}\label{thm:single_particle_number_state_momentum_density_correlation}
	The correlation function of the momentum density operator is
	\begin{equation}
		\expval{\hat\pi(x^\mu)\hat\pi(y^\mu)}{1_f}
		=
		\expval{\left(\Delta\hat\pi(x^\mu,y^\mu)\right)^2}{0}
		+
		2\Re\left\{
			\psi(x^\mu)
			\psi(y^\mu)^*
		\right\}
	\end{equation}
\end{lemma}
\begin{corollary}
	The variance of the momentum density for the single-particle number state is
	\begin{equation}
		\expval{\left(\Delta\hat\pi(t,\vb{x})\right)^2}{1_f}
		=
		\Delta\pi_0
		+
		2
		\abs{\psi(t,\vb{x})}^2
		.
	\end{equation}
\end{corollary}
Later, we are going to need
\begin{lemma}\label{thm:single_partiicle_number_state_inner_product_pn_smeared_kg_comm}
	The commutator of the smeared positive and negative frequency Klein-Gordon operators is equal to the overlap of the corresponding single-particle number states, i.e.,
	\begin{equation}
		\comm{\hat\phi^+[f]}{\hat\phi^-[g]}
		=
		\braket{1_f}{1_g}
		.
	\end{equation}
\end{lemma}

\begin{definition}[Multi-particle number state]
	A multi-particle number state with $n$ particles and spectrum $f$ is given by
	\begin{equation}
		\ket{n_f}
		=
		\frac{1}{\sqrt{n!}}
		\hat\phi[f]^n
		\ket{0}
		=
		\frac{1}{\sqrt{n!}}
		\left(
			\int\frac{\dd[3]{p}}{(2\pi)^3\sqrt{2\omega(\vb{p})}}
			f(\vb{p})
			\hat{a}^\dagger(\vb{p})
		\right)^n
		\ket{0}
	\end{equation}
\end{definition}
\begin{theorem}
	The multi-particle number state $\ket{n_f}$ is an eigenstate of the number operator $\hat{N}$ to eigenvalue $n$, i.e.,
	\begin{equation}
		\hat{N}
		\ket{n_f}
		=
		n
		\ket{n_f}
		.
	\end{equation}
\end{theorem}
\begin{corollary}
	The multi-particle number state $\ket{n_f}$ has zero uncertainty in the number observable.
\end{corollary}
\begin{theorem}\label{thm:multi_particle_number_state_inner_product}
	Let $\ket{n_f}$ be an $n$-particle number state with spectrum $f(\vb{p})$ and $\ket{m_g}$ a $m$-particle number state with spectrum $g(\vb{p})$, then their inner product is
	\begin{equation}
		\braket{n_f}{m_g}
		=
		\delta_{nm}
		\left(
			\int\frac{\dd[3]{p}}{(2\pi)^32\omega(\vb{p})}
			f(\vb{p})^*
			g(\vb{p})
		\right)^n
		.
	\end{equation}
\end{theorem}
\begin{corollary}
	Multi-particle number states with the same spectrum $f(\vb{p})$ are orthogonal.
\end{corollary}
\begin{lemma}
	The momentum density correlation of the multi-particle number state is
	\begin{equation}
		\expval{\hat\pi(x)\hat\pi(y)}{n_f}
		=
	\end{equation}
\end{lemma}
\begin{proof}
	\begin{equation*}
		\begin{split}
			\expval{
				\hat\pi(x)
				\hat\pi(y)
			}{n_f}
			&=
			\frac{1}{n!}
			\expval{
				\hat\phi[f]^n
				\hat\pi(x)
				\hat\pi(y)
				\hat\phi[f]^n
			}{0}
			\\
			&=
			\frac{1}{n!}
			\int\dd[4]{x_1}
			f(x_1)
			\dots
			\int\dd[4]{x_n}
			f(x_n)
			\int\dd[4]{y_1}
			f(y_1)
			\dots
			f(y_n)
			\int\dd[4]{y_n}
			\\
			&\times
			\expval{
				\hat\phi(x_1)
				\dots
				\hat\phi(x_n)
				\hat\pi(x)
				\hat\pi(y)
				\hat\phi(y_1)
				\dots
				\hat\phi(y_n)
			}{0}			
		\end{split}
	\end{equation*}
\end{proof}
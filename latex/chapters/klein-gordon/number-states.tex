\section{Number states}

\begin{definition}[Number state]\label{def:number_state}
	The $n$-particle number state with spectrum $f(\vb{p})$ is
	\begin{equation}
		\ket{n_f}
		=
		\frac{1}{\sqrt{n!}}
		\left(
			\int\frac{\dd[3]{p}}{(2\pi)^3\sqrt{2\omega(\vb{p})}}
			f(\vb{p})
			\hat{a}^\dagger(\vb{p})
		\right)^n
		\ket{0}
	\end{equation}
	and the spectrum is required to satisfy
	\begin{equation}
		\int\frac{\dd[3]{p}}{(2\pi)^32\omega(\vb{p})}
		\abs{f(\vb{p})}^2
		=
		1
		\label{eq:number_state_spectrum_constraint}
		.
	\end{equation}
\end{definition}
In axiomatic quantum field theory, e.g., Wightman quantum field theory, we define quantum field operators as operator-valued distributions acting on the Schwartz function space, see Ref.~\cite{Bogolubov1989} and Ref.~\cite{Streater2016}.
It turns out that there is a deep connection between the spectrum $f(\vb{p})$ and the smearing function $f(x^\mu)\in\mathcal{S}(\mathbb{R}^4,\mathbb{R})$ of the negative Klein-Gordon operator.
\begin{lemma}
	Let $f(x^\mu)$ be a smearing function with Fourier transform $f(p^\mu)=f(p_0,\vb{p})$ satisfying the spectral constraint \cref{eq:number_state_spectrum_constraint} with $f(\vb{p})=f(\omega(\vb{p}),\vb{p})$, and $\hat\phi^-(x^\mu)$ the negative frequency Klein-Gordon operator as defined in Wightman quantum field theory, then
	\begin{equation}
		\ket{n_f}
		=
		\frac{1}{\sqrt{n!}}
		\hat\phi^-[f]^n
		\ket{0}
		=
		\frac{1}{\sqrt{n!}}
		\left(
			\int\dd[4]{x}
			f(x^\mu)
			\hat\phi^-(x^\mu)
		\right)^n
		\ket{0}
		,
	\end{equation}
	i.e., the number state is a specific kind of smearing function acting together with the negative frequency Klein-Gordon operator on the vacuum state.
\end{lemma}
\begin{theorem}
	The multi-particle number state $\ket{n_f}$ is an eigenstate of the number operator $\hat{N}$ to eigenvalue $n$, i.e.,
	\begin{equation}
		\hat{N}
		\ket{n_f}
		=
		n
		\ket{n_f}
		.
	\end{equation}
\end{theorem}
\begin{theorem}\label{thm:number_state_inner_product}
	Let $\ket{n_f}$ be an $n$-particle number state with spectrum $f(\vb{p})$ and $\ket{m_g}$ a $m$-particle number state with spectrum $g(\vb{p})$, then their inner product is
	\begin{equation}
		\braket{n_f}{m_g}
		=
		\delta_{nm}
		\left(
			\int\frac{\dd[3]{p}}{(2\pi)^32\omega(\vb{p})}
			f(\vb{p})^*
			g(\vb{p})
		\right)^n
		.
	\end{equation}
\end{theorem}
\begin{corollary}
	Multi-particle number states with the same spectrum $f(\vb{p})$ are orthogonal, i.e.,
	\begin{equation}
		\braket{n_f}{m_f}
		=
		\delta_{nm}
		.
	\end{equation}
\end{corollary}
\begin{example}
	The spectrum of a Gaussian number state is
	\begin{equation}
		f(\vb{p})
		\propto
		\exp\left\{
			-
			\frac{(p_\mu-k_\mu)(p^\mu-k^\mu)}{4\sigma_P^2}
		\right\}
		\label{eq:covariant_gaussian_spectrum}
	\end{equation}
	where the spectrum has mean $k^\mu=(k_0,\vb{k})$ and variance $\sigma^2$.\footnote{See Ref.~\cite{Naumov2013,Naumov2009} for a in-depth discussion.}
\end{example}
\begin{lemma}\label{thm:single_particle_number_state_wave_function}
	The coordinate wave function of a single-particle number state is
	\begin{equation}
		\psi(t,\vb{x})
		=
		\int\frac{\dd[3]{p}}{(2\pi)^32\omega(\vb{p})}
		\eval{
			f(\vb{p})
			e^{-ip_\mu x^\mu}
		}_{p_0=\omega(\vb{p})}
		.
	\end{equation}
\end{lemma}
\begin{lemma}\label{thm:single_particle_number_state_group_velocity}
	The group velocity of a single-particle number state $\ket{1_f}$ is
	\begin{equation}
		\expval{\vb{v}}
		=
		\int\frac{\dd[3]{p}}{(2\pi)^32\omega(\vb{p})}
		\abs{f(\vb{p})}^2
		\frac{\vb{p}}{\omega(\vb{p})}
		\label{eq:single_particle_number_state_group_velocity}
		.
	\end{equation}
\end{lemma}
\begin{lemma}
	The single-particle number state is localized on a trajectory
	\begin{equation}
		\expval{\vb{x}(t)}
		=
		\expval{\vb{v}}t
	\end{equation}
	moving with the group velocity $\expval{\vb{v}}$.
\end{lemma}
Later, we are going to need
\begin{lemma}\label{thm:single_partiicle_number_state_inner_product_pn_smeared_kg_comm}
	The commutator of the smeared positive and negative frequency Klein-Gordon operators is equal to the overlap of the corresponding single-particle number states, i.e.,
	\begin{equation}
		\comm{\hat\phi^+[f]}{\hat\phi^-[g]}
		=
		\braket{1_f}{1_g}
		.
	\end{equation}
\end{lemma}
\section{Coherent states}

\begin{definition}[Displacement operator]
	The displacement operator with spectrum $\alpha(\vb{p})$ is
	\begin{equation}
		\hat{D}[\alpha]
		=
		\exp\left\{
			\int\frac{\dd[3]{p}}{(2\pi)^3\sqrt{2\omega(\vb{p})}}
			\left\{
				\alpha(\vb{p})
				\hat{a}^\dagger(\vb{p})
				-
				\alpha(\vb{p})^*
				\hat{a}(\vb{p})
			\right\}
		\right\}
		\label{eq:qkg_displacement_operator}
		.
	\end{equation}
\end{definition}
\begin{corollary}\label{thm:qkg_displacement_smeared}
	The displacement operators can be expressed in terms of the smeared positive and negative frequency Klein-Gordon operators
	\begin{equation}
		\hat{D}[\alpha]
		=
		\exp\left\{
			\hat\phi^-[\alpha]
			-
			\hat\phi^+[\alpha]
		\right\}
		.
	\end{equation}
\end{corollary}
\begin{restatable}{lemma}{qkgdisplacementordered}\label{thm:qkg_displacement_ordered}
	The displacement operator can be put in normal-
	\begin{equation}\label{eq:qkg_displacement_normal}
		\hat{D}[\alpha]
		=
		\exp\left\{
			+\hat\phi^-[\alpha]
		\right\}
		\exp\left\{
			-\hat\phi^+[\alpha]
		\right\}
		\exp\left\{
			-
			\frac{1}{2}
			\comm{\hat\phi^+[\alpha]}{\hat\phi^-[\alpha]}
		\right\}
	\end{equation}
	and antinormal-order
	\begin{equation}\label{eq:qkg_displacement_antinormal}
		\hat{D}[\alpha]
		=
		\exp\left\{
			-\hat\phi^+[\alpha]
		\right\}
		\exp\left\{
			+\hat\phi^-[\alpha]
		\right\}
		\exp\left\{
			+
			\frac{1}{2}
			\comm{\hat\phi^+[\alpha]}{\hat\phi^-[\alpha]}
		\right\}
	\end{equation}
	wherein the commutator evaluates to
	\begin{equation}
		\comm{\hat\phi^+[\alpha]}{\hat\phi^-[\alpha]}
		=
		\int\frac{\dd[3]{p}}{(2\pi)^32\omega(\vb{p})}
		\abs{\alpha(\vb{p})}^2
	\end{equation}
	in momentum space.
\end{restatable}
\begin{restatable}{lemma}{qkgdisplacementproduct}\label{thm:qkg_displacement_product}
	Let $\hat{D}[\alpha],\hat{D}[\beta]$ be two displacement operators with spectrum $\alpha(\vb{p}),\beta(\vb{p})$, then their product equals
	\begin{equation}\label{eq:qkg_displacement_product}
		\begin{split}
			\hat{D}[\alpha]
			\hat{D}[\beta]
			&=
			\hat{D}[\alpha+\beta]
			\exp\left\{
				-
				\frac{1}{2}
				\comm{\hat\phi^+[\alpha]}{\hat\phi^-[\beta]}
				+
				\frac{1}{2}
				\comm{\hat\phi^+[\beta]}{\hat\phi^-[\alpha]}
			\right\}
			\\
			&=
			\hat{D}[\alpha+\beta]
			\exp\left\{
				-
				\frac{1}{2}
				\iint\frac{\dd[3]{p}}{(2\pi)^32\omega(\vb{p})}
				\left\{
					\alpha(\vb{p})^*
					\beta(\vb{p})
					-
					\alpha(\vb{p})
					\beta(\vb{p})^*
				\right\}
			\right\}
			.
		\end{split}
	\end{equation}
\end{restatable}
\begin{restatable}{lemma}{qkgdisplacementunitary}\label{thm:qkg_displacement_unitary}
	The displacement operator is unitary
	\begin{equation}
		\hat{D}[\alpha]^{-1}
		=
		\hat{D}[\alpha]^\dagger
		=
		\hat{D}[-\alpha]
		.
	\end{equation}
\end{restatable}
\begin{definition}[Coherent state]
	A coherent state $\ket{\alpha}$ with spectrum $\alpha(\vb{p})$
	\begin{equation}
		\begin{split}
			\ket{\alpha}
			&=
			\exp\left\{
				-
				\frac{1}{2}
				\comm{\hat\phi^+[\alpha]}{\hat\phi^-[\alpha]}
			\right\}
			\exp\left\{
				+\hat\phi^-[\alpha]
			\right\}
			\ket{0}
			\\
			&=
			\exp\left\{
				-
				\frac{1}{2}
				\int\frac{\dd[3]{p}}{(2\pi)^32\omega(\vb{p})}
				\abs{\alpha(\vb{p})}^2
			\right\}
			\exp\left\{
				\int\frac{\dd[3]{p}}{(2\pi)^3\sqrt{2\omega(\vb{p})}}
				\alpha(\vb{p})
				\hat{a}^\dagger(\vb{p})
			\right\}
			\ket{0}
		\end{split}
	\end{equation}
	is a coherent superposition of number states without constrained spectrum.
\end{definition}
\begin{restatable}{lemma}{qkgdisplacementvacuum}\label{thm:qkg_displacement_vacuum}
	The displacement operator creates a coherent state from the vacuum
	\begin{equation}\label{eq:qkg_displacement_vacuum}
		\hat{D}[\alpha]
		\ket{0}
		=
		\ket{\alpha}
		.
	\end{equation}
\end{restatable}
\begin{corollary}
	The coherent state is normalized\footnote{In contrast to the number state, there is no constraint on the spectrum of the coherent state $\alpha(\vb{p})$.}
	\begin{equation}
		\braket{\alpha}
		=
		1
		.
	\end{equation}
\end{corollary}
\begin{restatable}{theorem}{qkgcoherenteigenstate}\label{thm:qkg_coherent_state_eigenstate}
	The coherent state is an eigenstate of the annihilation operator to eigenvalue $\alpha(\vb{p})/\sqrt{2\omega(\vb{p})}$, i.e.,
	\begin{equation}\label{eq:qkg_coherent_state_eigenstate}
		\hat{a}(\vb{p})
		\ket{\alpha}
		=
		\frac{\alpha(\vb{p})}{\sqrt{2\omega(\vb{p})}}
		\ket{\alpha}
		.
	\end{equation}
\end{restatable}
\begin{restatable}{lemma}{qkgcoherentenergy}\label{thm:qkg_coherent_state_energy}
	The mean energy of the coherent state is
	\begin{equation}
		\expval{\hat{H}}{\alpha}
		=
		\int\frac{\dd[3]{p}}{(2\pi)^32\omega(\vb{p})}
		\omega(\vb{p})
		\abs{\alpha(\vb{p})}^2
	\end{equation}
	and the variance is
	\begin{equation}
		\expval{\left(\Delta\hat{H}\right)^2}{\alpha}
		=
		\int\frac{\dd[3]{p}}{(2\pi)^32\omega(\vb{p})}
		\omega(\vb{p})^2
		\abs{\alpha(\vb{p})}^2
		.
	\end{equation}
\end{restatable}
Although, the mean energy of the coherent state is the same as the mean energy of the single-particle number state, the spectrum $\alpha(\vb{p})$ of the coherent state is not bound.
\begin{restatable}{lemma}{qkgcoherentnumber}\label{thm:qkg_coherent_state_number}
	The mean number of particles is
	\begin{equation}
		\overline{n}
		=
		\expval{\hat{N}}{\alpha}
		=
		\int\frac{\dd[3]{p}}{(2\pi)^32\omega(\vb{p})}
		\abs{\alpha(\vb{p})}^2
	\end{equation}
	and the variance is
	\begin{equation}
		\expval{\left(\Delta\hat{N}\right)^2}{\alpha}
		=
		\int\frac{\dd[3]{p}}{(2\pi)^32\omega(\vb{p})}
		\abs{\alpha(\vb{p})}^2
		=
		\overline{n}
		,
	\end{equation}
	i.e., the particle number is Poisson distributed.
\end{restatable}
\begin{restatable}{lemma}{qkgcoherentnumberinnerproduct}\label{thm:qkg_coherent_state_number_state_inner_product}
	The inner product between an $n$-particle number state with spectrum $f(\vb{p})$ and a coherent state with spectrum $\alpha(\vb{p})$ is
	\begin{equation}
		\braket{n_f}{\alpha}
		=
		\frac{1}{\sqrt{n!}}
		\left(
			\int\frac{\dd[3]{p}}{(2\pi)^32\omega(\vb{p})}
			f(\vb{p})^*
			\alpha(\vb{p})
		\right)^n
		e^{-\overline{n}/2}
	\end{equation}
	where $\overline{n}$ is the mean particle number of the coherent state.
\end{restatable}
\begin{corollary}
	If $\alpha(\vb{p})=f(\vb{p})$ the former inner product reduces to
	\begin{equation}
		\braket{n_f}{\alpha}
		=
		\frac{1}{\sqrt{n!}}
		e^{-\overline{n}/2}
	\end{equation}
	which absolute squared is a Poisson distribution with unit variance.
\end{corollary}
\begin{corollary}
	If the product of the spectrums is equal to
	\begin{equation}
		f(\vb{p})
		\alpha(\vb{p})
		=
		2\omega(\vb{p})
		(2\pi)^3
		\delta(\vb{p}-\vb{k})
	\end{equation}
	the inner product reduces to
	\begin{equation}
		\braket{n_f}{\alpha}
		=
		\frac{\alpha^n}{\sqrt{n!}}
		e^{-\overline{n}/2}
	\end{equation}
	and we recover the Poisson distribution known from single-mode quantum optics
	\begin{equation}
		p_n
		=
		\abs{\braket{n_f}{\alpha}}^2
		=
		\frac{\overline{n}^n}{\sqrt{n!}}
		e^{-\overline{n}}
	\end{equation}
	where $\overline{n}=\abs{\alpha}^2$.
\end{corollary}
\begin{restatable}{lemma}{qkgcoherentinnerproduct}\label{thm:coherent_state_inner_product}
	Let $\ket{\alpha},\ket{\beta}$ be two coherent states, then their inner product is
	\begin{equation}
		\begin{split}
			\braket{\alpha}{\beta}
			&=
			\exp\left\{
				-
				\frac{1}{2}
				\comm{\hat\phi^+[\alpha]}{\hat\phi^-[\alpha]}
				+
				\comm{\hat\phi^+[\alpha]}{\hat\phi^-[\beta]}
				-
				\frac{1}{2}
				\comm{\hat\phi^+[\beta]}{\hat\phi^-[\beta]}
			\right\}
			\\
			&=
			\exp\left\{
				-
				\frac{1}{2}
				\int\frac{\dd[3]{p}}{(2\pi)^32\omega(\vb{p})}
				\left\{
					\abs{\alpha(\vb{p})}^2
					+
					\abs{\beta(\vb{p})}^2
					-
					2\alpha(\vb{p})\beta(\vb{p})^*
				\right\}
			\right\}
		\end{split}
	\end{equation}
\end{restatable}
\begin{restatable}{lemma}{qkgcoherentwavefunction}\label{thm:coherent_state_wave_function}
	Let $\ket{\alpha}$ be a coherent state, then its coordinate wave function is
	\begin{equation}
		\psi(x^\mu)
		=
		\bra{0}
		\hat\phi(x^\mu)
		\ket{\alpha}
		=
		e^{-\overline{n}/2}
		\int\frac{\dd[3]{p}}{(2\pi)^32\omega(\vb{p})}
		\alpha(\vb{p})
		\eval{e^{-ip_\mu x^\mu}}_{p_0=\omega(\vb{p})}
		.
	\end{equation}
	The coherent state's coordinate wave function is equal to the single-particle number state's coordinate wave function with the difference that the wave function of the coherent state is suppressed by $e^{-\overline{n}/2}$ and the spectrum $\alpha(\vb{p})$ is not constrained.
\end{restatable}
\begin{corollary}
	Results obtained for the single-particle number state's coordinate wave function, e.g., the group velocity (\Cref{thm:number_state_single_group_velocity}) and the localization (\Cref{thm:number_state_single_localization}), carry over to the coherent state after performing the replacement $f(\vb{p})=e^{-\overline{n}/2}\alpha(\vb{p})$.
\end{corollary}

\begin{definition}
	The positive and negative frequency momentum density operator are
	\begin{align}
		\hat\pi^+(x^\mu)
		&=
		\int\frac{\dd[3]{p}}{(2\pi)^3}
		\left(
			+
			i\sqrt{\frac{p_0}{2}}
		\right)
		\hat{a}^\dagger(\vb{p})
		\eval{e^{+ip_\mu x^\mu}}_{p_0=\omega(\vb{p})}
		\\
		\hat\pi^-(x^\mu)
		&=
		\int\frac{\dd[3]{p}}{(2\pi)^3}
		\left(
			-
			i\sqrt{\frac{p_0}{2}}
		\right)
		\hat{a}(\vb{p})
		\eval{e^{-ip_\mu x^\mu}}_{p_0=\omega(\vb{p})}
		.
	\end{align}
\end{definition}
\begin{corollary}
	The momentum density operator can be written as a sum of positive and negative frequency momentum density operators
	\begin{equation}
		\hat\pi(x^\mu)
		=
		\hat\pi^+(x^\mu)
		+
		\hat\pi^-(x^\mu)
		.
	\end{equation}
\end{corollary}
\begin{corollary}
	The positive and negative frequency momentum density operator are related by the Hermitian conjugate
	\begin{equation}
		\hat\pi^-(x^\mu)
		=
		\hat\pi^+(x^\mu)^\dagger
		.
	\end{equation}	
\end{corollary}

\begin{lemma}
	The expected mean of the momentum density operator w.r.t. the coherent state is
	\begin{equation}
		\expval{\hat\pi(x^\mu)}{\alpha}
		=
		\int\frac{\dd[3]{p}}{(2\pi)^3}
		\Im\left\{
			\alpha\left(p_0,\vb{p}\right)
			e^{-ip_\mu x^\mu}	
		\right\}_{p_0=\omega(\vb{p})}
		.
	\end{equation}
\end{lemma}
\begin{proof}
	We decompose the momentum density operator
	\begin{equation*}
		\expval{\hat\pi(x^\mu)}{\alpha}
		=
		\expval{\hat\pi^-(x^\mu)}{\alpha}
		+
		\expval{\hat\pi^+(x^\mu)}{\alpha}
	\end{equation*}
	and evaluate the first term
	\begin{equation*}
		\begin{split}
			\expval{\hat\pi^-(x^\mu)}{\alpha}
			&=
			\int\frac{\dd[3]{p}}{(2\pi)^3}
			\left(
				-i
				\sqrt{\frac{\omega(\vb{p})}{2}}
			\right)
			\expval{\hat{a}(\vb{p})}{\alpha}
			\eval{e^{-ip_\mu x^\mu}}_{p_0=\omega(\vb{p})}
			\\
			&=
			\int\frac{\dd[3]{p}}{(2\pi)^3}
			\left(
				-i
				\sqrt{\frac{\omega(\vb{p})}{2}}
			\right)
			\frac{\alpha\left(\omega(\vb{p}),\vb{p}\right)}{\sqrt{2\omega(\vb{p})}}
			\eval{e^{-ip_\mu x^\mu}}_{p_0=\omega(\vb{p})}
			\\
			&=
			-
			\frac{i}{2}
			\int\frac{\dd[3]{p}}{(2\pi)^3}
			\alpha\left(\omega(\vb{p}),\vb{p}\right)
			\eval{e^{-ip_\mu x^\mu}}_{p_0=\omega(\vb{p})}
		\end{split}
	\end{equation*}
	and the second term is equal to the Hermitian conjugate, thus
	\begin{equation*}
		\begin{split}
			\expval{\hat\pi(x^\mu)}{\alpha}
			&=
			-
			\frac{i}{2}
			\int\frac{\dd[3]{p}}{(2\pi)^3}
			\alpha\left(\omega(\vb{p}),\vb{p}\right)
			\eval{e^{-ip_\mu x^\mu}}_{p_0=\omega(\vb{p})}
			+
			\frac{i}{2}
			\int\frac{\dd[3]{p}}{(2\pi)^3}
			\alpha\left(\omega(\vb{p}),\vb{p}\right)^*
			\eval{e^{+ip_\mu x^\mu}}_{p_0=\omega(\vb{p})}
			\\
			&=
			\frac{1}{2i}
			\int\frac{\dd[3]{p}}{(2\pi)^3}
			\left\{
				\alpha\left(p_0,\vb{p}\right)
				e^{-ip_\mu x^\mu}
				-
				\alpha\left(p_0,\vb{p}\right)^*
				e^{+ip_\mu x^\mu}
			\right\}_{p_0=\omega(\vb{p})}
			\\
			&=
			\int\frac{\dd[3]{p}}{(2\pi)^3}
			\Im\left\{
				\alpha\left(p_0,\vb{p}\right)
				e^{-ip_\mu x^\mu}			
			\right\}_{p_0=\omega(\vb{p})}
			.
		\end{split}
	\end{equation*}
\end{proof}
\begin{lemma}
	\begin{equation}
		\begin{split}
			\expval{\hat\pi(x^\mu)\hat\pi(y^\mu)}{\alpha}
			&=
			\frac{1}{2}
			\Re\left\{
				\int\frac{\dd[3]{p}}{(2\pi)^3}
				\left(
					\alpha(p_0,\vb{p})
					e^{-ip_\mu x^\mu}
				\right)^*_{p_0=\omega(\vb{p})}
				\int\frac{\dd[3]{q}}{(2\pi)^3}
				\left(
					\alpha(q_0,\vb{q})
					e^{-iq_\mu y^\mu}
				\right)_{q_0=\omega(\vb{q})}			
			\right\}
			\\
			&-
			\frac{1}{2}
			\Re\left\{
				\int\frac{\dd[3]{p}}{(2\pi)^3}
				\left(
					\alpha(p_0,\vb{p})
					e^{-ip_\mu x^\mu}
				\right)^*_{p_0=\omega(\vb{p})}
				\int\frac{\dd[3]{q}}{(2\pi)^3}
				\left(
					\alpha(q_0,\vb{q})
					e^{-iq_\mu y^\mu}
				\right)^*_{q_0=\omega(\vb{q})}
			\right\}
		\end{split}
	\end{equation}
\end{lemma}
\begin{proof}
	We write the expectation value in terms of positive and negative frequency momentum density operators
	\begin{equation*}
		\expval{\hat\pi(x^\mu)\hat\pi(y^\mu)}{\alpha}
		=
		\expval{\hat\pi^+(x^\mu)\hat\pi^+(y^\mu)}{\alpha}
		+
		\expval{\hat\pi^+(x^\mu)\hat\pi^-(y^\mu)}{\alpha}
		+
		\text{h.c.}
	\end{equation*}
	and evaluate the first term
	\begin{equation*}
		\begin{split}
			\expval{\hat\pi^+(x^\mu)\hat\pi^+(y^\mu)}{\alpha}
			&=
			\int\frac{\dd[3]{p}}{(2\pi)^3}
			\left(
				+i
				\sqrt{\frac{\omega(\vb{p})}{2}}
			\right)
			\int\frac{\dd[3]{q}}{(2\pi)^3}
			\left(
				+i
				\sqrt{\frac{\omega(\vb{q})}{2}}
			\right)
			\\
			&\times
			\expval{\hat{a}^\dagger(\vb{p})\hat{a}^\dagger(\vb{q})}{\alpha}
			\eval{e^{+ip_\mu x^\mu}}_{p_0=\omega(\vb{p})}
			\eval{e^{+iq_\mu y^\mu}}_{q_0=\omega(\vb{q})}
			\\
			&=
			\int\frac{\dd[3]{p}}{(2\pi)^3}
			\left(
				+i
				\sqrt{\frac{\omega(\vb{p})}{2}}
			\right)
			\int\frac{\dd[3]{q}}{(2\pi)^3}
			\left(
				+i
				\sqrt{\frac{\omega(\vb{q})}{2}}
			\right)
			\\
			&\times
			\frac{\alpha\left(\omega(\vb{p}),\vb{p}\right)^*}{\sqrt{2\omega(\vb{p})}}
			\frac{\alpha\left(\omega(\vb{q}),\vb{q}\right)^*}{\sqrt{2\omega(\vb{q})}}
			\eval{e^{+ip_\mu x^\mu}}_{p_0=\omega(\vb{p})}
			\eval{e^{+iq_\mu y^\mu}}_{q_0=\omega(\vb{q})}
			\\
			&=
			\frac{i}{2}
			\int\frac{\dd[3]{p}}{(2\pi)^3}
			\left(
				\alpha(p_0,\vb{p})
				e^{-ip_\mu x^\mu}
			\right)^*_{p_0=\omega(\vb{p})}
			\frac{i}{2}
			\int\frac{\dd[3]{q}}{(2\pi)^3}
			\left(
				\alpha(q_0,\vb{q})
				e^{-iq_\mu y^\mu}
			\right)^*_{q_0=\omega(\vb{q})}
			\\
			&=
			-
			\frac{1}{4}
			\int\frac{\dd[3]{p}}{(2\pi)^3}
			\left(
				\alpha(p_0,\vb{p})
				e^{-ip_\mu x^\mu}
			\right)^*_{p_0=\omega(\vb{p})}
			\int\frac{\dd[3]{q}}{(2\pi)^3}
			\left(
				\alpha(q_0,\vb{q})
				e^{-iq_\mu y^\mu}
			\right)^*_{q_0=\omega(\vb{q})}
		\end{split}
	\end{equation*}
	by using the Hermitian conjugate of the eigenvalue equation \Cref{thm:qkg_coherent_state_eigenstate} to the left.
	For the second term, we find
	\begin{equation*}
		\begin{split}
			\expval{\hat\pi^+(x^\mu)\hat\pi^-(y^\mu)}{\alpha}
			&=
			\int\frac{\dd[3]{p}}{(2\pi)^3}
			\left(
				+i
				\sqrt{\frac{\omega(\vb{p})}{2}}
			\right)
			\int\frac{\dd[3]{q}}{(2\pi)^3}
			\left(
				-i
				\sqrt{\frac{\omega(\vb{q})}{2}}
			\right)
			\\
			&\times
			\expval{\hat{a}^\dagger(\vb{p})\hat{a}(\vb{q})}{\alpha}
			\eval{e^{+ip_\mu x^\mu}}_{p_0=\omega(\vb{p})}
			\eval{e^{-iq_\mu y^\mu}}_{q_0=\omega(\vb{q})}
			\\
			&=
			\int\frac{\dd[3]{p}}{(2\pi)^3}
			\left(
				+i
				\sqrt{\frac{\omega(\vb{p})}{2}}
			\right)
			\int\frac{\dd[3]{q}}{(2\pi)^3}
			\left(
				-i
				\sqrt{\frac{\omega(\vb{q})}{2}}
			\right)
			\\
			&\times
			\frac{\alpha\left(\omega(\vb{p}),\vb{p}\right)}{\sqrt{2\omega(\vb{p})}}
			\frac{\alpha\left(\omega(\vb{q}),\vb{q}\right)^*}{\sqrt{2\omega(\vb{q})}}
			\eval{e^{+ip_\mu x^\mu}}_{p_0=\omega(\vb{p})}
			\eval{e^{-iq_\mu y^\mu}}_{q_0=\omega(\vb{q})}
			\\
			&=
			\frac{i}{2}
			\int\frac{\dd[3]{p}}{(2\pi)^3}
			\left(
				\alpha(p_0,\vb{p})
				e^{-ip_\mu x^\mu}
			\right)^*_{p_0=\omega(\vb{p})}
			\left(
				-
				\frac{i}{2}
			\right)
			\int\frac{\dd[3]{q}}{(2\pi)^3}
			\left(
				\alpha(q_0,\vb{q})
				e^{-iq_\mu y^\mu}
			\right)_{q_0=\omega(\vb{q})}
			\\
			&=
			+
			\frac{1}{4}
			\int\frac{\dd[3]{p}}{(2\pi)^3}
			\left(
				\alpha(p_0,\vb{p})
				e^{-ip_\mu x^\mu}
			\right)^*_{p_0=\omega(\vb{p})}
			\int\frac{\dd[3]{q}}{(2\pi)^3}
			\left(
				\alpha(q_0,\vb{q})
				e^{-iq_\mu y^\mu}
			\right)_{q_0=\omega(\vb{q})}
			,
		\end{split}
	\end{equation*}
	thus in sum
	\begin{equation*}
		\begin{split}
			\expval{\hat\pi(x^\mu)\hat\pi(y^\mu)}{\alpha}
			&=
			\frac{1}{2}
			\Re\left\{
				\int\frac{\dd[3]{p}}{(2\pi)^3}
				\left(
					\alpha(p_0,\vb{p})
					e^{-ip_\mu x^\mu}
				\right)^*_{p_0=\omega(\vb{p})}
				\int\frac{\dd[3]{q}}{(2\pi)^3}
				\left(
					\alpha(q_0,\vb{q})
					e^{-iq_\mu y^\mu}
				\right)_{q_0=\omega(\vb{q})}			
			\right\}
			\\
			&-
			\frac{1}{2}
			\Re\left\{
				\int\frac{\dd[3]{p}}{(2\pi)^3}
				\left(
					\alpha(p_0,\vb{p})
					e^{-ip_\mu x^\mu}
				\right)^*_{p_0=\omega(\vb{p})}
				\int\frac{\dd[3]{q}}{(2\pi)^3}
				\left(
					\alpha(q_0,\vb{q})
					e^{-iq_\mu y^\mu}
				\right)^*_{q_0=\omega(\vb{q})}
			\right\}
		\end{split}
	\end{equation*}
\end{proof}
\begin{corollary}
	The variance of the momentum density is
	\begin{equation}
		\expval{\left(\Delta\hat\pi(x^\mu)\right)^2}{\alpha}
		=
	\end{equation}
\end{corollary}
\begin{proof}
	The second moment follows from the previous lemma
	\begin{equation*}
		\begin{split}
			\expval{\hat\pi(x^\mu)^2}{\alpha}
			&=
			\frac{1}{2}
			\Re\left\{
				\int\frac{\dd[3]{p}}{(2\pi)^3}
				\left(
					\alpha(p_0,\vb{p})
					e^{-ip_\mu x^\mu}
				\right)^*_{p_0=\omega(\vb{p})}
				\int\frac{\dd[3]{q}}{(2\pi)^3}
				\left(
					\alpha(q_0,\vb{q})
					e^{-iq_\mu x^\mu}
				\right)_{q_0=\omega(\vb{q})}			
			\right\}
			\\
			&-
			\frac{1}{2}
			\Re\left\{
				\int\frac{\dd[3]{p}}{(2\pi)^3}
				\left(
					\alpha(p_0,\vb{p})
					e^{-ip_\mu x^\mu}
				\right)_{p_0=\omega(\vb{p})}
				\int\frac{\dd[3]{q}}{(2\pi)^3}
				\left(
					\alpha(q_0,\vb{q})
					e^{-iq_\mu x^\mu}
				\right)_{q_0=\omega(\vb{q})}
			\right\}
			\\
			&=
			\frac{1}{2}
			\abs{
				\int\frac{\dd[3]{p}}{(2\pi)^3}
				\left(
					\alpha(p_0,\vb{p})
					e^{-ip_\mu x^\mu}
				\right)_{p_0=\omega(\vb{p})}
			}^2
			-
			\frac{1}{2}
			\Re\left\{
				\int\frac{\dd[3]{p}}{(2\pi)^3}
				\left(
					\alpha(p_0,\vb{p})
					e^{-ip_\mu x^\mu}
				\right)_{p_0=\omega(\vb{p})}
			\right\}^2
		\end{split}
	\end{equation*}
\end{proof}
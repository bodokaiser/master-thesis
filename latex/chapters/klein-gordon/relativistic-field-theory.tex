\section{Relativistic field theory}

\begin{definition}[Klein-Gordon Lagrangian]
	The Klein-Gordon Lagrangian density
	\begin{equation}
		\mathcal{L}
		=
		\frac{1}{2}
		\left(\partial_\mu\phi\right)
		\left(\partial^\mu\phi\right)
		-
		\frac{1}{2}
		m^2\phi^2
		\label{eq:kg_lagrangian}
	\end{equation}
	describes a real-valued massive scalar field $\phi(t,\vb{x})$.
\end{definition}
\begin{corollary}
	The Klein-Gordon Lagrangian is manifest Lorentz-covariant, i.e., as a Lorentz scalar, the Klein-Gordon Lagrangian is invariant under Lorentz transformations and thereby valid in any reference frame.
\end{corollary}
\begin{theorem}[Relativistic energy-momentum relation]\label{th:relativistic_energy_momentum}
	Excitations of the Klein-Gordon field satisfy the relativistic energy-momentum relation
	\begin{equation}
		\omega(\vb{p})
		=
		\sqrt{\vb{p}^2+m^2}
		=
		E(\vb{p})
		\label{eq:energy_momentum_relation}
		.
	\end{equation}
\end{theorem}
\begin{theorem}[Mode expansion of the Klein-Gordon field]\label{thm:kg_fourier_expansion}
	The mode expansion of the Klein-Gordon field
	\begin{equation}
		\phi(x^\mu)
		=
		\int_{\mathbb{R}^3}\frac{\dd[3]{p}}{(2\pi)^3\sqrt{2\omega(\vb{p})}}
		\biggl\{
			a(\vb{p})
			e^{-ip_\mu x^\mu}
			+
			a(\vb{p})^*
			e^{+ip_\mu x^\mu}
		\biggr\}_{p_0=\omega(\vb{p})}
	\end{equation}
	satisfies the equations of motion for any choice of $a(\vb{p})=\phi(\omega(\vb{p}),\vb{p})$ where $\phi(p_0,\vb{p})=\phi(p^\mu)$ is the Fourier amplitude of the Klein-Gordon field $\phi(x^\mu)=\phi(t,\vb{x})$.
\end{theorem}
\begin{definition}[Energy-momentum tensor]
	The energy-momentum tensor of a scalar field is
	\begin{equation}
		T^{\mu\nu}
		=
		\pdv{\mathcal{L}}{(\partial_\mu\phi)}\partial^\nu\phi
		-
		g^{\mu\nu}\mathcal{L}
		\label{eq:energy_momentum_tensor}
		.
	\end{equation}
	The energy-momentum tensor's components encode the energy density $T^{00}$, the momentum density $T^{0i}$, and the stress densities $T^{ij}$.
\end{definition}
\begin{lemma}
	The energy density of the Klein-Gordon's energy-momentum tensor
	\begin{equation*}
		T^{00}
		=
		\frac{1}{2}
		\left(\partial_t\phi\right)^2
		+
		\frac{1}{2}
		\left(\grad\phi\right)^2
		+
		\frac{1}{2}
		\left(m\phi\right)^2
		=
		\mathcal{H}
	\end{equation*}
	is equal to its Hamiltonian density $\mathcal{H}$.
\end{lemma}
\begin{lemma}\label{thm:kg_total_energy_momentum}
	The total energy and the total momentum of the Klein-Gordon field are
	\begin{align}
		H
		=
		\int\frac{\dd[3]{p}}{(2\pi)^3}
		\omega(\vb{p})\abs{a(\vb{p})}^2
		&&
		\vb{P}
		=
		\int\frac{\dd[3]{p}}{(2\pi)^3}
		\vb{p}\abs{a(\vb{p})}^2
		\label{eq:kg_energy_momentum}
		.
	\end{align}
\end{lemma}
\begin{definition}
	The Klein-Gordon Hamiltonian density is
	\begin{equation}
		\mathcal{H}
		=
		\frac{1}{2}
		\pi^2
		+
		\frac{1}{2}
		\left(\grad\phi\right)^2
		+
		\frac{1}{2}
		\left(m\phi\right)^2
		.
	\end{equation}
	$\phi(x^\mu)$ is the position and $\pi(x^\mu)$ the momentum density.
\end{definition}
\begin{lemma}
	The mode expansions of the Klein-Gordon's position and momentum densities read
	\begin{align}
		\phi(x^\mu)
		&=
		\int_{\mathbb{R}^3}\frac{\dd[3]{p}}{(2\pi)^3}
		\frac{1}{\sqrt{2\omega(\vb{p})}}
		\biggl\{
			a(\vb{p})
			e^{-ip_\mu x^\mu}
			+
			a(\vb{p})^*
			e^{+ip_\mu x^\mu}
		\biggr\}_{p_0=\omega(\vb{p})}
		,
		\\
		\pi(x^\mu)
		&=
		\int_{\mathbb{R}^3}\frac{\dd[3]{p}}{(2\pi)^3}
		\left(-i\sqrt{\frac{\omega(\vb{p})}{2}}\right)
		\biggl\{
			a(\vb{p})
			e^{-ip_\mu x^\mu}
			-
			a(\vb{p})^*
			e^{+ip_\mu x^\mu}
		\biggr\}_{p_0=\omega(\vb{p})}
		.
	\end{align}
\end{lemma}
\begin{definition}
	The current
\end{definition}
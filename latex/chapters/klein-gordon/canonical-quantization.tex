\section{Canonical quantization}

\begin{definition}[Canonical quantization]
	In the canonical quantization procedure, the dynamical variables are promoted to operators satisfying the equal-time canonical commutation relations
	\begin{align}
		\comm{\hat\phi(\vb{x})}{\hat\pi(\vb{y})}
		&=
		i\delta^{(3)}(\vb{x}-\vb{y})
		\\
		\comm{\hat\phi(\vb{x})}{\hat\phi(\vb{y})}
		&=
		\comm{\hat\pi(\vb{x})}{\hat\pi(\vb{y})}
		=
		0
		\label{eq:qkg_comm_pm}
		.
	\end{align}
\end{definition}
\begin{theorem}[Klein-Gordon field operators]
	The Klein-Gordon field operators are
	\begin{align}
		\hat\phi(t,\vb{x})
		&=
		\int\frac{\dd[3]{p}}{(2\pi)^3}
		\frac{1}{\sqrt{2\omega(\vb{p})}}
		\left\{
			\hat{a}(\vb{p})
			e^{-ip_\mu x^\mu}
			+
			\hat{a}^\dagger(\vb{p})
			e^{+ip_\mu x^\mu}
		\right\}
		\label{eq:qkg_pos}
		\\
		\hat\pi(t,\vb{x})
		&=
		\int\frac{\dd[3]{p}}{(2\pi)^3}
		\left(-i\sqrt{2\omega(\vb{p})}\right)
		\left\{
			\hat{a}(\vb{p})
			e^{-ip_\mu x^\mu}
			-
			\hat{a}^\dagger(\vb{p})
			e^{+ip_\mu x^\mu}
		\right\}
		\label{eq:qkg_mom}
		,
	\end{align}
	where $\hat\phi(t,\vb{x})$ and $\hat\pi(t,\vb{x})$ satisfy the equal-time canonical commutation relations.
\end{theorem}
\begin{theorem}\label{thm:kg_comm_ac}
	The annihilation and creation operator of the Klein-Gordon field obey the commutation relations
	\begin{align}
		\comm{\hat{a}(\vb{p})}{\hat{a}^\dagger(\vb{q})}
		&=
		(2\pi)^3
		\delta^{(3)}(\vb{p}-\vb{q})
		\\
		\comm{\hat{a}^\dagger(\vb{p})}{\hat{a}^\dagger(\vb{q})}
		&=
		\comm{\hat{a}(\vb{p})}{\hat{a}(\vb{q})}
		=
		0
		\label{eq:kg_comm_ac}
		.
	\end{align}	
\end{theorem}

\subsection{Energy and momentum}

\begin{definition}[Normal-ordered product]
	The normal-ordered product of annihilation and creation operators
	\begin{equation}
		N\left\{
			\left(
				\hat{a}^\dagger
				\hat{a}
			\right)^l
		\right\}
		=
		\left(\hat{a}^\dagger\right)^l
		\hat{a}^l
	\end{equation}
	places all creation operators to the left and all annihilation operators to the right.
\end{definition}
\begin{definition}[Correspondence principle]
	The correspondence principle is a prescription to find the quantum from the classical observables by promoting the dynamical variables to operators in normal-order.\footnote{See Ref.~\cite[p.~20]{Mukhanov2007} for details on the problem of operator ordering.}
\end{definition}
\begin{corollary}
	The total energy and momentum operators of the Klein-Gordon field are
	\begin{align}
		\hat{H}
		=
		\int\frac{\dd[3]{p}}{(2\pi)^3}
		\omega(\vb{p})\hat{a}^\dagger(\vb{p})\hat{a}(\vb{p})
		&&
		\hat{\vb{P}}
		=
		\int\frac{\dd[3]{p}}{(2\pi)^3}
		\vb{p}\hat{a}^\dagger(\vb{p})\hat{a}(\vb{p})
		\label{eq:qkg_energy_momentum}
		.
	\end{align}
\end{corollary}
\begin{corollary}
	The number operator is the unweighted part of the energy and momentum operators
	\begin{equation}
		\hat{N}
		=
		\int\frac{\dd[3]{p}}{(2\pi)^3}
		\hat{a}^\dagger(\vb{p})
		\hat{a}(\vb{p})
		\label{eq:qkg_number}
		.
	\end{equation}
\end{corollary}

\subsection{Fock space}

\begin{definition}[Vacuum state]
	The vacuum state $\ket{0}$ is a unique state that is normalized
	\begin{equation}
		\braket{0}
		=
		1
	\end{equation}
	and satisfies
	\begin{align}
		\hat{P}_\mu
		\ket{0}
		&=
		0
		\\
		\hat{M}_{\mu\nu}
		\ket{0}
		&=
		\ket{0}
	\end{align}
	where $\hat{P}_\mu$ is the four-momentum and $\hat{M}_{\mu\nu}$ is the relativistic angular momentum operator, see Ref.~\cite[p.~270]{Greiner2013} for details.
\end{definition}
\begin{corollary}
	The vacuum state is invariant under spacetime translations
	\begin{equation}
		e^{i\hat{P}_\mu a^\mu}
		\ket{0}
		=
		\ket{0}
		.
	\end{equation}	
\end{corollary}
\begin{theorem}
	Let $n,m\in\mathbb{N}$ and $n\neq m$ then
	\begin{equation}
		\expval{
			\hat{a}(\vb{p}_1)
			\dots
			\hat{a}(\vb{p}_n)
			\hat{a}^\dagger(\vb{q}_1)
			\dots
			\hat{a}^\dagger(\vb{q}_m)
		}{0}
		=
		0
		.
	\end{equation}
\end{theorem}
\begin{proof}
	Without loss of generality, assume $n>m$, then
\end{proof}
\begin{definition}(Vacuum fluctuation)
	The vacuum fluctuation is
	\begin{equation}
		\Delta\pi_0
		=
		\int\frac{\dd[3]{p}}{(2\pi)^32\omega(\vb{p})}
		\left(2\omega(\vb{p})\right)^2
		.
	\end{equation}
\end{definition}
\begin{lemma}\label{thm:vacuum_state_momentum_density_observable}
	The mean momentum density of the vacuum state is zero, the correlation is
	\begin{equation}
		\expval{\hat\pi(x^\mu)\hat\pi(y^\mu)}{0}
		=
		\int\frac{\dd[3]{p}}{(2\pi)^32\omega(\vb{p})}
		\left(2\omega(\vb{p})\right)^2
		e^{-ip_\mu(x^\mu-y^\mu)}
	\end{equation}
	and the variance turns out to be equal to the vacuum fluctuations $\Delta\pi_0$.
\end{lemma}

\subsection{Propagators}

The section is based on Ref.~\cite[p.~26]{Peskin1995} and can be skipped if one is not interested in the proofs of the coherent state and interactions.
\begin{definition}[Propagator]
	We define the propagator as
	\begin{equation}
		D(x^\mu-y^\mu)
		=
		\int\frac{\dd[3]{p}}{(2\pi)^32\omega(\vb{p})}
		e^{-ip_\mu (x^\mu-y^\mu)}
		.
	\end{equation}
\end{definition}
\begin{definition}[Positive and negative frequency Klein-Gordon field operator]
	The positive and negative frequency decomposition of the Klein-Gordon field operator
	\begin{equation}
		\hat\phi(t,\vb{x})
		=
		\hat\phi^+(t,\vb{x})
		+
		\hat\phi^-(t,\vb{x})
	\end{equation}
	wherein the positive and negative frequency Klein-Gordon field operators
	\begin{equation}
		\begin{split}
			\hat\phi^+(t,\vb{x})
			&=
			\int\frac{\dd[3]{p}}{(2\pi)^3\sqrt{2\omega(\vb{p})}}
			e^{-ip_\mu x^\mu}
			\hat{a}(\vb{p})
			\\
			\hat\phi^-(t,\vb{x})
			&=
			\int\frac{\dd[3]{p}}{(2\pi)^3\sqrt{2\omega(\vb{p})}}
			e^{+ip_\mu x^\mu}
			\hat{a}^\dagger(\vb{p})
		\end{split}
		\label{eq:qkg_positive_negative_frequency}
	\end{equation}
	are related by the hermitian conjugate $\hat\phi^-(t,\vb{x})=\hat\phi^+(t,\vb{x})^\dagger$.
\end{definition}
\begin{lemma}\label{thm:qkg_full_comm_pn_comm}
	The positive and negative frequency Klein-Gordon field operators satisfy
	\begin{align}
		\comm{\hat\phi^+(x^\mu)}{\hat\phi^-(y^\mu)}
		&=
		D(x^\mu-y^\mu)
		\\
		\comm{\hat\phi^+(x^\mu)}{\hat\phi^+(y^\mu)}
		&=
		\comm{\hat\phi^-(x^\mu)}{\hat\phi^-(y^\mu)}
		=
		0
	\end{align}
\end{lemma}
\begin{lemma}\label{thm:qkg_propagator_kg_comm}
	The commutator of the Klein-Gordon operators is equal to
	\begin{equation}
		\comm{\hat\phi(x^\mu)}{\hat\phi(y^\mu)}
		=
		D(x^\mu-y^\mu)
		-
		D(y^\mu-x^\mu)
	\end{equation}
\end{lemma}
\begin{lemma}\label{thm:qkg_propagator_correlation_function}
	The propagator is equal to the correlation function
	\begin{equation}
		D(x^\mu-y^\mu)
		=
		\expval{\hat\phi(x^\mu)\hat\phi(y^\mu)}{0}
	\end{equation}
\end{lemma}
\begin{definition}
	The retarded propagator is defined to be
	\begin{equation}
		D_R(x^\mu-y^\mu)
		=
		\theta(x^0-y^0)
		\comm{\hat\phi(x^\mu)}{\hat\phi(y^\mu)}
		=
		\begin{cases}
			\comm{\hat\phi(x^\mu)}{\hat\phi(y^\mu)}	& \text{if}\ x^0<y^0
			\\
			0 & \text{otherwise}
		\end{cases}
		,
	\end{equation}
	the advanced propagator is defined to be
	\begin{equation}
		D_A(x^\mu-y^\mu)
		=
		\theta(x^0-y^0)
		\comm{\hat\phi(x^\mu)}{\hat\phi(y^\mu)}
		=
		\begin{cases}
			\comm{\hat\phi(x^\mu)}{\hat\phi(y^\mu)}	& \text{if}\ x^0>y^0
			\\
			0 & \text{otherwise}
		\end{cases}
		,
	\end{equation}
	and the Feynman propagator is defined to be
	\begin{equation}
		\begin{split}
			D_F(x^\mu-y^\mu)
			&=
			\theta(x^0-y^0)
			\expval{\hat\phi(x^\mu)\hat\phi(y^\mu)}{0}
			+
			\theta(y^0-x^0)
			\expval{\hat\phi(y^\mu)\hat\phi(x^\mu)}{0}
			\\
			&=
			\begin{cases}
				D(x^\mu-y^\mu) & \text{if}\ x^0>y^0 \\
				D(y^\mu-x^\mu) & \text{if}\ x^0<y^0 \\
				\frac{1}{2}
				D(x^\mu-y^\mu)
				+
				\frac{1}{2}
				D(y^\mu-x^\mu)
				& \text{if}\ x^0=y^0
			\end{cases}
		\end{split}
		.
	\end{equation}
\end{definition}
\begin{definition}
	The time-ordered product of two operators $\hat{A}(x^\mu),\hat{B}(y^\mu)$ is defined to be
	\begin{equation}
		\begin{split}
			\expval{T\left\{\hat{A}(x^\mu)\hat{B}(y^\mu)\right\}}{0}
			&=
			\theta(x^0-y^0)
			\expval{\hat{A}(x^\mu)\hat{B}(y^\mu)}{0}
			\\
			&+
			\theta(y^0-x^0)
			\expval{\hat{B}(y^\mu)\hat{A}(x^\mu)}{0}
		\end{split}
		.		
	\end{equation}
\end{definition}
\begin{corollary}
	The Feynman propagator is equal to the time-ordered correlation function
	\begin{equation}
		D_F(x^\mu-y^\mu)
		=
		\expval{T\left\{\hat\phi(x^\mu)\hat\phi(y^\mu)\right\}}{0}
		.
	\end{equation}
\end{corollary}
\begin{lemma}\label{thm:propagator_kg_solution}
	The propagators are solutions to the Klein-Gordon equation of a point-source
	\begin{equation}
		\left(
			\partial_\mu
			\partial^\mu
			+
			m^2
		\right)
		D(x^\mu-y^\mu)
		=
		-i\delta^{(4)}(x^\mu-y^\mu)
	\end{equation}
	but to different boundary conditions.
\end{lemma}
\begin{corollary}
	The Fourier transform of the propagators $D(p^\mu)$ satisfies
	\begin{equation}
		\left(
			-
			p_\mu p^\mu
			+
			m^2
		\right)
		D(p^\mu)
		=
		-i
		.
	\end{equation}
\end{corollary}

\subsection{Coordinate wave function}
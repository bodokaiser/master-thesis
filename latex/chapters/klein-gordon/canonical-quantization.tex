\section{Canonical quantization}

\begin{definition}[Canonical quantization]
	In the canonical quantization procedure, the dynamical variables are promoted to operators satisfying the equal-time canonical commutation relations
	\begin{align}
		\comm{\hat\phi(t,\vb{x})}{\hat\pi(t,\vb{y})}
		&=
		i\delta^{(3)}(\vb{x}-\vb{y})
		\\
		\comm{\hat\phi(t,\vb{x})}{\hat\phi(t,\vb{y})}
		&=
		\comm{\hat\pi(t,\vb{x})}{\hat\pi(t,\vb{y})}
		=
		0
		\label{eq:qkg_comm_pm}
		.
	\end{align}
\end{definition}
\begin{corollary}[Klein-Gordon field operators]
	The Klein-Gordon field operators are
	\begin{align}
		\hat\phi(x^\mu)
		&=
		\int\frac{\dd[3]{p}}{(2\pi)^3}
		\frac{1}{\sqrt{2\omega(\vb{p})}}
		\left\{
			\hat{a}(\vb{p})
			e^{-ip_\mu x^\mu}
			+
			\hat{a}^\dagger(\vb{p})
			e^{+ip_\mu x^\mu}
		\right\}_{p_0=\omega(\vb{p})}
		\label{eq:qkg_pos}
		\\
		\hat\pi(x^\mu)
		&=
		\int\frac{\dd[3]{p}}{(2\pi)^3}
		\left(-i\sqrt{\frac{\omega(\vb{p})}{2}}\right)
		\left\{
			\hat{a}(\vb{p})
			e^{-ip_\mu x^\mu}
			-
			\hat{a}^\dagger(\vb{p})
			e^{+ip_\mu x^\mu}
		\right\}_{p_0=\omega(\vb{p})}
		\label{eq:qkg_mom}
	\end{align}
	where $\hat\phi(x^\mu)$ and $\hat\pi(x^\mu)$ satisfy the equal-time canonical commutation relations.
\end{corollary}
\begin{restatable}{theorem}{qkgcommac}\label{thm:qkg_comm_ac}
	The operators $\hat{a}(\vb{p}),\hat{a}^\dagger(\vb{p})$ obey the commutation relations
	\begin{align}
		\comm{\hat{a}(\vb{p})}{\hat{a}^\dagger(\vb{q})}
		&=
		(2\pi)^3
		\delta^{(3)}(\vb{p}-\vb{q})
		\\
		\comm{\hat{a}^\dagger(\vb{p})}{\hat{a}^\dagger(\vb{q})}
		&=
		\comm{\hat{a}(\vb{p})}{\hat{a}(\vb{q})}
		=
		0
		\label{eq:kg_comm_ac}
		.
	\end{align}	
\end{restatable}
\begin{definition}[Energy and momentum operator]\label{def:qkg_energy_momentum}
	The Klein-Gordon's total energy and total momentum operators are
	\begin{align}
		\hat{H}
		=
		\int\frac{\dd[3]{p}}{(2\pi)^3}
		\omega(\vb{p})\hat{a}^\dagger(\vb{p})\hat{a}(\vb{p})
		&&
		\hat{\vb{P}}
		=
		\int\frac{\dd[3]{p}}{(2\pi)^3}
		\vb{p}\hat{a}^\dagger(\vb{p})\hat{a}(\vb{p})
		\label{eq:qkg_energy_momentum}
		.
	\end{align}
\end{definition}
\begin{definition}\label{def:vacuum_state}
	The vacuum state $\ket{0}$ is the unique eigenstate of the total energy and total momentum operators with eigenvalue zero, i.e.,
	\begin{align}
		\hat{H}
		\ket{0}
		=
		0
		&&
		\hat{\vb{P}}
		\ket{0}
		=
		0
		.
	\end{align}
\end{definition}
\begin{corollary}\label{thm:vacuum_state_ac}
	The definition of the vacuum state and the energy operator implies
	\begin{align}
		\hat{a}(\vb{p})
		\ket{0}
		=
		0
		&&
		\bra{0}
		\hat{a}^\dagger(\vb{p})
		=
		0
		.
	\end{align}
\end{corollary}
\begin{corollary}
	The commutators of the operators $\hat{a}(\vb{p}),\hat{a}^\dagger(\vb{p})$ with the Hamiltonian $\hat{H}$ yield
	\begin{align}
		\comm{\hat{H}}{\hat{a}^\dagger(\vb{p})}
		=
		\omega(\vb{p})
		\hat{a}^\dagger(\vb{p})
		&&
		\comm{\hat{H}}{\hat{a}(\vb{p})}
		=
		-
		\omega(\vb{p})
		\hat{a}(\vb{p})
		.
	\end{align}
\end{corollary}
\begin{definition}
	The number density and total number operators are
	\begin{align}
		\hat{n}(\vb{p})
		=
		\hat{a}^\dagger(\vb{p})
		\hat{a}(\vb{p})
		&&
		\hat{N}
		=
		\int\frac{\dd[3]{p}}{(2\pi)^3}
		\hat{n}(\vb{p})
		=
		\int\frac{\dd[3]{p}}{(2\pi)^3}
		\hat{a}^\dagger(\vb{p})
		\hat{a}(\vb{p})
		.
	\end{align}
\end{definition}
\begin{corollary}
	The total momentum and number observables are conserved
	\begin{equation}
		\comm{\hat{H}}{\hat{\vb{P}}}
		=
		0
		=
		\comm{\hat{H}}{\hat{N}}
		.
	\end{equation}
\end{corollary}

\begin{definition}[Positive and negative frequency Klein-Gordon operator]
	The Klein-Gordon operators can be decomposed into the sum
	\begin{equation}
		\hat\phi(x^\mu)
		=
		\hat\phi^+(x^\mu)
		+
		\hat\phi^-(x^\mu)
	\end{equation}
	wherein the positive and negative frequency Klein-Gordon operators are
	\begin{equation}
		\begin{split}
			\hat\phi^+(x^\mu)
			&=
			\int\frac{\dd[3]{p}}{(2\pi)^3\sqrt{2\omega(\vb{p})}}
			\eval{
				e^{-ip_\mu x^\mu}
				\hat{a}(\vb{p})
			}_{p_0=\omega(\vb{p})}
			\\
			\hat\phi^-(x^\mu)
			&=
			\int\frac{\dd[3]{p}}{(2\pi)^3\sqrt{2\omega(\vb{p})}}
			\eval{
				e^{+ip_\mu x^\mu}
				\hat{a}^\dagger(\vb{p})
			}_{p_0=\omega(\vb{p})}
		\end{split}
		\label{eq:qkg_positive_negative_frequency}
		.
	\end{equation}
	The positive and negative frequency Klein-Gordon operators are related by the hermitian conjugate
	\begin{equation}
		\hat\phi^-(x^\mu)
		=
		\hat\phi^+(x^\mu)^\dagger
		.
	\end{equation}
\end{definition}
\begin{corollary}\label{thm:vacuum_state_pn}
	The positive and negative frequency Klein-Gordon operators satisfy
	\begin{align}
		\hat\phi^+(x^\mu)
		\ket{0}
		=
		0
		&&
		\bra{0}
		\hat\phi^-(x^\mu)
		=
		0
		\label{eq:vacuum_state_pn}
		.
	\end{align}
\end{corollary}
\begin{definition}[Klein-Gordon Propagator]
	The Klein-Gordon propagator is
	\begin{equation}
		D(x^\mu-y^\mu)
		=
		\int\frac{\dd[3]{p}}{(2\pi)^32\omega(\vb{p})}
		\eval{
			e^{-ip_\mu (x^\mu-y^\mu)}
		}_{p_0=\omega(\vb{p})}
		.
	\end{equation}
\end{definition}
\begin{restatable}{lemma}{qkgcommpn}\label{thm:qkg_comm_pn}
	The positive and negative frequency Klein-Gordon operators satisfy the commutation relations
	\begin{align}
		\comm{\hat\phi^+(x^\mu)}{\hat\phi^-(y^\mu)}
		&=
		D(x^\mu-y^\mu)
		\\
		\comm{\hat\phi^+(x^\mu)}{\hat\phi^+(y^\mu)}
		&=
		\comm{\hat\phi^-(x^\mu)}{\hat\phi^-(y^\mu)}
		=
		0
	\end{align}
\end{restatable}
\begin{restatable}{lemma}{qkgpropcorr}\label{thm:qkg_prop_corr}
	The propagator is equal to the expectation value
	\begin{equation}
		D(x^\mu-y^\mu)
		=
		\expval{\hat\phi(x^\mu)\hat\phi(y^\mu)}{0}
		.
	\end{equation}
\end{restatable}

\begin{definition}
	The generalized quadrature operator is
	\begin{equation}
		\hat\chi(\theta)
		=
		\frac{1}{2}
		\int\frac{\dd[3]{p}}{(2\pi)^3}
		\left\{
			\hat{a}^\dagger(\vb{p})
			e^{+i\theta}
			+
			\hat{a}(\vb{p})
			e^{-i\theta}
		\right\}
	\end{equation}
	and reduces to the quadrature and in-phase operators
	\begin{align}
		\hat{X}
		&=
		\frac{1}{2}
		\int\frac{\dd[3]{p}}{(2\pi)^3}
		\left\{
			\hat{a}^\dagger(\vb{p})
			+
			\hat{a}(\vb{p})
		\right\}
		=
		\hat\chi(0)
		\\
		\hat{P}
		&=
		\frac{i}{2}
		\int\frac{\dd[3]{p}}{(2\pi)^3}
		\left\{
			\hat{a}^\dagger(\vb{p})
			-
			\hat{a}(\vb{p})
		\right\}
		=
		\hat\chi\left(\frac{\pi}{2}\right)
		.
	\end{align}
\end{definition}
\begin{lemma}
	\begin{equation}
		\comm{\hat\chi(\theta)}{\hat\chi(\theta^\prime)}
		\propto
		i\sin(\theta-\theta^\prime)
	\end{equation}
\end{lemma}
\begin{proof}
	\begin{equation*}
		\begin{split}
			\comm{\hat\chi(\theta)}{\hat\chi(\theta-\Delta\theta)}
			&=
			\frac{1}{4}
			\int\frac{\dd[3]{p}}{(2\pi)^3}
			\int\frac{\dd[3]{q}}{(2\pi)^3}
			\comm{
				\hat{a}^\dagger(\vb{p})
				e^{+i\theta}
				+
				\hat{a}(\vb{p})
				e^{-i\theta}
			}{
				\hat{a}^\dagger(\vb{q})
				e^{+i(\theta-\Delta\theta)}
				+
				\hat{a}(\vb{q})
				e^{-i(\theta-\Delta\theta)}
			}
			\\
			&=
			\frac{1}{4}
			\int\frac{\dd[3]{p}}{(2\pi)^3}
			\int\frac{\dd[3]{q}}{(2\pi)^3}
			\left\{
				\comm{\hat{a}^\dagger(\vb{p})}{\hat{a}(\vb{q})}
				e^{+i\Delta\theta}
				+
				\comm{\hat{a}(\vb{p})}{\hat{a}^\dagger(\vb{q})}
				e^{-i\Delta\theta}
			\right\}
			\\
			&=
			\frac{1}{4}
			\int\frac{\dd[3]{p}}{(2\pi)^3}
			\left\{
				-
				e^{+i\Delta\theta}
				+
				e^{-i\Delta\theta}
			\right\}
			\\
			&=
			\frac{i}{2}
			\int\frac{\dd[3]{p}}{(2\pi)^3}
			\Im e^{+i\Delta\theta}
			\\
			&=
			\frac{i}{2}
			\sin\Delta\theta
			\frac{4\pi}{(2\pi)^3}
			\int_0^\Lambda \dd{p}p^2
			\\
			&=
			i
			\sin\Delta\theta
			\frac{\Lambda^3}{3(2\pi)^2}
		\end{split}
	\end{equation*}
\end{proof}
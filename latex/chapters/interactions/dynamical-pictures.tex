\section{Dynamical pictures}

\begin{definition}
	In the Heisenberg picture, the operators satisfy the Heisenberg equation of motion
	\begin{equation}
		i\partial_t
		\hat{O}^H(t)
		=
		\comm{\hat{O}^H(t)}{\hat{H}}
	\end{equation}
	and the states are constant, i.e.,
	\begin{equation}
		\ket{\psi(t)}^H
		=
		\ket{\psi(0)}^H
		=
		\ket{\psi}^H
		.
	\end{equation}
\end{definition}
\begin{lemma}\label{thm:heisenberg_eom_sol}
	The Heisenberg equation of motion is solved by
	\begin{equation}
		\hat{O}^H(t)
		=
		e^{+i\hat{H}t}
		\hat{O}^H(0)
		e^{-i\hat{H}t}
		.
	\end{equation}
\end{lemma}
\begin{definition}
	In the Schrödinger picture, the operators are time-independent, and the states satisfy the Schrödinger equation
	\begin{equation}
		i\partial_t
		\ket{\psi(t)}^S
		=
		\hat{H}
		\ket{\psi(t)}^S
	\end{equation}
\end{definition}
\begin{lemma}\label{thm:schroedinger_eom_sol}
	For a time-independent Hamiltonian, the state solving the Schrödinger equation is
	\begin{equation}
		\ket{\psi(t)}^S
		=
		e^{-i\hat{H}t}
		\ket{\psi(0)}^S
		.
	\end{equation}
\end{lemma}
\begin{theorem}\label{thm:heisenberg_schroedinger_equivalence}
	Operators and states in the Heisenberg and Schrödinger picture are related by
	\begin{align}
		\hat{O}^S
		&=
		\hat{O}^H(0)
		\\
		\ket{\psi(0)}^S
		&=
		\ket{\psi}^H
		,
	\end{align}
	and are equivalent with respect to their physical predictions.
\end{theorem}
\begin{definition}
	A canonical transformation leaves the commutation relations unchanged.	
\end{definition}
\begin{lemma}\label{thm:heisenberg_schroedinger_canonical_transformation}
	Heisenberg and Schrödinger picture are related by a canonical transformation.
\end{lemma}
\begin{definition}
	If we can write the Hamiltonian $\hat{H}$ as the sum of a free $\hat{H}_0$ and an interacting part $\hat{H}_\text{int}$,
	\begin{equation}
		\hat{H}
		=
		\hat{H}_0
		+
		\hat{H}_\text{int}
		,
	\end{equation}
	the Dirac (or interaction) picture is defined as
	\begin{align}
		\hat{O}^D(t)
		&=
		e^{+i\hat{H}_0^St}
		\hat{O}^S
		e^{-i\hat{H}_0^St}
		\\
		\ket{\psi(t)}^D
		&=
		e^{+i\hat{H}_0^St}
		\ket{\psi(t)}^S
	\end{align}
\end{definition}
\begin{remark}
	For $\comm{\hat{H}_0^S}{\hat{H}_\text{int}^S}=0$, we have
	\begin{align}
		\hat{O}^D(t)
		&=
		e^{+i\hat{H}_0^St}
		\hat{O}^S
		e^{-i\hat{H}_0^St}
		\\
		\ket{\psi(t)}^D
		&=
		e^{-i\hat{H}_\text{int}^St}
		\ket{\psi(0)}^S
		,
	\end{align}
	i.e., the state evolves according to the Schrödinger picture with the interaction Hamiltonian $\hat{H}_\text{int}$ while the operators evolve according to the Heisenberg picture with the free Hamiltonian $\hat{H}_0$.
\end{remark}
\begin{corollary}
	For $\hat{H}_\text{int}=0$, the Dirac picture reduces to the Heisenberg picture.
\end{corollary}
\begin{corollary}
	At $t=0$, operators and states in Heisenberg, Schrödinger, and Dirac picture coincide.
\end{corollary}
\begin{lemma}
	The Dirac picture relates to the Heisenberg picture by
	\begin{align}
		\hat{O}^D(t)
		&=
		e^{+i\hat{H}_0^St}
		e^{-i\hat{H}t}
		\hat{O}^H(t)
		e^{+i\hat{H}t}
		e^{-i\hat{H}_0^St}
		\\
		\ket{\psi(t)}^D
		&=
		e^{+i\hat{H}_0^St}
		e^{-i\hat{H}t}
		\ket{\psi(t)}^S
		.
	\end{align}
\end{lemma}

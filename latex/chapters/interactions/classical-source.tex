\section{Classical source}

\begin{definition}
	The Lagrangian density describing the interaction between a quantized Klein-Gordon field $\hat\phi(x)$ and an external classical source $J(x)$ is
	\begin{equation}
		\mathcal{L}_\text{int}
		=
		J(x)
		\hat\phi(x)
		\label{eq:qkg_classical_interaction}
		.
	\end{equation}
\end{definition}
\begin{theorem}
	The scattering operator $\hat{S}$ of the Klein-Gordon field corresponding to the interaction with a classical source $J(x)$, see \cref{eq:qkg_classical_interaction}, is equal to
	\begin{equation}
		\hat{S}
		=
		N\left[
			\exp\left\{
				-i
				\int\dd[4]{x}
				J(x)
				\hat\phi(x)
			\right\}
		\right]
		\exp\left\{
			-
			\frac{1}{2}
			\iint\dd[4]{x}\dd[4]{y}
			J(x)
			D(x-y)
			J(y)
		\right\}
		\label{eq:scattering_operator_classical_interaction}
	\end{equation}
	wherein $D(x-y)$ is the Feynman propagator of the Klein-Gordon field.
\end{theorem}
\begin{proof}
	See Ref.~\cite{Zhang1999} and Ref.~\cite[p.~180]{Itzykson2012} where the scattering operator is calculated for the Maxwell field.
	
	\begin{equation*}
		\begin{split}
			\hat{S}
			&=
			T\left[
				\exp\left\{
					i
					\int\dd[4]{x}
					\mathcal{L}_\text{int}(x^\mu)
				\right\}
			\right]
			\\
			&=
			T\left[
				\exp\left\{
					i
					\int\dd[4]{x}
					J(x^\mu)
					\hat\phi(x^\mu)
				\right\}
			\right]
			\\
			&=
			\sum_{n=0}^\infty
			\frac{1}{n!}
			i^n
			\hat{S}_n
		\end{split}
	\end{equation*}
	where
	\begin{equation*}
		\begin{split}
			\hat{S}_n
			&=
			\int\dd[4]{x_1}J(x_1)\dots J(x_n)\dd[4]{x_n}
			T\left[
				\hat\phi(x_1)
				\dots
				\hat\phi(x_n)
			\right]
		\end{split}
	\end{equation*}
	and the time-ordered product
	\begin{equation*}
		\begin{split}
			T\left[
				\hat\phi(x_1)
				\dots
				\hat\phi(x_n)
			\right]
			&=
			N\left[
				\hat\phi(x_1)
				\dots
				\hat\phi(x_n)
			\right]
			+
			\text{sum of contracted normal products}			
		\end{split}
	\end{equation*}
	\begin{align*}
		S_{n=0,1}
		&=
		N\left[\hat\phi[J]^n\right]
		\\
		S_{n=2,3}
		&=
		N\left[\hat\phi[J]^n\right]
		+
		\binom{n}{2,n-2}
		\comm{\hat\phi^-[J]}{\hat\phi^+[J]}
		N\left[\hat\phi[J]^{n-2}\right]
		\\
		S_{n=4,5}
		&=
		N\left[\hat\phi[J]^n\right]
		+
		\binom{n}{2,n-2}
		\comm{\hat\phi^-[J]}{\hat\phi^+[J]}
		N\left[\hat\phi[J]^{n-2}\right]
		+
		\binom{n}{2,2,n-4}
		\comm{\hat\phi^-[J]}{\hat\phi^+[J]}^2
		N\left[\hat\phi[J]^{n-4}\right]
	\end{align*}
	\begin{equation*}
		\begin{split}
			\hat{S}
			&=
			\sum_{n=0}^\infty
			\frac{1}{n!}
			i^n
			\hat{S}_n
			\\
			&=
			\sum_{n=0}^\infty
			\frac{1}{n!}
			i^n
			N\left[\hat\phi[J]^n\right]
			+
			\sum_{n=2}^\infty
			\frac{1}{n!}
			i^n
			\binom{n}{2,n-2}
			\comm{\hat\phi^-[J]}{\hat\phi^+[J]}
			N\left[\hat\phi[J]^{n-2}\right]
			\\
			&+
			\sum_{n=4}^\infty
			\frac{1}{n!}
			i^n
			\binom{n}{2,2,n-4}
			\comm{\hat\phi^-[J]}{\hat\phi^+[J]}^2
			N\left[\hat\phi[J]^{n-4}\right]
			+
			\dots
			\\
			&=
			\sum_{n=0}^\infty
			\frac{1}{n!}
			i^n
			N\left[\hat\phi[J]^n\right]
			+
			\sum_{n=2}^\infty
			i^n
			\frac{1}{(n-2)!}
			\left(
				\frac{1}{2}
				\comm{\hat\phi^-[J]}{\hat\phi^+[J]}
			\right)
			N\left[\hat\phi[J]^{n-2}\right]
			\\
			&+
			\sum_{n=4}^\infty
			i^n
			\frac{1}{(n-4)!}
			\left(
				\frac{1}{2}
				\comm{\hat\phi^-[J]}{\hat\phi^+[J]}
			\right)^2
			N\left[\hat\phi[J]^{n-4}\right]
			+
			\dots
			\\
			&=
			\sum_{n=0}^\infty
			\frac{1}{n!}
			i^n
			N\left[\hat\phi[J]^n\right]
			+
			\sum_{n=0}^\infty
			i^n
			\frac{1}{n!}
			\left(
				-
				\frac{1}{2}
				\comm{\hat\phi^-[J]}{\hat\phi^+[J]}
			\right)
			N\left[\hat\phi[J]^n\right]
			\\
			&+
			\sum_{n=4}^\infty
			i^n
			\frac{1}{n!}
			\left(
				-
				\frac{1}{2}
				\comm{\hat\phi^-[J]}{\hat\phi^+[J]}
			\right)^2
			N\left[\hat\phi[J]^n\right]
			+
			\dots
			\\
			&=
			N\left[
				\sum_{n=0}^\infty
				\frac{1}{n!}
				i^n
				\hat\phi[J]^n
			\right]
			\sum_{m=0}^\infty
			\textcolor{red}{\frac{1}{m!}}
			\left(
				-
				\frac{1}{2}
				\comm{\hat\phi^-[J]}{\hat\phi^+[J]}
			\right)^m
			\\
			&=
			N\exp\left\{
				i
				\hat\phi[J]
			\right\}
			\exp\left\{
				-
				\frac{1}{2}
				\comm{\hat\phi^-[J]}{\hat\phi^+[J]}
			\right\}
		\end{split}
	\end{equation*}
\end{proof}
\begin{theorem}\label{thm:displacement_scattering_operator_equivalence}
	The $\hat{S}$ operator of \cref{eq:scattering_operator_classical_interaction} is equal to the phase-shifted displacement operator
	\begin{equation}
		\hat{S}
		=
		\hat{D}[-iJ]
		.
	\end{equation}
\end{theorem}

\begin{equation}
	\mathcal{L}_\text{int}(x^\mu)
	=
	-
	J_\nu(x^\mu)
	\hat{A}^\nu(x^\mu)
	=
	\vb{J}_\perp(x^\mu)
	\vdot
	\hat{\vb{A}}_\perp(x^\mu)
\end{equation}
\begin{equation}
	\begin{split}
		\hat{S}
		&=
		T
		\exp\left\{
			i
			\int\dd[4]{x}
			\mathcal{L}_\text{int}
		\right\}
		\\
		&=
		T
		\exp\left\{
			i
			\int\dd[4]{x}
			\vb{J}_\perp(x^\mu)
			\vdot
			\hat{\vb{A}}_\perp(x^\mu)
		\right\}
		\\
		&=
		\exp\left\{
			-
			\frac{1}{2}
			\comm{\hat{\vb{A}}_\perp^-[\vb{J}_\perp]}{\hat{\vb{A}}_\perp^+[\vb{J}_\perp]}
		\right\}
		N\exp\left\{
			i
			\hat{\vb{A}}_\perp[\vb{J}_\perp]
		\right\}
		\\
		&=
		\exp\left\{
			-
			\frac{1}{2}
			\comm{\hat{\vb{A}}_\perp^-[\vb{J}_\perp]}{\hat{\vb{A}}_\perp^+[\vb{J}_\perp]}
		\right\}
		\exp\left\{
			+
			\hat{\vb{A}}_\perp^+[-i\vb{J}_\perp]
		\right\}
		\exp\left\{
			-
			\hat{\vb{A}}_\perp^-[-i\vb{J}_\perp]
		\right\}
	\end{split}
\end{equation}
where
\begin{equation}
	\overline{n}
	=
	\sum_{\lambda=1,2}
	\int\frac{\dd[3]{p}}{(2\pi)^32\omega(\vb{p})}
	\abs{J_\lambda\left(\omega(\vb{p}),\vb{p}\right)}^2
	=
	\comm{\hat{\vb{A}}_\perp^-[\vb{J}_\perp]}{\hat{\vb{A}}_\perp^+[\vb{J}_\perp]}
\end{equation}
is the energy the Maxwell field gets from the source.

\begin{example}[Oscillating dipole]
	A point-particle with charge $e$ oscillating from $-\vb{r}$ to $+\vb{r}$ creates the current
	\begin{equation*}
		\begin{split}
			\vb{j}(x^\mu)
			&=
			e
			\int_\mathbb{R}\dd{\tau}
			\dv{\vb{r}}{\tau}
			\delta^{(1)}(x^0-\tau)
			\delta^{(3)}\left(\vb{x}-\vb{r}(\tau)\right)
			\\
			&=
			e\vb{r}_0\omega_0
			\int_\mathbb{R}\dd{\tau}
			\cos(\omega_0\tau)
			\delta^{(1)}(x^0-\tau)
			\delta^{(3)}\left(\vb{x}-\vb{r}_0\sin(\omega_0\tau)\right)
			.
		\end{split}
	\end{equation*}
	In momentum space, the current takes the form
	\begin{equation*}
		\begin{split}
			\vb{j}(p^\mu)
			=
			e\vb{r}_0\omega_0
			\int_\mathbb{R}\dd{\tau}
			\cos(\omega_0\tau)
			\exp\left\{
				i\left[
					p_0\tau
					-
					\vb{p}\vdot\vb{r}_0
					\sin(\omega_0\tau)
				\right]
			\right\}
		\end{split}
		.
	\end{equation*}
	In the near field, we have $\vb{r}_0\vdot\vb{p}\ll1$ and the current comprises the fundamental frequency $\omega_0$ and its second harmonic $2\omega_0$
	\begin{equation*}
		\begin{split}
			\vb{j}(p^\mu)
			&\approx
			e\vb{r}_0\omega_0
			\int_\mathbb{R}\dd{\tau}
			\cos(\omega_0\tau)
			e^{ip_0\tau}
			\left(
				1
				-
				i
				\vb{p}
				\vdot
				\vb{r}_0
				\sin(\omega_0\tau)
			\right)
			\\
			&=
			\frac{e\vb{r}_0\omega_0}{4}
			\left\{
				\left[
					\delta^{(1)}(p_0+\omega_0)
					+
					\delta^{(1)}(p_0-\omega_0)
				\right]
				-
				2
				\vb{p}
				\vdot
				\vb{r}_0
				\left[
					\delta^{(1)}(p_0+2\omega_0)
					-
					\delta^{(1)}(p_0-2\omega_0)
				\right]				
			\right\}
			.
		\end{split}
	\end{equation*}
	The projection of the polarization basis yields
	\begin{equation*}
		\boldsymbol{\varepsilon_\lambda}(\vb{p})
		\vdot
		\vb{j}(p^\mu)
	\end{equation*}
\end{example}
\begin{example}[Bremsstrahlung]
	A particle with charge $q$ changes momentum from $k_i^\mu$ to $k_f^\mu$ at proper time $\tau=0$, i.e.,
	\begin{equation}
		\vb{y}(\tau)
		=
		\begin{cases}
			\vb{k}_i\tau/m
			&
			\tau<0
			\\
			\vb{k}_f\tau/m
			&
			\tau>0
		\end{cases}
		.
	\end{equation}
	The coordinate representation of the current is~\cite[p.~178]{Peskin1995}
	\begin{equation}
		\begin{split}
			j^\mu(x^\nu)
			&=
			q\int_{-\infty}^{+\infty}\dd{\tau}
			\dv{y^\mu}{\tau}
			\delta^{(4)}\left(
				x^\nu
				-
				y^\nu(\tau)
			\right)
			\\
			&=
			\frac{qk_f^\mu}{m}
			\int_0^{+\infty}\dd{\tau}
			\delta^{(4)}\left(x^\nu-\frac{k_f^\nu}{m}\tau\right)
			+
			\frac{qk_i^\mu}{m}
			\int_{-\infty}^0\dd{\tau}
			\delta^{(4)}\left(x^\nu-\frac{k_i^\nu}{m}\tau\right)
		\end{split}
	\end{equation}
	and its momentum representation turns out to be~\cite[p.~39]{Itzykson2012}
	\begin{equation*}
		j^\mu(p^\nu)
		=
		iq\left(
			\frac{k_f^\mu}{p_\nu k_f^\nu}
			-
			\frac{k_i^\mu}{p_\nu k_i^\nu}
		\right)
		.
	\end{equation*}
	The number of emitted photons is~\cite[p.~41]{Itzykson2012}
	\begin{equation*}
		\begin{split}
			\overline{n}
			&=
			\int\frac{\dd[3]{p}}{(2\pi)^32\omega(\vb{p})}
			\sum_{\lambda=1,2}
			\abs{
				\boldsymbol{\varepsilon}_\lambda(\vb{p})
				\vdot
				\vb{j}\left(\omega(\vb{p}),\vb{p}\right)
			}^2
			\\
			&=
			\int\frac{\dd[3]{p}}{(2\pi)^32\omega(\vb{p})}
			\left\{
				\frac{2g_{\mu\nu}k_i^\mu k_f^\nu}{(p_\mu k_i^\mu)(p_\nu k_f^\nu)}
				-
				\frac{m^2}{(p_\mu k_i^\mu)^2}
				-
				\frac{m^2}{(p_\mu k_f^\mu)^2}
			\right\}_{p_0=\omega(\vb{k})}
		\end{split}
	\end{equation*}
	and the corresponding coherent state is
	\begin{equation*}
		\ket{\alpha}
		=
		\hat{D}[-i\vb{j}]
		\ket{0}
		=
		e^{-\overline{n}/2}
		\exp\left\{
			i
			\int\frac{\dd[3]{p}}{(2\pi)^3\sqrt{2\omega(\vb{p})}}
			\sum_{\lambda=1,2}
			\left[
				\boldsymbol{\varepsilon}_\lambda(\vb{p})
				\vdot
				\vb{j}\left(\omega(\vb{p}),\vb{p}\right)
			\right]^*
			\hat{a}^\dagger(\vb{p})
		\right\}
		\ket{0}
		.
	\end{equation*}
\end{example}
\begin{example}[Laser source]
	
\end{example}
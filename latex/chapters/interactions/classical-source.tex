\section{Classical source}

\begin{definition}
	The Lagrangian density describing the interaction between a quantized Klein-Gordon field $\hat\phi(x)$ and an external classical source $J(x)$ is
	\begin{equation}
		\mathcal{L}_\text{int}
		=
		J(x)
		\hat\phi(x)
		\label{eq:qkg_classical_interaction}
		.
	\end{equation}
\end{definition}
\begin{theorem}
	The scattering operator $\hat{S}$ of the Klein-Gordon field corresponding to the interaction with a classical source $J(x)$, see \cref{eq:qkg_classical_interaction}, is equal to
	\begin{equation}
		\hat{S}
		=
		N\left[
			\exp\left\{
				-i
				\int\dd[4]{x}
				J(x)
				\hat\phi(x)
			\right\}
		\right]
		\exp\left\{
			-
			\frac{1}{2}
			\iint\dd[4]{x}\dd[4]{y}
			J(x)
			D(x-y)
			J(y)
		\right\}
		\label{eq:scattering_operator_classical_interaction}
	\end{equation}
	wherein $D(x-y)$ is the Feynman propagator of the Klein-Gordon field.
\end{theorem}
\begin{proof}
	See Ref.~\cite{Zhang1999} and Ref.~\cite[p.~180]{Itzykson2012} where the scattering operator is calculated for the Maxwell field.
	
	\begin{equation*}
		\begin{split}
			\hat{S}
			&=
			T\left[
				\exp\left\{
					i
					\int\dd[4]{x}
					\mathcal{L}_\text{int}(x^\mu)
				\right\}
			\right]
			\\
			&=
			T\left[
				\exp\left\{
					i
					\int\dd[4]{x}
					J(x^\mu)
					\hat\phi(x^\mu)
				\right\}
			\right]
			\\
			&=
			\sum_{n=0}^\infty
			\frac{1}{n!}
			i^n
			\hat{S}_n
		\end{split}
	\end{equation*}
	where
	\begin{equation*}
		\begin{split}
			\hat{S}_n
			&=
			\int\dd[4]{x_1}J(x_1)\dots J(x_n)\dd[4]{x_n}
			T\left[
				\hat\phi(x_1)
				\dots
				\hat\phi(x_n)
			\right]
		\end{split}
	\end{equation*}
	and the time-ordered product
	\begin{equation*}
		\begin{split}
			T\left[
				\hat\phi(x_1)
				\dots
				\hat\phi(x_n)
			\right]
			&=
			N\left[
				\hat\phi(x_1)
				\dots
				\hat\phi(x_n)
			\right]
			+
			\text{sum of contracted normal products}			
		\end{split}
	\end{equation*}
	\begin{align*}
		S_{n=0,1}
		&=
		N\left[\hat\phi[J]^n\right]
		\\
		S_{n=2,3}
		&=
		N\left[\hat\phi[J]^n\right]
		+
		\binom{n}{2,n-2}
		\comm{\hat\phi^-[J]}{\hat\phi^+[J]}
		N\left[\hat\phi[J]^{n-2}\right]
		\\
		S_{n=4,5}
		&=
		N\left[\hat\phi[J]^n\right]
		+
		\binom{n}{2,n-2}
		\comm{\hat\phi^-[J]}{\hat\phi^+[J]}
		N\left[\hat\phi[J]^{n-2}\right]
		+
		\binom{n}{2,2,n-4}
		\comm{\hat\phi^-[J]}{\hat\phi^+[J]}^2
		N\left[\hat\phi[J]^{n-4}\right]
	\end{align*}
	\begin{equation*}
		\begin{split}
			\hat{S}
			&=
			\sum_{n=0}^\infty
			\frac{1}{n!}
			i^n
			\hat{S}_n
			\\
			&=
			\sum_{n=0}^\infty
			\frac{1}{n!}
			i^n
			N\left[\hat\phi[J]^n\right]
			+
			\sum_{n=2}^\infty
			\frac{1}{n!}
			i^n
			\binom{n}{2,n-2}
			\comm{\hat\phi^-[J]}{\hat\phi^+[J]}
			N\left[\hat\phi[J]^{n-2}\right]
			\\
			&+
			\sum_{n=4}^\infty
			\frac{1}{n!}
			i^n
			\binom{n}{2,2,n-4}
			\comm{\hat\phi^-[J]}{\hat\phi^+[J]}^2
			N\left[\hat\phi[J]^{n-4}\right]
			+
			\dots
			\\
			&=
			\sum_{n=0}^\infty
			\frac{1}{n!}
			i^n
			N\left[\hat\phi[J]^n\right]
			+
			\sum_{n=2}^\infty
			i^n
			\frac{1}{(n-2)!}
			\left(
				\frac{1}{2}
				\comm{\hat\phi^-[J]}{\hat\phi^+[J]}
			\right)
			N\left[\hat\phi[J]^{n-2}\right]
			\\
			&+
			\sum_{n=4}^\infty
			i^n
			\frac{1}{(n-4)!}
			\left(
				\frac{1}{2}
				\comm{\hat\phi^-[J]}{\hat\phi^+[J]}
			\right)^2
			N\left[\hat\phi[J]^{n-4}\right]
			+
			\dots
			\\
			&=
			\sum_{n=0}^\infty
			\frac{1}{n!}
			i^n
			N\left[\hat\phi[J]^n\right]
			+
			\sum_{n=0}^\infty
			i^n
			\frac{1}{n!}
			\left(
				-
				\frac{1}{2}
				\comm{\hat\phi^-[J]}{\hat\phi^+[J]}
			\right)
			N\left[\hat\phi[J]^n\right]
			\\
			&+
			\sum_{n=4}^\infty
			i^n
			\frac{1}{n!}
			\left(
				-
				\frac{1}{2}
				\comm{\hat\phi^-[J]}{\hat\phi^+[J]}
			\right)^2
			N\left[\hat\phi[J]^n\right]
			+
			\dots
			\\
			&=
			N\left[
				\sum_{n=0}^\infty
				\frac{1}{n!}
				i^n
				\hat\phi[J]^n
			\right]
			\sum_{m=0}^\infty
			\textcolor{red}{\frac{1}{m!}}
			\left(
				-
				\frac{1}{2}
				\comm{\hat\phi^-[J]}{\hat\phi^+[J]}
			\right)^m
			\\
			&=
			N\exp\left\{
				i
				\hat\phi[J]
			\right\}
			\exp\left\{
				-
				\frac{1}{2}
				\comm{\hat\phi^-[J]}{\hat\phi^+[J]}
			\right\}
		\end{split}
	\end{equation*}
\end{proof}
\begin{theorem}\label{thm:displacement_scattering_operator_equivalence}
	The $\hat{S}$ operator of \cref{eq:scattering_operator_classical_interaction} is equal to the phase-shifted displacement operator
	\begin{equation}
		\hat{S}
		=
		\hat{D}[-iJ]
		.
	\end{equation}
\end{theorem}

\subsection{Detector}

\begin{example}
	The Hamiltonian of a Klein-Gordon field coupled to a two-level quantum system is
	\begin{equation}
		\hat{H}
		=
		\int\frac{\dd[3]{p}}{(2\pi)^3\sqrt{2\omega(\vb{p})}}
		\omega(\vb{p})
		\hat{a}^\dagger(\vb{p})
		\hat{a}(\vb{p})
		+
		\int\frac{\dd[3]{p}}{(2\pi)^3\sqrt{2\omega(\vb{p})}}
		\left\{
			\gamma(\vb{p})
			\hat{a}^\dagger(\vb{p})
			\hat{b}
			+0
			\gamma(\vb{p})^*
			\hat{a}(\vb{p})
			\hat{b}^\dagger
		\right\}
		+
		\frac{1}{2}
		\omega_0
		\hat{b}^\dagger
		\hat{b}
	\end{equation}
\end{example}

\subsection{Laser}

\section{Classical source and sink}

\begin{equation}
	\mathcal{L}_\text{int}(x^\mu)
	=
	-
	J_\nu(x^\mu)
	\hat{A}^\nu(x^\mu)
	=
	\vb{J}_\perp(x^\mu)
	\vdot
	\hat{\vb{A}}_\perp(x^\mu)
\end{equation}
\begin{equation}
	\begin{split}
		\hat{S}
		&=
		T
		\exp\left\{
			i
			\int\dd[4]{x}
			\mathcal{L}_\text{int}
		\right\}
		\\
		&=
		T
		\exp\left\{
			i
			\int\dd[4]{x}
			\vb{J}_\perp(x^\mu)
			\vdot
			\hat{\vb{A}}_\perp(x^\mu)
		\right\}
		\\
		&=
		\exp\left\{
			-
			\frac{1}{2}
			\comm{\hat{\vb{A}}_\perp^-[\vb{J}_\perp]}{\hat{\vb{A}}_\perp^+[\vb{J}_\perp]}
		\right\}
		N\exp\left\{
			i
			\hat{\vb{A}}_\perp[\vb{J}_\perp]
		\right\}
		\\
		&=
		\exp\left\{
			-
			\frac{1}{2}
			\comm{\hat{\vb{A}}_\perp^-[\vb{J}_\perp]}{\hat{\vb{A}}_\perp^+[\vb{J}_\perp]}
		\right\}
		\exp\left\{
			+
			\hat{\vb{A}}_\perp^+[-i\vb{J}_\perp]
		\right\}
		\exp\left\{
			-
			\hat{\vb{A}}_\perp^-[-i\vb{J}_\perp]
		\right\}
	\end{split}
\end{equation}
where
\begin{equation}
	\overline{n}
	=
	\sum_{\lambda=1,2}
	\int\frac{\dd[3]{p}}{(2\pi)^32\omega(\vb{p})}
	\abs{J_\lambda\left(\omega(\vb{p}),\vb{p}\right)}^2
	=
	\comm{\hat{\vb{A}}_\perp^-[\vb{J}_\perp]}{\hat{\vb{A}}_\perp^+[\vb{J}_\perp]}
\end{equation}
is the energy the Maxwell field gets from the source.

\begin{example}[Electric dipole]
	Let's consider the source
	\begin{equation*}
		\vb{J}_\perp(t,\vb{x})
		=
		\vb{J}_\perp(\vb{x})
		\cos\omega_0 t
	\end{equation*}
	which in momentum space becomes
	\begin{equation*}
		\vb{J}_\perp(p_0,\vb{p})
		=
		(2\pi)^3
		\left[
			\delta^{(1)}(p_0-\omega_0)
			+
			\delta^{(1)}(p_0+\omega_0)
		\right]
		\vb{J}_\perp(\vb{p})
	\end{equation*}
	and the mean number of photons
	\begin{equation*}
		\overline{n}
		=
	\end{equation*}
\end{example}
\begin{example}[Bremsstrahlung]
	
\end{example}

\section{Quantum harmonic oscillator}

Interaction of the Maxwell field with a quantum harmonic oscillator (discrete energy levels).
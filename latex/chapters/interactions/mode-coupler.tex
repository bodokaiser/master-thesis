\section{Mode couplers}

Let $\ket{\boldsymbol{\alpha}_1}$ be a first coherent state $\vb{j}_1$ be a first classical current induced by absorption of the coherent state, i.e.,
\begin{equation*}
	\vb{j}_1(p^\mu)
	=
	i\boldsymbol{\alpha}_1(p^\mu)
	.
\end{equation*}
The first classical current is coupled to a second classical current by a classical transfer function $g(x^\mu)$
\begin{equation*}
	\vb{j}_2(x^\mu)
	=
	\int\dd[4]{y}
	g(x^\mu-y^\mu)
	\vb{j}_1(y^\mu)
\end{equation*}
which emits a second coherent state $\ket{\boldsymbol{\alpha}_2}$ with
\begin{equation*}
	\begin{split}
		\boldsymbol{\alpha}_2(p^\mu)
		=
		-i\vb{j}_2(p^\mu)
		&=
		-i
		\int\dd[4]{x}
		\vb{j}_2(x^\mu)
		e^{ip_\nu x^\nu}
		\\
		&=
		-i
		\int\dd[4]{x}
		g(x^\mu-y^\mu)
		e^{ip_\nu x^\nu}
		\int\dd[4]{y}
		\vb{j}_1(y^\mu)
		\\
		&=
		-i
		\int\dd[4]{x}
		g(x^\mu)
		e^{ip_\nu x^\nu}
		\int\dd[4]{y}
		i\boldsymbol{\alpha}_1(y^\mu)
		e^{ip_\nu y^\nu}
		\\
		&=
		g(p^\mu)
		\boldsymbol{\alpha}_1(p^\mu)
		.
	\end{split}
\end{equation*}
Alternatively, a quantum mode coupler with transfer function $g_{\lambda\lambda^\prime}$ is described by the interaction Lagrangian
\begin{equation}
	L_\text{int}
	=
	\sum_{\lambda,\lambda^\prime=1,2}
	\int\frac{\dd[3]{p}}{(2\pi)^32\omega(\vb{p})}
	\left\{
		g_{\lambda\lambda^\prime}\left(\omega(\vb{p}),\vb{p}\right)
		\hat{a}_{\text{out},\lambda^\prime}^\dagger(\vb{p})
		\hat{a}_{\text{in},\lambda}(\vb{p})
		+
		\text{h.c.}
	\right\}
	.
\end{equation}
In terms of smeared positive and negative frequency Maxwell operators, the interaction Lagrangian takes the form~\cite[p.~130]{Haroche2006}
\begin{equation*}
	\begin{split}
		L_\text{int}
		&=
		\int\dd[4]{x}
		\mathcal{L}_\text{int}
		=
		\hat{\vb{A}}_\text{out}^+[g_\text{out}]
		\hat{\vb{A}}_\text{in}^-[g_\text{in}]
		+
		\text{h.c.}
		\\
		&=
		\int\frac{\dd[3]{p}}{(2\pi)^3\sqrt{2\omega(\vb{p})}}
		\sum_{\lambda=1,2}
		g_{\text{out},\lambda}\left(\omega(\vb{p}),\vb{p}\right)^*
		\hat{a}_{\text{out},\lambda}^\dagger(\vb{p})
		\\
		&\times
		\int\frac{\dd[3]{q}}{(2\pi)^3\sqrt{2\omega(\vb{q})}}
		\sum_{\lambda^\prime=1,2}
		g_{\text{in},\lambda^\prime}\left(\omega(\vb{q}),\vb{q}\right)
		\hat{a}_{\text{in},\lambda^\prime}(\vb{q})
		+
		\text{h.c.}
		\\
		&=
		\int\frac{\dd[3]{p}}{(2\pi)^32\omega(\vb{p})}
		\sum_{\lambda,\lambda^\prime=1,2}
		g_{\lambda,\lambda^\prime}\left(\omega(\vb{p}),\vb{p}\right)
		\hat{a}_{\text{out},\lambda}^\dagger(\vb{p})
		\hat{a}_{\text{in},\lambda^\prime}(\vb{p})
	\end{split}
\end{equation*}
where we used
\begin{equation*}
	g_{\lambda,\lambda^\prime}\left(\omega(\vb{p}),\vb{p}\right)
	=
	(2\pi)^3\delta^{(3)}(\vb{q}-\vb{p})
	g_{\text{out},\lambda}\left(\omega(\vb{p}),\vb{p}\right)^*
	g_{\text{in},\lambda^\prime}\left(\omega(\vb{q}),\vb{q}\right)	
\end{equation*}
which appears somewhat reasonable when we consider first-order momentum transfer.
The scattering operator is
\begin{equation*}
	\begin{split}
		\hat{S}
		=
		T\exp\left\{
			iL_\text{int}
		\right\}
		&=
		T\exp\left\{
			i
			\hat{\vb{A}}_\text{out}^+[g_\text{out}]
			\hat{\vb{A}}_\text{in}^-[g_\text{in}]
			+
			i
			\hat{\vb{A}}_\text{in}^-[g_\text{in}]
			\hat{\vb{A}}_\text{out}^+[g_\text{out}]
		\right\}
		\\
		&=
		\sum_{n=0}^\infty
		\frac{i^n}{n!}
		T\left[
			\hat{\vb{A}}_\text{out}^+[g_\text{out}]
			\hat{\vb{A}}_\text{in}^-[g_\text{in}]
			+
			\hat{\vb{A}}_\text{in}^-[g_\text{in}]
			\hat{\vb{A}}_\text{out}^+[g_\text{out}]
		\right]^n
		\\
		&=
		\sum_{n=0}^\infty
		\frac{i^n}{n!}
		\sum_{m=0}^n
		\binom{n}{m}
		T\left[
			\hat{\vb{A}}_\text{out}^+[g_\text{out}]
			\hat{\vb{A}}_\text{in}^-[g_\text{in}]
		\right]^m
		T\left[
			\hat{\vb{A}}_\text{in}^-[g_\text{in}]
			\hat{\vb{A}}_\text{out}^+[g_\text{out}]
		\right]^{n-m}
	\end{split}
\end{equation*}
The first power term is
\begin{equation*}
	\begin{split}
		&
		T\left[
			\hat{\vb{A}}_\text{out}^+[g_\text{out}]
			\hat{\vb{A}}_\text{in}^-[g_\text{in}]
		\right]^m
		\\
		=&\
		T\left[
			\int\dd[4]{x}
			g_\text{out}(x^\mu)
			\hat{\vb{A}}_\text{out}^+(x^\mu)
			\int\dd[4]{y}
			g_\text{in}(y^\mu)
			\hat{\vb{A}}_\text{in}^-(y^\mu)
		\right]^m
		\\
		=&\
		\int\dd[4]{x_1}
		\int\dd[4]{y_1}
		g_\text{out}(x^\mu_1)
		g_\text{in}(y^\mu_1)
		\dots
		\int\dd[4]{x_m}
		\int\dd[4]{y_m}
		g_\text{out}(x^\mu_m)
		g_\text{in}(y^\mu_n)
		T\left[
			\hat{\vb{A}}_\text{out}^+(x^\mu)^m
			\hat{\vb{A}}_\text{in}^-(y^\mu)^m
		\right]
	\end{split}
\end{equation*}

\subsection{Phase modulator}

The action describing the absorption and emission of a field while adding a possibly time-dependent phase-shift is
\begin{equation*}
	\begin{split}
		S
		&=
		\frac{1}{2}
		\int\dd[4]{x}
		\int\dd[4]{y}
		\hat\phi_\text{out}^+(x^\mu)
		G(x^\mu-y^\mu)
		\hat\phi_\text{in}^-(y^\mu)
		\\
		&=
		\frac{1}{2}
		\int\frac{\dd[3]{p}}{(2\pi)^3}
		G(p^\mu)
		\int\dd[4]{x}
		\hat\phi_\text{out}^+(x^\mu)
		e^{-ip_\nu x^\nu}
		\int\dd[4]{y}
		\hat\phi_\text{in}^-(y^\mu)
		e^{+ip_\nu y^\nu}
		\\
		&=
		\frac{1}{2}
		\int\frac{\dd[3]{p}}{(2\pi)^32\omega(\vb{p})}
		G\left(\omega(\vb{p}),\vb{p}\right)
		\hat{a}_\text{out}^\dagger(\vb{p})
		\hat{a}_\text{in}(\vb{p})
	\end{split}
\end{equation*}
and the corresponding scattering operator becomes
\begin{equation*}
	\hat{S}
	=
	T\left[
		\exp\left\{
			iS
		\right\}
	\right]
	=
	T\left[
		\exp\left\{
			\frac{i}{2}
			\int\frac{\dd[3]{p}}{(2\pi)^32\omega(\vb{p})}
			G\left(\omega(\vb{p}),\vb{p}\right)
			\hat{a}_\text{out}^\dagger(\vb{p})
			\hat{a}_\text{in}(\vb{p})
		\right\}
	\right]
\end{equation*}
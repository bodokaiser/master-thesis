\section{Coherent source}

In the first chapter, we showed that a classical source $\vb{J}(t,\vb{x})$ generates a coherent state via the $\hat{S}$ operator from the vacuum
\begin{equation}
	\begin{split}
		\ket{\alpha(\vb{p})}
		&=
		\exp\left\{
			-i\int\dd[4]{x}
			\vb{J}(t,\vb{x})
			\vdot
			\vb{A}(t,\vb{x})
		\right\}
		\ket{0}
		\\
		&=
		\exp\left\{
			-\frac{1}{2}
			\sum_{\lambda=1,2}
			\int\dd[3]{p}
			\boldsymbol{\varepsilon}_\lambda(\vb{p})
			\vdot
			\vb{J}(\vb{p})
		\right\}
		\\
		&\times
		\exp\left\{
			\sum_{\lambda=1,2}
			\int\dd[3]{p}
			\boldsymbol{\varepsilon}_\lambda(\vb{p})
			\vdot
			\vb{J}(\vb{p})
			\hat{a}_\lambda^\dagger(\vb{p})
		\right\}
		\ket{0}
	\end{split}
\end{equation}
and we are left to find a mechanism for $\vb{J}(\vb{p})$.

\subsection{Oscillating electron}

\subsection{Hydrogen atom}

\subsection{Harmonic oscillators}
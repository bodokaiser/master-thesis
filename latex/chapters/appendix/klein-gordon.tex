\section{Relativistic field theory}

\rem
\begin{proof}
	According to the action principle, the dynamics of the field are determined by the equations of motion which can be found by the relativistic Euler-Lagrange equations
	\begin{equation*}
		0
		=
		\partial_\mu\pdv{\mathcal{L}}{(\partial_\mu\phi)}
		-
		\pdv{\mathcal{L}}{\phi}
		=
		\left(
			\partial_\mu\partial^\mu
			+
			m^2
		\right)
		\phi(t,\vb{x})
		.
	\end{equation*}
	Assuming the existence of the Klein-Gordon field's Fourier representation\footnote{See Ref.~\cite[p.~341]{Cohen2019} for a definition of a Minkowski Fourier transform.}
	\begin{equation*}
		\phi(t,\vb{x})
		=
		\int_{\mathbb{R}^3}\frac{\dd[3]{p}}{(2\pi)^3}
		\phi(t,\vb{p})
		e^{-i\vb{p}\vdot\vb{x}}
		=
		\int_{\mathbb{R}^4}\frac{\dd[4]{p}}{(2\pi)^4}
		\phi(p_0,\vb{p})
		e^{+ip_\mu x^\mu}
		,
	\end{equation*}
	the equation of motion in momentum space reduces to
	\begin{equation*}
		0
		=
		\left(
			ip_\mu ip^\mu
			+
			m^2
		\right)
		\phi(p_0,\vb{p})
		=
		-
		\left(
			p_0^2
			-
			\omega(\vb{p})^2
		\right)
		\phi(p_0,\vb{p})
	\end{equation*}
	which is satisfied if $p_0=\pm\omega(\vb{p})$.
\end{proof}

\kgmodeexp
\begin{proof}
	There are two approaches to prove \cref{thm:kg_mode_expansion}.
	In a first approach, we calculate with the proposed Fourier expansion
	\begin{equation*}
		\begin{split}
			\partial_\mu
			\partial^\mu
			\phi(t,\vb{x})
			&=
			\int\frac{\dd[3]{p}}{(2\pi)^3\sqrt{2\omega(\vb{p})}}
			\left\{
				a^*(\vb{p})
				\partial_\mu
				\partial^\mu
				e^{-ip_\mu x^\mu}
				+
				\text{c.c.}
			\right\}_{p_0=\omega(\vb{p})}
			\\
			&=
			\int\frac{\dd[3]{p}}{(2\pi)^3\sqrt{2\omega(\vb{p})}}
			\left\{
				a^*(\vb{p})
				\left(
					-
					p_\mu
					p^\mu
				\right)
				e^{-ip_\mu x^\mu}
				+
				\text{c.c.}
			\right\}_{p_0=\omega(\vb{p})}
		\end{split}
	\end{equation*}
	where $p_\mu p^\mu=\omega(\vb{p})^2-\vb{p}=m^2$ and therefore the equation of motion
	\begin{equation*}
			\partial_\mu
			\partial^\mu
			\phi(t,\vb{x})
			=
			-
			m^2
			\phi(t,\vb{x})
	\end{equation*}
	is satisfied.
	While the first approach successfully shows why the theorem is true, it is not obvious how to arrive at the Fourier expansion.
	Therefor, a second approach starts with the Fourier transform of the Klein-Gordon field
	\begin{equation*}
		\phi(t,\vb{x})
		=
		\int_V\frac{\dd[4]{p}}{(2\pi)^4}
		\phi(p_0,\vb{p})
		e^{+ip_\mu x^\mu}
	\end{equation*}
	where the integration domain is constrained to the momentum lightcone $V$ and therefore $\omega(\vb{p})^2=p_0^2$ is automatically satisfied.
	We are left to rewrite the constrained integration domain
	\begin{equation*}
		\phi(t,\vb{x})
		=
		\int_{\mathbb{R}^4}\frac{\dd[4]{p}}{(2\pi)^3}
		\delta^{(1)}\left(p_0^2-\omega(\vb{p})^2\right)
		\phi(p_0,\vb{p})
		e^{+ip_\mu x^\mu}
	\end{equation*}
	and using the composition property of the delta distribution
	\begin{equation*}
		\delta^{(1)}\left(p_0^2-\omega(\vb{p})^2\right)
		=
		\frac{
			\delta^{(1)}\left(p_0-\omega(\vb{p})\right)
			+
			\delta^{(1)}\left(p_0+\omega(\vb{p})\right)
		}{\sqrt{2\omega(\vb{p})}}
	\end{equation*}
	which leaves us with
	\begin{equation*}
		\phi(t,\vb{x})
		=
		\int_{\mathbb{R}^3}\frac{\dd[3]{p}}{(2\pi)^3\sqrt{2\omega(\vb{p})}}
		\biggl\{
			\phi(\omega(\vb{p}),\vb{p})
			e^{+i\omega(\vb{p})t}
			+
			\phi(-\omega(\vb{p}),\vb{p})
			e^{-i\omega(\vb{p})t}
		\biggr\}
		e^{-i\vb{p}\vdot\vb{x}}
		.
	\end{equation*}
	We now only need to perform the substitution $\vb{p}\to-\vb{p}$ on the second term
	\begin{equation*}
		\begin{split}
			\phi(t,\vb{x})
			&=
			\int_{\mathbb{R}^3}\frac{\dd[3]{p}}{(2\pi)^3\sqrt{2\omega(\vb{p})}}
			\biggl\{
				\phi(\omega(\vb{p}),\vb{p})
				e^{+i\omega(\vb{p})t}
				e^{-i\vb{p}\vdot\vb{x}}
				+
				\phi(-\omega(\vb{p}),\vb{p})
				e^{-i\omega(\vb{p})t}
				e^{-i\vb{p}\vdot\vb{x}}
			\biggr\}
			\\
			&=
			\int_{\mathbb{R}^3}\frac{\dd[3]{p}}{(2\pi)^3\sqrt{2\omega(\vb{p})}}
			\biggl\{
				\phi(\omega(\vb{p}),\vb{p})
				e^{+i\omega(\vb{p})t}
				e^{-i\vb{p}\vdot\vb{x}}
				+
				\phi(-\omega(\vb{p}),-\vb{p})
				e^{-i\omega(\vb{p})t}
				e^{+i\vb{p}\vdot\vb{x}}
			\biggr\}
			\\
			&=
			\int_{\mathbb{R}^3}\frac{\dd[3]{p}}{(2\pi)^3\sqrt{2\omega(\vb{p})}}
			\biggl\{
				\phi(\omega(\vb{p}),\vb{p})
				e^{+ip_\mu x^\mu}
				+
				\phi(\omega(\vb{p}),\vb{p})^*
				e^{-ip_\mu x^\mu}
			\biggr\}_{p_0=\omega(\vb{p})}
		\end{split}
	\end{equation*}
	and use the conjugate symmetry of the Fourier amplitudes
	\begin{equation*}
		\phi(p_0,\vb{p})^*
		=
		\phi(-p_0,-\vb{p})
	\end{equation*}
	which is present because $\phi(t,\vb{x})$ is real-valued.
\end{proof}

\Cref{thm:kg_energy_density}, \Cref{thm:kg_energy_momentum}, and \Cref{thm:kg_conjugate_momentum} are found in standard literature on relativistic field theory, for instance, in Ref.~\cite{Peskin1995}.

\section{Canonical quantization}

\qkgcommac
\begin{proof}
	The equal-time commutation relations are valid for any time $t$ so also for $t=0$, inserting \cref{eq:qkg_pos} into \cref{eq:qkg_comm_pm}, we find
	\begin{equation*}
		\begin{split}
			0
			=
			\comm{\hat\phi(0,\vb{x})}{\hat\phi(0,\vb{y})}
			&=
			\int\frac{\dd[3]{p}}{(2\pi)^3}
			\frac{1}{\sqrt{2\omega(\vb{p})}}
			\int\frac{\dd[3]{q}}{(2\pi)^3}
			\frac{1}{\sqrt{2\omega(\vb{q})}}
			\\
			&\times
			\comm{
				\hat{a}(\vb{p})
				e^{+i\vb{p}\vdot\vb{x}}
				+
				\hat{a}^\dagger(\vb{p})
				e^{-i\vb{p}\vdot\vb{x}}
			}{
				\hat{a}(\vb{q})
				e^{+i\vb{q}\vdot\vb{y}}
				+
				\hat{a}^\dagger(\vb{q})
				e^{-i\vb{q}\vdot\vb{y}}
			}
			,
		\end{split}
	\end{equation*}
		and for \cref{eq:qkg_mom}, we find
	\begin{equation*}
		\begin{split}
			0
			=
			\comm{\hat\phi(0,\vb{x})}{\hat\phi(0,\vb{y})}
			&=
			\int\frac{\dd[3]{p}}{(2\pi)^3}
			\left(
				-i
				\sqrt{\frac{\omega(\vb{p})}{2}}
			\right)
			\int\frac{\dd[3]{q}}{(2\pi)^3}
			\left(
				-i
				\sqrt{\frac{\omega(\vb{q})}{2}}
			\right)
			\\
			&\times
			\comm{
				\hat{a}(\vb{p})
				e^{+i\vb{p}\vdot\vb{x}}
				-
				\hat{a}^\dagger(\vb{p})
				e^{-i\vb{p}\vdot\vb{x}}
			}{
				\hat{a}(\vb{q})
				e^{+i\vb{q}\vdot\vb{y}}
				-
				\hat{a}^\dagger(\vb{q})
				e^{-i\vb{q}\vdot\vb{y}}
			}
			.
		\end{split}
	\end{equation*}
	The double integral is zero iff the integrand is zero, therefore
	\begin{align*}
		\comm{
			\hat{a}(\vb{p})
			e^{+i\vb{p}\vdot\vb{x}}
			+
			\hat{a}^\dagger(\vb{p})
			e^{-i\vb{p}\vdot\vb{x}}
		}{
			\hat{a}(\vb{q})
			e^{+i\vb{q}\vdot\vb{y}}
			+
			\hat{a}^\dagger(\vb{q})
			e^{-i\vb{q}\vdot\vb{y}}
		}
		&=
		0
		\\
		\comm{
			\hat{a}(\vb{p})
			e^{+i\vb{p}\vdot\vb{x}}
			-
			\hat{a}^\dagger(\vb{p})
			e^{-i\vb{p}\vdot\vb{x}}
		}{
			\hat{a}(\vb{q})
			e^{+i\vb{q}\vdot\vb{y}}
			-
			\hat{a}^\dagger(\vb{q})
			e^{-i\vb{q}\vdot\vb{y}}
		}
		&=
		0
		.
	\end{align*}
	Adding and subtracting these equations from another implies
	\begin{equation*}
		\comm{\hat{a}(\vb{p})}{\hat{a}(\vb{q})}
		=
		0
		=
		\comm{\hat{a}^\dagger(\vb{p})}{\hat{a}^\dagger(\vb{q})}
		.
	\end{equation*}
	Finally, inserting \cref{eq:qkg_pos} and \cref{eq:qkg_mom} into \cref{eq:qkg_comm_pm}, reveals
	\begin{equation*}
		\begin{split}
			i\delta^{(3)}(\vb{x}-\vb{y})
			=
			\comm{\hat\phi(0,\vb{x})}{\hat\pi(0,\vb{y})}
			&=
			\int\frac{\dd[3]{p}}{(2\pi)^3}
			\frac{1}{\sqrt{2\omega(\vb{p})}}
			\int\frac{\dd[3]{q}}{(2\pi)^3}
			\left(
				-i
				\sqrt{\frac{\omega(\vb{q})}{2}}
			\right)
			\\
			&\times
			\comm{
				\hat{a}(\vb{p})
				e^{+i\vb{p}\vdot\vb{x}}
				+
				\hat{a}^\dagger(\vb{p})
				e^{-i\vb{p}\vdot\vb{x}}
			}{
				\hat{a}(\vb{q})
				e^{+i\vb{q}\vdot\vb{y}}
				-
				\hat{a}^\dagger(\vb{q})
				e^{-i\vb{p}\vdot\vb{y}}
			}
			\\
			&=
			\int\frac{\dd[3]{p}}{(2\pi)^3}
			\int\frac{\dd[3]{q}}{(2\pi)^3}
			\left(
				-
				\frac{i}{2}
				\sqrt{\frac{\omega(\vb{q})}{\omega(\vb{p})}}
			\right)
			\\
			&\times
			\left\{
				-
				\comm{\hat{a}(\vb{p})}{\hat{a}^\dagger(\vb{q})}
				e^{+i\vb{p}\vdot\vb{x}}
				e^{-i\vb{q}\vdot\vb{y}}
				+
				\comm{\hat{a}^\dagger(\vb{p})}{\hat{a}(\vb{q})}
				e^{-i\vb{p}\vdot\vb{x}}
				e^{+i\vb{q}\vdot\vb{y}}
			\right\}
		\end{split}
	\end{equation*}
	which is satisfied for
	\begin{equation*}
		\comm{\hat{a}(\vb{p})}{\hat{a}^\dagger(\vb{q})}
		=
		(2\pi)^3
		\delta^{(3)}(\vb{q}-\vb{p})
		.
	\end{equation*}
\end{proof}

\qkgcommpn
\begin{proof}
	That positive respective negative frequency Klein-Gordon operators commute follows from \Cref{thm:qkg_comm_ac}.
	Using \Cref{thm:qkg_comm_ac} for the remaining commutator
	\begin{equation*}
		\begin{split}
			\comm{\hat\phi^+(x^\mu)}{\hat\phi^-(y^\mu)}
			&=
			\int\frac{\dd[3]{p}}{(2\pi)^3\sqrt{2\omega(\vb{p})}}
			\int\frac{\dd[3]{q}}{(2\pi)^3\sqrt{2\omega(\vb{q})}}
			\comm{\hat{a}(\vb{p})}{\hat{a}^\dagger(\vb{p})}
			\eval{
				e^{-ip_\mu x^\mu}
				e^{+iq_\mu y^\mu}
			}_{\substack{p_0=\omega(\vb{p})\\q_0=\omega(\vb{q})}}
			\\
			&=
			\int\frac{\dd[3]{p}}{(2\pi)^32\omega(\vb{p})}
			\eval{
				e^{-ip_\mu(x^\mu-y^\mu)}
			}_{p_0=\omega(\vb{p})}
			=
			D(x^\mu-y^\mu)
		\end{split}
	\end{equation*}
	recovers the initial claim after identification of the Klein-Gordon propagator.
\end{proof}
\qkgpropcorr
\begin{proof}
	Decomposing the Klein-Gordon operator into positive and negative frequency and using \Cref{thm:vacuum_state_pn}, we find
	\begin{equation*}
		\begin{split}
			\expval{\hat\phi(x^\mu)\hat\phi(y^\mu)}{0}
			&=
			\expval{\hat\phi^+(x^\mu)\hat\phi^-(y^\mu)}{0}
			\\
			&=
			\expval{\comm{\hat\phi^+(x^\mu)}{\hat\phi^-(y^\mu)}}{0}
			\\
			&=
			\expval{D(x^\mu-y^\mu)}{0}
			\\
			&=
			D(x^\mu-y^\mu)
		\end{split}
	\end{equation*}
	where we used \Cref{thm:qkg_comm_pn}.
\end{proof}

\section{Number states}

\qkgnumbersmearing
\begin{proof}
	Comparing \cref{eq:qkg_number_state_smearing} and \cref{eq:qkg_number_state} leaves us with showing
	\begin{equation*}
		\int\dd[4]{x}
		f(x^\mu)
		\hat\phi^-(x^\mu)
		=
		\int\frac{\dd[3]{p}}{(2\pi)^3\sqrt{2\omega(\vb{p})}}
		f(\vb{p})
		\hat{a}^\dagger(\vb{p})
		.
	\end{equation*}
	Inserting the negative frequency Klein-Gordon operator, \cref{eq:qkg_positive_negative_frequency}, into the left-hand side yields
	\begin{equation*}
		\begin{split}
			\int\dd[4]{x}
			f(x^\mu)
			\hat\phi^-(x^\mu)
			&=
			\int\dd[4]{x}
			f(x^\mu)
			\int\frac{\dd[3]{p}}{(2\pi)^3\sqrt{2\omega(\vb{p})}}
			\eval{
				e^{+ip_\mu x^\mu}
				\hat{a}^\dagger(\vb{p})
			}_{p_0=\omega(\vb{p})}
			\\
			&=
			\int\frac{\dd[3]{p}}{(2\pi)^3\sqrt{2\omega(\vb{p})}}
			\left(
				\int\dd[4]{x}
				f(x^\mu)
				e^{+ip_\mu x^\mu}
			\right)_{p_0=\omega(\vb{p})}
			\hat{a}^\dagger(\vb{p})
			\\
			&=
			\int\frac{\dd[3]{p}}{(2\pi)^3\sqrt{2\omega(\vb{p})}}
			\eval{f(p^\mu)}_{p_0=\omega(\vb{p})}
			\hat{a}^\dagger(\vb{p})
			\\
			&=
			\int\frac{\dd[3]{p}}{(2\pi)^3\sqrt{2\omega(\vb{p})}}
			f\left(\omega(\vb{p}),\vb{p}\right)
			\hat{a}^\dagger(\vb{p})
		\end{split}
	\end{equation*}
	which equals the right-hand side when we take $f(\vb{p})=f\left(\omega(\vb{p}),\vb{p}\right)$.
\end{proof}

\begin{lemma}\label{thm:qkg_comm_an}
	Let $n\in\mathbb{N}$, $\hat{a}(\vb{p})$ be the annihilation operator of the Klein-Gordon field, and $\hat\phi^-(x^\mu)$ be the negative frequency Klein-Gordon operator, then
	\begin{equation}
		\comm{\hat{a}(\vb{p})}{\hat\phi^-(x^\mu)^n}
		=
		n
		\frac{e^{+ip_\mu x^\mu}}{\sqrt{2\omega(\vb{p})}}
		\hat\phi^-(x^\mu)^{n-1}
		\label{eq:qkg_comm_an}
		.
	\end{equation}
\end{lemma}
\begin{proof}
	Induction start ($n=1$):
	\begin{equation*}
		\comm{\hat{a}(\vb{p})}{\hat\phi^-(x^\mu)}
		=
		\int\frac{\dd[3]{q}}{(2\pi)^3\sqrt{2\omega(\vb{q})}}
		e^{+iq_\mu x^\mu}
		\comm{\hat{a}(\vb{p})}{\hat{a}^\dagger(\vb{q})}
		=
		\frac{e^{+ip_\mu x^\mu}}{\sqrt{2\omega(\vb{p})}}
	\end{equation*}
	Induction start ($n\to n+1$):
	\begin{equation*}
		\begin{split}
			\comm{\hat{a}(\vb{p})}{\hat\phi^-(x^\mu)^{n+1}}
			&=
			\comm{\hat{a}(\vb{p})}{\hat\phi^-(x^\mu)\hat\phi^-(x^\mu)^n}
			\\
			&=
			\hat\phi^-(x^\mu)
			\comm{\hat{a}(\vb{p})}{\hat\phi^-(x^\mu)^n}
			+
			\comm{\hat{a}(\vb{p})}{\hat\phi^-(x^\mu)}
			\hat\phi^-(x^\mu)^n
			\\
			&=
			\hat\phi^-(x^\mu)
			n
			\frac{e^{+ip_\mu x^\mu}}{\sqrt{2\omega(\vb{p})}}
			\hat\phi^-(x^\mu)^{n-1}
			+
			\frac{e^{+ip_\mu x^\mu}}{\sqrt{2\omega(\vb{p})}}
			\hat\phi^-(x^\mu)^n
			\\
			&=
			(n+1)
			\frac{e^{+ip_\mu x^\mu}}{\sqrt{2\omega(\vb{p})}}
			\hat\phi^-(x^\mu)^n
		\end{split}
	\end{equation*}
\end{proof}
\begin{lemma}\label{thm:qkg_comm_cp}
	Let $n\in\mathbb{N}$, $\hat{a}^\dagger(\vb{p})$ be the creation operator of the Klein-Gordon field, and $\hat\phi^+(x^\mu)$ be the positive frequency Klein-Gordon operator, then
	\begin{equation}
		\comm{\hat{a}^\dagger(\vb{p})}{\hat\phi^+(x^\mu)^n}
		=
		-
		n
		\frac{e^{-ip_\mu x^\mu}}{\sqrt{2\omega(\vb{p})}}
		\hat\phi^+(x^\mu)^{n-1}
		\label{eq:qkg_comm_cp}
		.
	\end{equation}
\end{lemma}
\begin{proof}
	The hermitian conjugate of the left-hand side of \cref{eq:qkg_comm_an} is
	\begin{equation*}
		\comm{\hat{a}(\vb{p})}{\hat\phi^-(x^\mu)^n}^\dagger
		=
		-
		\comm{\hat{a}^\dagger(\vb{p})}{\hat\phi^+(x^\mu)^n}
	\end{equation*}
	and the hermitian conjugate of the right-hand side is
	\begin{equation*}
		\left(
			n
			\frac{e^{+ip_\mu x^\mu}}{\sqrt{2\omega(\vb{p})}}
			\hat\phi^-(x^\mu)^{n-1}
		\right)^\dagger
		=
		n
		\frac{e^{-ip_\mu x^\mu}}{\sqrt{2\omega(\vb{p})}}
		\hat\phi^+(x^\mu)^{n-1}
		.
	\end{equation*}
	Therefore, we have
	\begin{equation*}
		\comm{\hat{a}^\dagger(\vb{p})}{\hat\phi^+(x^\mu)^n}
		=
		-
		n
		\frac{e^{-ip_\mu x^\mu}}{\sqrt{2\omega(\vb{p})}}
		\hat\phi^+(x^\mu)^{n-1}
	\end{equation*}
	after equating both sides again.
\end{proof}
\begin{lemma}
	\begin{equation}
		\hat{a}(\vb{p})
		\ket{n_f}
		=
		\sqrt{n}
		\frac{e^{+ip_\mu x^\mu}}{\sqrt{2\omega(\vb{p})}}
		\ket{n-1_f}
	\end{equation}
\end{lemma}
\begin{proof}
	Using \Cref{thm:qkg_number_state_smearing}, \Cref{thm:qkg_comm_cp}, and \Cref{thm:vacuum_state_ac}, we find
	\begin{equation*}
		\begin{split}
			\hat{a}(\vb{p})
			\ket{n_f}
			=
			\frac{1}{\sqrt{n!}}
			\hat{a}(\vb{p})
			\hat\phi^-[f]^n
			\ket{0}
			=
			\frac{1}{\sqrt{n!}}
			n
			\frac{e^{+ip_\mu x^\mu}}{\sqrt{2\omega(\vb{p})}}
			\hat\phi^-[f]^{n-1}
			\ket{0}
			=
			\sqrt{n}
			\frac{e^{+ip_\mu x^\mu}}{\sqrt{2\omega(\vb{p})}}
			\ket{1-n_f}
			.
		\end{split}
	\end{equation*}
\end{proof}

\qkgnumbereigenstate
\begin{proof}
Foo	
\end{proof}

\section{Coherent states}
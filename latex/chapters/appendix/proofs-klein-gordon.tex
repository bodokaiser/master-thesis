\section{Number state}

\begin{proof}
	First, we note that the positive frequency Klein-Gordon operator vanishes,
	\begin{equation*}
		\int\dd[4]{x}
		f(x)
		\hat\phi(x)
		\ket{0}
		=
		\int\dd[4]{x}
		f(x)
		\hat\phi^-(x)
		\ket{0}
		,
	\end{equation*}
	because $\hat{a}(\vb{p})\ket{0}=0$, then we insert the definition of the negative frequency part \cref{eq:qkg_positive_negative_frequency}
	\begin{equation*}
		\begin{split}
			\int\dd[4]{x}
			f(x)
			\hat\phi^-(x)
			\ket{0}
			&=
			\int\dd[4]{x}
			f(x)
			\int\frac{\dd[3]{p}}{(2\pi)^3\sqrt{2\omega(\vb{p})}}
			e^{+ip_\mu x^\mu}
			\hat{a}^\dagger(\vb{p})
			\ket{0}
			\\
			&=
			\int\frac{\dd[3]{p}}{(2\pi)^3\sqrt{2\omega(\vb{p})}}
			\left(
				\int\dd[4]{x}
				f(x)
				e^{+ip_\mu x^\mu}
			\right)
			\hat{a}^\dagger(\vb{p})
			\ket{0}
			\\
			&=
			\int\frac{\dd[3]{p}}{(2\pi)^3\sqrt{2\omega(\vb{p})}}
			f(\vb{p})
			\hat{a}^\dagger(\vb{p})
			\ket{0}
			\\
			&=
			\ket{1_f}
		\end{split}
	\end{equation*}
	and recognize the single-particle number state.
\end{proof}
\begin{proof}
	Simply inserting the definition and using the commutation relations
	\begin{equation*}
		\begin{split}
			\bra{1_g}\ket{1_f}
			&=
			\int\frac{\dd[3]{p}}{(2\pi)^3\sqrt{2\omega(\vb{p})}}
			\int\frac{\dd[3]{q}}{(2\pi)^3\sqrt{2\omega(\vb{q})}}
			f(\vb{p})g(\vb{q})^*
			\expval{\hat{a}(\vb{q})\hat{a}^\dagger(\vb{p})}{0}
			\\
			&=
			\int\frac{\dd[3]{p}}{(2\pi)^3\sqrt{2\omega(\vb{p})}}
			\int\frac{\dd[3]{q}}{(2\pi)^3\sqrt{2\omega(\vb{q})}}
			f(\vb{p})g(\vb{q})^*
			(2\pi)^3\delta^{(3)}(\vb{q}-\vb{p})
			\\
			&=
			\int\frac{\dd[3]{p}}{(2\pi)^32\omega(\vb{p})}
			f(\vb{p})g(\vb{p})^*
		\end{split}
	\end{equation*}
	shows the equality.
\end{proof}
\begin{proof}
	The probabilistic interpretation of quantum mechanics requires the absolute square of the probability amplitude to equal one, i.e.,
	\begin{equation*}
		\abs{\braket{1_f}}^2
		=
		1
		.
	\end{equation*}
	From \Cref{th:single_particle_number_states_inner_product}, we know that
	\begin{equation*}
		\braket{1_f}
		=
		\int\frac{\dd[3]{p}}{(2\pi)^32\omega(\vb{p})}
		\abs{f(\vb{p})}^2
	\end{equation*}
	which is real-valued. Combining both insights, we must conclude
	\begin{equation*}
		\braket{1_f}
		=
		\int\frac{\dd[3]{p}}{(2\pi)^32\omega(\vb{p})}
		\abs{f(\vb{p})}^2
		=
		1
		.
	\end{equation*}
\end{proof}
\begin{proof}
	Inserting the definitions and carefully applying the commutation relations
	\begin{equation*}
		\begin{split}
			\hat{N}
			\ket{1_f}
			&=
			\int\frac{\dd[3]{p}}{(2\pi)^3}
			\hat{a}^\dagger(\vb{p})
			\hat{a}(\vb{p})
			\int\frac{\dd[3]{q}}{(2\pi)^3\sqrt{2\omega(\vb{q})}}
			f(\vb{q})
			\hat{a}^\dagger(\vb{q})
			\ket{0}
			\\
			&=
			\int\frac{\dd[3]{p}}{(2\pi)^3}
			\int\frac{\dd[3]{q}}{(2\pi)^3\sqrt{2\omega(\vb{q})}}
			f(\vb{q})
			\hat{a}^\dagger(\vb{p})
			\hat{a}(\vb{p})
			\hat{a}^\dagger(\vb{q})
			\ket{0}
			\\
			&=
			\int\frac{\dd[3]{p}}{(2\pi)^3}
			\int\frac{\dd[3]{q}}{(2\pi)^3\sqrt{2\omega(\vb{q})}}
			f(\vb{q})
			\hat{a}^\dagger(\vb{p})
			(2\pi)^3
			\delta^{(3)}(\vb{q}-\vb{p})
			\ket{0}
			\\
			&=
			\int\frac{\dd[3]{p}}{(2\pi)^3\sqrt{2\omega(\vb{p})}}
			f(\vb{p})
			\hat{a}^\dagger(\vb{p})
			\ket{0}
			\\
			&=
			1
			\ket{1_f}
		\end{split}
		,
	\end{equation*}
	we identify the single-particle state which has trivial eigenvalue $1$.
\end{proof}
\begin{proof}
	Using the auxiliary result
	\begin{equation*}
		\begin{split}
			\hat{H}
			\ket{1_f}
			&=
			\int\frac{\dd[3]{p}}{(2\pi)^3}
			\omega(\vb{p})
			\hat{a}^\dagger(\vb{p})
			\hat{a}(\vb{p})
			\int\frac{\dd[3]{q}}{(2\pi)^3\sqrt{2\omega(\vb{q})}}
			f(\vb{q})
			\hat{a}^\dagger(\vb{q})
			\ket{0}
			\\
			&=
			\int\frac{\dd[3]{q}}{(2\pi)^3\sqrt{2\omega(\vb{q})}}
			f(\vb{q})
			\int\frac{\dd[3]{p}}{(2\pi)^3}
			\omega(\vb{p})
			\hat{a}^\dagger(\vb{p})
			\hat{a}(\vb{p})
			\hat{a}^\dagger(\vb{q})
			\ket{0}
			\\
			&=
			\int\frac{\dd[3]{q}}{(2\pi)^3\sqrt{2\omega(\vb{q})}}
			f(\vb{q})
			\int\frac{\dd[3]{p}}{(2\pi)^3}
			\omega(\vb{p})
			\hat{a}^\dagger(\vb{p})
			(2\pi)^3
			\delta^{(3)}(\vb{p}-\vb{q})
			\ket{0}
			\\
			&=
			\int\frac{\dd[3]{q}}{(2\pi)^3\sqrt{2\omega(\vb{q})}}
			f(\vb{q})
			\omega(\vb{q})
			\hat{a}^\dagger(\vb{q})
			\ket{0}
		\end{split}
		,
	\end{equation*}
	the first moment turns out to be
	\begin{equation*}
		\begin{split}
			\expval{\hat{H}}{1_f}
			&=
			\int\frac{\dd[3]{p}}{(2\pi)^3\sqrt{2\omega(\vb{p})}}
			f(\vb{p})^*
			\bra{0}
			\hat{a}(\vb{p})
			\int\frac{\dd[3]{q}}{(2\pi)^3\sqrt{2\omega(\vb{q})}}
			f(\vb{q})
			\omega(\vb{q})
			\hat{a}^\dagger(\vb{q})
			\ket{0}
			\\
			&=
			\int\frac{\dd[3]{p}}{(2\pi)^3\sqrt{2\omega(\vb{p})}}
			\int\frac{\dd[3]{q}}{(2\pi)^3\sqrt{2\omega(\vb{q})}}
			\omega(\vb{q})
			f(\vb{p})^*
			f(\vb{q})
			\expval{
				\hat{a}(\vb{p})
				\hat{a}^\dagger(\vb{q})
			}{0}
			\\
			&=
			\int\frac{\dd[3]{p}}{(2\pi)^32\omega(\vb{p})}
			\omega(\vb{p})
			\abs{f(\vb{p})}^2
		\end{split}
	\end{equation*}
	in agreement with Ref.~\cite[eqs.~10 and 11]{Naumov2013}.
	The second moment turns out as
	\begin{equation*}
		\begin{split}
			\expval{\hat{H}^2}{1_f}
			&=
			\int\frac{\dd[3]{p}}{(2\pi)^3\sqrt{2\omega(\vb{p})}}
			f(\vb{p})^*
			\omega(\vb{p})
			\bra{0}
			\hat{a}(\vb{p})
			\int\frac{\dd[3]{q}}{(2\pi)^3\sqrt{2\omega(\vb{q})}}
			f(\vb{q})
			\omega(\vb{q})
			\hat{a}^\dagger(\vb{q})
			\ket{0}
			\\
			&=
			\int\frac{\dd[3]{p}}{(2\pi)^3\sqrt{2\omega(\vb{p})}}
			\int\frac{\dd[3]{q}}{(2\pi)^3\sqrt{2\omega(\vb{q})}}
			\omega(\vb{p})
			\omega(\vb{q})
			f(\vb{p})^*
			f(\vb{q})
			\expval{
				\hat{a}(\vb{p})
				\hat{a}^\dagger(\vb{q})
			}{0}
			\\
			&=
			\int\frac{\dd[3]{p}}{(2\pi)^32\omega(\vb{p})}
			\omega(\vb{p})^2
			\abs{f(\vb{p})}^2
		\end{split}
		.
	\end{equation*}
\end{proof}
\begin{proof}
	Inserting the definitions, we find
	\begin{equation*}
		\begin{split}
			\expval{\hat\pi(t,\vb{x})}{1_f}
			&=
			\int\frac{\dd[3]{p}}{(2\pi)^3\sqrt{2\omega(\vb{p})}}
			\left(
				-i
				2\omega(\vb{p})
			\right)
			\int\frac{\dd[3]{q_1}}{(2\pi)^3\sqrt{2\omega(\vb{q}_1)}}
			\int\frac{\dd[3]{q_2}}{(2\pi)^3\sqrt{2\omega(\vb{q}_2)}}
			\\
			&\times
			f(\vb{q}_1)^*
			f(\vb{q}_2)
			\expval{\hat{a}(\vb{q}_1)\hat{a}(\vb{p})\hat{a}^\dagger(\vb{q}_2)}{0}
			+
			\text{h.c.}			
		\end{split}
	\end{equation*}
	and
	\begin{equation*}
		\expval{\hat{a}(\vb{q}_1)\hat{a}(\vb{p})\hat{a}^\dagger(\vb{q}_2)}{0}
		=
		0
	\end{equation*}
	because we have an unequal number of annihilation and creation operators.
\end{proof}
\begin{proof}
	\begin{equation*}
		\begin{split}
			\expval{\hat\pi(x^\mu)\hat\pi(y^\mu)}{1_f}
			&=
			\int\frac{\dd[3]{p_1}}{(2\pi)^3\sqrt{2\omega(\vb{p}_1)}}
			\int\frac{\dd[3]{p_2}}{(2\pi)^3\sqrt{2\omega(\vb{p}_2)}}
			\left(-i2\omega(\vb{p}_1)\right)
			\left(-i2\omega(\vb{p}_2)\right)
			\\
			&\times
			\expval{
				\left[
					\hat{a}(\vb{p}_1)
					e^{-ip_1^\mu x_\mu}
					-
					\hat{a}^\dagger(\vb{p}_1)
					e^{+ip_1^\mu x_\mu}
				\right]
				\left[
					\hat{a}(\vb{p}_2)
					e^{-ip_2^\mu y_\mu}
					-
					\hat{a}^\dagger(\vb{p}_2)
					e^{+ip_2^\mu y_\mu}
				\right]
			}{1_f}
			\\
			&=
			\int\frac{\dd[3]{p_1}}{(2\pi)^3\sqrt{2\omega(\vb{p}_1)}}
			\int\frac{\dd[3]{p_2}}{(2\pi)^3\sqrt{2\omega(\vb{p}_2)}}
			2\omega(\vb{p}_1)
			2\omega(\vb{p}_2)
			\\
			&\times
			\expval{
				\hat{a}(\vb{p}_1)
				\hat{a}^\dagger(\vb{p}_2)
				e^{-ip_1^\mu x_\mu}
				e^{+ip_2^\mu y_\mu}
				+
				\hat{a}^\dagger(\vb{p}_1)
				\hat{a}(\vb{p}_2)
				e^{+ip_1^\mu x_\mu}
				e^{-ip_2^\mu y_\mu}
			}{1_f}
		\end{split}
	\end{equation*}
	The first term reduces to
	\begin{equation*}
		\expval{
			\hat{a}(\vb{p}_1)
			\hat{a}^\dagger(\vb{p}_2)
		}{1_f}
		=
		(2\pi)^3\delta^{(3)}(\vb{p}_2-\vb{p}_1)
		+
		\expval{
			\hat{a}^\dagger(\vb{p}_2)
			\hat{a}(\vb{p}_1)
		}{1_f}
	\end{equation*}
	where we used $\braket{1_f}=1$ and the commutation relation
	\begin{equation*}
		\begin{split}
			\expval{
				\hat{a}^\dagger(\vb{p}_2)
				\hat{a}(\vb{p}_1)
			}{1_f}
			&=
			\int\frac{\dd[3]{q_1}}{(2\pi)^3\sqrt{2\omega(\vb{q}_1)}}
			\int\frac{\dd[3]{q_2}}{(2\pi)^3\sqrt{2\omega(\vb{q}_2)}}
			f(\vb{q}_1)^*
			f(\vb{q}_2)
			\expval{
				\hat{a}(\vb{q}_1)
				\hat{a}^\dagger(\vb{p}_2)
				\hat{a}(\vb{p}_1)
				\hat{a}^\dagger(\vb{q}_2)
			}{0}
			\\
			&=
			\int\frac{\dd[3]{q_1}}{(2\pi)^3\sqrt{2\omega(\vb{q}_1)}}
			\int\frac{\dd[3]{q_2}}{(2\pi)^3\sqrt{2\omega(\vb{q}_2)}}
			f(\vb{q}_1)^*
			f(\vb{q}_2)
			(2\pi)^3\delta^{(3)}(\vb{q}_1-\vb{p}_2)
			(2\pi)^3\delta^{(3)}(\vb{q}_2-\vb{p}_1)
			\\
			&=
			\frac{f(\vb{p}_2)^*}{\sqrt{2\omega(\vb{p}_2)}}
			\frac{f(\vb{p}_1)}{\sqrt{2\omega(\vb{p}_1)}}
		\end{split}
	\end{equation*}
	and together, we get
	\begin{equation*}
		\begin{split}
			&
			\int\frac{\dd[3]{p_1}}{(2\pi)^3\sqrt{2\omega(\vb{p}_1)}}
			\int\frac{\dd[3]{p_2}}{(2\pi)^3\sqrt{2\omega(\vb{p}_2)}}
			2\omega(\vb{p}_1)
			2\omega(\vb{p}_2)
			\biggl\{
				(2\pi)^3\delta^{(3)}(\vb{p}_2-\vb{p}_1)
				e^{-ip_1^\mu x_\mu}
				e^{+ip_2^\mu y_\mu}
				\\
				&+
				\frac{f(\vb{p}_1)}{\sqrt{2\omega(\vb{p}_1)}}
				\frac{f(\vb{p}_2)^*}{\sqrt{2\omega(\vb{p}_2)}}
				e^{-ip_1^\mu x_\mu}
				e^{+ip_2^\mu y_\mu}
				+
				\frac{f(\vb{p}_1)^*}{\sqrt{2\omega(\vb{p}_1)}}
				\frac{f(\vb{p}_2)}{\sqrt{2\omega(\vb{p}_2)}}
				e^{+ip_1^\mu x_\mu}
				e^{-ip_2^\mu y_\mu}
			\biggr\}
			\\
			=&
			\int\frac{\dd[3]{p}}{(2\pi)^32\omega(\vb{p})}
			\left(2\omega(\vb{p})\right)^2
			e^{-ip_\mu(x^\mu-y^\mu)}
			+
			\int\frac{\dd[3]{p_1}}{(2\pi)^32\omega(\vb{p}_1)}
			f(\vb{p}_1)
			e^{-ip_1^\mu x_\mu}
			\int\frac{\dd[3]{p_2}}{(2\pi)^32\omega(\vb{p}_2)}
			f(\vb{p}_2)^*
			e^{+ip_2^\mu y_\mu}
			\\
			+&
			\int\frac{\dd[3]{p_1}}{(2\pi)^32\omega(\vb{p}_1)}
			f(\vb{p}_1)^*
			e^{+ip_1^\mu x_\mu}
			\int\frac{\dd[3]{p_2}}{(2\pi)^32\omega(\vb{p}_2)}
			f(\vb{p}_2)
			e^{-ip_2^\mu y_\mu}
			\\
			=&
			\expval{\left(\Delta\hat\pi(x^\mu,y^\mu)\right)^2}{0}
			+
			2\Re\left\{
				\int\frac{\dd[3]{p_1}}{(2\pi)^32\omega(\vb{p}_1)}
				f(\vb{p}_1)
				e^{-ip_1^\mu x_\mu}
				\int\frac{\dd[3]{p_2}}{(2\pi)^32\omega(\vb{p}_2)}
				f(\vb{p}_2)^*
				e^{+ip_2^\mu y_\mu}
			\right\}
		\end{split}
	\end{equation*}
\end{proof}

\begin{proof}
	We insert the definition of the smeared Klein-Gordon operator
	\begin{equation*}
		\begin{split}
			\comm{\hat\phi^+[f]}{\hat\phi^-[g]}
			&=
			\iint\dd[4]{x}\dd[4]{y}
			f(x^\mu)
			\comm{\hat\phi^+(x^\mu)}{\hat\phi^-(y^\mu)}
			g(y^\mu)
			\\
			&=
			\iint\dd[4]{x}\dd[4]{y}
			f(x^\mu)
			D(x^\mu-y^\mu)
			g(y^\mu)
			\\
			&=
			\iint\dd[4]{x}\dd[4]{y}
			f(x^\mu)
			\left(
				\int\frac{\dd[3]{p}}{(2\pi)^32\omega(\vb{p})}
				e^{-ip_\mu (x^\mu-y^\mu)}
			\right)
			g(y^\mu)
			\\
			&=
			\int\frac{\dd[3]{p}}{(2\pi)^32\omega(\vb{p})}
			\left(
				\int\dd[4]{x}
				f(x^\mu)
				e^{+ip_\mu x^\mu}
			\right)^*
			\left(
				\int\dd[4]{y}
				g(y^\mu)
				e^{+ip_\mu y^\mu}
			\right)
			\\
			&=
			\int\frac{\dd[3]{p}}{(2\pi)^32\omega(\vb{p})}
			f(\vb{p})^*
			g(\vb{p})
			\\
			&=
			\braket{1_f}{1_g}
		\end{split}
	\end{equation*}
	where we used \Cref{thm:qkg_full_comm_pn_comm} to replace the commutator with the propagator, inserted the definition of the propagator, and used the definition of the Fourier transform.
\end{proof}

\begin{lemma}\label{th:normal_ordered_a1_cn}
	Let $\hat{a}(\vb{p}),\hat{a}^\dagger(\vb{p})$ be the annihilation and creation operator of the Klein-Gordon field satisfying the canonical commutation relations then for $n\in\mathbb{N}$
	\begin{equation}
		\hat{a}(\vb{p})
		\prod_{j=1}^n
		\hat{a}^\dagger(\vb{q}_j)
		\ket{0}
		=
		\sum_{i=1}^n
		(2\pi)^3
		\delta^{(3)}(\vb{q}_i-\vb{p})
		\prod_{\substack{j=1\\j\neq i}}^n
		\hat{a}^\dagger(\vb{q}_j)
		\ket{0}
		\label{eq:normal_ordered_a1_cn}
		.
	\end{equation}
\end{lemma}
\begin{proof}
	Induction start $n=1$:
	\begin{equation*}
		\hat{a}(\vb{p})
		\prod_{j=1}^1
		\hat{a}^\dagger(\vb{q}_j)
		\ket{0}
		=
		\hat{a}(\vb{p})
		\hat{a}^\dagger(\vb{q}_1)
		\ket{0}
		=
		(2\pi)^3
		\delta^{(3)}(\vb{q}_1-\vb{p})
		\ket{0}
	\end{equation*}
	where we used the commutation relation and $\hat{a}(\vb{p})\ket{0}=0$.

	Induction step $n\to n+1$:
	\begin{equation*}
		\begin{split}
			\hat{a}(\vb{p})
			\prod_{j=1}^{n+1}
			\hat{a}^\dagger(\vb{q}_j)
			\ket{0}
			&=
			\hat{a}(\vb{p})
			\left(
				\prod_{j=1}^n
				\hat{a}^\dagger(\vb{q}_j)
			\right)
			\hat{a}^\dagger(\vb{q}_{n+1})
			\ket{0}
			\\
			&=
			\hat{a}(\vb{p})
			\hat{a}^\dagger(\vb{q}_{n+1})
			\left(
				\prod_{j=1}^n
				\hat{a}^\dagger(\vb{q}_j)
			\right)
			\ket{0}
			\\
			&=
			(2\pi)^3
			\delta^{(3)}(\vb{q}_{n+1}-\vb{p})
			\prod_{j=1}^n
			\hat{a}^\dagger(\vb{q}_j)
			\ket{0}
			+
			\hat{a}^\dagger(\vb{q}_{n+1})
			\hat{a}(\vb{p})
			\prod_{j=1}^n
			\hat{a}^\dagger(\vb{q}_j)
			\ket{0}
			\\
			&=
			(2\pi)^3
			\delta^{(3)}(\vb{q}_{n+1}-\vb{p})
			\prod_{j=1}^n
			\hat{a}^\dagger(\vb{q}_j)
			\ket{0}
			+
			\sum_{i=1}^n
			(2\pi)^3
			\delta^{(3)}(\vb{q}_i-\vb{p})
			\prod_{\substack{j=1\\j\neq i}}^{n+1}
			\hat{a}^\dagger(\vb{q}_j)
			\ket{0}
			\\
			&=
			\sum_{i=1}^{n+1}
			(2\pi)^3
			\delta^{(3)}(\vb{q}_i-\vb{p})
			\prod_{\substack{j=1\\j\neq i}}^{n+1}
			\hat{a}^\dagger(\vb{q}_j)
			\ket{0}
		\end{split}
	\end{equation*}
	where we used that creation operators commute in the first line, the canonical commutation relation in the second line, and the induction hypothesis in the third line.
\end{proof}
\begin{lemma}\label{thm:anti_normal_expvalue}
	Let $n,m\in\mathbb{N}$ then
	\begin{equation}
		\expval{
			\hat{a}(\vb{p}_1)
			\dots
			\hat{a}(\vb{p}_n)
			\hat{a}^\dagger(\vb{q}_1)
			\dots
			\hat{a}^\dagger(\vb{q}_m)
		}{0}
		=
		\delta_{nm}
		\sum_{\pi\in\textrm{perm}}
		\prod^n_{i=1}
		(2\pi)^3
		\delta^{(3)}(\vb{p}_i-\vb{q}_{\pi(i)})
	\end{equation}
	where the sum is over all pairwise permutations of $(i,j)\in\left\{1,\dots,n\right\}^2$.
\end{lemma}
\begin{proof}
	Assuming $s=n-m>0$, then
	\begin{equation*}
		\expval{
			\hat{a}(\vb{p}_1)
			\dots
			\hat{a}(\vb{p}_n)
			\hat{a}^\dagger(\vb{q}_1)
			\dots
			\hat{a}^\dagger(\vb{q}_m)
		}{0}
		\propto
		\expval{
			\hat{a}(\vb{p}_{i_1})
			\dots
			\hat{a}(\vb{p}_{i_s})
		}{0}
		=
		0
	\end{equation*}
	because of $\hat{a}(\vb{p})\ket{0}=0$. The case $s<0$ follows directly from the hermitian conjugate.
	We conclude that the expectation value is only non-zero iff $n=m$.
	
	Induction start $n=1$:
	\begin{equation*}
		\expval{
			\hat{a}(\vb{p}_1)
			\hat{a}^\dagger(\vb{q}_1)
			\dots
			\hat{a}^\dagger(\vb{q}_m)
		}{0}
		=
		\delta_{m1}
		\expval{
			\hat{a}(\vb{p}_1)
			\hat{a}^\dagger(\vb{q}_1)
		}{0}
		=
		\delta_{m1}
		(2\pi)^3
		\delta^{(3)}(\vb{q}_1-\vb{p}_1)
	\end{equation*}
	
	Induction step $n\to n+1$:
	\begin{equation*}
		\begin{split}
			&\
			\expval{
				\hat{a}(\vb{p}_1)
				\dots
				\hat{a}(\vb{p}_{n+1})
				\hat{a}^\dagger(\vb{q}_1)
				\dots
				\hat{a}^\dagger(\vb{q}_{m+1})
			}{0}
			\\
			=&\
			\delta_{nm}
			\expval{
				\hat{a}(\vb{p}_1)
				\dots
				\hat{a}(\vb{p}_{n+1})
				\hat{a}^\dagger(\vb{q}_{n+1})
				\hat{a}^\dagger(\vb{q}_1)
				\dots
				\hat{a}^\dagger(\vb{q}_n)
			}{0}
			\\
			=&\
			\delta_{nm}
			(2\pi)^3
			\delta^{(3)}(\vb{q}_{n+1}-\vb{p}_{n+1})
			\expval{
				\hat{a}(\vb{p}_1)
				\dots
				\hat{a}(\vb{p}_n)
				\hat{a}^\dagger(\vb{q}_1)
				\dots
				\hat{a}^\dagger(\vb{q}_n)
			}{0}
			\\
			+&\
			\delta_{nm}
			\expval{
				\hat{a}(\vb{p}_1)
				\dots
				\hat{a}(\vb{p}_n)
				\hat{a}^\dagger(\vb{q}_{n+1})
				\hat{a}(\vb{p}_{n+1})
				\hat{a}^\dagger(\vb{q}_1)
				\dots
				\hat{a}^\dagger(\vb{q}_n)
			}{0}
		\end{split}
	\end{equation*}
	the second term can be further simplified to
	\begin{equation*}
		\begin{split}
			&\
			\expval{
				\hat{a}(\vb{p}_1)
				\dots
				\hat{a}(\vb{p}_n)
				\hat{a}^\dagger(\vb{q}_{n+1})
				\hat{a}(\vb{p}_{n+1})
				\hat{a}^\dagger(\vb{q}_1)
				\dots
				\hat{a}^\dagger(\vb{q}_n)
			}{0}
			\\
			=&\
			\left(
				\hat{a}(\vb{q}_{n+1})
				\hat{a}^\dagger(\vb{p}_1)
				\dots
				\hat{a}^\dagger(\vb{p}_n)
				\ket{0}
			\right)^\dagger
			\left(
				\hat{a}(\vb{p}_{n+1})
				\hat{a}^\dagger(\vb{q}_1)
				\dots
				\hat{a}^\dagger(\vb{q}_n)
				\ket{0}
			\right)
			\\
			=&\
			\left(
				\sum^n_{i=1}
				(2\pi)^3
				\delta^{(3)}(\vb{p}_i-\vb{q}_{n+1})
				\prod_{\substack{j=1\\j\neq i}}^n
				\hat{a}^\dagger(\vb{p}_j)
				\ket{0}
			\right)^\dagger
			\left(
				\sum^n_{l=1}
				(2\pi)^3
				\delta^{(3)}(\vb{q}_l-\vb{p}_{n+1})
				\prod_{\substack{k=1\\k\neq l}}^n
				\hat{a}^\dagger(\vb{p}_k)
				\ket{0}
			\right)
			\\
			=&\
			\sum^n_{i,l=1}
			\delta^{(3)}(\vb{p}_i-\vb{q}_{n+1})
			\delta^{(3)}(\vb{q}_l-\vb{p}_{n+1})
			\expval{
				\left(
					\prod_{\substack{j=1\\j\neq i}}^n
					\hat{a}(\vb{p}_j)
				\right)
				\left(
					\prod_{\substack{k=1\\k\neq l}}^n
					\hat{a}^\dagger(\vb{p}_k)
				\right)
			}{0}
		\end{split}
	\end{equation*}
	applying the induction hypothesis and inserting the second term back, we complete the induction step as all permutations are accounted for.
\end{proof}
\begin{proof}
	Using \Cref{thm:anti_normal_expvalue} and noting that the sum over the permutations compensates for the factorials, we find
	\begin{equation*}
		\begin{split}
			\braket{n_f}{m_g}
			&=
			\expval{
				\frac{1}{\sqrt{n!}}
				\left(
					\int\frac{\dd[3]{p}}{(2\pi)^3\sqrt{2\omega(\vb{p})}}
					f(\vb{p})^*
					\hat{a}(\vb{p})
				\right)^n
				\frac{1}{\sqrt{m!}}
				\left(
					\int\frac{\dd[3]{q}}{(2\pi)^3\sqrt{2\omega(\vb{q})}}
					g(\vb{q})
					\hat{a}^\dagger(\vb{q})
				\right)^m
			}{0}
			\\
			&=
			\int\frac{\dd[3]{p_1}}{(2\pi)^3\sqrt{2\omega(\vb{p}_1)}}
			\dots
			\int\frac{\dd[3]{p_n}}{(2\pi)^3\sqrt{2\omega(\vb{p}_n)}}
			\int\frac{\dd[3]{q_1}}{(2\pi)^3\sqrt{2\omega(\vb{q}_1)}}
			\dots
			\int\frac{\dd[3]{q_m}}{(2\pi)^3\sqrt{2\omega(\vb{q}_m)}}
			\\
			&\times
			\frac{1}{\sqrt{n!}}
			f(\vb{p}_1)^*
			\dots
			f(\vb{p}_n)^*
			\frac{1}{\sqrt{m!}}
			g(\vb{q}_1)
			\dots
			g(\vb{q}_m)
			\expval{
				\hat{a}(\vb{p}_1)
				\dots
				\hat{a}(\vb{p}_n)
				\hat{a}^\dagger(\vb{q}_1)
				\dots
				\hat{a}^\dagger(\vb{q}_m)
			}{0}
			\\
			&=
			\delta_{nm}
			\int\frac{\dd[3]{p_1}}{(2\pi)^32\omega(\vb{p}_1)}
			f(\vb{p}_1)^*
			g(\vb{p}_1)
			\dots
			\int\frac{\dd[3]{p_n}}{(2\pi)^32\omega(\vb{p}_n)}
			f(\vb{p}_n)^*
			g(\vb{p}_n)
			\\
			&=
			\left(
				\int\frac{\dd[3]{p}}{(2\pi)^32\omega(\vb{p})}
				f(\vb{p})^*
				g(\vb{p})
			\right)^n
		\end{split}
	\end{equation*}
\end{proof}
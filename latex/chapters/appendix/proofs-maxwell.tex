\section{Relativistic field theory}

\begin{delayedproof}{thm:mw_local_gauge_invariance}
	The physical field-strength tensor transforms under \cref{eq:mw_local_gauge_transform} as
	\begin{equation*}
		\begin{split}
			F_{\mu\nu}
			\to
			F_{\mu\nu}^\prime
			&=
			\partial_\mu\left(A_\nu+\partial_\nu\Lambda\right)
			-
			\partial_\nu\left(A_\mu+\partial_\mu\Lambda\right)
			\\
			&=
			F_{\mu\nu}
			+
			\partial_\mu\partial_\nu\Lambda
			-
			\partial_\nu\partial_\mu\Lambda
			=
			F_{\mu\nu}
		\end{split}
	\end{equation*}	
\end{delayedproof}
\begin{lemma}
	The field-strength tensor is antisymmetric $F^{\mu\nu}=-F^{\nu\mu}$.
\end{lemma}
\begin{proof}
	\begin{equation*}
		F^{\mu\nu}
		=
		\partial^\mu A^\nu
		-
		\partial^\nu A^\mu
		=
		-
		\left(
			\partial^\nu A^\mu
			-
			\partial^\mu A^\nu
		\right)
		=
		-
		F^{\nu\mu}
	\end{equation*}
\end{proof}

\subsection{Maxwell equations}

\begin{delayedproof}{thm:tensor_maxwell_equations}
	The inhomogeneous Maxwell equations are the equations of motion which are found from the Euler-Lagrange equation
	\begin{equation*}
		0
		=
		\partial_\mu
		\pdv{\mathcal{L}}{(\partial_\mu A_\nu)}
		-
		\pdv{\mathcal{L}}{A_\nu}
		=
		-
		\partial_\mu
		F^{\mu\nu}
		+
		J^\nu
	\end{equation*}
	where we used
	\begin{equation*}
		\begin{split}
			\pdv{\mathcal{L}}{(\partial_\mu A_\nu)}
			&=
			\pdv{\mathcal{L}}{F_{\alpha\beta}}
			\pdv{F_{\alpha\beta}}{(\partial_\mu A_\nu)}
			\\
			&=
			-
			\frac{1}{4}
			\pdv{(F_{\sigma\rho}F^{\sigma\rho})}{F_{\alpha\beta}}
			\pdv{(\partial_\alpha A_\beta-\partial_\beta A_\alpha)}{(\partial_\mu A_\nu)}
			\\
			&=
			-
			\frac{1}{2}
			F^{\alpha\beta}
			\left(
				\delta_\alpha^\mu
				\delta_\beta^\nu
				-
				\delta_\alpha^\nu
				\delta_\beta^\mu
			\right)
			\\
			&=
			-
			\frac{1}{2}
			F^{\mu\nu}
			+
			\frac{1}{2}
			F^{\nu\mu}
			=
			-
			F^{\mu\nu}			
		\end{split}
	\end{equation*}
	The homogeneous Maxwell equations are a consequence of the Bianchi identity and antisymmetry of $F^{\mu\nu}$.
\end{delayedproof}
\begin{delayedproof}{thm:vector_maxwell_equations}
	Evaluating the time component of \cref{eq:mw_homo_vec} yields the Gauss' law for magnetism
	\begin{equation*}
		\begin{split}
			0
			=
			\varepsilon_{0\lambda\mu\nu}\partial^\lambda F^{\mu\nu}
			&=
			\varepsilon_{0ijk}\partial^iF^{jk}
			\\
			&=
			-
			\varepsilon_{ijk}\varepsilon_{ljk}
			\partial^i B_l
			=
			2\partial_iB^i
		\end{split}
		\label{eq:mw_gauss_law_mag}
	\end{equation*}
	and the spatial component yields Ampere's circuit law
	\begin{equation*}
		\begin{split}
			0
			=
			\varepsilon_{i\lambda\mu\nu}
			\partial^\lambda
			F^{\mu\nu}
			&=
			-
			\varepsilon_{ijk}
			\varepsilon^{ljk}
			\partial_t B_l
			-
			2\varepsilon_{ijk}
			\partial^jE^k
			\\
			&=
			\partial_tB_i
			+
			\varepsilon_{ijk}
			\partial^jE_k
		\end{split}
		\label{eq:mw_ampere_law}.
	\end{equation*}
	The time component of the inhomogeneous covariant Maxwell equation \cref{eq:mw_inhomo} yields Gauss' law
	\begin{equation*}
		J^0
		=
		\rho
		=
		\partial_\mu F^{\mu\nu}
		=
		\partial_i E^i
		\label{eq:mw_gauss_law},
	\end{equation*}
	and the spatial component yields Faraday's law of induction
	\begin{equation*}
		J^i
		=
		\partial_\mu F^{\mu i}
		=
		-\partial_t E^i
		+\varepsilon^{ijk}\partial_j B_k
		\label{eq:mw_faraday_law}.
	\end{equation*}
	and we derived the vector Maxwell equations from first principles.
\end{delayedproof}
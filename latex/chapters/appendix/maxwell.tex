\section{Relativistic field theory}

\mwfieldstrengthlagrangian
\begin{proof}
	Inserting the definition of the field-strength tensor \cref{eq:mw_field_strength_tensors} into the free term of the Lagrangian \cref{eq:mw_field_strength_lagrangian}
	\begin{equation*}
		\begin{split}
			-
			\frac{1}{4}
			F_{\mu\nu}
			F^{\mu\nu}
			&=
			-
			\frac{1}{4}
			\left(
				\partial_\mu A_\nu
				-
				\partial_\nu A_\mu
			\right)
			\left(
				\partial^\mu A^\nu
				-
				\partial^\nu A^\mu
			\right)
			\\
			&=
			-
			\frac{1}{4}
			\left[
				(\partial_\mu A_\nu)
				(\partial^\mu A^\nu)
				-
				(\partial_\mu A_\nu)
				(\partial^\nu A^\mu)
				-
				(\partial_\nu A_\mu)
				(\partial^\mu A^\nu)
				+
				(\partial_\nu A_\mu)
				(\partial^\nu A^\mu)
			\right]
			\\
			&=
			-
			\frac{1}{4}
			\left[
				(\partial_\mu A_\nu)
				(\partial^\mu A^\nu)
				-
				(\partial_\mu A_\nu)
				(\partial^\nu A^\mu)
				-
				(\partial_\mu A_\nu)
				(\partial^\nu A^\mu)
				+
				(\partial_\mu A_\nu)
				(\partial^\mu A^\nu)
			\right]
			\\
			&=
			-
			\frac{1}{2}
			\left[
				(\partial_\mu A_\nu)
				(\partial^\mu A^\nu)
				-
				(\partial_\mu A_\nu)
				(\partial^\nu A^\mu)
			\right]
			\\
			&=
			\frac{1}{2}
			(\partial_\mu A_\nu)
			\left(
				\partial_\mu A_\nu
				-
				\partial^\mu A^\nu
			\right)
		\end{split}
	\end{equation*}
	where we relabeled the indices in the two last terms at the second equal.
\end{proof}
\mweom
\begin{proof}
	Using the Lagrangian in terms of the field-strength tensor, we first calculate
	\begin{equation*}
		\pdv{\mathcal{L}}{(\partial_\mu A_\nu)}
		=
		-
		\frac{1}{2}
		F^{\alpha\beta}
		\pdv{F^{\alpha\beta}}{(\partial_\mu A_\nu)}
		=
		-
		\frac{1}{2}
		F^{\alpha\beta}
		\left(
			\delta^\alpha_\mu
			\delta^\beta_\nu
			-
			\delta^\alpha_\nu
			\delta^\beta_\mu
		\right)
		=
		-
		\frac{1}{2}
		F^{\mu\nu}
		+
		\frac{1}{2}
		F^{\nu\mu}
		=
		-
		F^{\mu\nu}
	\end{equation*}
	and use the result for the Euler-Lagrange equations
	\begin{equation*}
		0
		=
		\partial_\mu
		\pdv{\mathcal{L}}{(\partial_\mu A_\nu)}
		-
		\pdv{\mathcal{L}}{A_\nu}
		=
		-
		\partial_\mu
		F^{\mu\nu}
		+
		J^\nu
	\end{equation*}
	which produces \cref{eq:mw_eom}.
\end{proof}

\mwfieldstrengthcontracted
\begin{proof}
	For the first contraction, we find
	\begin{equation*}
		\begin{split}
			F_{\mu\nu}
			F^{\mu\nu}
			&=
			F_{0\nu}
			F^{0\nu}
			+
			F_{i\nu}
			F^{i\nu}
			=
			F_{0i}
			F^{0i}
			+
			F_{i0}
			F^{i0}
			+
			F_{ij}
			F^{ij}
			\\
			&=
			2
			F_{0i}
			F^{0i}
			+
			F_{ij}
			F^{ij}
			=
			-2
			E_i
			E^i
			+
			\varepsilon^{ijk}B_k
			\varepsilon_{ijl}B^l
			\\
			&=
			-
			2
			E_i
			E^i
			+
			2
			B_k
			B^k
			=
			-
			2
			\left(
				\vb{E}^2
				-
				\vb{B}^2
			\right)
		\end{split}
	\end{equation*}
	and for the second contraction
	\begin{equation*}
		\begin{split}
			\tilde{F}_{\mu\nu}
			F^{\mu\nu}
			&=
			\tilde{F}_{0\nu}
			F^{0\nu}
			+
			\tilde{F}_{i\nu}
			F^{i\nu}
			=
			\tilde{F}_{0i}
			F^{0i}
			+
			\tilde{F}_{i0}
			F^{i0}
			+
			\tilde{F}_{ij}
			F^{ij}
			\\
			&=
			2
			\tilde{F}_{0i}
			F^{0i}
			+
			\tilde{F}_{ij}
			F^{ij}
			=
			-
			2B_iE^i
			-
			\varepsilon_{ijk}E^k
			\varepsilon^{ijkl}B_l
			\\
			&=
			-
			2B_iE^i
			-
			2
			E^k
			\delta_k^l
			B_l
			=
			-
			4
			\vb{E}\vdot\vb{B}
			.
		\end{split}
	\end{equation*}
\end{proof}

\mweqtensor
\begin{proof}
	The inhomogeneous Maxwell equations are the equations of motion which are found from the Euler-Lagrange equation
	\begin{equation*}
		0
		=
		\partial_\mu
		\pdv{\mathcal{L}}{(\partial_\mu A_\nu)}
		-
		\pdv{\mathcal{L}}{A_\nu}
		=
		-
		\partial_\mu
		F^{\mu\nu}
		+
		J^\nu
	\end{equation*}
	where we used
	\begin{equation*}
		\begin{split}
			\pdv{\mathcal{L}}{(\partial_\mu A_\nu)}
			&=
			\pdv{\mathcal{L}}{F_{\alpha\beta}}
			\pdv{F_{\alpha\beta}}{(\partial_\mu A_\nu)}
			\\
			&=
			-
			\frac{1}{4}
			\pdv{(F_{\sigma\rho}F^{\sigma\rho})}{F_{\alpha\beta}}
			\pdv{(\partial_\alpha A_\beta-\partial_\beta A_\alpha)}{(\partial_\mu A_\nu)}
			\\
			&=
			-
			\frac{1}{2}
			F^{\alpha\beta}
			\left(
				\delta_\alpha^\mu
				\delta_\beta^\nu
				-
				\delta_\alpha^\nu
				\delta_\beta^\mu
			\right)
			\\
			&=
			-
			\frac{1}{2}
			F^{\mu\nu}
			+
			\frac{1}{2}
			F^{\nu\mu}
			=
			-
			F^{\mu\nu}			
		\end{split}
	\end{equation*}
	The homogeneous Maxwell equations are a consequence of the Bianchi identity and antisymmetry of $F^{\mu\nu}$.
\end{proof}
\mweqvector
\begin{proof}
	Evaluating the time component of \cref{eq:tensor_maxwell} yields the Gauss' law for magnetism
	\begin{equation*}
		0
		=
		\varepsilon_{0\lambda\mu\nu}\partial^\lambda F^{\mu\nu}
		=
		\varepsilon_{0ijk}\partial^iF^{jk}
		=
		-
		\varepsilon_{ijk}\varepsilon_{ljk}
		\partial^i B_l
		=
		2\partial_iB^i
	\end{equation*}
	and the spatial component yields Ampere's circuit law
	\begin{equation*}
		0
		=
		\varepsilon_{i\lambda\mu\nu}
		\partial^\lambda
		F^{\mu\nu}
		=
		-
		\varepsilon_{ijk}
		\varepsilon^{ljk}
		\partial_t B_l
		-
		2\varepsilon_{ijk}
		\partial^jE^k
		=
		\partial_tB_i
		+
		\varepsilon_{ijk}
		\partial^jE_k
		.
	\end{equation*}
	The time component of the inhomogeneous covariant Maxwell equation \cref{eq:mw_eq} yields Gauss' law
	\begin{equation*}
		J^0
		=
		\rho
		=
		\partial_\mu F^{\mu\nu}
		=
		\partial_i E^i
		,
	\end{equation*}
	and the spatial component yields Faraday's law of induction
	\begin{equation*}
		J^i
		=
		\partial_\mu F^{\mu i}
		=
		-
		\partial_t E^i
		+
		\varepsilon^{ijk}
		\partial_j
		B_k
		.
	\end{equation*}
	and we derived the vector Maxwell equations from first principles.
\end{proof}

\mwlogalgaugeinvariance
\begin{proof}
	The physical field-strength tensor transforms under \cref{eq:mw_local_gauge_transform} as
	\begin{equation*}
		\begin{split}
			F_{\mu\nu}
			\to
			F_{\mu\nu}^\prime
			&=
			\partial_\mu\left(A_\nu+\partial_\nu\Lambda\right)
			-
			\partial_\nu\left(A_\mu+\partial_\mu\Lambda\right)
			\\
			&=
			F_{\mu\nu}
			+
			\partial_\mu\partial_\nu\Lambda
			-
			\partial_\nu\partial_\mu\Lambda
			=
			F_{\mu\nu}
		\end{split}
	\end{equation*}	
\end{proof}

\mwcoulombfixing
\begin{proof}
	The time component of the Maxwell field transforms under the suggested local gauge transformation as
	\begin{equation*}
		A^0
		\to
		A^0
		+
		\partial^0\Lambda
		=
		A^0
		+
		\partial_t\Lambda
		=
		0
	\end{equation*}
	and the spatial components transform as
	\begin{equation*}
		A^i
		\to
		A^i
		+
		\partial^i\Lambda
		.
	\end{equation*}
	Taking the spatial derivative of the former, we see that the Coulomb gauge is satisfied
	\begin{equation*}
		\partial_i A^i
		\to
		\partial_i A^i
		+
		\partial_i \partial^i\Lambda
		=
		\partial_i A^i
		-
		\partial_i A^i
		=
		0
		.
	\end{equation*}
	It is important to note that it is incorrect to deduce
	\begin{equation*}
		\partial_i
		\Lambda
		=
		-
		\partial_i A^i
	\end{equation*}
	as then $A^i=0$.
\end{proof}
\mwcoulombtransverse
\begin{proof}
	We first show that the transverse Maxwell field $\vb{A}_\perp$ satisfies the Coulomb gauge
	\begin{equation*}
		\div\vb{A}
		=
		\partial_i
		\left(
			\delta^{ij}
			-
			\frac{\partial^i\partial^j}{\laplacian}
		\right)
		A_j
		=
		\partial_iA^i
		-
		\partial_jA^j
		=
		0
		.
	\end{equation*}
\end{proof}
\mwcoulombeom
\begin{proof}
	Starting from \cref{eq:mw_eom}, we expand the field-strength tensor
	\begin{equation*}
		\begin{split}
			J^\nu
			=
			\partial_\mu
			F^{\mu\nu}
			&=
			\partial_\mu
			\left(
				\partial^\mu
				A^\nu
				-
				\partial^\nu
				A^\mu
			\right)
			\\
			&=
			\partial_\mu
			\partial^\mu
			A^\nu
			-
			\partial^\nu
			\partial_\mu
			A^\mu
			\\
			&=
			\partial_\mu
			\partial^\mu
			A^\nu
			-
			\partial^\nu
			\left(
				\partial_t A^0
				+
				\partial_i A^i
			\right)
		\end{split}
	\end{equation*}
	and apply the Coulomb and temporal gauge conditions while assuming dynamical sources
	\begin{equation*}
		\vb{J}
		=
		\partial_\mu
		\partial^\mu
		\vb{A}
		.
	\end{equation*}
	Because the Maxwell field in the Coulomb and temporal gauge is transverse, the previous equation reduces further to
	\begin{equation*}
		\vb{J}_\perp
		=
		\partial_\mu
		\partial^\mu
		\vb{A}_\perp
		.		
	\end{equation*}
\end{proof}

\mwcoulombmodeexpansion
\begin{proof}
	Inserting \cref{eq:mw_coulomb_mode_expansion} into the free equations of motion \cref{eq:mw_coulomb_eom}, we find
	\begin{equation*}
		0
		=
		p_\mu p^\mu
		=
		p_0^2
		-
		\vb{p}^2
	\end{equation*}
	which is satisfied if $p_0=\omega(\vb{p})$.
	$\vb{p}\vdot\boldsymbol{\varepsilon}_\lambda(\vb{p})=0$ follows from the Coulomb gauge $\div\vb{A}=0$.
	Orthogonality of the polarization vectors $\boldsymbol{\varepsilon}_1(\vb{p})\perp\boldsymbol{\varepsilon}_2(\vb{p})$ is required to have a complete solution and choosing the polarization vectors to be orthonormal is a convenient choice.
\end{proof}
\mwcoulombmodeexpansionemfield
\begin{proof}
	The electric field components follow directly from inserting the mode expansion of the field into the definition
	\begin{equation*}
		\begin{split}
			E^i
			&=
			F^{0i}
			=
			\partial_t
			A^i
			\\
			&=
			\sum_{\lambda=1,2}
			\int_{\mathbb{R}^3}\frac{\dd[3]{p}}{(2\pi)^3\sqrt{2\omega(\vb{p})}}
			\partial_t
			\left\{
				a_\lambda(\vb{p})
				\boldsymbol{\epsilon}_\lambda(\vb{p})^i
				e^{-ip_\mu x^\mu}
				+
				\text{c.c.}
			\right\}_{p_0=\omega(\vb{p})}
			\\
			&=
			\sum_{\lambda=1,2}
			\int_{\mathbb{R}^3}\frac{\dd[3]{p}}{(2\pi)^3\sqrt{2\omega(\vb{p})}}
			\left\{
				a_\lambda(\vb{p})
				\boldsymbol{\epsilon}_\lambda(\vb{p})^i
				\partial_t
				e^{-ip_0t}
				e^{+i\vb{p}\vdot\vb{x}}
				+
				\text{c.c.}
			\right\}_{p_0=\omega(\vb{p})}
			\\
			&=
			\sum_{\lambda=1,2}
			\int_{\mathbb{R}^3}\frac{\dd[3]{p}}{(2\pi)^3\sqrt{2\omega(\vb{p})}}
			\left\{
				a_\lambda(\vb{p})
				\boldsymbol{\epsilon}_\lambda(\vb{p})^i
				(-ip_0)
				e^{-ip_0t}
				e^{+i\vb{p}\vdot\vb{x}}
				+
				\text{c.c.}
			\right\}_{p_0=\omega(\vb{p})}
			\\
			&=
			\sum_{\lambda=1,2}
			\int_{\mathbb{R}^3}\frac{\dd[3]{p}}{(2\pi)^3\sqrt{2\omega(\vb{p})}}
			\left(-i\omega(\vb{p})\right)
			\left\{
				a_\lambda(\vb{p})
				\boldsymbol{\epsilon}_\lambda(\vb{p})^i
				e^{-ip_\mu x^\mu}
				-
				\text{c.c.}
			\right\}_{p_0=\omega(\vb{p})}
			.
		\end{split}
	\end{equation*}
	For the magnetic field components, we first need to express the magnetic field components in terms of the field-strength tensor
	\begin{equation*}
		\varepsilon_{ijk}
		F^{jk}
		=
		\varepsilon_{ijk}
		\varepsilon^{jkl}
		B_l
		=
		\varepsilon_{jki}
		\varepsilon^{jkl}
		B_l
		=
		2
		\delta_i^l
		B_l
		=
		2B_i
	\end{equation*}
	and insert the mode expansion of the field
	\begin{equation*}
		\begin{split}
			B_i
			&=
			\frac{1}{2}
			\varepsilon_{ijk}
			F^{jk}
			=
			\frac{1}{2}
			\varepsilon_{ijk}
			\left(
				\partial^j
				A^k
				-
				\partial^k
				A^j
			\right)
			\\
			&=
			\frac{1}{2}
			\varepsilon_{ijk}
			\partial^j
			A^k
			+
			\frac{1}{2}
			\varepsilon_{jik}
			\partial^k
			A^j
			=
			\varepsilon_{ijk}
			\partial^j
			A^k
			\\
			&=
			\sum_{\lambda=1,2}
			\int_{\mathbb{R}^3}\frac{\dd[3]{p}}{(2\pi)^3\sqrt{2\omega(\vb{p})}}
			\varepsilon_{ijk}
			\partial^j
			\left\{
				a_\lambda(\vb{p})
				\boldsymbol{\epsilon}_\lambda(\vb{p})^i
				e^{-ip_\mu x^\mu}
				+
				\text{c.c.}
			\right\}_{p_0=\omega(\vb{p})}
			\\
			&=
			\sum_{\lambda=1,2}
			\int_{\mathbb{R}^3}\frac{\dd[3]{p}}{(2\pi)^3\sqrt{2\omega(\vb{p})}}
			\varepsilon_{ijk}
			\left\{
				a_\lambda(\vb{p})
				\boldsymbol{\epsilon}_\lambda(\vb{p})^i
				\partial^j
				e^{-ip_0t}
				e^{+i\vb{p}\vdot\vb{x}}
				+
				\text{c.c.}
			\right\}_{p_0=\omega(\vb{p})}
			\\
			&=
			\sum_{\lambda=1,2}
			\int_{\mathbb{R}^3}\frac{\dd[3]{p}}{(2\pi)^3\sqrt{2\omega(\vb{p})}}
			\varepsilon_{ijk}
			\left\{
				a_\lambda(\vb{p})
				\boldsymbol{\epsilon}_\lambda(\vb{p})^i
				(+ip^j)
				e^{-ip_0t}
				e^{+i\vb{p}\vdot\vb{x}}
				+
				\text{c.c.}
			\right\}_{p_0=\omega(\vb{p})}
			\\
			&=
			\sum_{\lambda=1,2}
			\int_{\mathbb{R}^3}\frac{\dd[3]{p}}{(2\pi)^3\sqrt{2\omega(\vb{p})}}
			\left(
				i
				\varepsilon_{ijk}
				p^j
			\right)
			\left\{
				a_\lambda(\vb{p})
				\boldsymbol{\epsilon}_\lambda(\vb{p})^i
				e^{-ip_\mu x^\mu}
				-
				\text{c.c.}
			\right\}_{p_0=\omega(\vb{p})}
			\\
			&=
			\sum_{\lambda=1,2}
			\int_{\mathbb{R}^3}\frac{\dd[3]{p}}{(2\pi)^3\sqrt{2\omega(\vb{p})}}
			\left(
				-i
				\varepsilon_{ijk}
				p^i
			\right)
			\left\{
				a_\lambda(\vb{p})
				\boldsymbol{\epsilon}_\lambda(\vb{p})^j
				e^{-ip_\mu x^\mu}
				-
				\text{c.c.}
			\right\}_{p_0=\omega(\vb{p})}
			.
		\end{split}
	\end{equation*}
	Raising the index of the magnetic field with the metric yields
	\begin{equation*}
		B^i
		=
		\eta^{ij}
		B_j
		=
		-
		\delta^{ij}
		B_j
		=
		-
		B_i
	\end{equation*}
	and in vector notation we recover \cref{eq:mw_coulomb_mode_expansion_em_field} which up to a sign is in agreement with Ref.~\cite[p.~198]{Greiner2013}.
\end{proof}

\mwcoulombcanonicalmomentum
\begin{proof}
	In \Cref{thm:mw_coulomb_eom} we found
	\begin{equation*}
		\pdv{\mathcal{L}}{(\partial_\mu A_\nu)}
		=
		-
		F^{\mu\nu}
	\end{equation*}
	thus
	\begin{equation*}
		\partial_t
		\pdv{\mathcal{L}}{(\partial_tA_i)}
		=
		-
		\partial_t
		F^{0i}
		=
		-
		\partial_t
		E^i
		.
	\end{equation*}
\end{proof}
\mwcoulombenergy
\begin{proof}
	Integration of the Hamiltonian density given in \Cref{thm:mw_hamiltonian} yields the total energy in terms of the electric and magnetic field amplitudes
	\begin{equation*}
		H
		=
		\int\dd[3]{x}
		\mathcal{H}
		=
		\frac{1}{2}
		\int\dd[3]{x}
		\left\{
			\vb{E}(x^\mu)^2
			+
			\vb{B}(x^\mu)^2
		\right\}
		.
	\end{equation*}
	Inserting the mode decomposition of \Cref{thm:mw_coulomb_mode_expansion_em_field} for the magnetic field, we evaluate
	\begin{equation*}
		\begin{split}
			\int\dd[3]{x}
			\vb{E}(x^\mu)^2
			&=
			\sum_{\lambda,\lambda^\prime=1,2}
			\int\frac{\dd[3]{p}}{(2\pi)^3\sqrt{2\omega(\vb{p})}}
			\int\frac{\dd[3]{q}}{(2\pi)^3\sqrt{2\omega(\vb{q})}}
			\left(
				-
				\omega(\vb{p})
				\omega(\vb{q})
			\right)
			\\
			&\times
			\biggl\{
				a_\lambda(\vb{p})
				a_{\lambda^\prime}(\vb{q})
				\boldsymbol{\varepsilon}_\lambda(\vb{p})
				\vdot
				\boldsymbol{\varepsilon}_{\lambda^\prime}(\vb{q})
				\int\dd[3]{x}
				e^{-i(p_\mu+q_\mu)x^\mu}
				\\
				& \ \ \
				-
				a_\lambda(\vb{p})
				a_{\lambda^\prime}(\vb{q})^*
				\boldsymbol{\varepsilon}_\lambda(\vb{p})
				\vdot
				\boldsymbol{\varepsilon}_{\lambda^\prime}(\vb{q})^*
				\int\dd[3]{x}
				e^{-i(p_\mu-q_\mu)x^\mu}
				+
				\text{c.c.}
			\biggr\}_{\substack{p_0=\omega(\vb{p})\\q_0=\omega(\vb{q})}}
		\end{split}
	\end{equation*}
	where we find
	\begin{equation*}
		\begin{split}
			\int\dd[3]{x}
			\eval{e^{-i(p_\mu\mp q_\mu)x^\mu}}_{\substack{p_0=\omega(\vb{p})\\q_0=\omega(\vb{q})}}
			&=
			e^{-i\omega(\vb{p})t}
			e^{\pm i\omega(\vb{q})t}
			\int\dd[3]{x}
			e^{+i(\vb{p}\mp\vb{q})\vdot\vb{x}}
			\\
			&=
			e^{-i\omega(\vb{p})t}
			e^{\pm i\omega(\vb{q})t}
			(2\pi)^3\delta^{(3)}(\vb{q}\mp\vb{p})
		\end{split}
	\end{equation*}
	which simplifies the first term in the curly brackets to
	\begin{equation*}
		\begin{split}
			&\
			a_\lambda(\vb{p})
			a_{\lambda^\prime}(\vb{q})
			\boldsymbol{\varepsilon}_\lambda(\vb{p})
			\vdot
			\boldsymbol{\varepsilon}_{\lambda^\prime}(\vb{q})
			e^{-i\omega(\vb{p})t}
			e^{-i\omega(\vb{q})t}
			(2\pi)^3\delta^{(3)}(\vb{q}+\vb{p})
			\\
			=&\
			a_\lambda(\vb{p})
			a_{\lambda^\prime}(-\vb{p})
			\boldsymbol{\varepsilon}_\lambda(\vb{p})
			\vdot
			\boldsymbol{\varepsilon}_{\lambda^\prime}(-\vb{p})
			e^{-i\omega(\vb{p})t}
			e^{-i\omega(-\vb{p})t}
			(2\pi)^3\delta^{(3)}(\vb{q}+\vb{p})
			\\
			=&\
			a_\lambda(\vb{p})
			a_{\lambda^\prime}(\vb{p})^*
			\boldsymbol{\varepsilon}_\lambda(\vb{p})
			\vdot
			\boldsymbol{\varepsilon}_{\lambda^\prime}(\vb{p})^*
			e^{-2i\omega(\vb{p})t}
			(2\pi)^3\delta^{(3)}(\vb{q}+\vb{p})
			\\
			=&\
			a_\lambda(\vb{p})
			a_{\lambda^\prime}(\vb{p})^*
			\delta_{\lambda\lambda^\prime}
			e^{-2i\omega(\vb{p})t}
			(2\pi)^3\delta^{(3)}(\vb{q}+\vb{p})
			\\
			=&\
			\abs{a_\lambda(\vb{p})}^2
			\delta_{\lambda\lambda^\prime}
			e^{-2i\omega(\vb{p})t}
			(2\pi)^3\delta^{(3)}(\vb{q}+\vb{p})
		\end{split}
	\end{equation*}
	and the second term to
	\begin{equation*}
		\begin{split}
			&\
			a_\lambda(\vb{p})
			a_{\lambda^\prime}(\vb{q})^*
			\boldsymbol{\varepsilon}_\lambda(\vb{p})
			\vdot
			\boldsymbol{\varepsilon}_{\lambda^\prime}(\vb{q})^*
			e^{-i\omega(\vb{p})t}
			e^{+i\omega(\vb{q})t}
			(2\pi)^3\delta^{(3)}(\vb{q}-\vb{p})
			\\
			=&\
			a_\lambda(\vb{p})
			a_{\lambda^\prime}(\vb{p})^*
			\boldsymbol{\varepsilon}_\lambda(\vb{p})
			\vdot
			\boldsymbol{\varepsilon}_{\lambda^\prime}(\vb{p})^*
			e^{-i\omega(\vb{p})t}
			e^{+i\omega(\vb{p})t}
			(2\pi)^3\delta^{(3)}(\vb{q}-\vb{p})
			\\
			=&\
			a_\lambda(\vb{p})
			a_{\lambda^\prime}(\vb{p})^*
			\delta_{\lambda\lambda^\prime}
			(2\pi)^3\delta^{(3)}(\vb{q}-\vb{p})
			\\
			=&\
			\abs{a_\lambda(\vb{p})}^2
			\delta_{\lambda\lambda^\prime}
			(2\pi)^3\delta^{(3)}(\vb{q}-\vb{p})
			.
		\end{split}
	\end{equation*}
	Inserting both terms back, we find
	\begin{equation*}
		\begin{split}
			\int\dd[3]{x}
			\vb{E}(x^\mu)^2
			&=
			\sum_{\lambda,\lambda^\prime=1,2}
			\int\frac{\dd[3]{p}}{(2\pi)^3\sqrt{2\omega(\vb{p})}}
			\int\frac{\dd[3]{q}}{(2\pi)^3\sqrt{2\omega(\vb{q})}}
			\left(
				-
				\omega(\vb{p})
				\omega(\vb{q})
			\right)
			\\
			&\times
			\left\{
				\abs{a_\lambda(\vb{p})}^2
				\delta_{\lambda\lambda^\prime}
				e^{-2i\omega(\vb{p})t}
				(2\pi)^3\delta^{(3)}(\vb{q}+\vb{p})
				-
				\abs{a_\lambda(\vb{p})}^2
				\delta_{\lambda\lambda^\prime}
				(2\pi)^3\delta^{(3)}(\vb{q}-\vb{p})
				+
				\text{c.c.}
			\right\}
			\\
			&=
			\sum_{\lambda=1,2}
			\int\frac{\dd[3]{p}}{(2\pi)^32\omega(\vb{p})}
			\left(
				-
				\omega(\vb{p})^2
			\right)
			\abs{a_\lambda(\vb{p})}^2
			\left\{
				e^{-2i\omega(\vb{p})t}
				-
				1
				+
				e^{+2i\omega(\vb{p})t}
				-
				1
			\right\}
			\\
			&=
			2
			\sum_{\lambda=1,2}
			\int\frac{\dd[3]{p}}{(2\pi)^32\omega(\vb{p})}
			\omega(\vb{p})^2
			\abs{a_\lambda(\vb{p})}^2
			\Re\left\{
				1
				-
				e^{-2i\omega(\vb{p})t}
			\right\}
			.
		\end{split}
	\end{equation*}
	For the magnetic field, we need to account for an additional sign flip of the momentum when evaluating the delta distribution, otherwise the steps are analog and we find
	\begin{equation*}
		\begin{split}
			\int\dd[3]{x}
			\vb{B}(x^\mu)^2
			&=
			2
			\sum_{\lambda,\lambda^\prime=1,2}
			\int\frac{\dd[3]{p}}{(2\pi)^3\sqrt{2\omega(\vb{p})}}
			\int\frac{\dd[3]{q}}{(2\pi)^3\sqrt{2\omega(\vb{q})}}
			(-1)
			\int\dd[3]{x}
			\\
			&\times
			\left\{
				a_\lambda(\vb{p})
				\left(
					\vb{p}
					\cp
					\boldsymbol{\varepsilon}_\lambda(\vb{p})
				\right)
				e^{-ip_\mu x^\mu}
				-
				\text{c.c.}
			\right\}_{p_0=\omega(\vb{p})}
			\\
			&\vdot
			\left\{
				a_{\lambda^\prime}(\vb{q})
				\left(
					\vb{q}
					\cp
					\boldsymbol{\varepsilon}_{\lambda^\prime}(\vb{q})
				\right)
				e^{-iq_\mu x^\mu}
				-
				\text{c.c.}
			\right\}_{q_0=\omega(\vb{q})}
		\end{split}
	\end{equation*}
	and the inner product evaluates to
	\begin{equation*}
		\begin{split}
			&\
			a_\lambda(\vb{p})
			a_{\lambda^\prime}(\vb{q})
			\left(
				\vb{p}
				\cp
				\boldsymbol{\varepsilon}_\lambda(\vb{p})
			\right)
			\vdot
			\left(
				\vb{q}
				\cp
				\boldsymbol{\varepsilon}_{\lambda^\prime}(\vb{q})
			\right)
			\int\dd[3]{x}
			\eval{e^{-i(p_\mu+q_\mu)x^\mu}}_{\substack{p_0=\omega(\vb{p})\\q_0=\omega(\vb{q})}}
			\\
			-&\
			a_\lambda(\vb{p})
			a_{\lambda^\prime}(\vb{q})^*
			\left(
				\vb{p}
				\cp
				\boldsymbol{\varepsilon}_\lambda(\vb{p})
			\right)
			\vdot
			\left(
				\vb{q}
				\cp
				\boldsymbol{\varepsilon}_{\lambda^\prime}(\vb{q})^*
			\right)
			\int\dd[3]{x}
			\eval{e^{-i(p_\mu-q_\mu)x^\mu}}_{\substack{p_0=\omega(\vb{p})\\q_0=\omega(\vb{q})}}
			+
			\text{c.c.}
			\\
			=&\
			a_\lambda(\vb{p})
			a_{\lambda^\prime}(\vb{q})
			\left(
				\vb{p}
				\cp
				\boldsymbol{\varepsilon}_\lambda(\vb{p})
			\right)
			\vdot
			\left(
				\vb{q}
				\cp
				\boldsymbol{\varepsilon}_{\lambda^\prime}(\vb{q})
			\right)
			e^{-i\omega(\vb{p})t}
			e^{-i\omega(\vb{q})t}
			(2\pi)^3\delta^{(3)}(\vb{q}+\vb{p})
			\\
			-&\
			a_\lambda(\vb{p})
			a_{\lambda^\prime}(\vb{q})^*
			\left(
				\vb{p}
				\cp
				\boldsymbol{\varepsilon}_\lambda(\vb{p})
			\right)
			\vdot
			\left(
				\vb{q}
				\cp
				\boldsymbol{\varepsilon}_{\lambda^\prime}(\vb{q})^*
			\right)
			e^{-i\omega(\vb{p})}
			e^{+i\omega(\vb{q})}
			(2\pi)^3\delta^{(3)}(\vb{q}-\vb{p})
			+
			\text{c.c.}
			\\
			=&\
			a_\lambda(\vb{p})
			a_{\lambda^\prime}(\vb{p})^*
			\left(
				\vb{p}
				\cp
				\boldsymbol{\varepsilon}_\lambda(\vb{p})
			\right)
			\vdot
			\left(
				-
				\vb{p}
				\cp
				\boldsymbol{\varepsilon}_{\lambda^\prime}(\vb{p})^*
			\right)
			e^{-2i\omega(\vb{p})t}
			(2\pi)^3\delta^{(3)}(\vb{q}+\vb{p})
			\\
			-&\
			a_\lambda(\vb{p})
			a_{\lambda^\prime}(\vb{q})^*
			\left(
				\vb{p}
				\cp
				\boldsymbol{\varepsilon}_\lambda(\vb{p})
			\right)
			\vdot
			\left(
				\vb{p}
				\cp
				\boldsymbol{\varepsilon}_{\lambda^\prime}(\vb{p})^*
			\right)
			(2\pi)^3\delta^{(3)}(\vb{q}-\vb{p})
			+
			\text{c.c.}
		\end{split}
	\end{equation*}
	and noting that
	\begin{equation*}
		\left(
			\vb{p}
			\cp
			\boldsymbol{\varepsilon}_\lambda(\vb{p})
		\right)
		\vdot
		\left(
			\vb{p}
			\cp
			\boldsymbol{\varepsilon}_{\lambda^\prime}(\vb{p})^*
		\right)
		=
		\left(
			\vb{p}^2
		\right)
		\left(
			\boldsymbol{\varepsilon}_\lambda(\vb{p})
			\vdot
			\boldsymbol{\varepsilon}_{\lambda^\prime}(\vb{p})^*
		\right)
		-
		\left(
			\vb{p}
			\vdot
			\boldsymbol{\varepsilon}_\lambda(\vb{p})
		\right)
		\left(
			\vb{p}
			\vdot
			\boldsymbol{\varepsilon}_{\lambda^\prime}(\vb{p})^*
		\right)
		=
		\vb{p}^2
		\delta_{\lambda\lambda^\prime}
	\end{equation*}
	we find
	\begin{equation*}
		\begin{split}
			\int\dd[3]{x}
			\vb{B}(x^\mu)^2
			&=
			\sum_{\lambda,\lambda^\prime=1,2}
			\int\frac{\dd[3]{p}}{(2\pi)^3\sqrt{2\omega(\vb{p})}}
			\int\frac{\dd[3]{q}}{(2\pi)^3\sqrt{2\omega(\vb{q})}}
			(-1)
			\\
			&\times
			\left\{
				-
				\abs{a_\lambda(\vb{p})}^2
				\vb{p}^2
				\delta_{\lambda\lambda^\prime}
				e^{-2i\omega(\vb{p})t}
				(2\pi)^3\delta^{(3)}(\vb{q}+\vb{p})
				-
				\abs{a_\lambda(\vb{p})}^2
				\vb{p}^2
				\delta_{\lambda\lambda^\prime}
				(2\pi)^3\delta^{(3)}(\vb{q}-\vb{p})
				+
				\text{c.c.}
			\right\}
			\\
			&=
			\sum_{\lambda=1,2}
			\int\frac{\dd[3]{p}}{(2\pi)^32\omega(\vb{p})}
			\abs{a_\lambda(\vb{p})}^2
			\vb{p}^2
			\left\{
				e^{-2i\omega(\vb{p})t}
				+
				1
				+
				\text{c.c.}
			\right\}
			\\
			&=
			2
			\sum_{\lambda=1,2}
			\int\frac{\dd[3]{p}}{(2\pi)^32\omega(\vb{p})}
			\abs{a_\lambda(\vb{p})}^2
			\vb{p}^2
			\Re\left\{
				1
				+
				e^{-2i\omega(\vb{p})t}
			\right\}
		\end{split}
		.
	\end{equation*}
	Adding the integral over the squared field amplitudes together and dividing by two gives the total energy
	\begin{equation*}
		\begin{split}
			H
			&=
			\frac{1}{2}
			\int\dd[3]{x}
			\vb{E}(x^\mu)^2
			+
			\frac{1}{2}
			\int\dd[3]{x}
			\vb{B}(x^\mu)^2
			\\
			&=
			2\sum_{\lambda=1,2}
			\int\frac{\dd[3]{p}}{(2\pi)^32\omega(\vb{p})}
			\omega(\vb{p})^2
			\abs{a_\lambda(\vb{p})}^2
			\\
			&=
			\int\frac{\dd[3]{p}}{(2\pi)^3}
			\omega(\vb{p})
			\abs{a_\lambda(\vb{p})}^2
		\end{split}
	\end{equation*}
	where we used $\vb{p}^2=\omega(\vb{p})^2$.
\end{proof}

\section{Canonical quantization}

%\qmwtransversepnsmeared
\begin{proof}
	Inserting the definition of the positive and negative transverse Maxwell field and the Fourier transform, we find
	\begin{equation*}
		\begin{split}
			\int\dd[4]{x}\
			\vb{f}(x^\mu)
			\vdot
			\hat{\vb{A}}_\perp^+(x^\mu)
			&=
			\int\dd[4]{x}\
			\vb{f}(x^\mu)
			\vdot
			\sum_{\lambda=1,2}
			\int\frac{\dd[3]{p}}{(2\pi)^3\sqrt{2\omega(\vb{p})}}
			\hat{a}_\lambda^\dagger(\vb{p})
			\boldsymbol{\varepsilon}_\lambda(\vb{p})^*
			\eval{e^{+ip_\mu x^\mu}}_{p_0=\omega(\vb{p})}
			\\
			&=
			\sum_{\lambda=1,2}
			\int\frac{\dd[3]{p}}{(2\pi)^3\sqrt{2\omega(\vb{p})}}
			\hat{a}_\lambda^\dagger(\vb{p})
			\boldsymbol{\varepsilon}_\lambda(\vb{p})^*
			\vdot
			\left(
				\int\dd[4]{x}\
				\vb{f}(x^\mu)^*
				e^{-ip_\mu x^\mu}
			\right)_{p_0=\omega(\vb{p})}^*
			\\
			&=
			\sum_{\lambda=1,2}
			\int\frac{\dd[3]{p}}{(2\pi)^3\sqrt{2\omega(\vb{p})}}
			\hat{a}_\lambda^\dagger(\vb{p})
			\left[
				\vb{f}\left(\omega(\vb{p}),\vb{p}\right)
				\vdot
				\boldsymbol{\varepsilon}_\lambda(\vb{p})
			\right]^*
			.
		\end{split}
	\end{equation*}
	Taking the Hermitian conjugate, we find the smeared negative frequency transverse Maxwell operator.
\end{proof}

\section{Connection to Klein-Gordon states}

\qmwqkgnumberstate
\begin{proof}
	The smeared positive Klein-Gordon operators $\hat\phi^+_1,\hat\phi^+_2$ commute and we can use the binomial theorem to write
	\begin{equation*}
		\begin{split}
			\ket{n_{\vb{f}}}
			&=
			\frac{1}{\sqrt{n!}}
			\hat{\vb{A}}_\perp^+[\vb{f}]^n
			\ket{0}
			\\
			&=
			\frac{1}{\sqrt{n!}}
			\left(
				\hat\phi_1^+[f_1]
				+
				\hat\phi_2^+[f_2]
			\right)^n
			\ket{0}
			\\
			&=
			\frac{1}{\sqrt{n!}}
			\sum^n_{m=0}
			\binom{n}{m}
			\hat\phi_1^+[f_1]^m
			\hat\phi_2^+[f_2]^{n-m}
			\ket{0}
			\\
			&=
			\sum^n_{m=0}
			\binom{n}{m}^{1/2}
			\ket{m_{f_1},n-m_{f_2}}
			.
		\end{split}
	\end{equation*}
\end{proof}
\qmwqkgnumberstateinnerproduct
\begin{proof}
	Inserting the transverse Maxwell number state in terms of two independent Klein-Gordon number states \Cref{thm:qmw_qkg_number_state}
	\begin{equation*}
		\begin{split}
			\braket{n_{\vb{f}}}{m_{\vb{g}}}
			&=
			\sum^n_{k=0}
			\sum^m_{l=0}
			\binom{n}{k}^{1/2}
			\binom{m}{l}^{1/2}
			\braket{k_{f_1},n-k_{f_2}}{l_{g_1},m-l_{g_2}}
			\\
			&=
			\sum^n_{k=0}
			\sum^m_{l=0}
			\binom{n}{k}^{1/2}
			\binom{m}{l}^{1/2}
			\braket{k_{f_1}}{l_{g_1}}
			\braket{n-k_{f_2}}{m-l_{g_2}}
			\\
			&=
			\sum^n_{k=0}
			\sum^m_{l=0}
			\binom{n}{k}^{1/2}
			\binom{m}{l}^{1/2}
			\delta_{k,l}
			\braket{1_{f_1}}{1_{g_1}}^k
			\delta_{n-k,m-l}
			\braket{1_{f_2}}{1_{g_2}}^{n-k}
			\\
			&=
			\sum^n_{k=0}
			\binom{n}{k}^{1/2}
			\binom{m}{k}^{1/2}
			\braket{1_{f_1}}{1_{g_1}}^k
			\delta_{n-k,m-k}
			\braket{1_{f_2}}{1_{g_2}}^{n-k}
			\\
			&=
			\delta_{n,m}
			\sum^n_{k=0}
			\binom{n}{k}
			\braket{1_{f_1}}{1_{g_1}}^k
			\braket{1_{f_2}}{1_{g_2}}^{n-k}
			\\
			&=
			\delta_{n,m}
			\left(
				\braket{1_{f_1}}{1_{g_1}}
				+
				\braket{1_{f_2}}{1_{g_2}}
			\right)^n
			.
		\end{split}
	\end{equation*}
\end{proof}

\qmwqkgcoherentstate
\begin{proof}
	\begin{equation*}
		\begin{split}
			\ket{\vb{\alpha}}
			&=
			\exp\left\{
				-
				\frac{1}{2}
				\comm{\hat{\vb{A}}_\perp^-[\vb{\alpha}]}{\hat{\vb{A}}_\perp^+[\vb{\alpha}]}
			\right\}
			\exp\left\{
				\hat{\vb{A}}_\perp^+[\vb{\alpha}]
			\right\}
			\ket{0}
			\\
			&=
			\exp\left\{
				-
				\frac{1}{2}
				\comm{
					\hat\phi_1^-[\alpha_1]
					+
					\hat\phi_2^-[\alpha_2]
				}{
					\hat\phi_1^+[\alpha_1]
					+
					\hat\phi_2^+[\alpha_2]
				}
			\right\}
			\exp\left\{
				\hat\phi_1^+[\alpha_1]
				+
				\hat\phi_2^+[\alpha_2]
			\right\}
			\ket{0}
			\\
			&=
			\exp\left\{
				-
				\frac{1}{2}
				\comm{\hat\phi_1^-[\alpha_1]}{\hat\phi_1^+[\alpha_1]}
				-
				\frac{1}{2}
				\comm{\hat\phi_2^-[\alpha_2]}{\hat\phi_2^+[\alpha_2]}
			\right\}
			\exp\left\{
				\hat\phi_1^+[\alpha_1]
				+
				\hat\phi_2^+[\alpha_2]
			\right\}
			\ket{0}
			\\
			&=
			\exp\left\{
				\frac{1}{2}
				\comm{\hat\phi_2^-[\alpha_2]}{\hat\phi_2^+[\alpha_2]}
			\right\}
			\exp\left\{
				\hat\phi_2^+[\alpha_2]
			\right\}
			\exp\left\{
				-
				\frac{1}{2}
				\comm{\hat\phi_1^-[\alpha_1]}{\hat\phi_1^+[\alpha_1]}
			\right\}
			\exp\left\{
				\hat\phi_1^+[\alpha_1]
			\right\}
			\ket{0}
			\\
			&=
			\hat{D}_1[\alpha_1]
			\hat{D}_2[\alpha_2]
			\ket{0}
			\\
			&=
			\ket{\alpha_1,\alpha_2}
			.
		\end{split}
	\end{equation*}
\end{proof}
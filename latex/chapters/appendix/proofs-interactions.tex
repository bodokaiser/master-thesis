\section{Dynamical pictures}

\begin{delayedproof}{thm:heisenberg_schroedinger_equivalence}
	Heisenberg and Schrödinger picture are unitarily equivalent by the Stone-von Neumann theorem.
	Let $\hat{O}^H(t)$ be an observable and $\ket{\psi}^H$ a state in the Heisenberg picture, then the expectation values agree
	\begin{equation*}
		\bra{\psi}^H
		\hat{O}^H(t)
		\ket{\psi}^H
		=
		\bra{\psi(0)}^H
		e^{+i\hat{H}t}
		\hat{O}^H(0)
		e^{-i\hat{H}t}
		\ket{\psi(0)}^H
		=
		\bra{\psi(t)}^S
		\hat{O}^S
		\ket{\psi(t)}^S
	\end{equation*}
	where we used \Cref{thm:heisenberg_eom_sol} and \Cref{thm:schroedinger_eom_sol}.
\end{delayedproof}
\begin{delayedproof}{thm:heisenberg_schroedinger_canonical_transformation}
	Let $\hat{A}^H,\hat{B}^H,\hat{C}^H$ be operators in the Heisenberg picture satisfying the commutation relation
	\begin{equation*}
		\comm{\hat{A}^H}{\hat{B}^H}
		=
		\hat{C}^H
	\end{equation*}
	then in the Schrödinger picture, we find
	\begin{equation*}
		\begin{split}
			\comm{\hat{A}^S}{\hat{B}^S}
			&=
			e^{-i\hat{H}t}
			\left\{
				\hat{A}^H
				e^{+i\hat{H}t}
				e^{-i\hat{H}t}
				\hat{B}^H
				-	
				\hat{B}^H
				e^{-i\hat{H}t}
				e^{+i\hat{H}t}
				\hat{A}^H
			\right\}
			e^{+i\hat{H}t}
			\\
			&=
			e^{-i\hat{H}t}
			\comm{\hat{A}^H}{\hat{B}^H}
			e^{+i\hat{H}t}
			\\
			&=
			e^{-i\hat{H}t}
			\hat{C}^H
			e^{+i\hat{H}t}
			=
			\hat{C}^S
		\end{split}
	\end{equation*}
	from Ref.~\cite[p.~213]{Greiner2013}.
\end{delayedproof}
\begin{delayedproof}{thm:dirac_schroedinger_eom}
	See Ref.~\cite[p.~214]{Greiner2013}.
\end{delayedproof}

\subsection{Time-evolution operator}

\begin{delayedproof}{thm:time_evolution_int}
	We integrate \cref{eq:time_evolution_diff} over the interval $[t_0,t]$
	\begin{equation*}
		\int_t^{t_0}\dd{t_1}
		\hat{H}_\text{int}^D(t_1,t_0)
		\hat{U}_D(t_1,t_0)
		=
		i\int_t^{t_0}\dd{t_1}
		\partial_{t_1}
		\hat{U}_D(t_1,t_0)
		=
		i\mathbb{I}
		-
		i\hat{U}_D(t,t_0)
	\end{equation*}
	where we used the fundamental theorem of calculus and the boundary condition $\hat{U}^D(t_0,t_0)=\mathbb{I}$ in the third line.
	We are left to rearrange the terms to find \cref{eq:time_evolution_int}.
\end{delayedproof}
\begin{delayedproof}{thm:time_evolution_exp_sol}
	Use \Cref{thm:time_ordered_integral} with $\hat{A}=\hat{H}_\text{int}^D$ to rewrite the iterative integral solution \Cref{thm:time_evolution_iter_sol}.
	A proof that \cref{eq:time_evolution_sol} satisfies \cref{eq:time_evolution_diff} can be found in Ref.~\cite[p.~219]{Greiner2013}.
\end{delayedproof}

\section{Classical source}

\begin{delayedproof}{thm:displacement_scattering_operator_equivalence}
	The double integral evaluates to
	\begin{equation*}
		\iint\dd[4]{x}\dd[4]{y}
		J(x)
		D(x-y)
		J(y)
		=
		\int\frac{\dd[3]{p}}{(2\pi)^32\omega(\vb{p})}
		\abs{J(\vb{p})}^2
	\end{equation*}
	see Ref.~\cite[p.~26]{Zee2010}.
	Furthermore, decomposing the Klein-Gordon operator into positive and negative frequency parts
	\begin{equation*}
		\int\dd[4]{x}
		J(x)
		\hat\phi(x)
		=
		\int\dd[4]{x}
		J(x)
		\hat\phi^+(x)
		+
		\int\dd[4]{x}
		J(x)
		\hat\phi^-(x)
		=
		\hat\phi^+[J]
		+
		\hat\phi^-[J]
		,
	\end{equation*}
	we note that the second term is the hermitian conjugate of the first.
	Similar to what we did for the single-particle number state, we rewrite the integral using the Fourier transform of the source
	\begin{equation*}
		\begin{split}
			\hat\phi^+[J]
			=
			\int\dd[4]{x}
			J(x)
			\hat\phi^+(x)
			&=
			\int\dd[4]{x}
			J(x)
			\int\frac{\dd[3]{p}}{(2\pi)^3\sqrt{2\omega(\vb{p})}}
			e^{-ip_\mu x^\mu}
			\hat{a}^\dagger(\vb{p})
			\\
			&=
			\int\frac{\dd[3]{p}}{(2\pi)^3\sqrt{2\omega(\vb{p})}}
			\left(
				\int\dd[4]{x}
				J(x)
				e^{-ip_\mu x^\mu}
			\right)
			\hat{a}^\dagger(\vb{p})
			\\
			&=
			\int\frac{\dd[3]{p}}{(2\pi)^3\sqrt{2\omega(\vb{p})}}
			J(\vb{p})
			\hat{a}^\dagger(\vb{p})
		\end{split}
		.
	\end{equation*}
	We therefore find
	\begin{equation*}
		\begin{split}
			\hat{S}
			&=
			N\exp\left\{
				\hat\phi^+[iJ]
				-
				\hat\phi^-[iJ]
			\right\}
			\exp\left\{
				\int\dd[4]{x}\dd[4]{y}
				J(x)
				D(x-y)
				J(y)
			\right\}
			\\
			&=
			\exp\left\{
				-
				\int\frac{\dd[3]{p}}{(2\pi)^3\sqrt{2\omega(\vb{p})}}
				iJ(\vb{p})
				\hat{a}^\dagger(\vb{p})
			\right\}
			\exp\left\{
				+
				\int\frac{\dd[3]{p}}{(2\pi)^3\sqrt{2\omega(\vb{p})}}
				iJ(\vb{p})
				\hat{a}^\dagger(\vb{p})
			\right\}
			\\
			&\times
			\exp\left\{
				-
				\frac{1}{2}
				\int\frac{\dd[3]{p}}{(2\pi)^32\omega(\vb{p})}
				\abs{J(\vb{p})}^2
			\right\}
			=
			\hat{D}[iJ]
		\end{split}
	\end{equation*}
	which shows that $\hat{S}=\hat{D}[iJ]$.
\end{delayedproof}
\section{Single-particle}

\begin{proof}
\begin{equation*}
	\begin{split}
		\phi(t,\vb{x})
		&=
		\int_V\frac{\dd[4]{p}}{(2\pi)^4}
		\phi(p_0,\vb{p})
		e^{+ip_\mu x^\mu}
		\\
		&=
		\int_{\mathbb{R}^4}\frac{\dd[4]{p}}{(2\pi)^3}
		\delta^{(1)}\left(p_0^2-\omega(\vb{p})^2\right)
		\phi(p_0,\vb{p})
		e^{+ip_\mu x^\mu}
		\\
		&=
		\int_{\mathbb{R}^4}\frac{\dd[4]{p}}{(2\pi)^3}
		\frac{
			\delta^{(1)}\left(p_0-\omega(\vb{p})\right)
			+
			\delta^{(1)}\left(p_0+\omega(\vb{p})\right)
		}{\sqrt{2\omega(\vb{p})}}
		\phi(p_0,\vb{p})
		e^{+ip_0t}
		e^{-i\vb{p}\vdot\vb{x}}
		\\
		&=
		\int_{\mathbb{R}^3}\frac{\dd[3]{p}}{(2\pi)^3\sqrt{2\omega(\vb{p})}}
		\biggl\{
			\phi(\omega(\vb{p}),\vb{p})
			e^{+i\omega(\vb{p})t}
			+
			\phi(-\omega(\vb{p}),\vb{p})
			e^{-i\omega(\vb{p})t}
		\biggr\}
		e^{-i\vb{p}\vdot\vb{x}}
		\\
		&=
		\int_{\mathbb{R}^3}\frac{\dd[3]{p}}{(2\pi)^3\sqrt{2\omega(\vb{p})}}
		\biggl\{
			\phi(\omega(\vb{p}),\vb{p})
			e^{+i\omega(\vb{p})t}
			e^{-i\vb{p}\vdot\vb{x}}
			+
			\phi(-\omega(\vb{p}),\vb{p})
			e^{-i\omega(\vb{p})t}
			e^{-i\vb{p}\vdot\vb{x}}
		\biggr\}
		\\
		&=
		\int_{\mathbb{R}^3}\frac{\dd[3]{p}}{(2\pi)^3\sqrt{2\omega(\vb{p})}}
		\biggl\{
			\phi(\omega(\vb{p}),\vb{p})
			e^{+i\omega(\vb{p})t}
			e^{-i\vb{p}\vdot\vb{x}}
			+
			\phi(-\omega(\vb{p}),-\vb{p})
			e^{-i\omega(\vb{p})t}
			e^{+i\vb{p}\vdot\vb{x}}
		\biggr\}
		\\
		&=
		\int_{\mathbb{R}^3}\frac{\dd[3]{p}}{(2\pi)^3\sqrt{2\omega(\vb{p})}}
		\biggl\{
			\phi(\omega(\vb{p}),\vb{p})
			e^{+ip_\mu x^\mu}
			+
			\phi(\omega(\vb{p}),\vb{p})^*
			e^{-ip_\mu x^\mu}
		\biggr\}_{p_0=\omega(\vb{p})}
	\end{split}
\end{equation*}
where we used the conjugate symmetry of $\phi(p_0,\vb{p})$ because $\phi(t,\vb{x})$ is real-valued.
\end{proof}

\subsection{Number, energy and momentum observables}

We start with the expectation values of the energy operator and begin with the auxiliary results
\begin{equation}
	\begin{split}
		\expval{\hat{a}(\vb{q})\hat{H}\hat{a}^\dagger(\vb{p})}{0}
		&=
		\expval{
			\hat{a}(\vb{q})
			\left(
				\int\frac{\dd[3]{k}}{(2\pi)^3}
				\omega(\vb{k})
				\hat{a}^\dagger(\vb{k})
				\hat{a}(\vb{k})
			\right)
			\hat{a}^\dagger(\vb{p})
		}{0}
		\\
		&=
		\int\frac{\dd[3]{k}}{(2\pi)^3}
		\omega(\vb{k})
		\expval{
			\hat{a}(\vb{q})
			\hat{a}^\dagger(\vb{k})
			\hat{a}(\vb{k})
			\hat{a}^\dagger(\vb{p})
		}{0}
		\\
		&=
		\int\frac{\dd[3]{k}}{(2\pi)^3}
		\omega(\vb{k})
		\expval{
			\hat{a}(\vb{q})
			\hat{a}^\dagger(\vb{k})
			(2\pi)^3\delta^{(3)}(\vb{k}-\vb{p})
		}{0}
		\\
		&=
		\omega(\vb{p})
		\expval{
			\hat{a}(\vb{q})
			\hat{a}^\dagger(\vb{p})
		}{0}
		\\
		&=
		\omega(\vb{p})
		(2\pi)^3\delta^{(3)}(\vb{q}-\vb{p})
	\end{split}
\end{equation}
and
\begin{equation}
	\begin{split}
		\expval{
			\hat{a}(\vb{q})
			\hat{H}^2\hat{a}^\dagger(\vb{p})
		}{0}
		&=
		\expval{
			\hat{a}(\vb{q})
			\left(
				\int\frac{\dd[3]{k_1}}{(2\pi)^3}
				\int\frac{\dd[3]{k_2}}{(2\pi)^3}
				\omega(\vb{k}_1)
				\omega(\vb{k}_2)
				\hat{a}^\dagger(\vb{k}_1)
				\hat{a}^\dagger(\vb{k}_2)
				\hat{a}(\vb{k}_1)
				\hat{a}(\vb{k}_2)
			\right)
			\hat{a}^\dagger(\vb{p})
		}{0}
		\\
		&=
		\int\frac{\dd[3]{k_1}}{(2\pi)^3}
		\int\frac{\dd[3]{k_2}}{(2\pi)^3}
		\omega(\vb{k}_1)
		\omega(\vb{k}_2)
		\expval{
			\hat{a}(\vb{q})
			\hat{a}^\dagger(\vb{k}_1)
			\hat{a}^\dagger(\vb{k}_2)
			\hat{a}(\vb{k}_1)
			\hat{a}(\vb{k}_2)
			\hat{a}^\dagger(\vb{p})
		}{0}
		\\
		&=
		\int\frac{\dd[3]{k_1}}{(2\pi)^3}
		\int\frac{\dd[3]{k_2}}{(2\pi)^3}
		\omega(\vb{k}_1)
		\omega(\vb{k}_2)
		\expval{
			(2\pi)^3
			\delta^{(3)}(\vb{k}_1-\vb{q})
			\hat{a}^\dagger(\vb{k}_2)
			\hat{a}(\vb{k}_1)
			(2\pi)^3
			\delta^{(3)}(\vb{k}_2-\vb{p})
		}{0}
		\\
		&=
		\omega(\vb{p})
		\omega(\vb{q})
		\expval{
			\hat{a}^\dagger(\vb{p})
			\hat{a}(\vb{q})
		}{0}
		\\
		&=
		\omega(\vb{p})^2
		(2\pi)^3
		\delta^{(3)}(\vb{q}-\vb{p})
	\end{split}
	.
\end{equation}
The first moment then turns out to be
\begin{equation}
	\begin{split}
		\expval{\hat{H}}{f}
		&=
		\left(
			\int\frac{\dd[3]{p}}{(2\pi)^3\sqrt{2\omega(\vb{p})}}
			f(\vb{p})
			\hat{a}^\dagger(\vb{p})
			\ket{0}
		\right)^\dagger
		\hat{H}
		\left(
			\int\frac{\dd[3]{p}}{(2\pi)^3\sqrt{2\omega(\vb{q})}}
			f(\vb{q})
			\hat{a}^\dagger(\vb{q})
			\ket{0}
		\right)^\dagger
		\\
		&=
		\int\frac{\dd[3]{p}}{(2\pi)^3\sqrt{2\omega(\vb{p})}}
		\int\frac{\dd[3]{p}}{(2\pi)^3\sqrt{2\omega(\vb{q})}}
		f(\vb{p})^*
		f(\vb{q})
		\expval{
			\hat{a}(\vb{p})
			\hat{H}
			\hat{a}^\dagger(\vb{q})
		}{0}
		\\
		&=
		\int\frac{\dd[3]{p}}{(2\pi)^3\sqrt{2\omega(\vb{p})}}
		\int\frac{\dd[3]{p}}{(2\pi)^3\sqrt{2\omega(\vb{q})}}
		f(\vb{p})^*
		f(\vb{q})		
		\omega(\vb{p})
		(2\pi)^3\delta^{(3)}(\vb{q}-\vb{p})
		\\
		&=
		\int\frac{\dd[3]{p}}{(2\pi)^3\sqrt{2\omega(\vb{p})}}
		\omega(\vb{p})
		\abs{f(\vb{p})}^2
	\end{split}
\end{equation}
and the second moment is
\begin{equation}
	\begin{split}
		\expval{\hat{H}^2}{f}
		&=
		\left(
			\int\frac{\dd[3]{p}}{(2\pi)^3\sqrt{2\omega(\vb{p})}}
			f(\vb{p})
			\hat{a}^\dagger(\vb{p})
			\ket{0}
		\right)^\dagger
		\hat{H}^2
		\left(
			\int\frac{\dd[3]{p}}{(2\pi)^3\sqrt{2\omega(\vb{q})}}
			f(\vb{q})
			\hat{a}^\dagger(\vb{q})
			\ket{0}
		\right)^\dagger
		\\
		&=
		\int\frac{\dd[3]{p}}{(2\pi)^3\sqrt{2\omega(\vb{p})}}
		\int\frac{\dd[3]{p}}{(2\pi)^3\sqrt{2\omega(\vb{q})}}
		f(\vb{p})^*
		f(\vb{q})
		\expval{
			\hat{a}(\vb{p})
			\hat{H}^2
			\hat{a}^\dagger(\vb{q})
		}{0}
		\\
		&=
		\int\frac{\dd[3]{p}}{(2\pi)^3\sqrt{2\omega(\vb{p})}}
		\int\frac{\dd[3]{p}}{(2\pi)^3\sqrt{2\omega(\vb{q})}}
		f(\vb{p})^*
		f(\vb{q})
		\omega(\vb{p})^2
		(2\pi)^3
		\delta^{(3)}(\vb{q}-\vb{p})
		\\
		&=
		\int\frac{\dd[3]{p}}{(2\pi)^32\omega(\vb{p})}
		\omega(\vb{p})^2
		\abs{f(\vb{p})}^2
	\end{split}
	.
\end{equation}
Replacing $\omega(\vb{p})$ with either $1$ or $\vb{p}$ yields the first two moments of the number $\hat{N}$ respective momentum $\hat{\vb{P}}$ operator.

\subsection{Center-of-mass position and dispersion}

The coordinate the wave function is
\begin{equation}
	\begin{split}
		\psi(t,\vb{x})
		&=
		\bra{0}
		\hat\phi(t,\vb{x})
		\ket{f}
		=
		\bra{0}
		\hat\phi^-(t,\vb{x})
		\ket{f}
		\\
		&=
		\left(
			\int\frac{\dd[3]{p}}{(2\pi)^3\sqrt{2\omega(\vb{p})}}
			e^{+ip_\mu x^\mu}
			\hat{a}^\dagger(\vb{p})
			\ket{0}
		\right)^\dagger
		\int\frac{\dd[3]{q}}{(2\pi)^3\sqrt{2\omega(\vb{q})}}
		f(\vb{q})
		\hat{a}^\dagger(\vb{q})
		\ket{0}
		\\
		&=
		\left(
			\int\frac{\dd[3]{p}}{(2\pi)^3\sqrt{2\omega(\vb{p})}}
			e^{-ip_\mu x^\mu}
			\bra{0}
			\hat{a}(\vb{p})
		\right)^\dagger
		\int\frac{\dd[3]{q}}{(2\pi)^3\sqrt{2\omega(\vb{q})}}
		f(\vb{q})
		\hat{a}^\dagger(\vb{q})
		\ket{0}
		\\
		&=
		\int\frac{\dd[3]{p}}{(2\pi)^3\sqrt{2\omega(\vb{p})}}
		\int\frac{\dd[3]{q}}{(2\pi)^3\sqrt{2\omega(\vb{q})}}
		f(\vb{q})
		e^{-ip_\mu x^\mu}
		\expval{\hat{a}(\vb{p})\hat{a}^\dagger(\vb{q})}{0}
		\\
		&=
		\int\frac{\dd[3]{p}}{(2\pi)^3\sqrt{2\omega(\vb{p})}}
		\int\frac{\dd[3]{q}}{(2\pi)^3\sqrt{2\omega(\vb{q})}}
		f(\vb{q})
		e^{-ip_\mu x^\mu}
		(2\pi)^3
		\delta^{(3)}(\vb{q}-\vb{p})
		\\
		&=
		\int\frac{\dd[3]{p}}{(2\pi)^32\omega(\vb{p})}
		f(\vb{p})
		e^{-ip_\mu x^\mu}
	\end{split}
\end{equation}
the probability current is a conserved Noether current~\cite[p.~18]{Peskin1995}
\begin{equation}
	\begin{split}
		j_\mu(t,\vb{x})
		&=
		i
		\bigl\{
			\psi(t,\vb{x})
			\partial_\mu
			\psi(t,\vb{x})^*
			-
			\psi(t,\vb{x})^*
			\partial_\mu
			\psi(t,\vb{x})
		\bigr\}
		=
		2
		\Im\left\{
			\psi(t,\vb{x})^*
			\partial_\mu
			\psi(t,\vb{x})
		\right\}
		\\
		&=
		2
		\Im\left\{
			\int\frac{\dd[3]{p}}{(2\pi)^32\omega(\vb{p})}
			f(\vb{p})^*e^{+ip_\mu x^\mu}
			\int\frac{\dd[3]{q}}{(2\pi)^32\omega(\vb{q})}
			f(\vb{q})iq_\mu e^{-iq_\mu x^\mu}
		\right\}
		\\
		&=
		2
		\int\frac{\dd[3]{p}}{(2\pi)^32\omega(\vb{p})}
		\int\frac{\dd[3]{q}}{(2\pi)^32\omega(\vb{q})}
		\Im\left\{
			f(\vb{p})^*
			f(\vb{q})
			iq_\mu
			e^{-i(q_\mu-p_\mu)x^\mu}
		\right\}
		\\
		&=
		2
		\int\frac{\dd[3]{p}}{(2\pi)^32\omega(\vb{p})}
		\int\frac{\dd[3]{q}}{(2\pi)^32\omega(\vb{q})}
		q_\mu
		\Re\left\{
			f(\vb{p})^*
			f(\vb{q})
			e^{-i(q_\mu-p_\mu)x^\mu}
		\right\}
		\\
		&=
		\int\frac{\dd[3]{p}}{(2\pi)^32\omega(\vb{p})}
		\int\frac{\dd[3]{q}}{(2\pi)^32\omega(\vb{q})}
		\left\{
			q_\mu
			+
			p_\mu
		\right\}
		f(\vb{p})^*
		f(\vb{q})
		e^{-i(q_\mu-p_\mu)x^\mu}
	\end{split}
\end{equation}
where we expanded the real part and renamed the integrations variables of the second term in the last line.

\subsubsection{Center-of-mass position}

The center-of-mass position can be expressed in terms of the velocity
\begin{equation}
	\begin{split}
		\overline{\vb{x}}
		&=
		\int\dd[3]{x}
		\vb{x}
		\rho(t,\vb{x})
		\\
		&=
		2\Re\left\{
			\int\dd[3]{x}
			\vb{x}
			\int\frac{\dd[3]{p}}{(2\pi)^32\omega(\vb{p})}
			\int\frac{\dd[3]{q}}{(2\pi)^32\omega(\vb{q})}
			\omega(\vb{q})
			f(\vb{q})
			f(\vb{p})^*
			e^{-i(p_\mu-q_\mu)x^\mu}
		\right\}
		\\
		&=
		2\Re\left\{
			\int\dd[3]{x}
			\vb{x}
			\int\frac{\dd[3]{p}}{(2\pi)^32\omega(\vb{p})}
			\int\frac{\dd[3]{q}}{(2\pi)^3}
			f(\vb{q})
			f(\vb{p})^*
			e^{-i(p_\mu-q_\mu)x^\mu}
		\right\}
		\\
		&=
		2\Re\left\{
			\int\frac{\dd[3]{p}}{(2\pi)^32\omega(\vb{p})}
			f(\vb{p})^*
			e^{-i\omega(\vb{p})t}
			\int\frac{\dd[3]{k}}{(2\pi)^3}
			f(\vb{p}+\vb{k})
			e^{+i\omega(\vb{k}+\vb{p})t}
			\int\dd[3]{x}
			\vb{x}
			e^{+i\vb{k}\vdot\vb{x}}
		\right\}
	\end{split}
\end{equation}
We take $g(\vb{k})=f(\vb{p}+\vb{k})e^{-i\omega(\vb{k}+\vb{p})t}$ and use partial integration with vanishing boundaries because $g(\vb{k})$ is a Schwartz function
\begin{equation}
	\begin{split}
		\int\frac{\dd[3]{k}}{(2\pi)^3}
		g(\vb{k})
		\int\dd[3]{x}
		\vb{x}
		e^{+i\vb{k}\vdot\vb{x}}
		&=
		\int\frac{\dd[3]{k}}{(2\pi)^3}
		g(\vb{k})
		\left(-i\grad_{\vb{k}}\right)
		\int\dd[3]{x}
		e^{+i\vb{k}\vdot\vb{x}}
		\\
		&=
		\int\frac{\dd[3]{k}}{(2\pi)^3}
		g(\vb{k})
		\left(-i\grad_{\vb{k}}\right)
		(2\pi)^3\delta^{(3)}(\vb{k})
		\\
		&=
		\int\dd[3]{k}
		\left(i\grad_{\vb{k}}g(\vb{k})\right)
		\delta^{(3)}(\vb{k})
		\\
		&=
		i\eval{\grad_{\vb{k}}}_{\vb{k}=0}
		f(\vb{p}+\vb{k})
		e^{+i\omega(\vb{k}+\vb{p})t}
		\\
		&=
		ie^{+i\omega(\vb{p})t}
		\left(
			\eval{\grad_{\vb{k}}}_{\vb{k}=0}
			f(\vb{p}+\vb{k})
			+
			f(\vb{p})
			\eval{\grad_{\vb{k}}}_{\vb{k}=0}
			it\omega(\vb{p}+\vb{k})
		\right)
		\\
		&=
		ie^{+i\omega(\vb{p})t}
		\left(
			\grad_{\vb{p}}
			f(\vb{p})
			+
			it
			f(\vb{p})
			\grad_{\vb{p}}
			\omega(\vb{p})
		\right)
	\end{split}
\end{equation}
Inserting this back
\begin{equation}
	\begin{split}
		\overline{\vb{x}}
		&=
		2\Re\left\{
			\int\frac{\dd[3]{p}}{(2\pi)^32\omega(\vb{p})}
			f(\vb{p})^*
			e^{-i\omega(\vb{p})t}
			ie^{+i\omega(\vb{p})t}
			\left(
				\grad_{\vb{p}}
				f(\vb{p})
				+
				it
				f(\vb{p})
				\grad_{\vb{p}}
				\omega(\vb{p})
			\right)
		\right\}
		\\
		&=
		2\Re\left\{
			\int\frac{\dd[3]{p}}{(2\pi)^32\omega(\vb{p})}
			f(\vb{p})^*
			i
			\left(
				\grad_{\vb{p}}
				f(\vb{p})
				+
				it
				f(\vb{p})
				\grad_{\vb{p}}
				\omega(\vb{p})
			\right)
		\right\}
		\\
		&=
		-
		2\Im\left\{
			\int\frac{\dd[3]{p}}{(2\pi)^32\omega(\vb{p})}
			f(\vb{p})^*
			\grad_{\vb{p}}
			f(\vb{p})
		\right\}
		-
		2
		\int\frac{\dd[3]{p}}{(2\pi)^32\omega(\vb{p})}
		\abs{f(\vb{p})}^2
		\grad_{\vb{p}}
		\omega(\vb{p})t
		\\
		&=
		2\Im\left\{
			\int\frac{\dd[3]{p}}{(2\pi)^3}
			f(\vb{p})
			\grad_{\vb{p}}
			\frac{f(\vb{p})^*}{2\omega(\vb{p})}
		\right\}
		-
		2
		\int\frac{\dd[3]{p}}{(2\pi)^32\omega(\vb{p})}
		\abs{f(\vb{p})}^2
		\grad_{\vb{p}}
		\omega(\vb{p})t
		\\
		&=
		2\Im\left\{
			\int\frac{\dd[3]{p}}{(2\pi)^3}
			f(\vb{p})
			\frac{
				\omega(\vb{p})
				\grad_{\vb{p}}
				f(\vb{p})^*
				-
				f(\vb{p})^*
				\grad_{\vb{p}}
				\omega(\vb{p})
			}{2\omega(\vb{p})^2}
		\right\}
		-
		2
		\int\frac{\dd[3]{p}}{(2\pi)^32\omega(\vb{p})}
		\abs{f(\vb{p})}^2
		\grad_{\vb{p}}
		\omega(\vb{p})t
	\end{split}
\end{equation}

If we want
\begin{equation}
	\overline{\vb{x}}
	=
	\int\frac{\dd[3]{p}}{(2\pi)^32\omega(\vb{p})}
	\frac{\vb{p}}{\omega(\vb{p})}
	\abs{f(\vb{p})}^2
	t
	=
	\overline{\vb{v}}t
\end{equation}
we need
\begin{equation}
			\int\frac{\dd[3]{k}}{(2\pi)^3}
			f(\vb{p}+\vb{k})
			e^{-i\omega(\vb{k}+\vb{p})t}
			\int\dd[3]{x}
			\vb{x}
			e^{+i\vb{k}\vdot\vb{x}}
			=
			\frac{\vb{p}}{\omega(\vb{p})}f(\vb{p})t	
\end{equation}

\section{Coherent state}

Ref.~\cite[p.~40]{Barnett2002} proves
\begin{equation}
	e^{\hat{A}}
	\hat{B}
	e^{-\hat{A}}
	=
	\hat{B}
	+
	\comm{\hat{A}}{\hat{B}}
	+
	\frac{1}{2!}
	\comm{\hat{A}}{\comm{\hat{A}}{\hat{B}}}
	+
	\frac{1}{3!}
	\comm{\hat{A}}{\comm{\hat{A}}{\comm{\hat{A}}{\hat{B}}}}
\end{equation}
which for $\comm{\hat{A}}{\hat{B}}=0$ reduces to
\begin{equation}
	e^{\hat{A}}
	\hat{B}
	e^{-\hat{A}}
	=
	\hat{B}
	+
	\comm{\hat{A}}{\hat{B}}
\end{equation}

Wick's theorem from Lancaster~\cite[p.~173]{Lancaster2014}
\begin{equation}
	T
	\hat\phi(x_1)
	\hat\phi(x_2)
	=
	N
	\hat\phi(x_1)
	\hat\phi(x_2)
	+
	D_F(x_1-x_2)	
\end{equation}
where the Feynman propagator $D_F(x_1-x_2)$ is the result of the contraction and odd orders vanish.
The next higher order is
\begin{equation}
	\begin{split}
		T
		\hat\phi(x_1)
		\hat\phi(x_2)
		\hat\phi(x_3)
		\hat\phi(x_4)
		&=
		N
		\hat\phi(x_1)
		\hat\phi(x_2)
		\hat\phi(x_3)
		\hat\phi(x_4)
		\\
		&+
		D_F(x_1-x_2)
		N
		\hat\phi(x_3)
		\hat\phi(x_4)
		+
		D_F(x_2-x_3)
		N
		\hat\phi(x_1)
		\hat\phi(x_4)
		\\
		&+
		D_F(x_3-x_4)
		N
		\hat\phi(x_1)
		\hat\phi(x_2)
		+
		D_F(x_1-x_3)
		N
		\hat\phi(x_2)
		\hat\phi(x_4)
		\\
		&+
		D_F(x_1-x_4)
		N
		\hat\phi(x_2)
		\hat\phi(x_3)
		+
		D_F(x_2-x_4)
		N
		\hat\phi(x_1)
		\hat\phi(x_3)
		\\
		&+
		D_F(x_1-x_2)
		D_F(x_3-x_4)
		+
		D_F(x_1-x_3)
		D_F(x_2-x_4)
		\\
		&+
		D_F(x_1-x_4)
		D_F(x_2-x_3)
	\end{split}
\end{equation}
which is already quite complex but fortunately simplifies after integration
\begin{align*}
	T
	\int\dd[4]{x_1}\dd[4]{x_2}
	J(x_1)
	J(x_2)
	\hat\phi(x_1)
	\hat\phi(x_2)
	&=
	N
	\int\dd[4]{x_1}\dd[4]{x_2}
	J(x_1)
	J(x_2)
	\hat\phi(x_1)
	\hat\phi(x_2)
	\\
	&+
	\int\dd[4]{x_1}\dd[4]{x_2}
	J(x_1)
	D_F(x_1-x_2)
	J(x_2)
	\\
	T
	\int\dd[4]{x_1}\dots\dd[4]{x_4}
	J(x_1)
	\dots
	J(x_4)
	\hat\phi(x_1)
	\dots
	\hat\phi(x_4)
	&=
	N
	\int\dd[4]{x_1}\dots\dd[4]{x_4}
	J(x_1)
	\dots
	J(x_4)
	\hat\phi(x_1)
	\dots
	\hat\phi(x_4)
	\\
	&+
	6\int\dd[4]{x_1}\dd[4]{x_2}
	J(x_1)
	D_F(x_1-x_2)
	J(x_2)
	\\
	&+
	3\left(
		\int\dd[4]{x_1}\dd[4]{x_2}
		J(x_1)
		D_F(x_1-x_2)
		J(x_2)
	\right)^2
\end{align*}
because we can relabel the integration variables to equalize the permutations.
We expect that this generalizes to
\begin{equation}
	\begin{split}
		&
		T
		\int\dd[4]{x_1}\dots\dd[4]{x_n}
		J(x_1)
		\dots
		J(x_n)
		\hat\phi(x_1)
		\dots
		\hat\phi(x_n)
		\\
		=&\
		N
		\int\dd[4]{x_1}\dots\dd[4]{x_n}
		J(x_1)
		\dots
		J(x_n)
		\hat\phi(x_1)
		\dots
		\hat\phi(x_n)
		\\
		+&\
		\binom{n}{2}
		\int\dd[4]{x_1}\dd[4]{x_2}
		J(x_1)
		D_F(x_1-x_2)
		J(x_2)
		\\
		+&\
		\binom{n}{4}
		\left(
			\int\dd[4]{x_1}\dd[4]{x_2}
			J(x_1)
			D_F(x_1-x_2)
			J(x_2)
		\right)^2
		+
		\dots
	\end{split}
\end{equation}

\begin{equation}
	\begin{split}
		&
		T\exp\left\{
			-i
			\int\dd[4]{x}
			\hat\phi(x)
			J(x)
		\right\}
		\\
		=&\
		\sum_{n=0}^\infty
		\frac{1}{n!}
		(-i)^n
		\int\dd[4]{x_1}\dots\dd[4]{x_n}
		J(x_1)\dots J(x_n)
		T\hat\phi(x_1)\dots\hat\phi(x_n)
		\\
		=&\
		N\exp\left\{
			-i
			\int\dd[4]{x}
			\hat\phi(x)
			J(x)
		\right\}
		\\
		\times&\
		\left\{
			\sum_{n=0}^\infty
			\frac{1}{n!}
			(-i)^{n}
			\int\dd[4]{x_1}\dots\dd[4]{x_n}
			J(x_1)\dots J(x_n)
			\sum_\text{perm}
			D_F(x_1-x_2)\dots D_F(x_{n-1}-x_n)
		\right\}
		\\
		=&\
		N\exp\left\{
			-i
			\int\dd[4]{x}
			\hat\phi(x)
			J(x)
		\right\}
		\\
		\times&\
		\left\{
			\sum_{n=0}^\infty
			\frac{1}{n!}
			(-i)^{n}
			\sum_\text{perm}
			\int\dd[4]{x_1}\dd[4]{x_2}
			J(x_1)D_F(x_1-x_2)J(x_2)
			\dots
			\int\dd[4]{x_{n-1}}\dd[4]{x_n}
			J(x_{n-1})D_F(x_{n-1}-x_n)J(x_n)
		\right\}
	\end{split}
\end{equation}
where we used Wick's theorem as in Ref.~\cite[p.~89]{Peskin1995}

\subsection{Eigenvalue of annihilation operator}

\begin{lemma}\label{th:normal_ordered_a1_cn}
	Let $\hat{a}(\vb{p}),\hat{a}^\dagger(\vb{p})$ be the annihilation and creation operator of the Klein-Gordon field satisfying the canonical commutation relations then for $n\in\mathbb{N}$
	\begin{equation}
		\hat{a}(\vb{p})
		\prod_{j=1}^n
		\hat{a}^\dagger(\vb{q}_j)
		\ket{0}
		=
		\sum_{i=1}^n
		(2\pi)^3
		\delta^{(3)}(\vb{q}_i-\vb{p})
		\prod_{\substack{j=1\\j\neq i}}^n
		\hat{a}^\dagger(\vb{q}_j)
		\ket{0}
		\label{eq:normal_ordered_a1_cn}
		.
	\end{equation}
\end{lemma}
\begin{proof}
	To prove by induction, we start with $n=1$
	\begin{equation*}
		\hat{a}(\vb{p})
		\prod_{j=1}^1
		\hat{a}^\dagger(\vb{q}_j)
		\ket{0}
		=
		\hat{a}(\vb{p})
		\hat{a}^\dagger(\vb{q}_1)
		\ket{0}
		=
		(2\pi)^3
		\delta^{(3)}(\vb{q}_1-\vb{p})
		\ket{0}
	\end{equation*}
	where we used the commutation relation and $\hat{a}(\vb{p})\ket{0}=0$.
	We assume the induction hypothesis, \cref{eq:normal_ordered_a1_cn}, to hold for $n\in\mathbb{N}$ and perform the induction step $n\to n+1$
	\begin{equation*}
		\begin{split}
			\hat{a}(\vb{p})
			\prod_{j=1}^{n+1}
			\hat{a}^\dagger(\vb{q}_j)
			\ket{0}
			&=
			\hat{a}(\vb{p})
			\left(
				\prod_{j=1}^n
				\hat{a}^\dagger(\vb{q}_j)
			\right)
			\hat{a}^\dagger(\vb{q}_{n+1})
			\ket{0}
			\\
			&=
			\hat{a}(\vb{p})
			\hat{a}^\dagger(\vb{q}_{n+1})
			\left(
				\prod_{j=1}^n
				\hat{a}^\dagger(\vb{q}_j)
			\right)
			\ket{0}
			\\
			&=
			(2\pi)^3
			\delta^{(3)}(\vb{q}_{n+1}-\vb{p})
			\prod_{j=1}^n
			\hat{a}^\dagger(\vb{q}_j)
			\ket{0}
			+
			\hat{a}^\dagger(\vb{q}_{n+1})
			\hat{a}(\vb{p})
			\prod_{j=1}^n
			\hat{a}^\dagger(\vb{q}_j)
			\ket{0}
			\\
			&=
			(2\pi)^3
			\delta^{(3)}(\vb{q}_{n+1}-\vb{p})
			\prod_{j=1}^n
			\hat{a}^\dagger(\vb{q}_j)
			\ket{0}
			+
			\sum_{i=1}^n
			(2\pi)^3
			\delta^{(3)}(\vb{q}_i-\vb{p})
			\prod_{\substack{j=1\\j\neq i}}^{n+1}
			\hat{a}^\dagger(\vb{q}_j)
			\ket{0}
			\\
			&=
			\sum_{i=1}^{n+1}
			(2\pi)^3
			\delta^{(3)}(\vb{q}_i-\vb{p})
			\prod_{\substack{j=1\\j\neq i}}^{n+1}
			\hat{a}^\dagger(\vb{q}_j)
			\ket{0}
		\end{split}
	\end{equation*}
	where we used that creation operators commute in the first line, the canonical commutation relation in the second line, and the induction hypothesis in the third line.
\end{proof}

\begin{theorem}
	Let $\ket{\alpha}$ be a coherent state of the Klein-Gordon field and $\hat{a}(\vb{p})$ the annihilation operator satisfying the canonical commutation relation of the Klein-Gordon field, then the coherent state is an eigenstate of the annihilation operator
	\begin{equation}
		\hat{a}(\vb{p})
		\ket{\alpha}
		=
		\frac{\alpha(\vb{p})}{\sqrt{2\omega(\vb{p})}}
		\ket{\alpha}
		.
	\end{equation}
\end{theorem}
\begin{proof}
	We use the series representation of the operator exponential and \cref{th:normal_ordered_a1_cn} to move the annihilation operator to the right
	\begin{equation*}
		\begin{split}
			&
			\hat{a}(\vb{p})
			\exp\left\{
				-i
				\int\frac{\dd[3]{p}\alpha(\vb{p})}{(2\pi)^3\sqrt{2\omega(\vb{p})}}
				\hat{a}^\dagger(\vb{p})
			\right\}
			\ket{0}
			\\
			=&\
			\hat{a}(\vb{p})
			\sum_{n=0}^\infty
			\frac{1}{n!}
			\left(
				-i
				\int\frac{\dd[3]{p}\alpha(\vb{p})}{(2\pi)^3\sqrt{2\omega(\vb{p})}}
				\hat{a}^\dagger(\vb{p})
			\right)^n
			\ket{0}
			\\
			=&\
			\sum_{n=0}^\infty
			\frac{1}{n!}
			(-i)^n
			\int\frac{\dd[3]{p_1}\alpha(\vb{p}_1)}{(2\pi)^3\sqrt{2\omega(\vb{p}_1)}}
			\dots
			\int\frac{\dd[3]{p_n}\alpha(\vb{p}_n)}{(2\pi)^3\sqrt{2\omega(\vb{p}_n)}}
			\hat{a}(\vb{p})
			\prod_{j=1}^n
			\hat{a}^\dagger(\vb{p}_j)
			\ket{0}
			\\
			=&\
			\sum_{n=1}^\infty
			\frac{1}{n!}
			(-i)^n
			\int\frac{\dd[3]{p_1}\alpha(\vb{p}_1)}{(2\pi)^3\sqrt{2\omega(\vb{p}_1)}}
			\dots
			\int\frac{\dd[3]{p_n}\alpha(\vb{p}_n)}{(2\pi)^3\sqrt{2\omega(\vb{p}_n)}}
			\sum_{i=1}^n
			(2\pi)^3
			\delta^{(3)}(\vb{p}_i-\vb{p})
			\prod_{\substack{j=1\\j\neq i}}^n
			\hat{a}^\dagger(\vb{p}_j)
			\ket{0}
			\\
			=&\
			\sum_{n=1}^\infty
			\frac{1}{n!}
			(-i)^n
			\sum_{i=1}^n
			\frac{\alpha(\vb{p})}{\sqrt{2\omega(\vb{p})}}
			\prod_{\substack{j=1\\j\neq i}}^n
			\int\frac{\dd[3]{p_j}\alpha(\vb{p}_j)}{(2\pi)^3\sqrt{2\omega(\vb{p}_j)}}
			\hat{a}^\dagger(\vb{p}_j)
			\ket{0}
			\\
			=&\
			\sum_{n=1}^\infty
			\frac{1}{(n-1)!}
			(-i)^n
			\frac{\alpha(\vb{p})}{\sqrt{2\omega(\vb{p})}}
			\left(
				\int\frac{\dd[3]{p}\alpha(\vb{p})}{(2\pi)^3\sqrt{2\omega(\vb{p})}}
				\hat{a}^\dagger(\vb{p})
			\right)^n
			\ket{0}
			\\
			=&\
			\frac{\alpha(\vb{p})}{\sqrt{2\omega(\vb{p})}}
			\sum_{n=0}^\infty
			\frac{1}{n!}
			\left(
				-i
				\int\frac{\dd[3]{p}\alpha(\vb{p})}{(2\pi)^3\sqrt{2\omega(\vb{p})}}
				\hat{a}^\dagger(\vb{p})
			\right)^n
			\ket{0}
			\\
			=&\
			\frac{\alpha(\vb{p})}{\sqrt{2\omega(\vb{p})}}
			\exp\left\{
				-i
				\int\frac{\dd[3]{p}\alpha(\vb{p})}{(2\pi)^3\sqrt{2\omega(\vb{p})}}
				\hat{a}^\dagger(\vb{p})
			\right\}
			\ket{0}
		\end{split}
	\end{equation*}
	and to obtain the eigenvalue equation we are left to multiply both sides with
	\begin{equation*}
		\exp\left\{
			-
			\frac{1}{2}
			\int\frac{\dd[3]{p}}{(2\pi)^3\sqrt{2\omega(\vb{p})}}
			\abs{\alpha(\vb{p})}^2
		\right\}
		.
	\end{equation*}
\end{proof}

\subsection{Number, energy, and momentum observables}

As for the single-particle case, we calculate the moments for $\hat{H}$ and replace $\omega(\vb{p})\to1,\vb{p}$ to obtain the results for $\hat{N},\hat{P}$
\begin{equation}
	\begin{split}
		\expval{\hat{H}}{\alpha}
		&=
		\int\frac{\dd[3]{p}}{(2\pi)^3}
		\omega(\vb{p})
		\expval{\hat{a}^\dagger(\vb{p})\hat{a}(\vb{p})}{\alpha}
		\\
		&=
		\int\frac{\dd[3]{p}}{(2\pi)^3}
		\omega(\vb{p})
		\expval{\frac{\alpha(\vb{p})^*}{2\omega(\vb{p})}\frac{\alpha(\vb{p})}{2\omega(\vb{p})}}{\alpha}
		\\
		&=
		\int\frac{\dd[3]{p}}{(2\pi)^32\omega(\vb{p})}
		\omega(\vb{p})
		\abs{\alpha(\vb{p})}^2
	\end{split}
\end{equation}
for the first and
\begin{equation}
	\begin{split}
		\expval{\hat{H}^2}{\alpha}
		&=
		\int\frac{\dd[3]{p}_1}{(2\pi)^3}
		\int\frac{\dd[3]{p}_2}{(2\pi)^3}
		\omega(\vb{p}_1)
		\omega(\vb{p}_2)
		\expval{
			\hat{a}^\dagger(\vb{p}_1)
			\hat{a}(\vb{p}_1)
			\hat{a}^\dagger(\vb{p}_2)
			\hat{a}(\vb{p}_2)
		}{\alpha}
		\\
		&=
		\int\frac{\dd[3]{p}_1}{(2\pi)^3}
		\int\frac{\dd[3]{p}_2}{(2\pi)^3}
		\omega(\vb{p}_1)
		\omega(\vb{p}_2)
		\expval{
			\hat{a}^\dagger(\vb{p}_1)
			(2\pi)^3
			\delta^{(3)}(\vb{p}_2-\vb{p}_1)
			\hat{a}(\vb{p}_2)
		}{\alpha}
		\\
		&=
		\int\frac{\dd[3]{p}}{(2\pi)^3}
		\omega(\vb{p})^2
		\expval{
			\hat{a}^\dagger(\vb{p})
			\hat{a}(\vb{p})
		}{\alpha}
		\\
		&=
		\int\frac{\dd[3]{p}}{(2\pi)^32\omega(\vb{p})}
		\omega(\vb{p})^2
		\abs{\alpha(\vb{p})}^2
	\end{split}
\end{equation}
for the second moment.

The probability to measure the vacuum given a coherent state is
\begin{equation}
	\begin{split}
		\braket{0}{\alpha}
		&=
		\exp\left\{
			-
			\frac{1}{2}
			\int\frac{\dd[3]{p}}{(2\pi)^32\omega(\vb{p})}
			\abs{\alpha(\vb{p})}^2
		\right\}
		\expval{0}
		\exp\left\{
			-i
			\int\frac{\dd[3]{q}}{(2\pi)^3\sqrt{2\omega(\vb{q})}}
			\alpha(\vb{q})
			\hat{a}^\dagger(\vb{q})
		\right\}
		\ket{0}
		\\
		&=
		\exp\left\{
			-
			\frac{1}{2}
			\int\frac{\dd[3]{p}}{(2\pi)^32\omega(\vb{p})}
			\abs{\alpha(\vb{p})}^2
		\right\}
		\left\{
			\braket{0}
			-
			i\int\frac{\dd[3]{q}}{(2\pi)^3\sqrt{2\omega(\vb{q})}}
			\alpha(\vb{q})
			\expval{\hat{a}^\dagger(\vb{q})}{0}
			+
			\dots
		\right\}
		\\
		&=
		\exp\left\{
			-
			\frac{1}{2}
			\int\frac{\dd[3]{p}}{(2\pi)^32\omega(\vb{p})}
			\abs{\alpha(\vb{p})}^2
		\right\}
		=
		\exp\left\{
			-
			\frac{1}{2}
			\expval{\hat{N}}{\alpha}
		\right\}
	\end{split}
\end{equation}
For instance, the overlap with a single-particle state is
\begin{equation}
	\begin{split}
		\braket{f}{\alpha}
		&=
		\left(
			\int\frac{\dd[3]{p}}{(2\pi)^3\sqrt{2\omega(\vb{p})}}
			f(\vb{p})
			\hat{a}^\dagger(\vb{p})
			\ket{0}
		\right)^\dagger
		\ket{\alpha}
		\\
		&=
		\int\frac{\dd[3]{p}}{(2\pi)^3\sqrt{2\omega(\vb{p})}}
		f(\vb{p})^*
		\bra{0}
		\hat{a}(\vb{p})
		\ket{\alpha}
		\\
		&=
		\int\frac{\dd[3]{p}}{(2\pi)^32\omega(\vb{p})}
		f(\vb{p})^*
		\alpha(\vb{p})
		\braket{0}{\alpha}
		\\
		&=
		\int\frac{\dd[3]{p}}{(2\pi)^32\omega(\vb{p})}
		f(\vb{p})^*
		\alpha(\vb{p})
		\exp\left\{
			-
			\frac{1}{2}
			\expval{\hat{N}}{\alpha}
		\right\}
	\end{split}
\end{equation}
\section{Single-particle}

\subsection{Number, energy and momentum observables}

We define the operator
\begin{equation}
	\hat{A}
	=
	g(\vb{k})
	\hat{a}^\dagger(\vb{k})
	\hat{a}(\vb{k})
\end{equation}
where $g$ is allowed to be vector-valued, e.g., $g(\vb{p})=\vb{p}$.
To calculate the first two moments of $\hat{A}$, we need the auxiliary results
\begin{equation}
	\begin{split}
		\expval{\hat{a}(\vb{q})\hat{A}\hat{a}^\dagger(\vb{p})}{0}
		&=
		\int\frac{\dd[3]{k}}{(2\pi)^3}
		g(\vb{k})
		\expval{\hat{a}(\vb{q})\hat{a}^\dagger(\vb{k})\hat{a}(\vb{k})\hat{a}^\dagger(\vb{p})}{0}
		\\
		&=
		\int\frac{\dd[3]{k}}{(2\pi)^3}
		g(\vb{k})
		\expval{\hat{a}(\vb{q})\hat{a}^\dagger(\vb{k})}{0}
		(2\pi)^3\delta^{(3)}(\vb{k}-\vb{p})
		\\
		&=
		g(\vb{p})
		\expval{\hat{a}(\vb{q})\hat{a}^\dagger(\vb{p})}{0}
		\\
		&=
		g(\vb{p})
		(2\pi)^3\delta^{(3)}(\vb{q}-\vb{p})
	\end{split}
\end{equation}
and
\begin{equation}
	\begin{split}
		\expval{\hat{a}(\vb{q})\hat{A}^2\hat{a}^\dagger(\vb{p})}{0}
		&=
		\int\frac{\dd[3]{k_1}}{(2\pi)^3}
		\int\frac{\dd[3]{k_2}}{(2\pi)^3}
		g(\vb{k}_1)g(\vb{k}_2)
		\\
		&\times
		\expval{
			\hat{a}(\vb{q})
			\hat{a}^\dagger(\vb{k}_1)
			\hat{a}(\vb{k}_1)
			\hat{a}^\dagger(\vb{k}_2)
			\hat{a}(\vb{k}_2)
			\hat{a}^\dagger(\vb{p})
		}{0}
		\\
		&=
		\int\frac{\dd[3]{k_1}}{(2\pi)^3}
		\int\frac{\dd[3]{k_2}}{(2\pi)^3}
		g(\vb{k}_1)g(\vb{k}_2)
		\\
		&\times
		(2\pi)^3\delta^{(3)}(\vb{k}_1-\vb{q})
		\expval{
			\hat{a}(\vb{k}_1)
			\hat{a}^\dagger(\vb{k}_2)
		}{0}
		(2\pi)^3\delta^{(3)}(\vb{k}_2-\vb{p})
		\\
		&=
		g(\vb{q})g(\vb{p})
		\expval{
			\hat{a}(\vb{q})
			\hat{a}^\dagger(\vb{p})
		}{0}
		\\
		&=
		g(\vb{p})^2
		(2\pi)^3\delta^{(3)}(\vb{q}-\vb{p})
	\end{split}
	.
\end{equation}
The first moment then turns out to be
\begin{equation}
	\begin{split}
		\expval{\hat{A}}{f}
		&=
		\int\frac{\dd[3]{p}}{(2\pi)^3\sqrt{2\omega(\vb{p})}}
		\int\frac{\dd[3]{q}}{(2\pi)^3\sqrt{2\omega(\vb{q})}}
		f(\vb{p})f(\vb{q})^*
		\expval{\hat{a}(\vb{q})\hat{A}\hat{a}^\dagger(\vb{p})}{0}
		\\
		&=
		\int\frac{\dd[3]{p}}{(2\pi)^3\sqrt{2\omega(\vb{p})}}
		\int\frac{\dd[3]{q}}{(2\pi)^3\sqrt{2\omega(\vb{q})}}
		f(\vb{p})f(\vb{q})^*
		g(\vb{p})
		(2\pi)^3\delta^{(3)}(\vb{q}-\vb{p})
		\\
		&=
		\int\frac{\dd[3]{p}}{(2\pi)^32\omega(\vb{p})}
		g(\vb{p})
		\abs{f(\vb{p})}^2
	\end{split}
\end{equation}
and the second moment is
\begin{equation}
	\begin{split}
		\expval{\hat{A}^2}{f}
		&=
		\int\frac{\dd[3]{p}}{(2\pi)^3\sqrt{2\omega(\vb{p})}}
		\int\frac{\dd[3]{q}}{(2\pi)^3\sqrt{2\omega(\vb{q})}}
		f(\vb{p})f(\vb{q})^*
		\expval{\hat{a}(\vb{q})\hat{A}^2\hat{a}^\dagger(\vb{p})}{0}
		\\
		&=
		\int\frac{\dd[3]{p}}{(2\pi)^3\sqrt{2\omega(\vb{p})}}
		\int\frac{\dd[3]{q}}{(2\pi)^3\sqrt{2\omega(\vb{q})}}
		f(\vb{p})f(\vb{q})^*
		g(\vb{p})^2
		(2\pi)^3\delta^{(3)}(\vb{q}-\vb{p})
		\\
		&=
		\int\frac{\dd[3]{p}}{(2\pi)^32\omega(\vb{p})}
		g(\vb{p})^2
		\abs{f(\vb{p})}^2
	\end{split}
	.
\end{equation}
We can now identify $g(\vb{p})$ with $1,\omega(\vb{p}),\vb{p}$ to obtain the first two moments of number, Hamilton, or momentum operator.

\subsection{Center-of-mass position and dispersion}

The coordinate the wave function is
\begin{equation}
	\begin{split}
		\psi(t,\vb{x})
		&=
		\bra{0}
		\hat\phi(t,\vb{x})
		\ket{f}
		\\
		&=
		\left(
			\int\frac{\dd[3]{p}}{(2\pi)^3\sqrt{2\omega(\vb{p})}}
			e^{+ip_\mu x^\mu}
			\hat{a}^\dagger(\vb{p})
			\ket{0}
		\right)^\dagger
		\int\frac{\dd[3]{q}}{(2\pi)^3\sqrt{2\omega(\vb{q})}}
		f(\vb{q})
		\hat{a}^\dagger(\vb{q})
		\ket{0}
		\\
		&=
		\int\frac{\dd[3]{p}}{(2\pi)^3\sqrt{2\omega(\vb{p})}}
		e^{-ip_\mu x^\mu}
		\bra{0}
		\hat{a}(\vb{p})
		\int\frac{\dd[3]{q}}{(2\pi)^3\sqrt{2\omega(\vb{q})}}
		f(\vb{q})
		\hat{a}^\dagger(\vb{q})
		\ket{0}
		\\
		&=
		\int\frac{\dd[3]{p}}{(2\pi)^3\sqrt{2\omega(\vb{p})}}
		\int\frac{\dd[3]{q}}{(2\pi)^3\sqrt{2\omega(\vb{q})}}
		f(\vb{q})
		e^{-ip_\mu x^\mu}
		\expval{\hat{a}(\vb{p})\hat{a}^\dagger(\vb{q})}{0}
		\\
		&=
		\int\frac{\dd[3]{p}}{(2\pi)^3\sqrt{2\omega(\vb{p})}}
		\int\frac{\dd[3]{q}}{(2\pi)^3\sqrt{2\omega(\vb{q})}}
		f(\vb{q})
		e^{-ip_\mu x^\mu}
		(2\pi)^3
		\delta^{(3)}(\vb{q}-\vb{p})
		\\
		&=
		\int\frac{\dd[3]{p}}{(2\pi)^32\omega(\vb{p})}
		f(\vb{p})
		e^{-ip_\mu x^\mu}
	\end{split}
\end{equation}
the probability current is a conserved Noether current\cite[p.~18]{Peskin1995}
\begin{equation}
	\begin{split}
		j_\mu(t,\vb{x})
		&=
		i
		\bigl\{
			\psi(t,\vb{x})
			\partial_\mu
			\psi(t,\vb{x})^*
			-
			\psi(t,\vb{x})^*
			\partial_\mu
			\psi(t,\vb{x})
		\bigr\}
		=
		2
		\Im\left\{
			\psi(t,\vb{x})^*
			\partial_\mu
			\psi(t,\vb{x})
		\right\}
		\\
		&=
		2
		\Im\left\{
			\int\frac{\dd[3]{p}}{(2\pi)^32\omega(\vb{p})}
			f(\vb{p})^*e^{+ip_\mu x^\mu}
			\int\frac{\dd[3]{q}}{(2\pi)^32\omega(\vb{q})}
			f(\vb{q})iq_\mu e^{-iq_\mu x^\mu}
		\right\}
		\\
		&=
		2
		\int\frac{\dd[3]{p}}{(2\pi)^32\omega(\vb{p})}
		\int\frac{\dd[3]{q}}{(2\pi)^32\omega(\vb{q})}
		\Im\left\{
			f(\vb{p})^*
			f(\vb{q})
			iq_\mu
			e^{-i(q_\mu-p_\mu)x^\mu}
		\right\}
		\\
		&=
		2
		\int\frac{\dd[3]{p}}{(2\pi)^32\omega(\vb{p})}
		\int\frac{\dd[3]{q}}{(2\pi)^32\omega(\vb{q})}
		q_\mu
		\Re\left\{
			f(\vb{p})^*
			f(\vb{q})
			e^{-i(q_\mu-p_\mu)x^\mu}
		\right\}
		\\
		&=
		\int\frac{\dd[3]{p}}{(2\pi)^32\omega(\vb{p})}
		\int\frac{\dd[3]{q}}{(2\pi)^32\omega(\vb{q})}
		\left\{
			q_\mu
			+
			p_\mu
		\right\}
		f(\vb{p})^*
		f(\vb{q})
		e^{-i(q_\mu-p_\mu)x^\mu}
	\end{split}
\end{equation}

\subsubsection{Center-of-mass position}

The center-of-mass position can be expressed in terms of the velocity
\begin{equation}
	\begin{split}
		\overline{\vb{x}}
		&=
		\int\dd[3]{x}
		\vb{x}
		\rho(t,\vb{x})
		\\
		&=
		2\Re\left\{
			\int\dd[3]{x}
			\vb{x}
			\int\frac{\dd[3]{p}}{(2\pi)^32\omega(\vb{p})}
			\int\frac{\dd[3]{q}}{(2\pi)^32\omega(\vb{q})}
			\omega(\vb{q})
			f(\vb{q})
			f(\vb{p})^*
			e^{-i(p_\mu-q_\mu)x^\mu}
		\right\}
		\\
		&=
		2\Re\left\{
			\int\dd[3]{x}
			\vb{x}
			\int\frac{\dd[3]{p}}{(2\pi)^32\omega(\vb{p})}
			\int\frac{\dd[3]{q}}{(2\pi)^3}
			f(\vb{q})
			f(\vb{p})^*
			e^{-i(p_\mu-q_\mu)x^\mu}
		\right\}
		\\
		&=
		2\Re\left\{
			\int\frac{\dd[3]{p}}{(2\pi)^32\omega(\vb{p})}
			f(\vb{p})^*
			e^{-i\omega(\vb{p})t}
			\int\frac{\dd[3]{k}}{(2\pi)^3}
			f(\vb{p}+\vb{k})
			e^{+i\omega(\vb{k}+\vb{p})t}
			\int\dd[3]{x}
			\vb{x}
			e^{+i\vb{k}\vdot\vb{x}}
		\right\}
	\end{split}
\end{equation}
We take $g(\vb{k})=f(\vb{p}+\vb{k})e^{-i\omega(\vb{k}+\vb{p})t}$ and use partial integration with vanishing boundaries because $g(\vb{k})$ is a Schwartz function
\begin{equation}
	\begin{split}
		\int\frac{\dd[3]{k}}{(2\pi)^3}
		g(\vb{k})
		\int\dd[3]{x}
		\vb{x}
		e^{+i\vb{k}\vdot\vb{x}}
		&=
		\int\frac{\dd[3]{k}}{(2\pi)^3}
		g(\vb{k})
		\left(-i\grad_{\vb{k}}\right)
		\int\dd[3]{x}
		e^{+i\vb{k}\vdot\vb{x}}
		\\
		&=
		\int\frac{\dd[3]{k}}{(2\pi)^3}
		g(\vb{k})
		\left(-i\grad_{\vb{k}}\right)
		(2\pi)^3\delta^{(3)}(\vb{k})
		\\
		&=
		\int\dd[3]{k}
		\left(i\grad_{\vb{k}}g(\vb{k})\right)
		\delta^{(3)}(\vb{k})
		\\
		&=
		i\eval{\grad_{\vb{k}}}_{\vb{k}=0}
		f(\vb{p}+\vb{k})
		e^{+i\omega(\vb{k}+\vb{p})t}
		\\
		&=
		ie^{+i\omega(\vb{p})t}
		\left(
			\eval{\grad_{\vb{k}}}_{\vb{k}=0}
			f(\vb{p}+\vb{k})
			+
			f(\vb{p})
			\eval{\grad_{\vb{k}}}_{\vb{k}=0}
			it\omega(\vb{p}+\vb{k})
		\right)
		\\
		&=
		ie^{+i\omega(\vb{p})t}
		\left(
			\grad_{\vb{p}}
			f(\vb{p})
			+
			it
			f(\vb{p})
			\grad_{\vb{p}}
			\omega(\vb{p})
		\right)
	\end{split}
\end{equation}
Inserting this back
\begin{equation}
	\begin{split}
		\overline{\vb{x}}
		&=
		2\Re\left\{
			\int\frac{\dd[3]{p}}{(2\pi)^32\omega(\vb{p})}
			f(\vb{p})^*
			e^{-i\omega(\vb{p})t}
			ie^{+i\omega(\vb{p})t}
			\left(
				\grad_{\vb{p}}
				f(\vb{p})
				+
				it
				f(\vb{p})
				\grad_{\vb{p}}
				\omega(\vb{p})
			\right)
		\right\}
		\\
		&=
		2\Re\left\{
			\int\frac{\dd[3]{p}}{(2\pi)^32\omega(\vb{p})}
			f(\vb{p})^*
			i
			\left(
				\grad_{\vb{p}}
				f(\vb{p})
				+
				it
				f(\vb{p})
				\grad_{\vb{p}}
				\omega(\vb{p})
			\right)
		\right\}
		\\
		&=
		-
		2\Im\left\{
			\int\frac{\dd[3]{p}}{(2\pi)^32\omega(\vb{p})}
			f(\vb{p})^*
			\grad_{\vb{p}}
			f(\vb{p})
		\right\}
		-
		2
		\int\frac{\dd[3]{p}}{(2\pi)^32\omega(\vb{p})}
		\abs{f(\vb{p})}^2
		\grad_{\vb{p}}
		\omega(\vb{p})t
		\\
		&=
		2\Im\left\{
			\int\frac{\dd[3]{p}}{(2\pi)^3}
			f(\vb{p})
			\grad_{\vb{p}}
			\frac{f(\vb{p})^*}{2\omega(\vb{p})}
		\right\}
		-
		2
		\int\frac{\dd[3]{p}}{(2\pi)^32\omega(\vb{p})}
		\abs{f(\vb{p})}^2
		\grad_{\vb{p}}
		\omega(\vb{p})t
		\\
		&=
		2\Im\left\{
			\int\frac{\dd[3]{p}}{(2\pi)^3}
			f(\vb{p})
			\frac{
				\omega(\vb{p})
				\grad_{\vb{p}}
				f(\vb{p})^*
				-
				f(\vb{p})^*
				\grad_{\vb{p}}
				\omega(\vb{p})
			}{2\omega(\vb{p})^2}
		\right\}
		-
		2
		\int\frac{\dd[3]{p}}{(2\pi)^32\omega(\vb{p})}
		\abs{f(\vb{p})}^2
		\grad_{\vb{p}}
		\omega(\vb{p})t
	\end{split}
\end{equation}

If we want
\begin{equation}
	\overline{\vb{x}}
	=
	\int\frac{\dd[3]{p}}{(2\pi)^32\omega(\vb{p})}
	\frac{\vb{p}}{\omega(\vb{p})}
	\abs{f(\vb{p})}^2
	t
	=
	\overline{\vb{v}}t
\end{equation}
we need
\begin{equation}
			\int\frac{\dd[3]{k}}{(2\pi)^3}
			f(\vb{p}+\vb{k})
			e^{-i\omega(\vb{k}+\vb{p})t}
			\int\dd[3]{x}
			\vb{x}
			e^{+i\vb{k}\vdot\vb{x}}
			=
			\frac{\vb{p}}{\omega(\vb{p})}f(\vb{p})t	
\end{equation}

\section{Coherent state}

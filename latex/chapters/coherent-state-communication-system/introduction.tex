The present chapter describes a coherent state communication system from a signal-processing point of view.
We discuss concepts common to signal-processing, extend the concepts to quantum signals whenever necessary, and give references to the implementation details of the previous chapter.
In the next chapter, the coherent state communication system is extended to a \gls{cvqkd} communication system by choosing a random sequence of complex symbols, adding classical post-processing, and discussing security aspects.
% TODO: cite Hans' paper where transmitter and receiver configuration is described

\begin{figure}[htb]
	\centering
	\includestandalone[mode=buildnew]{figures/tikz/communication-system}
	\caption{Flow diagram of an abstract communication system delivering information encoded in a sequence of symbols from the transmitter to the receiver over a physical channel.}\label{fig:communication_system}
\end{figure}
\Cref{fig:communication_system} depicts the flow diagram of an abstract communication system.
A sequence of complex symbols $\alpha_1,\dots,\alpha_n\in\mathbb{C}$ is streamed to a transmitter.
The transmitter converts the symbol stream to an analog signal $\alpha(t)$ and modulates it onto a carrier $\alpha(t)e^{-i\omega_0t}$.
The signal leaves the channel as $\alpha^\prime(t)e^{-i\omega_0t}$.
Finally, the receiver removes the carrier and recovers the symbols stream from $\alpha^\prime(t)$.

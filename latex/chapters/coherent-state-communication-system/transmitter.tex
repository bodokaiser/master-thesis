\section{Transmitter}

The transmitter takes a stream of complex symbols $\alpha_1,\dots,\alpha_n$ and outputs a quantum signal $\alpha^\prime(t)e^{-i\omega_0t}$.

\begin{figure}[htb]
	\centering
	\includestandalone{figures/tikz/transmitter}
	\caption{Flow diagram of the transmitter converting a sequence of complex symbols to a coherent state signal.}\label{fig:transmitter}
\end{figure}
\Cref{fig:transmitter} depicts the process steps of the transmitter.
First, an analog signal $\alpha(t)$ is reconstructed from the symbols.
Second, the analog signal $\alpha(t)$ is modulated onto a carrier with optical frequency $\omega_0$.
Finally, a variable-optical attenuator reduces the signal strength down to the quantum regime.

\subsection{Pulse-shaping}

Let $\alpha[n]\in\mathbb{C}$ be a complex sample, then we define the corresponding real samples
\begin{align}
	x[n]=\Re{\alpha[n]}
	&&
	p[n]=\Im{\alpha[n]}
\end{align}
through the real and imaginary part.
The signal reconstruction is performed independent for the real symbol sequences.
\begin{figure}[htb]
	\centering
	\includestandalone[mode=buildnew]{figures/tikz/pulse-shaping}
	\caption{Flow diagram of the signal reconstruction where an analog signal $x(t)$ is reconstructed from a stream of samples.}\label{fig:transmitter}
\end{figure}

% TODO: analog <-> digital (mathematical)
% TODO: samples -> upsampling -> pulse-shaping

\subsection{Up-conversion}

Let $x(t),p(t)$ be the analog signals reconstructed from the real respective imaginary parts of the complex symbol sequence, then up-conversion multiplies these signals such that the output signal is~\cite[p.~25]{Madhow2008}
\begin{equation}
	\Re\left\{
		\alpha(t)
		e^{-i\omega_0t}
	\right\}
	=
	x(t)
	\cos(\omega_0t)
	+
	p(t)
	\sin(\omega_0t)
\end{equation}
and one refers to $x(t)$ as the in-phase and $p(t)$ as the quadrature component of $\alpha(t)$.

\Cref{fig:iqmod} shows a diagram of an I/Q-modulator which can be used to obtain the previous equation.
\begin{figure}[htb]
	\centering
	\includestandalone{figures/circuitikz/iq-modulator}
	\caption{Diagram of an in-phase and quadrature modulator which can be used for signal up-conversion.}\label{fig:iqmod}
\end{figure}
The output of a sinusoidal oscillator with frequency $\omega_0$ is split into two paths.
The first path is mixed with the the $p(t)$ signal while the second path is phase-shifted by \SI{90}{\degree} and mixed with the $x(t)$ signal.
Finally, the outputs of both mixers are combined to obtain the real part of the complex baseband signal
\begin{equation}
	\alpha(t)
	=
	x(t)
	+
	ip(t)
	.
\end{equation}
One often works with the complex signal as up-conversion takes the simple form of a multiplication.
\begin{figure}[htb]
	\centering
	\includestandalone{figures/circuitikz/up-converter}
	\caption{For a complex signal $\alpha(t)$ up-conversion is simply the multiplication with $e^{-i\omega_0t}$.}
\end{figure}
However, one should keep in mind that the true physical signal is real-valued, thus only the real-part of the complex signal has physical relevance!

\subsection{Attenuation}

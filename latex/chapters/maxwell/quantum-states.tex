\section{Quantum states}

\begin{corollary}\label{thm:qmw_qkg}
	Let $\vb{f}$ be a smearing vector function, then the smeared positive frequency transverse Maxwell operator
	\begin{equation}
		\hat{\vb{A}}_\perp^+[\vb{f}]
		=
		\sum_{\lambda=1,2}
		\int\frac{\dd[3]{p}}{(2\pi)^3\sqrt{2\omega(\vb{p})}}
		f_\lambda\left(\omega(\vb{p}),\vb{p}\right)^*
		\hat{a}_\lambda^\dagger(\vb{p})
		=
		\hat\phi_1^+[f_1]
		+
		\hat\phi_2^+[f_2]
	\end{equation}
	 reduces to the sum of two independent smeared positive frequency Klein-Gordon operator.
\end{corollary}
\begin{restatable}{lemma}{qmwqkgnumberstate}\label{thm:qmw_qkg_number_state}
	The transverse Maxwell fields' $n$-particle number state with smearing vector function $\vb{f}$ is
	\begin{equation}
		\label{eq:qmw_number_state}
		\ket{n_{\vb{f}}}
		=
		\frac{1}{\sqrt{n!}}
		\hat{\vb{A}}_\perp^+[\vb{f}]^n
		\ket{0}
	\end{equation}
	where the smearing vector function is required to satisfy~\cite[p.~175]{Itzykson2012}
	\begin{equation}
		\int\frac{\dd[3]{p}}{(2\pi)^32\omega(\vb{p})}
		\norm{\vb{f}\left(\omega(\vb{p}),\vb{p}\right)}^2
		=
		\sum_{\lambda=1,2}
		\int\frac{\dd[3]{p}}{(2\pi)^32\omega(\vb{p})}
		\abs{f_\lambda\left(\omega(\vb{p}),\vb{p}\right)}^2
		=
		1
		.
	\end{equation}
\end{restatable}
\begin{corollary}
	Using \Cref{thm:qmw_qkg}, we can write the transverse Maxwell $n$-particle state as the tensor product of $n$-particle Klein-Gordon states
	\begin{equation}
		\ket{n_{\vb{f}}}
		=
		\sum^n_{m=0}
		\binom{n}{m}^{1/2}
		\ket{m_{f_1},n-m_{f_2}}
		.
	\end{equation}
\end{corollary}
\begin{restatable}{lemma}{qmwqkgnumberstateinnerproduct}\label{thm:qmw_qkg_number_state_inner_product}
	Let $\ket{n_{\vb{f}}}$ be an $n$-particle number state with spectrum $\vb{f}$, and $\ket{m_{\vb{g}}}$ an $m$-particle number state with spectrum $\vb{g}$, then their inner product is
	\begin{equation}
		\braket{n_{\vb{f}}}{m_{\vb{g}}}
		=
		\delta_{n,m}
		\left[
			\braket{1_{f_1}}{1_{g_1}}
			+
			\braket{1_{f_2}}{1_{g_2}}
		\right]^n
		.
	\end{equation}
\end{restatable}
\begin{example}
	Let $\vb{f}$ be a smearing vector function with transverse components
	\begin{align*}
		f_1(p_0,\vb{p})
		=
		\sqrt{2\omega(\vb{p}_0)}
		\left(\frac{2\pi}{\sigma^2}\right)^{\frac{3}{4}}
		\exp\left\{
			-
			\frac{(\vb{p}-\vb{p}_0)^2}{4\sigma^2}
		\right\}
		&&
		f_2(p_0,\vb{p})
		=
		0
		.
	\end{align*}
	The smearing vector function is normalized
	\begin{equation*}
		\begin{split}
			\int\frac{\dd[3]{p}}{(2\pi)^32\omega(\vb{p})}
			\norm{\vb{f}\left(\omega(\vb{p}),\vb{p}\right)}^2
			&=
			2\omega(\vb{p}_0)
			\left(\frac{2\pi}{\sigma^2}\right)^{\frac{3}{2}}
			\int\frac{\dd[3]{p}}{(2\pi)^32\omega(\vb{p})}
			\exp\left\{
				-
				\frac{(\vb{p}-\vb{p}_0)^2}{2\sigma^2}
			\right\}
			\\
			&=
			\left(\frac{2\pi}{\sigma^2}\right)^{\frac{3}{2}}
			\left(
				\int\frac{\dd[3]{p}}{2\pi}
				\exp\left\{
					-
					\frac{(p-p_0)^2}{2\sigma^2}
				\right\}
			\right)^3
			\\
			&=
			\left(\frac{2\pi}{\sigma^2}\right)^{\frac{3}{2}}
			\left(
				\frac{1}{2\pi}
				\sqrt{2\pi\sigma^2}
			\right)^3
			\\
			&=
			\left(\frac{2\pi}{\sigma^2}\right)^{\frac{3}{2}}
			\left(
				\frac{\sigma^2}{2\pi}
			\right)^{\frac{3}{2}}
			=
			1
		\end{split}
	\end{equation*}
	where we used the mean value theorem in the second line.
	The coordinate representation of the smearing function is
	\begin{equation*}
		\begin{split}
			f_1(t,\vb{x})
			&=
			\int\frac{\dd[4]{p}}{(2\pi)^4}
			f(p_0,\vb{p})
			e^{+ip_\mu x^\mu}
			\\
			&=
			\sqrt{2\omega(\vb{p}_0)}
			\left(\frac{2\pi}{\sigma^2}\right)^{\frac{3}{4}}
			\int\frac{\dd[4]{p}}{(2\pi)^4}
			\exp\left\{
				-
				\frac{(\vb{p}-\vb{p}_0)^2}{4\sigma^2}
			\right\}
			e^{+i(p_0t-\vb{p}\vdot\vb{x})}
			\\
			&=
			\sqrt{2\omega(\vb{p}_0)}
			\left(\frac{2\pi}{\sigma^2}\right)^{\frac{3}{4}}
			\int\frac{\dd{p_0}}{2\pi}
			e^{-ip_0t}
			\int\frac{\dd[3]{p}}{(2\pi)^3}
			\exp\left\{
				-
				\frac{(\vb{p}-\vb{p}_0)^2}{4\sigma^2}
				-
				i\vb{p}\vdot\vb{x}
			\right\}
			\\
			&=
			\sqrt{2\omega(\vb{p}_0)}
			\left(\frac{2\pi}{\sigma^2}\right)^{\frac{3}{4}}
			\delta(t)
			e^{-i\vb{p}_0\vdot\vb{x}}
			\left(
				\int\frac{\dd{q}}{2\pi}
				\exp\left\{
					-
					\frac{q^2+2q(2i\sigma^2x)+(2i\sigma^2x)^2-(2i\sigma^2x)^2}{4\sigma^2}
				\right\}
			\right)^3
			\\
			&=
			\sqrt{2\omega(\vb{p}_0)}
			\left(\frac{2\pi}{\sigma^2}\right)^{\frac{3}{4}}
			\delta(t)
			e^{-i\vb{p}_0\vdot\vb{x}}
			e^{-(\sigma\vb{x})^2}
			\left(
				\int\frac{\dd{q}}{2\pi}
				\exp\left\{
					-
					\frac{(q+i\sigma^2x)^2}{4\sigma^2}
				\right\}
			\right)^3
			\\
			&=
			\sqrt{2\omega(\vb{p}_0)}
			\left(\frac{2\pi}{\sigma^2}\right)^{\frac{3}{4}}
			\delta(t)
			e^{-i\vb{p}_0\vdot\vb{x}}
			e^{-(\sigma\vb{x})^2}
			\left(
				\frac{1}{2\pi}
				\sqrt{4\pi\sigma^2}
			\right)^3
			\\
			&=
			\sqrt{2\omega(\vb{p}_0)}
			\left(\frac{2\pi}{\sigma^2}\right)^{\frac{3}{4}}
			\delta(t)
			e^{-i\vb{p}_0\vdot\vb{x}}
			e^{-(\sigma\vb{x})^2}
			\left(
				\frac{\sigma^2}{\pi}
			\right)^\frac{3}{2}
			\\
			&=
			\sqrt{2\omega(\vb{p}_0)}
			\left(\frac{2\sigma^2}{\pi}\right)^{\frac{3}{4}}
			\delta(t)
			e^{-i\vb{p}_0\vdot\vb{x}}
			e^{-(\sigma\vb{x})^2}
			,
		\end{split}
	\end{equation*}
	hence, at time $t=0$, the wave packet is localized at $\vb{x}=0$ with spatial spread $\sigma$.
	Thereby the wave packet is perfectly localized in time, equivalent with knowing the momentum distribution of the wave packet at $t=0$ or $f_1(0,\vb{x})$.
	However, we know that there must be a time-frequency uncertainty.
	We find the time-frequency uncertainty in the coordinate representation of the wave function
	\begin{equation*}
		\begin{split}
			\psi_1(t,\vb{x})
			&=
			\int\frac{\dd[3]{p}}{(2\pi)^32p_0}
			\eval{
				f(p_0,\vb{p})^*
				e^{+ip_\mu x^\mu}
			}_{p_0=\omega(\vb{p})}
			\\
			&=
			2\omega(\vb{p}_0)
			\left(\frac{2\pi}{\sigma^2}\right)^{\frac{3}{2}}
			\int\frac{\dd[3]{p}}{(2\pi)^32\omega(\vb{p})}
			\exp\left\{
				-
				\frac{(\vb{p}-\vb{p}_0)^2}{4\sigma^2}
			\right\}
			e^{+i\omega(\vb{p})t-i\vb{p}\vdot\vb{x}}
			\\
			&=
			\left(\frac{2\pi}{\sigma^2}\right)^{\frac{3}{2}}
			\int\frac{\dd[3]{p}}{(2\pi)^3}
			\exp\left\{
				-
				\frac{(\vb{p}-\vb{p}_0)^2}{4\sigma^2}
				-
				i\vb{p}\vdot\vb{x}
				+
				i\vb{p}^2t
			\right\}
			\\
			&=
			\left(\frac{2\pi}{\sigma^2}\right)^{\frac{3}{2}}
			\int\frac{\dd[3]{q}}{(2\pi)^3}
			\exp\left\{
				-
				\frac{\vb{q}^2}{4\sigma^2}
				-
				i(\vb{q}+\vb{p}_0)\vdot\vb{x}
				+
				i(\vb{q}+\vb{p}_0)^2t
			\right\}
			\\
			&=
			\left(\frac{2\pi}{\sigma^2}\right)^{\frac{3}{2}}
			e^{i\omega(\vb{p}_0)t-i\vb{p}_0\vdot\vb{x}}
			\int\frac{\dd[3]{q}}{(2\pi)^3}
			\exp\left\{
				-
				\frac{\vb{q}^2}{4\sigma^2}
				+
				i\vb{q}^2t
				+
				2i\vb{q}\vdot\vb{p}_0t
				-
				i\vb{q}\vdot\vb{x}
			\right\}
			\\
			&=
			\left(\frac{2\pi}{\sigma^2}\right)^{\frac{3}{2}}
			e^{i\omega(\vb{p}_0)t-i\vb{p}_0\vdot\vb{x}}
			\int\frac{\dd[3]{q}}{(2\pi)^3}
			\exp\left\{
				-
				\frac{
					(1-4i\sigma^2t)\vb{q}^2
					+
					22i(\sigma^2\vb{x}-2\sigma^2\vb{p}_0t)
					\vdot
					\vb{q}
				}{4\sigma^2}
			\right\}
			\\
			&=
			\left(\frac{2\pi}{\sigma^2}\right)^{\frac{3}{2}}
			\frac{e^{i\omega(\vb{p}_0)t-i\vb{p}_0\vdot\vb{x}}}{(1-4i\sigma^2t)^\frac{3}{2}}
			\int\frac{\dd[3]{v}}{(2\pi)^3}
			\exp\left\{
				-
				\frac{
					\vb{v}^2
					+
					2
					(2i\vb{b})
					\vdot
					\vb{v}
				}{4\sigma^2}
			\right\}
			\\
			&=
			\left(\frac{2\pi}{\sigma^2}\right)^{\frac{3}{2}}
			\frac{e^{i\omega(\vb{p}_0)t-i\vb{p}_0\vdot\vb{x}}}{(1-4i\sigma^2t)^\frac{3}{2}}
			\int\frac{\dd[3]{v}}{(2\pi)^3}
			\exp\left\{
				-
				\frac{
					\left(
						\vb{v}
						+
						2i\vb{b}
					\right)^2
					-
					(2i\vb{b})^2
				}{4\sigma^2}
			\right\}
			\\
			&=
			\left(\frac{2\pi}{\sigma^2}\right)^{\frac{3}{2}}
			\frac{e^{i\omega(\vb{p}_0)t-i\vb{p}_0\vdot\vb{x}}}{(1-4i\sigma^2t)^\frac{3}{2}}
			e^{-\vb{b}^2}
			\int\frac{\dd[3]{u}}{(2\pi)^3}
			\exp\left\{
				-
				\frac{\vb{u}^2}{4\sigma^2}
			\right\}
			\\
			&=
			\left(\frac{2\pi}{\sigma^2}\right)^{\frac{3}{2}}
			\frac{e^{i\omega(\vb{p}_0)t-i\vb{p}_0\vdot\vb{x}}}{(1-4i\sigma^2t)^\frac{3}{2}}
			e^{-\vb{b}^2}
			\left(\frac{\sigma^2}{\pi}\right)^\frac{3}{2}
			\\
			&=
			\left(2\right)^{\frac{3}{2}}
			\frac{e^{i\omega(\vb{p}_0)t-i\vb{p}_0\vdot\vb{x}}}{(1-4i\sigma^2t)^\frac{3}{2}}
			\exp\left\{
				-
				\sigma^4
				\frac{(\vb{x}-2\vb{p}_0t)^2}{1-4i\sigma^2 t}
			\right\}
		\end{split}
	\end{equation*}
	where we have
	\begin{equation*}
		\vb{b}
		=
		\sigma^2
		\frac{\vb{x}-2\vb{p}_0t}{\sqrt{1-4i\sigma^2t}}
		.
	\end{equation*}
	Comparing the smearing and wave function at $t=0$, we find
	\begin{align*}
		f_1(0,\vb{x})
		&=
		\sqrt{2\omega(\vb{p}_0)}
		\left(\frac{2\sigma^2}{\pi}\right)^\frac{3}{2}
		e^{-i\vb{p}_0\vdot\vb{x}}
		e^{-(\sigma^2\vb{x})^2}
		\\
		\psi_1(0,\vb{x})
		&=
		2^\frac{3}{4}
		e^{-i\vb{p}_0\vdot\vb{x}}
		e^{-(\sigma^2\vb{x})^2}
	\end{align*}
	and we conclude that
	\begin{equation*}
		f_1(0,\vb{x})
		=
		\sqrt{2\omega(\vb{p}_0)}
		\left(\frac{\sigma^2}{\pi}\right)^\frac{3}{4}
		\psi_1(0,\vb{x})
	\end{equation*}
	or just $f_1(0,\vb{x})\propto\psi_1(0,\vb{x})$ meaning that the smearing function is actually the (re-scaled) wave function at some initial time $t=0$ and the wave function has built-in time uncertainty.
\end{example}

\begin{definition}[Coherent state]
	The coherent state with spectrum $\vb{\alpha}(p_0,\vb{p})$ is
	\begin{equation}
		\label{eq:qmw_coherent_state}
		\ket{\vb{\alpha}}
		=
		\exp\left\{
			-
			\frac{1}{2}
			\comm{\hat{\vb{A}}_\perp^-[\vb{\alpha}]}{\hat{\vb{A}}_\perp^+[\vb{\alpha}]}
		\right\}
		\exp\left\{
			\hat{\vb{A}}_\perp^+[\vb{\alpha}]
		\right\}
		\ket{0}
		.
	\end{equation}
\end{definition}
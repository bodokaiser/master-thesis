\section{Comparison with quantum optics}

\begin{lemma}
	By absorbing $1/\sqrt{2\omega(\vb{p})}$ into the definition of the spectrum $f(\vb{p})$, the number state simplifies to
	\begin{equation}
		\ket{n}
		=
		\frac{1}{n!}
		\left(
			\int\frac{\dd[3]{p}}{(2\pi)^3}
			f(\vb{p})
			\hat{a}^\dagger(\vb{p})
		\right)^n
		\ket{0}
	\end{equation}
	and the coherent state simplifies to
	\begin{equation}
		\ket{\beta}
		=
		e^{-\overline{n}/2}
		\exp\left\{
			\int\frac{\dd[3]{p}}{(2\pi)^3}
			\beta(\vb{p})
			\hat{a}^\dagger(\vb{p})
		\right\}
		\ket{0}
	\end{equation}
	in exchange for manifest Lorentz-invariance.
\end{lemma}
\begin{theorem}
	The spectrum expanded in spherical harmonics is
	\begin{equation}
		f(\vb{p})
		=
		f(\norm{\vb{p}},\theta,\varphi)
	\end{equation}
\end{theorem}
\begin{proof}
	The spectrum satisfies the Klein-Gordon equation.
	In the massless case, the Klein-Gordon equation is equal to the relativistic wave equation
	\begin{equation}
		0
		=
		\partial_t^2
		f(t,\vb{x})
		-
		\laplacian
		f(t,\vb{x})
		.
	\end{equation}
	According to Ref.~\cite[p.~538]{Jackson2007}, we then can write
	\begin{equation}
		f(t,\vb{x})
		=
		\sum_{l=0}^\infty
		\sum_{m=-l}^l
		\int\dd{\omega}
		f_l(\omega\norm{\vb{x}})
		Y_l^m(\theta,\varphi)
		e^{-i\omega t}
	\end{equation}
\end{proof}

In the previous sections we derived the number state which is an eigenstate of the number operator $\hat{N}$
\begin{align}
	\ket{n}
	&=
	\frac{1}{n!}
	\left(
		\int\frac{\dd[3]{p}}{(2\pi)^3\sqrt{2\omega(\vb{p})}}
		f(\vb{p})
		\hat{a}^\dagger(\vb{p})
	\right)^n
	\ket{0}
	\\
	\ket{\alpha}
	&=
	e^{-\overline{n}/2}
	\exp\left\{
		\int\frac{\dd[3]{p}}{(2\pi)^3\sqrt{2\omega(\vb{p})}}
		\alpha(\vb{p})
		\hat{a}^\dagger(\vb{p})
	\right\}
	\ket{0}
\end{align}
and the coherent state which is an eigenstate of the annihilation operator.
For the number observable of these two states, we found
\begin{align}
	\expval{\hat{N}}{n}
	=
	n
	&&
	\expval{(\Delta\hat{N})^2}{n}
	=
	0
	\\
	\expval{\hat{N}}{\alpha}
	=
	\overline{n}
	&&
	\expval{(\Delta\hat{N})^2}{n}
	=
	\overline{n}
\end{align}
while for the energy observable, we found.

Our definitions of the number and coherent state are manifest Lorentz-covariant as the integral measure and the integrand are both Lorentz-invariant.
However, our measurements are not Lorentz-invariant, i.e., our spectrum analyzer measures $\beta(\vb{p})=\alpha(\vb{p})/\sqrt{2\omega(\vb{p})}$ and not $\alpha(\vb{p})$ for the spectrum.
Therefore, we can redefine the spectrum and use

without loss of generality.\footnote{We kept the Lorentz-invariant measure and spectrum throughout the chapter to compare the results directly with the literature.}

Another sensitive approximation to make is to transform to spherical coordinates where the $z$ axis is along the propagation direction, i.e., where $f(\vb{p})$ is maximal, then we write $f(\norm{\vb{p}},\theta,\varphi)$.
It is also sensitive, to assume that the spectrum is symmetric around the propagation axis, hence $f(\norm{\vb{p}},\theta)$, then we can write
\begin{align}
	\ket{n}
	&=
	\frac{1}{n!}
	\left(
		\int\frac{\dd[3]{p}}{(2\pi)^3}
		g(\norm{\vb{p}},\theta)
		\hat{a}^\dagger(\vb{p})
	\right)^n
	\ket{0}
	\\
	\ket{\beta}
	&=
	e^{-\overline{n}/2}
	\exp\left\{
		\int\frac{\dd[3]{p}}{(2\pi)^3}
		\beta(\norm{\vb{p}},\theta)
		\hat{a}^\dagger(\vb{p})
	\right\}
	\ket{0}
\end{align}
By the 
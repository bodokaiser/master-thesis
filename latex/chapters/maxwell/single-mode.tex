\section{Approximation and comparison}

\subsection{Continuous-mode}

Barnett defines the single-particle number state~\cite[p.~51]{Barnett2002}
\begin{align}
	\int\dd{\omega}
	f(\omega)
	\hat{a}^\dagger(\omega)
	\ket{0}
	&&
	\text{with}\
	\int\dd{\omega}
	\abs{f(\omega)}^2
	=
	1
\end{align}
and the coherent state~\cite[p.~65]{Barnett2002}
\begin{equation}
	\exp\left\{
		\int\dd{\omega}
		\left[
			\alpha(\omega)
			\hat{a}^\dagger(\omega)
			-
			\alpha(\omega)^*
			\hat{a}(\omega)
		\right]
	\right\}
	\ket{0}
	.
\end{equation}
Our formalism reproduces the states proposed by Barnett for
\begin{equation}
	g_\lambda(\vb{p})
	=
	\sqrt{2\omega(\vb{p})}
	(2\pi)\delta(p_1)
	(2\pi)\delta(p_2)
	f(p_3)
	\delta_{\lambda1}
\end{equation}
and defining $\hat{a}^\dagger(\omega)=\hat{a}_1^\dagger(0,0,p_3),\omega=p_3/(2\pi)$.
We conclude that Barnett's continuous-mode formalism is obtained under the assumptions that we consider
\begin{enumerate}
	\item a single polarization mode $\lambda=1$,
	\item a single propagation direction $p_3$,
	\item a non-Lorentz-invariant quantities $\sqrt{2\omega(\vb{p})}$.
\end{enumerate}
While the first and third assumptions appear to be sensible for any practical considerations, the second assumption fails to describe any angular momentum distribution.
Furthermore, keeping $\omega(\vb{p})$ allows for semi-classical models of non-linear dispersion as known from waveguides.

\subsection{Single-mode}

Alternatively, quantum optics imposes the dipole approximation
\begin{equation}
	\frac{\lambda}{2\pi}
	=
	\frac{1}{\norm{\vb{p}}}
	\gg
	\vb{x}
\end{equation}
where $\vb{x}$ has magnitude of the interacting object, e.g., an atom.
Then, we have
\begin{equation*}
	e^{\pm i\vb{p}\vdot\vb{x}}
	\approx
	1
	\pm
	i\vb{p}\vdot\vb{x}
\end{equation*}
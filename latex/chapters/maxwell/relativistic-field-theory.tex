\section{Relativistic field theory}

\begin{definition}[Maxwell Lagrangian]
	The Lagrangian of the Maxwell field $A^\mu(x^\mu)$ interacting with an external source $J^\mu(x^\mu)$ is~\cite[p.~339]{Srednicki2007}.
	\begin{equation}
		\mathcal{L}
		=
		\frac{1}{2}
		(\partial_\mu A_\nu)
		\left(
			\partial^\nu A^\mu
			-
			\partial^\mu A^\nu
		\right)
		-
		J_\mu A^\mu
		\label{eq:mw_lagrangian}
		.
	\end{equation}
\end{definition}
\begin{definition}[Field-strength tensor]
	The Maxwell's field covariant field-strength tensor and its dual are
	\begin{align}
		F^{\mu\nu}
		=
		\partial^\mu A^\nu
		-
		\partial^\nu A^\mu
		&&
		\tilde{F}^{\mu\nu}
		=
		\frac{1}{2}
		\varepsilon^{\mu\nu\alpha\beta}
		F_{\alpha\beta}
		\label{eq:mw_field_strength_tensors}
	\end{align}
	where $\varepsilon^{\mu\nu\alpha\beta}$ is the generalized antisymmetric tensor.
	The components of the field-strength tensor are
	\begin{align}
		F^{0i}
		=
		E^i
		&&
		F^{ij}
		=
		\varepsilon^{ijk}B_k
		\label{eq:mw_field_strength_components}
	\end{align}
	where $E^i$ and $B_k$ are electric respective magnetic field components~\cite[p.~20]{Carroll1997}.
\end{definition}
\begin{corollary}
	The field-strength tensors are antisymmetric
	\begin{align}
		F^{\mu\nu}
		=
		-
		F^{\nu\mu}
		&&
		\tilde{F}^{\mu\nu}
		=
		-
		\tilde{F}^{\nu\mu}
	\end{align}
	implying vanishing diagonal components $F^{\mu\mu}=0=\tilde{F}^{\mu\mu}$.
\end{corollary}
\begin{restatable}{lemma}{mwfieldstrengthlagrangian}\label{thm:mw_field_strength_lagrangian}
	Using the field-strength tensor $F^{\mu\nu}$, the Maxwell Lagrangian becomes
	\begin{equation}
		\mathcal{L}
		=
		-
		\frac{1}{4}
		F_{\mu\nu}
		F^{\mu\nu}
		-
		J_\mu A^\mu
		\label{eq:mw_field_strength_lagrangian}
	\end{equation}
\end{restatable}
\begin{restatable}{lemma}{mweom}\label{thm:mw_eom}
	The dynamics of the Maxwell field are governed by the equations of motion
	\begin{equation}
		J^\mu
		=
		\partial_\mu
		F^{\mu\nu}
		\label{eq:mw_eom}
		.
	\end{equation}
\end{restatable}
\begin{restatable}{lemma}{mwfieldstrengthcontracted}\label{thm:mw_field_strength_contracted}
	Contracting the field-strength tensor and its dual yields
		\begin{align}
		F_{\mu\nu}
		F^{\mu\nu}
		=
		-2
		\left(
			\vb{E}^2
			-
			\vb{B}^2
		\right)
		&&
		\tilde{F}_{\mu\nu}
		F^{\mu\nu}
		=
		-
		4
		\vb{E}
		\vdot
		\vb{B}
		.
	\end{align}
\end{restatable}
\begin{corollary}
	Using \Cref{thm:mw_field_strength_contracted}, we can write the Maxwell Lagrangian in terms of the electromagnetic fields
	\begin{equation}
		\mathcal{L}
		=
		\frac{1}{2}
		\left(
			\vb{E}^2
			-
			\vb{B}^2
		\right)
		-
		\rho A^0
		+
		\vb{J}\vdot\vb{A}
	\end{equation}
	where $\rho$ is a charge and $\vb{J}$ a current density which together form the relativistic current $J^\mu(\rho,\vb{J})$.
\end{corollary}

\begin{restatable}{theorem}{mweqtensor}\label{thm:tensor_maxwell}
	The manifest Lorentz-covariant Maxwell equations are
	\begin{align}
		0
		=
		\partial_\mu
		\tilde{F}^{\mu\nu}
		&&
		J^\nu
		=
		\partial_\mu
		F^{\mu\nu}
		\label{eq:tensor_maxwell}.
	\end{align}
\end{restatable}
\begin{restatable}{theorem}{mweqvector}\label{thm:vector_maxwell}
	The non-covariant Maxwell equations are
	\begin{align}
		\div\vb{E}
		=
		\rho
		&&
		\div\vb{B}
		=
		0
		\label{eq:vector_maxwell_homo}
		\\
		\curl\vb{E}
		=
		-
		\partial_t\vb{B}
		&&
		\curl\vb{B}
		=
		\vb{J}
		+
		\partial_t\vb{E}
		\label{eq:vector_maxwell_inhomo}
	\end{align}
\end{restatable}

\begin{definition}[Local gauge transformation]
	A local gauge transformation
	\begin{equation}
		A_\mu(t,\vb{x})
		\to
		A_\mu^\prime(t,\vb{x})
		=
		A_\mu(t,\vb{x})
		+
		\partial_\mu\Lambda(t,\vb{x})
		\label{eq:mw_local_gauge_transform}
	\end{equation}
	adds the gradient of a scalar field to a (vector) field.\footnote{In a strict sense, a local gauge transformation is defined with respect to the Dirac field and \cref{eq:mw_local_gauge_transform} is a consequence that the Dirac combined with the Maxwell Lagrangian should be invariant under such transformations.}
\end{definition}
\begin{restatable}{theorem}{mwlogalgaugeinvariance}\label{thm:mw_local_gauge_invariance}
	The physical observables of the Maxwell field are invariant under local gauge transformations.
\end{restatable}
Unphysical degrees of freedom of the Maxwell field can be removed by choosing a specific scalar field $\Lambda(t,\vb{x})$ and performing a local gauge transformation.
The procedure is known as gauge fixing or imposing a gauge and the resulting restriction on the field is known as gauge condition.
Typical gauge conditions are:
\begin{definition}[Lorenz gauge]
	The Lorentz gauge,
	\begin{equation}
		\partial_\mu
		A^\mu
		=
		0
		,
	\end{equation}
	where the four-gradient of the Maxwell field is zero.
\end{definition}
The Lorenz gauge is manifestly Lorentz-covariant and calculations in the Lorenz gauge are valid in any reference frame.\footnote{See, for instance, Ref.~\cite[p.~144]{Greiner2013}.}
However, a drawback of the Lorentz gauge is that the quantization procedure becomes more elaborate giving rise to scalar and longitudinal polarization vectors which have to be removed by projecting out unphysical states of the Hilbert space, see Gupta-Beuler quantization in Ref.~\cite[p.~180]{Greiner2013}.
\begin{definition}[Coulomb gauge]
	In the Coulomb gauge, the three-gradient of the Maxwell field is zero, $\partial_iA^i=\div\vb{A}=0$.
\end{definition}
In contrast to the Lorentz gauge, the Coulomb gauge removes unphysical longitudinal and scalar degrees of freedom from the start by making the Maxwell field transverse at the cost of manifest Lorentz-covariance.
\begin{definition}[Temporal gauge]
	In the temporal gauge, the temporal component of the Maxwell field is zero, $A_0=0$.
\end{definition}
The Coulomb gauge leaves a residual gauge freedom which is further removed by the temporal gauge.
The temporal removes Coulomb interactions, i.e., interactions with a static charge distribution~\cite[p.~200]{Greiner2013}.

Quantum optical communication is described in the rest frame of the communication system and it appears natural to renounce manifest Lorentz-covariance for a simplified quantization procedure.
\begin{restatable}{lemma}{mwcoulombfixing}
	Choosing the local gauge field $\Lambda$ such that
	\begin{align}
		\laplacian
		\Lambda
		=
		-
		\div\vb{A}
		&&
		\partial_t
		\Lambda
		=
		-
		A_0
	\end{align}
	implements the Coulomb and temporal gauge.
\end{restatable}
\begin{restatable}{lemma}{mwcoulombtransverse}
	In the Coulomb gauge, the Maxwell field becomes transverse
	\begin{equation}
		\vb{A}
		\to
		P_\perp
		\vb{A}
		=
		\vb{A}_\perp
		.
	\end{equation}
\end{restatable}
\begin{restatable}{lemma}{mwcoulombeom}\label{thm:mw_coulomb_eom}
	The Maxwell field in the Coulomb gauge has the equation of motion becomes
	\begin{equation}
		\vb{J}_\perp
		=
		\partial_\mu \partial^\mu
		\vb{A}_\perp
		=
		\left(
			\partial_t^2
			-
			\laplacian
		\right)
		\vb{A}_\perp
		\label{eq:mw_coulomb_eom}
		.
	\end{equation}
\end{restatable}
\begin{restatable}{theorem}{mwcoulombmodeexpansion}\label{thm:mw_coulomb_mode_expansion}
	The mode expansion of the Maxwell field in the Coulomb gauge is
	\begin{equation}
		\vb{A}_\perp(x^\mu)
		=
		\sum_{\lambda=1,2}
		\int_{\mathbb{R}^3}\frac{\dd[3]{p}}{(2\pi)^3\sqrt{2\omega(\vb{p})}}
		\left\{
			a_\lambda(\vb{p})
			\boldsymbol{\epsilon}_\lambda(\vb{p})
			e^{-ip_\mu x^\mu}
			+
			\text{c.c.}
		\right\}_{p_0=\omega(\vb{p})}
		\label{eq:mw_coulomb_mode_expansion}
	\end{equation}
	with $\omega(\vb{p})=\norm{\vb{p}}$ where the polarization basis vectors satisfy
	\begin{align}
		\vb{p}
		\vdot
		\boldsymbol{\varepsilon}_\lambda(\vb{p})
		=
		0
		&&
		\boldsymbol{\varepsilon}_\alpha(\vb{p})
		\vdot
		\boldsymbol{\varepsilon}_\beta(\vb{p})^*
		=
		\delta_{\alpha\beta}
		\label{eq:mw_coulomb_polarization_basis}
	\end{align}
	and are complete~\cite[p.~341]{Srednicki2007}
	\begin{equation}
		\sum_{\lambda=1,2}
		\boldsymbol{\varepsilon}_\lambda^i(\vb{p})
		\boldsymbol{\varepsilon}_\lambda^j(\vb{p})^*
		=
		\delta^{ij}
		-
		\frac{p^ip^j}{\vb{p}^2}
		=
		P_\perp
		\label{eq:mw_coulomb_polarization_complete}
		.
	\end{equation}
\end{restatable}
\begin{restatable}{theorem}{mwcoulombmodeexpansionemfield}\label{thm:mw_coulomb_mode_expansion_em_field}
	The mode expansion of the electric and magnetic fields reads
	\begin{align}
		\vb{E}(x^\mu)
		&=
		\sum_{\lambda=1,2}
		\int\frac{\dd[3]{p}}{(2\pi)^3\sqrt{2\omega(\vb{p})}}
		\left(
			-i
			\omega(\vb{p})
		\right)
		\left\{
			a_\lambda(\vb{p})
			\boldsymbol{\varepsilon}_\lambda(\vb{p})
			e^{-ip_\mu x^\mu}
			-
			\text{c.c.}
		\right\}_{p_0=\omega(\vb{p})}
		\\
		\vb{B}(x^\mu)
		&=
		\sum_{\lambda=1,2}
		\int\frac{\dd[3]{p}}{(2\pi)^3\sqrt{2\omega(\vb{p})}}
		\left(
			i\vb{p}
		\right)
		\cross
		\left\{
			a_\lambda(\vb{p})
			\boldsymbol{\varepsilon}_\lambda(\vb{p})
			e^{-ip_\mu x^\mu}
			-
			\text{c.c.}
		\right\}_{p_0=\omega(\vb{p})}
		\label{eq:mw_coulomb_mode_expansion_em_field}
		.
	\end{align}
\end{restatable}
\begin{restatable}{lemma}{mwcoulombcanonicalmomentum}
	The canonical momentum density
	\begin{equation}
		\pi^i(x^\mu)
		=
		\partial_t
		\pdv{\mathcal{L}}{(\partial_tA_i)}
		=
		-
		E^i(x^\mu)
		\label{eq:mw_coulomb_canonical_momentum}
	\end{equation}
	is equal to the electric field.
\end{restatable}
\begin{corollary}\label{thm:mw_hamiltonian}
	The Hamiltonian density is
	\begin{equation}
		\mathcal{H}
		=
		\frac{1}{2}
		\boldsymbol{\pi}^2
		-
		\mathcal{L}
		=
		\frac{1}{2}
		\left(
			\vb{E}^2
			+
			\vb{B}^2
		\right)
		\label{thm:mw_hamiltonian}
		.
	\end{equation}
\end{corollary}
\begin{restatable}{lemma}{mwenergy}\label{thm:mw_energy}
	The total energy is
	\begin{equation}
		H
		=
		\sum_{\lambda=1,2}
		\int\frac{\dd[3]{p}}{(2\pi)^3}
		\omega(\vb{p})
		\abs{a_\lambda(\vb{p})}^2
		\label{thm:mw_energy}
		.
	\end{equation}
\end{restatable}
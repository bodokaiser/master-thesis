\section{Canonical quantization}

\begin{definition}[Transverse delta distribution]\label{def:delta_distribution_transverse}
	The transverse delta distribution
	\begin{equation}
		\label{eq:delta_distribution_transverse}
		\delta_{\perp ij}^{(3)}(\vb{x}-\vb{y})
		=
		P^{ij}_\perp
		\delta^{(3)}(\vb{x}-\vb{y})
		=
		\int\frac{\dd[3]{p}}{(2\pi)^3}
		\left(
			\delta_{ij}
			-
			\frac{p_ip_j}{\vb{p}^2}
		\right)
		e^{i\vb{p}\vdot(\vb{x}-\vb{y})}
		.
	\end{equation}
\end{definition}
\begin{corollary}[Transverse Maxwell operator]\label{thm:qmw_transverse}
	Promoting the dynamical field variables of the transverse Maxwell field to operators
	\begin{align}
		\label{eq:qmw_transverse}
		\hat{\vb{A}}_\perp(x^\mu)
		&=
		\sum_{\lambda=1,2}
		\int_{\mathbb{R}^3}\frac{\dd[3]{p}}{(2\pi)^3}
		\frac{1}{\sqrt{2\omega(\vb{p})}}
		\left\{
			\hat{a}_\lambda(\vb{p})
			\boldsymbol{\epsilon}_\lambda(\vb{p})
			e^{-ip_\mu x^\mu}
			+
			\text{c.c.}
		\right\}_{p_0=\omega(\vb{p})}
		\\
		\hat{\vb{E}}_\perp(x^\mu)
		&=
		\sum_{\lambda=1,2}
		\int_{\mathbb{R}^3}\frac{\dd[3]{p}}{(2\pi)^3}
		\left(
			-i
			\sqrt{\frac{\omega(\vb{p})}{2}}
		\right)
		\left\{
			\hat{a}_\lambda(\vb{p})
			\boldsymbol{\epsilon}_\lambda(\vb{p})
			e^{-ip_\mu x^\mu}
			-
			\text{c.c.}
		\right\}_{p_0=\omega(\vb{p})}
	\end{align}
	satisfying the equal-time commutation relations~\cite[p.~197]{Greiner2013}
	\begin{equation}
		\label{eq:qmw_transverse_comm}
		\comm{\hat{A}_\perp^i(t,\vb{x})}{\hat{E}_\perp^j(t,\vb{y})}
		=
		-i
		\delta_{\perp ij}^{(3)}(\vb{x}-\vb{y})
		=
		\comm{\hat{A}_\perp^i(t,\vb{x})}{-\hat\pi_\perp^j(t,\vb{y})}
	\end{equation}
	where $\hat\pi_\perp$ is the canonical momentum density operator, classically defined in \Cref{thm:mw_coulomb_canonical_momentum}.
\end{corollary}
\begin{definition}\label{thm:qmw_hamilton}
	The Hamilton operator of the transverse Maxwell field is the normal-ordered classical energy
	\begin{equation}
		\label{eq:qmw_hamilton}
		\hat{H}
		=
		\sum_{\lambda=1,2}
		\int\frac{\dd[3]{p}}{(2\pi)^3}
		\omega(\vb{p})
		\hat{a}_\lambda^\dagger(\vb{p})
		\hat{a}_\lambda(\vb{p})
	\end{equation}
	where $\omega(\vb{p})=\norm{\vb{p}}$.
\end{definition}
\begin{definition}\label{thm:qmw_number}
	The (particle) number operator of the transverse Maxwell field is the unweighted part of the Hamilton operator
	\begin{equation}
		\label{eq:qmw_number}
		\hat{N}
		=
		\sum_{\lambda=1,2}
		\int\frac{\dd[3]{p}}{(2\pi)^3}
		\hat{a}_\lambda^\dagger(\vb{p})
		\hat{a}_\lambda(\vb{p})
		.
	\end{equation}
\end{definition}
\begin{restatable}{lemma}{qmwcommac}\label{thm:qmw_comm_ac}
	The annihilation and creation operators of the transverse Maxwell field satisfy
	\begin{align}
		\label{eq:qmw_comm_ac}
		\comm{\hat{a}_\lambda(\vb{p})}{\hat{a}_{\lambda^\prime}^\dagger(\vb{q})}
		&=
		(2\pi)^3
		\delta^{(3)}(\vb{q}-\vb{p})
		\delta_{\lambda\lambda^\prime}
		\\
		\comm{\hat{a}_\lambda^\dagger(\vb{p})}{\hat{a}_{\lambda^\prime}^\dagger(\vb{q})}
		&=
		\comm{\hat{a}_\lambda(\vb{p})}{\hat{a}_{\lambda^\prime}(\vb{q})}
		=
		0
		.
	\end{align}
\end{restatable}
\begin{proof}
	foobar
\end{proof}

\begin{definition}[Smearing vector function]
	Let $f_1,f_2,f_3\in\mathcal{S}(\mathbb{R},\mathbb{R}^3)$ be smearing functions, then
	\begin{equation}
		\begin{split}
			\vb{f}
			\colon
			\mathbb{R}\times \mathbb{R}^3
			&\to
			\mathbb{R}
			\\
			x^\mu
			&\mapsto
			\vb{f}(x^\mu)
			=
			\begin{pmatrix}
				f_1(x^\mu) \\
				f_2(x^\mu) \\
				f_3(x^\mu)				
			\end{pmatrix}
		\end{split}
	\end{equation}
	defines the smearing vector functioin.
\end{definition}
\begin{definition}[Transverse smearing function]
	Let $\vb{f}$ be a smearing vector, then
	\begin{equation}
		\label{eq:qmw_transverse_smearing_function}
		f_\lambda\left(\omega(\vb{p}),\vb{p}\right)
		=
		\vb{f}\left(\omega(\vb{p}),\vb{p}\right)
		\vdot
		\boldsymbol{\varepsilon}_\lambda(\vb{p})
	\end{equation}
	defines the transverse $\lambda$ component of the smearing vector function.
\end{definition}

\begin{definition}\label{thm:qmw_transverse_pn}
	The positive and negative transverse Maxwell operator are
	\begin{align}
		\label{eq:qmw_transverse_pn}
		\hat{\vb{A}}^+_\perp(x^\mu)
		&=
		\sum_{\lambda=1,2}
		\int_{\mathbb{R}^3}\frac{\dd[3]{p}}{(2\pi)^3\sqrt{2\omega(\vb{p})}}
		\hat{a}_\lambda^\dagger(\vb{p})
		\boldsymbol{\epsilon}_\lambda(\vb{p})^*
		\eval{e^{+ip_\mu x^\mu}}_{p_0=\omega(\vb{p})}
		\\
		\hat{\vb{A}}^-_\perp(x^\mu)
		&=
		\sum_{\lambda=1,2}
		\int_{\mathbb{R}^3}\frac{\dd[3]{p}}{(2\pi)^3\sqrt{2\omega(\vb{p})}}
		\hat{a}_\lambda(\vb{p})
		\boldsymbol{\epsilon}_\lambda(\vb{p})
		\eval{e^{-ip_\mu x^\mu}}_{p_0=\omega(\vb{p})}
		.
	\end{align}
\end{definition}
\begin{restatable}{lemma}{qmwcommpn}\label{thm:qmw_comm_pn}
	\begin{align}
		\label{eq:qmw_comm_ac}
		\comm{\hat{\vb{A}}_\perp^-(x^\mu)}{\hat{\vb{A}}_\perp^+(y^\mu)}
		&=
		\\
		\comm{\hat{\vb{A}}_\perp^-(x^\mu)}{\hat{\vb{A}}_\perp^-(y^\mu)}
		&=
		\comm{\hat{\vb{A}}_\perp^+(x^\mu)}{\hat{\vb{A}}_\perp^+(y^\mu)}
		=
		0
		.
	\end{align}
\end{restatable}

\begin{definition}\label{def:qmw_transverse_pn_smeared}
	Let $\vb{f}$ be a smearing vector function, then the smeared positive and negative frequency transverse Maxwell operators are
	\begin{align}
		\label{eq:qmw_transverse_pn_smeared}
		\hat{\vb{A}}_\perp^+[\vb{f}]
		&=
		\int\dd[4]{x}\
		\vb{f}(x^\mu)
		\vdot
		\hat{\vb{A}}_\perp^+(x^\mu)
		\\
		\hat{\vb{A}}_\perp^-[\vb{f}]
		&=
		\int\dd[4]{x}\
		\vb{f}(x^\mu)
		\vdot
		\hat{\vb{A}}_\perp^-(x^\mu)
		.
	\end{align}
\end{definition}
\begin{restatable}{lemma}{qmwtransversepnsmeared}\label{thm:qmw_transverse_pn_smeared}
	The mode expanded smeared positive and negative frequency transverse Maxwell operators are
	\begin{align}
		\label{eq:qmw_transverse_pn}
		\hat{\vb{A}}^+_\perp(x^\mu)
		&=
		\sum_{\lambda=1,2}
		\int_{\mathbb{R}^3}\frac{\dd[3]{p}}{(2\pi)^3\sqrt{2\omega(\vb{p})}}
		f_\lambda\left(\omega(\vb{p}),\vb{p}\right)^*
		\hat{a}_\lambda^\dagger(\vb{p})
		\\
		\hat{\vb{A}}^-_\perp(x^\mu)
		&=
		\sum_{\lambda=1,2}
		\int_{\mathbb{R}^3}\frac{\dd[3]{p}}{(2\pi)^3\sqrt{2\omega(\vb{p})}}
		f_\lambda\left(\omega(\vb{p}),\vb{p}\right)
		\hat{a}_\lambda(\vb{p})
		.
	\end{align}	
\end{restatable}

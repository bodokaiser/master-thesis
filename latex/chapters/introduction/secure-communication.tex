\section{Secure communication}

\begin{figure}[htb]
	\centering
	\includestandalone{figures/tikz/secure-communication}
	\caption{Foobar}
\end{figure}

% eternal secure communication
% symmetric ciphers

\subsection{Algorithmic key exchange}

% public-key ciphers
% asymmetric ciphers for key-exchange
% problems with asymmetric ciphers

% post-quantum algorithms

\subsection{Quantum-key distribution}



\begin{figure}[htb]
	\centering
	\includestandalone{figures/tikz/qkd-parties}
	\caption{\Gls{qkd} system used to create a shared key between between two spatially distanced parties, Alice and Bob, secret to a third party, Eve. Eve has full access to the quantum channel but can only read information from the classical authenticated channel.}
\end{figure}

% explain superoperator (effect of quantum channel)

% how does this transfer to the time-continuous case? -> in the previous chapter we should have argued how we can recover symbols
% derive an equation how Bob's beta relates to Alice's alpha (quantum channel, quantum uncertainty in measurement)
% show that the coherent state's quadratures are a bivariate normal random variable
% chapter overview: what is awaiting the user and why is it important?
% rough idea why CV-QKD is secure (plot with bivariate Gaussian, Eve's measurement, Bob's measurement -> two sigma)

\begin{enumerate}
	\item Alice samples $n$ complex symbols $\alpha_1,\dots,\alpha_n\in\mathbb{C}$ from a complex normal distribution.
	\item Alice encodes the complex symbols onto a coherent state $\ket{\alpha(t)}$ and sends it through a quantum channel.
	\item The quantum channel maps the coherent state $\ket{\alpha(t)}$ to a coherent state $\ket{\alpha(t)}$ by a superoperator.
	\item Bob decodes the complex symbols $\beta_1,\dots,\beta_n\in\mathbb{C}$ from the received coherent state $\ket{\beta(t)}$.
	\item Alice and Bob perform the classical post-processing over the authenticated classical communication channel to distill a shared secret bit string from the correlated symbols $\alpha_1,\dots,\alpha_n$ and $\beta_1,\dots,\beta_n$.
\end{enumerate}

% very basic idea
% assumptions
% protocol (Alice sends state, Bob measures)
% dv most simple, developed in academia, difficult for practical use because of ....
% cv developed in industry because of benefits ...

\begin{figure}[htb]
	\centering
	\includestandalone{figures/pgfplots/phase-space}
	\caption{Phase space density plot of the $X,P$ variables of a coherent state $\ket{\alpha}$.}
\end{figure}
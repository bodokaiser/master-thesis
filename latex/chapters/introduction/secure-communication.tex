\section{Secure communication}

Two spatially distanced parties, Alice and Bob, share a communication channel that allows them to send and receive information.
We assume the communication channel to be a one-to-one map between Alice and Bob, i.e., information is received as transmitted, for instance, by employing some error correction, e.g., LDPC codes~\cite{Gallager1962}.
How do we enable secure communication between Alice and Bob?
To promote the communication channel to be secure against an adversarial third party, Eve, we need to extend the communication protocol to assure
\begin{enumerate}
	\item \textbf{confidentiality}, and
	\item \textbf{integrity}.
\end{enumerate}
Confidentiality ensures that Eve cannot read messages and is implemented using symmetric encryption like the \gls{otp}~\cite{Shannon1949} or the more practical \gls{aes}~\cite{Daemen1999}.
Integrity ensures that Eve cannot alter messages in transit and is implemented using a \gls{mac}, e.g., universal hash functions~\cite{Carter1979}.
\begin{figure}[htb]
	\centering
	\includestandalone{figures/tikz/secure-communication}
	\caption{Secure communication between Alice and Bob using a classical channel to which an adversarial third party, Eve, has read and write access: A plaintext $p$ is encrypted with a secret key $k$ through a symmetric encryption function $e(k,p)$ yielding a ciphertext $c$. The ciphertext is amended by a hashing function $h(k,c)$ yielding the message $m$. The message is sent through an error-free channel to Bob. Bob checks the integrity of the message using his copy of the secret key $k$ and the hash inside the message $m$. If he finds the message $m$ to be genuine, he removes the hash to obtain the ciphertext $c$ which he decrypts using his key $k$ and the decryption function $d(k,c)$. After decryption, Bob has access to the plaintext $p$ initially prepared by Alice.}\label{fig:secure_communication}
\end{figure}
\Cref{fig:secure_communication} depicts a communication between Alice and Bob secure against a third adversarial party, Eve.
The plaintext is first encrypted and then appended a \gls{mac} as advised by Ref.~\cite{Kohno2003,Krawczyk2001,Bellare2000}.

Given a truly random and secret key $k$ shared between Alice and Bob such a system has been proven to be eternal secure when using the \gls{otp} encryption scheme~\cite{Shannon1949}.
However, one needs to employ a practical mechanism to generate and distribute the key securely.

\FloatBarrier
\subsection{Public-key distribution}

The standard attempt to solve the secret key distribution problem is to use an asymmetric cipher.
Asymmetric ciphers like RSA, comprise a public key for encryption and a private key for decryption.
Using such an asymmetric cipher, secret key distribution between Alice and Bob involves the following steps:
\begin{enumerate}
	\item Bob generates a public and a private key.
	\item Alice generates a secret key.
	\item Bob discloses the public key to Alice.
	\item Alice encrypts the secret key with the public key and sends it to Bob.
	\item Bob decrypts Alice secret key with the private key.
\end{enumerate}
Assuming the asymmetric cipher to be secure, Alice and Bob now share a secret key.
Public-key distribution algorithms, for instance, Diffie-Hellman~\cite{Diffie1976} and variants thereof, e.g., \gls{ecdh}, are heavily employed in todays internet for key agreement.
One of the advantages of public-key distribution is that it is software-defined, allowing for easy installation and upgrades.
The core principle behind asymmetric ciphers is the concept of one-way functions:
Functions that are easy to compute but difficult to invert.
A typical class of one-way functions appears in prime number factorization: Given two prime numbers $p$ and $q$, it is easy to compute the product $pq$ but difficult to factorize $pq$ into its prime numbers $p$ and $q$.
Here, easy and difficult refer to the computational complexity, which gives an upper bound of the best (known) algorithm to solve the problem.
So far, there has been no mathematical proof that one-way functions are indeed computational complex, and it may be that we are just not yet aware of efficient algorithms placing asymmetric cryptography at all in a questionable position.
With the advent of quantum computers, new algorithms with significant speed up to their classical counterpart were found.
Shor, for example, presented an algorithm for prime number factorization in polynomial time~\cite{Shor1994} - the same task used by the Diffie-Hellman key exchange.
Consequently, post-quantum algorithms have been proposed~\cite{Bernstein2017,Chen2016} to use a different class of one-way functions where no efficient (quantum) algorithm is known.
As long as there is no proof for the computational complexity for a specific family of one-way functions, public-key distribution is only safe for key lengths that are not computationally feasible.
However, what is not computationally feasible today might be tomorrow, and it is conceivable that encrypted communication recorded today will be broken in the feature.

\FloatBarrier
\subsection{Quantum-key distribution}

\begin{figure}[htb]
	\centering
	\includestandalone{figures/tikz/qkd-system}
	\caption{\Gls{qkd} system used to create a shared key between between two spatially distanced parties, Alice and Bob, secret to a third party, Eve. Eve has full access to the quantum channel but can only read information from the classical authenticated channel.}
\end{figure}

% explain superoperator (effect of quantum channel)

% how does this transfer to the time-continuous case? -> in the previous chapter we should have argued how we can recover symbols
% derive an equation how Bob's beta relates to Alice's alpha (quantum channel, quantum uncertainty in measurement)
% show that the coherent state's quadratures are a bivariate normal random variable
% chapter overview: what is awaiting the user and why is it important?
% rough idea why CV-QKD is secure (plot with bivariate Gaussian, Eve's measurement, Bob's measurement -> two sigma)

\begin{enumerate}
	\item Alice samples $n$ complex symbols $\alpha_1,\dots,\alpha_n\in\mathbb{C}$ from a complex normal distribution.
	\item Alice encodes the complex symbols onto a coherent state $\ket{\alpha(t)}$ and sends it through a quantum channel.
	\item The quantum channel maps the coherent state $\ket{\alpha(t)}$ to a coherent state $\ket{\alpha(t)}$ by a superoperator.
	\item Bob decodes the complex symbols $\beta_1,\dots,\beta_n\in\mathbb{C}$ from the received coherent state $\ket{\beta(t)}$.
	\item Alice and Bob perform the classical post-processing over the authenticated classical communication channel to distill a shared secret bit string from the correlated symbols $\alpha_1,\dots,\alpha_n$ and $\beta_1,\dots,\beta_n$.
\end{enumerate}

% very basic idea
% assumptions
% protocol (Alice sends state, Bob measures)
% dv most simple, developed in academia, difficult for practical use because of ....
% cv developed in industry because of benefits ...

\begin{figure}[htb]
	\centering
	\includestandalone{figures/pgfplots/phase-space}
	\caption{Phase space density plot of the $X,P$ variables of a coherent state $\ket{\alpha}$.}
\end{figure}
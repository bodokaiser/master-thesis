\section{Conventions}

Throughout the thesis, we work in natural units in which the reduced Planck constant $\hbar$, the speed of light $c$, the vacuum permittivity $\varepsilon_0$, the electron charge $e_0$, and the electron mass $m_e$ are set to one, i.e.,
\begin{equation}
	\hbar
	=
	c
	=
	\varepsilon_0
	=
	e_0
	=
	m_e
	=
	1
	.
\end{equation}
The advantage of using the natural unit system is that it reduces the clutter when performing long calculations.
It might appear confusing that for instance energy $E$ is sometimes used interchangeably with the angular frequency $\omega$ because $E=\omega$ in natural units.

To distinguish between elements of the Euclidean and Minkowski vector space, we use boldface for Euclidean vectors.
We use the Latin alphabet to index the components of Euclidean vectors and the Greek alphabet for Minkowski vectors.
For example, we write the four-position vector of Minkowski spacetime
\begin{equation}
	x
	=
	(x^0,x^1,x^2,x^3)
	=
	(t,\vb{x})
\end{equation}
where $\vb{x}$ is the Euclidean position vector, and the four-momentum
\begin{equation}
	p
	=
	(p^0,p^1,p^2,p^3)
	=
	(\omega,\vb{p})
	.
\end{equation}
Often, it is sufficient to only consider the components $x^\mu,p^\mu$ of a Minkowski vector.
We use the mostly-minus convention for the Minkowski metric
\begin{equation}
	\eta
	=
	\begin{pmatrix}
		1 & 0 & 0 & 0 \\
		0 & -1 & 0 & 0 \\
		0 & 0 & -1 & 0 \\
		0 & 0 & 0 & -1
	\end{pmatrix}
	.
\end{equation}
Let $a,b$ be two Minkowski vectors, then their Minkowski product is
\begin{equation}
	\eta(a,b)
	=
	a^\mu
	\eta_{\mu\nu}
	b^\mu
	=
	a_\mu
	b^\mu
	=
	a_0b_0
	-
	\vb{a}\vdot\vb{b}
	.
\end{equation}
The former equation tell'su 
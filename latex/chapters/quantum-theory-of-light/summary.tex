\section*{Summary}

Our approach to a theoretical framework for quantum optical communication has been unconventional but insightful:
We developed the quantum Maxwell and electric field operators in the Coulomb gauge, \cref{eq:maxwell_operator} and \cref{eq:electric_operator}, from field-theoretic arguments.
We constructed generalized number and coherent quantum states from the quantum field operators using mathematical considerations.
Contrary to the quantum states of the harmonic oscillator, the generalized number states are not eigenstates of the energy and momentum operator.
Furthermore, we find that the variance of, for instance, the electric field operator, contains additional terms depending on the spectrum which are not present in the harmonic oscillator picture.
Otherwise, we find the generalized number and coherent states to share many properties of their quantum optics counterparts.

The Coulomb gauge limits the generality of our results.
Choosing the Coulomb gauge restricts our results to a stationary reference frame.
The assumption of a stationary reference frame appears reasonable for terrestrial quantum optical communication but must be questioned for deep-space applications.
Invoking further approximations allows the comparison of our results with the literature.
We already used the scalar or one-polarization approximation in the construction of the quantum states.
Strictly speaking, this is not even an approximation as we can always recover both polarizations by constructing the tensor product.
For optical communication the transverse momentum profile transversal to the propagation axis carries no information but is determined by the communication channel.
It is reasonable to ignore the transversal \gls{dof}s and reduce the field to one-dimension~ \cite[p.~53]{Cohen2019}
\begin{equation}
	\int_{\mathbb{R}^3}
	\frac{\dd[3]{p}}{(2\pi)^3\sqrt{2\omega(\vb{p})}}
	\to
	\int_{-\infty}^{\infty}
	\frac{\dd{p}}{(2\pi)\sqrt{2p}}
	.
\end{equation}
The approximation can be made even stronger when neglecting back-scattering and reflection effects and restricting the momentum distribution to forward momentum, i.e.,
\begin{equation}
	\int_{-\infty}^{\infty}
	\frac{\dd{p}}{(2\pi)\sqrt{2p}}
	\approx	
	\int_{0}^{\infty}
	\frac{\dd{p}}{(2\pi)\sqrt{2p}}	
\end{equation}
where the identification of the forward momentum with the frequency, $p=\omega$ is often made.
Finally, we can remove the Lorentz factor in the integration measure by assuming a bandwidth-limited detection.
Invoking the mean-value theorem for definite integrals
\begin{equation}
	\int_{0}^{\infty}
	\frac{\dd{\omega}}{(2\pi)\sqrt{2\omega}}
	\approx
	\frac{1}{\sqrt{2\omega_0}}
	\int_{0}^{\infty}
	\frac{\dd{\omega}}{2\pi}
	,
\end{equation}
we can absorb the prefactor into the field.
Combining these approximations and applying them to our results, we find our results in agreement with Ref.~\cite{Barnett2002}.
We conclude that (continuous-mode) quantum optics is the scalar, one-dimensional forward, quasi-monochromatic approximation of quantum Maxwell field theory in the Coulomb gauge.
We are in the acceleration stage of a second quantum revolution.
While the first quantum revolution unveiled us the - often counterintuitive - rules of quantum mechanics, the second quantum revolution breaches a new level of understanding and control of applied quantum systems.
Notable results of the last century include, for instance, the widespread adaption of the laser, the transistor, and nuclear-magnetic-resonance tomography.
Recently, the second quantum revolution gained further momentum by attracting widespread public and political interest.
Quantum technologies are seen as critical technologies for the upcoming decades, and public institutions fear losing the technological race.

% quantum science and technology / second quantum revolution
% growth market prediction
% possible citations:
% Quantum technology: the second quantum revolution (https://arxiv.org/pdf/quant-ph/0206091.pdf)

\section{Quantum technologies}

% applications of quantum technology (fields: quantum communication, computation, simulation, sensing/metrology, ...)
% focus area: quantum communication -> secure communication -> quantum key distribution

\begin{figure}[htb]
	\centering
	\includestandalone{figures/tikz/quantum-technologies}
	\caption{Relevant quantum technologies for the European quantum flagship bla.}
\end{figure}

% cite The quantum technologies roadmap (https://iopscience.iop.org/article/10.1088/1367-2630/aad1ea/pdf) for overview of quantum technologies

% \section{Quantum communication}
% actually quantum communication -> quantum key-distribution
% most mature quantum technology
% commercial adaption (name some examples, QKD networks)

\section{Secure communication}

% eternal secure communication
% symmetric ciphers
% asymmetric ciphers for key-exchange
% problems with asymmetric ciphers
% quantum key-distribution as mechanism for secure key exchange

\section{Quantum-key distribution}

% very basic idea
% assumptions
% protocol (Alice sends state, Bob measures)
% dv most simple, developed in academia, difficult for practical use because of ....
% cv developed in industry because of benefits ...

\begin{figure}[htb]
	\centering
	\includestandalone{figures/tikz/qkd-parties}
	\caption{\Gls{qkd} system used to create a shared key between between two spatially distanced parties, Alice and Bob, secret to a third party, Eve. Eve has full access to the quantum channel but can only read information from the classical authenticated channel.}
\end{figure}

\begin{figure}[htb]
	\centering
	\includestandalone{figures/pgfplots/phase-space}
	\caption{Phase space density plot of the $X,P$ variables of a coherent state $\ket{\alpha}$.}
\end{figure}

\section{Problem statement and motivation}

% no complete and connected description of DV-QKD (because not discussed much in academia)
% standard quantum optics formalism neither time-dependent nor frequency continuous
% signal-processing uses continuous frequency spectras while quantum optics uses single frequency modes
% no complete fully-connected description of a practical QKD device reported so far
% intersections of different disciplines: signal-processing, quantum optics, quantum information theory, quantum field theory

\section{Thesis overview and structure}

% bottom-up approach
% chapter corresponds to different abstraction levels (cv-qkd chapter -> quantum information theory, coherent state communication system -> signal processing, ...)

\section{Conventions and notation}

% Minkowski space, four vectors
% why we use p instead of omega for modes -> to distinguish between frequency and momentum

\begin{align}
	f(t)
	=
	\int_{\mathbb{R}}\frac{\dd{\omega}}{2\pi}
	f(\omega)
	e^{+i\omega t}
	&&
	f(\omega)
	=
	\int_{\mathbb{R}}\dd{t}
	f(t)
	e^{-i\omega t}
\end{align}
\begin{align}
	f(\vb{x})
	=
	\int_{\mathbb{R}^3}\frac{\dd[3]{p}}{(2\pi)^3}
	f(\vb{p})
	e^{-i\vb{p}\vdot\vb{x}}
	&&
	f(\vb{p})
	=
	\int_{\mathbb{R}^3}\dd[3]{x}
	f(\vb{x})
	e^{+i\vb{p}\vdot\vb{x}}
\end{align}
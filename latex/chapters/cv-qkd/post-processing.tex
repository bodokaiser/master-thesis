\section{Classical post-processing}

% aim of classical post-processing (correlated variables -> shared secret, estimate error -> protocol abortion)}

% what about (base) sifting?

\begin{figure}[htb]
	\centering
	\includestandalone{figures/tikz/post-processing}
	\caption{\Gls{qkd} transmission system from a signal-processing perspective.}
\end{figure}

\begin{figure}[htb]
	\centering
	\includestandalone{figures/tikz/post-processing-dv}
	\caption{\Gls{dvqkd} transmission system from a signal-processing perspective.}
\end{figure}

\subsection{Reconciliation}

% classical post-processing in DV-QKD
% classical post-processing in CV-QKD (differences)

\subsection{Error correction}

% LDPC codes

\subsection{Privacy amplification}

% XOR-ing using Toeplitz matrices
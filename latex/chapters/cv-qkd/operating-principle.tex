\section{Operating principle}

\begin{enumerate}
	\item Alice samples $n$ complex symbols $\alpha_1,\dots,\alpha_n\in\mathbb{C}$ from a complex normal distribution.
	\item Alice encodes the complex symbols onto a coherent state $\ket{\alpha(t)}$ and sends it through a quantum channel.
	\item The quantum channel maps the coherent state $\ket{\alpha(t)}$ to a coherent state $\ket{\alpha(t)}$ by a superoperator.
	\item Bob decodes the complex symbols $\beta_1,\dots,\beta_n\in\mathbb{C}$ from the received coherent state $\ket{\beta(t)}$.
	\item Alice and Bob perform the classical post-processing over the authenticated classical communication channel to distill a shared secret bit string from the correlated symbols $\alpha_1,\dots,\alpha_n$ and $\beta_1,\dots,\beta_n$.
\end{enumerate}

% explain superoperator (effect of quantum channel)

% how does this transfer to the time-continuous case? -> in the previous chapter we should have argued how we can recover symbols
% derive an equation how Bob's beta relates to Alice's alpha (quantum channel, quantum uncertainty in measurement)
% show that the coherent state's quadratures are a bivariate normal random variable
% chapter overview: what is awaiting the user and why is it important?
% rough idea why CV-QKD is secure (plot with bivariate Gaussian, Eve's measurement, Bob's measurement -> two sigma)
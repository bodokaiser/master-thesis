\section{Transmitter}

\subsection{Mach-Zehnder modulator}

\begin{figure}[htb]
	\centering
	\includestandalone{figures/pstricks/mzi-symmetric}
	\caption{Free-space setup of a symmetric \gls{mzm}: The input light mode $\hat{a}_1$ enters a first beam splitter BS1 from the left. A vacuum light mode $\hat{a}_2$ enters BS2 from the top. The transformed mode $\hat{a}_1^\prime$ and $\hat{a}_2^\prime$ exit BS1 to the right and the bottom. A first phase shifter and first mirror M1, right to BS1, add a relative phase of $\varphi_1+\pi$ from mode $\hat{a}_1$ to $\hat{a}_1^{\prime\prime}$. Below BS1, a second mirror M2 directs the light to a right second phase shifter, both adding a relative phase of $\varphi_2+\pi$ from mode $\hat{a}_2^\prime$ to $\hat{a}_2^{\prime\prime}$. A second beam splitter BS2 transforms the input modes $\hat{a}_1^{\prime\prime}$ and $\hat{a}_2^{\prime\prime}$ to the output modes $\hat{a}_1^{\prime\prime\prime}$ and $\hat{a}_2^{\prime\prime\prime}$.}
\end{figure}

\begin{equation}
	\begin{split}
		\vb{\hat{a}}^{\prime\prime\prime}
		=
		U_\text{BS2}
		\vb{\hat{a}}^{\prime\prime}
		=
		U_\text{BS2}
		U_\text{PS}(\phi_1,\phi_2)
		\vb{\hat{a}}^{\prime}
		=
		U_\text{BS2}
		U_\text{PS}(\phi_1,\phi_2)
		U_\text{BS1}
		\vb{\hat{a}}
		=
		U_\text{MZM}(\phi_1,\phi_2)
		\vb{\hat{a}}
	\end{split}
\end{equation}

\begin{equation}
	U_\text{BS}(\theta,\varphi)
	=
	\begin{pmatrix}
		\cos(\theta/2) & ie^{+i\varphi}\sin(\theta/2) \\
		ie^{-i\varphi}\sin(\theta/2) & \cos(\theta/2) \\
	\end{pmatrix}
\end{equation}

\begin{equation}
	U_\text{BS1}
	=
	\frac{1}{\sqrt{2}}
	\begin{pmatrix}
		1 & i \\
		i & 1
	\end{pmatrix}
	=
	U_\text{BS}(\pi/2,0)	
\end{equation}
\begin{equation}
	U_\text{PS}
	=
	\begin{pmatrix}
		e^{i\phi_1} & 0 \\
		0 & e^{i\phi_2}
	\end{pmatrix}	
\end{equation}
\begin{equation}
	U_\text{BS2}
	=
	\frac{1}{\sqrt{2}}
	\begin{pmatrix}
		i & 1 \\
		1 & i
	\end{pmatrix}
	=
	e^{-i\pi/2}
	U_\text{BS}(\pi,0)
	U_\text{BS}(\pi/2,0)
\end{equation}

\begin{equation}
	U_\text{MZM}(\phi_1,\phi_2)
	=
	ie^{i\frac{\phi_1+\phi_2}{2}}
	\begin{pmatrix}
		\cos(\frac{\phi_2-\phi_1}{2}) & \sin(\frac{\phi_2-\phi_1}{2}) \\
		-\sin(\frac{\phi_2-\phi_1}{2}) & \cos(\frac{\phi_2-\phi_1}{2})
	\end{pmatrix}
\end{equation}
\begin{table}[htb]
	\centering
	\begin{tabular}{lcc}
		\toprule
		Configuration & Phase relation & Modulation \\
		\midrule
		Push-pull & $\phi_1=-\phi_2$ & Amplitude \\
		Push-push & $\phi_1=+\phi_2$ & Phase \\
		\bottomrule
	\end{tabular}
	\caption{Configurations of a symmetric \gls{mzm}.}
\end{table}
\begin{align}
	\phi_+
	&=
	\phi_2+\phi_1
	&
	\phi_-
	&=
	\phi_2-\phi_1
\end{align}
\begin{equation}
	U_\text{MZM}(\phi_+,\phi_-)
	=
	ie^{i\phi_+/2}
	\begin{pmatrix}
		\cos(\phi_-/2)- & \sin(\phi_-/2) \\
		-\sin(\phi_-/2) & \cos(\phi_-/2)
	\end{pmatrix}
\end{equation}
\begin{equation}
	U_\text{BS}(\Theta,3\pi/2)
	=
	\begin{pmatrix}
		\cos(\Theta/2) & \sin(\Theta/2) \\
		-\sin(\Theta/2) & \cos(\Theta/2)
	\end{pmatrix}
	=
	R(\Theta)
\end{equation}
The \gls{mzm} transform is a beam splitter with flexible splitting ratio.

\begin{equation}
	R(\Theta)
	\vb{\hat{a}}
	=
	\hat{R}^\dagger(\Theta)
	\vb{\hat{a}}
	\hat{R}(\Theta)
	=
	e^{-i\Theta\hat{L}_y}
	\vb{\hat{a}}
	e^{+i\Theta\hat{L}_y}
\end{equation}
wherein~\cite[p.~97]{Leonhardt2010}
\begin{equation}
	\hat{L}_y
	=
	\frac{1}{2}
	\vb{\hat{a}}^\dagger
	\sigma_y
	\vb{\hat{a}}
	=
	\frac{i}{2}
	\left(
		\hat{a}_2^\dagger
		\hat{a}_1
		-
		\hat{a}_1^\dagger
		\hat{a}_2
	\right)
	.
\end{equation}

\begin{equation}
	\begin{split}
		\hat{R}(\Theta)
		\ket{n,0}
		&=
		\frac{1}{\sqrt{n!}}
		\hat{R}(\Theta)
		\left(\hat{a}_1^\dagger\right)^n
		\hat{R}^\dagger(\Theta)
		\hat{R}(\Theta)
		\ket{0,0}
		\\
		&=
		\frac{1}{\sqrt{n!}}
		\left[
			\hat{R}(\Theta)
			\hat{a}_1^\dagger
			\hat{R}^\dagger(\Theta)
		\right]^n
		\ket{0,0}
		\\
		&=
		\frac{1}{\sqrt{n!}}
		\left(
			\left[
				\hat{R}(\Theta)
				\hat{a}_1
				\hat{R}^\dagger(\Theta)
			\right]^\dagger
		\right)^n
		\ket{0,0}
		\\
		&=
		\frac{1}{\sqrt{n!}}
		\left(
			\left[
				\hat{R}^\dagger(-\Theta)
				\hat{a}_1
				\hat{R}(\Theta)
			\right]^\dagger
		\right)^n
		\ket{0,0}
		\\
		&=
		\frac{1}{\sqrt{n!}}
		\left(
			\left[
				\cos(\Theta/2)
				\hat{a}_1
				-
				\sin(\Theta/2)
				\hat{a}_2
			\right]^\dagger
		\right)^n
		\ket{0,0}
		\\
		&=
		\frac{1}{\sqrt{n!}}
		\left(
			\cos(\Theta/2)
			\hat{a}_1^\dagger
			-
			\sin(\Theta/2)
			\hat{a}_2^\dagger
		\right)^n
		\ket{0,0}
		\\
		&=
		\frac{1}{\sqrt{n!}}
		\sum_{m=0}^n
		\binom{n}{m}
		\cos(\Theta/2)^m
		\left(
			-\sin(\Theta/2)
		\right)^{n-m}
		\left(\hat{a}_1^\dagger\right)^m
		\left(\hat{a}_2^\dagger\right)^{n-m}
		\ket{0,0}
		\\
		&=
		\sum_{m=0}^n
		\binom{n}{m}^\frac{1}{2}
		(-1)^{n-m}
		\cos(\Theta/2)^m
		\sin(\Theta/2)^{n-m}
		\ket{n,n-m}
	\end{split}
\end{equation}

The number basis is complete and we can use our previous result to find the transformation property of a coherent state
\begin{equation}
	\begin{split}
		\hat{R}(\Theta)
		\ket{\alpha,0}
		&=
		e^{-\overline{n}/2}
		\sum_{n=0}^\infty
		\frac{\alpha^n}{\sqrt{n!}}
		\hat{R}(\Theta)
		\ket{n,0}
		\\
		&=
		e^{-\overline{n}/2}
		\sum_{n=0}^\infty
		\frac{1}{n!}
		\left(
			\alpha
			\cos(\Theta/2)
			\hat{a}_1^\dagger
			-
			\alpha
			\sin(\Theta/2)
			\hat{a}_2^\dagger
		\right)^n
		\ket{0,0}
		\\
		&=
		e^{-\overline{n}/2}
		\exp\left\{
			\alpha
			\cos(\Theta/2)
			\hat{a}_1^\dagger
			-
			\alpha
			\sin(\Theta/2)
			\hat{a}_2^\dagger
		\right\}
		\ket{0,0}
		\\
		&=
		e^{-\overline{n}\cos(\Theta/2)^2/2}
		\exp\left\{
			\alpha
			\cos(\Theta/2)
			\hat{a}_1^\dagger
		\right\}
		e^{-\overline{n}\sin(\Theta/2)^2/2}
		\exp\left\{
			-
			\alpha
			\sin(\Theta/2)
			\hat{a}_2^\dagger
		\right\}
		\ket{0,0}
		\\
		&=
		\hat{D}_1\left(
			\alpha
			\cos(\Theta/2)
		\right)
		\hat{D}_2\left(
			-
			\alpha
			\sin(\Theta/2)
		\right)
		\ket{0,0}
		\\
		&=
		\ket{\alpha\cos(\Theta/2),-\alpha\sin(\Theta/2)}
	\end{split}
\end{equation}

Tracing out one output of the MZM for a coherent state gives the projection.
That is not true in general, and tracing out an entangled state yields a mixed state!
\begin{equation}
	\ket{\psi}
	=
	\sum_{n,m}
	c_{n,m}
	\ket{n,m}
\end{equation}
\begin{equation}
	\rho
	=
	\ketbra{\psi}
	=
	\sum_{n,m,k,l}
	c_{n,m}
	c_{k,l}^*
	\ketbra{n,m}{k,l}
\end{equation}
\begin{equation}
	\begin{split}
		\rho_1
		=
		\tr_2\rho
		&=
		\sum_{n,m,k,l}
		c_{n,m}
		c_{k,l}^*
		\tr_2\ketbra{n,m}{k,l}
		\\
		&=
		\sum_{n,m,k,l}
		c_{n,m}
		c_{k,l}^*
		\ketbra{n,m}
		\delta_{k,l}
		\\
		&=
		\sum_{n,m}
		c_{n,m}
		\ketbra{n,m}
		\sum_k
		c_{k,k}^*
	\end{split}
\end{equation}

\subsection{In-phase and quadrature modulator}

\begin{figure}[htb]
	\centering
	\includestandalone{figures/pstricks/iqm}
	\caption{\gls{iq}-modulator comprising three \gls{mzm} and connected by an input and output fiber. MZM 1 and MZM 2, are used in push-pull configuration for \gls{am}. MZM 3 is used in push-push configuration for setting the relative phase between the upper and lower branch. Vacuum in- and outputs are indicated by the red dashed fiber.}
\end{figure}

\begin{equation}
	\begin{split}
		\begin{pmatrix}
			\hat{a}_1^{\prime\prime\prime} \\
			\hat{a}_2^{\prime\prime\prime} \\
			\hat{a}_3^{\prime\prime\prime} \\
			\hat{a}_4^{\prime\prime\prime}
		\end{pmatrix}
		&=
		\begin{pmatrix}
			 1 & 0 & 0 & 0 \\
			 0 & \multicolumn{2}{c}{\multirow{2}{*}{$U_\text{BS2}U_\text{PS}$}} & 0 \\
			 0 & & & 0 \\
			 0 & 0 & 0 & 1
		\end{pmatrix}
		\begin{pmatrix}
			\hat{a}_1^{\prime\prime} \\
			\hat{a}_2^{\prime\prime} \\
			\hat{a}_3^{\prime\prime} \\
			\hat{a}_4^{\prime\prime}
		\end{pmatrix}
		\\
		&=
		\begin{pmatrix}
			 1 & 0 & 0 & 0 \\
			 0 & \multicolumn{2}{c}{\multirow{2}{*}{$U_\text{BS2}U_\text{PS}$}} & 0 \\
			 0 & & & 0 \\
			 0 & 0 & 0 & 1
		\end{pmatrix}
		\begin{pmatrix}
			 \multicolumn{2}{c}{\multirow{2}{*}{$U_\text{MZM1}$}} & 0 & 0 \\
			 & & 0 & 0 \\
			 0 & 0 & \multicolumn{2}{c}{\multirow{2}{*}{$U_\text{MZM2}$}} \\
			 0 & 0 & & \\
		\end{pmatrix}
		\begin{pmatrix}
			\hat{a}_1^\prime \\
			\hat{a}_2^\prime \\
			\hat{a}_3^\prime \\
			\hat{a}_4^\prime
		\end{pmatrix}
		\\
		&=
		\begin{pmatrix}
			 1 & 0 & 0 & 0 \\
			 0 & \multicolumn{2}{c}{\multirow{2}{*}{$U_\text{BS2}U_\text{PS}$}} & 0 \\
			 0 & & & 0 \\
			 0 & 0 & 0 & 1
		\end{pmatrix}
		\begin{pmatrix}
			 \multicolumn{2}{c}{\multirow{2}{*}{$U_\text{MZM1}(\theta_1)$}} & 0 & 0 \\
			 & & 0 & 0 \\
			 0 & 0 & \multicolumn{2}{c}{\multirow{2}{*}{$U_\text{MZM2}(\theta_2)$}} \\
			 0 & 0 & & \\
		\end{pmatrix}
		\begin{pmatrix}
			 1 & 0 & 0 & 0 \\
			 0 & \multicolumn{2}{c}{\multirow{2}{*}{$U_\text{BS1}$}} & 0 \\
			 0 & & & 0 \\
			 0 & 0 & 0 & 1
		\end{pmatrix}
		\begin{pmatrix}
			\hat{a}_1 \\
			\hat{a}_2 \\
			\hat{a}_3 \\
			\hat{a}_4
		\end{pmatrix}
	\end{split}
\end{equation}

\begin{equation}
	\vb{\hat{a}}^{\prime\prime\prime}
	=
	U_\text{IQM}
	=
	\left[
		\mathbb{I}_1
		\otimes
		U_\text{BS2}
		U_\text{PS}(\phi)
		\otimes
		\mathbb{I}_1
	\right]
	\left[
		U_\text{MZM1}(\theta_1)
		\otimes
		U_\text{MZM2}(\theta_2)
	\right]
	\left[
		\mathbb{I}_1
		\otimes
		U_\text{BS1}
		\otimes
		\mathbb{I}_1
	\right]
	\vb{\hat{a}}
\end{equation}

\begin{equation}
	\begin{split}
		U_\text{IQM}(\theta_1,\theta_2,\phi)
		=
		\\		
		\begin{pmatrix}
			\cos\theta_1 & \frac{1}{\sqrt{2}}\sin\theta_1 & \frac{1}{\sqrt{2}}\sin\theta_1 & 0 \\
			\frac{1}{\sqrt{2}}\sin\theta_1e^{-i\phi/2} & \frac{1}{2}\cos\theta_1e^{+i\phi/2}-\frac{1}{2}\cos\theta_2 & \frac{1}{2}\cos\theta_1e^{+i\phi/2}+\frac{1}{2}\cos\theta_2 & \frac{1}{\sqrt{2}}\sin\theta_2 \\
			\frac{1}{\sqrt{2}}\sin\theta_1 & -\frac{1}{2}\cos\theta_1-\frac{1}{2}\cos\theta_2e^{-i\phi/2} & -\frac{1}{2}\cos\theta_1+\frac{1}{2}\cos\theta_2e^{-i\phi/2} & \frac{1}{\sqrt{2}}\sin\theta_2e^{-i\phi2} \\
			0 & \frac{1}{\sqrt{2}}\sin\theta_2 & -\frac{1}{\sqrt{2}}\sin\theta_2 & \cos\theta_2
		\end{pmatrix}
	\end{split}
\end{equation}

\begin{equation}
	\begin{split}
		\hat{U}_\text{IQM}(\theta_1,\theta_2,\phi)
		\ket{0,\alpha,0,0}
		&=
		\ket{\frac{1}{\sqrt{2}}\alpha,\frac{1}{2}\left[\cos\theta_1e^{+i\phi/2}-\cos\theta_2\right],-\frac{1}{2}\left[\cos\theta_1-\cos\theta_2e^{-i\phi/2}\right],\frac{1}{\sqrt{2}}\sin\theta_2}
	\end{split}
\end{equation}

\begin{align}
	U_\text{BS1}
	=
	U_\text{BS2}
	&=
	U_\text{MZM}(\pi)
	&
	U_\text{PS}
	&=
	\frac{1}{\sqrt{2}}
	\begin{pmatrix}
		e^{+i\phi/2} & 0 \\
		0 & e^{-i\phi/2}
	\end{pmatrix}
\end{align}

\begin{equation}
	\hat{U}_\text{IQ}(\theta_1,\theta_2,\phi)
	=
	e^{i\frac{\pi}{2}\hat{L}_y^{(2,3)}}
	e^{i\theta_1\hat{L}_y^{(1,2)}}
	e^{i\theta_2\hat{L}_y^{(3,4)}}
	e^{i\phi\hat{L}_z^{(2,3)}}
	e^{i\frac{\pi}{2}\hat{L}_y^{(2,3)}}
\end{equation}
wherein~\cite{Leonhardt2010}
\begin{align}
	\hat{L}_y^{(1,2)}
	&=
	\frac{i}{2}
	\left(
		\hat{a}_2^\dagger
		\hat{a}_1
		-
		\hat{a}_1^\dagger
		\hat{a}_2
	\right)
	\\
	\hat{L}_y^{(2,3)}
	&=
	\frac{i}{2}
	\left(
		\hat{a}_3^\dagger
		\hat{a}_2
		-
		\hat{a}_2^\dagger
		\hat{a}_3
	\right)
	\\
	\hat{L}_y^{(3,4)}
	&=
	\frac{i}{2}
	\left(
		\hat{a}_4^\dagger
		\hat{a}_3
		-
		\hat{a}_3^\dagger
		\hat{a}_4
	\right)
	\\
	\hat{L}_z^{(2,3)}
	&=
	\frac{1}{2}
	\left(
		\hat{a}_2^\dagger
		\hat{a}_2
		-
		\hat{a}_3^\dagger
		\hat{a}_3
	\right)
\end{align}

\begin{equation}
	\begin{split}
		\hat{U}_\text{IQ}^\dagger(\theta_1,\theta_2,\phi)
		&=
		e^{-i\frac{\pi}{2}\hat{L}_y^{(2,3)}}
		e^{-i\phi\hat{L}_z^{(2,3)}}
		e^{-i\theta_2\hat{L}_y^{(3,4)}}
		e^{-i\theta_1\hat{L}_y^{(1,2)}}
		e^{-i\frac{\pi}{2}\hat{L}_y^{(2,3)}}
	\end{split}
\end{equation}

\begin{equation}
	\begin{split}
		\comm{\hat{L}_z^{(2,3)}}{\hat{L}_y^{(1,2)}}
		&=
		\frac{i}{4}
		\comm{
			\hat{a}_2^\dagger
			\hat{a}_2
			-
			\hat{a}_3^\dagger
			\hat{a}_3
		}{
			\hat{a}_2^\dagger
			\hat{a}_1
			-
			\hat{a}_1^\dagger
			\hat{a}_2
		}
		\\
		&=
		\frac{i}{4}
		\comm{
			\hat{a}_2^\dagger
			\hat{a}_2
		}{
			\hat{a}_2^\dagger
			\hat{a}_1
		}
		-
		\frac{i}{4}
		\comm{
			\hat{a}_2^\dagger
			\hat{a}_2
		}{
			\hat{a}_1^\dagger
			\hat{a}_2
		}
		\\
		&=
		\frac{i}{4}
		\hat{a}_2^\dagger
		\comm{\hat{a}_2}{\hat{a}_2^\dagger}
		\hat{a}_1
		-
		\frac{i}{4}
		\hat{a}_1^\dagger
		\comm{\hat{a}_2^\dagger}{\hat{a}_2}
		\hat{a}_2
		\\
		&=
		\frac{1}{2}
		\frac{i}{2}
		\left(
			\hat{a}_2^\dagger
			\hat{a}_1
			+
			\hat{a}_1^\dagger
			\hat{a}_2
		\right)
		\\
		&=
		\frac{i}{2}\hat{L}_x^{(1,2)}
	\end{split}
\end{equation}
\begin{equation}
	\begin{split}
		\comm{\hat{L}_z^{(2,3)}}{\hat{L}_y^{(3,4)}}
		&=
		\frac{i}{4}
		\comm{
			\hat{a}_2^\dagger
			\hat{a}_2
			-
			\hat{a}_3^\dagger
			\hat{a}_3
		}{
			\hat{a}_4^\dagger
			\hat{a}_3
			-
			\hat{a}_3^\dagger
			\hat{a}_4		
		}
		\\
		&=
		\frac{i}{4}
		\comm{
			\hat{a}_3^\dagger
			\hat{a}_3
		}{
			\hat{a}_3^\dagger
			\hat{a}_4		
		}
		-
		\frac{i}{4}
		\comm{
			\hat{a}_3^\dagger
			\hat{a}_3
		}{
			\hat{a}_4^\dagger
			\hat{a}_3
		}
		\\
		&=
		\frac{i}{2}
		\frac{1}{2}
		\left(
			\hat{a}_3^\dagger
			\hat{a}_4
			+
			\hat{a}_4^\dagger
			\hat{a}_3
		\right)
		\\
		&=
		\frac{i}{2}
		\hat{L}_x^{(3,4)}
	\end{split}
\end{equation}

\begin{equation}
	\begin{split}
		\hat{U}_\text{IQ}(\theta_1,\theta_2,\phi)
		\hat{a}_1^\dagger
		\hat{U}_\text{IQ}^\dagger(\theta_1,\theta_2,\phi)
		&=
		e^{+i\frac{\pi}{2}\hat{L}_y^{(2,3)}}
		e^{+i\theta_1\hat{L}_y^{(1,2)}}
		e^{+i\theta_2\hat{L}_y^{(3,4)}}
		e^{+i\phi\hat{L}_z^{(2,3)}}
		e^{+i\frac{\pi}{2}\hat{L}_y^{(2,3)}}
		\\
		&\times
		\hat{a}_1^\dagger
		e^{-i\frac{\pi}{2}\hat{L}_y^{(2,3)}}
		e^{-i\phi\hat{L}_z^{(2,3)}}
		e^{-i\theta_2\hat{L}_y^{(3,4)}}
		e^{-i\theta_1\hat{L}_y^{(1,2)}}
		e^{-i\frac{\pi}{2}\hat{L}_y^{(2,3)}}
	\end{split}
\end{equation}

\subsection{Pockels effect}

% figure (?): what time scales are there, what area of physics do they belong to, are these relevant (?)
% non-vanishing commutators inside the light-cone (QFT), see Tohen-Canoudji
% effective potential in the interaction encoding time effects
% time-dependency of these effects

% discussion of Pockels effect
From a purely phenomenological point of view, the Pockels effect - also known as the linear electro-optical effect - describes a linear change of the refractive index of a dielectric material in the presence of an external electric field $E$, i.e.,
\begin{equation}
	n(E)
	=
	n^{(0)}
	+
	n^{(1)}E
\end{equation}
wherein $n^{(0)}=n(0)$ is the refractive index without an external field and $n^{(1)}$ is a proportionality constant.
One way to create the external electric field $E$ is to place the Pockels dielectric inside a plate capacitor and apply a voltage $V$ to the plates.
Neglecting boundary effects, we find a homogeneous static electric field between the plates of $E=V/d$ where $d$ is the plate distance.
Such a configuration is known as Pockels cell and depicted in \Cref{fig:pockels_cell} where the field inside the Pockels cell is labeled $E_x(t)$.
\begin{figure}[htb]
    \centering
    \includegraphics{figures/tikz/pockels-cell}
    \caption{Pockels cell of length $l$ and thickness $d$ embedded in a waveguide with constant dielectric permittivity $\varepsilon_1$: The dielectric permittivity of the Pockels cell $\varepsilon_2(t)$ depends on the electric field $E_x(t)$ across the Pockels cell plates. The optical field $E_z(t)$ enters the Pockels cell from the left and leaves it to the right as $E^\prime_z(t)$.}\label{fig:pockels_cell}
\end{figure}
Let us first consider a monochromatic electromagnetic wave of frequency $\omega_0$ propagating through a dielectric of length $l$ with refractive index $n_1^{(0)}=\sqrt{\varepsilon_1}$.
Inside the dielectric, the wave propagates with phase velocity $c/n_1^{(0)}$ and takes $T_1=ln_1^{(0)}/c$ to pass through the dielectric.
Directly after the transit, the wave accumulated a total phase shift of
\begin{equation*}
	\phi_1
	=
	\omega_0T_1
	=
	2\pi\frac{n_1^{(0)}l}{\lambda_0}
\end{equation*}
where $\lambda_0=c/f$ is the vacuum wave length.
If we now consider the same monochromatic electromagnetic wave propagating through a Pockels cell of the same length $l$ but without applied voltage, the wave would accumulate a phase shift of
\begin{equation}
	\phi_2
	=
	\omega_0T_2
	=
	2\pi\frac{n_2^{(0)}l}{\lambda_0}
	.
\end{equation}
So even when there is no electric field inside the Pockels cell, we find a phase difference of
\begin{equation}
	\Delta\phi
	=
	\phi_2
	-
	\phi_1
	=
	2\pi\frac{n_2^{(0)}-n_1^{(0)}}{\lambda_0}l
\end{equation}
because of the different dielectric constants of the materials.
If we apply a voltage to the Pockels cell, the refractive index of the Pockels dielectric contributes a second phase shift of
\begin{equation}
	\varphi_2
	=
	2\pi\frac{n_2^{(1)}E}{\lambda_0}l
\end{equation}
and the total phase difference is $\varphi_2+\Delta\phi$.
Extending our discussion to wave packets in dispersive media, we need to replace the phase velocity $v_p$ with the group velocity $v_g$.
The group velocity can be expressed in terms of the refractive index via~\cite[p.~211]{Jackson2007}
\begin{equation}
	v_g(\omega_0)
	=
	\left[
		n
		+
		\omega
		\pdv{n}{\omega}
	\right]^{-1}_{\omega=\omega_0}
\end{equation}
where $\omega_0$ is the center frequency of the wave packet.
The concept of group velocity can be extended to quantum states, see, for instance, Ref.~\cite[p.~3]{Naumov2013}.
We refrain from a further discussion which would require specific knowledge of the wave packet's pulse shape and frequency-dependency of the refractive index and is best simulated using finite element methods.

Thus far, we have treated a static electric field in the Pockels cell.
We are now going to extend our model to allow for time-dependent electric field inside the Pockels cell and justify why we can treat the time-dependency of the modulation field to be effectively static.
An external electric field induces a dipole moment in the constituents of a dielectric.
The average dipole moment per volume is referred to the macroscopic polarization $\vb{P}$.
Assuming the dielectric to be a time-invariant system, i.e., to be memoryless, we can expand each component of the macroscopic polarization $P^i$ up to second-order in terms of electric susceptibility tensors $\chi$~\cite[p.~17]{Murti2014}
\begin{equation}
	P_i(t)
	=
	\int_{\mathbb{R}}\dd{t^\prime}
	\chi^{(1)}_{ij}(t-t^\prime)
	E^j(t^\prime)
	+
	\iint_{\mathbb{R}^2}\dd{t^\prime}\dd{t^{\prime\prime}}
	\chi^{(2)}_{ijk}(t-t^\prime,t-t^{\prime\prime})
	E^j(t^\prime)E^k(t^{\prime\prime})
\end{equation}
where causality demands $\chi^{(n)}(t-t^\prime)=0$ for $t^\prime>t$.
Inserting the Fourier representation, we find
\begin{align}
	P_i^{(1)}(t)
	&=
	\int_{\mathbb{R}}
	\frac{\dd{\omega_1}}{2\pi}
	\chi^{(1)}_{ij}(\omega_1)
	E^j(\omega_1)
	e^{+i\omega_1t}
	\\
	P_i^{(2)}(t)
	&=
	\iint_{\mathbb{R}^2}
	\frac{\dd{\omega_1}}{2\pi}
	\frac{\dd{\omega_2}}{2\pi}
	\chi^{(2)}_{ijk}(\omega_1,\omega_2)
	E^j(\omega_1)E^k(\omega_2)
	e^{+i(\omega_1+\omega_2)t}	
\end{align}
for the first two terms of the expansion.
The inverse Fourier transform yields the macroscopic polarization in the frequency domain~\cite[p.~1070]{Mandel1995} yields for the first term
\begin{equation}
	P_i^{(1)}(\omega)
	=
	\int\dd{\omega^\prime}
	\chi^{(1)}_{ij}(\omega^\prime)
	E^j(\omega^\prime)
	\delta^{(1)}(\omega-\omega^\prime)
	=
	\chi^{(1)}_{ij}(\omega)
	E^j(\omega)
\end{equation}
which we will later identify as part of the constant refractive index with $E^j$ being the optical field passing through the Pockels cell
For the second term we find
\begin{equation}
	\begin{split}
		P_i^{(2)}(\omega)
		&=
		\iint\frac{\dd{\omega^\prime}\dd{\omega^{\prime\prime}}}{2\pi}
		\chi^{(2)}_{ijk}(\omega^\prime,\omega^{\prime\prime})
		E^j(\omega^\prime)
		E^k(\omega^{\prime\prime})
		(2\pi)	
		\delta^{(1)}(\omega-\omega^\prime-\omega^{\prime\prime})
		\\
		&=
		\int\frac{\dd{\omega^\prime}}{2\pi}
		\chi^{(2)}_{ijk}(\omega-\omega^\prime,\omega^\prime)
		E^j(\omega-\omega^\prime)
		E^k(\omega^\prime)
	\end{split}
\end{equation}
where we will later identify $E^k$ to be the radio frequency field driving the Pockels cell.
The specific form indicates the presence of frequency sidebands by modulation.
Let's assume radio frequency field $E^k$ to have non-zero support, i.e., bandwidth, $\Delta\Omega$, then according to the mean value theorem
\begin{equation}
	\begin{split}
		P_i^{(2)}(\omega)
		&=
		\int_{\Delta\Omega}\frac{\dd{\omega^\prime}}{2\pi}
		\chi^{(2)}_{ijk}(\omega-\omega^\prime,\omega^\prime)
		E^j(\omega-\omega^\prime)
		E^k(\omega^\prime)
		\\
		&=
		\frac{\Delta\Omega}{2\pi}
		\chi^{(2)}_{ijk}(\omega-\Omega_0,\Omega_0)
		E^j(\omega-\Omega_0)
		E^k(\Omega_0)
	\end{split}
\end{equation}
where $\Omega_0$ is the mean frequency of $E^k$.
Restricting $P_i(\omega)$ to the optical domain, we have $\omega\gg\Omega_0$ which let's us Taylor expand $E^j(\omega-\Omega_0)$ around $\omega$, i.e.,
\begin{equation}
	P_i^{(2)}(\omega)
	\approx
	\frac{\Delta\Omega}{2\pi}
	\chi^{(2)}_{ijk}(\omega,\Omega_0)
	\left(
		E^j(\omega)
		+
		\pdv{E^j}{\omega}(\omega)\Omega_0
	\right)
	E^k(\Omega_0)
\end{equation}
where we assumed a flat response $\pdv{\chi_{ijk}^{(2)}}{\Omega_0}\approx0$.
Finally, we redefine the electric susceptibility tensor such that
\begin{equation}
	P_i(\omega)
	\approx
	\left(
		\tilde{\chi}^{(1)}_{ij}(\omega)
		+
		\tilde{\chi}^{(2)}_{ijk}(\omega)
		E^k(\Omega_0)
	\right)
	E^j(\omega)
\end{equation}
which is valid if $\omega$ is an optical frequency.
The dielectric permittivity tensor $\varepsilon_{ij}$ is defined implicitly through the displacement field
\begin{equation}
	D_i(\omega)
	=
	E_i(\omega)
	+
	P_i(\omega)
	=
	\varepsilon_{ij}(\omega)
	E^j(\omega)
\end{equation}
and we identify by comparison the dielectric permittivity tensor to relate to the electric susceptibility by
\begin{equation}
	\varepsilon_{ij}(\omega)
	=
	1
	+
	\tilde{\chi}^{(1)}_{ij}(\omega)
	+
	\tilde{\chi}^{(2)}_{ijk}(\omega)
	E^k(\Omega_0)
	.
\end{equation}
The refractive index tensor for a dielectric, non-magnetic and non-chiral medium is given after a series expansion by~\cite{Rerat2020}
\begin{equation}
	n_{ij}(\omega)
	=
	\sqrt{\varepsilon_{ij}(\omega)}
	\approx
	n^{(0)}_{ij}(\omega)
	+
	n^{(1)}_{ijk}(\omega)
	E^k(\Omega_0)
\end{equation}
where we defined the refractive index tensors
\begin{align}
	n^{(0)}_{ij}(\omega)
	=
	\sqrt{1+\tilde{\chi}^{(1)}_{ij}(\omega)}
	&&
	n^{(1)}_{ijk}(\omega)
	=
	\frac{\tilde{\chi}^{(2)}_{ijk}(\omega)
	E^k(\Omega_0)}{2\sqrt{1+\tilde{\chi}^{(1)}_{ij}(\omega)}}
\end{align}
which recovers the linear electro-optical effect we first discussed in the phenomenological approach.
For the Pockels cell depicted in \Cref{fig:pockels_cell}, the refractive index relevant for the optical field takes the explicit form
\begin{equation}
	n_{zz}(\omega)
	\approx
	n^{(0)}_{zz}(\omega)
	+
	n^{(1)}_{zzx}(\omega)
	E^x(\Omega_0)
	.
\end{equation}
For practical Pockels media, e.g., Lithium-Niobate, the tensorial \gls{dof} can be substantially reduced by considering crystal symmetries, see, for instance, Ref.~\cite[p.~237]{Yariv1984}.

We derived the relation between the macroscopic polarization, response and refractive index and justified why we can treat the modulation field to be effectively static compared to the fast oscillating optical field.
For the last step where we summarize the unitary transformation associated with a Pockels phase modulator, we assume the dielectric to be constant on the time scale of the electrical field.

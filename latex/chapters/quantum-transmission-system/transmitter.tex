\section{Transmitter}

\subsection{Mach-Zehnder modulator}

\begin{figure}[htb]
	\centering
	\includestandalone{figures/pstricks/mzi-symmetric}
	\caption{Free-space setup of a symmetric \gls{mzi}: The input light field enters a first beam splitter BS1 from the left. The light field exits BS1 to the right and the bottom. Right of BS1, a first phase shifter adds a relative phase of $\varphi_1$. Right of the first phase shifter, a first mirror M1 reflects the light to the bottom, hitting a second beam splitter BS2 from the top. Below BS1, a second mirror M2 directs the light to the right, where a second phase shifter adds a relative phase of $\varphi_2$, and the light hits BS2 from the left.}
\end{figure}
\begin{equation}
	\begin{split}
		\begin{pmatrix}
			\hat{a}_1^\prime \\
			\hat{a}_2^\prime
		\end{pmatrix}
		&=
		\frac{1}{\sqrt{2}}
		\begin{pmatrix}
			i & 1 \\
			1 & i
		\end{pmatrix}
		\begin{pmatrix}
			ie^{i\varphi_1} & 0 \\
			0 & ie^{i\varphi_2}
		\end{pmatrix}
		\frac{1}{\sqrt{2}}
		\begin{pmatrix}
			1 & i \\
			i & 1
		\end{pmatrix}
		\begin{pmatrix}
			\hat{a}_1 \\
			\hat{a}_2
		\end{pmatrix}
		\\
		&=
		e^{i\frac{\varphi_1+\varphi_2}{2}+i\pi}
		\begin{pmatrix}
			\cos(\frac{\varphi_2-\varphi_1}{2}) & \sin(\frac{\varphi_2-\varphi_1}{2}) \\
			-\sin(\frac{\varphi_2-\varphi_1}{2}) & \cos(\frac{\varphi_2-\varphi_1}{2})
		\end{pmatrix}
		\begin{pmatrix}
			\hat{a}_1 \\
			\hat{a}_2
		\end{pmatrix}
	\end{split}
\end{equation}

\begin{table}[htb]
	\centering
	\begin{tabular}{lcc}
		\toprule
		Configuration & Phase relation & Modulation \\
		\midrule
		Push-pull & $\varphi_1=-\varphi_2$ & Amplitude \\
		Push-push & $\varphi_1=+\varphi_2$ & Phase \\
		\bottomrule
	\end{tabular}
	\caption{Configurations of a symmetric \gls{mzm}.}
\end{table}

\begin{equation}
	\begin{pmatrix}
		\hat{a}_1^\prime \\
		\hat{a}_2^\prime
	\end{pmatrix}
		\begin{pmatrix}
			\cos(\theta) & \sin(\theta) \\
			-\sin(\theta) & \cos(\theta)
		\end{pmatrix}
		\begin{pmatrix}
			\hat{a}_1 \\
			\hat{a}_2
		\end{pmatrix}
\end{equation}

\begin{equation}
	\hat{U}(\theta)
	=
	e^{i\theta\left(
		\hat{a}_1
		\hat{a}_2^\dagger
		-
		\hat{a}_1^\dagger
		\hat{a}_2
	\right)}
\end{equation}

\begin{align}
	\hat{a}_1^\prime
	&=
	\hat{U}^\dagger(\theta)
	\hat{a}_1
	\hat{U}(\theta)
	=
	\cos(\theta)
	\hat{a}_1
	+
	\sin(\theta)
	\hat{a}_2
	\\
	\hat{a}_2^\prime
	&=
	\hat{U}^\dagger(\theta)
	\hat{a}_2
	\hat{U}(\theta)
	=
	\cos(\theta)
	\hat{a}_2
	-
	\sin(\theta)
	\hat{a}_1
\end{align}

\subsection{I/Q modulator}

\begin{figure}[htb]
	\centering
	\includestandalone{figures/pstricks/iqm}
	\caption{\gls{iq}-modulator comprising three \gls{mzm} and connected by an input and output fiber. MZM 1 and MZM 2, are used in push-pull configuration for \gls{am}. MZM 3 is used in push-push configuration for setting the relative phase between the upper and lower branch. Vacuum in- and outputs are indicated by the red dashed fiber.}
\end{figure}

\begin{equation}
	\begin{pmatrix}
		\hat{a}_1^\prime \\
		\hat{a}_2^\prime
	\end{pmatrix}
	=
	\frac{1}{\sqrt{2}}
	\begin{pmatrix}
		 1 & ie^{+i\varphi_1} \\
		 ie^{-i\varphi_1} & 1
	\end{pmatrix}
	\begin{pmatrix}
		\hat{a}_1 \\
		\hat{a}_2
	\end{pmatrix}
\end{equation}

\begin{equation}
	\begin{pmatrix}
		\hat{a}_3^{\prime\prime} \\
		\hat{a}_1^{\prime\prime}
	\end{pmatrix}
	=
	\begin{pmatrix}
		 \cos\theta_1 & \sin\theta_1 \\
		 -\sin\theta_1 & \cos\theta_1
	\end{pmatrix}
	\begin{pmatrix}
		\hat{a}_3^\prime \\
		\hat{a}_1^\prime
	\end{pmatrix}
\end{equation}

\begin{equation}
	\begin{pmatrix}
		\hat{a}_1^{\prime\prime} \\
		\hat{a}_2^{\prime\prime} \\
		\hat{a}_3^{\prime\prime} \\
		\hat{a}_4^{\prime\prime}
	\end{pmatrix}
	=
	\begin{pmatrix}
		 \cos\theta_1 & 0 & -\sin\theta_1 & 0 \\
		 0 & \cos\theta_2 & 0 & \sin\theta_2 \\
		 \sin\theta_1 & 0 & \cos\theta_1 & 0 \\
		 0 & -\sin\theta_2 & 0 & \cos\theta_2 \\
	\end{pmatrix}
	\begin{pmatrix}
		\hat{a}_1^{\prime} \\
		\hat{a}_2^{\prime} \\
		\hat{a}_3^{\prime} \\
		\hat{a}_4^{\prime}
	\end{pmatrix}
\end{equation}

\begin{equation}
	\begin{pmatrix}
		\hat{a}_2^{\prime\prime} \\
		\hat{a}_4^{\prime\prime}
	\end{pmatrix}
	=
	\begin{pmatrix}
		 \cos\theta_2 & \sin\theta_2 \\
		 -\sin\theta_2 & \cos\theta_2
	\end{pmatrix}
	\begin{pmatrix}
		\hat{a}_2^\prime \\
		\hat{a}_4^\prime
	\end{pmatrix}
\end{equation}


\begin{equation}
	\begin{split}
		\begin{pmatrix}
			\hat{a}_1^{\prime\prime\prime\prime} \\
			\hat{a}_2^{\prime\prime\prime\prime}
		\end{pmatrix}
		&=
		\frac{1}{\sqrt{2}}
		\begin{pmatrix}
			1 & ie^{+i\varphi_2} \\
			ie^{-i\varphi_2} & 1
		\end{pmatrix}
		\begin{pmatrix}
			\hat{a}_1^{\prime\prime\prime} \\
			\hat{a}_2^{\prime\prime\prime}
		\end{pmatrix}
		\\
		&=
		\frac{1}{\sqrt{2}}
		\begin{pmatrix}
			1 & ie^{+i\varphi_2} \\
			ie^{-i\varphi_2} & 1
		\end{pmatrix}
		\begin{pmatrix}
			e^{+i\phi} & 0 \\
			0 & e^{-i\phi}
		\end{pmatrix}
		\begin{pmatrix}
			\hat{a}_1^{\prime\prime} \\
			\hat{a}_2^{\prime\prime}
		\end{pmatrix}
		\\
		&=
		\frac{1}{\sqrt{2}}
		\begin{pmatrix}
			e^{+i\phi} & ie^{+i(\varphi_2-\phi)} \\
			ie^{-i(\varphi_2-\phi)} & e^{-i\phi}
		\end{pmatrix}
		\begin{pmatrix}
			\hat{a}_1^{\prime\prime} \\
			\hat{a}_2^{\prime\prime}
		\end{pmatrix}
	\end{split}
\end{equation}
\section{Transmitter}

% TODO: overview figure of how we show "equivalence" between the tensor product of coherent states with the time-continuous coherent state for the transmitter

\subsection{Pulse-shaping}

Pulse-shaping turns a sequence of (complex) symbols into a time-continuous signal.
For practical implementations, the pulse-shaping is performed in the digital domain.
Moving the pulse-shaping to the optical domain allows for a quantum mechanical description.
\begin{figure}[htb]
	\centering
    \includestandalone{figures/pstricks/quantum-pulse-shaping}
    \caption{Optical setup to perform pulse-shaping in the (quantum) optical domain: First, a single coherent state is modulated and a time delay corresponding to the symbol index is added. Second, the pulses are superimposed and optically filtered.}
\end{figure}

The \gls{iq}-modulated coherent state in the Schrödinger picture is
\begin{equation}
	\ket{\alpha(p)}
	=
	e^{-\overline{n}/2}
	\exp\left\{
		\int_0^\infty\dd{p}
		\alpha(p)
		\hat{a}^\dagger(p)
	\right\}
	\ket{0}
\end{equation}
wherein $\overline{n}$ is the mean photon number and
\begin{equation}
	\alpha(p)
	\propto
	\exp\left\{
		-\frac{1}{4}\left(\frac{p-p_0}{\sigma}\right)^2
	\right\}
\end{equation}
is a (complex) Gaussian spectrum centered at $p_0$ with spread $\sigma$.
In the limit of an extremely peaked Gaussian spectrum $\sigma\to0$, the Gaussian spectrum becomes a delta distribution
\begin{equation}
	\alpha(p)
	\xrightarrow{\sigma\to0}
	\alpha(p_0)
	\delta^{(1)}(p-p_0)
\end{equation}
and we obtain the single-mode coherent state
\begin{equation}
	\ket{\alpha(p_0)}
	=
	e^{-\overline{n}/2}
	\exp\left\{
		\alpha(p_0)
		\hat{a}^\dagger(p_0)
	\right\}
	\ket{0}
\end{equation}
usually used in quantum information theory.

% step 1: convert frequency- to time-pulse
% use a frequency-dependent beam splitter as optical filter?

% step 2: add time delay to time-pulse

% step 3: superimpose (delayed) time-pulses

% step 4: perform pulse-shaping

% step 5: perform up-conversion using sum-to-frequency generation

\FloatBarrier
\subsection{Up-conversion}

% step 6: perform attenuation using beam splitter
\section{Transmitter}

The transmitter consists of a coherent light source, an \gls{iqm}, and a \gls{voa}.

Practically, a laser implements the coherent light source.
Advanced quantum stochastic models predicting laser characteristics (gain, stability, coherence, linewidth) are established and found in  Ref.~\cite[p.~900]{Mandel1995} and Ref.~\cite{Haken2012}.
For our purpose, it is sufficient to know that lasers emit coherent states with a frequency spectrum.\footnote{For an intuitive argument why laser emit coherent states based on decoherence, see Ref.~\cite{Gea1998}.}

The \gls{voa} attenuates the \gls{iq}-modulated signal such that the signal power becomes comparable with the quantum noise.
It is straightforward to model the \gls{voa}'s with a beam splitter transform, such that we will not draw further attention to it.

The \gls{iq} modulation is the most important step in the transmitter as it indicates the borderline between the classical and quantum signals.
It is also the most mysterious as we are unaware of any published quantum mechanical investigation.
The modulation occurs at three stages: the Pockels (phase) modulator, the \gls{mzm} (amplitude) modulator, and the \gls{iqm} modulator.
The transition from the classical signal to the quantum state occurs at the phase modulation, while the MZM and \gls{iqm} modulator operate on the quantum state.

\subsection{Pockels modulator}

% discussion of Pockels effect
From a purely phenomenological point of view, the Pockels effect - also known as the linear electro-optical effect - describes a linear change of the refractive index of a dielectric material in the presence of an external electric field $E$, i.e.,
\begin{equation}
	n(E)
	=
	n^{(0)}
	+
	n^{(1)}E
\end{equation}
wherein $n^{(0)}=n(0)$ is the refractive index without an external field and $n^{(1)}$ is a proportionality constant.
One way to create the external electric field $E$ is to place the Pockels dielectric inside a plate capacitor and apply a voltage $V$ to the plates.
Neglecting boundary effects, we find a homogeneous static electric field between the plates of $E=V/d$ where $d$ is the plate distance.
Such a configuration is known as Pockels cell and depicted in \Cref{fig:pockels_cell} where the field inside the Pockels cell is labeled $E_x(t)$.
\begin{figure}[htb]
    \centering
    \includegraphics{figures/tikz/pockels-cell}
    \caption{Pockels cell of length $l$ and thickness $d$ embedded in a waveguide with constant dielectric permittivity $\varepsilon_1$: The dielectric permittivity of the Pockels cell $\varepsilon_2(t)$ depends on the electric field $E_x(t)$ across the Pockels cell plates. The optical field $E_z(t)$ enters the Pockels cell from the left and leaves it to the right as $E^\prime_z(t)$.}\label{fig:pockels_cell}
\end{figure}
Let us first consider a monochromatic electromagnetic wave of frequency $\omega_0$ propagating through a dielectric of length $l$ with refractive index $n_1^{(0)}=\sqrt{\varepsilon_1}$.
Inside the dielectric, the wave propagates with phase velocity $c/n_1^{(0)}$ and takes $T_1=ln_1^{(0)}/c$ to pass through the dielectric.
Directly after the transit, the wave accumulated a total phase shift of
\begin{equation*}
	\phi_1
	=
	\omega_0T_1
	=
	2\pi\frac{n_1^{(0)}l}{\lambda_0}
\end{equation*}
where $\lambda_0=c/f$ is the vacuum wave length.
If we now consider the same monochromatic electromagnetic wave propagating through a Pockels cell of the same length $l$ but without applied voltage, the wave would accumulate a phase shift of
\begin{equation}
	\phi_2
	=
	\omega_0T_2
	=
	2\pi\frac{n_2^{(0)}l}{\lambda_0}
	.
\end{equation}
So even when there is no electric field inside the Pockels cell, we find a phase difference of
\begin{equation}
	\Delta\phi
	=
	\phi_2
	-
	\phi_1
	=
	2\pi\frac{n_2^{(0)}-n_1^{(0)}}{\lambda_0}l
\end{equation}
because of the different dielectric constants of the materials.
If we apply a voltage to the Pockels cell, the refractive index of the Pockels dielectric contributes a second phase shift of
\begin{equation}
	\varphi_2
	=
	2\pi\frac{n_2^{(1)}E}{\lambda_0}l
\end{equation}
and the total phase difference is $\varphi_2+\Delta\phi$.
Extending our discussion to wave packets in dispersive media, we need to replace the phase velocity $v_p$ with the group velocity $v_g$.
The group velocity can be expressed in terms of the refractive index via~\cite[p.~211]{Jackson2007}
\begin{equation}
	v_g(\omega_0)
	=
	\left[
		n
		+
		\omega
		\pdv{n}{\omega}
	\right]^{-1}_{\omega=\omega_0}
\end{equation}
where $\omega_0$ is the center frequency of the wave packet.
The concept of group velocity can be extended to quantum states, see, for instance, Ref.~\cite[p.~3]{Naumov2013}.
We refrain from a further discussion which would require specific knowledge of the wave packet's pulse shape and frequency-dependency of the refractive index and is best simulated using finite element methods.

Thus far, we have treated a static electric field in the Pockels cell.
We are now going to extend our model to allow for time-dependent electric field inside the Pockels cell and justify why we can treat the time-dependency of the modulation field to be effectively static.
An external electric field induces a dipole moment in the constituents of a dielectric.
The average dipole moment per volume is referred to the macroscopic polarization $\vb{P}$.
Assuming the dielectric to be a time-invariant system, i.e., to be memoryless, we can expand each component of the macroscopic polarization $P^i$ up to second-order in terms of electric susceptibility tensors $\chi$~\cite[p.~17]{Murti2014}
\begin{equation}
	P_i(t)
	=
	\int_{\mathbb{R}}\dd{t^\prime}
	\chi^{(1)}_{ij}(t-t^\prime)
	E^j(t^\prime)
	+
	\iint_{\mathbb{R}^2}\dd{t^\prime}\dd{t^{\prime\prime}}
	\chi^{(2)}_{ijk}(t-t^\prime,t-t^{\prime\prime})
	E^j(t^\prime)E^k(t^{\prime\prime})
\end{equation}
where causality demands $\chi^{(n)}(t-t^\prime)=0$ for $t^\prime>t$.
Inserting the Fourier representation, we find
\begin{align}
	P_i^{(1)}(t)
	&=
	\int_{\mathbb{R}}
	\frac{\dd{\omega_1}}{2\pi}
	\chi^{(1)}_{ij}(\omega_1)
	E^j(\omega_1)
	e^{+i\omega_1t}
	\\
	P_i^{(2)}(t)
	&=
	\iint_{\mathbb{R}^2}
	\frac{\dd{\omega_1}}{2\pi}
	\frac{\dd{\omega_2}}{2\pi}
	\chi^{(2)}_{ijk}(\omega_1,\omega_2)
	E^j(\omega_1)E^k(\omega_2)
	e^{+i(\omega_1+\omega_2)t}	
\end{align}
for the first two terms of the expansion.
The inverse Fourier transform yields the macroscopic polarization in the frequency domain~\cite[p.~1070]{Mandel1995} yields for the first term
\begin{equation}
	P_i^{(1)}(\omega)
	=
	\int\dd{\omega^\prime}
	\chi^{(1)}_{ij}(\omega^\prime)
	E^j(\omega^\prime)
	\delta^{(1)}(\omega-\omega^\prime)
	=
	\chi^{(1)}_{ij}(\omega)
	E^j(\omega)
\end{equation}
which we will later identify as part of the constant refractive index with $E^j$ being the optical field passing through the Pockels cell
For the second term we find
\begin{equation}
	\begin{split}
		P_i^{(2)}(\omega)
		&=
		\iint\frac{\dd{\omega^\prime}\dd{\omega^{\prime\prime}}}{2\pi}
		\chi^{(2)}_{ijk}(\omega^\prime,\omega^{\prime\prime})
		E^j(\omega^\prime)
		E^k(\omega^{\prime\prime})
		(2\pi)	
		\delta^{(1)}(\omega-\omega^\prime-\omega^{\prime\prime})
		\\
		&=
		\int\frac{\dd{\omega^\prime}}{2\pi}
		\chi^{(2)}_{ijk}(\omega-\omega^\prime,\omega^\prime)
		E^j(\omega-\omega^\prime)
		E^k(\omega^\prime)
	\end{split}
\end{equation}
where we will later identify $E^k$ to be the radio frequency field driving the Pockels cell.
The specific form indicates the presence of frequency sidebands by modulation.
Let's assume radio frequency field $E^k$ to have non-zero support, i.e., bandwidth, $\Delta\Omega$, then according to the mean value theorem
\begin{equation}
	\begin{split}
		P_i^{(2)}(\omega)
		&=
		\int_{\Delta\Omega}\frac{\dd{\omega^\prime}}{2\pi}
		\chi^{(2)}_{ijk}(\omega-\omega^\prime,\omega^\prime)
		E^j(\omega-\omega^\prime)
		E^k(\omega^\prime)
		\\
		&=
		\frac{\Delta\Omega}{2\pi}
		\chi^{(2)}_{ijk}(\omega-\Omega_0,\Omega_0)
		E^j(\omega-\Omega_0)
		E^k(\Omega_0)
	\end{split}
\end{equation}
where $\Omega_0$ is the mean frequency of $E^k$.
Restricting $P_i(\omega)$ to the optical domain, we have $\omega\gg\Omega_0$ which let's us Taylor expand $E^j(\omega-\Omega_0)$ around $\omega$, i.e.,
\begin{equation}
	P_i^{(2)}(\omega)
	\approx
	\frac{\Delta\Omega}{2\pi}
	\chi^{(2)}_{ijk}(\omega,\Omega_0)
	\left(
		E^j(\omega)
		+
		\pdv{E^j}{\omega}(\omega)\Omega_0
	\right)
	E^k(\Omega_0)
\end{equation}
where we assumed a flat response $\pdv{\chi_{ijk}^{(2)}}{\Omega_0}\approx0$.
Finally, we redefine the electric susceptibility tensor such that
\begin{equation}
	P_i(\omega)
	\approx
	\left(
		\tilde{\chi}^{(1)}_{ij}(\omega)
		+
		\tilde{\chi}^{(2)}_{ijk}(\omega)
		E^k(\Omega_0)
	\right)
	E^j(\omega)
\end{equation}
which is valid if $\omega$ is an optical frequency.
The dielectric permittivity tensor $\varepsilon_{ij}$ is defined implicitly through the displacement field
\begin{equation}
	D_i(\omega)
	=
	E_i(\omega)
	+
	P_i(\omega)
	=
	\varepsilon_{ij}(\omega)
	E^j(\omega)
\end{equation}
and we identify by comparison the dielectric permittivity tensor to relate to the electric susceptibility by
\begin{equation}
	\varepsilon_{ij}(\omega)
	=
	1
	+
	\tilde{\chi}^{(1)}_{ij}(\omega)
	+
	\tilde{\chi}^{(2)}_{ijk}(\omega)
	E^k(\Omega_0)
	.
\end{equation}
The refractive index tensor for a dielectric, non-magnetic and non-chiral medium is given after a series expansion by~\cite{Rerat2020}
\begin{equation}
	n_{ij}(\omega)
	=
	\sqrt{\varepsilon_{ij}(\omega)}
	\approx
	n^{(0)}_{ij}(\omega)
	+
	n^{(1)}_{ijk}(\omega)
	E^k(\Omega_0)
\end{equation}
where we defined the refractive index tensors
\begin{align}
	n^{(0)}_{ij}(\omega)
	=
	\sqrt{1+\tilde{\chi}^{(1)}_{ij}(\omega)}
	&&
	n^{(1)}_{ijk}(\omega)
	=
	\frac{\tilde{\chi}^{(2)}_{ijk}(\omega)
	E^k(\Omega_0)}{2\sqrt{1+\tilde{\chi}^{(1)}_{ij}(\omega)}}
\end{align}
which recovers the linear electro-optical effect we first discussed in the phenomenological approach.
For the Pockels cell depicted in \Cref{fig:pockels_cell}, the refractive index relevant for the optical field takes the explicit form
\begin{equation}
	n_{zz}(\omega)
	\approx
	n^{(0)}_{zz}(\omega)
	+
	n^{(1)}_{zzx}(\omega)
	E^x(\Omega_0)
	.
\end{equation}
For practical Pockels media, e.g., Lithium-Niobate, the tensorial \gls{dof} can be substantially reduced by considering crystal symmetries, see, for instance, Ref.~\cite[p.~237]{Yariv1984}.

We derived the relation between the macroscopic polarization, response and refractive index and justified why we can treat the modulation field to be effectively static compared to the fast oscillating optical field.
For the last step where we summarize the unitary transformation associated with a Pockels phase modulator, we assume the dielectric to be constant on the time scale of the electrical field.

\subsection{Mach-Zehnder modulator}

The \gls{mzm} arranges two phase modulators to perform amplitude and phase modulation on an optical input field.
Using two electrical-driven phase modulators, the \gls{mzm} performs amplitude and phase modulation on an optical input field.
To begin with, we derive the quantum operator matrix transform from a specific implementation of the \gls{mzm}, the symmetric free-space \gls{mzi}.
Then, we link the quantum operator matrix transform to more general unitary operators.
Finally, we derive the quantum state transform for a number state as input to the \gls{mzm} and use it to derive the quantum state transform for a coherent state.

\begin{figure}[htb]
	\centering
	\includestandalone{figures/pstricks/mzi-symmetric}
	\caption{Free-space setup of a symmetric \gls{mzm}: The input light mode $\hat{a}_1$ enters a first beam splitter BS1 from the left. A vacuum light mode $\hat{a}_2$ enters BS2 from the top. The transformed mode $\hat{a}_1^\prime$ and $\hat{a}_2^\prime$ exit BS1 to the right and the bottom. A first phase shifter and first mirror M1, right to BS1, add a relative phase of $\varphi_1+\pi$ from mode $\hat{a}_1$ to $\hat{a}_1^{\prime\prime}$. Below BS1, a second mirror M2 directs the light to a right second phase shifter, both adding a relative phase of $\varphi_2+\pi$ from mode $\hat{a}_2^\prime$ to $\hat{a}_2^{\prime\prime}$. A second beam splitter BS2 transforms the input modes $\hat{a}_1^{\prime\prime}$ and $\hat{a}_2^{\prime\prime}$ to the output modes $\hat{a}_1^{\prime\prime\prime}$ and $\hat{a}_2^{\prime\prime\prime}$.}\label{fig:mzi_symmetric}
\end{figure}
\Cref{fig:mzi_symmetric} shows a free-space optics setup of a symmetric \gls{mzi} with one signal input; the other input being in the vacuum state.
The most crucial components of the \gls{mzi} are a splitter, a coupler, and two independent phase modulators.
The splitter divides the input light into two branches.
Each branch adds a relative phase shift using an independent phase modulator, i.e., $\phi_1$ and $\phi_2$.
The coupler recombines both branches into two outputs.
For our free-space setup, two cubic beam splitters implement the splitter (BS1) and the coupler (BS2).
For additional beam alignment, our free-space setup utilizes two mirrors (M1 and M2).

Finding the transformation of the annihilation operators at each stage of the \gls{mzi} is sufficient to find the quantum input-output relations.
Idealizing the symmetric \gls{mzi}'s passive components as lossless allows relating the annihilation operators by two-dimensional unitary matrices.
We label the input annihilation operators of the \gls{mzi} $\hat{a}_1,\hat{a}_2$ and the output annihilation operators $\hat{a}_1^{\prime\prime\prime},\hat{a}_2^{\prime\prime\prime}$.
The annihilation operators after splitting and before (after) phase shifting are denoted by two (three) primes, i.e., $\hat{a}_1^{\prime},\hat{a}_2^{\prime}$ and $\hat{a}_1^{\prime\prime},\hat{a}_2^{\prime\prime}$.
Going backwards through the transformations from the output to the input annihilation operators
\begin{equation}
	\begin{split}
		\vb{\hat{a}}^{\prime\prime\prime}
		=
		U_\text{BS2}
		\vb{\hat{a}}^{\prime\prime}
		=
		U_\text{BS2}
		U_\text{PS}
		\vb{\hat{a}}^{\prime}
		=
		U_\text{BS2}
		U_\text{PS}
		U_\text{BS1}
		\vb{\hat{a}}
		=
		U_\text{MZI}
		\vb{\hat{a}}
	\end{split}
\end{equation}
we find the symmetric \gls{mzi}'s unitary matrix transform $U_\text{MZM}$ to be equal to the matrix product of the second beam splitter's, the phase shifts', and the first beam splitter's unitary matrix transform  $U_\text{BS2}U_\text{PS}U_\text{BS1}$.

An ideal cubic beam splitter with a single dielectric layer has the unitary matrix transform~\cite[p.~139]{Gerry2005}
\begin{equation}
	U_\text{BS1}
	=
	\frac{1}{\sqrt{2}}
	\begin{pmatrix}
		1 & i \\
		i & 1
	\end{pmatrix}
\end{equation}
where the off-diagonal elements of $U_\text{BS1}$, $i/\sqrt{2}$, account for the phase-shift due to the reflection at the diagonal of the cubic beam splitter.
The matrix encoding the phase shifts from the phase modulation $\phi_1,\phi_2$ and the reflection at the mirrors M1 and M2, $\pi$ is
\begin{equation}
	U_\text{PS}
	=
	\begin{pmatrix}
		ie^{i\phi_1} & 0 \\
		0 & ie^{i\phi_2}
	\end{pmatrix}
\end{equation}
For the second beam splitter, BS2, we again assume an ideal cubic beam splitter with a single dielectric layer.
The corresponding matrix transform is
\begin{equation}
	U_\text{BS2}
	=
	\frac{1}{\sqrt{2}}
	\begin{pmatrix}
		i & 1 \\
		1 & i
	\end{pmatrix}	
\end{equation}
where we exchanged the rows for consistency with the input labels.

Performing the matrix multiplication and writing the exponentials as trigonometric functions, we find the matrix transform of the symmetric \gls{mzi} to be
\begin{equation}
	U_\text{MZI}
	=
	-
	\begin{pmatrix}
		\cos(\frac{\phi_2-\phi_1}{2}) & \sin(\frac{\phi_2-\phi_1}{2}) \\
		-\sin(\frac{\phi_2-\phi_1}{2}) & \cos(\frac{\phi_2-\phi_1}{2})
	\end{pmatrix}
	e^{i\frac{\phi_1+\phi_2}{2}}
	.
\end{equation}
It appears useful to define the common-mode and differential-mode phases
\begin{align}
	\phi_+
	&=
	\phi_2
	+
	\phi_1
	&
	\phi_-
	&=
	\phi_2-\phi_1
\end{align}
for which the matrix transform simplifies to
\begin{equation}
	U_\text{MZI}
	=
	-
	\begin{pmatrix}
		\cos(\phi_-/2) & \sin(\phi_-/2) \\
		-\sin(\phi_-/2) & \cos(\phi_-/2)
	\end{pmatrix}
	e^{i\phi_+/2}
\end{equation}
and we note that the common-mode phase $\phi_+$ adds a global phase shift of $\phi_+/2$ while the differential-mode phase $\phi_-$ changes the splitting ratios at the output.

For a general lossless \gls{mzm}, we propose the generic unitary matrix transform~\cite[p.~95]{Leonhardt2010}
\begin{equation}
	\begin{split}
		U_\text{MZM}
		&=
		e^{i\Lambda/2}
		\begin{pmatrix}
			\cos(\Theta/2)e^{i(+\Phi+\Psi)/2} & \sin(\Theta/2)e^{i(+\Phi-\Psi)/2} \\
			-\sin(\Theta/2)e^{+(-\Phi+\Psi)/2} & \cos(\Theta/2)e^{i(-\Phi-\Psi)/2}
		\end{pmatrix}
		\\
		&=
		e^{i\Lambda/2}
		\begin{pmatrix}
			e^{+i\Phi/2} & 0 \\
			0 & e^{-i\Phi/2}
		\end{pmatrix}
		\begin{pmatrix}
			\cos(\Theta/2) & \sin(\Theta/2) \\
			-\sin(\Theta/2) & \cos(\Theta/2)
		\end{pmatrix}
		\begin{pmatrix}
			e^{+i\Psi/2} & 0 \\
			0 & e^{-i\Psi/2}
		\end{pmatrix}
	\end{split}
	\label{eq:mzm_matrix}
\end{equation}
wherein the global phase $\Theta$ and the rotation angle $\Lambda$ are time-dependent but $\Phi,\Psi$ are constants.
We obtain the matrix transform of the symmetric free-space $U_\text{MZI}$ after identification of $\Theta$ with the differential-mode phase, $\Lambda+\pi$ with the common-mode phase, and choosing $\Psi=0=\Phi$.
One advantage of the proposed decomposition in \cref{eq:mzm_matrix} is the one-to-one correspondence between the unitary operator~\cite[p.~99]{Leonhardt2010}
\begin{equation}
	\hat{U}_\text{MZM}
	=
	e^{i\Lambda\hat{L}_t}
	e^{i\Phi\hat{L}_z}
	e^{i\Theta\hat{L}_y}
	e^{i\Psi\hat{L}_z}
	\label{eq:mzm_operator}
\end{equation}
wherein $\hat{L}_j$ denote the Jordan-Schwinger operators~\cite[p.~97]{Leonhardt2010}
\begin{equation}
	\hat{L}_j
	=
	\frac{1}{2}
	\begin{pmatrix}
		\hat{a}_1^\dagger, \hat{a}_2^\dagger
	\end{pmatrix}
	\sigma_j
	\begin{pmatrix}
		\hat{a}_1 \\
		\hat{a}_2
	\end{pmatrix}
	=
	\frac{1}{2}
	\hat{\vb{a}}^\dagger
	\sigma_j
	\hat{\vb{a}}
\end{equation}
with $\sigma_j$ being the two-dimensional Pauli matrices and $\sigma_0$ being the identity matrix, and the unitary matrix transform
\begin{equation}
	U_\text{MZM}
	\hat{\vb{a}}
	=
	\hat{U}_\text{MZM}^\dagger
	\hat{\vb{a}}
	\hat{U}_\text{MZM}
	.
\end{equation}

Given the unitary operator for the \gls{mzm}, \cref{eq:mzm_operator}, we find the number state transform
\begin{equation}
	\begin{split}
		\hat{U}_\text{MZM}
		\ket{n,0}
		&=
		\frac{1}{\sqrt{n!}}
		\hat{U}_\text{MZM}
		\left(\hat{a}_1^\dagger\right)^n
		\hat{U}_\text{MZM}^\dagger
		\hat{U}_\text{MZM}
		\ket{0,0}
		\\
		&=
		\frac{1}{\sqrt{n!}}
		\left(
			\hat{U}_\text{MZM}
			\hat{a}_1^\dagger
			\hat{U}_\text{MZM}^\dagger
		\right)^n
		\ket{0,0}
	\end{split}
	\label{eq:mzm_number}
\end{equation}
where we used the invariance of the vacuum state $\ket{0,0}$ in the second line.
We note that
\begin{equation}
	\begin{split}
		\hat{U}_\text{MZM}(\Lambda,\Phi,\Psi,\Theta)^\dagger
		&=
		e^{-i\Psi\hat{L}_z}
		e^{-i\Theta\hat{L}_y}
		e^{-i\Phi\hat{L}_z}
		e^{-i\Lambda\hat{L}_t}
		\\
		&=
		e^{-i\Lambda\hat{L}_t}
		e^{-i\Psi\hat{L}_z}
		e^{-i\Theta\hat{L}_y}
		e^{-i\Phi\hat{L}_z}
		\\
		&=
		\hat{U}_\text{MZM}(-\Lambda,\Psi,\Phi,-\Theta)
	\end{split}
\end{equation}
because $\hat{L}_t$ commutes with the other Jordan-Schwinger operators, and rewrite the unitary transform of the creation operator
\begin{equation}
	\begin{split}
		\hat{U}_\text{MZM}(\Lambda,\Phi,\Psi,\Theta)
		\hat{a}_1^\dagger
		\hat{U}_\text{MZM}(\Lambda,\Phi,\Psi,\Theta)^\dagger
		&=
		\left[
			\hat{U}_\text{MZM}(\Lambda,\Phi,\Psi,\Theta)
			\hat{a}_1
			\hat{U}_\text{MZM}(\Lambda,\Phi,\Psi,\Theta)^\dagger
		\right]^\dagger
		\\
		&=
		\left[
			\hat{U}_\text{MZM}(-\Lambda,\Psi,\Phi,-\Theta)^\dagger
			\hat{a}_1
			\hat{U}_\text{MZM}(-\Lambda,\Psi,\Phi,-\Theta)
		\right]^\dagger
	\end{split}
\end{equation}
Using the correspondence between the unitary and matrix \gls{mzm} transform, we identify the transformed creation operator
\begin{equation}
	\left(\hat{a}_1^\prime\right)^\dagger
	=
	\left[
		\hat{a}_1^\dagger
		\cos(\Theta/2)
		e^{-i(\Phi+\Psi)/2}
		+
		\hat{a}_2^\dagger
		\sin(\Theta/2)
		e^{-i(\Phi-\Psi)/2}
	\right]
	e^{-i\Lambda/2}
	.
\end{equation}
We insert the transformed creation operator into \cref{eq:mzm_number} and find
\begin{equation}
	\begin{split}
		\hat{U}_\text{MZM}
		\ket{n,0}
		&=
		\frac{1}{\sqrt{n!}}
		\left[
			\hat{a}_1^\dagger
			\cos(\Theta/2)
			e^{-i(\Phi+\Psi)/2}
			+
			\hat{a}_2^\dagger
			\sin(\Theta/2)
			e^{-i(\Phi-\Psi)/2}
		\right]^n
		e^{-in\Lambda/2}
		\ket{0,0}
		\\
		&=
	\end{split}
\end{equation}

As the number basis is complete, we can use the previous result to find the transformation property of a coherent and vacuum input state
\begin{equation}
	\begin{split}
		\hat{R}(\Theta)
		\ket{\alpha,0}
		&=
		e^{-\overline{n}/2}
		\sum_{n=0}^\infty
		\frac{\alpha^n}{\sqrt{n!}}
		\hat{R}(\Theta)
		\ket{n,0}
		\\
		&=
		e^{-\overline{n}/2}
		\sum_{n=0}^\infty
		\frac{1}{n!}
		\left(
			\alpha
			\cos(\Theta/2)
			\hat{a}_1^\dagger
			-
			\alpha
			\sin(\Theta/2)
			\hat{a}_2^\dagger
		\right)^n
		\ket{0,0}
		\\
		&=
		e^{-\overline{n}/2}
		\exp\left\{
			\alpha
			\cos(\Theta/2)
			\hat{a}_1^\dagger
			-
			\alpha
			\sin(\Theta/2)
			\hat{a}_2^\dagger
		\right\}
		\ket{0,0}
		\\
		&=
		e^{-\overline{n}\cos(\Theta/2)^2/2}
		\exp\left\{
			\alpha
			\cos(\Theta/2)
			\hat{a}_1^\dagger
		\right\}
		e^{-\overline{n}\sin(\Theta/2)^2/2}
		\exp\left\{
			-
			\alpha
			\sin(\Theta/2)
			\hat{a}_2^\dagger
		\right\}
		\ket{0,0}
		\\
		&=
		\hat{D}_1\left(
			\alpha
			\cos(\Theta/2)
		\right)
		\hat{D}_2\left(
			-
			\alpha
			\sin(\Theta/2)
		\right)
		\ket{0,0}
		\\
		&=
		\ket{\alpha\cos(\Theta/2),-\alpha\sin(\Theta/2)}
		.
	\end{split}
\end{equation}
A rotation splits the coherent input state among two coherent output states.
Tracing out one output of the \gls{mzm} for a coherent input state is equivalent to a projection
\begin{equation}
	\tr_2\left\{
		\hat{R}(\Theta/2)
		\ketbra{\alpha,0}
		\hat{R}^\dagger(\Theta/2)
	\right\}
	=
	\ketbra{\alpha\cos(\Theta/2)}
\end{equation}
but for an entangled state tracing out one output yields a mixed state!
We conclude that only coherent input states yield simple output states while number input states yield complex entangled output states.

\subsection{In-phase and quadrature modulator}

The \gls{iq}-modulator uses two \gls{mzm} to modulate each of the two quadratures of a coherent state.

\Cref{fig:iqm} shows a fiber-optical embodiment of the \gls{iq}-modulator.
First, a splitter divides the input light into two branches with an independent \gls{mzm} each.
Second, a coupler recombines both branches and outputs the \gls{iq}-modulated signal.
\begin{figure}[htb]
	\centering
	\includestandalone{figures/pstricks/iqm}
	\caption{\gls{iq}-modulator comprising three \gls{mzm} and connected by an input and output fiber. MZM 1 and MZM 2, are used in push-pull configuration for \gls{am}. MZM 3 is used in push-push configuration for setting the relative phase between the upper and lower branch. Vacuum in- and outputs are indicated by the red dashed fiber.}\label{fig:iqm}
\end{figure}
The vacuum input states and dumped output states are denoted by the dashed orange lines.

Instead of using the number basis, we could have used that the coherent state is an eigenstate of the annihilation operator
\begin{equation}
	\ket{\boldsymbol{\alpha}}
	=
	\bigotimes_{n=1}^N
	\ket{\alpha_n}
\end{equation}
\begin{equation}
	\hat{U}
	\ket{\boldsymbol{\alpha}}
	=
	\hat{U}
	\hat{D}(\boldsymbol{\alpha})
	\hat{U}^\dagger
	\hat{U}
	\ket{\vb{0}}
	=
	\hat{U}
	\hat{D}(\boldsymbol{\alpha})
	\hat{U}^\dagger
	\ket{\vb{0}}
\end{equation}
\begin{equation}
	\hat{D}(\boldsymbol{\alpha})
	=
	\exp\left\{
		\hat{\boldsymbol{\alpha}}^\trans
		\hat{\vb{a}}^\dagger
		-
		\text{h.c.}
	\right\}
	=
	\exp\left\{
		\sum_{n=1}^N
		\alpha_n
		\hat{a}_n^\dagger
		-
		\text{h.c.}
	\right\}
\end{equation}
\begin{equation}
	\begin{split}
		\hat{U}
		\hat{D}(\boldsymbol{\alpha})
		\hat{U}^\dagger
		&=
		\exp\left\{
			\hat{\boldsymbol{\alpha}}^\trans
			\hat{U}
			\hat{\vb{a}}^\dagger
			\hat{U}^\dagger
			-
			\text{h.c.}
		\right\}
		\\
		&=
		\exp\left\{
			\hat{\boldsymbol{\alpha}}^\trans
			U
			\hat{\vb{a}}^\dagger
			-
			\text{h.c.}
		\right\}		
	\end{split}
\end{equation}

We relate the different annihilation operators using tensor products of two-dimensional unitary matrices describing the splitter and coupler $U_\text{BS1},U_\text{BS2}$ and the independent \gls{mzm}s $U_\text{MZM1},U_\text{MZM2}$, i.e.,
\begin{equation}
	\begin{split}
		\begin{pmatrix}
			\hat{a}_1^{\prime\prime\prime} \\
			\hat{a}_2^{\prime\prime\prime} \\
			\hat{a}_3^{\prime\prime\prime} \\
			\hat{a}_4^{\prime\prime\prime}
		\end{pmatrix}
		&=
		\begin{pmatrix}
			 1 & 0 & 0 & 0 \\
			 0 & \multicolumn{2}{c}{\multirow{2}{*}{$U_\text{BS2}$}} & 0 \\
			 0 & & & 0 \\
			 0 & 0 & 0 & 1
		\end{pmatrix}
		\begin{pmatrix}
			\hat{a}_1^{\prime\prime} \\
			\hat{a}_2^{\prime\prime} \\
			\hat{a}_3^{\prime\prime} \\
			\hat{a}_4^{\prime\prime}
		\end{pmatrix}
		\\
		&=
		\begin{pmatrix}
			 1 & 0 & 0 & 0 \\
			 0 & \multicolumn{2}{c}{\multirow{2}{*}{$U_\text{BS2}$}} & 0 \\
			 0 & & & 0 \\
			 0 & 0 & 0 & 1
		\end{pmatrix}
		\begin{pmatrix}
			 \multicolumn{2}{c}{\multirow{2}{*}{$U_\text{MZM1}$}} & 0 & 0 \\
			 & & 0 & 0 \\
			 0 & 0 & \multicolumn{2}{c}{\multirow{2}{*}{$U_\text{MZM2}$}} \\
			 0 & 0 & & \\
		\end{pmatrix}
		\begin{pmatrix}
			\hat{a}_1^\prime \\
			\hat{a}_2^\prime \\
			\hat{a}_3^\prime \\
			\hat{a}_4^\prime
		\end{pmatrix}
		\\
		&=
		\begin{pmatrix}
			 1 & 0 & 0 & 0 \\
			 0 & \multicolumn{2}{c}{\multirow{2}{*}{$U_\text{BS2}$}} & 0 \\
			 0 & & & 0 \\
			 0 & 0 & 0 & 1
		\end{pmatrix}
		\begin{pmatrix}
			 \multicolumn{2}{c}{\multirow{2}{*}{$U_\text{MZM1}$}} & 0 & 0 \\
			 & & 0 & 0 \\
			 0 & 0 & \multicolumn{2}{c}{\multirow{2}{*}{$U_\text{MZM2}$}} \\
			 0 & 0 & & \\
		\end{pmatrix}
		\begin{pmatrix}
			 1 & 0 & 0 & 0 \\
			 0 & \multicolumn{2}{c}{\multirow{2}{*}{$U_\text{BS1}$}} & 0 \\
			 0 & & & 0 \\
			 0 & 0 & 0 & 1
		\end{pmatrix}
		\begin{pmatrix}
			\hat{a}_1 \\
			\hat{a}_2 \\
			\hat{a}_3 \\
			\hat{a}_4
		\end{pmatrix}
		.
	\end{split}
\end{equation}
In tensor product notation, we can compactly write
\begin{equation}
	\vb{\hat{a}}^{\prime\prime\prime}
	=
	U_\text{IQM}
	\vb{\hat{a}}
	=
	\left[
		\mathbb{I}_1
		\otimes
		U_\text{BS2}
		\otimes
		\mathbb{I}_1
	\right]
	\left[
		U_\text{MZM1}
		\otimes
		U_\text{MZM2}
	\right]
	\left[
		\mathbb{I}_1
		\otimes
		U_\text{BS1}
		\otimes
		\mathbb{I}_1
	\right]
	\vb{\hat{a}}
	.
\end{equation}

We do not assume a specific kind of balanced splitter or coupler and use
\begin{equation}
	U_\text{BS1}
	=
	U_\text{BS2}
	=
	\frac{1}{\sqrt{2}}
	\begin{pmatrix}
		1 & ie^{+i\varphi} \\
		ie^{-i\varphi} & 1
	\end{pmatrix}
	=
	U_\text{BS}(\varphi)
\end{equation}
as the matrix transform of the splitter and coupler.
While the second \gls{mzm} has the transform derived in the previous section, the first \gls{mzm} has exchanged in- and outputs which we can write
\begin{equation}
	\begin{split}
		U_\text{MZM1}
		&=
		e^{i\pi}
		U_\text{BS}(\pi,0)
		U_\text{MZM}(\phi_{1+},\phi_{1-})
		U_\text{BS}(\pi,0)
		\\
		&=
		\begin{pmatrix}
			0 & i \\
			i & 0
		\end{pmatrix}
		e^{i\phi_{1+}/2}
		\begin{pmatrix}
			\cos(\phi_{1-}/2) & \sin(\phi_{1-}/2) \\
			-\sin(\phi_{1-}/2) & \cos(\phi_{1-}/2)
		\end{pmatrix}
		\begin{pmatrix}
			0 & i \\
			i & 0
		\end{pmatrix}
		\\
		&=
		-
		e^{i\phi_{1+}/2}
		\begin{pmatrix}
			\cos(\phi_{1-}/2) & -\sin(\phi_{1-}/2) \\
			\sin(\phi_{1-}/2) & \cos(\phi_{1-}/2)
		\end{pmatrix}
	\end{split}
\end{equation}
We choose $\phi_{1+}=2\pi$ to compensate for the phase shift and $\Theta_1=\phi_{1-}$, i.e.,
\begin{equation}
	U_\text{MZM1}
	=
	\begin{pmatrix}
		\cos(\Theta_1/2) & -\sin(\Theta_1/2) \\
		\sin(\Theta_1/2) & \cos(\Theta_1/2)
	\end{pmatrix}
	=
	R(-\Theta_1)
\end{equation}
The matrix transform of the second \gls{mzm} is
\begin{equation}
	U_\text{MZM2}
	=
	e^{i\phi_{2+}/2}
	\begin{pmatrix}
		\cos(\phi_{2-}/2) & \sin(\phi_{2-}/2) \\
		-\sin(\phi_{2-}/2) & \cos(\phi_{2-}/2)
	\end{pmatrix}
\end{equation}
but we choose $\phi_{2+}=-\pi$ and $\Theta_2=\phi_{2-}$ such that
\begin{equation}
	U_\text{MZM2}(\Theta_2)
	=
	-i
	\begin{pmatrix}
		\cos(\Theta_1/2) & \sin(\Theta_1/2) \\
		-\sin(\Theta_1/2) & \cos(\Theta_1/2)
	\end{pmatrix}
	=
	-i
	R(\Theta_2)
	.
\end{equation}
Using $\phi_{2+}=-\pi$ turns out to yield orthogonal modulation of both \gls{mzm} as desired for \gls{iq}-modulation.

Performing the matrix multiplication we find
\begin{equation}
	\begin{split}
		U_\text{IQM}
		=
		\begin{pmatrix}
			c_1 & -\frac{1}{\sqrt{2}}s_1 & -\frac{1}{\sqrt{2}}e^{+i\varphi}s_1 & 0 \\
			\frac{1}{\sqrt{2}}s_1 & \frac{1}{2}(c_1+ic_2) & \frac{1}{2}e^{i\varphi}(ic_1+c_2) & \frac{1}{\sqrt{2}}e^{i\varphi}s_2 \\
			\frac{1}{\sqrt{2}}ie^{-i\varphi}s_1 & \frac{1}{2}e^{-i\varphi}(ic_1+c_2) & -\frac{1}{2}(c_1+ic_2) & -\frac{1}{\sqrt{2}}is_2 \\
			0 & -\frac{1}{\sqrt{2}}e^{-i\varphi}s_2 & \frac{1}{\sqrt{2}}is_2 & -ic_2
		\end{pmatrix}
	\end{split}
\end{equation}
where we introduced the shorthand notation
\begin{align}
	c_j
	&=
	\cos(\Theta_j/2)
	&
	s_j
	&=
	\sin(\Theta_j/2)
	.
\end{align}

For a coherent input state the matrix transform 
\begin{equation}
	\begin{pmatrix}
		\alpha_1^{\prime\prime\prime} \\
		\alpha_2^{\prime\prime\prime} \\
		\alpha_3^{\prime\prime\prime} \\
		\alpha_4^{\prime\prime\prime}
	\end{pmatrix}
	=
	U_\text{IQM}(\Theta_1,\Theta_2,\varphi)
	\begin{pmatrix}
		0 \\
		\alpha \\
		0 \\ 
		0
	\end{pmatrix}
	=
	\frac{\alpha}{\sqrt{2}}
	\begin{pmatrix}
		-\sin(\Theta_1/2) \\
		\frac{1}{\sqrt{2}}\left(\cos(\Theta_1/2)+i\cos(\Theta_2/2)\right) \\
		\frac{1}{\sqrt{2}}\left(i\cos(\Theta_1/2)+\cos(\Theta_2/2)\right)e^{-i\varphi} \\
		-\sin(\Theta_2/2)e^{-i\varphi}
	\end{pmatrix}
\end{equation}
tracing out the other modes is equivalent to projecting the second mode if we only deal with coherent states
\begin{equation}
	\hat{P}_2
	U_\text{IQM}(\Theta_1,\Theta_2,\varphi)
	\ket{0,\alpha,0,0}
	=
	\ket{\frac{1}{2}\alpha\left(\cos(\Theta_1/2)+i\cos(\Theta_2/2)\right)}
\end{equation}
which is independent of $\varphi$ does we don't care about the specific phase properties of BS1 and BS2 as long as they are symmetric.

For $\alpha=e^{-i\omega_ct}$ and $\Theta_j\to\Theta_j-\pi$, we find
\begin{equation}
	\hat{P}_2
	U_\text{IQM}(\Theta_1,\Theta_2)
	\ket{0,e^{-i\omega_ct},0,0}
	=
	\ket{\frac{1}{2}\left(\sin(\Theta_1/2)+i\sin(\Theta_2/2)\right)e^{-i\omega_ct}}
\end{equation}
for small modulation $\Theta_1,\Theta_2\ll1$ the small-angle approximation yields
\begin{equation}
	\hat{P}_2
	U_\text{IQM}(\Theta_1,\Theta_2)
	\ket{0,e^{-i\omega_ct},0,0}
	\approx
	\ket{\alpha e^{-i\omega_ct}}
\end{equation}
where we identified
\begin{equation}
	\alpha
	=
	\frac{1}{4}
	\left(\Theta_1+i\Theta_2\right)
\end{equation}
with the complex baseband signal.

% mention number states
% mention generator transform
\section{Transmitter}

\subsection{Mach-Zehnder modulator}

\begin{figure}[htb]
	\centering
	\includestandalone{figures/pstricks/mzi-symmetric}
	\caption{Free-space setup of a symmetric \gls{mzi}: The input light field enters a first beam splitter BS1 from the left. The light field exits BS1 to the right and the bottom. Right of BS1, a first phase shifter adds a relative phase of $\varphi_1$. Right of the first phase shifter, a first mirror M1 reflects the light to the bottom, hitting a second beam splitter BS2 from the top. Below BS1, a second mirror M2 directs the light to the right, where a second phase shifter adds a relative phase of $\varphi_2$, and the light hits BS2 from the left.}
\end{figure}

\begin{align}
	U_\text{PS}(\varphi,\phi)
	&=
	\begin{pmatrix}
		e^{i\varphi} & 0 \\
		0 & e^{i\phi}
	\end{pmatrix}
	&
	U_\text{BS}(\theta)
	&=
	\begin{pmatrix}
		\cos(\theta/2) & i\sin(\theta/2) \\
		i\sin(\theta/2) & \cos(\theta/2)
	\end{pmatrix}
\end{align}

\begin{equation}
	\begin{split}
		U_\text{MZM}(\varphi_1,\varphi_2)
		&=
		\frac{1}{\sqrt{2}}
		\begin{pmatrix}
			1 & i \\
			i & 1
		\end{pmatrix}
		\begin{pmatrix}
			e^{i\pi} & 0 \\
			0 & e^{i\varphi_2}
		\end{pmatrix}
		\begin{pmatrix}
			e^{i\varphi_1} & 0 \\
			0 & e^{i\pi}
		\end{pmatrix}
		\frac{1}{\sqrt{2}}
		\begin{pmatrix}
			1 & i \\
			i & 1
		\end{pmatrix}
		\\
		&=
		\frac{1}{2}
		\begin{pmatrix}
			e^{i\varphi_1}-e^{i\varphi_2} & ie^{i\varphi_1}+ie^{i\varphi_2} \\
			ie^{i\varphi_1}+ie^{i\varphi_2} & -e^{i\varphi_1}+e^{i\varphi_2}
		\end{pmatrix}
		\\
		&=
		i
		\begin{pmatrix}
			\sin(\frac{\varphi_1-\varphi_2}{2}) & \cos(\frac{\varphi_1-\varphi_2}{2}) \\
			\cos(\frac{\varphi_1-\varphi_2}{2}) & -\sin(\frac{\varphi_1-\varphi_2}{2})
		\end{pmatrix}
		e^{i(\varphi_1+\varphi_2)/2}
	\end{split}
\end{equation}

\begin{figure}[htb]
	\centering
	\includestandalone{figures/pstricks/mzm-symmetric}
	\caption{Fiber-optical setup of a symmetric \gls{mzm}: A first coupler is connected with an input fiber and a vacuum input (red dashed line). The first coupler splits into an upper and lower branch. The upper branch passes phase modulator with phase shift $\varphi_1$. The lower branch passes a phase modulator with phase shift $\varphi_2$. A second coupler combines the output of the phase modulators. One output of the second coupler is connected to an output fiber while the other output is dumped (red dashed line).}
\end{figure}

\begin{table}[htb]
	\centering
	\begin{tabular}{lcc}
		\toprule
		Configuration & Phase relation & Modulation \\
		\midrule
		Push-pull & $\varphi_1=-\varphi_2$ & Amplitude \\
		Push-push & $\varphi_1=+\varphi_2$ & Phase \\
		\bottomrule
	\end{tabular}
	\caption{Configurations of a symmetric \gls{mzm}.}
\end{table}

\begin{equation}
	U_\text{mzm}(\varphi_1=\varphi,\varphi_2=-\varphi)
	=
	i
	\begin{pmatrix}
		\sin(\varphi) & \cos(\varphi) \\
		\cos(\varphi) & -\sin(\varphi)
	\end{pmatrix}
\end{equation}
\begin{equation}
	U_\text{mzm}(\varphi_1=\varphi,\varphi_2=\varphi)
	=
	i
	e^{i\varphi}
	\begin{pmatrix}
		0 & 1 \\
		1 & 0
	\end{pmatrix}
\end{equation}

\subsection{I/Q modulator}

\begin{figure}[htb]
	\centering
	\includestandalone{figures/pstricks/iqm}
	\caption{\gls{iq}-modulator comprising three \gls{mzm} and connected by an input and output fiber. MZM 1 and MZM 2, are used in push-pull configuration for \gls{am}. MZM 3 is used in push-push configuration for setting the relative phase between the upper and lower branch. Vacuum in- and outputs are indicated by the red dashed fiber.}
\end{figure}
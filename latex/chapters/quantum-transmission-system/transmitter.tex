\section{Transmitter}

The transmitter consists of a coherent light source, an \gls{iqm}, and a \gls{voa}.

Practically, a laser implements the coherent light source.
Advanced quantum stochastic models predicting laser characteristics (gain, stability, coherence, linewidth) are established and found in  Ref.~\cite[p.~900]{Mandel1995} and Ref.~\cite{Haken2012}.
For our purpose, it is sufficient to know that lasers emit coherent states with a frequency spectrum.\footnote{For an intuitive argument why laser emit coherent states based on decoherence, see Ref.~\cite{Gea1998}.}

The \gls{voa} attenuates the \gls{iq}-modulated signal such that the signal power becomes comparable with the quantum noise.
It is straightforward to model the \gls{voa}'s with a beam splitter transform, such that we will not draw further attention to it.

The \gls{iq} modulation is the most important step in the transmitter as it indicates the borderline between the classical and quantum signals.
It is also the most mysterious as we are unaware of any published quantum mechanical investigation.
The modulation occurs at three stages: the Pockels (phase) modulator, the \gls{mzm} (amplitude) modulator, and the \gls{iqm} modulator.
The transition from the classical signal to the quantum state occurs at the phase modulation, while the MZM and \gls{iqm} modulator operate on the quantum state.

\subsection{Pockels modulator}

% discussion of Pockels effect
From a purely phenomenological point of view, the Pockels effect - also known as the linear electro-optical effect - describes a linear change of the refractive index of a dielectric material in the presence of an external electric field $E$, i.e.,
\begin{equation}
	n(E)
	=
	n^{(0)}
	+
	n^{(1)}E
\end{equation}
wherein $n^{(0)}=n(0)$ is the refractive index without an external field and $n^{(1)}$ is a proportionality constant.
One way to create the external electric field $E$ is to place the Pockels dielectric inside a plate capacitor and apply a voltage $V$ to the plates.
Neglecting boundary effects, we find a homogeneous static electric field between the plates of $E=V/d$ where $d$ is the plate distance.
Such a configuration is known as Pockels cell and depicted in \Cref{fig:pockels_cell} where the field inside the Pockels cell is labeled $E_x(t)$.
\begin{figure}[htb]
    \centering
    \includegraphics{figures/tikz/pockels-cell}
    \caption{Pockels cell of length $l$ and thickness $d$ embedded in a waveguide with constant dielectric permittivity $\varepsilon_1$: The dielectric permittivity of the Pockels cell $\varepsilon_2(t)$ depends on the electric field $E_x(t)$ across the Pockels cell plates. The optical field $E_z(t)$ enters the Pockels cell from the left and leaves it to the right as $E^\prime_z(t)$.}\label{fig:pockels_cell}
\end{figure}
Let us first consider a monochromatic electromagnetic wave of frequency $\omega_0$ propagating through a dielectric of length $l$ with refractive index $n_1^{(0)}=\sqrt{\varepsilon_1}$.
Inside the dielectric, the wave propagates with phase velocity $c/n_1^{(0)}$ and takes $T_1=ln_1^{(0)}/c$ to pass through the dielectric.
Directly after the transit, the wave accumulated a total phase shift of
\begin{equation*}
	\phi_1
	=
	\omega_0T_1
	=
	2\pi\frac{n_1^{(0)}l}{\lambda_0}
\end{equation*}
where $\lambda_0=c/f$ is the vacuum wave length.
If we now consider the same monochromatic electromagnetic wave propagating through a Pockels cell of the same length $l$ but without applied voltage, the wave would accumulate a phase shift of
\begin{equation}
	\phi_2
	=
	\omega_0T_2
	=
	2\pi\frac{n_2^{(0)}l}{\lambda_0}
	.
\end{equation}
So even when there is no electric field inside the Pockels cell, we find a phase difference of
\begin{equation}
	\Delta\phi
	=
	\phi_2
	-
	\phi_1
	=
	2\pi\frac{n_2^{(0)}-n_1^{(0)}}{\lambda_0}l
\end{equation}
because of the different dielectric constants of the materials.
If we apply a voltage to the Pockels cell, the refractive index of the Pockels dielectric contributes a second phase shift of
\begin{equation}
	\varphi_2
	=
	2\pi\frac{n_2^{(1)}E}{\lambda_0}l
\end{equation}
and the total phase difference is $\varphi_2+\Delta\phi$.
Extending our discussion to wave packets in dispersive media, we need to replace the phase velocity $v_p$ with the group velocity $v_g$.
The group velocity can be expressed in terms of the refractive index via~\cite[p.~211]{Jackson2007}
\begin{equation}
	v_g(\omega_0)
	=
	\left[
		n
		+
		\omega
		\pdv{n}{\omega}
	\right]^{-1}_{\omega=\omega_0}
\end{equation}
where $\omega_0$ is the center frequency of the wave packet.
The concept of group velocity can be extended to quantum states, see, for instance, Ref.~\cite[p.~3]{Naumov2013}.
We refrain from a further discussion which would require specific knowledge of the wave packet's pulse shape and frequency-dependency of the refractive index and is best simulated using finite element methods.

Thus far, we have treated a static electric field in the Pockels cell.
We are now going to extend our model to allow for time-dependent electric field inside the Pockels cell and justify why we can treat the time-dependency of the modulation field to be effectively static.
An external electric field induces a dipole moment in the constituents of a dielectric.
The average dipole moment per volume is referred to the macroscopic polarization $\vb{P}$.
Assuming the dielectric to be a time-invariant system, i.e., to be memoryless, we can expand each component of the macroscopic polarization $P^i$ up to second-order in terms of electric susceptibility tensors $\chi$~\cite[p.~17]{Murti2014}
\begin{equation}
	P_i(t)
	=
	\int_{\mathbb{R}}\dd{t^\prime}
	\chi^{(1)}_{ij}(t-t^\prime)
	E^j(t^\prime)
	+
	\iint_{\mathbb{R}^2}\dd{t^\prime}\dd{t^{\prime\prime}}
	\chi^{(2)}_{ijk}(t-t^\prime,t-t^{\prime\prime})
	E^j(t^\prime)E^k(t^{\prime\prime})
\end{equation}
where causality demands $\chi^{(n)}(t-t^\prime)=0$ for $t^\prime>t$.
Inserting the Fourier representation, we find
\begin{align}
	P_i^{(1)}(t)
	&=
	\int_{\mathbb{R}}
	\frac{\dd{\omega_1}}{2\pi}
	\chi^{(1)}_{ij}(\omega_1)
	E^j(\omega_1)
	e^{+i\omega_1t}
	\\
	P_i^{(2)}(t)
	&=
	\iint_{\mathbb{R}^2}
	\frac{\dd{\omega_1}}{2\pi}
	\frac{\dd{\omega_2}}{2\pi}
	\chi^{(2)}_{ijk}(\omega_1,\omega_2)
	E^j(\omega_1)E^k(\omega_2)
	e^{+i(\omega_1+\omega_2)t}	
\end{align}
for the first two terms of the expansion.
The inverse Fourier transform yields the macroscopic polarization in the frequency domain~\cite[p.~1070]{Mandel1995} yields for the first term
\begin{equation}
	P_i^{(1)}(\omega)
	=
	\int\dd{\omega^\prime}
	\chi^{(1)}_{ij}(\omega^\prime)
	E^j(\omega^\prime)
	\delta^{(1)}(\omega-\omega^\prime)
	=
	\chi^{(1)}_{ij}(\omega)
	E^j(\omega)
\end{equation}
which we will later identify as part of the constant refractive index with $E^j$ being the optical field passing through the Pockels cell
For the second term we find
\begin{equation}
	\begin{split}
		P_i^{(2)}(\omega)
		&=
		\iint\frac{\dd{\omega^\prime}\dd{\omega^{\prime\prime}}}{2\pi}
		\chi^{(2)}_{ijk}(\omega^\prime,\omega^{\prime\prime})
		E^j(\omega^\prime)
		E^k(\omega^{\prime\prime})
		(2\pi)	
		\delta^{(1)}(\omega-\omega^\prime-\omega^{\prime\prime})
		\\
		&=
		\int\frac{\dd{\omega^\prime}}{2\pi}
		\chi^{(2)}_{ijk}(\omega-\omega^\prime,\omega^\prime)
		E^j(\omega-\omega^\prime)
		E^k(\omega^\prime)
	\end{split}
\end{equation}
where we will later identify $E^k$ to be the radio frequency field driving the Pockels cell.
The specific form indicates the presence of frequency sidebands by modulation.
Let's assume radio frequency field $E^k$ to have non-zero support, i.e., bandwidth, $\Delta\Omega$, then according to the mean value theorem
\begin{equation}
	\begin{split}
		P_i^{(2)}(\omega)
		&=
		\int_{\Delta\Omega}\frac{\dd{\omega^\prime}}{2\pi}
		\chi^{(2)}_{ijk}(\omega-\omega^\prime,\omega^\prime)
		E^j(\omega-\omega^\prime)
		E^k(\omega^\prime)
		\\
		&=
		\frac{\Delta\Omega}{2\pi}
		\chi^{(2)}_{ijk}(\omega-\Omega_0,\Omega_0)
		E^j(\omega-\Omega_0)
		E^k(\Omega_0)
	\end{split}
\end{equation}
where $\Omega_0$ is the mean frequency of $E^k$.
Restricting $P_i(\omega)$ to the optical domain, we have $\omega\gg\Omega_0$ which let's us Taylor expand $E^j(\omega-\Omega_0)$ around $\omega$, i.e.,
\begin{equation}
	P_i^{(2)}(\omega)
	\approx
	\frac{\Delta\Omega}{2\pi}
	\chi^{(2)}_{ijk}(\omega,\Omega_0)
	\left(
		E^j(\omega)
		+
		\pdv{E^j}{\omega}(\omega)\Omega_0
	\right)
	E^k(\Omega_0)
\end{equation}
where we assumed a flat response $\pdv{\chi_{ijk}^{(2)}}{\Omega_0}\approx0$.
Finally, we redefine the electric susceptibility tensor such that
\begin{equation}
	P_i(\omega)
	\approx
	\left(
		\tilde{\chi}^{(1)}_{ij}(\omega)
		+
		\tilde{\chi}^{(2)}_{ijk}(\omega)
		E^k(\Omega_0)
	\right)
	E^j(\omega)
\end{equation}
which is valid if $\omega$ is an optical frequency.
The dielectric permittivity tensor $\varepsilon_{ij}$ is defined implicitly through the displacement field
\begin{equation}
	D_i(\omega)
	=
	E_i(\omega)
	+
	P_i(\omega)
	=
	\varepsilon_{ij}(\omega)
	E^j(\omega)
\end{equation}
and we identify by comparison the dielectric permittivity tensor to relate to the electric susceptibility by
\begin{equation}
	\varepsilon_{ij}(\omega)
	=
	1
	+
	\tilde{\chi}^{(1)}_{ij}(\omega)
	+
	\tilde{\chi}^{(2)}_{ijk}(\omega)
	E^k(\Omega_0)
	.
\end{equation}
The refractive index tensor for a dielectric, non-magnetic and non-chiral medium is given after a series expansion by~\cite{Rerat2020}
\begin{equation}
	n_{ij}(\omega)
	=
	\sqrt{\varepsilon_{ij}(\omega)}
	\approx
	n^{(0)}_{ij}(\omega)
	+
	n^{(1)}_{ijk}(\omega)
	E^k(\Omega_0)
\end{equation}
where we defined the refractive index tensors
\begin{align}
	n^{(0)}_{ij}(\omega)
	=
	\sqrt{1+\tilde{\chi}^{(1)}_{ij}(\omega)}
	&&
	n^{(1)}_{ijk}(\omega)
	=
	\frac{\tilde{\chi}^{(2)}_{ijk}(\omega)
	E^k(\Omega_0)}{2\sqrt{1+\tilde{\chi}^{(1)}_{ij}(\omega)}}
\end{align}
which recovers the linear electro-optical effect we first discussed in the phenomenological approach.
For the Pockels cell depicted in \Cref{fig:pockels_cell}, the refractive index relevant for the optical field takes the explicit form
\begin{equation}
	n_{zz}(\omega)
	\approx
	n^{(0)}_{zz}(\omega)
	+
	n^{(1)}_{zzx}(\omega)
	E^x(\Omega_0)
	.
\end{equation}
For practical Pockels media, e.g., Lithium-Niobate, the tensorial \gls{dof} can be substantially reduced by considering crystal symmetries, see, for instance, Ref.~\cite[p.~237]{Yariv1984}.

We derived the relation between the macroscopic polarization, response and refractive index and justified why we can treat the modulation field to be effectively static compared to the fast oscillating optical field.
For the last step where we summarize the unitary transformation associated with a Pockels phase modulator, we assume the dielectric to be constant on the time scale of the electrical field.

\subsection{Mach-Zehnder modulator}

The \gls{mzm} arranges two phase modulators to perform amplitude and phase modulation on an optical input field.
Using two electrical-driven phase modulators, the \gls{mzm} performs amplitude and phase modulation on an optical input field.
To begin with, we derive the quantum operator matrix transform from a specific implementation of the \gls{mzm}, the symmetric free-space \gls{mzi}.
Then, we link the quantum operator matrix transform to more general unitary operators.
Finally, we derive the quantum state transform for a coherent input state to the \gls{mzm}.

\begin{figure}[htb]
	\centering
	\includestandalone{figures/pstricks/mzi-symmetric}
	\caption{Free-space setup of a symmetric \gls{mzm}: The input light mode $\hat{a}_1$ enters a first beam splitter BS1 from the left. A vacuum light mode $\hat{a}_2$ enters BS2 from the top. The transformed mode $\hat{a}_1^\prime$ and $\hat{a}_2^\prime$ exit BS1 to the right and the bottom. A first phase shifter and first mirror M1, right to BS1, add a relative phase of $\varphi_1+\pi$ from mode $\hat{a}_1$ to $\hat{a}_1^{\prime\prime}$. Below BS1, a second mirror M2 directs the light to a right second phase shifter, both adding a relative phase of $\varphi_2+\pi$ from mode $\hat{a}_2^\prime$ to $\hat{a}_2^{\prime\prime}$. A second beam splitter BS2 transforms the input modes $\hat{a}_1^{\prime\prime}$ and $\hat{a}_2^{\prime\prime}$ to the output modes $\hat{a}_1^{\prime\prime\prime}$ and $\hat{a}_2^{\prime\prime\prime}$.}\label{fig:mzi_symmetric}
\end{figure}
\Cref{fig:mzi_symmetric} shows a free-space optics setup of a symmetric \gls{mzi} with one signal input; the other input being in the vacuum state.
The most crucial components of the \gls{mzi} are a splitter, a coupler, and two independent phase modulators.
The splitter divides the input light into two branches.
Each branch adds a relative phase shift using an independent phase modulator, i.e., $\phi_1$ and $\phi_2$.
The coupler recombines both branches into two outputs.
For our free-space setup, two cubic beam splitters implement the splitter (BS1) and the coupler (BS2).
For additional beam alignment, our free-space setup utilizes two mirrors (M1 and M2).

Finding the transformation of the annihilation operators at each stage of the \gls{mzi} is sufficient to find the quantum input-output relations.
Idealizing the symmetric \gls{mzi}'s passive components as lossless allows relating the annihilation operators by two-dimensional unitary matrices.
We label the input annihilation operators of the \gls{mzi} $\hat{a}_1,\hat{a}_2$ and the output annihilation operators $\hat{a}_1^{\prime\prime\prime},\hat{a}_2^{\prime\prime\prime}$.
The annihilation operators after splitting and before (after) phase shifting are denoted by two (three) primes, i.e., $\hat{a}_1^{\prime},\hat{a}_2^{\prime}$ and $\hat{a}_1^{\prime\prime},\hat{a}_2^{\prime\prime}$.
Going backwards through the transformations from the output to the input annihilation operators
\begin{equation}
	\vb{\hat{a}}^{\prime\prime\prime}
	=
	U_\text{BS2}
	\hat{\vb{a}}^{\prime\prime}
	=
	U_\text{BS2}
	U_\text{PS}
	\hat{\vb{a}}^{\prime}
	=
	U_\text{BS2}
	U_\text{PS}
	U_\text{BS1}
	\hat{\vb{a}}
	=
	U_\text{MZI}
	\vb{\hat{a}}
\end{equation}
we find the symmetric \gls{mzi}'s unitary matrix transform $U_\text{MZM}$ to be equal to the matrix product of the second beam splitter's, the phase shifts', and the first beam splitter's unitary matrix transform  $U_\text{BS2}U_\text{PS}U_\text{BS1}$.

An ideal cubic beam splitter with a single dielectric layer has the unitary matrix transform~\cite[p.~139]{Gerry2005}
\begin{equation}
	U_\text{BS1}
	=
	\frac{1}{\sqrt{2}}
	\begin{pmatrix}
		1 & i \\
		i & 1
	\end{pmatrix}
\end{equation}
where the off-diagonal elements of $U_\text{BS1}$, $i/\sqrt{2}$, account for the phase-shift due to the reflection at the diagonal of the cubic beam splitter.
The matrix encoding the phase shifts from the phase modulation $\phi_1,\phi_2$ and the reflection at the mirrors M1 and M2, $\pi$ is
\begin{equation}
	U_\text{PS}
	=
	\begin{pmatrix}
		ie^{i\phi_1} & 0 \\
		0 & ie^{i\phi_2}
	\end{pmatrix}
\end{equation}
For the second beam splitter, BS2, we again assume an ideal cubic beam splitter with a single dielectric layer.
The corresponding matrix transform is
\begin{equation}
	U_\text{BS2}
	=
	\frac{1}{\sqrt{2}}
	\begin{pmatrix}
		i & 1 \\
		1 & i
	\end{pmatrix}	
\end{equation}
where we exchanged the rows for consistency with the input labels.

Performing the matrix multiplication and writing the exponentials as trigonometric functions, we find the matrix transform of the symmetric \gls{mzi} to be
\begin{equation}
	U_\text{MZI}
	=
	-
	\begin{pmatrix}
		\cos(\frac{\phi_2-\phi_1}{2}) & \sin(\frac{\phi_2-\phi_1}{2}) \\
		-\sin(\frac{\phi_2-\phi_1}{2}) & \cos(\frac{\phi_2-\phi_1}{2})
	\end{pmatrix}
	e^{i\frac{\phi_1+\phi_2}{2}}
	.
\end{equation}
It appears useful to define the common-mode and differential-mode phases
\begin{align}
	\phi_+
	&=
	\phi_2
	+
	\phi_1
	&
	\phi_-
	&=
	\phi_2-\phi_1
\end{align}
for which the matrix transform simplifies to
\begin{equation}
	U_\text{MZI}
	=
	-
	\begin{pmatrix}
		\cos(\phi_-/2) & \sin(\phi_-/2) \\
		-\sin(\phi_-/2) & \cos(\phi_-/2)
	\end{pmatrix}
	e^{i\phi_+/2}
\end{equation}
and we note that the common-mode phase $\phi_+$ adds a global phase shift of $\phi_+/2$ while the differential-mode phase $\phi_-$ changes the splitting ratios at the output.

For a general lossless \gls{mzm}, we propose the generic unitary matrix transform~\cite[p.~95]{Leonhardt2010}
\begin{equation}
	\begin{split}
		U_\text{MZM}
		&=
		e^{i\Lambda/2}
		\begin{pmatrix}
			\cos(\Theta/2)e^{i(+\Phi+\Psi)/2} & \sin(\Theta/2)e^{i(+\Phi-\Psi)/2} \\
			-\sin(\Theta/2)e^{+(-\Phi+\Psi)/2} & \cos(\Theta/2)e^{i(-\Phi-\Psi)/2}
		\end{pmatrix}
		\\
		&=
		e^{i\Lambda/2}
		\begin{pmatrix}
			e^{+i\Phi/2} & 0 \\
			0 & e^{-i\Phi/2}
		\end{pmatrix}
		\begin{pmatrix}
			\cos(\Theta/2) & \sin(\Theta/2) \\
			-\sin(\Theta/2) & \cos(\Theta/2)
		\end{pmatrix}
		\begin{pmatrix}
			e^{+i\Psi/2} & 0 \\
			0 & e^{-i\Psi/2}
		\end{pmatrix}
	\end{split}
	\label{eq:mzm_matrix}
\end{equation}
wherein the global phase $\Theta$ and the rotation angle $\Lambda$ are time-dependent but $\Phi,\Psi$ are constants.
We obtain the matrix transform of the symmetric free-space $U_\text{MZI}$ after identification of $\Theta$ with the differential-mode phase, $\Lambda+\pi$ with the common-mode phase, and choosing $\Psi=0=\Phi$.
One advantage of the proposed decomposition in \cref{eq:mzm_matrix} is the one-to-one correspondence between the unitary operator~\cite[p.~99]{Leonhardt2010}
\begin{equation}
	\hat{U}_\text{MZM}
	=
	e^{i\Lambda\hat{L}_t}
	e^{i\Phi\hat{L}_z}
	e^{i\Theta\hat{L}_y}
	e^{i\Psi\hat{L}_z}
	\label{eq:mzm_operator}
\end{equation}
wherein $\hat{L}_j$ denote the Jordan-Schwinger operators~\cite[p.~97]{Leonhardt2010}
\begin{equation}
	\hat{L}_j
	=
	\frac{1}{2}
	\begin{pmatrix}
		\hat{a}_1^\dagger, \hat{a}_2^\dagger
	\end{pmatrix}
	\sigma_j
	\begin{pmatrix}
		\hat{a}_1 \\
		\hat{a}_2
	\end{pmatrix}
	=
	\frac{1}{2}
	\hat{\vb{a}}^\dagger
	\sigma_j
	\hat{\vb{a}}
\end{equation}
with $\sigma_j$ being the two-dimensional Pauli matrices and $\sigma_0$ being the identity matrix, and the unitary matrix transform
\begin{equation}
	U_\text{MZM}
	\hat{\vb{a}}
	=
	\hat{U}_\text{MZM}^\dagger
	\hat{\vb{a}}
	\hat{U}_\text{MZM}
	.
\end{equation}

We apply the unitary operator for the \gls{mzm}, \cref{eq:mzm_operator}, and 
Let us consider the input state
\begin{align}
	\ket{\vb{\alpha}}
	&=
	\ket{\alpha_1,\alpha_2}
	&
	\vb{\alpha}
	=
	\begin{pmatrix}
		\alpha_1,
		\alpha_2
	\end{pmatrix}
	\in
	\mathbb{C}^{2\times1}
\end{align}
to a \gls{mzm}\footnote{We can always set $alpha=0$ to set the second input to vacuum.}, i.e.,
\begin{equation}
	\hat{U}_\text{MZM}
	\ket{\vb{\alpha}}
	=
	\hat{U}_\text{MZM}
	\hat{D}(\vb{\alpha})
	\hat{U}_\text{MZM}^\dagger
	\hat{U}_\text{MZM}
	\ket{0,0}
	=
	\hat{U}_\text{MZM}
	\hat{D}(\vb{\alpha})
	\hat{U}_\text{MZM}^\dagger
	\ket{0,0}
\end{equation}
where we used the invariance of the vacuum state $\ket{0,0}$ for the second equal.
The displacement operator transforms as
\begin{equation}
	\begin{split}
		\hat{U}
		\hat{D}(\vb{\alpha})
		\hat{U}^\dagger
		&=
		\hat{U}
		\exp\left\{
			\vb{\alpha}^\trans
			\hat{\vb{a}}^\dagger
			-
			\hat{\vb{a}}
			\vb{\alpha}^*
		\right\}
		\hat{U}^\dagger
		\\
		&=
		\exp\left\{
			\vb{\alpha}^\trans
			\hat{U}
			\hat{\vb{a}}^\dagger
			\hat{U}^\dagger
			-
			\hat{U}
			\hat{\vb{a}}
			\hat{U}^\dagger
			\vb{\alpha}^*
		\right\}
		\\
		&=
		\exp\left\{
			\vb{\alpha}^\trans
			\left(
				\hat{U}
				\hat{\vb{a}}
				\hat{U}^\dagger
			\right)^\dagger
			-
			\hat{U}
			\hat{a}
			\hat{U}^\dagger
			\vb{\alpha}^*
		\right\}
	\end{split}
\end{equation}
where we used the operator identity
\begin{equation}
	\hat{U}
	e^{\hat{A}}
	\hat{U}^\dagger
	=
	\sum_{n=0}^\infty
	\frac{1}{n!}
	\hat{U}
	\hat{A}^n
	\hat{U}^\dagger
	=
	\sum_{n=0}^\infty
	\frac{1}{n!}
	\hat{U}
	\hat{A}
	\hat{U}^\dagger
	\cdots
	\hat{U}
	\hat{A}
	\hat{U}^\dagger
	=
	\sum_{n=0}^\infty
	\frac{1}{n!}
	\left(
		\hat{U}
		\hat{A}
		\hat{U}^\dagger
	\right)^n
	=
	e^{\hat{U}\hat{A}\hat{U}^\dagger}
\end{equation}
which follows from inserting $\hat{U}\hat{U}^\dagger=\mathbb{1}$ between the $\hat{A}$s.
For the unitary operator of the \gls{mzm}, we note that
\begin{equation}
	\begin{split}
		\hat{U}_\text{MZM}(\Lambda,\Phi,\Psi,\Theta)^\dagger
		&=
		e^{-i\Psi\hat{L}_z}
		e^{-i\Theta\hat{L}_y}
		e^{-i\Phi\hat{L}_z}
		e^{-i\Lambda\hat{L}_t}
		\\
		&=
		e^{-i\Lambda\hat{L}_t}
		e^{-i\Psi\hat{L}_z}
		e^{-i\Theta\hat{L}_y}
		e^{-i\Phi\hat{L}_z}
		\\
		&=
		\hat{U}_\text{MZM}(-\Lambda,\Psi,\Phi,-\Theta)
	\end{split}
\end{equation}
where we used in the second line that $\hat{L}_t$ commutes with the other Jordan-Schwinger operators $\hat{L}_y,\hat{L}_z$.
The transformed annihilation operators turn out to be
\begin{equation}
	\begin{split}
		\hat{\vb{a}}^\prime
		&=
		\hat{U}_\text{MZM}(\Lambda,\Phi,\Psi,\Theta)
		\hat{\vb{a}}
		\hat{U}_\text{MZM}(\Lambda,\Phi,\Psi,\Theta)^\dagger
		\\
		&=
		\hat{U}_\text{MZM}(-\Lambda,\Psi,\Phi,-\Theta)^\dagger
		\hat{\vb{a}}
		\hat{U}_\text{MZM}(-\Lambda,\Psi,\Phi,-\Theta)
		\\
		&=
		U_\text{MZM}(-\Lambda,\Psi,\Phi,-\Theta)
		\hat{\vb{a}}
		\\
		&=
		U_\text{MZM}(\Lambda,\Psi,\Phi,\Theta)^\dagger
		\hat{\vb{a}}
		.
	\end{split}
\end{equation}
Using the transformed operators, we find the transformed displacement operator to be~\cite[p.~210]{Vogel2006}
\begin{equation}
	\hat{D}^\prime(\vb{\alpha})
	=
	\exp\left\{
		\vb{\alpha}^\trans
		\left(\hat{\vb{a}}^\prime\right)^\dagger
		-
		\hat{\vb{a}}^\prime
		\vb{\alpha}^*
	\right\}
	=
	\exp\left\{
		\left(\vb{\alpha}^\prime\right)^\trans
		\hat{\vb{a}}^\dagger
		-
		\hat{\vb{a}}
		\left(\vb{\alpha}^\prime\right)^*
	\right\}
	.
\end{equation}
We therefore find the \gls{mzm} to transform the coherent input states according to
\begin{align}
	\hat{U}_\text{MZM}
	\ket{\vb{\alpha}}
	&=
	\ket{\vb{\alpha}^\prime}
	&
	\vb{\alpha}^\prime
	&=
	U_\text{MZM}
	\vb{\alpha}
	.
\end{align}
Using the explicit matrix elements in \cref{eq:mzm_matrix} and setting the second input to zero $\alpha_2=0$ and identifying the first as $\alpha_1=\alpha$, we find
\begin{equation}
	\hat{U}_\text{MZM}
	\ket{\alpha,0}
	=
	\ket{\alpha\cos(\Theta/2)e^{+i(\Psi+\Phi)/2},-\alpha\sin(\Theta/2)e^{-i(\Psi-\Phi)/2}}
	.
\end{equation}

The \gls{mzm} produces highly entangled output states for (displaced) number states, see, for instance, Ref.~\cite{Windhager2011}.

\subsection{In-phase and quadrature modulator}

The \gls{iqm} independently modulates the quadrature components of coherent light by first splitting the signal into two branches, where each branch has an \gls{mzm}, and then recombining the two branches.
The \gls{mzm} \gls{pm} must hold a total phase difference of $\pi/2$ between the branches.
Otherwise, the quadrature components become correlated.

\Cref{fig:iqm} presents a fiber-optical embodiment of an \gls{iqm} where we labeled the in- and output modes using the corresponding quantum annihilation operator.
We indicated non-signal in- and outputs by a dashed orange line.
The non-signal inputs must be in a vacuum state.
The non-signal outputs may be dumped or monitored.
Two voltage signals, not indicated in the figure, drive each of the \gls{mzm}.
\begin{figure}[htb]
	\centering
	\includestandalone{figures/pstricks/iqm}
	\caption{Fiber-optical setup of a symmetric \gls{iqm}: A first coupler splits the input modes denoted by the quantum operator $\hat{a}_2,\hat{a}_3$ among the output modes $\hat{a}_2^\prime,\hat{a}_3^\prime$. A first \gls{mzm}, MZM1, transforms the input modes $\hat{a}_1=\hat{a}_1^\prime$ and $\hat{a}_2^\prime$ into the outputs modes $\hat{a}_1^{\prime\prime}=\hat{a}_1^{\prime\prime\prime}$ and $\hat{a}_2^{\prime\prime}$. A second \gls{mzm}, MZM2, transforms input modes $\hat{a}_3^\prime$ and $\hat{a}_4=\hat{a}_4^{\prime}$ into the outputs modes $\hat{a}_3^{\prime\prime}$ and $\hat{a}_4^{\prime\prime}=\hat{a}_4^{\prime\prime\prime}$. 	The input mode $\hat{a}_2$ is the signal input. The other input modes, $\hat{a}_1,\hat{a}_3,\hat{a}_4$, are vacuum. A second coupler, combines modes $\hat{a}_2^{\prime\prime}$ and $\hat{a}_3^{\prime\prime}$ into output modes $\hat{a}_2^{\prime\prime\prime}$ and $\hat{a}_3^{\prime\prime\prime}$. Output mode $\hat{a}_2^{\prime\prime\prime}$ is the signal output.}\label{fig:iqm}
\end{figure}

In the previous section, we motivated the result that the amplitudes of a coherent input state transform with the matrix as the annihilation operators.
We assume that this result generalizes to composite systems as we have in \gls{iqm}, and we are left to construct a possible matrix from the known matrix transforms of the coupler and the \gls{mzm}, i.e.,
\begin{equation}
	\begin{split}
		\begin{pmatrix}
			\alpha_1^{\prime\prime\prime} \\
			\alpha_2^{\prime\prime\prime} \\
			\alpha_3^{\prime\prime\prime} \\
			\alpha_4^{\prime\prime\prime}
		\end{pmatrix}
		&=
		\begin{pmatrix}
			 1 & 0 & 0 & 0 \\
			 0 & \multicolumn{2}{c}{\multirow{2}{*}{$U_\text{BS2}$}} & 0 \\
			 0 & & & 0 \\
			 0 & 0 & 0 & 1
		\end{pmatrix}
		\begin{pmatrix}
			\alpha_1^{\prime\prime} \\
			\alpha_2^{\prime\prime} \\
			\alpha_3^{\prime\prime} \\
			\alpha_4^{\prime\prime}
		\end{pmatrix}
		\\
		&=
		\begin{pmatrix}
			 1 & 0 & 0 & 0 \\
			 0 & \multicolumn{2}{c}{\multirow{2}{*}{$U_\text{BS2}$}} & 0 \\
			 0 & & & 0 \\
			 0 & 0 & 0 & 1
		\end{pmatrix}
		\begin{pmatrix}
			 \multicolumn{2}{c}{\multirow{2}{*}{$U_\text{MZM1}$}} & 0 & 0 \\
			 & & 0 & 0 \\
			 0 & 0 & \multicolumn{2}{c}{\multirow{2}{*}{$U_\text{MZM2}$}} \\
			 0 & 0 & & \\
		\end{pmatrix}
		\begin{pmatrix}
			\alpha_1^\prime \\
			\alpha_2^\prime \\
			\alpha_3^\prime \\
			\alpha_4^\prime
		\end{pmatrix}
		\\
		&=
		\begin{pmatrix}
			 1 & 0 & 0 & 0 \\
			 0 & \multicolumn{2}{c}{\multirow{2}{*}{$U_\text{BS2}$}} & 0 \\
			 0 & & & 0 \\
			 0 & 0 & 0 & 1
		\end{pmatrix}
		\begin{pmatrix}
			 \multicolumn{2}{c}{\multirow{2}{*}{$U_\text{MZM1}$}} & 0 & 0 \\
			 & & 0 & 0 \\
			 0 & 0 & \multicolumn{2}{c}{\multirow{2}{*}{$U_\text{MZM2}$}} \\
			 0 & 0 & & \\
		\end{pmatrix}
		\begin{pmatrix}
			 1 & 0 & 0 & 0 \\
			 0 & \multicolumn{2}{c}{\multirow{2}{*}{$U_\text{BS1}$}} & 0 \\
			 0 & & & 0 \\
			 0 & 0 & 0 & 1
		\end{pmatrix}
		\begin{pmatrix}
			\alpha_1 \\
			\alpha_2 \\
			\alpha_3 \\
			\alpha_4
		\end{pmatrix}
		,
	\end{split}
	\label{eq:iqm_amplitude}
\end{equation}
wherein $U_\text{BS1},U_\text{BS2},U_\text{MZM1},U_\text{MZM2}$ are the unitary matrices corresponding to the operator transformation of the respective components.

We assume the two couplers to be identical and balanced with an unknown relative phase shift of $\varphi$.
The matrix transform of such couplers is
\begin{equation}
	U_\text{BS1}(\varphi)
	=
	\frac{1}{\sqrt{2}}
	\begin{pmatrix}
		1 & ie^{+i\varphi} \\
		ie^{-i\varphi} & 1
	\end{pmatrix}
	=
	U_\text{BS2}(\varphi)
	.
\end{equation}
For the \gls{mzm}s, we use the matrix transformation derived in the previous section, \cref{eq:mzm_matrix}, where we set the relative phases to zero $\Psi=\Phi=0$
\begin{equation}
	U_\text{MZM}(\Lambda,0,0,\Theta)
	=
	\begin{pmatrix}
		\cos(\Theta/2) & \sin(\Theta/2) \\
		-\sin(\Theta/2) & \cos(\Theta/2)
	\end{pmatrix}
	e^{i\Lambda/2}
	.
\end{equation}
Setting the relative phases to zero is sufficient for outlining the rough operating principle.
For any practical implementations, we need to assume that these relative phases are non-zero, unknown, and possibly time-dependent due to thermal effects.
Comparing MZM1 and MZM2 we recognize exchanged in- and outputs.
Nevertheless, we assume MZM1 and MZM2 to be designed such that we do not need to permute the matrix transform, i.e.,
\begin{align}
	U_\text{MZM1}(\Theta_1)
	&=
	U_\text{MZM}(0,0,0,\Theta_1)
	\\
	U_\text{MZM2}(\Lambda,\Theta_2)
	&=
	U_\text{MZM}(\Lambda,0,0,\Theta_2)
\end{align}
where we have set one phase factor of the \gls{mzm}s to zero.

Inserting the previous matrices into \cref{eq:iqm_amplitude} and performing the matrix multiplication we find
\begin{equation}
	\begin{split}
		U_\text{IQM}
		=
		\begin{pmatrix}
			c_1
			&
			s_1/\sqrt{2}
			&
			ie^{+i\varphi}s_1/\sqrt{2}
			&
			0
			\\
			-s_1/\sqrt{2}
			&
			\left(c_1-e^{i\Lambda/2}c_2\right)/2
			&
			ie^{+i\varphi}\left(c_1+e^{i\Lambda/2}c_2\right)/2
			&
			ie^{+i\varphi}e^{i\Lambda/2}s_2/\sqrt{2}
			\\
			-ie^{-i\varphi}s_1/\sqrt{2}
			&
			ie^{-i\varphi}\left(c_1+e^{i\Lambda/2}c_2\right)/2
			&
			-\left(c_1-e^{i\Lambda/2}c_2\right)/2
			&
			e^{i\Lambda/2}s_2/\sqrt{2}
			\\
			0
			&
			-ie^{-i\varphi}e^{i\Lambda/2}s_2/\sqrt{2}
			&
			-e^{i\Lambda/2}s_2/\sqrt{2}
			&
			e^{i\Lambda/2}c_2
		\end{pmatrix}
	\end{split}
\end{equation}
where we introduced the shorthand notation
\begin{align}
	c_j
	&=
	\cos(\Theta_j/2)
	&
	s_j
	&=
	\sin(\Theta_j/2)
	.
\end{align}
We set $\Lambda=-\pi$ and write
\begin{equation}
	z
	=
	c_1
	-
	e^{-\pi/2}
	c_2
	=
	c_1
	+
	ic_2
\end{equation}
with which the matrix transform of the \gls{iqm} further simplifies to
\begin{equation}
	\begin{split}
		U_\text{IQM}
		=
		\begin{pmatrix}
			c_1
			&
			s_1/\sqrt{2}
			&
			ie^{+i\varphi}s_1/\sqrt{2}
			&
			0
			\\
			-s_1/\sqrt{2}
			&
			z/2
			&
			ie^{+i\varphi}z^*/2
			&
			e^{+i\varphi}s_2/\sqrt{2}
			\\
			-ie^{-i\varphi}s_1/\sqrt{2}
			&
			ie^{-i\varphi}z^*/2
			&
			-z/2
			&
			-is_2/\sqrt{2}
			\\
			0
			&
			-e^{-i\varphi}s_2/\sqrt{2}
			&
			is_2/\sqrt{2}
			&
			-ic_2
		\end{pmatrix}
		.
	\end{split}
\end{equation}
Assuming the input state
\begin{equation}
	\ket{0,\alpha,0,0}
\end{equation}
corresponding to a coherent signal state at the second input mode and the vacuum state in the other modes,
we find the transformed amplitudes to be
\begin{equation}
	\begin{pmatrix}
		\alpha_1^{\prime\prime\prime} \\
		\alpha_2^{\prime\prime\prime} \\
		\alpha_3^{\prime\prime\prime} \\
		\alpha_4^{\prime\prime\prime}
	\end{pmatrix}
	=
	U_\text{IQM}
	\begin{pmatrix}
		0 \\
		\alpha \\
		0 \\
		0
	\end{pmatrix}
	=
	\alpha
	\begin{pmatrix}
		s_1/\sqrt{2} \\
		z/2 \\
		ie^{-\varphi}z^*/2 \\
		-e^{-\varphi}s_2/\sqrt{2}
	\end{pmatrix}
	.
\end{equation}
Our main interest is in the second signal output mode which we obtain by projection of the second output mode, i.e.,
\begin{equation}
	\hat{P}_2
	\hat{U}_\text{IQM}
	\ket{0,\alpha,0,0}
	=
	\ket{\alpha z/2}
	\label{eq:iqm_state}
\end{equation}
wherein $\hat{P}_2$ is the projector of the second mode.
\Cref{eq:iqm_state} summarizes the final result of this section: The relation between the coherent signal in- and outputs of a \gls{iqm} is a multiplication with a complex number
\begin{equation}
	z
	=
	\cos(\Theta_1/2)
	+
	i\cos(\Theta_2/2)
\end{equation}
which real and imaginary part are determined by the cosine of the two \gls{mzm}s \gls{am} angles $\Theta_1,\Theta_2$.
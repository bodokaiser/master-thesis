\section{Transmitter}

% TODO: overview figure of how we show "equivalence" between the tensor product of coherent states with the time-continuous coherent state for the transmitter

\subsection{Pulse-shaping}

Pulse-shaping turns a sequence of (complex) symbols into a time-continuous signal.
For practical implementations, the pulse-shaping is performed in the digital domain.
Moving the pulse-shaping to the optical domain allows for a quantum mechanical description.

\begin{figure}[htb]
	\centering
    \includestandalone{figures/pstricks/quantum-pulse-shaping}
    \caption{Fiber-optical setup to perform pulse-shaping in the (quantum) optical domain. Three independent branches comprising a pulsed laser, \gls{iq}-modulator and a variable delay line are coupled to an optical \gls{lp} filter. The output of the optical \gls{lp} is passed on to a fiber.}\label{fig:quantum_pulse_shaping}
\end{figure}
\Cref{fig:quantum_pulse_shaping} presents a fiber-optical setup implementing pulse-shaping in the optical domain.
The generalization to $n$ independent symbols is straightforward by adding more \gls{iq}-modulated pulsed-lasers with a time delay to the coupler.
First, we need to produce independent pulses directly encoding the symbols, symbol pulses, by \gls{iq}-modulating pulsed lasers.
Second, we add a time delay $nT$ to each symbol pulse equal to the symbol period $T$ multiplied by the symbol index $n$.
Third, we superimpose the time-delayed symbol pulses through an optical coupler yielding a symbol pulse train.
Assuming the symbol pulses to be strongly peaked in the time domain, we can approximate the symbol pulse train as a sequence of Dirac pulses.
Pulse-shaping occurs when the symbol pulse train passes an optical filter.
We can model the optical filter as a frequency-dependent beam splitter with one vacuum input mode and one output mode dumped.

Let us present the quantum states and transformations relevant to \Cref{fig:quantum_pulse_shaping}.
The symbol pulse is a continuous-mode coherent state of the form
\begin{equation}
	\ket{\alpha(x,t)}
	=
	e^{-\overline{n}/2}
	\exp\left\{
		\int_0^\infty\dd{p}
		\alpha(p)
		e^{-ip(x-t)}
		\hat{a}^\dagger(p)
	\right\}
	\ket{0}
\end{equation}
wherein $\overline{n}$ is the mean photon number and
\begin{equation}
	\alpha(p)
	\propto
	\exp\left\{
		-\frac{1}{4}\left(\frac{p-p_0}{\sigma}\right)^2
	\right\}
\end{equation}
is a (complex) Gaussian spectrum centered at $p_0$ with spread $\sigma$.
In the limit of an extremely peaked Gaussian spectrum $\sigma\to0$, the Gaussian spectrum becomes a delta distribution and we obtain the single-mode limit
\begin{equation}
	\ket{\alpha(x,t)}
	\xrightarrow{\sigma\to0}
	e^{-\overline{n}/2}
	\exp\left\{
		\alpha(p_0)
		e^{-ip_0(x-t)}
		\hat{a}^\dagger(p_0)
	\right\}
	\ket{0}
	.
\end{equation}
The single-mode state abbreviates symbolically a strongly peaked Gaussian.
We should keep this in mind and transition to the correct continuous-mode description when necessary.

Adding a time delay $nT$ is equivalent to time-evolution from the initial time $t$ to $t+nT$, i.e.,
\begin{equation}
	\hat{U}(t,t+nT)
	\ket{\alpha(x,t)}
	=
	\ket{\alpha(x,t+nT)}
	=
	\ket{\alpha(x,t)e^{-ip_0nT}}
\end{equation}
wherein the time-shifted spectrum is equal to the initial spectrum times a time-delay as expected from Fourier theory.

The total quantum state is the tensor product of the individual symbol pulse states
\begin{equation}
	\bigotimes_{n=1}^N
	\ket{\alpha_n(x,t+nT)}
\end{equation}
where each symbol pulse is time delayed by the symbol period $T$ multiplied by the symbol index $n$.
According to Ref.~\cite{Zukowski1997}, a symmetric multiport beam splitter has matrix elements
\begin{equation}
	U_{mn}
	=
	\frac{1}{\sqrt{N}}
	\left(e^{i2\pi/N}\right)^{(n-1)(m-1)}
\end{equation}
and the first output mode of the multiport beam splitter provided the tensor product of symbol pulses is the symbol pulse train
\begin{equation}
	\ket{\alpha_1^\prime(x,t)}
	=
	\ket{\sum_{n=1}^NU_{1n}\alpha_n(x,t+nT)}
	=
	\ket{\frac{1}{\sqrt{N}}\sum_{n=1}^N\alpha_n(x,t+nT)}
	.
\end{equation}

For pulse-shaping, we use a time-dependent beam splitter with the symbol pulse train and a vacuum state as input.
The first output mode takes the form
\begin{equation}
	\ket{\alpha^{\prime\prime}(t)}
	=
	\ket{\cos[\theta(t)]\alpha_1^\prime(x,t)}
\end{equation}

% perform pulse-shaping using time-dependent (?) beam splitter

\FloatBarrier
\subsection{Up-conversion}

% perform up-conversion using sum-to-frequency generation
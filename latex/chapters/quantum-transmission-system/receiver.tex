\section{Receiver}

\subsection{Photodiode}

In most physics literature, the photocurrent of a photodiode illuminated by monochromatic light with angular frequency $\omega_0$ is given by~\cite[p.~650]{Saleh2007}
\begin{equation}
	i
	=
	\eta
	e
	\Phi_\gamma
\end{equation}
wherein $\eta$ is the \gls{qe}, $e$ is the electron charge, and $\Phi_\gamma$ is the photon flux impinging on the detector.
We again adapt natural units in which $e=1$.
The generalization to broadband light is straight forward,
\begin{equation}
	i(\omega)
	=
	\eta(\omega)
	\Phi_\gamma(\omega)
	,
\end{equation}
and in the time domain, we find the photocurrent to be
\begin{equation}
	i(t)
	=
	\int_\mathbb{R}\dd{\omega}
	\eta(\omega)
	\Phi_\gamma(\omega)
	=
	e\left(\eta*\Phi_\gamma\right)(t)
	,
\end{equation}
i.e., a convolution of the photon flux with the \gls{qe}.
Using the divergence theorem, we relate the photon flux onto the detector with the photons absorbed by the detector

% why is the usual description insufficient? -> no annihilation of the field, WHAT ELSE?
% what is the actual proposed POVM for photodetection?
The \gls{povm} for detecting $m$ photons in the time interval $[t,t+\Delta t]$ is~\cite[p.~192]{Vogel2006}
\begin{equation}
	\hat{P}_m
	=
	\colon
	\frac{1}{m!}
	\left(\eta\hat{n}\right)^m
	e^{-\eta\hat{n}}
	\colon
\end{equation}

% description of photodetection from a solid-state physics perspective!

% most complete and up-to-date review in Vogel (compare with Gardiner and others - literature review)
% schematically outline derivation of photodetection given in Vogel.

\subsection{Balanced detection}

% general citations:
% \cite{Kikuchi2016}
% 

% \cite[p.~206]{Vogel2006}
\begin{figure}[htb]
    \centering
    \includestandalone{figures/tikz/balanced-detector}
    \caption{Optical part of the balanced detector comprising a beam splitter and two photodetectors: The signal mode $\hat{a}_s$ enters the beam splitter from the left. The \gls{lo} mode $\hat{a}_l$ enters the beam splitter from the top. The first output mode $\hat{a}_+$ exits the beam splitter to the right where a first photodetector emits a photocurrent $i_+$. The second output mode $\hat{a}_-$ exits the beam splitter to the bottom where a second photodetector emits a photocurrent $i_-$.}\label{fig:balanced_detector_optics}
\end{figure}

\begin{figure}[htb]
    \centering
    \includestandalone{figures/circuitikz/balanced-detector}
    \caption{Electronic part of the balanced detector comprising two biassed photodiodes in balanced configuration with a \gls{tia} frontend: The cathode of a first photodiode PD+ is biased with a positive bias voltage $+V_b$. The anode of a second photodiode PD- is biased with a negative bias voltage $-V_b$. The photocurrent difference $i_+-i_-$ runs through the line connecting the anode of PD+ and the cathode of PD-. A \gls{tia} with complex feedback impedance $Z_f$ converts and amplifies the photocurrent difference $i_+-i_-$ to an output voltage $V_0$.}\label{fig:balanced_detector_electronics}
\end{figure}

\section{Receiver}

\subsection{Photodiode}

In most physics literature, the photocurrent of a photodiode illuminated by monochromatic light with angular frequency $\omega_0$ is given by~\cite[p.~650]{Saleh2007}
\begin{equation}
	i
	=
	\eta
	e
	\Phi_\gamma
\end{equation}
wherein $\eta$ is the \gls{qe}, $e$ is the electron charge, and $\Phi_\gamma$ is the photon flux impinging on the detector.
We again adapt natural units in which $e=1$.
The generalization to broadband light is straight forward,
\begin{equation}
	i(\omega)
	=
	\eta(\omega)
	\Phi_\gamma(\omega)
	,
\end{equation}
and in the time domain, we find the photocurrent to be
\begin{equation}
	i(t)
	=
	\int_{\mathbb{R}}\dd{\omega}
	\eta(\omega)
	\Phi_\gamma(\omega)
	=
	\left(\eta*\Phi_\gamma\right)(t)
	,
\end{equation}
i.e., a convolution of the photon flux with the \gls{qe}.
Using the divergence theorem, we relate the photon flux onto the detector with the photons absorbed by the detector

% why is the usual description insufficient? -> no annihilation of the field, WHAT ELSE?
% what is the actual proposed POVM for photodetection?
The \gls{povm} for detecting $m$ photons in the time interval $[t,t+\Delta t]$ is~\cite[p.~192]{Vogel2006}
\begin{equation}
	\hat{P}_m
	=
	\colon
	\frac{1}{m!}
	\left(\eta\hat{n}\right)^m
	e^{-\eta\hat{n}}
	\colon
\end{equation}

% description of photodetection from a solid-state physics perspective!

% most complete and up-to-date review in Vogel (compare with Gardiner and others - literature review)
% schematically outline derivation of photodetection given in Vogel.

\subsection{Balanced detection}

% literature review
% \cite{Kikuchi2016}
% \cite{Shapiro2009}
% \cite{Loudon2000}

% \cite[p.~206]{Vogel2006}
\begin{figure}[htb]
    \centering
    \includestandalone{figures/tikz/balanced-detector}
    \caption{Optical part of the balanced detector comprising a beam splitter and two photodetectors: A first signal mode $\hat{a}_s=\hat{a}_1$ enters the beam splitter from the left. A second \gls{lo} mode $\hat{a}_l=\hat{a}_2$ enters the beam splitter from the top. A first output mode $\hat{a}_1^\prime$ exits the beam splitter to the right where a first photodetector with photocurrent $i_1$ is placed. A second output mode $\hat{a}_2^\prime$ exits the beam splitter to the bottom where a second photodetector with photocurrent $i_2$ is placed.}\label{fig:balanced_detector_optics}
\end{figure}

We write the frequency-dependent beam splitter \cite[p.~207]{Vogel2006}
\begin{equation}
	\begin{pmatrix}
		\hat{a}_1^\prime(\omega) \\
		\hat{a}_2^\prime(\omega)
	\end{pmatrix}
	=
	\begin{pmatrix}
		t(\omega) & r(\omega) \\
		-r^*(\omega) & t^*(\omega)
	\end{pmatrix}
	\begin{pmatrix}
		\hat{a}_s(\omega) \\
		\hat{a}_l(\omega)
	\end{pmatrix}
	=
	\begin{pmatrix}
		t(\omega)\hat{a}_s(\omega) + r(\omega)\hat{a}_l(\omega) \\
		t^*(\omega)\hat{a}_l(\omega) - r^*(\omega)\hat{a}_s(\omega)
	\end{pmatrix}
\end{equation}
and find the transformed number operators of the output modes to be
\begin{align}
	\hat{n}_1^\prime(\omega)
	&=
	\abs{t(\omega)}^2
	\hat{n}_s(\omega)
	+
	\abs{r(\omega)}^2
	\hat{n}_l(\omega)
	+
	\left[
		t(\omega)^*
		r(\omega)
		\hat{a}_s^\dagger(\omega)
		\hat{a}_l(\omega)
		+
		\text{h.c.}
	\right]
	\\
	\hat{n}_2^\prime(\omega)
	&=
	\abs{r(\omega)}^2
	\hat{n}_s(\omega)
	+
	\abs{t(\omega)}^2
	\hat{n}_l(\omega)
	-
	\left[
		t(\omega)^*
		r(\omega)
		\hat{a}_s^\dagger(\omega)
		\hat{a}_l(\omega)
		+
		\text{h.c.}
	\right]
\end{align}
Assuming the frequency-dependent beam splitter to be sufficiently balanced
\begin{equation}
	\abs{t(\omega)}
	\approx
	\abs{r(\omega)}
\end{equation}
within the frequencies of the detector bandwidth $B$, the photon number difference between the detector is
\begin{equation}
	\begin{split}
		\hat{n}^\prime_\Delta(\omega)
		=
		\hat{n}_1^\prime(\omega)
		-
		\hat{n}_2^\prime(\omega)
		&\approx
		2\left[
			t(\omega)^*
			r(\omega)
			\hat{a}_s^\dagger(\omega)
			\hat{a}_l(\omega)
			+
			\text{h.c.}
		\right]
		\\
		&=
		2\abs{t(\omega)r(\omega)}
		\left[
			\hat{a}_s^\dagger(\omega)
			\hat{a}_l(\omega)
			e^{i(\phi_r-\phi_t)}
			+
			\text{h.c.}
		\right]
	\end{split}
\end{equation}
Assuming the quantum state~\cite[p.~213]{Vogel2006}
\begin{equation}
	\hat\rho(t)
	=
	\hat\rho_s(t)
	\otimes
	\hat\rho_l(t)
	=
	\hat\rho_s(t)
	\otimes
	\ketbra{\alpha_l(t)}
\end{equation}
wherein the \gls{lo} coherent state is
\begin{equation}
	\ket{\alpha_l(t)}
	=
	e^{-\overline{n}_l(t)/2}
	\exp\left\{
		\int_0^\infty\dd{\omega}
		\alpha_l(\omega,t)
		\hat{a}_l^\dagger(\omega)
	\right\}
	\ket{0}
	.
\end{equation}
We then find
\begin{equation}
	\begin{split}
		\overline{n}^\prime_\Delta(t)
		&=
		\int_0^\infty\frac{\dd{\omega}}{2\pi}
		\tr\left\{
			\hat\rho(t)
			\hat{n}_\Delta^\prime(\omega)
		\right\}
		\\
		&=
		2
		\int_0^\infty\frac{\dd{\omega}}{2\pi}
		\abs{t(\omega)r(\omega)}
		\tr_s\left\{
			\hat\rho_s(t)
			\expval{
				\hat{a}_s^\dagger(\omega)
				\hat{a}_l(\omega)
				e^{+i(\phi_r-\phi_t)}
				+
				\text{h.c.}
			}{\alpha_l(t)}
		\right\}
		\\
		&=
		2
		\int_0^\infty\frac{\dd{\omega}}{2\pi}
		\abs{t(\omega)r(\omega)}
		\tr_s\left\{
			\hat\rho_s(t)
			\expval{
				\hat{a}_s^\dagger(\omega)
				\alpha_l(\omega,t)
				e^{+i(\phi_r-\phi_t)}
				+
				\text{h.c.}
			}{\alpha_l(t)}
		\right\}
		\\
		&=
		2
		\int_0^\infty\frac{\dd{\omega}}{2\pi}
		\abs{t(\omega)r(\omega)\alpha_l(\omega)}
		\tr_s\left\{
			\hat\rho_s(t)
			\left[
				\hat{a}_s^\dagger(\omega)
				e^{-i\vartheta(t)}
				+
				\text{h.c.}
			\right]
		\right\}
		\\
		&=
		2\sqrt{2}
		\int_0^\infty\frac{\dd{\omega}}{2\pi}
		\abs{t(\omega)r(\omega)\alpha_l(\omega)}
		\tr_s\left\{
			\hat\rho_s(t)
			\hat{X}\left(\omega,\vartheta(t)\right)
		\right\}
		\\
		&=
		2\sqrt{2}
		\int_0^\infty\frac{\dd{\omega}}{2\pi}
		\abs{t(\omega)r(\omega)\alpha_l(\omega)}
		\expval{\hat{X}\left(\omega,\vartheta(\omega,t)\right)}_s
	\end{split}
\end{equation}
with
\begin{equation}
	\vartheta(\omega,t)
	=
	\phi_t-\phi_r-\varphi_l(\omega,t)
\end{equation}
and
\begin{equation}
	\hat{X}\left(\omega,\vartheta(t)\right)
	=
	\frac{1}{\sqrt{2}}
	\left(
		\hat{a}_s(\omega)
		e^{+i\vartheta(t)}
		+
		\hat{a}_s^\dagger(\omega)
		e^{-i\vartheta(t)}
	\right)
	.
\end{equation}
We can further write
\begin{equation}
	\overline{n}^\prime_\Delta(t)
	=
	\left(h*\overline{X}\right)(t)
\end{equation}
where we defined the filter
\begin{equation}
	h(\omega)
	=
	2\sqrt{2}
	\abs{t(\omega)r(\omega)\alpha_l(\omega)}
	.
\end{equation}

We are left to relate the photon number difference operator with the analog signals.
\Cref{fig:balanced_detector_electronics} shows the schematic of the balanced detector and \gls{tia} circuitry.
The photodiodes are both biased with bias voltage $\pm V_b$ to improve frequency response.
At the node between the anode of $\text{PD}_1$ and the cathode of $\text{PD}_2$, the two photocurrents $i_1$ and $i_2$ are directly subtracted.
\begin{figure}[htb]
    \centering
    \includestandalone{figures/circuitikz/balanced-detector}
    \caption{Electronic part of the balanced detector comprising two biassed photodiodes in balanced configuration with a \gls{tia} frontend: The cathode of a first photodiode $\text{PD}_1$ is biased with a positive bias voltage $+V_b$. The anode of a second photodiode $\text{PD}_2$ is biased with a negative bias voltage $-V_b$. The photocurrent difference $i_1-i_2$ runs through the line connecting the anode of $\text{PD}_1$ and the cathode of $\text{PD}_2$. A \gls{tia} with complex feedback impedance $Z_f$ converts and amplifies the photocurrent difference $i_1-i_2$ to an output voltage $V_o$.}\label{fig:balanced_detector_electronics}
\end{figure}
The mean of the photocurrent difference is equal to
\begin{equation}
	\overline{\Delta i}(t)
	=
	\overline{i}_1(t)
	-
	\overline{i}_2(t)
	=
	\int\dd{t^\prime}
	\eta(t^\prime)
	\overline{n}_\Delta^\prime(t-t^\prime)
\end{equation}
wherein $\eta$ is the \gls{qe} or frequency response of the photodiodes.
The mean photocurrent difference is amplified and converted to a voltage signal $V_0$ by the \gls{tia} frontend.
In particular,
\begin{equation}
	V_o(\omega)
	=
	-
	Z_f(\omega)
	\overline{\Delta i}(\omega)
\end{equation}
where $Z_F$ is the complex feedback impedance.
The mean voltage signal is equal to
\begin{equation}
	V_o(t)
	=
	-
	\left(Z_f*\overline{\Delta i}\right)(t)
	=
	-
	\left(Z_f*h*\overline{X}\right)(t)
	.
\end{equation}

The variance of a photodetector with efficiency $\eta$ is given by~\cite[p.~194]{Vogel2006}
\begin{equation}
	\begin{split}
		\overline{(\Delta n^\prime)^2}(\omega)
		&=
		\eta(\omega)
		\expval{\hat{n}^\prime(\omega)}
		+
		\eta(\omega)^2
		\expval{\colon\left(\Delta\hat{n}^\prime(\omega)\right)^2\colon}
		\\
		&=
		\eta(\omega)\left(1-\eta(\omega)\right)
		\expval{\hat{n}^\prime(\omega)}
		+
		\eta(\omega)^2
		\expval{\left(\Delta\hat{n}^\prime(\omega)\right)^2}
	\end{split}
\end{equation}
\begin{equation}
	\begin{split}
		\expval{\colon\hat{n}_\Delta^\prime(\omega)^2\colon}
		&=
		4\abs{t(\omega)r(\omega)}^2
		\expval{
			\colon
			\left[
				\hat{a}_s^\dagger(\omega)
				\hat{a}_l(\omega)
				e^{+i(\phi_r-\phi_l)}
				+
				\hat{a}_s(\omega)
				\hat{a}_l^\dagger(\omega)
				e^{-i(\phi_r-\phi_l)}
			\right]^2
			\colon
		}
		\\
		&=
		4\abs{t(\omega)r(\omega)}^2
		\expval{
			\hat{a}_s^\dagger(\omega)^2
			\hat{a}_l(\omega)^2
			e^{+2i(\phi_r-\phi_l)}
			+
			\hat{a}_l^\dagger(\omega)^2
			\hat{a}_s(\omega)^2
			e^{-2i(\phi_r-\phi_l)}
			+
			2
			\hat{a}_s^\dagger(\omega)
			\hat{a}_l^\dagger(\omega)
			\hat{a}_l(\omega)
			\hat{a}_s(\omega)
		}
		\\
		&=
		4\abs{t(\omega)r(\omega)}^2
		\left[
			2
			\overline{n}_s(t)
			\overline{n}_l(t)
			+
			\dots
		\right]
	\end{split}
\end{equation}
\section{Receiver}

\subsection{Photodiode}

In most physics literature, the photocurrent of a photodiode illuminated by monochromatic light with angular frequency $\omega_0$ is given by~\cite[p.~650]{Saleh2007}
\begin{equation}
	i
	=
	\eta
	e
	\Phi_\gamma
\end{equation}
wherein $\eta$ is the \gls{qe}, $e$ is the electron charge, and $\Phi_\gamma$ is the photon flux impinging on the detector.
We again adapt natural units in which $e=1$.
The generalization to broadband light is straight forward,
\begin{equation}
	i(\omega)
	=
	\eta(\omega)
	\Phi_\gamma(\omega)
	,
\end{equation}
and in the time domain, we find the photocurrent to be
\begin{equation}
	i(t)
	=
	\int_\mathbb{R}\dd{\omega}
	\eta(\omega)
	\Phi_\gamma(\omega)
	=
	\left(\eta*\Phi_\gamma\right)(t)
	,
\end{equation}
i.e., a convolution of the photon flux with the \gls{qe}.
Using the divergence theorem, we relate the photon flux onto the detector with the photons absorbed by the detector

% why is the usual description insufficient? -> no annihilation of the field, WHAT ELSE?
% what is the actual proposed POVM for photodetection?
The \gls{povm} for detecting $m$ photons in the time interval $[t,t+\Delta t]$ is~\cite[p.~192]{Vogel2006}
\begin{equation}
	\hat{P}_m
	=
	\colon
	\frac{1}{m!}
	\left(\eta\hat{n}\right)^m
	e^{-\eta\hat{n}}
	\colon
\end{equation}

% description of photodetection from a solid-state physics perspective!

% most complete and up-to-date review in Vogel (compare with Gardiner and others - literature review)
% schematically outline derivation of photodetection given in Vogel.

\subsection{Balanced detection}

% literature review
% \cite{Kikuchi2016}
% \cite{Shapiro2009}
% \cite{Loudon2000}

% \cite[p.~206]{Vogel2006}
\begin{figure}[htb]
    \centering
    \includestandalone{figures/tikz/balanced-detector}
    \caption{Optical part of the balanced detector comprising a beam splitter and two photodetectors: A first signal mode $\hat{a}_s=\hat{a}_1$ enters the beam splitter from the left. A second \gls{lo} mode $\hat{a}_l=\hat{a}_2$ enters the beam splitter from the top. A first output mode $\hat{a}_1^\prime$ exits the beam splitter to the right where a first photodetector with photocurrent $i_1$ is placed. A second output mode $\hat{a}_2^\prime$ exits the beam splitter to the bottom where a second photodetector with photocurrent $i_2$ is placed.}\label{fig:balanced_detector_optics}
\end{figure}

We write the frequency-dependent beam splitter \cite[p.~207]{Vogel2006}
\begin{equation}
	\begin{pmatrix}
		\hat{a}_1^\prime(\omega) \\
		\hat{a}_2^\prime(\omega)
	\end{pmatrix}
	=
	\begin{pmatrix}
		t(\omega) & r(\omega) \\
		-r^*(\omega) & t^*(\omega)
	\end{pmatrix}
	\begin{pmatrix}
		\hat{a}_s(\omega) \\
		\hat{a}_l(\omega)
	\end{pmatrix}
	=
	\begin{pmatrix}
		t(\omega)\hat{a}_s(\omega) + r(\omega)\hat{a}_l(\omega) \\
		t^*(\omega)\hat{a}_l(\omega) - r^*(\omega)\hat{a}_s(\omega)
	\end{pmatrix}
\end{equation}
and find the transformed number operators of the output modes to be
\begin{align}
	\hat{n}_1^\prime(\omega)
	&=
	T(\omega)
	\hat{n}_s(\omega)
	+
	R(\omega)
	\hat{n}_l(\omega)
	+
	\left[
		t(\omega)^*
		r(\omega)
		\hat{a}_s^\dagger(\omega)
		\hat{a}_l(\omega)
		+
		\text{h.c.}
	\right]
	\\
	\hat{n}_2^\prime(\omega)
	&=
	R(\omega)
	\hat{n}_s(\omega)
	+
	T(\omega)
	\hat{n}_l(\omega)
	-
	\left[
		t(\omega)^*
		r(\omega)
		\hat{a}_s^\dagger(\omega)
		\hat{a}_l(\omega)
		+
		\text{h.c.}
	\right]
\end{align}
wherein $R(\omega)=\abs{r(\omega)}^2$ is the reflection and $T(\omega)=\abs{t(\omega)}^2$ the transmission coefficient.

The quantum state comprises the signal and the \gls{lo}~\cite[p.~213]{Vogel2006}
\begin{equation}
	\hat\rho(t)
	=
	\hat\rho_s(t)
	\otimes
	\hat\rho_l(t)
	=
	\hat\rho_s(t)
	\otimes
	\ketbra{\alpha_l(t)}
\end{equation}
Then, the average photons impinging onto the detector in the time interval $[t,t+\Delta t]$ is
\begin{equation}
	\begin{split}
		\overline{n}_1^\prime(t)
		=
		\tr\left\{
			\hat\rho(t)
			\hat{n}_1^\prime
		\right\}
		&=
		\left(T*\overline{n}_s\right)(t)
		+
		\left(R*\overline{n}_l\right)(t)
		\\
		&+
		\int_\mathbb{R}\dd{\omega}
		\left[
			t(\omega)^*
			r(\omega)
			\alpha(\omega)
			\tr_s\left\{
				\hat\rho_s(t)
				\hat{a}_s^\dagger
			\right\}
			+
			\text{h.c.}
		\right]
	\end{split}
\end{equation}
and we find the photocurrents to be equal to
\begin{equation}
	i_j(t)
	=
	\int_{\mathbb{R}}\dd{t^\prime}
	\eta(t^\prime)
	\overline{n}(t-t^\prime)
	=
	\left(\eta*\overline{n}_j\right)(t)
\end{equation}
where $\eta(t)$ is the \gls{qe} of the photodiode.
\begin{figure}[htb]
    \centering
    \includestandalone{figures/circuitikz/balanced-detector}
    \caption{Electronic part of the balanced detector comprising two biassed photodiodes in balanced configuration with a \gls{tia} frontend: The cathode of a first photodiode $\text{PD}_1$ is biased with a positive bias voltage $+V_b$. The anode of a second photodiode $\text{PD}_2$ is biased with a negative bias voltage $-V_b$. The photocurrent difference $i_1-i_2$ runs through the line connecting the anode of $\text{PD}_1$ and the cathode of $\text{PD}_2$. A \gls{tia} with complex feedback impedance $Z_f$ converts and amplifies the photocurrent difference $i_1-i_2$ to an output voltage $V_o$.}\label{fig:balanced_detector_electronics}
\end{figure}

\begin{equation}
	\begin{split}
		\overline{n}_\pm(t)
		=
		\expval{\hat{n}_\pm(t)}
		&=
		\tr\left\{
			\hat\rho(t)
			\hat{n}_\pm(t)
		\right\}
		\\
		&=
		\frac{1}{2}
		\left[
			\tr_s\left\{
				\hat\rho_s(t)
				\hat{n}_s
			\right\}
			+
			\tr_l\left\{
				\hat\rho_l(t)
				\hat{n}_l
			\right\}
			-
			\sqrt{2}
			\abs{\alpha_l}
			\expval{\hat{X}_s(\vartheta)}
		\right]
	\end{split}
\end{equation}
wherein we found the generalized quadrature operator for the signal mode
\begin{equation}
	\hat{X}_s(\vartheta)
	=
	\frac{1}{\sqrt{2}}
	\left(
		\hat{a}_s
		e^{+i\vartheta}
		+
		\hat{a}_s^\dagger
		e^{-i\vartheta}
	\right)
\end{equation}
with $\vartheta$ being the phase difference between the signal and \gls{lo}.


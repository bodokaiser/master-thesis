\gls{qkd} heavily exploits light's quantum properties and demanding for a consistent quantum theory of light.
A suitable quantum theory of light should build on a few well-established fundamental axioms and transparently extend these to model physical reality.

The usual approach to a quantum theory of light using non-relativistic quantum mechanics, as found in, for instance, Ref.~\cite{Gerry2005} or Ref.~\cite{Fox2006}, is limited to a single frequency mode.
While the single-mode description successfully explains many quantum optics experiments, it fails to account for the inherent time- and bandwidth limitations of any realistic experiment.
Even worse, a single-frequency mode corresponding to a plane-wave is not square-integrable and therefore not an element of the Hilbert space, thereby violating one of quantum mechanics' fundamental axioms.
Only two books in the quantum optics literature, known by the author, describing a continuous-mode quantum theory of light: \citetitle{Barnett2002} by S. Barnet and P. M. Radmore and \citetitle{Loudon2000} by R. Loudon.
While the book by S. Barnet~\cite{Barnett2002} gives an in-depth presentation of continuous-mode light states, it fails to elaborate on essential observable statistics like the electric field strength's expectation value.
Loudon's book~\cite{Loudon2000} presents observable statistics in a continuous-mode formalism but emphasizes the breadth over the depth.
Outside the quantum optics literature, J. Shapiro provides a practical continuous-mode description from a quantum optical communication viewpoint~\cite{Shapiro2009}.
Instead of transparently extending well-established axioms, these references are sparse in justifying their claims.
If we approach the quantized electromagnetic field from the theoretical particle physics as in \citetitle{Peskin1995} by M. Peskin and D. Schroeder, and \citetitle{Greiner2013} by W. Greiner, we find a transparent construction of quantum fields from first principles.
Nevertheless, most literature on quantum field theory has very little reference to optical applications.
Furthermore, most literature on quantum field theory is limited to the plane-wave picture  with the exception of few advanced books like \citetitle{Itzykson2012} and \citetitle{Fulling1989}.
Interestingly enough, normalizable wave packets appear in the mathematical physics literature and axiomatic quantum field theory as smearing (or test) functions.
Unfortunately, these sources are difficult to understand with formal physics training and often lose sight of the reality they should describe.
Altogether, there has been no attempt to motivate a continuous-mode quantum optics formalism from fundamental quantum field theory in a comprehensible language.

The present chapter summarizes the literature's key insights to derive a consistent continuous-mode quantum theory of light transparently from first principles.
We start with a recap of the harmonic oscillator and introduce the Lagrange and Hamilton formalism of classical mechanics, followed by the canonical quantization procedure tp derive and discuss the quantum harmonic oscillator.
In Klein-Gordon theory, we perform canonical quantization of a relativistic field theory to obtain the essential building block of a quantum field theory, the quantum Klein-Gordon theory.
Most results of the quantum Klein-Gordon field translate to a scalar treatment of the Maxwell field and we are left to discuss complications arising from the gauge invariance and the vector form
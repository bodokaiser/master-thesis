\section{Gaussian-based protocols}

In \gls{cvqkd}, the receiver performs a continuous-value measurement, i.e., the set of possible outcomes is uncountable.
In comparison to \gls{dvqkd}, \gls{cvqkd} generates random bits faster but with higher error.
Grosshans initially proposed \gls{cvqkd} in 2002 as an alternative to DV-QKD~\cite{Grosshans2002}.
Today, CV-QKD has grown into solid competition with DV-QKD~\cite{Diamanti2016}.
One significant advantage of CV-QKD over DV-QKD is that the CV-QKD receiver uses standard telecommunication components allowing for cheap high-performance integration.

\subsection{Coherent-encoding}
% relevance of Gaussian RV to CV-QKD?
For an introduction to 

% citations:
% \cite{Weedbrook2012,Ferraro2005} Gaussian quantum information (quantum channel, cv-qkd, measurements)
% \cite{Lodewyck2007} complete description of CV-QKD device (theoretical and experimental)

% measurement of a Gaussian variable (1d)
% protocol table
% post-processing

\begin{figure}[htb]
	\centering
	\includestandalone{figures/pstricks/gaussian-coherent-transmitter}
	\caption{Fiber-optical setup of the phase-encoding \gls{cvqkd} protocol:}
\end{figure}

\begin{figure}[htb]
	\centering
	\includestandalone{figures/pstricks/gaussian-coherent-receiver-active}
	\caption{Fiber-optical setup of the phase-encoding \gls{cvqkd} protocol:}
\end{figure}

\begin{figure}[htb]
	\centering
	\includestandalone{figures/pstricks/gaussian-coherent-receiver-passive}
	\caption{Fiber-optical setup of the phase-encoding \gls{cvqkd} protocol:}
\end{figure}

% dual homodyne

% tensor product input state
\section*{Summary}

\begin{table}[htb]
	\centering	
	\begin{tabular}{lcc}
		\toprule
			& Qubit (spin) & Boson (quadrature) \\
		\midrule
			Visualization & Bloch sphere & Phase space \\
			Hilbert space (dim) & Finite (two) & Uncountable (infinite) \\
			Description & Density matrix & Wigner distribution \\
			Observable & $\vb{\hat{S}}(\vb{n})=\hat{S}_in^i$ & $\hat{X}(\vartheta)=\frac{1}{\sqrt{2}}\left(\hat{a}e^{-i\vartheta}+\hat{a}^\dagger e^{+i\vartheta}\right)$ \\
			Standard basis & $\left\{\ket{0},\ket{1}\right\}$ & $\left\{\ket{x}\right\}_{x\in\mathbb{R}}$ or $\left\{\ket{p}\right\}_{p\in\mathbb{R}}$ \\
			Basis transformation & Rotation & Unitary \\
			Other relevant states & $\ket{x_\pm},\ket{y_\pm},\ket{z_\pm}$ & $\ket{n},\ket{\alpha},\ket{\alpha,\xi}$ \\
		\bottomrule
	\end{tabular}
	\caption{Possible physical systems to encode a qubit. Many of the physical systems have a much higher dimension than the qubit space but can pre reduced to a qubit space by a proper mapping.}
\end{table}

\Cref{fig:qkd_protocol} illustrates our proposed notion of a \gls{qkd} protocol, with the feature being a logical quantum system from which the random bits are encoded and decoded.
The logical quantum system is a subspace of the physical quantum system.
The physical quantum system depends strongly on the physical implementation and quantum encoding.
\begin{figure}[htb]
	\centering
	\includestandalone{figures/tikz/qkd-protocol}
	\caption{A \gls{qkd} protocol comprises a binary encoder, a logical quantum system, and a binary decoder. The binary encoder maps bits $\vb{b}\in\{0,1\}^n$ onto a quantum state of the logical quantum system $\ket{\psi}$. The binary decoder extracts the bits $\vb{b}$ back from the quantum state $\ket{\psi^\prime}$. The logical quantum system is a subspace of a larger physical quantum system. The state encoder and decoder map between the logical and physical quantum states.}\label{fig:qkd_protocol}
\end{figure}
The distinction between logical and quantum systems is vital to separate the implementation and security concerns.
Many security proofs show equivalence between the physical implementation and the logical system to use an established security proof.
One should keep in mind that such a separation implicitly assumes no loopholes from the particular implementation.
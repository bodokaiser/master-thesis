\section{Boson- or quadrature-based protocols}

The quantum system of interest in boson-based protocols is a single bosonic mode, i.e., a quantum harmonic oscillator.

The central observable is the generalized quadrature operator~\cite[p.~36]{Barnett2002}
\begin{equation}
	\hat{X}(\vartheta)
	=
	\frac{1}{\sqrt{2}}
	\left(
		\hat{a}
		e^{-i\vartheta}
		+
		\hat{a}^\dagger
		e^{+i\vartheta}
	\right)
	=
	\hat{X}
	\cos\vartheta
	+
	\hat{P}
	\sin\vartheta
\end{equation}
where $\hat{X}=\hat{X}(0)$ and $\hat{P}=\hat{X}(\pi/2)$ are the position respectively momentum operators.
The generalized quadrature operator has commutator
\begin{equation}
	\comm{\hat{X}(\vartheta)}{\hat{X}(\vartheta+\Delta\vartheta)}
	=
	i\sin\Delta\vartheta
\end{equation}
implying maximal uncertainty for orthogonal quadratures $\Delta\vartheta=\pi/2$.
The Robertson uncertainty relation provides a lower bound for the product of quadrature variances differing by the angle $\Delta\vartheta$
\begin{equation}
	\expval{\Delta\hat{X}(\vartheta)}
	\expval{\Delta\hat{X}(\vartheta+\Delta\vartheta)}
	\geq
	\frac{1}{2}
	\abs{\expval{\comm{\hat{X}(\vartheta)}{\hat{X}(\vartheta+\Delta\vartheta)}}}
	=
	\frac{1}{2}
	\sin\Delta\vartheta
\end{equation}
and thereby generalizing Heisenberg's uncertainty principle.
Let us assume the existence of position $\ket{x}$ and momentum $\ket{p}$ eigenstates satisfying
\begin{align}
	\hat{X}(\vartheta)
	\ket{x}
	&=
	x
	\ket{x}
	&
	\hat{X}(\vartheta+\pi/2)
	\ket{p}
	&=
	p
	\ket{p}
	,
\end{align}
then the uncertainty relation implies that position and momentum are conjugate variables, i.e., increasing the precision of one variable decreases the precision of the other.
Unsurprisingly, we can show that position and momentum eigenstates are non-orthogonal~\cite[p.~29]{Mukhanov2007}
\begin{equation}
	\braket{x}{p}
	=
	\frac{e^{ipx}}{\sqrt{2\pi}}
	\label{eq:position_momentum_product}
	.
\end{equation}
Using \cref{eq:position_momentum_product} and the completeness of the position and momentum basis, we can relate them by a unitary transform, the Fourier transform,
\begin{equation}
	\ket{x}
	=
	\int\dd{p}
	\braket{p}{x}
	\ket{p}
	=
	\int\frac{\dd{p}}{\sqrt{2\pi}}
	e^{-ipx}
	\ket{p}
	=
	\hat{U}
	\ket{p}
	.
\end{equation}
Therefore, we can view the uncertainty relation between position and momentum in a semi-classical picture as encoded in the Fourier transform.

The non-orthogonality of position and momentum eigenstates, makes the bosonic system a perfect candidate for \gls{qkd} and we can easily construct a boson-based \gls{qkd} protocol similar to the qubit-based BB84:
\begin{enumerate}
	\item Alice prepares either a position eigenstate $\ket{x_i}$ or a momentum eigenstate $\ket{p_j}$ and sends it to Bob.
	\item Bob performs a measurement either in the position $\left\{\ket{x}\right\}_{x\in\mathbb{R}}$ or the momentum eigenbasis $\left\{\ket{p}\right\}_{p\in\mathbb{R}}$.
\end{enumerate}
If Bob measures in the correct basis, he (theoretically) is able to resolve perfectly $x_i,p_j\in\mathbb{R}$.
Otherwise, Bob measures an outcome completely uncorrelated with what Alice has prepared, see \Cref{tab:boson_transmission_sequence}.
\begin{table}[htb]
	\centering
	\begin{tabular}{llccccc}
		\toprule
		& & \multicolumn{5}{c}{Transmission} \\
		\cmidrule{3-7}
		Party & Step & 1 & 2 & 3 & 4 & 5 \\ 
		\midrule
		\multirow{3}{*}{Alice} & State value & $p_1$ & $x_1$ & $x_2$ & $p_2$ & $x_3$ \\
		& State basis & $P$ & $X$ & $X$ & $P$ & $X$ \\
		& Prepared state & $\ket{p_1}$ & $\ket{x_1}$ & $\ket{x_2}$ & $\ket{p_2}$ & $\ket{x_3}$ \\
		\cmidrule{1-1}
		\multirow{2}{*}{Bob} & Measurement basis & $X$ & $P$ & $X$ & $P$ & $P$ \\
		& Sifted outcome & - & - & $x_2$ & $p_2$ & - \\
		\bottomrule
	\end{tabular}
	\caption{Possible transmission sequence for boson-based BB84: Alice randomly selects a value and a basis, encodes this information into a quantum state and sends it to Bob. Bob randomly selects a measurement outcome. Only if Alice's and Bob's basis match, is Bob's outcome correlated with Alice's value.}\label{tab:boson_transmission_sequence}
\end{table}
To convert the sifted outcome to bits, we can simply assign the bit value according to the sign.
While the suggested boson-based BB84 highlights the differences of boson- with qubit-based \gls{qkd} it cannot be implemented as there are no physical position or momentum eigenstates as there is always an uncertainty involved.

However, we can use squeezed states as an approximation to the position and momentum eigenstates.
We are going to discuss such a squeeze-encoding in the subsequent section.

\FloatBarrier
\subsection{Squeezed-coherent-encoding}

A squeezed coherent state, denoted $\ket{\alpha,\xi}$, has expected quadrature~\cite[p.~91,94]{Vogel2006}
\begin{equation}
	\expval{\hat{X}(\vartheta)}{\alpha,\xi}
	=
	\frac{1}{\sqrt{2}}
	\left[
		\left(\mu\beta-\nu\beta^*\right)
		e^{+i\vartheta}
		+
		\text{c.c.}
	\right]
\end{equation}
with standard deviation~\cite[p.~95]{Vogel2006}
\begin{equation}
	\expval{\Delta\hat{X}(\vartheta)}{\alpha,\xi}
	=
	\abs{\mu e^{+i\vartheta}-\nu^* e^{-i\vartheta}}
	\label{eq:squeezed_quadrature_std}
\end{equation}
wherein parameters $\nu,\mu$ relate to the complex squeezing parameter $\xi=\abs{\xi}e^{i\varphi_\xi}$ via~\cite[p.~90]{Vogel2006}
\begin{align}
	\mu
	&=
	\cosh\abs{\xi}
	=
	1+\abs{\nu}^2
	&
	\nu
	&=
	e^{i\varphi_\xi}
	\sinh\abs{\xi}
	=
	\abs{\nu}
	e^{i\varphi_\xi}
	\label{eq:squeezing_parameters}
	.
\end{align}
Inserting the polar representation of the squeezing parameters \cref{eq:squeezing_parameters} into the qaudrature standard deviation \cref{eq:squeezed_quadrature_std}, we find that there exists two phase relation between $\vartheta$ and $\varphi_\xi$ which yield minimum and maximum quadrature uncertainties~\cite[p.~96]{Vogel2006}
\begin{equation}
	\expval{\Delta\hat{X}(\vartheta)}{\alpha,\xi}_{\vartheta=\vartheta_{\text{max}/\text{min}}}
	=
	e^{\pm\abs{\xi}}
	.
\end{equation}
In the limit of infinite squeezing magnitude $\abs{\xi}\to\infty$, we obtain position and momentum eigenstates for $\hat{X}(\vartheta_\text{min})$ and $\hat{X}(\vartheta_\text{max})=\hat{X}(\vartheta_\text{min}+\pi/2)$.
\Cref{fig:state_space_squeezed} depicts the optical phase space of two highly squeezed states.
The green ellipsis indicates the variances of a $\hat{P}$-quadrature squeezed state and the orange ellipsis indicated the variance of a $\hat{X}$-quadrature squeezed state.
If Alice prepares such states and Bob attempts to measure them, then his outcome will have almost no correlation if he selects the unsqueezed quadrature as measurement basis.
\begin{figure}[htb]
	\centering
	\includestandalone{figures/pgfplots/phase-space-squeezed}
	\caption{Phase space representation of squeezed-coherent states where the minimum uncertainty is in the $\hat{X}$ quadrature (orange) and the $\hat{P}$ quadrature (green).}\label{fig:phase_space_squeezed}
\end{figure}
To implement the quadrature measurement, we can employ a homodyne detection.
A fiber-optical setup for homodyne detection is depicted in \Cref{fig:coherent_receiver_active}.
At its heart, the homodyne detector consists of a \gls{lo}, a balanced coupler (or beam splitter) and two photodiodes in balanced configuration.
The \gls{lo} is superimposed with the signal through the coupler and the two coupler outputs are monitored by one photodiode.
In balanced configuration, the photocurrent of the photodiodes is subtracted removing the constant power of the signal and \gls{lo}.
\begin{figure}[htb]
	\centering
	\includestandalone{figures/pstricks/coherent-receiver-active}
	\caption{Fiber-optical setup of a single homodyne detector implementing active measurement basis section: A \gls{lo} is synced to the optical carrier and phase-shifted by either $0,\pi/2$ to select one of two orthogonal quadratures. The phase-shifted \gls{lo} is superimposed with the received signal in a balanced coupler. The two coupler outputs are monitored by photodiodes in balanced configuration.}\label{fig:coherent_receiver_active}
\end{figure}
Assuming a perfect detector and strong \gls{lo} with coherent state $\ket{\alpha_l}$ and $\abs{\alpha_l}\gg1$, the mean balanced photodiode current is proportional to~\cite[p.~217]{Vogel2006}
\begin{equation*}
	\expval{\Delta\hat{N}^\prime}
	=
	\expval{\hat{N}_1^\prime}
	-
	\expval{\hat{N}_2^\prime}
	=
	\abs{\alpha_l}
	\expval{\hat{X}(\vartheta)}
\end{equation*}
wherein $\vartheta$ is the phase difference between the signal and the \gls{lo}.
Moreover, it can be shown that the \gls{povm} of an ideal homodyne detector is~\cite[p.~220]{Vogel2006}
\begin{equation}
	\left\{\hat{P}_{\Delta n}=\frac{1}{\abs{\alpha_l}}\ketbra{x(\vartheta)}\right\}_{\Delta n\in\mathbb{N}_0}
\end{equation}
wherein $\abs{x(\varphi)}=\Delta n/\abs{\alpha_l}$.
The single homodyne detector corresponds to an active measurement basis selection of Bob.
As in the case of the polarization-encoding qubit-based \gls{qkd}, Bob can also use a second homodyne detector to implement passive measurement basis selection.
Such a setup is illustrated in \Cref{fig:coherent_receiver_passive}.
\begin{figure}[htb]
	\centering
	\includestandalone{figures/pstricks/coherent-receiver-passive}
	\caption{Fiber-optical setup of a dual homodyne detector implementing passive measurement basis section: A \gls{lo} is synced to the optical carrier and split into two branches. The first branch is used for a first homodyne detector while the second branch is phase-shifted by $\pi/2$ and used for a second homodyne detector. As the two homodyne detectors have orthogonal \gls{lo}, the two orthogonal quadratures can be resolved at the same time.}\label{fig:coherent_receiver_passive}
\end{figure}
The squeezed-coherent-encoding is a more realistic implementation of a boson-based \gls{qkd} protocol as squeezed-coherent states are physically contrary to position and momentum states.
The production of squeezed-coherent states requires nonlinear interactions, which are challenging to control.
Furthermore, squeezed-coherent states quickly lose their squeezing by attenuation.

\FloatBarrier
\subsection{Coherent-encoding}

Coherent-encoding is the most practical encoding for boson-based \gls{qkd}:
It is easier to create and manipulate coherent states using standard telecommunication hardware, and channel loss only deteriorates the amplitude but does not change the kind of quantum state.
The expected quadrature of a coherent state $\ket{\alpha}$ is
\begin{equation}
	\expval{\hat{X}(\vartheta)}{\alpha}
	=
	\frac{1}{\sqrt{2}}
	\left(
		\alpha
		e^{-i\vartheta}
		+
		\alpha^*
		e^{+i\vartheta}
	\right)
	=
	\sqrt{2}
	\Re\left(
		\alpha
		e^{-i\vartheta}
	\right)
	=
	\sqrt{2}
	\abs{\alpha}
	\cos(\phi-\vartheta)
\end{equation}
where we used the complex polar representation $\alpha=\abs{\alpha}e^{i\phi}$ and variance~\cite[p.~59]{Barnett2002}
\begin{equation}
	\expval{\Delta\hat{X}(\vartheta)}{\alpha}
	=
	\frac{1}{\sqrt{2}}
	.
\end{equation}
Consequently, coherent states have equal uncertainty in both quadratures and Bob's measurements will always be off Alice's encoded symbol requiring sophisticated error correction techniques.
Let us use visualize the quantum transmission of coherent-encoding boson-based \gls{qkd} in phase space.
\Cref{fig:phase_space_coherent} depicts the variance of Alice's prepared coherent state (blue) and how it looses amplitude through attenuation of the channel (green).
The variance of Bob's measurement is highlighted by the orange circle which includes electronic noise from the detector.
\begin{figure}[htb]
	\centering
	\includestandalone{figures/pgfplots/phase-space-coherent}
	\caption{Phase space representation of quantum transmission in coherent-encoding boson-based \gls{qkd}: Alice prepares a coherent state with mean and shot noise variance (blue circle). Bob receives the attenuated coherent state with mean and shot noise variance (green circle). Bob's measurement (orange circle) has an increased variance due to other noise sources, e.g., electronic noise.}\label{fig:phase_space_coherent}
\end{figure}
If Eve attempts an intercept-resend attack (\Cref{fig:phase_space_intercept_resend}), she measures Alice's coherent state with some outcome most likely in the blue circle but unlikely to be the exact value Alice used for encoding.
Eve's best guess is to prepare a new state onto which she encodes her measurement outcome (red circle).
The channel attenuation reduces the power of Eve's state.
Including electronic noise, Bob measures an outcome most likely inside the orange circle.
When Alice and Bob perform error correction they will notice a higher than usual error from Eve's attempt to copy Alice's state.
\begin{figure}[htb]
	\centering
	\includestandalone{figures/pgfplots/phase-space-intercept-resend}
	\caption{Phase space representation of quantum transmission in coherent-encoding boson-based \gls{qkd} including intercept-resend attack from Eve: Alice prepares a coherent state with mean and shot noise variance (blue circle). Eve intercepts the state and resends a new state to Bob (red ellipse). Bob receives the Eve's modified coherent state and performs a measurement. When Alice and Bob perform error correction the imperfect copy Eve sent increases the error noticeable.}\label{fig:phase_space_intercept_resend}
\end{figure}
\Cref{fig:phase_space_intercept_resend_squeezed} illustrates Eve's intercept-resent attack if she uses a squeezed state in an attempt to "hide" her error.
\begin{figure}[htb]
	\centering
	\includestandalone{figures/pgfplots/phase-space-intercept-resend-squeezed}
	\caption{Phase space representation of quantum transmission in coherent-encoding boson-based \gls{qkd} including intercept-resend attack from Eve using squeezing: Eve intercepts Alice's coherent state and prepares a squeezed state to reduce the error she introduced. Bobs measurement variance is now comparable to an untampered transmission for one quadrature but the variance of the other quadrature is dramatically increased. If Bob uses passive measurement basis selection, he will instantly notice the high error in the other quadrature. If Bob uses active measurement basis selection, he will too notice a large error introduced by Eve in the asymptotic limit of many measurements.}\label{fig:phase_space_intercept_resend_squeezed}
\end{figure}
If Bob uses a dual homodyne receiver, he will directly notice the increase of noise in one of the quadratures.
If Bob uses a single homodyne receiver, he will only notice every second measurement on average, which is still sufficient to detect Eve's tempering.

To implement coherent-encoding, we can use the same receiver setup as described for squeezed-coherent-encoding.
A possible transmitter setup is presented in \label{fig:coherent_transmitter}.
\begin{figure}[htb]
	\centering
	\includestandalone{figures/pstricks/coherent-transmitter}
	\caption{Fiber-optical setup of a coherent transmitter for coherent-encoding boson-based \gls{qkd}: \gls{iq}-modulation is performed on a transmit laser to encode the complex symbol onto the optical carrier. A \gls{voa} reduces the power of the transmit signal such that the signal strength is comparable to the quantum noise.}\label{fig:coherent_transmitter}
\end{figure}
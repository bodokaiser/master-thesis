\section{Boson- or quadrature-based protocols}

The quantum system of interest in boson-based protocols is a single bosonic mode, i.e., a quantum harmonic oscillator.

The central observable is the generalized quadrature operator~\cite[p.~36]{Barnett2002}
\begin{equation}
	\hat{X}(\vartheta)
	=
	\frac{1}{\sqrt{2}}
	\left(
		\hat{a}
		e^{-i\vartheta}
		+
		\hat{a}^\dagger
		e^{+i\vartheta}
	\right)
\end{equation}
wherein $\hat{a}^\dagger,\hat{a}$ are the bosonic creation and annihilation operators satisfying the canonical commutation relation
\begin{align}
	\comm{\hat{a}}{\hat{a}^\dagger}
	=
	1
	&&
	\comm{\hat{a}}{\hat{a}}
	=
	0
	=
	\comm{\hat{a}^\dagger}{\hat{a}^\dagger}
	.
\end{align}
It follows that the generalized quadrature operator satisfies the commutator
\begin{equation}
	\comm{\hat{X}(\vartheta)}{\hat{X}(\vartheta+\Delta\vartheta)}
	=
	i\sin\Delta\vartheta
	.
\end{equation}
The Robertson uncertainty relation provides a lower bound for the product of the standard deviation of two operators in terms of their commutator.
The Robertson uncertainty relation for the generalized quadrature operator,
\begin{equation}
	\expval{\Delta\hat{X}(\vartheta)}
	\expval{\Delta\hat{X}(\vartheta+\Delta\vartheta)}
	\geq
	\frac{1}{2}
	\abs{\expval{\comm{\hat{X}(\vartheta)}{\hat{X}(\vartheta+\Delta\vartheta)}}}
	=
	\frac{1}{2}
	\sin\Delta\vartheta
	,
\end{equation}
generalizes Heisenberg's uncertainty relation and implies maximal uncertainty for orthogonal quadratures $\Delta\vartheta=\pi/2$.
Let us assume the existence of an eigenstate $\ket{x,\vartheta}$ of the generalized quadrature operator $\Delta\hat{X}(\vartheta)$ with eigenvalue $x\in\mathbb{R}$, i.e.,\footnote{Actually, the quadrature eigenstates only exist on the extended Hilbert space as they itself are not square-integrable.}
\begin{equation}
	\hat{X}(\vartheta)
	\ket{x,\vartheta}
	=
	x\ket{x,\vartheta}
	,
\end{equation}
then the uncertainty relation implies that $\ket{x,\vartheta}$ and $\ket{p,\vartheta+\Delta\vartheta}$ for $\Delta\vartheta>0$ are conjugate variables, i.e., increasing the precision of one variable decreases the precision of the other.
Unsurprisingly, we can show that these eigenstates are non-orthogonal~\cite[p.~29]{Mukhanov2007}
\begin{equation}
	\braket{x,\vartheta}{p,\vartheta+\Delta\vartheta}
	=
	\frac{e^{ipx}}{\sqrt{2\pi}}
	\label{eq:position_momentum_product}
	.
\end{equation}
Furthermore, by using the completeness of eigenstates, we can show that the eigenstates of orthogonal quadratures are related by an unitary transform,
\begin{equation}
	\ket{x,\vartheta}
	=
	\int\frac{\dd{p}}{\sqrt{2\pi}}
	e^{-ipx}
	\ket{p,\vartheta+\pi/2}
	=
	\hat{U}
	\ket{p,\vartheta+\pi/2}
	,
\end{equation}
the Fourier transform.
The non-orthogonality of the quadrature eigenstates, makes the bosonic system a candidate for \gls{qkd}.
For instance, we can envision a bosonic BB84 protocol:
\begin{enumerate}
	\item Alice prepares the state $\ket{\pm x,\vartheta+\Delta\vartheta}$ where she randomly picks the sign of the eigenvalue $\pm x$ and $\Delta\vartheta=0,\pi/2$.
	\item Bob performs a homodyne measurement of one (active) or both (passive) quadratures.
\end{enumerate}
\Cref{fig:phase_space_quadrature} visualizes Alice's four quantum states in the optical phase space.
\begin{figure}[htb]
	\centering
	\includestandalone{figures/pgfplots/phase-space-quadrature}
	\caption{Phase space representation of Alice's quantum states in bosonic BB84: The analog of Alice's basis selection is to choose between the two orthogonal quadratures with $\Delta\vartheta=\pi/2$. The analog of Alice's basis element selection is to choose the sign of the eigenvalue $\pm x$.}\label{fig:phase_space_quadrature}
\end{figure}
If she chooses the quadrature corresponding to $\Delta\vartheta=0$, a measurement in the orthogonal quadrature eigenbasis yields a completely uncorrelated outcome.
Only if Bob measures in the same quadrature, can he decode the sign of the eigenvalue.
\Cref{tab:boson_transmission_sequence} summarizes a possible quantum transmission sequence of bosonic BB84.
\begin{table}[htb]
	\centering
	\begin{tabular}{llccccc}
		\toprule
		& & \multicolumn{5}{c}{Transmission} \\
		\cmidrule{3-7}
		Party & Step & 1 & 2 & 3 & 4 & 5 \\ 
		\midrule
		\multirow{3}{*}{Alice} & Quadrature value & $+x$ & $-x$ & $-x$ & $+x$ & $-x$ \\
		& Quadrature angle & $0$ & $\pi/2$ & $0$ & $\pi/2$ & $0$ \\
		& Prepared state & $\ket{+x,0}$ & $\ket{-x,\pi/2}$ & $\ket{-x,0}$ & $\ket{+x,\pi/2}$ & $\ket{-x,0}$ \\
		\cmidrule{1-2}
		\multirow{2}{*}{Bob} & Quadrature angle & $\pi/2$ & $\pi/2$ & $0$ & $\pi/2$ & $0$ \\
		& Sifted outcome & - & $-x$ & $-x$ & $+x$ & $-x$ \\
		\bottomrule
	\end{tabular}
	\caption{Possible quantum transmission sequence for bosonic BB84 with active basis selection: Alice randomly selects the eigenvalue sign $x/\abs{x}$ and a quadrature $\vartheta\in\{0,\pi/2\}$ which she encodes into a quantum state $\ket{\pm x,\vartheta}$ and sends it to Bob. Bob selects a quadrature for measurement. Only if Alice's and Bob's basis match, is Bob's outcome correlated with Alice's value.}\label{tab:boson_transmission_sequence}
\end{table}
To convert the sifted outcome to bits, we can simply assign the bit value according to the sign.
More advanced symbol mapping techniques are discussed in the post-processing section.

The suggested bosonic BB84 highlights the differences and similarities of boson- and qubit-based \gls{qkd}.
However, it cannot be implemented as no quadrature eigenstates exist - not even theoretically on the Hilbert space!
That said, we can use squeezed coherent states as an approximation for quadrature eigenstates as discussed in the next section.

\FloatBarrier
\subsection{Squeezed-coherent-encoding BB84}

A squeezed coherent state, denoted $\ket{\alpha,\xi}$, has the special property that the quadrature standard deviation is parametrized~\cite[p.~95]{Vogel2006}
\begin{equation}
	\expval{\Delta\hat{X}(\vartheta)}{\alpha,\xi}
	=
	\abs{\mu e^{+i\vartheta}-\nu^* e^{-i\vartheta}}
	\label{eq:squeezed_quadrature_std}
\end{equation}
wherein parameters $\nu,\mu$ relate to the complex squeezing parameter $\xi=\abs{\xi}e^{i\varphi}$ via~\cite[p.~90]{Vogel2006}
\begin{align}
	\mu
	&=
	\cosh\abs{\xi}
	=
	1+\abs{\nu}^2
	&
	\nu
	&=
	e^{i\varphi}
	\sinh\abs{\xi}
	=
	\abs{\nu}
	e^{i\varphi}
	\label{eq:squeezing_parameters}
	.
\end{align}
If the squeezing angle $\varphi$ satisfies a particular phase relation with the quadrature angle $\vartheta$, the quadrature standard deviation takes the form~\cite[p.~96]{Vogel2006}
\begin{equation}
	\expval{\Delta\hat{X}(\vartheta)}{\alpha,\xi_{\text{max}/\text{min}}}
	=
	\exp(\pm\abs{\xi_{\text{max}/\text{min}}})
	.
\end{equation}
In the limit of infinite squeezing magnitude $\abs{\xi_{\text{max}/\text{min}}}\to\infty$, we obtain the previously discussed quadrature eigenstates $\ket{x,\vartheta}$.
Therefore, we can implement bosonic BB84 using strongly squeezed coherent states.
\Cref{fig:phase_space_squeezed} depicts the optical phase space for Alice's strongly squeezed coherent states for bosonic BB84.
Contrary to quadrature eigenstates, the uncertainty in the unsqueezed quadrature is not infinite.
Measurements of the unsqueezed quadrature are not completely uncorrelated.
However, the squeezing magnitude $\abs{\xi}$ can (in theory) be chosen arbitrarily large such that the correlation can be arbitrarily reduced.
\begin{figure}[htb]
	\centering
	\includestandalone{figures/pgfplots/phase-space-squeezed}
	\caption{Phase space representation of Alice's squeezed coherent states in bosonic BB84: The uncertainty is sufficiently squeezed to approximate the quadrature eigenstates.}\label{fig:phase_space_squeezed}
\end{figure}
To implement the quadrature measurement, Bob can employ a homodyne detection.
A fiber-optical setup for homodyne detection is depicted in \Cref{fig:coherent_receiver_active}.
At its heart, the homodyne detector consists of a \gls{lo}, a balanced coupler (or beam splitter) and two photodiodes in balanced configuration.
The \gls{lo} is superimposed with the signal through the coupler and the two coupler outputs are monitored by one photodiode.
In balanced configuration, the photocurrent of the photodiodes is subtracted removing the constant power of the signal and \gls{lo}.
\begin{figure}[htb]
	\centering
	\includestandalone{figures/pstricks/coherent-receiver-active}
	\caption{Fiber-optical setup of a single homodyne detector, attached to an isolated polarization controlled quantum channel, implementing active measurement basis selection: The receiver laser is synced in phase and frequency to the optical carrier and phase-shifted by $\varphi\in\{0,\pi/2\}$. The phase-shifted receiver laser is superimposed with the received signal in a balanced coupler. The two coupler outputs are monitored by photodiodes in balanced configuration.}\label{fig:coherent_receiver_active}
\end{figure}
Assuming a perfect detector and strong \gls{lo} with coherent state $\ket{\alpha_l}$ and $\abs{\alpha_l}\gg1$, the mean balanced photodiode current is proportional to~\cite[p.~217]{Vogel2006}
\begin{equation}
	\expval{\Delta\hat{N}^\prime}
	=
	\expval{\hat{N}_1^\prime}
	-
	\expval{\hat{N}_2^\prime}
	=
	\abs{\alpha_l}
	\expval{\hat{X}(\vartheta)}
\end{equation}
wherein $\vartheta$ is the phase difference between the signal and the \gls{lo}.
Moreover, it can be shown that the \gls{povm} of an ideal homodyne detector is~\cite[p.~220]{Vogel2006}
\begin{equation}
	\left\{\hat{P}_{\Delta n}=\frac{1}{\abs{\alpha_l}}\ketbra{x,\vartheta}\right\}_{\Delta n\in\mathbb{Z}}
\end{equation}
wherein $\ket{x,\vartheta}$ has quadrature eigenvalue $x=\Delta n/\abs{\alpha_l}$.
The single homodyne detector corresponds to an active measurement basis selection of Bob.
As in the case of the polarization-encoding qubit-based \gls{qkd}, Bob can also use a second homodyne detector to implement passive measurement basis selection.
Such a setup is illustrated in \Cref{fig:coherent_receiver_passive}.
\begin{figure}[htb]
	\centering
	\includestandalone{figures/pstricks/coherent-receiver-passive}
	\caption{Fiber-optical setup of a dual homodyne detector, attached to an isolated polarization controlled quantum channel, implementing passive measurement basis selection: The receiver laser is synced in frequency to the optical carrier and split into two branches. The first branch is used for a first homodyne detector while the second branch is phase-shifted by $\pi/2$ and used for a second homodyne detector. Because of the phase-shift, two orthogonal quadratures can be resolved at the same time.}\label{fig:coherent_receiver_passive}
\end{figure}
Contrary to the quadrature eigenstates, squeezed coherent states are physical and have been prepared and measured with up to \SI{15}{\decibel} squeezing~\cite{Vahlbruch2016}.
However, the production of squeezed coherent states requires nonlinear interactions, which are challenging to control and require with the current state of art a complex optical setup.
Additionally, squeezed states quickly loose their squeezing by attenuation.
We conclude that although squeezed coherent states present a possible quantum state for boson-based \gls{qkd}, we prefer a more practical and reliable quantum state.

\FloatBarrier
\subsection{Coherent-encoding GG02}

Coherent states are the "most classical" quantum states and can be prepared using standard telecommunication components, see, for instance, \Cref{fig:coherent_transmitter}.
\begin{figure}[htb]
	\centering
	\includestandalone{figures/pstricks/coherent-transmitter}
	\caption{Fiber-optical setup of a coherent transmitter: \gls{iq}-modulation is performed on a transmit laser to encode the complex symbol onto the optical carrier. A \gls{voa} reduces the power of the transmit signal such that the signal strength is comparable to the quantum noise. \SI{99}{\percent} of the optical power is monitored by a power meter in a control loop with the \gls{voa}. The remaining \SI{1}{\percent} of the optical power are passed through an isolated quantum channel.}\label{fig:coherent_transmitter}
\end{figure}
Moreover, coherent states do not deteriorate to a different quantum state when interacting with the environment, e.g., inside a fiber channel.
In sum, coherent states seem to be the most practical quantum states.
One apparent disadvantage of coherent states is that the quadrature standard deviation is independent of the quadrature angle and the state parameters~\cite[p.~59]{Barnett2002}
\begin{equation}
	\expval{\Delta\hat{X}(\vartheta)}{\alpha}
	=
	\frac{1}{\sqrt{2}}
	.
\end{equation}
We cannot emulate a basis selection by changing the statistics of the quadratures and we need to rethink our approach to \gls{qkd}:
Instead of slowly producing highly correlated variables, we quickly produce hardly correlated variables.
By employing sophisticated error correction techniques, Alice and Bob can still distill a shared bit string, see the post-processing section for details.

\Cref{fig:phase_space_coherent} highlights the quantum transmission of two coherent states in coherent-encoding boson-based \gls{qkd}.
The mean (quadrature) of the coherent state is indicated by the colored marker.
The variance of the coherent state is indicated by the shaded circles around the markers.
The channel attenuates the coherent state by reducing the mean but the variance remains constant.
\begin{figure}[htb]
	\centering
	\includestandalone{figures/pgfplots/phase-space-coherent}
	\caption{Phase space representation of two coherent state transmissions: Alice prepares the coherent states $\ket{\alpha_1}$ (green), $\ket{\alpha_2}$ (blue) with mean quadratures $\alpha_1$ (green dot) respectively $\alpha_2$ (blue dot) and sends them to Bob. Bob receives the attenuated coherent states $\ket{\alpha_1^\prime}$ (orange) and $\ket{\alpha_2^\prime}$ (red). Every coherent state has variance $1/2$ per quadrature indicated by the shaded circles.}\label{fig:phase_space_coherent}
\end{figure}
An intercept-resend attempt by Eve is illustrated in \Cref{fig:phase_space_intercept_resend}.
When Eve intercepts and measures Alice's coherent state $\ket{\alpha}$, her outcome $\beta$ (red star) is distributed around the mean $\alpha$ due to the quadrature uncertainty.
Eve's best guess is to prepare a new (squeezed) coherent state $\ket{\beta}$ ($\ket{\beta,\xi}$) where she prepares the mean $\beta$ to be equal to her measurement outcome $\beta$.
Although, the channel attenuation strongly deteriorates the \gls{snr}, Bob's measurement distribution will be ragged due to Eve's imperfect state copy.
Alice and Bob notice a higher than usual error when performing error correction.
\begin{figure}[htb]
	\centering
	\includestandalone{figures/pgfplots/phase-space-intercept-resend}
	\caption{Phase space representation of coherent state transmission with an intercept-resend attack from Eve using a coherent state (left) and a squeezed coherent state (right). When Eve intercepts and measures Alice's coherent state $\ket{\alpha}$, her outcome (red star) is distributed around the mean $\alpha$ (green circle). }\label{fig:phase_space_intercept_resend}
\end{figure}
If Bob uses a dual homodyne receiver, he will directly notice the increase of noise in one of the quadratures.
If Bob uses a single homodyne receiver, he will only notice every second measurement on average, which is still sufficient to detect Eve's tempering.
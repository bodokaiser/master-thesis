\section{Boson-based protocols}

The quantum system of interest in boson-based protocols is a single bosonic mode, i.e., a quantum harmonic oscillator.
Contrary to a qubit's Hilbert space which has dimension two, the Hilbert space of a single bosonic mode is uncountable.
Consequently, different bases of a bosonic system are related by a Fourier transform, while different bases of a qubit are related by a rotation.
We distinguish between the position $\left\{\ket{x}\right\}_{x\in\mathbb{R}}$ and the momentum eigenbasis $\left\{\ket{p}\right\}_{p\in\mathbb{R}}$.
Position and momentum eigenstates satisfy the eigenvalue equations
\begin{align}
	\hat{X}\ket{x}
	&=
	x\ket{x}
	&
	\hat{P}\ket{p}
	&=
	p\ket{p}
\end{align}
where $\hat{X}$ and $\hat{P}$ are the position and momentum operator with canonical commutation relation
\begin{align}
	\comm{\hat{X}}{\hat{X}}
	&=
	0
	&
	\comm{\hat{X}}{\hat{P}}
	&=
	i
	&
	\comm{\hat{P}}{\hat{P}}
	&=
	0
	.
\end{align}
Using the eigenvalue equations and the commutators, we can show that position and momentum eigenstates
are non-orthogonal~\cite[p.~29]{Mukhanov2007}
\begin{equation}
	\braket{x}{p}
	=
	\frac{e^{ipx}}{\sqrt{2\pi}}
\end{equation}
and therefore perfectly suited for \gls{qkd}.

We can construct a boson-based \gls{qkd} protocol similar to the qubit-based BB84:
\begin{enumerate}
	\item Alice prepares either a position eigenstate $\ket{x_i}$ or a momentum eigenstate $\ket{p_j}$ and sends it to Bob.
	\item Bob performs a measurement either in the position $\left\{\ket{x}\right\}_{x\in\mathbb{R}}$ or the momentum eigenbasis $\left\{\ket{p}\right\}_{p\in\mathbb{R}}$.
\end{enumerate}
If Bob measures in the correct basis, he (theoretically) is able to resolve perfectly $x_i,p_j\in\mathbb{R}$.
Otherwise, Bob measures an outcome completely uncorrelated with what Alice has prepared, see \Cref{tab:boson_transmission_sequence}.
\begin{table}[htb]
	\centering
	\begin{tabular}{llccccc}
		\toprule
		& & \multicolumn{5}{c}{Transmission} \\
		\cmidrule{3-7}
		Party & Step & 1 & 2 & 3 & 4 & 5 \\ 
		\midrule
		\multirow{3}{*}{Alice} & State value & $p_1$ & $x_1$ & $x_2$ & $p_2$ & $x_3$ \\
		& State basis & $P$ & $X$ & $X$ & $P$ & $X$ \\
		& Prepared state & $\ket{p_1}$ & $\ket{x_1}$ & $\ket{x_2}$ & $\ket{p_2}$ & $\ket{x_3}$ \\
		\cmidrule{1-1}
		\multirow{2}{*}{Bob} & Measurement basis & $X$ & $P$ & $X$ & $P$ & $P$ \\
		& Sifted outcome & - & - & $x_2$ & $p_2$ & - \\
		\bottomrule
	\end{tabular}
	\caption{Possible transmission sequence for boson-based BB84: Alice randomly selects a value and a basis, encodes this information into a quantum state and sends it to Bob. Bob randomly selects a measurement outcome. Only if Alice's and Bob's basis match, is Bob's outcome correlated with Alice's value.}\label{tab:boson_transmission_sequence}
\end{table}
While the suggested boson-based BB84 highlights the true differences of boson- with qubit-based QKD it is cannot be implemented as there are no true position or momentum eigenstates.

However, we can use squeezed states as an approximation to the position and momentum eigenstates.
We are going to discuss such a squeeze-encoding in the subsequent section.
For more information on boson information theory, see Ref.~\cite{Weedbrook2012,Ferraro2005}.

\FloatBarrier
\subsection{Squeeze-encoding}

The generalized quadrature operator in the single-mode quantum optics is~\cite[p.~36]{Barnett2002}
\begin{equation}
	\hat{X}(\theta)
	=
	\frac{1}{\sqrt{2}}
	\left(
		\hat{a}
		e^{-i\theta}
		+
		\hat{a}^\dagger
		e^{+i\theta}
	\right)
	.
\end{equation}
For a single-mode coherent state $\ket{\alpha}$, we find mean expectation value
\begin{equation}
	\expval{\hat{X}(\theta)}{\alpha}
	=
	\frac{1}{\sqrt{2}}
	\left(
		\alpha
		e^{-i\theta}
		+
		\alpha^*
		e^{+i\theta}
	\right)
	=
	\sqrt{2}
	\Re\left(
		\alpha
		e^{-i\theta}
	\right)
	=
	\sqrt{2}
	\abs{\alpha}
	\cos(\phi-\theta)
\end{equation}
where we used the complex polar representation $\alpha=\abs{\alpha}e^{i\phi}$ and variance~\cite[p.~59]{Barnett2002}
\begin{equation}
	\expval{\left(\Delta\hat{X}(\theta)\right)^2}{\alpha}
	=
	\frac{1}{2}
	.
\end{equation}

% distinction between tensor- and time-encoding purely on a signal/conceptional level

\FloatBarrier
\subsection{Coherent-encoding}

% citations:
% Gaussian quantum information (quantum channel, cv-qkd, measurements)
% \cite{Lodewyck2007} complete description of CV-QKD device (theoretical and experimental)

%In \gls{cvqkd}, the receiver performs a continuous-value measurement, i.e., the set of possible outcomes is uncountable.
%In comparison to \gls{dvqkd}, \gls{cvqkd} generates random bits faster but with higher error.
%Grosshans initially proposed \gls{cvqkd} in 2002 as an alternative to DV-QKD~\cite{Grosshans2002}.
%Today, CV-QKD has grown into solid competition with DV-QKD~\cite{Diamanti2016}.
%One significant advantage of CV-QKD over DV-QKD is that the CV-QKD receiver uses standard telecommunication components allowing for cheap high-performance integration.


\begin{figure}[htb]
	\centering
	\includestandalone{figures/pgfplots/state-space-quadrature}
	\caption{Phase space representation of coherent states: Alice prepares a coherent state $\ket{\alpha}$ with mean $\alpha$ and shot noise variance (blue circle). Bob receives the attenuated coherent state $\ket{\beta}$ with mean $\beta$ and shot noise variance (green circle). Bob's measurement (orange circle) has an increased variance due to other noise sources, e.g., electronic noise.}
\end{figure}

\begin{figure}[htb]
	\centering
	\includestandalone{figures/pgfplots/state-space-quadrature-cloning}
	\caption{Phase space representation of coherent states: Alice prepares a coherent state $\ket{\alpha}$ with mean $\alpha$ and shot noise variance (blue circle). Bob receives the attenuated coherent state $\ket{\beta}$ with mean $\beta$ and shot noise variance (green circle). Bob's measurement (orange circle) has an increased variance due to other noise sources, e.g., electronic noise.}
\end{figure}

\begin{figure}[htb]
	\centering
	\includestandalone{figures/pgfplots/state-space-quadrature-squeezing}
	\caption{Phase space representation of coherent states: Alice prepares a coherent state $\ket{\alpha}$ with mean $\alpha$ and shot noise variance (blue circle). Bob receives the attenuated coherent state $\ket{\beta}$ with mean $\beta$ and shot noise variance (green circle). Bob's measurement (orange circle) has an increased variance due to other noise sources, e.g., electronic noise.}
\end{figure}

\begin{figure}[htb]
	\centering
	\includestandalone{figures/pstricks/coherent-transmitter}
	\caption{Fiber-optical setup of the phase-encoding \gls{cvqkd} protocol:}
\end{figure}

\begin{figure}[htb]
	\centering
	\includestandalone{figures/pstricks/coherent-receiver-active}
	\caption{Fiber-optical setup of the phase-encoding \gls{cvqkd} protocol:}
\end{figure}

\begin{figure}[htb]
	\centering
	\includestandalone{figures/pstricks/coherent-receiver-passive}
	\caption{Fiber-optical setup of the phase-encoding \gls{cvqkd} protocol:}
\end{figure}

% outlook encoding symbols continuously on a time-continuous coherent state

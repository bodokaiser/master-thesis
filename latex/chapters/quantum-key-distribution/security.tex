\section{Security analysis}

For completeness, we provide a brief introduction to the security analysis of QKD in which we show under which assumptions a \gls{qkd} protocol is secure.
An overview of security proofs, including background information, can be found in Ref.~\cite{Scarani2009}.
For a security analysis of \gls{cvqkd}, see Ref.~\cite{Diamanti2015} and Ref.~\cite{Laudenbach2018}. 
A mathematical treatment using recent information-theoretical tools, see Ref.~\cite{Wolf2021}.

Every \gls{qkd} security proof assumes fundamentally~\cite[p.~10]{Scarani2009}:
\begin{enumerate}
	\item Quantum theory to be complete and correct.
	\item Authenticated communication to be possible.
\end{enumerate}
The first assumption provides us with the framework of quantum (information) theory to formulate our proof.
Furthermore, it states that an adversary is only limited by physical - not technological - means.
The second assumption is vital to exclude man-in-the-middle attacks from an adversary.
It can be practical implemented using \gls{mac}s, for a security proof of Wegman-Carter-Shoup-type authenticators, see Ref.~\cite{Bernstein2005}.
Most security proofs further assume ideal implementation~\cite[p.~124]{Wolf2021}:
\begin{enumerate}
	\item Isolation of the transmitter and receiver from the adversary.
	\item Perfect quantum state preparation and measurement.
	\item True randomness in the state and bases selection.
	\item Perfect timing and synchronization of the transmitter and receiver.
	\item Post-processing protocols are secure and work as intended.
\end{enumerate}
After establishing the security proof of the ideal protocol implementation, we can discuss side-channel attacks originating from imperfect implementations separately.
For example, Ref.~\cite[p.~8]{Lo2014} discusses attacks due to hardware imperfections, Ref.~\cite{Fung2010} analysis the security of a practical post-processing pipeline for BB84, and Ref.~\cite{Renner2005} gives a security proof of privacy amplification in the context of \gls{qkd}.

So far, we have been rather vague about the notion of security.
In particular, we need to parametrize the security of a key as there is no strict security.
For example, consider the security of a binary key of length $n$.
The probability for an adversary to guess the correct key is $\varepsilon=2^{-n}$.
Such a brute-force attack marks the absolute floor of a key's security which we refer to as $\varepsilon$-secure.

More formally, we define a $\epsilon$-secure key obtained by a \gls{qkd} protocol to satisfy~\cite[p.~10]{Scarani2009}
\begin{equation}
	\frac{1}{2}
	\norm{\rho_{AE}-\rho_U\otimes\rho_E}_{\trace{}}
	\leq
	\varepsilon
	\label{eq:qkd_security}
\end{equation}
where $\rho_{AE}$ is the quantum state encoding the correlations between Alice's final key\footnote{It is sufficient to only consider Alice's state as Bob's shares the exact same state after post-processing.} and Eve, $\rho_U$ is the mixed state of possible key configurations, $\rho_E$ is a generic state of Eve, and $\norm{\cdot}_{\trace{}}$ is the trace norm which measures the distance between quantum states, see Ref.~\cite[p.~49]{Wolf2021}.
Intuitively \Cref{eq:qkd_security} encodes the distance between an ideal key state $\rho_U\otimes\rho_E$ and a real key state $\rho_{AE}$.
The real key state $\rho_{AE}$ may be entangled with Eve's system while for the ideal key state, Eve's state $\rho_E$ factorizes as a tensor product with the key state $\rho_U$, i.e., $\rho_E$ and $\rho_U$ describe independent systems.
Further definitions with respect to security, for instance, $\epsilon$-correctness, -robustness, and composability, are formalized in Ref.~\cite[p.~119]{Wolf2021}.
Additionally, one classifies Eve's eavesdropping strategies according to how her ancillas interact with the states Alice sends and whether she performs a local or global measurement of her ancillas, see \Cref{tab:eavesdropping_strategies}.
\begin{table}[htb]
	\centering
	\begin{tabular}{cccc}
		\toprule
			Attack & Alice's & \multicolumn{2}{c}{Eve's} \\
			& State & Unitary & Measurement \\
		\midrule
			Individual & $\rho_A^j$ & Local & Local \\
			Collective & $\rho_A^j$ & Local & Global \\
			Coherent & $\bigotimes_{j=1}^n\rho_A^j$ & Global & Global \\
		\bottomrule
	\end{tabular}
	\caption{Summary of Eve's attacks according to Ref.~\cite[p.~128]{Wolf2021}: Alice sends $n$ quantum states $\rho_A^1,\dots,\rho_A^n$ to Bob. For the individual attack, Eve attaches a single ancilla system to each of Alice's states $\rho_A^j$, applies a local unitary transformation and performs a separate measurement. The collective attack generalizes the individual attack by Eve performing a global measurement of all ancillas. For the coherent attack, Eve interacts with all of Alice's states $\bigotimes_{j=1}^n\rho_A^j$ at once and performs a single global measurement.}\label{tab:eavesdropping_strategies}
\end{table}
Sometimes it is possible to reduce coherent attacks, the most powerful attacks, to collective attacks, see the de Finetti theorem~\cite[p.~148]{Wolf2021}.

One approach for a security proof is to derive an inequality of the form~\cite[p.~11]{Scarani2009}
\begin{equation}
	\mathbb{P}\left[
		\frac{1}{2}
		\norm{\rho_{AE}-\rho_U\otimes\rho_E}_{\trace{}}
		\leq
		\varepsilon
	\right]
	\lesssim
	e^{l-F(\rho_{AE},\varepsilon)}
	\label{eq:qkd_inequality}
\end{equation}
where $l$ is the secret key length and $F$ is the secret key length reference scale.
The secret key length reference scale $F$ depends on the protocol and channel parameters.
For secure key generation, the secret key length must be chosen smaller than the secret key length reference scale, i.e., $l\ll F$.
In another approach, the Devetak-Winter rate~\cite[p.~144]{Wolf2021} is derived.
The Devetak-Winter rate provides a lower bound on the asymptotic secret key rate
\begin{equation}
	r_\text{sec}
	\approxprop
	r_\text{raw}
	(I_{AB}-\chi_{BE})
\end{equation}
wherein $r_\text{raw}$ is the raw transmission rate, $I_{AB}$ is the mutual information between Alice and Bob, and $\chi_{BE}$ is the Holevo information encoding Eve's information on Bob's measurements.
\begin{figure}[htb]
	\centering
	\includestandalone{figures/pgfplots/cvsim}
	\caption{Estimated minimum secret key rate for our \gls{cvqkd} protocol as a function of the channel length for different values of excess noise.}\label{fig:cvsim}
\end{figure}

A systematic approach to security proofs, first pioneered by Lo and Chau~\cite{Lo1999} and later extended by Shor~\cite{Shor2000}, includes the following steps:
\begin{enumerate}
	\item Convert the prepare-and-measure to an entanglement-based description.\footnote{If the protocol is entanglement-based, we can skip this step. For an equivalence proof of the BB84, see Ref.~\cite[p.~106]{Wolf2021}.}
	\item Employ quantum error correction codes to correct Alice's and Bob's qubits which then equals the secret key state.
	\item Show equivalence of classical with quantum post-processing.
\end{enumerate}
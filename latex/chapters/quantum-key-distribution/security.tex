\section{Security analysis}

For completeness, we provide a brief introduction to the security analysis of QKD in which we show under which assumptions a \gls{qkd} protocol is secure.
An overview of security proofs, including background information, can be found in Ref.~\cite{Scarani2009}.
For a security analysis of \gls{cvqkd}, see Ref.~\cite{Diamanti2015} and Ref.~\cite{Laudenbach2018}. 
A mathematical treatment using recent information-theoretical tools, see Ref.~\cite{Wolf2021}.

Every \gls{qkd} security proof assumes fundamentally~\cite[p.~10]{Scarani2009}:
\begin{enumerate}
	\item Quantum theory to be complete and correct.
	\item Authenticated communication to be possible.
\end{enumerate}
The first assumption provides us with the framework of quantum (information) theory to formulate our proof.
Furthermore, it states that an adversary is only limited by physical - not technological - means.
The second assumption is vital to exclude man-in-the-middle attacks from an adversary.
It can be practical implemented using \gls{mac}s, for a security proof of Wegman-Carter-Shoup-type authenticators, see Ref.~\cite{Bernstein2005}.
Most security proofs further assume ideal implementation~\cite[p.~124]{Wolf2021}:
\begin{enumerate}
	\item Isolation of the transmitter and receiver from the adversary.
	\item Perfect quantum state preparation and measurement.
	\item True randomness in the state and bases selection.
	\item Perfect timing and synchronization of the transmitter and receiver.
	\item Post-processing protocols are secure and work as intended.
\end{enumerate}
After establishing the security proof of the ideal protocol implementation, we can discuss side-channel attacks originating from imperfect implementations separately.
For example, Ref.~\cite[p.~8]{Lo2014} discusses attacks due to hardware imperfections, Ref.~\cite{Fung2010} analysis the security of a practical post-processing pipeline for BB84, and Ref.~\cite{Renner2005} gives a security proof of privacy amplification in the context of \gls{qkd}.

So far, we have been rather vague about the notion of security.
In particular, we need to parametrize the security of a key as there is no strict security.
For example, consider the security of a binary key of length $n$.
The probability for an adversary to guess the correct key is $\varepsilon=2^{-n}$.
Such a brute-force attack marks the absolute floor of a key's security which we refer to as $\varepsilon$-secure.

More formally, we define a $\epsilon$-secure key obtained by a \gls{qkd} protocol to satisfy~\cite[p.~10]{Scarani2009}
\begin{equation}
	\frac{1}{2}
	\norm{\rho_{AE}-\rho_U\otimes\rho_E}_{\trace{}}
	\leq
	\varepsilon
	\label{eq:qkd_security}
\end{equation}
where $\rho_{AE}$ is the quantum state encoding the correlations between Alice's final key\footnote{It is sufficient to only consider Alice's state as Bob's shares the exact same state after post-processing.} and Eve, $\rho_U$ is the mixed state of possible key configurations, and $\rho_E$ is a generic state of Eve.
Intuitively \Cref{eq:qkd_security} encodes the distance between an ideal key state $\rho_U\otimes\rho_E$ and a real key state $\rho_{AE}$.
The real key state $\rho_{AE}$ may be entangled with Eve's system while for the ideal key state, Eve's state $\rho_E$ factorizes as a tensor product with the key state $\rho_U$, i.e., $\rho_E$ and $\rho_U$ describe independent systems.
Further definitions with respect to security, for instance, $\epsilon$-correctness, -robustness, and composability, are formalized in Ref.~\cite[p.~119]{Wolf2021}.

The ultimate objective of a security proof is to prove an inequality of the form~\cite[p.~11]{Scarani2009}
\begin{equation}
	\mathbb{P}\left[
		\frac{1}{2}
		\norm{\rho_{AE}-\rho_U\otimes\rho_E}_{\trace{}}
		\leq
		\varepsilon
	\right]
	\lesssim
	e^{l-F(\rho_{AE},\varepsilon)}
\end{equation}
where $l$ is the secret key length and $F$ encodes the information leakage to Eve.
Alternatively, one can derive a lower bound for the secret key rate~\cite{Brunner2017}
\begin{equation}
	r_\text{sec}
	\propto
	r_\text{raw}
	(1-\nu)
	(\beta I_{AB}-\chi_{BE})
\end{equation}
wherein $r_\text{raw}$ is the raw transmission rate, $\nu$ is the fraction of data revealed for parameter estimation, $\beta$ is the error correction efficiency, $I_{AB}$ is the mutual information between Alice and Bob, and $\chi_{BE}$ is the Holevo information encoding Eve's information on Bob's measurements.
% TODO: plots showing the secret key rate with different parameters

\begin{table}[htb]
	\centering
	\begin{tabular}{c}
		\toprule
			Attack \\
		\midrule
			Individual \\
			Collective \\
			Coherent \\
		\bottomrule
	\end{tabular}
	\caption{Summary of Eve's attacks:~\cite[p.~128]{Wolf2021}.}
\end{table}
A systematic approach to security proofs first converts a prepare-and-measure to an entanglement-based protocol.
See, for example, Ref.~\cite[p.~106]{Wolf2021} to show equivalence between prepare-and-measure and entanglement-based BB84.
In the entanglement-based picture, quantum post-processing steps distill key qubit states from the quantum system of Alice and Eve.
The qubit states attributed to Eve's state after quantum post-processing directly estimate Eve's information about the key.
Shor and Peskill first suggested the entanglement distillation-based security proofs in providing a simple proof for BB84~\cite{Shor2000}.
Finally, one shows the equivalence of the quantum post-processing to the classical post-processing.
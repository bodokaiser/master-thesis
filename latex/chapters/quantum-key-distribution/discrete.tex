\section{Discrete-variable (BB84)}

BB84~\cite{Bennett1984} is one of the most popular \gls{dvqkd} protocols and is commonly used as foundation for more practical protocols due to its simple security proof~\cite{Shor2000}.
While the original BB84~\cite{Bennett1984} uses polariation-encoding, many different encodings are possible, see, for instance, \Cref{tab:dvqkd_encodings}.
\begin{table}[htb]
	\centering	
	\begin{tabular}{lcc}
		\toprule
		& \multicolumn{2}{c}{Standard basis} \\
		\cmidrule{2-3}
		Encoding variable & $\ket{0}$ & $\ket{1}$ \\
		\midrule
		Polarization & Horizontal & Vertical \\
		Photon number & Vacuum & Single-photon \\
		Squeezing & Amplitude & Phase \\
		Time-bin & Early & Late \\
		Phase-bin & \SI{0}{\deg} & \SI{180}{\deg} \\
		\bottomrule
	\end{tabular}
	\caption{Encoding variables and their usual assigned basis state for two-state \gls{dvqkd}: The specific choice of the encoding variable does not matter as long as the encoding variable can be described as a two-state quantum (qubit) system.}\label{tab:dvqkd_encodings}
\end{table}
In the following, we discuss the practical time-phase-encoding BB84 protocol and show its equivalence to the polarization-encoding BB84.
The basic setup is illustrated in \Cref{fig:time_phase_encoding_bb84} and comprises a single-photon source and a first \gls{mzi} on Alice's side as well as a second \gls{mzi} and two single-photon detecctors on Bob's side.
\begin{figure}[htb]
	\centering
	\includestandalone{figures/pstricks/phase-encoding-bb84}
	\caption{Fiber-optical setup of the phase-encoding BB84 DV-QKD protocol: Alice creates an entangled photon state using a first \gls{mzi} with phase $\theta=0,\pi/2,\pi,3\pi/2$ and sends it to Bob. Bob detects the photon state using a second \gls{mzi} with phase $\phi=0,\pi$ and two single-photon detectors monitoring the outputs.}\label{fig:time_phase_encoding_bb84}
\end{figure}
To understand the time-phase encoding, we analyze the action of the (asymmetric) \gls{mzi} with variable phase $\varphi$ on a photon pulse $\ket{t_0}$ arriving at time $t_0$, see \Cref{fig:mzi_asym}.
\begin{figure}[htb]
    \centering
    \includegraphics{figures/pstricks/mzi-asym}
     \caption{Asymmetric \gls{mzi} comprising two beam splitters, two mirrors and a phase shifter: An input state enters the first beam splitter BS1 to the left. BS1 splits the state to a longer upper path and a shorter lower path. The lower path}\label{fig:mzi_asym}
\end{figure}
An ideal (lossless) and symmetric beam splitter transforms the single-photon input states into a superposition according to~\cite[p.~134]{Haroche2006}
\begin{align}
	\hat{U}_\text{BS}
	\ket{1,0}
	&=
	\frac{1}{\sqrt{2}}
	\left(\ket{1,0}+i\ket{0,1}\right)
	\\
	\hat{U}_\text{BS}
	\ket{0,1}
	&=
	\frac{1}{\sqrt{2}}
	\left(i\ket{1,0}+\ket{0,1}\right)
	.
\end{align}
Then, the first beam splitter BS1 in \Cref{fig:mzi_asym} (instantly) splits a photon pulse $\ket{t_0}$ arriving at $t_0$ into the superposition
\begin{equation}
	\hat{U}_\text{BS}
	\ket{t_0,0}
	=
	\frac{1}{\sqrt{2}}
	\left(\ket{t_0,0}+i\ket{0,t_0}\right)
\end{equation}
where the first mode corresponds to the upper and the second mode to the lower optical path in \Cref{fig:mzi_asym}.
The phase shifter adds a relative phase of $\varphi$ between the upper and lower path and the input state to the second beam splitter BS2 is
\begin{equation}
	\hat{U}_\text{PS}
	\hat{U}_\text{BS}
	\ket{t_0,0}
	=
	\frac{1}{\sqrt{2}}
	\left(
		e^{i\varphi}
		\ket{t_0+\tau,0}
		+
		i
		\ket{0,t_0+\tau+\Delta\tau}
	\right)
\end{equation}
wherein $\tau$ is the time delay the pulse accumulates over the short upper path and $\Delta\tau$ is the difference in time delay between the shorter, upper and longer, lower path.
The output state of BS2 is equal to the action of the \gls{mzi}
\begin{equation}
	\begin{split}
		\hat{U}_\text{MZM}
		\ket{t_0,0}
		&=
		\hat{U}_\text{BS}
		\hat{U}_\text{PS}
		\hat{U}_\text{BS}
		\ket{t_0,0}
		\\
		&=
		\frac{1}{\sqrt{2}}
		\biggl[
			e^{i\varphi}
			\left(
				\ket{t_0+\tau,0}
				+
				i\ket{0,t_0+\tau}
			\right)
			+
			i
			\left(
				i\ket{t_0+\tau,0}
				+
				\ket{0,t_0+\tau}
			\right)
		\biggr]
	\end{split}
\end{equation}

% dv-qkd (BB84)
% qubit/spin-state system (Pauli eigenbasis)
% equations for phase-encoding QKD
\cite{Bennett1992} % BB92 (also using phase-encoding)


When Alice sends a single-photon $\ket{1}$ through her \gls{mzm} with phase $\theta$, she creates the entangled state\footnote{See Ref.~\cite[p.~137]{Haroche2006} and Ref.~\cite[p.~143]{Gerry2005} for a discussion of the \gls{mzi} with photon number states.}
\begin{equation}
	\ket{\theta}
	=
	\frac{1}{2}
	\left(1+e^{i\theta}\right)
	\ket{0}
	+
	\frac{i}{2}
	\left(1-e^{i\theta}\right)
	\ket{1}
\end{equation}
which we can express in the $Z$ and $Y$ Pauli eigenbasis
\begin{equation}
	\begin{split}
		\ket{\theta}
		=
		\frac{1}{2}
		\left(1+e^{i\theta}\right)
		\ket{z_+}
		+
		\frac{i}{2}
		\left(1-e^{i\theta}\right)
		\ket{z_i}
		=
		\frac{1}{\sqrt{2}}
		\left(
			\ket{y_+}
			+
			e^{i\theta}
			\ket{y_-}
		\right)
		.
	\end{split}
\end{equation}
For the discrete values of the phase $\theta$, Alice's state reduces to the $X,Z$ eigenbasis of the Pauli matrices, i.e.,
\begin{align}
	\ket{z_+}
	=
	\ket{\theta=0}
	&&
	\ket{z_-}
	=
	\ket{\theta=\pi}
	&&
	\ket{x_+}
	=
	\ket{\theta=\frac{\pi}{2}}
	&&
	\ket{x_-}
	=
	\ket{\theta=\frac{3\pi}{2}}
\end{align}
suggesting equivalence to the BB84 protocol on the transmitter side.
On the receiver side, Bob is to measure the state
\begin{equation}
	\ket{\theta,\varphi}
	=
	\frac{1}{2}
	\left(1+e^{i\theta}\right)
	\ket{0,0}
	+
	\frac{i}{4}
	\left(1-e^{i\theta}\right)
	\biggl[
		\left(1+e^{i\varphi}\right)
		\ket{0,1}
		+
		i\left(1-e^{i\varphi}\right)
		\ket{1,0}
	\biggr]
	.
\end{equation}
More precisely, the probabilities that Bob's detectors click are
\begin{align}
	P_1(\theta,\varphi)
	=
	\frac{1}{4}
	\left(1-\cos\theta\right)
	\left(1-\cos\varphi\right)
	&&
	P_2(\theta,\varphi)
	=
	\frac{1}{4}
	\left(1-\cos\theta\right)
	\left(1+\cos\varphi\right)
\end{align}
\begin{table}[htb]
	\centering
	\begin{tabular}{cccc}
		\toprule
		\multicolumn{2}{c}{Phase} & \multicolumn{2}{c}{Detector click probability} \\
		\cmidrule{1-2}
		\cmidrule{3-4}
		$\theta$ & $\varphi$ & $P_1(\theta,\varphi)$ & $P_2(\theta,\varphi)$ \\
		\midrule
		\multirow{2}{*}{$0$} & $0$ & \SI{0}{\percent} & \SI{0}{\percent} \\
		& $\pi$ & \SI{0}{\percent} & \SI{0}{\percent} \\
		\cmidrule{1-2}
		\multirow{2}{*}{$\pi$} & $0$ & \SI{0}{\percent} & \SI{100}{\percent} \\
		& $\pi$ & \SI{100}{\percent} & \SI{0}{\percent} \\
		\cmidrule{1-2}
		\multirow{2}{*}{$\frac{\pi}{2}$} & $0$ & \SI{0}{\percent} & \SI{100}{\percent} \\
		& $\pi$ & \SI{100}{\percent} & \SI{0}{\percent} \\
		\cmidrule{1-2}
		\multirow{2}{*}{$\frac{3\pi}{2}$} & $0$ & \SI{0}{\percent} & \SI{100}{\percent} \\
		& $\pi$ & \SI{100}{\percent} & \SI{0}{\percent} \\
		\bottomrule
	\end{tabular}
	\caption{Foobar}
\end{table}
\begin{table}[htb]
	\centering
	\begin{tabular}{llccccc}
		\toprule
		& & \multicolumn{5}{c}{Transmission} \\
		\cmidrule{3-7}
		Party & Step & 1 & 2 & 3 & 4 & 5 \\ 
		\midrule
		\multirow{3}{*}{Alice} & Initial key bit & \num{0} & \num{1} & \num{1} & \num{0} & \num{0} \\
		& State basis & $Z$ & $X$ & $X$ & $Z$ & $X$ \\
		& Prepared state & $\ket{z_+}$ & $\ket{x_-}$ & $\ket{x_-}$ & $\ket{z_+}$ & $\ket{x_+}$ \\
		\cmidrule{1-1}
		\multirow{3}{*}{Bob} & Measurement basis & $X$ & $Z$ & $X$ & $Z$ & $Z$ \\
		& Possible outcomes & \num{0},\num{1} & \num{0},\num{1} & \num{1} & \num{0} & \num{0},\num{1} \\
		& Sifted key bit & - & - & 1 & 0 & - \\
		\bottomrule
	\end{tabular}
	\caption{Possible prepare-and-measure sequence for BB84: Alice randomly selects an initial key bit \num{0} or \num{1} and a state basis $X$ or $Z$ where $X$ respective $Z$ denote the eigenbasis of the Pauli $\sigma_x$ respective $\sigma_z$ matrix. Alice's initial key bit and selected basis determine the quantum state she prepares and sends to Bob. Bob randomly chooses a measurement basis. Only if Alice's and Bob's basis agree, the key bit is not discarded.}
\end{table}
\begin{figure}[htb]
	\centering
	\includestandalone{figures/tikz/post-processing-dv}
	\caption{Post-processing procedure for \gls{dvqkd} protocols: First Alice's and Bob's key bits are correlated and partially secret. Key sifting discards key bits where Alice and Bob chose a different basis or where Bob did not detect an event while error correction removes the remaining differences between Alice's and Bob's partially secret key bits. Comparing the correlated key bits with the corrected key bits, Alice and Bob can estimate an error and thereby an upper bound on information loss to an adversary. Alice and Bob can remove the remaining partial information an adversary potentially has by performing privacy amplification. Finally, Alice and Bob share a secret key bit string.}
\end{figure}
\cite{Duvsek2006} % overview of DV-QKD protocols
% variants to DV-QKD (six state, decoy states, active/passive)
% active vs passive basis selection on the receiver (compare homodyne/heterodyne measurement with polarization filter vs. beam splitter + polarization filters)

\section{Taxonomy of protocols}

Before diving into specific \gls{qkd} protocols, we introduce a taxonomy of \gls{qkd} protocols.
By identifying common protocol features, allows for a reductionist model of \gls{qkd}.

The literature typically distinguishes between \gls{dvqkd}, \gls{cvqkd}, and \gls{dpsqkd}.
Ignoring \gls{dpsqkd}, it is unclear what exact features unambiguously differentiate between \gls{cvqkd} and \gls{dvqkd}.
For instance, most practical \gls{dvqkd} protocols use weak coherent states~\cite{Duvsek2006}, which are anything but discrete.
Fr this reason, the accepted opinion considers a protocol discrete when using a single-photon and continuous when using a coherent detector.
This view has been recently challenged by proposing a BB84-like protocol using coherent detection~\cite{Qi2021}.
\begin{figure}[htb]
	\centering
	\includestandalone[scale=0.8]{figures/tikz/qkd-classification}
	\caption{Common features among \gls{qkd} protocols: Detection, physical encoding, logical state space, measurement basis selection and schema.}\label{fig:qkd_classification}
\end{figure}
We propose a more subtle classification based on definite protocol features.
\Cref{fig:qkd_classification} illustrates common features we identified among \gls{qkd} protocols that can be uniquely determined.
Many of these features are implementation details such as physical encoding, measurement basis selection, and detection.
In contrast, other features are more fundamental, like the logical Hilbert space or the schema.
We will discuss and exemplify these features in the next two sections.

In the next two sections, we present \gls{qkd} protocols that differ by their logical Hilbert space.
The logical Hilbert space defines the quantum system assumed on an abstract protocol level, e.g., a qubit or boson, and is a more precise concept than discrete and continuous~\cite[p.~2]{Weedbrook2012}}.
The logical quantum system does not need to be equal to the quantum system physically encoding the quantum states, for example, a weak coherent state mimicking a single-photon.
\section{Taxonomy of protocols}

\subsection{Common features among the protocol zoo}

\Cref{fig:qkd_classification} illustrates common features among the \gls{qkd} protocols.
For an overview of \gls{qkd} protocols, see Ref.~\cite{Duvsek2006}.
\begin{figure}[htb]
	\centering
	\includestandalone{figures/tikz/qkd-classification}
	\caption{Common features among \gls{qkd} protocols: Detection, physical encoding, logical state space, measurement basis selection and schema.}\label{fig:qkd_classification}
\end{figure}
Every \gls{qkd} system requires a detector, e.g., a coherent detector or a single-photon (click) detector.
The detection does not necessarily imply the dimension of the logical state space.
For instance, BB84 having a two-dimensional logical state space has been implemented with coherent detection~\cite{Qi2021} having an uncountable detection space.
It remains to discuss where to place differential-phase-shift \gls{dps} \gls{qkd} into the proposed categories.
Concerning measurement basis selection, Bob can either actively choose a random measurement basis for every transmission or passively measure all (orthogonal) bases for measurement basis selection.
We will cover both active and passive measurement basis selection in the discussion of the polarization-encoding BB84 protocol.
Finally, the \gls{qkd} schema determines if either Alice prepares a state and sends it to Bob for measurement (prepare-and-measure) or if Alice and Bob share an entangled state (entanglement-based).
Most practical \gls{qkd} implementations use prepare-and-measure.
On a theoretical level, both schemas are equivalent, and security proofs are often more convenient in an entanglement-based setting.

\subsection{Abstract structure of a protocol}

\Cref{fig:qkd_protocol} illustrates our proposed notion of a \gls{qkd} protocol, with the feature being a logical quantum system from which the random bits are encoded and decoded.
The logical quantum system is a subspace of the physical quantum system.
The physical quantum system depends strongly on the physical implementation and quantum encoding.
\begin{figure}[htb]
	\centering
	\includestandalone{figures/tikz/qkd-protocol}
	\caption{A \gls{qkd} protocol comprises a binary encoder, a logical quantum system, and a binary decoder. The binary encoder maps bits $\vb{b}\in\{0,1\}^n$ onto a quantum state of the logical quantum system $\ket{\psi}$. The binary decoder extracts the bits $\vb{b}$ back from the quantum state $\ket{\psi^\prime}$. The logical quantum system is a subspace of a larger physical quantum system. The state encoder and decoder map between the logical and physical quantum states.}\label{fig:qkd_protocol}
\end{figure}
The distinction between logical and quantum systems is vital to separate the implementation and security concerns.
Many security proofs show equivalence between the physical implementation and the logical system to use an established security proof.
One should keep in mind that such a separation implicitly assumes no loopholes from the particular implementation.

\subsection{Comparison of qubit- and boson-based protocols}

% TODO: critic about CV-, DV-QKD, DPS-QKD

\begin{table}[htb]
	\centering	
	\begin{tabular}{lcc}
		\toprule
			& Qubit (spin) & Boson (quadrature) \\
		\midrule
			Visualization & Bloch sphere & Phase space \\
			Hilbert space (dim) & Finite (two) & Uncountable (infinite) \\
			Description & Density matrix & Wigner distribution \\
			Observable & $\vb{\hat{S}}(\vb{n})=\hat{S}_in^i$ & $\hat{X}(\vartheta)=\frac{1}{\sqrt{2}}\left(\hat{a}e^{-i\vartheta}+\hat{a}^\dagger e^{+i\vartheta}\right)$ \\
			Standard basis & $\left\{\ket{0},\ket{1}\right\}$ & $\left\{\ket{x}\right\}_{x\in\mathbb{R}}$ or $\left\{\ket{p}\right\}_{p\in\mathbb{R}}$ \\
			Basis transformation & Rotation & Unitary \\
			Other relevant states & $\ket{x_\pm},\ket{y_\pm},\ket{z_\pm}$ & $\ket{n},\ket{\alpha},\ket{\alpha,\xi}$ \\
		\bottomrule
	\end{tabular}
	\caption{Possible physical systems to encode a qubit. Many of the physical systems have a much higher dimension than the qubit space but can pre reduced to a qubit space by a proper mapping.}
\end{table}
\section{Conjugate-based protocols}

% relevance of Gaussian RV to CV-QKD?

% citations:
% \cite{Weedbrook2012,Ferraro2005} Gaussian quantum information (quantum channel, cv-qkd, measurements)
% \cite{Lodewyck2007} complete description of CV-QKD device (theoretical and experimental)

% measurement of a Gaussian variable (1d)
% protocol table
% post-processing

%In \gls{cvqkd}, the receiver performs a continuous-value measurement, i.e., the set of possible outcomes is uncountable.
%In comparison to \gls{dvqkd}, \gls{cvqkd} generates random bits faster but with higher error.
%Grosshans initially proposed \gls{cvqkd} in 2002 as an alternative to DV-QKD~\cite{Grosshans2002}.
%Today, CV-QKD has grown into solid competition with DV-QKD~\cite{Diamanti2016}.
%One significant advantage of CV-QKD over DV-QKD is that the CV-QKD receiver uses standard telecommunication components allowing for cheap high-performance integration.

The generalized quadrature operator in the single-mode quantum optics is~\cite[p.~36]{Barnett2002}
\begin{equation}
	\hat{X}(\theta)
	=
	\frac{1}{\sqrt{2}}
	\left(
		\hat{a}
		e^{-i\theta}
		+
		\hat{a}^\dagger
		e^{+i\theta}
	\right)
	.
\end{equation}
For a single-mode coherent state $\ket{\alpha}$, we find mean expectation value
\begin{equation}
	\expval{\hat{X}(\theta)}{\alpha}
	=
	\frac{1}{\sqrt{2}}
	\left(
		\alpha
		e^{-i\theta}
		+
		\alpha^*
		e^{+i\theta}
	\right)
	=
	\sqrt{2}
	\Re\left(
		\alpha
		e^{-i\theta}
	\right)
	=
	\sqrt{2}
	\abs{\alpha}
	\cos(\phi-\theta)
\end{equation}
where we used the complex polar representation $\alpha=\abs{\alpha}e^{i\phi}$ and variance~\cite[p.~59]{Barnett2002}
\begin{equation}
	\expval{\left(\Delta\hat{X}(\theta)\right)^2}{\alpha}
	=
	\frac{1}{2}
	.
\end{equation}

\subsection{Squeeze-encoding}

\begin{figure}[htb]
	\centering
	\includestandalone{figures/pgfplots/state-space-conjugate}
	\caption{Phase space representation of coherent states: Alice prepares a coherent state $\ket{\alpha}$ with mean $\alpha$ and shot noise variance (blue circle). Bob receives the attenuated coherent state $\ket{\beta}$ with mean $\beta$ and shot noise variance (green circle). Bob's measurement (orange circle) has an increased variance due to other noise sources, e.g., electronic noise.}
\end{figure}

\begin{figure}[htb]
	\centering
	\includestandalone{figures/pgfplots/state-space-conjugate-cloning}
	\caption{Phase space representation of coherent states: Alice prepares a coherent state $\ket{\alpha}$ with mean $\alpha$ and shot noise variance (blue circle). Bob receives the attenuated coherent state $\ket{\beta}$ with mean $\beta$ and shot noise variance (green circle). Bob's measurement (orange circle) has an increased variance due to other noise sources, e.g., electronic noise.}
\end{figure}

\begin{figure}[htb]
	\centering
	\includestandalone{figures/pgfplots/state-space-conjugate-squeezing}
	\caption{Phase space representation of coherent states: Alice prepares a coherent state $\ket{\alpha}$ with mean $\alpha$ and shot noise variance (blue circle). Bob receives the attenuated coherent state $\ket{\beta}$ with mean $\beta$ and shot noise variance (green circle). Bob's measurement (orange circle) has an increased variance due to other noise sources, e.g., electronic noise.}
\end{figure}

% distinction between tensor- and time-encoding purely on a signal/conceptional level

\subsection{Coherent-encoding}

\begin{figure}[htb]
	\centering
	\includestandalone{figures/pstricks/coherent-transmitter}
	\caption{Fiber-optical setup of the phase-encoding \gls{cvqkd} protocol:}
\end{figure}

\begin{figure}[htb]
	\centering
	\includestandalone{figures/pstricks/coherent-receiver-active}
	\caption{Fiber-optical setup of the phase-encoding \gls{cvqkd} protocol:}
\end{figure}

\begin{figure}[htb]
	\centering
	\includestandalone{figures/pstricks/coherent-receiver-passive}
	\caption{Fiber-optical setup of the phase-encoding \gls{cvqkd} protocol:}
\end{figure}

% outlook encoding symbols continuously on a time-continuous coherent state

\section{Post-processing}

\begin{enumerate}
	\item bit mapping (encoding,decoding)
	\item error correction and estimation
	\item privacy amplification
\end{enumerate}


% aim of classical post-processing (correlated variables -> shared secret, estimate error -> protocol abortion)}

% what about (base) sifting?

\begin{figure}[htb]
	\centering
	\includestandalone{figures/tikz/post-processing}
	\caption{\Gls{qkd} transmission system from a signal-processing perspective.}
\end{figure}

% citations
\cite{Silberhorn2002} % post-selection mechanism to mitigate beam splitter attack
\cite{Fung2010} % security analysis and overview of post-processing

%\subsection{Reconciliation}
% reconciliation
\cite{Leverrier2008} % multidimensional (sphere) 
\cite{Elkouss2011} % simpler reconciliation scheme

\subsection{Information reconciliation}

Information reconciliation summarizes methods required for Alice and Bob to agree on shared data.
It includes error correction, and discarding of data failed to correct.

Let us first consider procedures for error correction.
Error correction is a subdiscipline of coding theory, or more precisely, channel coding, which studies the arrangement of data for efficient and reliable transmission, see \Cref{fig:error_correction_codes}.
\begin{figure}[htb]
	\centering
	\includestandalone{figures/tikz/error-correction-codes}
	\caption{Taxonomy of codes in coding theory with emphasis on linear block codes for error correction.}\label{fig:error_correction_codes}
\end{figure}

\subsection{Privacy amplification}

% XOR-ing using Toeplitz matrices

\cite{Bennett1995} % Generalized privacy amplfication
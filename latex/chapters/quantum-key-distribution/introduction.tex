\textit{The following chapter presents the fundamental idea of quantum-key distribution and its many facets, usually overlooked in the introductory material. First, we dedicate our attention to \gls{dvqkd} as it is more familiar than its counterpart \gls{cvqkd} which we discuss in the second part. For \gls{dvqkd}, we focus on the most common BB84 protocol, which we first discuss on an abstract protocol level and then practical implementations, particularly the polarization and time-phase encoding. We hope to convey to the reader the difference between the protocol and the encoding, which is not apparent when considering the basic polarization-encoding BB84. However, the difference between protocol and encoding will become central in the later stages of the CV-QKD thesis.}

core principle: non-orthogonality of quantum states?

% internet layer for QKD protocols: bit encoder/decoder, logical quantum system, physical quantum system

%  -> EB for security proof, PM for implementation (cite equivalence)

\begin{enumerate}
	\item entanglement-based (EB) vs prepare-and-measure (PM)
	\item continuous- vs. discrete (definition)
	\item active vs passive measurement
\end{enumerate}

\begin{figure}[htb]
	\centering
	\includestandalone{figures/tikz/post-processing-dv}
	\caption{Post-processing procedure for \gls{dvqkd} protocols: First Alice's and Bob's key bits are correlated and partially secret. Key sifting discards key bits where Alice and Bob chose a different basis or where Bob did not detect an event while error correction removes the remaining differences between Alice's and Bob's partially secret key bits. Comparing the correlated key bits with the corrected key bits, Alice and Bob can estimate an error and thereby an upper bound on information loss to an adversary. Alice and Bob can remove the remaining partial information an adversary potentially has by performing privacy amplification. Finally, Alice and Bob share a secret key bit string.}\label{fig:post_processing}
\end{figure}
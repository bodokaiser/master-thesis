The present chapter attempts to formalize a \gls{qkd} protocol by classifying the rich diversity of \gls{qkd} protocols and identifying common characteristics.
We apply our results by presenting the popular BB84 and Gaussian \gls{cvqkd} protocols.

\Cref{fig:qkd_classification} presents an attempt to characterize \gls{qkd} protocols among three properties:
the type of random variable, the schema, and the measurement basis selection.
\begin{figure}[htb]
	\centering
	\includestandalone{figures/tikz/qkd-classification}
	\caption{An attempt to classify \gls{qkd} protocol: The random variable generating the random numbers can be either discrete or continuous. The schema can be either prepare-and-measure or entanglement-based. The basis selection for the basis can be either active or passive.}\label{fig:qkd_classification}
\end{figure}
The type of the random variable used for random number generation can be either continuous or discrete, i.e., the random number generator's domain is either an uncountable or countable set.
It is ambiguous to discriminate among \gls{dvqkd} and \gls{cvqkd} protocols based on the detector, i.e., click or coherent, as recently demonstrated by Qi~\cite{Qi2021}.
Concerning measurement basis selection, Bob can either actively choose a random measurement basis for every transmission or passively measure all (orthogonal) bases for measurement basis selection.
We will cover both active and passive measurement basis selection in the discussion of the polarization-encoding BB84 protocol.
Finally, the \gls{qkd} schema determines if either Alice prepares a state and sends it to Bob for measurement (prepare-and-measure) or if Alice and Bob share an entangled state (entanglement-based).
Most practical \gls{qkd} implementations use prepare-and-measure.
On a theoretical level, both schemas are equivalent, and security proofs are often more convenient in an entanglement-based setting.

\begin{figure}[htb]
	\centering
	\includestandalone{figures/tikz/qkd-protocol}
	\caption{A \gls{qkd} protocol comprises a binary encoder, a logical quantum system, and a binary decoder. The binary encoder maps bits $\vb{b}\in\{0,1\}^n$ onto a quantum state of the logical quantum system $\ket{\psi}$. The binary decoder extracts the bits $\vb{b}$ back from the quantum state $\ket{\psi^\prime}$. The logical quantum system is a subspace of a larger physical quantum system. The state encoder and decoder map between the logical and physical quantum states.}
\end{figure}


The present chapter identifies common characteristics of \gls{qkd} protocols and attempts to formalize the notion of a \gls{qkd} protocol.
We test the proposed formula with qubit-based \gls{qkd} protocols like BB84 and the six-state protocol as well as Gaussian \gls{cvqkd} protocols.
The chapter ends with a summary and literature review regarding practical post-processing and security analysis of \gls{qkd} protocols.

For more information on boson information theory, see Ref.~\cite{Weedbrook2012,Ferraro2005}.
% Gaussian quantum information (quantum channel, cv-qkd, measurements)
% \cite{Lodewyck2007} complete description of CV-QKD device (theoretical and experimental)

%In \gls{cvqkd}, the receiver performs a continuous-value measurement, i.e., the set of possible outcomes is uncountable.
%In comparison to \gls{dvqkd}, \gls{cvqkd} generates random bits faster but with higher error.
%Grosshans initially proposed \gls{cvqkd} in 2002 as an alternative to DV-QKD~\cite{Grosshans2002}.
%Today, CV-QKD has grown into solid competition with DV-QKD~\cite{Diamanti2016}.
%One significant advantage of CV-QKD over DV-QKD is that the CV-QKD receiver uses standard telecommunication components allowing for cheap high-performance integration.


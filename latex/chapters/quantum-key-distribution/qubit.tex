\section{Qubit- or spin-based protocols}

Many \gls{dvqkd} protocols, e.g., the BB84~\cite{Bennett1984} or six-state protocol~\cite{Bechmann1999}, are qubit-based in that the logical quantum system underlying the key generation is a two-state quantum system.

A qubit state $\ket{\psi}$ is an element of a two-dimensional complex Hilbert space with norm one, i.e., $\abs{\braket{\psi}}^2=1$.
In the qubit basis $\{\ket{0},\ket{1}\}$, a generic qubit state takes the form
\begin{align}
	\ket{\psi}
	=
	c_1\ket{0}
	+
	c_2\ket{1}
	&&
	\text{with}\
	\abs{c_1}^2
	+
	\abs{c_2}^2
	=
	1
	.
\end{align}
The state space equals the surface of a three-dimensional unit sphere.
A useful visualization of qubit states is the Bloch sphere, see \Cref{fig:bloch_sphere}.
\begin{figure}[htb]
	\centering
	\includestandalone{figures/tikz/bloch-sphere}
	\caption{Two-state quantum system in the Bloch sphere representation: A (pure) quantum state  $\ket{\psi}=c_1\ket{0}+c_2\ket{1}$ with $\abs{c_1}^2+\abs{c_2}^2=1$ takes a point on the surface on a unit radius sphere. By convention, the standard basis $\left\{\ket{0},\ket{1}\right\}$ is set to equal the $Z$ Pauli eigenbasis $\left\{\ket{z_+},\ket{z_-}\right\}$. The $Y,Z$ Pauli eigenbasis are both orthogonal to the $Z$ basis.}\label{fig:bloch_sphere}
\end{figure}
The Bloch sphere is embedded in three-dimensional space, and there are three independent bases.
Two antiparallel vectors on the Bloch sphere denote two orthonormal basis elements.
The observable operator central to the qubit is the generalized spin operator
\begin{equation}
	\hat{S}(\vu{n})
	=
	\vb{\hat{S}}\vdot\vu{n}
	=
	\frac{1}{2}
	\hat{\sigma}_jn^j
	=
	\begin{pmatrix}
		n_3 & n_1 - in_2 \\
		n_1 + in_2 & -n+3
	\end{pmatrix}
\end{equation}
wherein $\vu{n}\in\mathbb{R}^3$ is a unit norm vector and $\hat{\sigma}_j$ is the $j$th Pauli matrix.
Let $\ket{\pm,\vu{n}}$ be the eigenstate of the generalized spin operator $\hat{S}(\vu{n})$ to eigenvalues $\pm1/2$, i.e.,
\begin{equation}
	\hat{S}(\vu{n})
	\ket{\pm,\vu{n}}
	=
	\pm\frac{1}{2}
	\ket{\pm,\vu{n}},
\end{equation}
then for $\vu{n}=\vu{e}_j$ we obtain the $\hat{\sigma}_j$ Pauli eigenstates, i.e.,
\begin{align}
	\hat{S}_x
	\ket{x_\pm}
	=	
	\vb{\hat{S}}(\vu{e}_x)
	\ket{\pm,\vu{e}_x}
	&=
	\pm\frac{1}{2}
	\ket{\pm,\vu{e}_x}
	=
	\pm\frac{1}{2}
	\ket{x_\pm}
	\\
	\hat{S}_y
	\ket{y_\pm}
	=	
	\vb{\hat{S}}(\vu{e}_y)
	\ket{\pm,\vu{e}_y}
	&=
	\pm\frac{1}{2}
	\ket{\pm,\vu{e}_y}
	=
	\pm\frac{1}{2}
	\ket{y_\pm}
	\\
	\hat{S}_z
	\ket{z_\pm}
	=	
	\vb{\hat{S}}(\vu{e}_z)
	\ket{\pm,\vu{e}_z}
	&=
	\pm\frac{1}{2}
	\ket{\pm,\vu{e}_z}
	=
	\pm\frac{1}{2}
	\ket{z_\pm}
	.
\end{align}
By convention one identifies the $Z$ Pauli eigenbasis with the qubit basis $\left\{\ket{0},\ket{1}\right\}$.
\Cref{tab:qubit_encodings} lists different physical quantum systems  which allow encoding of a qubit.
\begin{table}[htb]
	\centering	
	\begin{tabular}{lcc}
		\toprule
		& \multicolumn{2}{c}{Standard basis} \\
		\cmidrule{2-3}
		Encoding variable & $\ket{0}$ & $\ket{1}$ \\
		\midrule
		Polarization & Horizontal & Vertical \\
		Photon number & Vacuum & Single-photon \\
		Time-bin & Early & Late \\
		Phase-bin & \SI{0}{\deg} & \SI{180}{\deg} \\
		\bottomrule
	\end{tabular}
	\caption{Possible physical systems to encode a qubit. Many of the physical systems have a much higher dimension than the qubit space but can be reduced to a qubit space.}\label{tab:qubit_encodings}
\end{table}
\begin{figure}[htb]
	\centering
	\includestandalone{figures/pgfplots/state-space-qubit}
	\caption{Two-dimensional state space spanned by the $X,Z$ Pauli eigenbases: Projecting the $\ket{x_-}$ state onto the $Z$ eigenbasis yields a constant probability amplitude of $1/\sqrt{2}$.}
\end{figure}

For the BB84 (six state) protocol, Alice and Bob must agree on two (or three) orthogonal bases\footnote{BB92 using non-orthogonal bases can be implemented by using the generalized spin operator with non-orthogonal vectors.} and a mapping between the basis states and some bit sequence, then
\begin{enumerate}
	\item Alice encodes her bits into the state $\ket{\psi}$ and sends it to Bob.
	\item Bob receives the state $\ket{\psi}$ from Alice and performs a measurement decoding some bits.
\end{enumerate}
If Alice and Bob select the same basis, Bob can accurately decode Alice's key bit from the measurement.
Alice and Bob's probability of choosing the same basis for one transmission is one divided by the number of orthogonal bases Alice and Bob have agreed on, e.g., \SI{50}{\percent} if Alice and Bob agreed to use the $X$ and $Z$ Pauli eigenbasis, also called the \gls{qber}.
In the asymptotic limit of many transmissions, the \gls{qber} should approach the theoretical limit.
Otherwise, an opposing third party, Eve, might have tempered with the transmission.
\Cref{tab:qubit_transmission_sequence} displays a possible transmission sequence between Alice and Bob.
\begin{table}[htb]
	\centering
	\begin{tabular}{llccccc}
		\toprule
		& & \multicolumn{5}{c}{Transmission} \\
		\cmidrule{3-7}
		Party & Step & 1 & 2 & 3 & 4 & 5 \\ 
		\midrule
		\multirow{3}{*}{Alice} & Initial key bit & \num{0} & \num{1} & \num{1} & \num{0} & \num{0} \\
		& State basis & $Z$ & $X$ & $X$ & $Z$ & $X$ \\
		& Prepared state & $\ket{z_+}$ & $\ket{x_-}$ & $\ket{x_-}$ & $\ket{z_+}$ & $\ket{x_+}$ \\
		\cmidrule{1-1}
		\multirow{3}{*}{Bob} & Measurement basis & $X$ & $Z$ & $X$ & $Z$ & $Z$ \\
		& Possible outcomes & \num{0},\num{1} & \num{0},\num{1} & \num{1} & \num{0} & \num{0},\num{1} \\
		& Sifted outcomes & - & - & 1 & 0 & - \\
		\bottomrule
	\end{tabular}
	\caption{Possible transmission sequence for qubit-based \gls{qkd}: Alice randomly selects an initial key bit \num{0} or \num{1} and a state basis $X$ or $Z$ where $X$ respective $Z$ denote the eigenbasis of the Pauli $\sigma_x$ respective $\sigma_z$ matrix. Alice's initial key bit and selected basis determine the quantum state she prepares and sends to Bob. Bob randomly chooses a measurement basis. Only if Alice's and Bob's basis agree, the key bit is not discarded.}\label{tab:qubit_transmission_sequence}
\end{table}
After the transmission sequence, Alice and Bob hold a partially correlated and partially secret bit string from which they can distill a shared secret bit string using classical post-processing.

\FloatBarrier
\subsection{Polarization-encoding}

In polarization-encoding qubit-based \gls{qkd} protocols, the polarization of light is used as physical quantum system to encode the logical qubit system.
Let $\ket{\leftrightarrow}$ and $\ket{\updownarrow}$ denote the horizontal respective vertical polarization states forming the rectilinear basis.
Let $\ket{\nwsearrow}$ and $\ket{\neswarrow}$ denote the left- and right-diagonal polarization states forming the diagonal basis.
Let $\ket{\circlearrowleft}$ and $\ket{\circlearrowright}$ denote the left- and right-circular polarization states forming the circular basis.
We can express the diagonal and circular basis elements in terms of the rectilinear basis elements:
\begin{align}
	\ket{\nwsearrow}
	&=
	\frac{1}{\sqrt{2}}
	\left(
		\ket{\leftrightarrow}
		+
		\ket{\updownarrow}
	\right)
	&
	\ket{\neswarrow}
	&=
	\frac{1}{\sqrt{2}}
	\left(
		\ket{\leftrightarrow}
		-
		\ket{\updownarrow}
	\right)
	\\
	\ket{\circlearrowleft}
	&=
	\frac{1}{\sqrt{2}}
	\left(
		\ket{\leftrightarrow}
		+
		i\ket{\updownarrow}
	\right)
	&
	\ket{\circlearrowright}
	&=
	\frac{1}{\sqrt{2}}
	\left(
		\ket{\leftrightarrow}
		-
		i\ket{\updownarrow}
	\right)
\end{align}
For clarity, we restrict the following discussion to qubit-based \gls{qkd} protocols where two orthogonal bases are used, e.g., rectilinear and diagonal.
Other protocols exist that use three orthogonal bases (six-state protocol) or even non-orthogonal bases.

A possible optical setup to implement such polarization-encoding is depicted in \Cref{fig:polarization_encoding_active}.
\begin{figure}[htb]
	\centering
	\includestandalone{figures/pstricks/bb84-polarization-active}
	\caption{Optical setup to implement polarization-encoding BB84 with active basis selection at the receiver: Alice configures her linear polarizer to select a basis element of the rectilinear or diagonal polarization basis. Bob receives Alice's polarization state through the polarization controlled quantum channel. He rotates a rectilinear polarized beam splitter by either \SI{0}{\degree} or \SI{45}{\degree} to detect either rectilinear or diagonal polarized photons with his two single-photon detectors placed at the beam splitter output.}\label{fig:polarization_encoding_active}
\end{figure}
Alice selects one of four polarization states $\ket{\leftrightarrow},\ket{\updownarrow},\ket{\nwsearrow},\ket{\neswarrow}$ by adjusting her linear polarizer to one of four angles $\theta=0,\pi,\pi/2,3\pi/2$.
We can write Alice's state as
\begin{equation}
	\ket{\theta}
	=
	\frac{1}{\sqrt{2}}
	\left(
		\ket{\circlearrowleft}
		+
		e^{i\theta}
		\ket{\circlearrowright}
	\right)
	.
\end{equation}
Unrotated, Bob's rectilinear polarized beam splitter monitored by two single-photon detectors is equivalent to the \gls{povm} for detecting rectilinear-polarized light
\begin{equation}
	\biggl\{
		\hat{P}_{\leftrightarrow}
		=
		\ketbra{\leftrightarrow},
		\hat{P}_{\updownarrow}
		=
		\ketbra{\updownarrow}
	\biggr\}
	.
\end{equation}
Rotated by \SI{45}{\degree}, Bob's rectilinear polarized beam splitter monitored by two single-photon detectors is equivalent to the \gls{povm} for detecting diagonal-polarized light
\begin{equation}
	\biggl\{
		\hat{P}_{\nwsearrow}
		=
		\ketbra{\nwsearrow},
		\hat{P}_{\neswarrow}
		=
		\ketbra{\neswarrow}
	\biggr\}
	.
\end{equation}
Instead of Bob actively selecting the measurement basis, he can passively let the quantum randomness decide by splitting the photon with an unpolarized beam splitter towards a rectilinear and diagonal polarization detector.
\Cref{fig:polarization_encoding_passive} shows an optical setup implementing polarization-encoding BB84 with passive measurement basis selection.
While Alice's transmitter setup is unchanged, Bob has two polarization detectors.
One polarization detector comprises a rectilinear-polarized beam splitter and two single-photon detectors.
Another polarization detector comprises a diagonal-polarized beam splitter.
\begin{figure}[htb]
	\centering
	\includestandalone{figures/pstricks/bb84-polarization-passive}
	\caption{Optical setup to implement polarization-encoding BB84 with passive basis selection at the receiver: Alice configures her linear polarizer to select a basis element of the rectilinear or diagonal polarization. Bob receives Alice's polarization state through the polarization controlled quantum channel. He first splits the incoming light into two branches. For the first branch a polarization measurement in the rectilinear basis is performed. For the second branch a polarization measurement in the diagonal basis is performed.}\label{fig:polarization_encoding_passive}
\end{figure}
The \gls{povm} describing Bob's measurement with passive basis selection is
\begin{equation}
	\biggl\{
		\frac{1}{2}\hat{P}_{\leftrightarrow},
		\frac{1}{2}\hat{P}_{\updownarrow},
		\frac{1}{2}\hat{P}_{\nwsearrow},
		\frac{1}{2}\hat{P}_{\neswarrow}
	\biggr\}
	.
\end{equation}
Bob may still have inconclusive measurements.
For instance, if he receives a horizontal polarization state $\ket{\leftrightarrow}$ and the photon chooses the path towards the diagonal polarization detector, the clicks among the two single-photon detectors are equally distributed.

The polarization of light is a qubit and we can simply relabel the polarization states with the Pauli eigenstates, i.e.,
\begin{align}
	\ket{\nwsearrow}
	=
	\ket{x_+},
	&
	\ket{\neswarrow}
	=
	\ket{x_-},
	&
	\ket{\circlearrowleft}
	=
	\ket{y_+},
	&
	\ket{\circlearrowright}
	=
	\ket{y_-},
	&
	\ket{\leftrightarrow}
	=
	\ket{z_+},
	&
	\ket{\updownarrow}
	=
	\ket{z_-}
\end{align}
to show equivalence to the general qubit description.

\FloatBarrier
\subsection{Time-phase-encoding}

In the following, we discuss the practical time-phase-encoding BB84 protocol and show its equivalence to the polarization-encoding BB84.
The idea of using phase-encoding was first proposed as part of the BB92 protocol~\cite{Bennett1992}.
The basic setup is illustrated in \Cref{fig:qubit_time_phase_active} and comprises a single-photon source and a first \gls{mzi} on Alice's side as well as a second \gls{mzi} and two single-photon detectors on Bob's side.
\begin{figure}[htb]
	\centering
	\includestandalone{figures/pstricks/bb84-time-phase-active}
	\caption{Fiber-optical setup of the phase-encoding BB84: Alice creates an entangled photon state using a first \gls{mzi} with phase $\theta\in\{0,\pi/2,\pi,3\pi/2\}$ and sends it to Bob. Bob detects the photon state using a second \gls{mzi} with phase $\phi\in\{0,\pi/2\}$ and two single-photon detectors monitoring the outputs.}\label{fig:qubit_time_phase_active}
\end{figure}

To understand the time-phase encoding, we analyze the action of the asymmetric \gls{mzi} with variable phase $\varphi$ on a photon pulse $\ket{t_0}$ arriving at time $t_0$, see \Cref{fig:mzi_asymmetric}.
\begin{figure}[htb]
    \centering
    \includegraphics{figures/pstricks/mzi-asymmetric}
     \caption{Asymmetric \gls{mzi} adding a constant time delay and variable phase difference between the upper and lower path: A pulsed state enters the first beam splitter BS1 to the left and is split among a longer upper path and a shorter lower path. A first mirror M1 directs the pulse from the upper path to a phase shifter which adds a relative phase of $\varphi$ between the upper and lower path. A second mirror M2 directs the pulse from the phase shifter to a second beam splitter BS2 while the lower path is between BS1 and BS2.}\label{fig:mzi_asymmetric}
\end{figure}
An ideal (lossless) and symmetric beam splitter transforms the single-photon input states into a superposition according to\footnote{See Ref.~\cite[p.~137]{Haroche2006} and Ref.~\cite[p.~143]{Gerry2005}}
\begin{align}
	\hat{U}_\text{BS}
	\ket{1,0}
	&=
	\frac{1}{\sqrt{2}}
	\left(\ket{1,0}+i\ket{0,1}\right)
	\\
	\hat{U}_\text{BS}
	\ket{0,1}
	&=
	\frac{1}{\sqrt{2}}
	\left(i\ket{1,0}+\ket{0,1}\right)
	.
\end{align}
Then, the first beam splitter BS1 in \Cref{fig:mzi_asymmetric} (instantly) splits a photon pulse $\ket{t_0}$ arriving at $t_0$ into the superposition
\begin{equation}
	\hat{U}_\text{BS}
	\ket{t_0,0}
	=
	\frac{1}{\sqrt{2}}
	\left(\ket{t_0,0}+i\ket{0,t_0}\right)
\end{equation}
where the first mode corresponds to the upper and the second mode to the lower optical path in \Cref{fig:mzi_asymmetric}.
The phase shifter adds a relative phase of $\varphi$ between the upper and lower path and the input state to the second beam splitter BS2 is
\begin{equation}
	\hat{U}_\text{PS}
	\hat{U}_\text{BS}
	\ket{t_0,0}
	=
	\frac{1}{\sqrt{2}}
	\left(
		\ket{t_0+\tau,0}
		+
		ie^{i\varphi}
		\ket{0,t_0+\tau+\Delta\tau}
	\right)
\end{equation}
wherein $\tau$ is the time delay the pulse accumulates over the short upper path and $\Delta\tau$ is the difference in time delay between the shorter, upper and longer, lower path.
The output state of BS2 is equal to the action of the \gls{mzi}
\begin{equation}
	\begin{split}
		\hat{U}_\text{MZM}
		\ket{t_0,0}
		&=
		\hat{U}_\text{BS}
		\hat{U}_\text{PS}
		\hat{U}_\text{BS}
		\ket{t_0,0}
		\\
		&=
		\frac{1}{2}
		\biggl[
			\left(
				\ket{t_0+\tau,0}
				+
				i\ket{0,t_0+\tau}
			\right)
			+
			ie^{i\varphi}
			\left(
				i\ket{t_0+\tau+\Delta\tau,0}
				+
				\ket{0,t_0+\tau+\Delta\tau}
			\right)
		\biggr]
		\\
		&=
		\frac{1}{2}
		\biggl[
			\ket{t_0+\tau,0}
			-
			e^{i\varphi}
			\ket{t_0+\tau+\Delta\tau,0}
			+
			i
			\left(
				\ket{0,t_0+\tau}
				+
				e^{i\varphi}
				\ket{0,t_0+\tau+\Delta\tau}
			\right)
		\biggr]
		.
	\end{split}
	\label{eq:mzi_asymmetric}
\end{equation}

Back to the time-phase-encoding BB84 setup depicted in \Cref{fig:qubit_time_phase_active}, we note that Alice's transmitter consists of a single-photon source and an asymmetric \gls{mzi} where on output is dumped.
Therefore, Alice's states are parametrized by the relative phase $\theta$,
\begin{equation}
	\ket{t_0,\theta}
	=
	\frac{1}{\sqrt{2}}
	\left(
		\ket{t_0}
		-
		e^{i\theta}
		\ket{t_0+\Delta\tau}
	\right)
	,
\end{equation}
which is obtained from \cref{eq:mzi_asymmetric} by absorbing the time delay of the shorter path $\tau$ into the time reference $t_0$ and projecting the first output mode of the \gls{mzi}.
By adding a time delay to the states of $\Delta\tau$, we receive the states for the time-delayed signal
\begin{align}
	\ket{t_1,\phi}_1
	&=
	\frac{1}{\sqrt{2}}
	\left(
		\ket{t_1}
		-
		e^{i\phi}
		\ket{t_1+\Delta\tau}
	\right)
	\\
	\ket{t_1,\phi}_2
	&=
	\frac{i}{\sqrt{2}}
	\left(
		\ket{t_1}
		+
		e^{i\phi}
		\ket{t_1+\Delta\tau}
	\right)
	\label{eq:qubit_time_phase_bob_nodelay}
\end{align}
where we again choose the time $t_1$ such that it cancels the time delay of the short path $\tau$.
If Bob receives a pulse with time delay $\Delta\tau$ at some time $t_1$, i.e., $\ket{t_1+\Delta\tau}$, then his \gls{mzi} provides the two detectors with the states
\begin{align}
	\ket{t_1+\Delta\tau,\phi}_1
	&=
	\frac{1}{\sqrt{2}}
	\left(
		\ket{t_1+\Delta\tau}
		-
		e^{i\phi}
		\ket{t_1+2\Delta\tau}
	\right)
	\\
	\ket{t_1+\Delta\tau,\phi}_2
	&=
	\frac{i}{\sqrt{2}}
	\left(
		\ket{t_1+\Delta\tau}
		+
		e^{i\phi}
		\ket{t_1+2\Delta\tau}
	\right)
	.
	\label{eq:qubit_time_phase_bob}
\end{align}
We note that these are superpositions of states at three different time instances $0,\Delta\tau,2\Delta\tau$.
We drop the pulse time and introduce the state
\begin{equation}
	\ket{\Delta\tau=m}
	=
	\ket{t_1+m\Delta\tau}
\end{equation}
corresponding to the $m$th detection time slot.
In the new notation, Bob's detectors receive a superposition of Alice's states, \cref{eq:qubit_time_phase_bob} and \cref{eq:qubit_time_phase_bob_delayed},
\begin{equation}
	\ket{\theta,\phi}_\pm
	=
	\frac{c_\pm(\theta-\phi)}{\sqrt{2}}
	\biggl[
		\ket{\Delta\tau=0}
		\mp
		\left(
			e^{i\phi}
			\pm
			e^{i\theta}
		\right)
		\ket{\Delta\tau=1}
		\pm
		e^{i(\phi+\theta)}
		\ket{\Delta\tau=2}
	\biggr]
	\label{eq:qubit_time_phase_bob_delayed}
\end{equation}
with phase-dependent normalization constant
\begin{equation}
	c_\pm(\theta-\phi)
	=
	\frac{1}{\sqrt{2\pm\cos(\theta-\phi)}}
	.
\end{equation}
Using the \gls{povm} for detecting a click at time slot $m$,
\begin{equation}
	\left\{
		\hat{P}_m
		=
		\ketbra{\Delta\tau=m}
	\right\}_{m=0,1,2}
	,
\end{equation}
we find the click probabilities for the plus and minus detectors to equal
\begin{equation}
	\begin{split}
		p_{\pm,m}
		=
		\trace{\hat\rho_\pm\hat{P}_m}
		&=
		\expval{\hat{P}_m}{\theta,\phi}_{\pm}
		\\
		&=
		\begin{cases}
			\abs{c_\pm(\theta-\phi)}^2\left(1\pm\cos(\theta-\phi)\right) & m=1 \\
			\abs{c_\pm(\theta-\phi)}^2\frac{1}{2} & m=0,2
		\end{cases}
		.
	\end{split}
\end{equation}
If we configure the detectors to trigger only on the $m=1$ time slot, we find the probability for a click of the plus and minus detectors to be
\begin{equation}
	p_\pm(\theta-\phi)
	=
	\frac{1}{2}
	\left(1\pm\cos(\theta-\phi)\right)
	.
\end{equation}
\Cref{tab:qubit_time_phase_clicks} summarizes the click probability of the plus and minus detectors triggered on the $m=1$ time slot for a restricted choice of phases.
\begin{table}[htb]
	\centering
	\begin{tabular}{cccc}
		\toprule
		\multicolumn{2}{c}{Phase} & \multicolumn{2}{c}{Detector click probability} \\
		\cmidrule{1-2}
		\cmidrule{3-4}
		$\theta$ & $\phi$ & $p_1(\theta-\phi)$ & $p_2(\theta-\phi)$ \\
		\midrule
		\multirow{2}{*}{$0$} & $0$ & \SI{100}{\percent} & \SI{0}{\percent} \\
		& $\pi/2$ & \SI{50}{\percent} & \SI{50}{\percent} \\
		\cmidrule{1-2}
		\multirow{2}{*}{$\pi$} & $0$ & \SI{0}{\percent} & \SI{100}{\percent} \\
		& $\pi/2$ & \SI{50}{\percent} & \SI{50}{\percent} \\
		\cmidrule{1-2}
		\multirow{2}{*}{$\pi/2$} & $0$ & \SI{50}{\percent} & \SI{50}{\percent} \\
		& $\pi/2$ & \SI{100}{\percent} & \SI{0}{\percent} \\
		\cmidrule{1-2}
		\multirow{2}{*}{$3\pi/2$} & $0$ & \SI{50}{\percent} & \SI{50}{\percent} \\
		& $\pi/2$ & \SI{0}{\percent} & \SI{100}{\percent} \\
		\bottomrule
	\end{tabular}
	\caption{Click probabilities for the time-phase-encoding BB84 protocol: Alice choosing $\theta\in\{0,\pi\}$ corresponds to choosing the $Z$ eigenbasis while $\theta\in\{\pi/2,3\pi/2\}$ corresponds to her choosing the $X$ eigenbasis. Bob using no phase shift $\phi=0$ corresponds to a selection of the $X$ basis while Bob adding a phase shift of $\phi=\pi/2$ corresponds to selection of $X$ as measurement basis. Only if Alice and Bob choose the same basis, Bob's click is perfectly correlated with Alice's choice for a basis element. Otherwise, it is completely random.}\label{tab:qubit_time_phase_clicks}
\end{table}
Comparing the click probabilities of \Cref{tab:qubit_time_phase_clicks} with the probabilities of the qubit-based BB84, suggests equivalence of the time-phase-encoding BB84 with the more general qubit-based description of BB84.
While we can identify Alice's state in the time basis in terms with the $Y$ qubit basis,
\begin{equation}
	\ket{t_0,\theta}
	=
	\frac{1}{\sqrt{2}}
	\left(
		\ket{t_0}
		-
		e^{i\theta}
		\ket{t_0+\Delta\tau}
	\right)
	=
	\frac{1}{\sqrt{2}}
	\left(
		\ket{y_+}
		-
		e^{i\theta}
		\ket{y_-}
	\right)
	=
	\ket{\theta}
	,
\end{equation}
the receiver side cannot simply be relabeled into the qubit-based description:
Bob's Hilbert space, spanned by the three time slot states, $\ket{\Delta\tau=m}_{m=0,1,2}$, has one additional dimension compared to the qubit Hilbert space.
Such complication can be addressed using the "squashing"~\cite{Beaudry2008,Gittsovich2014}:
We first find a unitary transformation for the input mode of Bob's receiver.
Second, we show that the \gls{povm} yields the same probability distribution as the qubit-based description for all possible quantum states.
The number state basis $\left\{\ket{n}\right\}_{n\in\mathbb{N}_0}$ is complete and countable allowing a proof by induction.
It is important to show equivalence for all number states as Eve's not limited to the single-photon state.
\section{Overview}

\subsection{Operating principle}

non-orthogonality of quantum states?

\subsection{Classifications}

%  -> EB for security proof, PM for implementation (cite equivalence)

\begin{enumerate}
	\item entanglement-based (EB) vs prepare-and-measure (PM)
	\item continuous- vs. discrete (definition)
	\item active vs passive measurement
\end{enumerate}

\subsection{Post-processing}

\begin{enumerate}
	\item bit mapping (encoding,decoding)
	\item error correction and estimation
	\item privacy amplification
\end{enumerate}

\begin{figure}[htb]
	\centering
	\includestandalone{figures/tikz/post-processing-dv}
	\caption{Post-processing procedure for \gls{dvqkd} protocols: First Alice's and Bob's key bits are correlated and partially secret. Key sifting discards key bits where Alice and Bob chose a different basis or where Bob did not detect an event while error correction removes the remaining differences between Alice's and Bob's partially secret key bits. Comparing the correlated key bits with the corrected key bits, Alice and Bob can estimate an error and thereby an upper bound on information loss to an adversary. Alice and Bob can remove the remaining partial information an adversary potentially has by performing privacy amplification. Finally, Alice and Bob share a secret key bit string.}\label{fig:post_processing}
\end{figure}

\FloatBarrier
\subsection{Security analysis}

For completeness, we provide a short literature review regarding \gls{qkd} security proofs, but a detailed security proof is not the subject matter of the present thesis.

\begin{enumerate}
	\item main idea how a security proof works?
	\item individual, collective and coherent attacks
	\item reduction of coherent to collective attacks~\cite{Renner2009}
	\item outline idea of a security proof
	\item $\varepsilon$-security
\end{enumerate}

Ref.~\cite{Lo2014} and Ref.~\cite{Laudenbach2018} provides an introduction to practical \gls{qkd} security including implementation vectors.
A more theoretical but still practical security analysis for \gls{qkd} is found in Ref.~\cite{Scarani2009}.
For a security analysis particular to \gls{cvqkd}, see Ref.~\cite{Diamanti2015}.

For a in-depth security analysis of the post-processing, see Ref.~\cite{Fung2010}.
For a proof of privacy amplification in the \gls{qkd} context, see Ref.~\cite{Renner2005}
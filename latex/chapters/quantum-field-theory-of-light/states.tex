\section{Quantum states}

\subsection{Single-particle number states}

\begin{definition}[Single-particle number state]\label{def:single_particle_number_state}
	We define a number state of the Klein-Gordon field by
	\begin{equation}
		\ket{f}
		=
		\int\frac{\dd[3]{p}}{(2\pi)^3\sqrt{2\omega(\vb{p})}}
		f(\vb{p})
		\hat{a}^\dagger(\vb{p})
		\ket{0}
	\end{equation}
	where the Lorentz-invariant spectrum $f(\vb{p})=f(\omega(\vb{p}),\vb{p})\in\mathcal{S}(\mathbb{R}^4,\mathbb{C})$ encodes the spectral distribution of $\ket{f}$.
\end{definition}
\begin{remark}
	If we remind ourselves that quantum field operators are operator-valued distributions mapping from the Schwartz function space to quantum operators, the definition follows from the smeared Klein-Gordon field operator acting on the vacuum $\hat\phi[f]\ket{0}=\phi^-[f]\ket{0}=\ket{f}$.
\end{remark}

\begin{lemma}[Inner product of two single-particle number states]\label{th:number_states_non_orthogonal}
	Let $\ket{f},\ket{g}$ be two number states then
	\begin{equation}
		\braket{g}{f}
		=
		\int\frac{\dd[3]{p}}{(2\pi)^32\omega(\vb{p})}
		f(\vb{p})g(\vb{p})^*
	\end{equation}
	indicates that number states are, in general, not orthogonal.
\end{lemma}
\begin{proof}
	For the proof, we insert the definition and use the commutation relation of the annihilation and creation operators of the Klein-Gordon field
	\begin{equation}
		\begin{split}
			\bra{g}\ket{f}
			&=
			\int\frac{\dd[3]{p}}{(2\pi)^3\sqrt{2\omega(\vb{p})}}
			\int\frac{\dd[3]{q}}{(2\pi)^3\sqrt{2\omega(\vb{q})}}
			f(\vb{p})g(\vb{q})^*
			\expval{\hat{a}(\vb{q})\hat{a}^\dagger(\vb{p})}{0}
			\\
			&=
			\int\frac{\dd[3]{p}}{(2\pi)^3\sqrt{2\omega(\vb{p})}}
			\int\frac{\dd[3]{q}}{(2\pi)^3\sqrt{2\omega(\vb{q})}}
			f(\vb{p})g(\vb{q})^*
			(2\pi)^3\delta^{(3)}(\vb{q}-\vb{p})
			\\
			&=
			\int\frac{\dd[3]{p}}{(2\pi)^32\omega(\vb{p})}
			f(\vb{p})g(\vb{p})^*
		\end{split}
	\end{equation}
\end{proof}

\begin{corollary}[Normalization of a single-particle number state spectrum]
	\Cref{th:number_states_non_orthogonal} and the probabilistic interpretation of quantum mechanics require the spectrum of a number state to be constrained by
	\begin{equation}
		\bra{f}\ket{f}
		=
		\int\frac{\dd[3]{p}}{(2\pi)^32\omega(\vb{p})}
		\abs{f(\vb{p})}^2
		=
		1
		.
	\end{equation}
\end{corollary}

\begin{remark}
	In the coordinate representation is not a practical representation as we have integrals involving the Feynman propagator.
	For instance, the inner product of two number states in coordinate representation is
	\begin{equation}
		\begin{split}
			\bra{g}\ket{f}
			=
			\expval{\hat\phi[g]\hat\phi[f]}{0}
			&=
			\int\dd[4]{x}
			\int\dd[4]{y}
			f(x^0,\vb{x})g(y^0,\vb{y})^*
			\expval{\hat\phi(x^0,\vb{x})\hat\phi(y^0,\vb{y})}{0}
			\\
			&=
			\int\dd[4]{x}
			\int\dd[4]{y}
			f(x^0,\vb{x})g(y^0,\vb{y})^*
			D(x^0-y^0,\vb{x}-\vb{y})
		\end{split}
	\end{equation}
	where $D(x^0-y^0,\vb{x}-\vb{y})$ is the Feynman propagator~\cite[p.~27]{Peskin1995}.
\end{remark}

\begin{example}[Covariant Gaussian single-particle number state]
	Naumov~\cite{Naumov2013,Naumov2009} claims $\ket{f}$ with
	\begin{equation}
		f(\vb{p})
		\propto
		\exp\left\{
			\frac{(p_\mu-k_\mu)(p^\mu-k^\mu)}{4\sigma_P^2}
		\right\}
		=
		\exp\left\{
			-
			\frac{\omega(\vb{p})\omega(\vb{k})-\vb{k}\vdot\vb{p}}{2\sigma^2}
		\right\}
		\label{eq:covariant_gaussian_spectrum}
	\end{equation}
	to describe a single-particle number state with spectrum peaked at $k^\mu=(k_0,\vb{k})$ with variance $\sigma^2$.
\end{example}
\begin{remark}
	In Ref.~\cite{Naumov2009} a discussion of the relativistic effects on the wave packet.
	In particular, a complete covariant description requires renormalization.
	However, close to the maximum of $f(\vb{p})$ a non-covariant approximation is justified.
\end{remark}
\begin{example}[Approximate massive Gaussian single-particle number state]
	According to Naumov~\cite{Naumov2013}, the spectrum of the covariant Gaussian number state can be approximated with
	\begin{equation}
		f(\vb{p})
		=
		\sqrt{2m}
		\left(
			\frac{2\pi}{\sigma^2}
		\right)^{3/4}
		\exp\left\{
			-
			\frac{(\vb{k}-\vb{p}_L)^2}{4\sigma_L^2}
			-
			\frac{\vb{p}_T^2}{4\sigma_T^2}
		\right\}
	\end{equation}
	for $\vb{p}\approx\vb{k}$, wherein $\vb{p}_T,\vb{p}_L$ are the momentum transversal and longitudinal to $\vb{k}$, and $\sigma_L=\omega(\vb{p})\sigma/m$ and $\sigma_T=\sigma_P$ are the longitudinal and transversal momentum spreads.
\end{example}
\begin{example}[Massless Gaussian single-particle number state]
	Evaluating \cref{eq:covariant_gaussian_spectrum} with $\omega(\vb{p})=\norm{\vb{p}}$ yields
	\begin{equation}
		f(\vb{p})
		\propto
		\exp\left\{
			-
			\frac{\norm{\vb{p}}\norm{\vb{k}}-\vb{k}\vdot\vb{p}}{2\sigma^2}
		\right\}
	\end{equation}
\end{example}

\begin{lemma}[Energy moments of the single-paricle number state]
	Calculating the first two moments of the energy observables, \cref{qkg:energy}, with respect to a single-particle number state $\ket{f}$ yields
	\begin{align}
		\expval{\hat{H}}{f}
		&=
		\int\frac{\dd[3]{p}}{(2\pi)^32\omega(\vb{p})}
		\omega(\vb{p})
		\abs{f(\vb{p})}^2
		\\
		\expval{\hat{H}^2}{f}
		&=
		\int\frac{\dd[3]{p}}{(2\pi)^32\omega(\vb{p})}
		\omega(\vb{p})^2
		\abs{f(\vb{p})}^2
		\label{eq:single_particle_number_state_energy}
	\end{align}
	for the first two moments implying non-zero energy fluctuations
	\begin{equation}
		\expval{\left(\Delta\hat{H}\right)^2}{f}
		=
		\expval{\hat{H}^2}{f}
		-
		\expval{\hat{H}}{f}^2
		>
		0
	\end{equation}
	in agreement with Ref.~\cite[eqs.~10 and 11]{Naumov2013}.
\end{lemma}

\begin{corollary}[Single-particle number state is eigenstate of the number operator]
	Performing the replacement $\omega(\vb{p})\to1$ in \cref{eq:single_particle_number_state_energy} and employing the normalization condition yields
	\begin{equation}
		\expval{\hat{N}}{f}
		=
		1
		=
		\expval{\hat{N}^2}{f}
	\end{equation}
	which leads to zero variance in the number observable, implying that $\ket{f}$ is a number eigenstate.
\end{corollary}

\begin{definition}[Coordinate wave function]
	We define the coordinate wave function of a state $\ket{\psi}$
	\begin{equation}
		\psi(t,\vb{x})
		=
		\bra{0}\hat\phi(t,\vb{x})\ket{\psi}
	\end{equation}
\end{definition}
\begin{example}[Coordinate wave function of single-particle number state]
	The coordinate wave function of a single-particle number state turns out to be
	\begin{equation}
		\psi(t,\vb{x})
		=
		\int\frac{\dd[3]{p}}{(2\pi)^32\omega(\vb{p})}
		f(\vb{p})e^{-ip_\mu x^\mu}
	\end{equation}
\end{example}

\begin{definition}[Probability current]
	The covariant probability current
	\begin{equation}
		j_\mu(t,\vb{x})
		=
		2
		\Im\left\{
			\psi(t,\vb{x})^*
			\partial_\mu
			\psi(t,\vb{x})
		\right\}
		\label{eq:qkg_probability_current}
	\end{equation}	
\end{definition}
\begin{lemma}[Probability current conservation]
	The probability current is the conserved Noether's current of the Klein-Gordon field.
\end{lemma}
\begin{proof}
	See Ref.~\cite[p.~18]{Peskin1995})
\end{proof}

\begin{definition}[Group velocity]
	We refer to the spatial integration of the probability current $\vb{j}$
	\begin{equation}
		\expval{\vb{v}}
		=
		\int\dd[3]{x}
		\vb{j}(t,\vb{x})
		\label{eq:group_velocity}
	\end{equation}
	as the group velocity.
\end{definition}
\begin{example}[Group velocity of a single-particle number state]
	The group velocity of a single-particle number state $\ket{f}$ is
	\begin{equation}
		\expval{\vb{v}}
		=
		\int\frac{\dd[3]{p}}{(2\pi)^32\omega(\vb{p})}
		\abs{f(\vb{p})}^2
		\frac{\vb{p}}{\omega(\vb{p})}
		\label{eq:single_particle_number_state_group_velocity}
	\end{equation}
\end{example}
\begin{definition}[Mean localization]
	We define the mean localization to be
	\begin{equation}
		\expval{\vb{x}(t)}
		=
		\int\dd[3]{x}
		\vb{x}
		\rho(t,\vb{x})
	\end{equation}
\end{definition}
\begin{example}[Mean localization of a single-particle number state]
	\begin{equation}
		\expval{\vb{x}(t)}
		=
		\expval{\vb{v}}t
	\end{equation}	
\end{example}
\begin{definition}[Spatial dispersion]
	The spatial dispersion is defined by
	\begin{equation}
		\sigma_x(t)^2
		=
		\expval{\vb{x}(t)^2}
		-
		\expval{\vb{x}(t)}^2		
	\end{equation}
\end{definition}
\begin{example}[Spatial dispersion for massive Gaussian single-particle number states]
	See Ref.~\cite{Naumov2013}.
\end{example}

\subsection{Coherent states}

\begin{equation}
	\begin{split}
		\ket{\alpha}
		&=
		\hat{D}[J]
		\ket{0}
		\\
		&=
		\exp\left\{
			-
			\frac{1}{2}
			\int\frac{\dd[3]{p}}{(2\pi)^32\omega(\vb{p})}
			\abs{\alpha(\vb{p})}^2
		\right\}
		\\
		&\times
		\exp\left\{
			-i
			\int\frac{\dd[3]{p}}{(2\pi)^3\sqrt{2\omega(\vb{p})}}
			\alpha(\vb{p})
			\hat{a}^\dagger(\vb{p})
		\right\}
		\ket{0}
	\end{split}
\end{equation}

\subsubsection{Number, energy, and momentum}

The coherent state is eigenstate of the annihilation operator
\begin{equation}
	\hat{a}(\vb{p})
	\ket{\alpha}
	=
	\frac{\alpha(\vb{p})}{\sqrt{2\omega(\vb{p})}}
	\ket{\alpha}
\end{equation}

The first two moments of the energy operator are exactly the same as for the sidle-particle state
\begin{align}
	\expval{\hat{H}}{\alpha}
	&=
	\int\frac{\dd[3]{p}}{(2\pi)^32\omega(\vb{p})}
	\omega(\vb{p})
	\abs{\alpha(\vb{p})}^2
	\\
	\expval{\hat{H}^2}{\alpha}
	&=
	\int\frac{\dd[3]{p}}{(2\pi)^32\omega(\vb{p})}
	\omega(\vb{p})^2
	\abs{\alpha(\vb{p})}^2
\end{align}
which is similar to our single-particle case with the big difference that is $\alpha(\vb{p})$ not normalized.
The same follows for the momentum moments
\begin{align}
	\expval{\hat{\vb{P}}}{\alpha}
	&=
	\int\frac{\dd[3]{p}}{(2\pi)^32\omega(\vb{p})}
	\vb{p}
	\abs{\alpha(\vb{p})}^2
	\\
	\expval{\hat{\vb{P}}^2}{\alpha}
	&=
	\int\frac{\dd[3]{p}}{(2\pi)^32\omega(\vb{p})}
	\vb{p}^2
	\abs{\alpha(\vb{p})}^2
\end{align}
and the number moments
\begin{align}
	\expval{\hat{N}}{\alpha}
	&=
	\int\frac{\dd[3]{p}}{(2\pi)^32\omega(\vb{p})}
	\abs{\alpha(\vb{p})}^2
	\\
	\expval{\hat{N}^2}{\alpha}
	&=
	\int\frac{\dd[3]{p}}{(2\pi)^32\omega(\vb{p})}
	\abs{\alpha(\vb{p})}^2
\end{align}
the variance of the number operator

The probability of detecting a single-particle given a coherent state is
\begin{equation}
	\begin{split}
		\braket{f}{\alpha}
		&=
		\exp\left\{
			-
			\frac{1}{2}
			\int\frac{\dd[3]{p}}{(2\pi)^32\omega(\vb{p})}
			\abs{\alpha(\vb{p})}^2
		\right\}
		\int\frac{\dd[3]{p}}{(2\pi)^32\omega(\vb{p})}
		f(\vb{p})^*
		\alpha(\vb{p})
	\end{split}
\end{equation}
Assuming spectral overlap, i.e., $\alpha(\vb{p})=\alpha f(\vb{p})$, we recover the Poisson distribution
\begin{equation}
	\begin{split}
		\braket{f}{\alpha}
		&=
		\alpha
		\exp\left\{
			-
			\frac{1}{2}
			\abs{\alpha}^2
		\right\}
	\end{split}
\end{equation}

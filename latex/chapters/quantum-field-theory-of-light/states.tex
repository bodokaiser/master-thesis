\section{Quantum states}

\subsection{Number states}

\subsubsection{Wightman axioms and smearing functions}

According to Wightman quantum field theory~\cite[p.~324]{Bogolubov1989}, a quantum field $\hat\Phi$ is an operator-valued tempered distribution which needs be combined with some Lorentz-invariant Schwartz function $f\in\mathcal{S}(\mathbb{R}^4,\mathbb{K})$
\begin{equation}
	\hat\Phi[f]
	=
	\int\dd[4]{x}
	f(t,\vb{x})
	\hat\Phi(t,\vb{x})
\end{equation}
to avoid ambiguities arising from the idealized non-physical ultra-localized momentum states which we would recover for
\begin{equation}
	\hat\Phi[\delta_x]
	=
	\int\dd[4]{y}
	\delta^{(1)}(y^0-x^0)
	\delta^{(3)}(\vb{y}-\vb{x})
	\hat\Phi(y^0,\vb{y})
	=
	\hat\Phi(x^0,\vb{x})
	.
\end{equation}

\subsubsection{Smeared single-particle state and overlap}

Smearing the Klein-Gordon field operator $\hat\phi^-$ acting on the vacuum state $\ket{0}$ with $f\in\mathcal{S}(\mathbb{R}^4,\mathbb{R})$ and defining $f(\vb{p})=f(\omega(\vb{p}),\vb{p})$
\begin{align}
	\hat\phi[f]
	\ket{0}
	&=
	\int\dd[4]{x}
	f(t,\vb{x})
	\hat\phi^-(t,\vb{x})
	\ket{0}
	\\
	&=
	\int\frac{\dd[3]{p}}{(2\pi)^3\sqrt{2\omega(\vb{p})}}
	\left(
		\int\dd[4]{x}
		f(t,\vb{x})
		e^{+ip_\mu x^\mu}
	\right)
	\hat{a}^\dagger(\vb{p})
	\ket{0}
	\\
	&=
	\int\frac{\dd[3]{p}}{(2\pi)^3\sqrt{2\omega(\vb{p})}}
	f(\vb{p})
	\hat{a}^\dagger(\vb{p})
	\ket{0}
\end{align}
and extrapolating from the known action of $\hat\phi$ on the vacuum, suggests
\begin{equation}
	\ket{f}
	=
	\int\frac{\dd[3]{p}}{(2\pi)^3\sqrt{2\omega(\vb{p})}}
	f(\vb{p})
	\hat{a}^\dagger(\vb{p})
	\ket{0}
\end{equation}
to represent a smeared particle state with localization in spacetime $f(t,\vb{x})$ respective momentum space $f(\vb{p})$.
The probability amplitude to measure $\ket{g}$ given $\ket{f}$ was prepared is equal to the overlap of the smearing functions
\begin{equation}
	\begin{split}
		\bra{g}\ket{f}
		&=
		\int\frac{\dd[3]{p}}{(2\pi)^3\sqrt{2\omega(\vb{p})}}
		\int\frac{\dd[3]{q}}{(2\pi)^3\sqrt{2\omega(\vb{q})}}
		f(\vb{p})g(\vb{q})^*
		\expval{\hat{a}(\vb{q})\hat{a}^\dagger(\vb{p})}{0}
		\\
		&=
		\int\frac{\dd[3]{p}}{(2\pi)^3\sqrt{2\omega(\vb{p})}}
		\int\frac{\dd[3]{q}}{(2\pi)^3\sqrt{2\omega(\vb{q})}}
		f(\vb{p})g(\vb{q})^*
		(2\pi)^3\delta^{(3)}(\vb{q}-\vb{p})
		\\
		&=
		\int\frac{\dd[3]{p}}{(2\pi)^32\omega(\vb{p})}
		f(\vb{p})g(\vb{p})^*
	\end{split}
\end{equation}
and suggests the normalization to be
\begin{equation}
	\bra{f}\ket{f}
	=
	\int\frac{\dd[3]{p}}{(2\pi)^32\omega(\vb{p})}
	\abs{f(\vb{p})}^2
	=
	1
\end{equation}
Alternatively, in coordinate space
\begin{equation}
	\begin{split}
		\bra{g}\ket{f}
		=
		\expval{\hat\phi[g]\hat\phi[f]}{0}
		&=
		\int\dd[4]{x}
		\int\dd[4]{y}
		f(x^0,\vb{x})g(y^0,\vb{y})^*
		\expval{\hat\phi(x^0,\vb{x})\hat\phi(y^0,\vb{y})}{0}
		\\
		&=
		\int\dd[4]{x}
		\int\dd[4]{y}
		f(x^0,\vb{x})g(y^0,\vb{y})^*
		D(x^0-y^0,\vb{x}-\vb{y})
	\end{split}
\end{equation}
where $D(x^0-y^0,\vb{x}-\vb{y})$ is the Feynman propagator~\cite[p.~27]{Peskin1995}.

\subsubsection{Number, energy, and momentum expectation values}

The first two moments of the number operator are
\begin{align}
	\expval{\hat{N}}{f}
	&=
	\int\frac{\dd[3]{p}}{(2\pi)^32\omega(\vb{p})}
	\abs{f(\vb{p})}^2
	=
	1
	\\
	\expval{\hat{N}^2}{f}
	&=
	\int\frac{\dd[3]{p}}{(2\pi)^32\omega(\vb{p})}
	\abs{f(\vb{p})}^2
	=
	1
\end{align}
suggesting $\ket{f}$ to be a single-particle number state because $\expval{\left(\Delta\hat{N}\right)}{f}=0$
Similar the first two moments of the total energy are
\begin{align}
	\expval{\hat{H}}{f}
	&=
	\int\frac{\dd[3]{p}}{(2\pi)^32\omega(\vb{p})}
	\omega(\vb{p})
	\abs{f(\vb{p})}^2
	\\
	\expval{\hat{H}^2}{f}
	&=
	\int\frac{\dd[3]{p}}{(2\pi)^32\omega(\vb{p})}
	\omega(\vb{p})^2
	\abs{f(\vb{p})}^2
\end{align}
implying non-zero energy fluctuations $\expval{\left(\Delta\hat{H}\right)^2}{f}>0$.
By an completely analog calculation, we find for the total momentum
\begin{align}
	\expval{\hat{\vb{P}}}{f}
	&=
	\int\frac{\dd[3]{p}}{(2\pi)^32\omega(\vb{p})}
	\vb{p}
	\abs{f(\vb{p})}^2
	\\
	\expval{\hat{\vb{P}}^2}{f}
	&=
	\int\frac{\dd[3]{p}}{(2\pi)^32\omega(\vb{p})}
	\vb{p}^2
	\abs{f(\vb{p})}^2
\end{align}
in agreement with results reported in~\cite[eqs.~10 and 11]{Naumov2013}.

\subsubsection{Group velocity, mean position and spatial dispersion}

From the spacetime representation of the wave function
\begin{equation}
	\psi(t,\vb{x})
	=
	\bra{0}\hat\phi(t,\vb{x})\ket{f}
	=
	\int\frac{\dd[3]{p}}{(2\pi)^32\omega(\vb{p})}
	f(\vb{p})e^{-ip_\mu x^\mu}
\end{equation}
we can calculate the probability current being the conserved Noether's current of the Klein-Gordon field~\cite[p.~18]{Peskin1995})
\begin{equation}
	j_\mu(t,\vb{x})
	=
	2
	\Im\left\{
		\psi(t,\vb{x})^*
		\partial_\mu
		\psi(t,\vb{x})
	\right\}
	\label{eq:qkg_probability_current}.
\end{equation}
Spatial integration of the probability current yields the group velocity of $\ket{f}$
\begin{equation}
	\expval{\vb{v}}
	=
	\int\dd[3]{x}
	\vb{j}(t,\vb{x})
	=
	\int\frac{\dd[3]{p}}{(2\pi)^32\omega(\vb{p})}
	\abs{f(\vb{p})}^2
	\frac{\vb{p}}{\omega(\vb{p})}
	\label{qkg:group_velocity}.
\end{equation}
Indeed in the non-relativistic regime $\omega(\vb{p})^2\sim m^2$
and the group velocity equals $\vb{p}/(2m^2)$ weighted by $\abs{f(\vb{p})}^2$.
The probability center-of-mass position is
\begin{equation}
	\expval{\vb{x}(t)}
	=
	\int\dd[3]{x}\vb{x}\rho(t,\vb{x})
	=
	\expval{\vb{v}}t
	,
\end{equation}
with the variance quantifying the spatial dispersion in time
\begin{equation}
	\sigma_x(t)^2
	=
	\expval{\vb{x}(t)^2}
	-
	\expval{\vb{x}(t)}^2
\end{equation}
which can be found in Naumov~\cite{Naumov2013}.

\subsubsection{Example of a massless Gaussian single-particle state}

Naumov~\cite{Naumov2013} discusses a massive Gaussian wave packets.

Let $k_\mu=(k_0,\vb{k})$ be a constant four-momentum vector with $k_0=\omega(\vb{k})$ (?), then the function
\begin{equation}
	f(\vb{p})
	\propto
	\exp\left\{
		\frac{(p_\mu-k_\mu)(p^\mu-k^\mu)}{4\sigma_p^2}
	\right\}
	=
	\exp\left\{
		\frac{p_\mu p^\mu-2k_\mu p^\mu+k_\mu k^\mu}{4\sigma_p^2}
	\right\}
\end{equation}
is an element of the Schwartz space $\mathcal{S}(\mathbb{R}^4,\mathbb{R})$ and is as a function of Lorentz scalars manifest Lorentz invariant, thus, a candidate for a smearing function.

For a massless Klein-Gordon field the dispersion relation is $\omega(\vb{p})=\norm{\vb{p}}$ and the smearing function simplifies to
\begin{equation}
	\begin{split}
		f(\vb{p})
		&\propto
		\exp\left\{
			-
			\frac{1}{2\sigma_p^2}
			\left(
				k_0
				\norm{\vb{p}}
				-
				\vb{k}\vdot\vb{p}
			\right)
		\right\}
		\\
		&=
		\exp\left\{
			-
			\frac{k_0}{2\sigma_p^2}
			\norm{\vb{p}}
			\left(
				1	
				-
				\cos\theta
			\right)
		\right\}
	\end{split}
\end{equation}

\subsection{Coherent states}


\begin{equation}
	\begin{split}
		\ket{\alpha}
		&=
		\hat{D}[J]
		\ket{0}
		\\
		&=
		\exp\left\{
			-
			\frac{1}{2}
			\int\frac{\dd[3]{p}}{(2\pi)^32\omega(\vb{p})}
			\abs{\alpha(\vb{p})}^2
		\right\}
		\\
		&\times
		\exp\left\{
			-i
			\int\frac{\dd[3]{p}}{(2\pi)^3\sqrt{2\omega(\vb{p})}}
			\alpha(\vb{p})
			\hat{a}^\dagger(\vb{p})
		\right\}
		\ket{0}
	\end{split}
\end{equation}

\subsubsection{Number, energy, and momentum}

The coherent state is eigenstate of the annihilation operator
\begin{equation}
	\hat{a}(\vb{p})
	\ket{\alpha}
	=
	\frac{\alpha(\vb{p})}{\sqrt{2\omega(\vb{p})}}
	\ket{\alpha}
\end{equation}

The first two moments of the energy operator are exactly the same as for the sidle-particle state
\begin{align}
	\expval{\hat{H}}{\alpha}
	&=
	\int\frac{\dd[3]{p}}{(2\pi)^32\omega(\vb{p})}
	\omega(\vb{p})
	\abs{\alpha(\vb{p})}^2
	\\
	\expval{\hat{H}^2}{\alpha}
	&=
	\int\frac{\dd[3]{p}}{(2\pi)^32\omega(\vb{p})}
	\omega(\vb{p})^2
	\abs{\alpha(\vb{p})}^2
\end{align}
which is similar to our single-particle case with the big difference that is $\alpha(\vb{p})$ not normalized.
The same follows for the momentum moments
\begin{align}
	\expval{\hat{\vb{P}}}{\alpha}
	&=
	\int\frac{\dd[3]{p}}{(2\pi)^32\omega(\vb{p})}
	\vb{p}
	\abs{\alpha(\vb{p})}^2
	\\
	\expval{\hat{\vb{P}}^2}{\alpha}
	&=
	\int\frac{\dd[3]{p}}{(2\pi)^32\omega(\vb{p})}
	\vb{p}^2
	\abs{\alpha(\vb{p})}^2
\end{align}
and the number moments
\begin{align}
	\expval{\hat{N}}{\alpha}
	&=
	\int\frac{\dd[3]{p}}{(2\pi)^32\omega(\vb{p})}
	\abs{\alpha(\vb{p})}^2
	\\
	\expval{\hat{N}^2}{\alpha}
	&=
	\int\frac{\dd[3]{p}}{(2\pi)^32\omega(\vb{p})}
	\abs{\alpha(\vb{p})}^2
\end{align}
the variance of the number operator

The probability of detecting a single-particle given a coherent state is
\begin{equation}
	\begin{split}
		\braket{f}{\alpha}
		&=
		\exp\left\{
			-
			\frac{1}{2}
			\int\frac{\dd[3]{p}}{(2\pi)^32\omega(\vb{p})}
			\abs{\alpha(\vb{p})}^2
		\right\}
		\int\frac{\dd[3]{p}}{(2\pi)^32\omega(\vb{p})}
		f(\vb{p})^*
		\alpha(\vb{p})
	\end{split}
\end{equation}
Assuming spectral overlap, i.e., $\alpha(\vb{p})=\alpha f(\vb{p})$, we recover the Poisson distribution
\begin{equation}
	\begin{split}
		\braket{f}{\alpha}
		&=
		\alpha
		\exp\left\{
			-
			\frac{1}{2}
			\abs{\alpha}^2
		\right\}
	\end{split}
\end{equation}

\section{Quantum states}

\subsection{Vacuum state}

\begin{definition}[Vacuum state]
	The vacuum state $\ket{0}$ is a unique state that is normalized
	\begin{equation}
		\braket{0}
		=
		1
	\end{equation}
	and satisfies
	\begin{align}
		\hat{P}_\mu
		\ket{0}
		&=
		0
		\\
		\hat{M}_{\mu\nu}
		\ket{0}
		&=
		\ket{0}
	\end{align}
	where $\hat{P}_\mu$ is the four-momentum and $\hat{M}_{\mu\nu}$ is the relativistic angular momentum operator, see Ref.~\cite[p.~270]{Greiner2013} for details.
\end{definition}
\begin{corollary}
	The vacuum state is invariant under spacetime translations
	\begin{equation}
		e^{i\hat{P}_\mu a^\mu}
		\ket{0}
		=
		\ket{0}
		.
	\end{equation}	
\end{corollary}

\subsection{Momentum state}

\begin{definition}[Momentum state]
	The creation operator acting on the vacuum state
	\begin{equation}
		\ket{\vb{p}}
		=
		\sqrt{2\omega(\vb{p})}
		\hat{a}^\dagger(\vb{p})
		\ket{0}
	\end{equation}
	creates a state with momentum $\vb{p}$, the momentum state.
\end{definition}
\begin{lemma}\label{thm:momentum_state_non_normalizable}
	The momentum state $\ket{\vb{p}}$ is not normalizable.
\end{lemma}
\begin{theorem}\label{thm:momentum_state_eigenstate}
	The momentum state $\ket{\vb{p}}$ is an eigenstate of the momentum operator $\hat{\vb{P}}$
	\begin{equation}
		\hat{\vb{P}}
		\ket{\vb{p}}
		=
		\vb{p}
		\ket{\vb{p}}
	\end{equation}
	with eigenvalue $\vb{p}$.
\end{theorem}
\begin{lemma}\label{thm:momentum_state_wave_function}
	The coordinate wave function of a momentum state $\ket{\vb{p}}$ is a plane-wave $\psi(t,\vb{x})=e^{-ip_\mu x^\mu}$.
\end{lemma}

\subsection{Single-particle number state}

\begin{definition}[Single-particle number state]\label{def:single_particle_number_state}
	The single-particle number state with Lorentz-invariant spectrum $f(\vb{p})=f(\omega(\vb{p}),\vb{p})\in\mathcal{S}(\mathbb{R}^4,\mathbb{C})$ is
	\begin{equation}
		\ket{1_f}
		=
		\int\frac{\dd[3]{p}}{(2\pi)^3\sqrt{2\omega(\vb{p})}}
		f(\vb{p})
		\hat{a}^\dagger(\vb{p})
		\ket{0}
		.
	\end{equation}
\end{definition}
\begin{lemma}\label{thm:single_particle_number_state_smeared_kg}
	The single-particle number state is equal to the smeared Klein-Gordon field operator acting on the vacuum state
	\begin{equation}
		\hat\phi[f]
		\ket{0}
		=
		\int\dd[4]{x}
		f(x)
		\hat\phi(x)
		\ket{0}
		=
		\ket{1_f}
	\end{equation}
	known from Wightman quantum field theory.\footnote{Wightman quantum field theory defines quantum field operators as operator-valued distributions acting on the Schwartz function space, see Ref.~\cite{Bogolubov1989} and Ref.~\cite{Streater2016}.}
\end{lemma}
\begin{lemma}\label{th:single_particle_number_states_inner_product}
	Let $\ket{1_f},\ket{1_g}$ be two single-particle number states then their inner product is equal to
	\begin{equation}
		\braket{1_g}{1_f}
		=
		\int\frac{\dd[3]{p}}{(2\pi)^32\omega(\vb{p})}
		f(\vb{p})g(\vb{p})^*
	\end{equation}
	indicating that the single-particle number states are overcomplete.
\end{lemma}
\begin{lemma}\label{thm:single_particle_number_state_normalization}
	The spectrum of a single-particle number state satisfies
	\begin{equation}
		\bra{1_f}\ket{1_f}
		=
		\int\frac{\dd[3]{p}}{(2\pi)^32\omega(\vb{p})}
		\abs{f(\vb{p})}^2
		=
		1
		.
	\end{equation}
\end{lemma}
Let us consider some examples for the single-particle number state spectrum $f(\vb{p})$.
\begin{example}
	The spectrum of a Gaussian single-particle number state is
	\begin{equation}
		f(\vb{p})
		\propto
		\exp\left\{
			\frac{(p_\mu-k_\mu)(p^\mu-k^\mu)}{4\sigma_P^2}
		\right\}
		\label{eq:covariant_gaussian_spectrum}
	\end{equation}
	where the spectrum has mean $k^\mu=(k_0,\vb{k})$ and variance $\sigma^2$.\footnote{See Ref.~\cite{Naumov2013,Naumov2009} for a in-depth discussion.}
\end{example}
\begin{theorem}\label{thm:single_particle_number_state_number_eigenstate}
	The single-particle number state $\ket{1_f}$ is an eigenstate of the number operator $\hat{N}$
	\begin{equation}
		\hat{N}
		\ket{1_f}
		=
		1
		\ket{1_f}
	\end{equation}
	with eigenvalue $1$.
\end{theorem}
\begin{lemma}\label{thm:single_particle_number_state_energy}
	The first energy moment of the single-particle number state $\ket{1_f}$ is
	\begin{equation}
		\expval{\hat{H}}{1_f}
		=
		\int\frac{\dd[3]{p}}{(2\pi)^3\sqrt{2\omega(\vb{p})}}
		\omega(\vb{p})
		\abs{f(\vb{p})}^2
	\end{equation}
	and the second moment is
	\begin{equation}
		\expval{\hat{H}^2}{1_f}
		=
		\int\frac{\dd[3]{p}}{(2\pi)^3\sqrt{2\omega(\vb{p})}}
		\omega(\vb{p})^2
		\abs{f(\vb{p})}^2
		.
	\end{equation}
\end{lemma}
\begin{corollary}
	The single-particle number state has non-zero energy fluctuations, i.e.,
	\begin{equation}
		\expval{\left(\Delta\hat{H}\right)^2}{1_f}
		>
		0
		.
	\end{equation}
\end{corollary}
\begin{lemma}\label{thm:single_particle_number_state_momentum_density_mean}
	The mean momentum density of the single-particle number state is zero, i.e.,
	\begin{equation}
		\expval{\hat\pi(t,\vb{x})}{1_f}
		=
		0
	\end{equation}
\end{lemma}
\begin{lemma}\label{thm:single_particle_number_state_momentum_density_correlation}
	The correlation function of the momentum density operator is
	\begin{equation}
		\expval{\hat\pi(x^\mu)\hat\pi(y^\mu)}{1_f}
		=
		2
		\left(
			\int\frac{\dd[3]{p_1}}{(2\pi)^3}
			f(\omega(\vb{p}_1),\vb{p}_1)
			e^{-ip_1^\mu x^\mu}
		\right)^*
		\left(
			\int\frac{\dd[3]{p_2}}{(2\pi)^3}
			f(\omega(\vb{p}_2),\vb{p}_2)
			e^{-ip_2^\mu y^\mu}
		\right)
	\end{equation}
\end{lemma}
\begin{corollary}
	The variance of the momentum density observable is
	\begin{equation}
		\expval{\left(\Delta\hat\pi(x^\mu)\right)^2}{1_f}
		=
		2
		\abs{
			\int\frac{\dd[3]{p}}{(2\pi)^3}
			f(\omega(\vb{p}),\vb{p})
			e^{-ip^\mu y^\mu}
		}^2
		.
	\end{equation}
\end{corollary}
\begin{lemma}\label{thm:single_particle_number_state_wave_function}
	The coordinate wave function of a single-particle number state is
	\begin{equation}
		\psi(t,\vb{x})
		=
		\int\frac{\dd[3]{p}}{(2\pi)^32\omega(\vb{p})}
		f(\vb{p})e^{-ip_\mu x^\mu}
		.
	\end{equation}
\end{lemma}
\begin{lemma}\label{thm:single_particle_number_state_group_velocity}
	The group velocity of a single-particle number state $\ket{1_f}$ is
	\begin{equation}
		\expval{\vb{v}}
		=
		\int\frac{\dd[3]{p}}{(2\pi)^32\omega(\vb{p})}
		\abs{f(\vb{p})}^2
		\frac{\vb{p}}{\omega(\vb{p})}
		\label{eq:single_particle_number_state_group_velocity}
		.
	\end{equation}
\end{lemma}
\begin{lemma}
	The single-particle number state is localized on a trajectory
	\begin{equation}
		\expval{\vb{x}(t)}
		=
		\expval{\vb{v}}t
	\end{equation}
	moving with the group velocity $\expval{\vb{v}}$.
\end{lemma}
Later, we are going to need
\begin{lemma}\label{thm:single_partiicle_number_state_inner_product_pn_smeared_kg_comm}
	The commutator of the smeared positive and negative frequency Klein-Gordon operators is equal to the overlap of the corresponding single-particle number states, i.e.,
	\begin{equation}
		\comm{\hat\phi^+[f]}{\hat\phi^-[g]}
		=
		\braket{1_f}{1_g}
		.
	\end{equation}
\end{lemma}

\subsection{Multi-particle number state}

\begin{definition}[Multi-particle number state]
	A multi-particle number state with $n$ particles and spectrum $f$ is given by
	\begin{equation}
		\ket{n_f}
		=
		\frac{1}{\sqrt{n!}}
		\hat\phi[f]^n
		\ket{0}
		=
		\frac{1}{\sqrt{n!}}
		\left(
			\int\frac{\dd[3]{p}}{(2\pi)^3\sqrt{2\omega(\vb{p})}}
			f(\vb{p})
			\hat{a}^\dagger(\vb{p})
		\right)^n
		\ket{0}
	\end{equation}
\end{definition}
\begin{theorem}
	The multi-particle number state $\ket{n_f}$ is an eigenstate of the number operator $\hat{N}$ to eigenvalue $n$, i.e.,
	\begin{equation}
		\hat{N}
		\ket{n_f}
		=
		n
		\ket{n_f}
		.
	\end{equation}
\end{theorem}
\begin{corollary}
	The multi-particle number state $\ket{n_f}$ has zero uncertainty in the number observable.
\end{corollary}
\begin{theorem}\label{thm:multi_particle_number_state_inner_product}
	Let $\ket{n_f}$ be an $n$-particle number state with spectrum $f(\vb{p})$ and $\ket{m_g}$ a $m$-particle number state with spectrum $g(\vb{p})$, then their inner product is
	\begin{equation}
		\braket{n_f}{m_g}
		=
		\delta_{nm}
		\left(
			\int\frac{\dd[3]{p}}{(2\pi)^32\omega(\vb{p})}
			f(\vb{p})^*
			g(\vb{p})
		\right)^n
		.
	\end{equation}
\end{theorem}
\begin{corollary}
	Multi-particle number states with the same spectrum $f(\vb{p})$ are orthogonal.
\end{corollary}

\subsection{Coherent state}

\begin{definition}[Displacement operator]
	The displacement operator with spectrum $\alpha(\vb{p})$ is
	\begin{equation}
		\hat{D}[\alpha]
		=
		\exp\left\{
			\int\frac{\dd[3]{p}}{(2\pi)^3\sqrt{2\omega(\vb{p})}}
			\left\{
				\alpha(\vb{p})
				\hat{a}^\dagger(\vb{p})
				-
				\alpha(\vb{p})^*
				\hat{a}(\vb{p})
			\right\}
		\right\}
		\label{eq:displacement_operator}
		.
	\end{equation}
\end{definition}
\begin{lemma}\label{thm:displacement_operator_normal_ordered}
	The displacement operator in normal-order is
	\begin{equation}
		\begin{split}
			\hat{D}[\alpha]
			&=
			\exp\left\{
				+
				\int\frac{\dd[3]{p}}{(2\pi)^3\sqrt{2\omega(\vb{p})}}
				\alpha(\vb{p})
				\hat{a}^\dagger(\vb{p})
			\right\}
			\\
			&\times
			\exp\left\{
				-
				\int\frac{\dd[3]{p}}{(2\pi)^3\sqrt{2\omega(\vb{p})}}
				\alpha(\vb{p})^*
				\hat{a}(\vb{p})
			\right\}
			\\
			&\times
			\exp\left\{
				-
				\frac{1}{2}
				\int\frac{\dd[3]{p}}{(2\pi)^32\omega(\vb{p})}
				\abs{\alpha(\vb{p})}^2
			\right\}
		\end{split}
		\label{eq:displacement_operator_normal}
	\end{equation}
	and the displacement operator in antinormal-order is
	\begin{equation}
		\begin{split}
			\hat{D}[\alpha]
			&=
			\exp\left\{
				-
				\int\frac{\dd[3]{p}}{(2\pi)^3\sqrt{2\omega(\vb{p})}}
				\alpha(\vb{p})^*
				\hat{a}(\vb{p})
			\right\}
			\\
			&\times
			\exp\left\{
				+
				\int\frac{\dd[3]{p}}{(2\pi)^3\sqrt{2\omega(\vb{p})}}
				\alpha(\vb{p})
				\hat{a}^\dagger(\vb{p})
			\right\}
			\\
			&\times
			\exp\left\{
				+
				\frac{1}{2}
				\int\frac{\dd[3]{p}}{(2\pi)^32\omega(\vb{p})}
				\abs{\alpha(\vb{p})}^2
			\right\}
		\end{split}
		\label{eq:displacement_operator_antinormal}
	\end{equation}
\end{lemma}
\begin{lemma}\label{thm:displacement_operator_simplified}
	The displacement operators can be expressed in terms of the smeared positive and negative frequency Klein-Gordon operators
	\begin{equation}
		\begin{split}
			\hat{D}[\alpha]
			&=
			\exp\left\{
				+\hat\phi^-[\alpha]
			\right\}
			\exp\left\{
				-\hat\phi^+[\alpha]
			\right\}
			\exp\left\{
				-
				\frac{1}{2}
				\comm{\hat\phi^+[\alpha]}{\hat\phi^-[\alpha]}
			\right\}
			\\
			&=
			\exp\left\{
				-\hat\phi^+[\alpha]
			\right\}
			\exp\left\{
				+\hat\phi^-[\alpha]
			\right\}
			\exp\left\{
				+
				\frac{1}{2}
				\comm{\hat\phi^+[\alpha]}{\hat\phi^-[\alpha]}
			\right\}
			\\
			&=
			\exp\left\{
				\hat\phi^-[\alpha]
				-
				\hat\phi^+[\alpha]
			\right\}
			.
		\end{split}
	\end{equation}
\end{lemma}
\begin{lemma}\label{thm:displacement_operator_product}
	Let $\hat{D}[\alpha],\hat{D}[\beta]$ be two displacement operators with spectrum $\alpha(\vb{p}),\beta(\vb{p})$, then their product equals
	\begin{equation}
		\begin{split}
			\hat{D}[\alpha]
			\hat{D}[\beta]
			&=
			\hat{D}[\alpha+\beta]
			\exp\left\{
				-
				\frac{1}{2}
				\comm{\hat\phi^+[\alpha]}{\hat\phi^-[\beta]}
				+
				\frac{1}{2}
				\comm{\hat\phi^+[\beta]}{\hat\phi^-[\alpha]}
			\right\}
			\\
			&=
			\hat{D}[\alpha+\beta]
			\exp\left\{
				-
				\frac{1}{2}
				\iint\frac{\dd[3]{p}}{(2\pi)^32\omega(\vb{p})}
				\left\{
					\alpha(\vb{p})^*
					\beta(\vb{p})
					-
					\alpha(\vb{p})
					\beta(\vb{p})^*
				\right\}
			\right\}
			.
		\end{split}
	\end{equation}
\end{lemma}
\begin{lemma}\label{thm:displacement_operator_unitary}
	The displacement operator is unitary
	\begin{equation}
		\hat{D}[\alpha]^{-1}
		=
		\hat{D}[\alpha]^\dagger
		=
		\hat{D}[-\alpha]
		.
	\end{equation}
\end{lemma}
\begin{definition}[Coherent state]
	A coherent state $\ket{\alpha}$ with spectrum $\alpha(\vb{p})$ is
	\begin{equation}
		\begin{split}
			\ket{\alpha}
			&=
			\exp\left\{
				-
				\frac{1}{2}
				\comm{\hat\phi^+[\alpha]}{\hat\phi^-[\alpha]}
			\right\}
			\exp\left\{
				+\hat\phi^-[\alpha]
			\right\}
			\ket{0}
			\\
			&=
			\exp\left\{
				-
				\frac{1}{2}
				\int\frac{\dd[3]{p}}{(2\pi)^32\omega(\vb{p})}
				\abs{\alpha(\vb{p})}^2
			\right\}
			\\
			&\times
			\exp\left\{
				\int\frac{\dd[3]{p}}{(2\pi)^3\sqrt{2\omega(\vb{p})}}
				\alpha(\vb{p})
				\hat{a}^\dagger(\vb{p})
			\right\}
			\ket{0}
		\end{split}
	\end{equation}
\end{definition}
\begin{lemma}
	The displacement operator creates a coherent state from the vacuum
	\begin{equation}
		\hat{D}[\alpha]
		\ket{0}
		=
		\ket{\alpha}
		.
	\end{equation}
\end{lemma}
\begin{corollary}
	The coherent state is normalized\footnote{In contrast to the number state, there is no constraint on the spectrum of the coherent state $\alpha(\vb{p})$.}
	\begin{equation}
		\braket{\alpha}
		=
		1
		.
	\end{equation}
\end{corollary}
\begin{theorem}\label{thm:coherent_state_annihilation_eigenvalue}
	The coherent state is an eigenstate of the annihilation operator to eigenvalue $\alpha(\vb{p})/\sqrt{2\omega(\vb{p})}$, i.e.,
	\begin{equation}
		\hat{a}(\vb{p})
		\ket{\alpha}
		=
		\frac{\alpha(\vb{p})}{\sqrt{2\omega(\vb{p})}}
		\ket{\alpha}
		.
	\end{equation}
\end{theorem}
\begin{lemma}\label{thm:coherent_state_energy_observable}
	The mean energy of the coherent state is
	\begin{equation}
		\expval{\hat{H}}{\alpha}
		=
		\int\frac{\dd[3]{p}}{(2\pi)^32\omega(\vb{p})}
		\omega(\vb{p})
		\abs{\alpha(\vb{p})}^2
	\end{equation}
	and the variance is
	\begin{equation}
		\expval{\left(\Delta\hat{H}\right)^2}{\alpha}
		=
		\int\frac{\dd[3]{p}}{(2\pi)^32\omega(\vb{p})}
		\omega(\vb{p})^2
		\abs{\alpha(\vb{p})}^2
		.
	\end{equation}
\end{lemma}
Although, the mean energy of the coherent state is the same as the mean energy of the single-particle number state, the spectrum $\alpha(\vb{p})$ of the coherent state is not bound.
\begin{lemma}\label{thm:coherent_state_number_observable}
	The mean number of particles is
	\begin{equation}
		\overline{n}
		=
		\expval{\hat{N}}{\alpha}
		=
		\int\frac{\dd[3]{p}}{(2\pi)^32\omega(\vb{p})}
		\abs{\alpha(\vb{p})}^2
	\end{equation}
	and the variance is
	\begin{equation}
		\expval{\left(\Delta\hat{N}\right)^2}{\alpha}
		=
		\int\frac{\dd[3]{p}}{(2\pi)^32\omega(\vb{p})}
		\abs{\alpha(\vb{p})}^2
		=
		\overline{n}
		,
	\end{equation}
	i.e., the particle number is Poisson distributed.
\end{lemma}
\begin{lemma}\label{thm:coherent_state_number_state_inner_product}
	The inner product between an $n$-particle number state with spectrum $f(\vb{p})$ and a coherent state with spectrum $\alpha(\vb{p})$ is
	\begin{equation}
		\braket{n_f}{\alpha}
		=
		\frac{1}{\sqrt{n!}}
		\left(
			\int\frac{\dd[3]{p}}{(2\pi)^32\omega(\vb{p})}
			f(\vb{p})^*
			\alpha(\vb{p})
		\right)^n
		e^{-\overline{n}/2}
	\end{equation}
	where $\overline{n}$ is the mean particle number of the coherent state.
\end{lemma}
\begin{corollary}
	If $\alpha(\vb{p})=f(\vb{p})$ the former inner product reduces to
	\begin{equation}
		\braket{n_f}{\alpha}
		=
		\frac{1}{\sqrt{n!}}
		e^{-\overline{n}/2}
	\end{equation}
	which absolute squared is a Poisson distribution with unit variance.
\end{corollary}
\begin{corollary}
	If the product of the spectrums is equal to
	\begin{equation}
		f(\vb{p})
		\alpha(\vb{p})
		=
		2\omega(\vb{p})
		(2\pi)^3
		\delta(\vb{p}-\vb{k})
	\end{equation}
	the inner product reduces to
	\begin{equation}
		\braket{n_f}{\alpha}
		=
		\frac{\alpha^n}{\sqrt{n!}}
		e^{-\overline{n}/2}
	\end{equation}
	and we recover the Poisson distribution known from single-mode quantum optics
	\begin{equation}
		p_n
		=
		\abs{\braket{n_f}{\alpha}}^2
		=
		\frac{\overline{n}^n}{\sqrt{n!}}
		e^{-\overline{n}}
	\end{equation}
	where $\overline{n}=\abs{\alpha}^2$.
\end{corollary}
\begin{lemma}\label{thm:coherent_state_inner_product}
	Let $\ket{\alpha},\ket{\beta}$ be two coherent states, then their inner product is
	\begin{equation}
		\begin{split}
			\braket{\alpha}{\beta}
			&=
			\exp\left\{
				-
				\frac{1}{2}
				\comm{\hat\phi^+[\alpha]}{\hat\phi^-[\alpha]}
				+
				\comm{\hat\phi^+[\alpha]}{\hat\phi^-[\beta]}
				-
				\frac{1}{2}
				\comm{\hat\phi^+[\beta]}{\hat\phi^-[\beta]}
			\right\}
			\\
			&=
			\exp\left\{
				-
				\frac{1}{2}
				\int\frac{\dd[3]{p}}{(2\pi)^32\omega(\vb{p})}
				\left\{
					\abs{\alpha(\vb{p})}^2
					+
					\abs{\beta(\vb{p})}^2
					-
					2\alpha(\vb{p})\beta(\vb{p})^*
				\right\}
			\right\}
		\end{split}
	\end{equation}
\end{lemma}
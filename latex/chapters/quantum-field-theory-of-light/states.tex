\section{Quantum states}

\subsection{Single-particle number states}

\begin{definition}[Single-particle number state]\label{def:single_particle_number_state}
	We define a number state of the Klein-Gordon field by
	\begin{equation}
		\ket{f}
		=
		\int\frac{\dd[3]{p}}{(2\pi)^3\sqrt{2\omega(\vb{p})}}
		f(\vb{p})
		\hat{a}^\dagger(\vb{p})
		\ket{0}
	\end{equation}
	where the Lorentz-invariant spectrum $f(\vb{p})=f(\omega(\vb{p}),\vb{p})\in\mathcal{S}(\mathbb{R}^4,\mathbb{C})$ encodes the spectral distribution of $\ket{f}$.
\end{definition}
\begin{lemma}
	The single-particle number state is equal to the smeared Klein-Gordon field operator acting on the vacuum state
	\begin{equation}
		\hat\phi[f]
		\ket{0}
		=
		\int\dd[4]{x}
		f(x)
		\hat\phi(x)
		\ket{0}
		=
		\ket{f}
	\end{equation}
\end{lemma}
\begin{proof}
	First, we note that the positive frequency Klein-Gordon operator vanishes,
	\begin{equation}
		\int\dd[4]{x}
		f(x)
		\hat\phi(x)
		\ket{0}
		=
		\int\dd[4]{x}
		f(x)
		\hat\phi^-(x)
		\ket{0}
		,
	\end{equation}
	because $\hat{a}(\vb{p})\ket{0}=0$, then we insert the definition of the negative frequency part \cref{eq:qkg_positive_negative_frequency}
	\begin{equation}
		\begin{split}
			\int\dd[4]{x}
			f(x)
			\hat\phi^-(x)
			\ket{0}
			&=
			\int\dd[4]{x}
			f(x)
			\int\frac{\dd[3]{p}}{(2\pi)^3\sqrt{2\omega(\vb{p})}}
			e^{+ip_\mu x^\mu}
			\hat{a}^\dagger(\vb{p})
			\ket{0}
			\\
			&=
			\int\frac{\dd[3]{p}}{(2\pi)^3\sqrt{2\omega(\vb{p})}}
			\left(
				\int\dd[4]{x}
				f(x)
				e^{+ip_\mu x^\mu}
			\right)
			\hat{a}^\dagger(\vb{p})
			\ket{0}
			\\
			&=
			\int\frac{\dd[3]{p}}{(2\pi)^3\sqrt{2\omega(\vb{p})}}
			f(\vb{p})
			\hat{a}^\dagger(\vb{p})
			\ket{0}
			=
			\ket{f}
		\end{split}
	\end{equation}
	and the last integral is the definition of the single-particle state.
\end{proof}
\begin{lemma}[Inner product of two single-particle number states]\label{th:single_particle_number_states_inner_product}
	Let $\ket{f},\ket{g}$ be two single-particle number states then their inner product is equal to
	\begin{equation}
		\braket{g}{f}
		=
		\int\frac{\dd[3]{p}}{(2\pi)^32\omega(\vb{p})}
		f(\vb{p})g(\vb{p})^*
	\end{equation}
	indicating that the single-particle number states are overcomplete.
\end{lemma}
\begin{proof}
	For the proof, we insert the definition and use the commutation relation of the annihilation and creation operators of the Klein-Gordon field
	\begin{equation}
		\begin{split}
			\bra{g}\ket{f}
			&=
			\int\frac{\dd[3]{p}}{(2\pi)^3\sqrt{2\omega(\vb{p})}}
			\int\frac{\dd[3]{q}}{(2\pi)^3\sqrt{2\omega(\vb{q})}}
			f(\vb{p})g(\vb{q})^*
			\expval{\hat{a}(\vb{q})\hat{a}^\dagger(\vb{p})}{0}
			\\
			&=
			\int\frac{\dd[3]{p}}{(2\pi)^3\sqrt{2\omega(\vb{p})}}
			\int\frac{\dd[3]{q}}{(2\pi)^3\sqrt{2\omega(\vb{q})}}
			f(\vb{p})g(\vb{q})^*
			(2\pi)^3\delta^{(3)}(\vb{q}-\vb{p})
			\\
			&=
			\int\frac{\dd[3]{p}}{(2\pi)^32\omega(\vb{p})}
			f(\vb{p})g(\vb{p})^*
		\end{split}
		.
	\end{equation}
\end{proof}
\begin{lemma}[Normalization of a single-particle number state spectrum]
	The spectrum of a single-particle number state satisfies
	\begin{equation}
		\bra{f}\ket{f}
		=
		\int\frac{\dd[3]{p}}{(2\pi)^32\omega(\vb{p})}
		\abs{f(\vb{p})}^2
		=
		1
		.
	\end{equation}
\end{lemma}
\begin{proof}
	The probabilistic interpretation of quantum mechanics requires $\braket{f}=1$ and \cref{th:single_particle_number_states_inner_product} with $\ket{g}=\ket{f}$ relates $\braket{f}=1$ with the integral.
\end{proof}
Let us consider some examples for the single-particle number state spectrum $f(\vb{p})$.
\begin{example}[Covariant Gaussian single-particle number state]
	The covariant Gaussian spectrum with mean $k^\mu=(k_0,\vb{k})$ and variance $\sigma^2$ is
	\begin{equation}
		f(\vb{p})
		\propto
		\exp\left\{
			\frac{(p_\mu-k_\mu)(p^\mu-k^\mu)}{4\sigma_P^2}
		\right\}
		=
		\exp\left\{
			-
			\frac{\omega(\vb{p})\omega(\vb{k})-\vb{k}\vdot\vb{p}}{2\sigma^2}
		\right\}
		\label{eq:covariant_gaussian_spectrum}
	\end{equation}
	according to Ref.~\cite{Naumov2013,Naumov2009}.
\end{example}
\begin{example}[Approximate massive Gaussian single-particle number state]
	According to Naumov~\cite{Naumov2013}, the spectrum of the covariant Gaussian number state can be approximated with
	\begin{equation}
		f(\vb{p})
		=
		\sqrt{2m}
		\left(
			\frac{2\pi}{\sigma^2}
		\right)^{3/4}
		\exp\left\{
			-
			\frac{(\vb{k}-\vb{p}_L)^2}{4\sigma_L^2}
			-
			\frac{\vb{p}_T^2}{4\sigma_T^2}
		\right\}
	\end{equation}
	for $\vb{p}\approx\vb{k}$, wherein $\vb{p}_T,\vb{p}_L$ are the momentum transversal and longitudinal to $\vb{k}$, and $\sigma_L=\omega(\vb{p})\sigma/m$ and $\sigma_T=\sigma_P$ are the longitudinal and transversal momentum spreads.
\end{example}
\begin{example}[Massless Gaussian single-particle number state]
	Evaluating \cref{eq:covariant_gaussian_spectrum} with $\omega(\vb{p})=\norm{\vb{p}}$ yields
	\begin{equation}
		f(\vb{p})
		\propto
		\exp\left\{
			-
			\frac{\norm{\vb{p}}\norm{\vb{k}}-\vb{k}\vdot\vb{p}}{2\sigma^2}
		\right\}
	\end{equation}
\end{example}
\begin{theorem}[Energy observable of the single-particle number state]\label{thm:single_particle_number_state_energy}
	Calculating the first two moments of the energy observables, \cref{qkg:energy}, with respect to a single-particle number state $\ket{f}$ yields
	\begin{align}
		\expval{\hat{H}}{f}
		&=
		\int\frac{\dd[3]{p}}{(2\pi)^32\omega(\vb{p})}
		\omega(\vb{p})
		\abs{f(\vb{p})}^2
		\\
		\expval{\hat{H}^2}{f}
		&=
		\int\frac{\dd[3]{p}}{(2\pi)^32\omega(\vb{p})}
		\omega(\vb{p})^2
		\abs{f(\vb{p})}^2
		\label{eq:single_particle_number_state_energy}
	\end{align}
	for the first two moments implying non-zero energy fluctuations
	\begin{equation}
		\expval{\left(\Delta\hat{H}\right)^2}{f}
		=
		\expval{\hat{H}^2}{f}
		-
		\expval{\hat{H}}{f}^2
		>
		0
	\end{equation}.
\end{theorem}
The proof is found in \proofref{the appendix}{thm:single_particle_number_state_energy}.

\begin{corollary}[Single-particle number state is eigenstate of the number operator]
	Performing the replacement $\omega(\vb{p})\to1$ in \cref{eq:single_particle_number_state_energy} and employing the normalization condition yields
	\begin{equation}
		\expval{\hat{N}}{f}
		=
		1
		=
		\expval{\hat{N}^2}{f}
	\end{equation}
	which leads to zero variance in the number observable, implying that $\ket{f}$ is a number eigenstate.
\end{corollary}

\begin{definition}[Coordinate wave function]
	We define the coordinate wave function of a state $\ket{\psi}$
	\begin{equation}
		\psi(t,\vb{x})
		=
		\bra{0}\hat\phi(t,\vb{x})\ket{\psi}
	\end{equation}
\end{definition}
\begin{lemma}[Coordinate wave function of single-particle number state]\label{thm:single_particle_number_state_wave_function}
	The coordinate wave function of a single-particle number state turns out to be
	\begin{equation}
		\psi(t,\vb{x})
		=
		\int\frac{\dd[3]{p}}{(2\pi)^32\omega(\vb{p})}
		f(\vb{p})e^{-ip_\mu x^\mu}
	\end{equation}
\end{lemma}
The proof is given in \proofref{the appendix}{thm:single_particle_number_state_wave_function}.
\begin{definition}[Probability current]
	The covariant probability current
	\begin{equation}
		j_\mu(t,\vb{x})
		=
		2
		\Im\left\{
			\psi(t,\vb{x})^*
			\partial_\mu
			\psi(t,\vb{x})
		\right\}
		\label{eq:qkg_probability_current}
	\end{equation}	
\end{definition}
The probability current is the conserved Noether's current of the Klein-Gordon field, see Ref.~\cite[p.~18]{Peskin1995}).
\begin{definition}[Group velocity]
	We refer to the spatial integration of the probability current $\vb{j}$
	\begin{equation}
		\expval{\vb{v}}
		=
		\int\dd[3]{x}
		\vb{j}(t,\vb{x})
		\label{eq:group_velocity}
	\end{equation}
	as the group velocity.
\end{definition}
\begin{lemma}[Group velocity of a single-particle number state]\label{thm:single_particle_number_state_group_velocity}
	The group velocity of a single-particle number state $\ket{f}$ is
	\begin{equation}
		\expval{\vb{v}}
		=
		\int\frac{\dd[3]{p}}{(2\pi)^32\omega(\vb{p})}
		\abs{f(\vb{p})}^2
		\frac{\vb{p}}{\omega(\vb{p})}
		\label{eq:single_particle_number_state_group_velocity}
	\end{equation}
\end{lemma}
\begin{definition}[Mean localization]
	We define the mean localization to be
	\begin{equation}
		\expval{\vb{x}(t)}
		=
		\int\dd[3]{x}
		\vb{x}
		\rho(t,\vb{x})
	\end{equation}
\end{definition}
\begin{example}[Mean localization of a single-particle number state]
	\begin{equation}
		\expval{\vb{x}(t)}
		=
		\expval{\vb{v}}t
	\end{equation}	
\end{example}
\begin{definition}[Spatial dispersion]
	The spatial dispersion is defined by
	\begin{equation}
		\sigma_x(t)^2
		=
		\expval{\vb{x}(t)^2}
		-
		\expval{\vb{x}(t)}^2		
	\end{equation}
\end{definition}
\begin{example}[Spatial dispersion for massive Gaussian single-particle number states]
	See Ref.~\cite{Naumov2013}.
\end{example}

\subsection{Coherent state}

\begin{definition}
	We define the displacement operator
	\begin{equation}
		\hat{D}[\alpha]
		=
		\exp\left\{
			\int\frac{\dd[3]{p}}{(2\pi)^3\sqrt{2\omega(\vb{p})}}
			\left\{
				\alpha(\vb{p})
				\hat{a}^\dagger(\vb{p})
				-
				\alpha(\vb{p})^*
				\hat{a}(\vb{p})
			\right\}
		\right\}
		\label{eq:displacement_operator}
	\end{equation}
	where $\alpha(\vb{p})$ is a spectral function.
\end{definition}
\begin{lemma}
	The displacement operator can be rewritten in normal-order and is equal to
	\begin{equation}
		\begin{split}
			\hat{D}[\alpha]
			&=
			\exp\left\{
				+
				\int\frac{\dd[3]{p}}{(2\pi)^3\sqrt{2\omega(\vb{p})}}
				\alpha(\vb{p})
				\hat{a}^\dagger(\vb{p})
			\right\}
			\\
			&\times
			\exp\left\{
				-
				\int\frac{\dd[3]{p}}{(2\pi)^3\sqrt{2\omega(\vb{p})}}
				\alpha(\vb{p})^*
				\hat{a}(\vb{p})
			\right\}
			\\
			&\times
			\exp\left\{
				-
				\frac{1}{2}
				\int\frac{\dd[3]{p}}{(2\pi)^32\omega(\vb{p})}
				\abs{\alpha(\vb{p})}^2
			\right\}
		\end{split}
		\label{eq:displacement_operator_normal}
	\end{equation}
\end{lemma}
\begin{proof}
	See Ref.~\cite[p.~48]{Barnett2002}.
\end{proof}
\begin{definition}[Coherent state]
	We define a coherent state $\ket{\alpha}$ with spectrum $\alpha(\vb{p})$ as
	\begin{equation}
		\begin{split}
			\ket{\alpha}
			&=
			\exp\left\{
				-
				\frac{1}{2}
				\int\frac{\dd[3]{p}}{(2\pi)^32\omega(\vb{p})}
				\abs{\alpha(\vb{p})}^2
			\right\}
			\\
			&\times
			\exp\left\{
				-i
				\int\frac{\dd[3]{p}}{(2\pi)^3\sqrt{2\omega(\vb{p})}}
				\alpha(\vb{p})
				\hat{a}^\dagger(\vb{p})
			\right\}
			\ket{0}
		\end{split}
	\end{equation}
\end{definition}
\begin{lemma}
	The displacement operator acting on the vacuum state is a coherent state, i.e.,
	\begin{equation}
		\hat{D}[\alpha]
		\ket{0}
		=
		\ket{\alpha}
	\end{equation}
\end{lemma}
\begin{theorem}[Coherent state is eigenstate of the annihilation operator]\label{thm:coherent_state_annihilation_eigenvalue}
	The coherent state is an eigenstate of the annihilation operator
	\begin{equation}
		\hat{a}(\vb{p})
		\ket{\alpha}
		=
		\frac{\alpha(\vb{p})}{\sqrt{2\omega(\vb{p})}}
		\ket{\alpha}
	\end{equation}
\end{theorem}
The proof is given in \proofref{the appendix}{thm:coherent_state_annihilation_eigenvalue}.

\begin{lemma}[Energy moments of the coherent state]\label{thm:coherent_state_energy_observable}
	\begin{align}
		\expval{\hat{H}}{\alpha}
		&=
		\int\frac{\dd[3]{p}}{(2\pi)^32\omega(\vb{p})}
		\omega(\vb{p})
		\abs{\alpha(\vb{p})}^2
		\\
		\expval{\hat{H}^2}{\alpha}
		&=
		\expval{\hat{H}}{\alpha}^2
	\end{align}	
\end{lemma}
The proof is given in \proofref{the appendix}{thm:coherent_state_energy_observable}.
The energy moments look very similar but in contrast to the single-particle number state spectrum $f(\vb{p})$ we have to keep in mind that there is no constraint on the coherent state spectrum $\alpha(\vb{p})$.
\begin{corollary}[Number moments of the coherent state]
	Performing the replacement $\omega(\vb{p})\to1$ in the energy expectation moments of the coherent state, we find the moments of the number observable to be
	\begin{align}
		\expval{\hat{N}}{\alpha}
		&=
		\int\frac{\dd[3]{p}}{(2\pi)^32\omega(\vb{p})}
		\abs{\alpha(\vb{p})}^2
		\\
		\expval{\hat{N}^2}{\alpha}
		&=
		\expval{\hat{N}}{\alpha}^2
	\end{align}
	.
\end{corollary}

\begin{lemma}
	The probability of detecting a single-particle given a coherent state is
	\begin{equation}
		\begin{split}
			\braket{f}{\alpha}
			&=
			\exp\left\{
				-
				\frac{1}{2}
				\int\frac{\dd[3]{p}}{(2\pi)^32\omega(\vb{p})}
				\abs{\alpha(\vb{p})}^2
			\right\}
			\int\frac{\dd[3]{p}}{(2\pi)^32\omega(\vb{p})}
			f(\vb{p})^*
			\alpha(\vb{p})
		\end{split}
	\end{equation}
	Assuming spectral overlap, i.e., $\alpha(\vb{p})=\alpha f(\vb{p})$, we recover the Poisson distribution
	\begin{equation}
		\begin{split}
			\braket{f}{\alpha}
			&=
			\alpha
			\exp\left\{
				-
				\frac{1}{2}
				\abs{\alpha}^2
			\right\}
		\end{split}
	\end{equation}
\end{lemma}

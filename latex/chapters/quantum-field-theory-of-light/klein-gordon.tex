\section{Klein-Gordon field}

\subsection{Relativistic field theory}

\begin{definition}[Klein-Gordon Lagrangian]
	The Lorentz-invariant Lagrangian density
	\begin{equation}
		\mathcal{L}
		=
		\frac{1}{2}
		\left(\partial_\mu\phi\right)
		\left(\partial^\mu\phi\right)
		-
		\frac{1}{2}
		m^2\phi^2
		\label{eq:kg_lagrangian}
	\end{equation}
	describes a real scalar field $\phi(t,\vb{x})$ with mass $m>0$, the (massive) Klein-Gordon field.
\end{definition}
\begin{theorem}[Relativistic energy-momentum relation]\label{th:relativistic_energy_momentum}
	Excitations of the Klein-Gordon field satisfy the relativistic energy-momentum relation
	\begin{equation}
		\omega(\vb{p})
		=
		\sqrt{\vb{p}^2+m^2}
		=
		E(\vb{p})
		\label{eq:energy_momentum_relation}
	\end{equation}
	and the physical momentum space is constrained to the momentum lightcone
	\begin{equation}
		V
		=
		\left\{
			(p_0,\vb{p})\in\mathbb{R}^4
			\mid
			p_0^2=\omega(\vb{p})^2
		\right\}
		\label{eq:momentum_lightcone}
		.
	\end{equation}
\end{theorem}
\begin{proof}
	According to the action principle, the dynamics of the field are determined by the equations of motion which can be found by the relativistic Euler-Lagrange equations
	\begin{equation*}
		0
		=
		\partial_\mu\pdv{\mathcal{L}}{(\partial_\mu\phi)}
		-
		\pdv{\mathcal{L}}{\phi}
		=
		\left(
			\partial_\mu\partial^\mu
			+
			m^2
		\right)
		\phi(t,\vb{x})
		.
	\end{equation*}
	Assuming the existence of the Klein-Gordon field's Fourier representation~\cite[p.~341]{Cohen2019}
	\begin{equation}
		\phi(t,\vb{x})
		=
		\int_{\mathbb{R}^3}\frac{\dd[3]{p}}{(2\pi)^3}
		\phi(t,\vb{p})
		e^{-i\vb{p}\vdot\vb{x}}
		=
		\int_{\mathbb{R}^4}\frac{\dd[4]{p}}{(2\pi)^4}
		\phi(p_0,\vb{p})
		e^{+ip_\mu x^\mu}
		,
	\end{equation}
	the equation of motion in momentum space reduces to
	\begin{equation}
		0
		=
		\left(
			ip_\mu ip^\mu
			+
			m^2
		\right)
		\phi(p_0,\vb{p})
		=
		-
		\left(
			p_0^2
			-
			\omega(\vb{p})^2
		\right)
		\phi(p_0,\vb{p})
	\end{equation}
	which is satisfied if $p_0=\pm\omega(\vb{p})$.
\end{proof}

\begin{lemma}[Fourier expansion of the Klein-Gordon field]
	The Fourier expansion of the Klein-Gordon field
	\begin{equation}
		\phi(t,\vb{x})
		=
		\int_{\mathbb{R}^3}\frac{\dd[3]{p}}{(2\pi)^3\sqrt{2\omega(\vb{p})}}
		\biggl\{
			a(\vb{p})^*
			e^{-ip_\mu x^\mu}
			+
			a(\vb{p})
			e^{+ip_\mu x^\mu}
		\biggr\}_{p_0=\omega(\vb{p})}
	\end{equation}
	satisfies the equations of motion for any choice of $a(\vb{p})=\phi(\omega(\vb{p}),\vb{p})^*$.
\end{lemma}

\begin{remark}
	From now on assume $p_0=\omega(\vb{p})$ if not stated otherwise.
\end{remark}

\begin{corollary}[Conjugate momentum density of the Klein-Gordon field]
	The conjugate momentum density of the Klein-Gordon field is
	\begin{equation}
		\pi(t,\vb{x})
		=
		\int\frac{\dd[3]{p}}{(2\pi)^3}
		\left(-i\sqrt{2\omega(\vb{p})}\right)
		\left\{
			a(\vb{p})
			e^{-ip_\mu x^\mu}
			-
			a(\vb{p})^*
			e^{+ip_\mu x^\mu}
		\right\}
	\end{equation}
	which follows directly from the Legendre transformation
	\begin{equation}
		\pi(t,\vb{x})
		=
		\partial_t\pdv{\mathcal{L}}{(\partial_t\phi)}
		=
		\partial_t\phi(t,\vb{x})
	\end{equation}
	fundamental to Hamiltonian mechanics.
\end{corollary}

\begin{definition}[Energy-momentum tensor]
	We define the energy-momentum tensor
	\begin{equation}
		T^{\mu\nu}
		=
		\pdv{\mathcal{L}}{(\partial_\mu\phi)}\partial^\nu\phi
		-
		g^{\mu\nu}\mathcal{L}
		\label{eq:energy_momentum_tensor}
	\end{equation}
	which encodes the field's observables in its components.
	For instance, the energy density is encoded in the
	\begin{equation*}
		T^{00}
		=
		\frac{1}{2}
		\left(\partial_t\phi\right)^2
		+
		\frac{1}{2}
		\left(\grad\phi\right)^2
		+
		\frac{1}{2}
		\left(m\phi\right)^2		
	\end{equation*}
	component and the momentum density in the
	\begin{equation*}
		T^{0i}
		=
		-\pi\partial_i\phi		
	\end{equation*}
	component, see Ref.~\cite{Peskin1995}.
\end{definition}

\begin{lemma}
	The total energy and momentum observables of the Klein-Gordon field are
	\begin{align}
		H
		=
		\int\frac{\dd[3]{p}}{(2\pi)^3}
		\omega(\vb{p})\abs{a(\vb{p})}^2
		&&
		\vb{P}
		=
		\int\frac{\dd[3]{p}}{(2\pi)^3}
		\vb{p}\abs{a(\vb{p})}^2
	\end{align}
\end{lemma}
\begin{proof}
	Perform a spatial integration over the energy and momentum densities.
\end{proof}

\subsection{Canonical quantization}

In the canonical quantization procedure, we promote the dynamical field variables to operators
\begin{align}
	\hat\phi(t,\vb{x})
	&=
	\int\frac{\dd[3]{p}}{(2\pi)^3}
	\frac{1}{\sqrt{2\omega(\vb{p})}}
	\left\{
		\hat{a}(\vb{p})
		e^{-ip_\mu x^\mu}
		+
		\hat{a}^\dagger(\vb{p})
		e^{+ip_\mu x^\mu}
	\right\}
	\label{eq:qkg_pos}
	\\
	\hat\pi(t,\vb{x})
	&=
	\int\frac{\dd[3]{p}}{(2\pi)^3}
	\left(-i\sqrt{2\omega(\vb{p})}\right)
	\left\{
		\hat{a}(\vb{p})
		e^{-ip_\mu x^\mu}
		-
		\hat{a}^\dagger(\vb{p})
		e^{+ip_\mu x^\mu}
	\right\}
	\label{eq:qkg_mom}
\end{align}
satisfying the canonical commutation relations
\begin{align}
	\comm{\hat\phi(\vb{x})}{\hat\pi(\vb{y})}
	=
	i\delta^{(3)}(\vb{x}-\vb{y})
	&&
	\comm{\hat\phi(\vb{x})}{\hat\phi(\vb{y})}
	=
	0
	=
	\comm{\hat\pi(\vb{x})}{\hat\pi(\vb{y})}
	\label{eq:qkg_comm_pm}	
\end{align}
which encode the uncertainty relation.
Inserting \cref{eq:qkg_pos} and \cref{eq:qkg_mom} into \cref{eq:qkg_comm_pm} reveals the commutation relations for the Fourier mode operators
\begin{align}
	\comm{\hat{a}(\vb{p})}{\hat{a}^\dagger(\vb{q})}
	=
	(2\pi)^3\delta^{(3)}(\vb{p}-\vb{q})
	&&
	\comm{\hat{a}(\vb{p})}{\hat{a}(\vb{q})}
	=
	0
	=
	\comm{\hat{a}^\dagger(\vb{p})}{\hat{a}^\dagger(\vb{q})}
	\label{eq:kg_comm_ac}.
\end{align}
Extending the corresponding principle by normal-ordering and applying it to the classical energy and momentum observables, \cref{eq:kg_total_energy} and \cref{eq:kg_total_momentum}, we find
\begin{align}
	\hat{H}
	=
	\int\frac{\dd[3]{p}}{(2\pi)^3}
	\omega(\vb{p})\hat{a}^\dagger(\vb{p})\hat{a}(\vb{p})
	&&
	\hat{\vb{P}}
	=
	\int\frac{\dd[3]{p}}{(2\pi)^3}
	\vb{p}\hat{a}^\dagger(\vb{p})\hat{a}(\vb{p})
	\label{eq:qkg_total_energy_momentum}.
\end{align}
In analogy with the quantum harmonic oscillator, we define the number operator to be
\begin{equation}
	\hat{N}
	=
	\int\frac{\dd[3]{p}}{(2\pi)^3}
	\hat{a}^\dagger(\vb{p})
	\hat{a}(\vb{p})
	\label{eq:qkg_number}.
\end{equation}
Observables always need to be normal-ordered, hence, higher moments are given by
\begin{equation}
	\hat{N}^2
	=
	\int\frac{\dd[3]{p_1}}{(2\pi)^3}
	\int\frac{\dd[3]{p_2}}{(2\pi)^3}
	\hat{a}^\dagger(\vb{p}_1)
	\hat{a}^\dagger(\vb{p}_2)
	\hat{a}(\vb{p}_1)
	\hat{a}(\vb{p}_2)
\end{equation}

It will be useful to decompose the Klein-Gordon field into a positive and negative frequency part
\begin{equation}
	\hat\phi(t,\vb{x})
	=
	\hat\phi^+(t,\vb{x})
	+
	\hat\phi^+(t,\vb{x})
\end{equation}
where the positive and negative frequency parts are defined as~\cite[p.~26]{Peskin1995}
\begin{align}
	\hat\phi^+(t,\vb{x})
	&=
	\int\frac{\dd[3]{p}}{(2\pi)^3\sqrt{2\omega(\vb{p})}}
	e^{-ip_\mu x^\mu}
	\hat{a}(\vb{p})
	\\
	\hat\phi^-(t,\vb{x})
	&=
	\int\frac{\dd[3]{p}}{(2\pi)^3\sqrt{2\omega(\vb{p})}}
	e^{+ip_\mu x^\mu}
	\hat{a}^\dagger(\vb{p})
\end{align}
and both parts are related by the hermitian conjugate $\hat\phi^-(t,\vb{x})=\hat\phi^+(t,\vb{x})^\dagger$.

So far, we have not considered any states.
The most important state is the vacuum state $\ket{0}$ which is axiomatically defined to be invariant under spacetime translations~\cite[p.~276]{Bogolubov1989}
\begin{equation}
	e^{i\hat{P}_\mu a^\mu}
	\ket{0}
	=
	\ket{0}
\end{equation}
suggesting that the vacuum state is energy and momentum eigenstate with respective zero eigenvalue $\hat{P}_\mu\ket{0}=0$.
From the vacuum state any other state can be constructed by applying the appropriate operators to it.

The first and second moment of the momentum operator
\begin{align}
	\expval{\hat{a}(\vb{p})\hat{\vb{P}}\hat{a}^\dagger(\vb{p})}{0}
	\propto
	\vb{p}
	&&
	\expval{\hat{a}(\vb{p})\hat{\vb{P}}^2\hat{a}^\dagger(\vb{p})}{0}
	\propto
	\vb{p}^2
\end{align}
where we only used the commutation relations, suggest that $\hat{a}^\dagger(\vb{p})\ket{0}$ is a momentum eigenstate.
The expectation value
\begin{equation}
	\expval{\hat{\vb{P}}}{\vb{p}_1,\vb{p}_2}
	\propto
	\vb{p}_1+\vb{p}_2
\end{equation}
where we defined
\begin{equation}
	\ket{\vb{p}_1,\vb{p}_2}
	\propto
	\hat{a}^\dagger(\vb{p}_1)
	\hat{a}^\dagger(\vb{p}_2)
	\ket{0}
	\propto
	\ket{\vb{p}_2,\vb{p}_1}
\end{equation}
suggests that $\hat{a}^\dagger(\vb{p})$ acting on a state to the right, adds an excitation with momentum $\vb{p}$ to the field.

From $\hat{a}(\vb{p})\ket{0}=0$, we find that $\hat\phi^+(t,\vb{x})\ket{0}=0$ and thus
\begin{equation}
	\hat\phi(t,\vb{x})
	\ket{0}
	=
	\hat\phi^-(t,\vb{x})
	\ket{0}
	=
	\int\frac{\dd[3]{p}}{(2\pi)^3\sqrt{2\omega(\vb{p})}}
	e^{+ip_\mu x^\mu}
	\hat{a}^\dagger(\vb{p})
	\ket{0}
\end{equation}
which resembles the Fourier transform of of the momentum eigenstate.
In that sense, $\hat\phi$ respective $\hat\phi^-$ creates a particle ideally localized at the spacetime event $(t,\vb{x})$.

One critical problem of the momentum eigenstates is that they are not normalizable
\begin{equation}
	\bra{\vb{p}}\ket{\vb{p}}
	=
	\expval{\hat{a}(\vb{p})\hat{a}^\dagger(\vb{p})}{0}
	\propto
	\delta^{(3)}(0)
\end{equation}
because $\delta^{(3)}(0)$ has no consistent mathematical definition.
Actually, the coordinate wave function of the momentum space~\cite[p.~25]{Peskin1995}
\begin{equation}
	\begin{split}
		\psi(t,\vb{x})
		=
		\bra{0}
		\hat\phi(t,\vb{x})
		\hat{a}^\dagger(\vb{p})
		\ket{0}
		&=
		\left(
			\int\frac{\dd[3]{q}}{(2\pi)^3\sqrt{2\omega(\vb{q})}}
			e^{+iq_\mu x^\mu}
			\hat{a}^\dagger(\vb{q})
			\ket{0}
		\right)^\dagger
		\hat{a}^\dagger(\vb{p})
		\ket{0}
		\\
		&=
		\int\frac{\dd[3]{q}}{(2\pi)^3\sqrt{2\omega(\vb{q})}}
		e^{-iq_\mu x^\mu}
		\expval{\hat{a}(\vb{q})\hat{a}^\dagger(\vb{p})}{0}
		\\
		&=
		\int\frac{\dd[3]{q}}{(2\pi)^3\sqrt{2\omega(\vb{q})}}
		e^{-iq_\mu x^\mu}
		(2\pi)^3
		\delta^{(3)}(\vb{q}-\vb{p})
		\\
		&\propto
		e^{-ip_\mu x^\mu}
	\end{split}
\end{equation}
is a plane-wave which is not square-integrable and therefore is not even an element of the Hilbert space.

\subsection{Smearing and single-particle states}

\subsubsection{Wightman axioms and smearing functions}

According to Wightman quantum field theory~\cite[p.~324]{Bogolubov1989}, a quantum field $\hat\Phi$ is an operator-valued tempered distribution which needs be combined with some Lorentz-invariant Schwartz function $f\in\mathcal{S}(\mathbb{R}^4,\mathbb{K})$
\begin{equation}
	\hat\Phi[f]
	=
	\int\dd[4]{x}
	f(t,\vb{x})
	\hat\Phi(t,\vb{x})
\end{equation}
to avoid ambiguities arising from the idealized non-physical ultra-localized momentum states which we would recover for
\begin{equation}
	\hat\Phi[\delta_x]
	=
	\int\dd[4]{y}
	\delta^{(1)}(y^0-x^0)
	\delta^{(3)}(\vb{y}-\vb{x})
	\hat\Phi(y^0,\vb{y})
	=
	\hat\Phi(x^0,\vb{x})
	.
\end{equation}

\subsubsection{Smeared single-particle state and overlap}

Smearing the Klein-Gordon field operator $\hat\phi^-$ acting on the vacuum state $\ket{0}$ with $f\in\mathcal{S}(\mathbb{R}^4,\mathbb{R})$ and defining $f(\vb{p})=f(\omega(\vb{p}),\vb{p})$
\begin{align}
	\hat\phi[f]
	\ket{0}
	&=
	\int\dd[4]{x}
	f(t,\vb{x})
	\hat\phi^-(t,\vb{x})
	\ket{0}
	\\
	&=
	\int\frac{\dd[3]{p}}{(2\pi)^3\sqrt{2\omega(\vb{p})}}
	\left(
		\int\dd[4]{x}
		f(t,\vb{x})
		e^{+ip_\mu x^\mu}
	\right)
	\hat{a}^\dagger(\vb{p})
	\ket{0}
	\\
	&=
	\int\frac{\dd[3]{p}}{(2\pi)^3\sqrt{2\omega(\vb{p})}}
	f(\vb{p})
	\hat{a}^\dagger(\vb{p})
	\ket{0}
\end{align}
and extrapolating from the known action of $\hat\phi$ on the vacuum, suggests
\begin{equation}
	\ket{f}
	=
	\int\frac{\dd[3]{p}}{(2\pi)^3\sqrt{2\omega(\vb{p})}}
	f(\vb{p})
	\hat{a}^\dagger(\vb{p})
	\ket{0}
\end{equation}
to represent a smeared particle state with localization in spacetime $f(t,\vb{x})$ respective momentum space $f(\vb{p})$.
The probability amplitude to measure $\ket{g}$ given $\ket{f}$ was prepared is equal to the overlap of the smearing functions
\begin{equation}
	\begin{split}
		\bra{g}\ket{f}
		&=
		\int\frac{\dd[3]{p}}{(2\pi)^3\sqrt{2\omega(\vb{p})}}
		\int\frac{\dd[3]{q}}{(2\pi)^3\sqrt{2\omega(\vb{q})}}
		f(\vb{p})g(\vb{q})^*
		\expval{\hat{a}(\vb{q})\hat{a}^\dagger(\vb{p})}{0}
		\\
		&=
		\int\frac{\dd[3]{p}}{(2\pi)^3\sqrt{2\omega(\vb{p})}}
		\int\frac{\dd[3]{q}}{(2\pi)^3\sqrt{2\omega(\vb{q})}}
		f(\vb{p})g(\vb{q})^*
		(2\pi)^3\delta^{(3)}(\vb{q}-\vb{p})
		\\
		&=
		\int\frac{\dd[3]{p}}{(2\pi)^32\omega(\vb{p})}
		f(\vb{p})g(\vb{p})^*
	\end{split}
\end{equation}
and suggests the normalization to be
\begin{equation}
	\bra{f}\ket{f}
	=
	\int\frac{\dd[3]{p}}{(2\pi)^32\omega(\vb{p})}
	\abs{f(\vb{p})}^2
	=
	1
\end{equation}
Alternatively, in coordinate space
\begin{equation}
	\begin{split}
		\bra{g}\ket{f}
		=
		\expval{\hat\phi[g]\hat\phi[f]}{0}
		&=
		\int\dd[4]{x}
		\int\dd[4]{y}
		f(x^0,\vb{x})g(y^0,\vb{y})^*
		\expval{\hat\phi(x^0,\vb{x})\hat\phi(y^0,\vb{y})}{0}
		\\
		&=
		\int\dd[4]{x}
		\int\dd[4]{y}
		f(x^0,\vb{x})g(y^0,\vb{y})^*
		D(x^0-y^0,\vb{x}-\vb{y})
	\end{split}
\end{equation}
where $D(x^0-y^0,\vb{x}-\vb{y})$ is the Feynman propagator~\cite[p.~27]{Peskin1995}.

\subsubsection{Number, energy, and momentum expectation values}

The first two moments of the number operator are
\begin{align}
	\expval{\hat{N}}{f}
	&=
	\int\frac{\dd[3]{p}}{(2\pi)^32\omega(\vb{p})}
	\abs{f(\vb{p})}^2
	=
	1
	\\
	\expval{\hat{N}^2}{f}
	&=
	\int\frac{\dd[3]{p}}{(2\pi)^32\omega(\vb{p})}
	\abs{f(\vb{p})}^2
	=
	1
\end{align}
suggesting $\ket{f}$ to be a single-particle number state because $\expval{\left(\Delta\hat{N}\right)}{f}=0$
Similar the first two moments of the total energy are
\begin{align}
	\expval{\hat{H}}{f}
	&=
	\int\frac{\dd[3]{p}}{(2\pi)^32\omega(\vb{p})}
	\omega(\vb{p})
	\abs{f(\vb{p})}^2
	\\
	\expval{\hat{H}^2}{f}
	&=
	\int\frac{\dd[3]{p}}{(2\pi)^32\omega(\vb{p})}
	\omega(\vb{p})^2
	\abs{f(\vb{p})}^2
\end{align}
implying non-zero energy fluctuations $\expval{\left(\Delta\hat{H}\right)^2}{f}>0$.
By an completely analog calculation, we find for the total momentum
\begin{align}
	\expval{\hat{\vb{P}}}{f}
	&=
	\int\frac{\dd[3]{p}}{(2\pi)^32\omega(\vb{p})}
	\vb{p}
	\abs{f(\vb{p})}^2
	\\
	\expval{\hat{\vb{P}}^2}{f}
	&=
	\int\frac{\dd[3]{p}}{(2\pi)^32\omega(\vb{p})}
	\vb{p}^2
	\abs{f(\vb{p})}^2
\end{align}
in agreement with results reported in~\cite[eqs.~10 and 11]{Naumov2013}.

\subsubsection{Group velocity, mean position and spatial dispersion}

From the spacetime representation of the wave function
\begin{equation}
	\psi(t,\vb{x})
	=
	\bra{0}\hat\phi(t,\vb{x})\ket{f}
	=
	\int\frac{\dd[3]{p}}{(2\pi)^32\omega(\vb{p})}
	f(\vb{p})e^{-ip_\mu x^\mu}
\end{equation}
we can calculate the probability current being the conserved Noether's current of the Klein-Gordon field~\cite[p.~18]{Peskin1995})
\begin{equation}
	j_\mu(t,\vb{x})
	=
	2
	\Im\left\{
		\psi(t,\vb{x})^*
		\partial_\mu
		\psi(t,\vb{x})
	\right\}
	\label{eq:qkg_probability_current}.
\end{equation}
Spatial integration of the probability current yields the group velocity of $\ket{f}$
\begin{equation}
	\expval{\vb{v}}
	=
	\int\dd[3]{x}
	\vb{j}(t,\vb{x})
	=
	\int\frac{\dd[3]{p}}{(2\pi)^32\omega(\vb{p})}
	\abs{f(\vb{p})}^2
	\frac{\vb{p}}{\omega(\vb{p})}
	\label{qkg:group_velocity}.
\end{equation}
Indeed in the non-relativistic regime $\omega(\vb{p})^2\sim m^2$
and the group velocity equals $\vb{p}/(2m^2)$ weighted by $\abs{f(\vb{p})}^2$.
The probability center-of-mass position is
\begin{equation}
	\expval{\vb{x}(t)}
	=
	\int\dd[3]{x}\vb{x}\rho(t,\vb{x})
	=
	\expval{\vb{v}}t
	,
\end{equation}
with the variance quantifying the spatial dispersion in time
\begin{equation}
	\sigma_x(t)^2
	=
	\expval{\vb{x}(t)^2}
	-
	\expval{\vb{x}(t)}^2
\end{equation}
which can be found in Naumov~\cite{Naumov2013}.

\subsubsection{Example of a massless Gaussian single-particle state}

Naumov~\cite{Naumov2013} discusses a massive Gaussian wave packets.

Let $k_\mu=(k_0,\vb{k})$ be a constant four-momentum vector with $k_0=\omega(\vb{k})$ (?), then the function
\begin{equation}
	f(\vb{p})
	\propto
	\exp\left\{
		\frac{(p_\mu-k_\mu)(p^\mu-k^\mu)}{4\sigma_p^2}
	\right\}
	=
	\exp\left\{
		\frac{p_\mu p^\mu-2k_\mu p^\mu+k_\mu k^\mu}{4\sigma_p^2}
	\right\}
\end{equation}
is an element of the Schwartz space $\mathcal{S}(\mathbb{R}^4,\mathbb{R})$ and is as a function of Lorentz scalars manifest Lorentz invariant, thus, a candidate for a smearing function.

For a massless Klein-Gordon field the dispersion relation is $\omega(\vb{p})=\norm{\vb{p}}$ and the smearing function simplifies to
\begin{equation}
	\begin{split}
		f(\vb{p})
		&\propto
		\exp\left\{
			-
			\frac{1}{2\sigma_p^2}
			\left(
				k_0
				\norm{\vb{p}}
				-
				\vb{k}\vdot\vb{p}
			\right)
		\right\}
		\\
		&=
		\exp\left\{
			-
			\frac{k_0}{2\sigma_p^2}
			\norm{\vb{p}}
			\left(
				1	
				-
				\cos\theta
			\right)
		\right\}
	\end{split}
\end{equation}

\subsection{Interactions and coherent states}

The coherent state can be found by treating the full Lagrangian with an external source~\cite{Peskin1995} or through the $\hat{S}$ operator formalism which has the advantage of extending to other interactions.
The $\hat{S}$ operator is defined by
\begin{equation}
	\hat{S}
	=
	T\exp\left\{
		-i
		\int\dd[4]{x}
		\mathcal{L}_\text{int}
	\right\}
\end{equation}
where $\mathcal{L}_\text{int}$ is the interaction term of the Lagrangian and $T$ is the time-ordering symbol.
For a classical source, we find
\begin{equation}
	\begin{split}
		\hat{S}
		&=
		T\exp\left\{
			-i
			\int\dd[4]{x}
			J(t,\vb{x})
			\hat\phi(t,\vb{x})
		\right\}
		\\
		&=
		\exp\left\{
			-i
			\int\dd[4]{x}
			J(t,\vb{x})
			\hat\phi(t,\vb{x})
			-i\theta
		\right\}
	\end{split}
\end{equation}
where the second equal is motived in Ref.~\cite[p.~180]{Itzykson2012} for the Maxwell field and
\begin{equation}
	\theta
	=
	\frac{1}{2}
	\int\dd[4]{x}\dd[4]{y}
	J(x^0,\vb{x})
	D_\text{ret}(x^0-y^0,\vb{x}-\vb{y})
	J(y^0,\vb{y})
\end{equation}
where $D_\text{ret}$ is the retarded propagator of the Klein-Gordon field.
We are going to neglect the phase for now.
We expand the field into the positive and negative frequency part and apply the BCH formula to obtain the normal-ordered form (up to a phase factor)
\begin{equation}
	\begin{split}
		\hat{S}
		&=
		\exp\left\{
			-i
			\int\dd[4]{x}
			\hat\phi^-(t,\vb{x})
			J(t,\vb{x})
			-i
			\int\dd[4]{x}
			\hat\phi^+(t,\vb{x})
			J(t,\vb{x})
		\right\}
		\\
		&=
		\exp\left\{
			\frac{1}{2}
			\iint\dd[4]{x}\dd[4]{y}
			\comm{\hat\phi^-(x)J(x)}{\hat\phi^+(y)J(y)}
		\right\}
		\\
		&\times
		\exp\left\{
			-i
			\int\dd[4]{x}
			\hat\phi^-(x)
			J(x)
		\right\}
		\exp\left\{
			-i
			\int\dd[4]{y}
			\hat\phi^+(y)
			J(y)
		\right\}
	\end{split}
\end{equation}
We rewrite
\begin{equation}
	\begin{split}
		&
		\frac{1}{2}
		\iint\dd[4]{x}\dd[4]{y}
		\comm{\hat\phi^-(x)J(x)}{\hat\phi^+(y)J(y)}
		\\
		=&\
		\frac{1}{2}
		\iint\dd[4]{x}\dd[4]{y}
		J(x)
		\comm{\hat\phi^-(x)}{\hat\phi^+(y)}
		J(y)
		\\
		=&\
		\frac{1}{2}
		\iint\dd[4]{x}\dd[4]{y}
		J(x)
		D(x-y)
		J(y)
		\\
		=&\
		-
		\frac{1}{2}
		\int\frac{\dd[3]{p}}{(2\pi)^32\omega(\vb{p})}
		\abs{J(\vb{p})}^2
	\end{split}
\end{equation}
check factors!
and $\int\dd[4]{x}\hat\phi^-(x)J(x)=\hat\phi^-[J]$ which looks similar to the smeared particle creation operator.

\subsubsection{Displacement operator}

We then identify the $S$-matrix operator as the displacement operator
\begin{equation}
	\begin{split}
		\hat{D}[\alpha]
		&=
		\exp\left\{
			-
			\frac{1}{2}
			\int\frac{\dd[3]{p}}{(2\pi)^32\omega(\vb{p})}
			\abs{\alpha(\vb{p})}^2
		\right\}
		\\
		&\times
		\exp\left\{
			-i
			\int\frac{\dd[3]{p}}{(2\pi)^3\sqrt{2\omega(\vb{p})}}
			\alpha(\vb{p})
			\hat{a}^\dagger(\vb{p})
		\right\}
		\\
		&\times
		\exp\left\{
			-i
			\int\frac{\dd[3]{p}}{(2\pi)^3\sqrt{2\omega(\vb{p})}}
			\alpha(\vb{p})^*
			\hat{a}(\vb{p})
		\right\}
	\end{split}
\end{equation}
which when acting on the vacuum state, creates a coherent state with spectrum $\alpha(\vb{p})$
\begin{equation}
	\begin{split}
		\ket{\alpha}
		&=
		\hat{D}[J]
		\ket{0}
		\\
		&=
		\exp\left\{
			-
			\frac{1}{2}
			\int\frac{\dd[3]{p}}{(2\pi)^32\omega(\vb{p})}
			\abs{\alpha(\vb{p})}^2
		\right\}
		\\
		&\times
		\exp\left\{
			-i
			\int\frac{\dd[3]{p}}{(2\pi)^3\sqrt{2\omega(\vb{p})}}
			\alpha(\vb{p})
			\hat{a}^\dagger(\vb{p})
		\right\}
		\ket{0}
	\end{split}
\end{equation}

\subsubsection{Number, energy, and momentum}

The coherent state is eigenstate of the annihilation operator
\begin{equation}
	\hat{a}(\vb{p})
	\ket{\alpha}
	=
	\frac{\alpha(\vb{p})}{\sqrt{2\omega(\vb{p})}}
	\ket{\alpha}
\end{equation}

The first two moments of the energy operator are exactly the same as for the sidle-particle state
\begin{align}
	\expval{\hat{H}}{\alpha}
	&=
	\int\frac{\dd[3]{p}}{(2\pi)^32\omega(\vb{p})}
	\omega(\vb{p})
	\abs{\alpha(\vb{p})}^2
	\\
	\expval{\hat{H}^2}{\alpha}
	&=
	\int\frac{\dd[3]{p}}{(2\pi)^32\omega(\vb{p})}
	\omega(\vb{p})^2
	\abs{\alpha(\vb{p})}^2
\end{align}
which is similar to our single-particle case with the big difference that is $\alpha(\vb{p})$ not normalized.
The same follows for the momentum moments
\begin{align}
	\expval{\hat{\vb{P}}}{\alpha}
	&=
	\int\frac{\dd[3]{p}}{(2\pi)^32\omega(\vb{p})}
	\vb{p}
	\abs{\alpha(\vb{p})}^2
	\\
	\expval{\hat{\vb{P}}^2}{\alpha}
	&=
	\int\frac{\dd[3]{p}}{(2\pi)^32\omega(\vb{p})}
	\vb{p}^2
	\abs{\alpha(\vb{p})}^2
\end{align}
and the number moments
\begin{align}
	\expval{\hat{N}}{\alpha}
	&=
	\int\frac{\dd[3]{p}}{(2\pi)^32\omega(\vb{p})}
	\abs{\alpha(\vb{p})}^2
	\\
	\expval{\hat{N}^2}{\alpha}
	&=
	\int\frac{\dd[3]{p}}{(2\pi)^32\omega(\vb{p})}
	\abs{\alpha(\vb{p})}^2
\end{align}
the variance of the number operator

The probability of detecting a single-particle given a coherent state is
\begin{equation}
	\begin{split}
		\braket{f}{\alpha}
		&=
		\exp\left\{
			-
			\frac{1}{2}
			\int\frac{\dd[3]{p}}{(2\pi)^32\omega(\vb{p})}
			\abs{\alpha(\vb{p})}^2
		\right\}
		\int\frac{\dd[3]{p}}{(2\pi)^32\omega(\vb{p})}
		f(\vb{p})^*
		\alpha(\vb{p})
	\end{split}
\end{equation}
Assuming spectral overlap, i.e., $\alpha(\vb{p})=\alpha f(\vb{p})$, we recover the Poisson distribution
\begin{equation}
	\begin{split}
		\braket{f}{\alpha}
		&=
		\alpha
		\exp\left\{
			-
			\frac{1}{2}
			\abs{\alpha}^2
		\right\}
	\end{split}
\end{equation}
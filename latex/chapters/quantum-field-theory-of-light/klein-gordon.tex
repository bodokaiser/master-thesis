\section{Klein-Gordon field}

\subsection{Relativistic field theory}

The most simple non-trivial, free relativistic Lagrangian density of a real-valued scalar field $\phi$ to be constructed is
\begin{equation}
	\mathcal{L}
	=
	\frac{1}{2}
	\left(\partial_\mu\phi\right)
	\left(\partial^\mu\phi\right)
	-
	\frac{1}{2}
	m^2\phi^2
	\label{eq:kg_lagrangian}.
\end{equation}
The field equation of motion are found by the relativistic Euler-Lagrange equations
\begin{equation}
	0
	=
	\partial_\mu\pdv{\mathcal{L}}{(\partial_\mu\phi)}
	-
	\pdv{\mathcal{L}}{\phi}
	=
	\left(\partial_\mu\partial^\mu+m^2\right)
	\phi(t,\vb{x})
	\label{eq:kg_eom}
\end{equation}
which minimize the action.
Assuming the existence of the Fourier transform of the field
\begin{equation}
	\phi(t,\vb{x})
	=
	\int_{\mathbb{R}^3}\frac{\dd[3]{p}}{(2\pi)^3}
	\phi(t,\vb{p})e^{+i\vb{p}\vdot\vb{x}}
	=
	\int_{\mathbb{R}^4}\frac{\dd[4]{p}}{(2\pi)^4}
	\phi(p_0,\vb{p})e^{-ip_\mu x^\mu}
	\label{eq:kg_ft_naive},
\end{equation}
and inserting the Fourier transform into the equation of motion reduces \cref{eq:kg_eom} to
\begin{equation}
	\begin{split}
		0
		&=
		(ip_\mu ip^\mu+m^2)\phi(p_0,\vb{p})
		\\
		&=
		-(p_\mu p^\mu-m^2)\phi(p_0,\vb{p})
		\\
		&=
		-(p_0^2-\vb{p}^2-m^2)\phi(p_0,\vb{p})
		\\
		&=
		-\left(p_0^2-\omega(\vb{p})^2\right)\phi(p_0,\vb{p})
	\end{split}
	\label{eq:kg_eom_ft}
\end{equation}
where we defined the relativistic energy-momentum (or dispersion) relation
\begin{equation}
	E(\vb{p})
	=
	\omega(\vb{p})
	=
	\sqrt{\vb{p}^2+m^2}
	\label{eq:energy_momentum_relation}.
\end{equation}
By restricting the integration domain to the momentum lightcone
\begin{equation}
	V
	=
	\left\{
		(p_0,\vb{p})\in\mathbb{R}^4
		\mid
		p_0^2=\omega(\vb{p})^2
	\right\}
	\label{eq:momentum_lightcone},
\end{equation}
we ensure that the Fourier modes satisfy the equation of motion.
It is sensible to rewrite the restricted Fourier mode expansion
\begin{equation}
	\begin{split}
		\phi(t,\vb{x})
		&=
		\int_V\frac{\dd[4]{p}}{(2\pi)^4}
		\phi(p_0,\vb{p})e^{-ip_\mu x^\mu}
		\\
		&=
		\int_{\mathbb{R}^4}\frac{\dd[4]{p}}{(2\pi)^3}
		\delta^{(1)}\left(p_0^2-\omega(\vb{p})^2\right)
		\phi(p_0,\vb{p})e^{-ip_\mu x^\mu}
		\\
		&=
		\int_{\mathbb{R}^4}\frac{\dd[4]{p}}{(2\pi)^3\sqrt{2\omega(\vb{p})}}
		\left\{
			a(\vb{p})
			e^{-i\omega(\vb{p})t}
			+
			a(\vb{p})
			e^{+i\omega(\vb{p})t}
		\right\}
		e^{i\vb{p}\vdot\vb{x}}
		\\
		&=
		\int_{\mathbb{R}^4}\frac{\dd[4]{p}}{(2\pi)^3\sqrt{2\omega(\vb{p})}}
		\left\{
			a(\vb{p})
			e^{-i\omega(\vb{p})t}
			e^{+i\vb{p}\vdot\vb{x}}
			+
			a(-\vb{p})
			e^{+i\omega(\vb{p})t}
			e^{-i\vb{p}\vdot\vb{x}}
		\right\}
		\\
		&=
		\int_{\mathbb{R}^4}\frac{\dd[4]{p}}{(2\pi)^3\sqrt{2\omega(\vb{p})}}
		\eval{\left\{
			a(\vb{p})
			e^{-ip_\mu x^\mu}
			+
			a(-\vb{p})
			e^{+ip_\mu x^\mu}
		\right\}}_{p_0=\omega(\vb{p})}
	\end{split}
	\label{eq:kg_ft_final}
\end{equation}
where we used the composition property of the delta distribution
\begin{equation}
	\delta^{(1)}\left(p_0^2-\omega(\vb{p})^2\right)
	=
	\frac{
		\delta^{(1)}\left(p_0-\omega(\vb{p})\right)
		+
		\delta^{(1)}\left(p_0+\omega(\vb{p})\right)
	}{\sqrt{2\omega(\vb{p})}}
	\label{eq:kg_em_delta}
\end{equation}
and defined $a(\vb{p})=\phi(\omega(\vb{p}),\vb{p})$ which has conjugate symmetry, $a(-\vb{p})=a(\vb{p})^*$ as $\phi(t,\vb{x})$ is real-valued.

We close the chapter by transitioning from Lagrangian to Hamiltonian field mechanics.
The canonical momentum density is given by
\begin{equation}
	\pi(t,\vb{x})
	=
	\partial_t\pdv{\mathcal{L}}{(\partial_t\phi)}
	=
	\partial_t\phi(t,\vb{x})
	\label{eq:canonical_momentum}
\end{equation}
and we find the conjugate variables to be
\begin{align}
	\hat\phi(t,\vb{x})
	&=
	\int\frac{\dd[3]{p}}{(2\pi)^3}
	\frac{1}{\sqrt{2\omega(\vb{p})}}
	\left\{
		a(\vb{p})
		e^{+ip_\mu x^\mu}
		+
		a(\vb{p})^*
		e^{-ip_\mu x^\mu}
	\right\}
	\label{eq:kg_pos}
	\\
	\hat\phi(t,\vb{p})
	&=
	\int\frac{\dd[3]{p}}{(2\pi)^3}
	\left(-i\sqrt{2\omega(\vb{p})}\right)
	\left\{
		a(\vb{p})
		e^{+ip_\mu x^\mu}
		-
		a(\vb{p})^*
		e^{-ip_\mu x^\mu}
	\right\}
	\label{eq:kg_mom}.
\end{align}
Field observables are encoded in the canonical energy-momentum, sometimes called (Hilbert) stress-energy, tensor
\begin{equation}
	T^{\mu\nu}
	=
	\pdv{\mathcal{L}}{(\partial_\mu\phi)}\partial^\nu\phi
	-
	g^{\mu\nu}\mathcal{L}
	\label{eq:energy_momentum_tensor}.
\end{equation}
For instance, the time-time component
\begin{equation}
	T^{00}
	=
	\frac{1}{2}
	\left\{
		\left(\partial_t\phi\right)^2
		+
		\left(\grad\phi\right)^2
		+
		\left(m\phi\right)^2
	\right\}
	=
	\mathcal{H}
	\label{eq:kg_energy_density}
\end{equation}
yields the energy density which integrated becomes the total energy
\begin{equation}
	H
	=
	\int\dd[3]{x}\mathcal{H}
	=
	\int\frac{\dd[3]{p}}{(2\pi)^3}
	\omega(\vb{p})\abs{a(\vb{p})}^2
	\label{eq:kg_total_energy},
\end{equation}
and the time-spatial components evaluate to the momentum density
\begin{equation}
	T^{0i}
	=
	-\pi(t,\vb{x})\partial_i\phi(t,\vb{x})
	=
	p_i
	\label{eq:kg_momentum_density}
\end{equation}
which integrated yields the total momentum
\begin{equation}
	\vb{P}
	=
	\int\dd[3]{x}T^{0i}
	=
	\int\frac{\dd[3]{p}}{(2\pi)^3}
	\vb{p}\abs{a(\vb{p})}^2
	\label{eq:kg_total_momentum}.
\end{equation}

\subsection{Canonical quantization}

In the canonical quantization procedure, we promote the dynamical field variables to operators
\begin{align}
	\hat\phi(t,\vb{x})
	&=
	\int\frac{\dd[3]{p}}{(2\pi)^3}
	\frac{1}{\sqrt{2\omega(\vb{p})}}
	\left\{
		\hat{a}(\vb{p})
		e^{+ip_\mu x^\mu}
		+
		\hat{a}^\dagger(\vb{p})
		e^{-ip_\mu x^\mu}
	\right\}
	\label{eq:qkg_pos}
	\\
	\hat\phi(t,\vb{p})
	&=
	\int\frac{\dd[3]{p}}{(2\pi)^3}
	\left(-i\sqrt{2\omega(\vb{p})}\right)
	\left\{
		\hat{a}(\vb{p})
		e^{+ip_\mu x^\mu}
		-
		\hat{a}^\dagger(\vb{p})
		e^{-ip_\mu x^\mu}
	\right\}
	\label{eq:qkg_mom}
\end{align}
satisfying the canonical commutation relations
\begin{align}
	\comm{\hat\phi(\vb{x})}{\hat\pi(\vb{y})}
	=
	i\delta^{(3)}(\vb{x}-\vb{y})
	&&
	\comm{\hat\phi(\vb{x})}{\hat\phi(\vb{y})}
	=
	0
	=
	\comm{\hat\pi(\vb{x})}{\hat\pi(\vb{y})}
	\label{eq:qkg_comm_pm}	
\end{align}
which encode the uncertainty relation.
Inserting \cref{eq:qkg_pos} and \cref{eq:qkg_mom} into \cref{eq:qkg_comm_pm} reveals the commutation relations for the Fourier mode operators
\begin{align}
	\comm{\hat{a}(\vb{p})}{\hat{a}^\dagger(\vb{q})}
	=
	(2\pi)^3\delta^{(3)}(\vb{p}-\vb{q})
	&&
	\comm{\hat{a}(\vb{p})}{\hat{a}(\vb{q})}
	=
	0
	=
	\comm{\hat{a}^\dagger(\vb{p})}{\hat{a}^\dagger(\vb{q})}
	\label{eq:kg_comm_ac}.
\end{align}
Extending the corresponding principle by normal-ordering and applying it to the classical energy and momentum observables, \cref{eq:kg_total_energy} and \cref{eq:kg_total_momentum}, we find
\begin{align}
	\hat{H}
	=
	\int\frac{\dd[3]{p}}{(2\pi)^3}
	\omega(\vb{p})\hat{a}^\dagger(\vb{p})\hat{a}(\vb{p})
	&&
	\hat{\vb{P}}
	=
	\int\frac{\dd[3]{p}}{(2\pi)^3}
	\vb{p}\hat{a}^\dagger(\vb{p})\hat{a}(\vb{p})
	\label{eq:qkg_total_energy_momentum}.
\end{align}

\subsubsection{Vacuum and momentum eigenstates}
Existence of a unique (?) vacuum state $\ket{0}$ for which we find
\begin{equation}
	\expval{\hat{H}}{0}
	=
	0
	=
	\expval{\hat{\vb{p}}}{0}
	\label{eq:qkg_expval_energy_momentum_vacuum}.
\end{equation}
The first and second moment of the momentum operator
\begin{align}
	\expval{\hat{a}(\vb{p})\hat{\vb{P}}\hat{a}^\dagger(\vb{p})}{0}
	\propto
	\vb{p}
	&&
	\expval{\hat{a}(\vb{p})\hat{\vb{P}}^2\hat{a}^\dagger(\vb{p})}{0}
	\propto
	\vb{p}^2
\end{align}
suggest that $\hat{a}^\dagger(\vb{p})\ket{0}=\ket{\vb{p}}$ is a momentum eigenstate, and
\begin{equation}
	\expval{\hat{\vb{P}}}{\vb{p}_1,\vb{p}_2}
	\propto
	\vb{p}_1+\vb{p}_2
\end{equation}
where we defined
\begin{equation}
	\ket{\vb{p}_1,\vb{p}_2}
	=
	\hat{a}^\dagger(\vb{p}_2)\ket{\vb{p}_1}
	=
	\hat{a}^\dagger(\vb{p}_2)\hat{a}^\dagger(\vb{p}_1)\ket{0}
	=
	\hat{a}^\dagger(\vb{p}_1)\hat{a}^\dagger(\vb{p}_2)\ket{0}
	=
	\ket{\vb{p}_2,\vb{p}_1}
\end{equation}
suggests that $\hat{a}^\dagger(\vb{p})$ acting on a state to the right, adds an excitation with momentum $\vb{p}$ to the field.

The momentum eigenstates are not normalizable
\begin{equation}
	\bra{\vb{p}}\ket{\vb{p}}
	=
	\expval{\hat{a}(\vb{p})\hat{a}^\dagger(\vb{p})}{0}
	=
	(2\pi)^3\delta^{(3)}(0)
\end{equation}
because $\delta^{(3)}(0)$ has no consistent mathematical definition but $\bra{\vb{p}}\ket{\vb{p}}$ appears. in all observables associated with the momentum eigenstates.
Actually, the wave function
\begin{equation}
	\psi(0,\vb{x})
	=
	\bra{0}\hat\phi(0,\vb{x})\ket{\vb{p}}
	=
	\int\frac{\dd[3]{q}}{(2\pi)^3}
	(2\pi)^3\delta^{(3)}(\vb{q}-\vb{p})
	e^{-i\vb{q}\vdot\vb{x}}
	=
	e^{-i\vb{p}\vdot\vb{x}}
\end{equation}
is a plane-wave which is not square-integrable and therefore not an element of the Hilbert space.

\subsection{Smearing and single-particle states}

\subsubsection{Wightman axioms and smearing}

\begin{align}
	\hat\Phi[f]
	&=
	\int\dd[4]{x}f(t,\vb{x})\hat\Phi(t,\vb{x})
\end{align}

\begin{align}
	\hat\phi^+[f]
	&=
	\int\dd[4]{x}
	f(t,\vb{x})
	\hat\phi^+(t,\vb{x})
\end{align}

\subsubsection{Definition}

\begin{equation}
	\ket{f}
	=
	\int\frac{\dd[3]{p}}{(2\pi)^3\sqrt{2\omega(\vb{p})}}
	f(\vb{p})\hat{a}^\dagger(\vb{p})\ket{0}
\end{equation}

\begin{equation}
	\bra{f}\ket{f}
	=
	\int\frac{\dd[3]{p}}{(2\pi)^32\omega(\vb{p})}
	\abs{f(\vb{p})}^2
\end{equation}

\subsubsection{Expectation values}

\begin{equation}
	\psi(t,\vb{x})
	=
	\bra{0}\hat\phi(t,\vb{x})\ket{f}
	=
\end{equation}

\subsubsection{Covariant Gaussian wave packets}

\subsection{Interactions and coherent states}

\section{Klein-Gordon field}

\subsection{Relativistic field theory}

The most simple non-trivial, free relativistic Lagrangian density of a real-valued scalar field $\phi$ to be constructed is
\begin{equation}
	\mathcal{L}
	=
	\frac{1}{2}
	\left(\partial_\mu\phi\right)
	\left(\partial^\mu\phi\right)
	-
	\frac{1}{2}
	m^2\phi^2
	\label{eq:kg_lagrangian}.
\end{equation}
The field equation of motion are found by the relativistic Euler-Lagrange equations
\begin{equation}
	0
	=
	\partial_\mu\pdv{\mathcal{L}}{(\partial_\mu\phi)}
	-
	\pdv{\mathcal{L}}{\phi}
	=
	\left(\partial_\mu\partial^\mu+m^2\right)
	\phi(t,\vb{x})
	\label{eq:kg_eom}
\end{equation}
which minimize the action.
Assuming the existence of the Fourier transform of the field~\cite[p.~341]{Cohen2019}
\begin{equation}
	\phi(t,\vb{x})
	=
	\int_{\mathbb{R}^3}\frac{\dd[3]{p}}{(2\pi)^3}
	\phi(t,\vb{p})e^{-i\vb{p}\vdot\vb{x}}
	=
	\int_{\mathbb{R}^4}\frac{\dd[4]{p}}{(2\pi)^4}
	\phi(p_0,\vb{p})e^{+ip_\mu x^\mu}
	\label{eq:kg_ft_naive},
\end{equation}
and inserting the Fourier transform into the equation of motion reduces \cref{eq:kg_eom} to
\begin{equation}
	0
	=
	\left(
		ip_\mu ip^\mu
		+
		m^2
	\right)
	\phi(p_0,\vb{p})
	=
	-
	\left(
		p_0^2
		-
		\omega(\vb{p})^2
	\right)
	\phi(p_0,\vb{p})
	\label{eq:kg_eom_ft}
\end{equation}
where we defined the relativistic energy-momentum (or dispersion) relation
\begin{equation}
	\omega(\vb{p})
	=
	\sqrt{\vb{p}^2+m^2}
	=
	E(\vb{p})
	\label{eq:energy_momentum_relation}.
\end{equation}
By restricting the integration domain to the momentum lightcone
\begin{equation}
	V
	=
	\left\{
		(p_0,\vb{p})\in\mathbb{R}^4
		\mid
		p_0^2=\omega(\vb{p})^2
	\right\}
	\label{eq:momentum_lightcone},
\end{equation}
we ensure that the Fourier modes satisfy the equation of motion.
It is sensible to rewrite the restricted Fourier mode expansion
\begin{equation*}
	\begin{split}
		\phi(t,\vb{x})
		&=
		\int_V\frac{\dd[4]{p}}{(2\pi)^4}
		\phi(p_0,\vb{p})
		e^{+ip_\mu x^\mu}
		\\
		&=
		\int_{\mathbb{R}^4}\frac{\dd[4]{p}}{(2\pi)^3}
		\delta^{(1)}\left(p_0^2-\omega(\vb{p})^2\right)
		\phi(p_0,\vb{p})
		e^{+ip_\mu x^\mu}
		\\
		&=
		\int_{\mathbb{R}^4}\frac{\dd[4]{p}}{(2\pi)^3}
		\frac{
			\delta^{(1)}\left(p_0-\omega(\vb{p})\right)
			+
			\delta^{(1)}\left(p_0+\omega(\vb{p})\right)
		}{\sqrt{2\omega(\vb{p})}}
		\phi(p_0,\vb{p})
		e^{+ip_0t}
		e^{-i\vb{p}\vdot\vb{x}}
		\\
		&=
		\int_{\mathbb{R}^3}\frac{\dd[3]{p}}{(2\pi)^3\sqrt{2\omega(\vb{p})}}
		\biggl\{
			\phi(\omega(\vb{p}),\vb{p})
			e^{+i\omega(\vb{p})t}
			+
			\phi(-\omega(\vb{p}),\vb{p})
			e^{-i\omega(\vb{p})t}
		\biggr\}
		e^{-i\vb{p}\vdot\vb{x}}
		\\
		&=
		\int_{\mathbb{R}^3}\frac{\dd[3]{p}}{(2\pi)^3\sqrt{2\omega(\vb{p})}}
		\biggl\{
			\phi(\omega(\vb{p}),\vb{p})
			e^{+i\omega(\vb{p})t}
			e^{-i\vb{p}\vdot\vb{x}}
			+
			\phi(-\omega(\vb{p}),\vb{p})
			e^{-i\omega(\vb{p})t}
			e^{-i\vb{p}\vdot\vb{x}}
		\biggr\}
		\\
		&=
		\int_{\mathbb{R}^3}\frac{\dd[3]{p}}{(2\pi)^3\sqrt{2\omega(\vb{p})}}
		\biggl\{
			\phi(\omega(\vb{p}),\vb{p})
			e^{+i\omega(\vb{p})t}
			e^{-i\vb{p}\vdot\vb{x}}
			+
			\phi(-\omega(\vb{p}),-\vb{p})
			e^{-i\omega(\vb{p})t}
			e^{+i\vb{p}\vdot\vb{x}}
		\biggr\}
		\\
		&=
		\int_{\mathbb{R}^3}\frac{\dd[3]{p}}{(2\pi)^3\sqrt{2\omega(\vb{p})}}
		\biggl\{
			\phi(\omega(\vb{p}),\vb{p})
			e^{+ip_\mu x^\mu}
			+
			\phi(\omega(\vb{p}),\vb{p})^*
			e^{-ip_\mu x^\mu}
		\biggr\}_{p_0=\omega(\vb{p})}
	\end{split}
\end{equation*}
where we used the conjugate symmetry of $\phi(p_0,\vb{p})$ because $\phi(t,\vb{x})$ is real-valued.
We define
\begin{equation}
	a(\vb{p})
	=
	\phi(\omega(\vb{p}),\vb{p})^*
\end{equation}
and from now on assume $p_0=\omega(\vb{p})$ if nothing different is stated.

We close the chapter by transitioning from Lagrangian to Hamiltonian field mechanics.
The canonical momentum density is given by
\begin{equation}
	\pi(t,\vb{x})
	=
	\partial_t\pdv{\mathcal{L}}{(\partial_t\phi)}
	=
	\partial_t\phi(t,\vb{x})
	\label{eq:canonical_momentum}
\end{equation}
and we find the conjugate variables to be
\begin{align}
	\phi(t,\vb{x})
	&=
	\int\frac{\dd[3]{p}}{(2\pi)^3}
	\frac{1}{\sqrt{2\omega(\vb{p})}}
	\left\{
		a(\vb{p})
		e^{-ip_\mu x^\mu}
		+
		a(\vb{p})^*
		e^{+ip_\mu x^\mu}
	\right\}
	\label{eq:kg_pos}
	\\
	\pi(t,\vb{x})
	&=
	\int\frac{\dd[3]{p}}{(2\pi)^3}
	\left(-i\sqrt{2\omega(\vb{p})}\right)
	\left\{
		a(\vb{p})
		e^{-ip_\mu x^\mu}
		-
		a(\vb{p})^*
		e^{+ip_\mu x^\mu}
	\right\}
	\label{eq:kg_mom}.
\end{align}
Field observables are encoded in the canonical energy-momentum tensor, sometimes called (Hilbert) stress-energy tensor
\begin{equation}
	T^{\mu\nu}
	=
	\pdv{\mathcal{L}}{(\partial_\mu\phi)}\partial^\nu\phi
	-
	g^{\mu\nu}\mathcal{L}
	\label{eq:energy_momentum_tensor}.
\end{equation}
For instance, the time components yield the energy and momentum density
\begin{align}
	T^{00}
	=
	\frac{1}{2}
	\left(\partial_t\phi\right)^2
	+
	\frac{1}{2}
	\left(\grad\phi\right)^2
	+
	\frac{1}{2}
	\left(m\phi\right)^2
	&&
	T^{0i}
	&=
	-\pi\partial_i\phi
\end{align}
which integrated upon becomes the total energy and momentum
\begin{align}
	H
	&=
	\int\dd[3]{x}T^{00}
	=
	\int\frac{\dd[3]{p}}{(2\pi)^3}
	\omega(\vb{p})\abs{a(\vb{p})}^2
	\label{eq:kg_total_energy}
	\\
	\vb{P}
	&=
	\int\dd[3]{x}T^{0i}
	=
	\int\frac{\dd[3]{p}}{(2\pi)^3}
	\vb{p}\abs{a(\vb{p})}^2
	\label{eq:kg_total_momentum}
	.
\end{align}

\subsection{Canonical quantization}

In the canonical quantization procedure, we promote the dynamical field variables to operators
\begin{align}
	\hat\phi(t,\vb{x})
	&=
	\int\frac{\dd[3]{p}}{(2\pi)^3}
	\frac{1}{\sqrt{2\omega(\vb{p})}}
	\left\{
		\hat{a}(\vb{p})
		e^{-ip_\mu x^\mu}
		+
		\hat{a}^\dagger(\vb{p})
		e^{+ip_\mu x^\mu}
	\right\}
	\label{eq:qkg_pos}
	\\
	\hat\pi(t,\vb{x})
	&=
	\int\frac{\dd[3]{p}}{(2\pi)^3}
	\left(-i\sqrt{2\omega(\vb{p})}\right)
	\left\{
		\hat{a}(\vb{p})
		e^{-ip_\mu x^\mu}
		-
		\hat{a}^\dagger(\vb{p})
		e^{+ip_\mu x^\mu}
	\right\}
	\label{eq:qkg_mom}
\end{align}
satisfying the canonical commutation relations
\begin{align}
	\comm{\hat\phi(\vb{x})}{\hat\pi(\vb{y})}
	=
	i\delta^{(3)}(\vb{x}-\vb{y})
	&&
	\comm{\hat\phi(\vb{x})}{\hat\phi(\vb{y})}
	=
	0
	=
	\comm{\hat\pi(\vb{x})}{\hat\pi(\vb{y})}
	\label{eq:qkg_comm_pm}	
\end{align}
which encode the uncertainty relation.
Inserting \cref{eq:qkg_pos} and \cref{eq:qkg_mom} into \cref{eq:qkg_comm_pm} reveals the commutation relations for the Fourier mode operators
\begin{align}
	\comm{\hat{a}(\vb{p})}{\hat{a}^\dagger(\vb{q})}
	=
	(2\pi)^3\delta^{(3)}(\vb{p}-\vb{q})
	&&
	\comm{\hat{a}(\vb{p})}{\hat{a}(\vb{q})}
	=
	0
	=
	\comm{\hat{a}^\dagger(\vb{p})}{\hat{a}^\dagger(\vb{q})}
	\label{eq:kg_comm_ac}.
\end{align}
Extending the corresponding principle by normal-ordering and applying it to the classical energy and momentum observables, \cref{eq:kg_total_energy} and \cref{eq:kg_total_momentum}, we find
\begin{align}
	\hat{H}
	=
	\int\frac{\dd[3]{p}}{(2\pi)^3}
	\omega(\vb{p})\hat{a}^\dagger(\vb{p})\hat{a}(\vb{p})
	&&
	\hat{\vb{P}}
	=
	\int\frac{\dd[3]{p}}{(2\pi)^3}
	\vb{p}\hat{a}^\dagger(\vb{p})\hat{a}(\vb{p})
	\label{eq:qkg_total_energy_momentum}.
\end{align}
In analogy with the quantum harmonic oscillator, we define the number operator to be
\begin{equation}
	\hat{N}
	=
	\int\frac{\dd[3]{p}}{(2\pi)^3}
	\hat{a}^\dagger(\vb{p})
	\hat{a}(\vb{p})
	\label{eq:qkg_number}.
\end{equation}
It will be useful to decompose the Klein-Gordon field into a positive and negative frequency part
\begin{equation}
	\hat\phi(t,\vb{x})
	=
	\hat\phi^+(t,\vb{x})
	+
	\hat\phi^+(t,\vb{x})
\end{equation}
where the positive and negative frequency parts are defined as~\cite[p.~26]{Peskin1995}
\begin{align}
	\hat\phi^+(t,\vb{x})
	&=
	\int\frac{\dd[3]{p}}{(2\pi)^3\sqrt{2\omega(\vb{p})}}
	e^{-ip_\mu x^\mu}
	\hat{a}(\vb{p})
	\\
	\hat\phi^-(t,\vb{x})
	&=
	\int\frac{\dd[3]{p}}{(2\pi)^3\sqrt{2\omega(\vb{p})}}
	e^{+ip_\mu x^\mu}
	\hat{a}^\dagger(\vb{p})
\end{align}
and both parts are related by the hermitian conjugate $\hat\phi^-(t,\vb{x})=\hat\phi^+(t,\vb{x})^\dagger$.

So far, we have not considered any states.
The most important state is the vacuum state $\ket{0}$ which is axiomatically defined to be invariant under spacetime translations~\cite[p.~276]{Bogolubov1989}
\begin{equation}
	e^{i\hat{P}_\mu a^\mu}
	\ket{0}
	=
	\ket{0}
\end{equation}
suggesting that the vacuum state is energy and momentum eigenstate with respective zero eigenvalue $\hat{P}_\mu\ket{0}=0$.
From the vacuum state any other state can be constructed by applying the appropriate operators to it.

The first and second moment of the momentum operator
\begin{align}
	\expval{\hat{a}(\vb{p})\hat{\vb{P}}\hat{a}^\dagger(\vb{p})}{0}
	\propto
	\vb{p}
	&&
	\expval{\hat{a}(\vb{p})\hat{\vb{P}}^2\hat{a}^\dagger(\vb{p})}{0}
	\propto
	\vb{p}^2
\end{align}
where we only used the commutation relations, suggest that $\hat{a}^\dagger(\vb{p})\ket{0}$ is a momentum eigenstate.
The expectation value
\begin{equation}
	\expval{\hat{\vb{P}}}{\vb{p}_1,\vb{p}_2}
	\propto
	\vb{p}_1+\vb{p}_2
\end{equation}
where we defined
\begin{equation}
	\ket{\vb{p}_1,\vb{p}_2}
	\propto
	\hat{a}^\dagger(\vb{p}_1)
	\hat{a}^\dagger(\vb{p}_2)
	\ket{0}
	\propto
	\ket{\vb{p}_2,\vb{p}_1}
\end{equation}
suggests that $\hat{a}^\dagger(\vb{p})$ acting on a state to the right, adds an excitation with momentum $\vb{p}$ to the field.

From $\hat{a}(\vb{p})\ket{0}=0$, we find that $\hat\phi^+(t,\vb{x})\ket{0}=0$ and thus
\begin{equation}
	\hat\phi(t,\vb{x})
	\ket{0}
	=
	\hat\phi^-(t,\vb{x})
	\ket{0}
	=
	\int\frac{\dd[3]{p}}{(2\pi)^3\sqrt{2\omega(\vb{p})}}
	e^{+ip_\mu x^\mu}
	\hat{a}^\dagger(\vb{p})
	\ket{0}
\end{equation}
which resembles the Fourier transform of of the momentum eigenstate.
In that sense, $\hat\phi$ respective $\hat\phi^-$ creates a particle ideally localized at the spacetime event $(t,\vb{x})$.

One critical problem of the momentum eigenstates is that they are not normalizable
\begin{equation}
	\bra{\vb{p}}\ket{\vb{p}}
	=
	\expval{\hat{a}(\vb{p})\hat{a}^\dagger(\vb{p})}{0}
	\propto
	\delta^{(3)}(0)
\end{equation}
because $\delta^{(3)}(0)$ has no consistent mathematical definition.
Actually, the coordinate wave function of the momentum space~\cite[p.~25]{Peskin1995}
\begin{equation}
	\begin{split}
		\psi(t,\vb{x})
		=
		\bra{0}
		\hat\phi(t,\vb{x})
		\hat{a}^\dagger(\vb{p})
		\ket{0}
		&=
		\left(
			\int\frac{\dd[3]{q}}{(2\pi)^3\sqrt{2\omega(\vb{q})}}
			e^{+iq_\mu x^\mu}
			\hat{a}^\dagger(\vb{q})
			\ket{0}
		\right)^\dagger
		\hat{a}^\dagger(\vb{p})
		\ket{0}
		\\
		&=
		\int\frac{\dd[3]{q}}{(2\pi)^3\sqrt{2\omega(\vb{q})}}
		e^{-iq_\mu x^\mu}
		\expval{\hat{a}(\vb{q})\hat{a}^\dagger(\vb{p})}{0}
		\\
		&=
		\int\frac{\dd[3]{q}}{(2\pi)^3\sqrt{2\omega(\vb{q})}}
		e^{-iq_\mu x^\mu}
		(2\pi)^3
		\delta^{(3)}(\vb{q}-\vb{p})
		\\
		&\propto
		e^{-ip_\mu x^\mu}
	\end{split}
\end{equation}
is a plane-wave which is not square-integrable and therefore is not even an element of the Hilbert space.

\subsection{Smearing and single-particle states}

\subsubsection{Wightman axioms and smearing}

According to Wightman quantum field theory~\cite[p.~324]{Bogolubov1989}, a quantum field $\hat\Phi$ is an operator-valued tempered distribution which needs be combined with some Lorentz-invariant Schwartz function $f\in\mathcal{S}(\mathbb{R}^4,\mathbb{K})$
\begin{equation}
	\hat\Phi[f]
	=
	\int\dd[4]{x}
	f(t,\vb{x})
	\hat\Phi(t,\vb{x})
\end{equation}
to avoid ambiguities arising from the idealized non-physical ultra-localized momentum states which we would recover for
\begin{equation}
	\hat\Phi[\delta_x]
	=
	\int\dd[4]{y}
	\delta^{(1)}(y^0-x^0)
	\delta^{(3)}(\vb{y}-\vb{x})
	\hat\Phi(y^0,\vb{y})
	=
	\hat\Phi(x^0,\vb{x})
	.
\end{equation}

\subsubsection{Definition smeared single-particle state}

Smearing the negative frequency field operator $\hat\phi^-$  with $f$ and defining $f(\vb{p})=f(\omega(\vb{p}),\vb{p})$
\begin{align}
	\hat\phi^-[f]
	&=
	\int\dd[4]{x}
	f(t,\vb{x})
	\hat\phi^-(t,\vb{x})
	\\
	&=
	\int\frac{\dd[3]{p}}{(2\pi)^3\sqrt{2\omega(\vb{p})}}
	\left(
		\int\dd[4]{x}
		f(t,\vb{x})
		e^{+ip_\mu x^\mu}
	\right)
	\hat{a}^\dagger(\vb{p})
	\\
	&=
	\int\frac{\dd[3]{p}}{(2\pi)^3\sqrt{2\omega(\vb{p})}}
	f(\vb{p})
	\hat{a}^\dagger(\vb{p})
\end{align}
and extrapolating from the known action of $\hat\phi$ on the vacuum, suggests
\begin{equation}
	\ket{f}
	=
	\int\frac{\dd[3]{p}}{(2\pi)^3\sqrt{2\omega(\vb{p})}}
	f(\vb{p})
	\hat{a}^\dagger(\vb{p})
	\ket{0}
\end{equation}
to represent a smeared particle state with localization in spacetime $f(t,\vb{x})$ respective momentum space $f(\vb{p})$.
The probability amplitude to measure $\ket{g}$ given $\ket{f}$ was prepared is equal to the overlap of the smearing functions
\begin{equation}
	\begin{split}
		\bra{g}\ket{f}
		&=
		\int\frac{\dd[3]{p}}{(2\pi)^3\sqrt{2\omega(\vb{p})}}
		\int\frac{\dd[3]{q}}{(2\pi)^3\sqrt{2\omega(\vb{q})}}
		f(\vb{p})g(\vb{q})^*
		\expval{\hat{a}(\vb{q})\hat{a}^\dagger(\vb{p})}{0}
		\\
		&=
		\int\frac{\dd[3]{p}}{(2\pi)^3\sqrt{2\omega(\vb{p})}}
		\int\frac{\dd[3]{q}}{(2\pi)^3\sqrt{2\omega(\vb{q})}}
		f(\vb{p})g(\vb{q})^*
		(2\pi)^3\delta^{(3)}(\vb{q}-\vb{p})
		\\
		&=
		\int\frac{\dd[3]{p}}{(2\pi)^32\omega(\vb{p})}
		f(\vb{p})g(\vb{p})^*
	\end{split}
\end{equation}
and suggests the normalization to be
\begin{equation}
	\bra{f}\ket{f}
	=
	\int\frac{\dd[3]{p}}{(2\pi)^32\omega(\vb{p})}
	\abs{f(\vb{p})}^2
	=
	1
\end{equation}
Alternatively, in coordinate space
\begin{equation}
	\begin{split}
		\bra{g}\ket{f}
		=
		\expval{\hat\phi[g]\hat\phi[f]}{0}
		&=
		\int\dd[4]{x}
		\int\dd[4]{y}
		f(x^0,\vb{x})g(y^0,\vb{y})^*
		\expval{\hat\phi(x^0,\vb{x})\hat\phi(y^0,\vb{y})}{0}
		\\
		&=
		\int\dd[4]{x}
		\int\dd[4]{y}
		f(x^0,\vb{x})g(y^0,\vb{y})^*
		D(x^0-y^0,\vb{x}-\vb{y})
	\end{split}
\end{equation}
where $D(x^0-y^0,\vb{x}-\vb{y})$ is the Feynman propagator~\cite[p.~27]{Peskin1995}.

\subsubsection{Expectation values of field observables}

The first two moments of the number operator are
\begin{align}
	\expval{\hat{N}}{f}
	&=
	\int\frac{\dd[3]{p}}{(2\pi)^32\omega(\vb{p})}
	\abs{f(\vb{p})}^2
	=
	1
	\\
	\expval{\hat{N}^2}{f}
	&=
	\int\frac{\dd[3]{p}}{(2\pi)^32\omega(\vb{p})}
	\abs{f(\vb{p})}^2
	=
	1
\end{align}
suggesting $\ket{f}$ to be a single-particle number state because $\expval{\left(\Delta\hat{N}\right)}{f}=0$
Similar the first two moments of the total energy are
\begin{align}
	\expval{\hat{H}}{f}
	&=
	\int\frac{\dd[3]{p}}{(2\pi)^32\omega(\vb{p})}
	\omega(\vb{p})
	\abs{f(\vb{p})}^2
	\\
	\expval{\hat{H}^2}{f}
	&=
	\int\frac{\dd[3]{p}}{(2\pi)^32\omega(\vb{p})}
	\omega(\vb{p})^2
	\abs{f(\vb{p})}^2
\end{align}
implying non-zero energy fluctuations $\expval{\left(\Delta\hat{H}\right)^2}{f}>0$.
By an completely analog calculation, we find for the total momentum
\begin{align}
	\expval{\hat{\vb{P}}}{f}
	&=
	\int\frac{\dd[3]{p}}{(2\pi)^32\omega(\vb{p})}
	\vb{p}
	\abs{f(\vb{p})}^2
	\\
	\expval{\hat{\vb{P}}^2}{f}
	&=
	\int\frac{\dd[3]{p}}{(2\pi)^32\omega(\vb{p})}
	\vb{p}^2
	\abs{f(\vb{p})}^2
\end{align}
in agreement with results reported in~\cite[eqs.~10 and 11]{Naumov2013}.

\subsubsection{Expectation values of wave packet observables}

From the spacetime representation of the wave function
\begin{equation}
	\psi(t,\vb{x})
	=
	\bra{0}\hat\phi(t,\vb{x})\ket{f}
	=
	\int\frac{\dd[3]{p}}{(2\pi)^32\omega(\vb{p})}
	f(\vb{p})e^{ip_\mu x^\mu}
\end{equation}
we can calculate the probability current
\begin{equation}
	\begin{split}
		j_\mu(t,\vb{x})
		&=
		2\Im\left\{
			\psi(t,\vb{x})^*
			\partial_\mu
			\psi(t,\vb{x})
		\right\}
		=
		\left(
			\rho(t,\vb{x}),
			\vb{j}(t,\vb{x})
		\right)
		\\
		&=
		\int\frac{\dd[3]{p}}{(2\pi)^32\omega(\vb{p})}
		\int\frac{\dd[3]{q}}{(2\pi)^32\omega(\vb{q})}
		\left\{
			q_\mu
			+
			p_\mu
		\right\}
		f(\vb{q})^*
		f(\vb{p})
		e^{-i(q_\mu-p_\mu)x^\mu}
	\end{split}
	\label{eq:qkg_probability_current}
\end{equation}
where we need to keep in mind that $q_0=\omega(\vb{q})$.
\Cref{eq:qkg_probability_current} is in agreement with results reported by Naumov~\cite[p.~9]{Naumov2013}.

The probability density is normalized
\begin{equation}
	\int\dd[3]{x}
	\rho(t,\vb{x})
	=
	1
\end{equation}
and the probability current yields the (group) velocity
\begin{equation}
	\overline{\vb{v}}
	=
	\int\dd[3]{x}
	\vb{j}(t,\vb{x})
	=
	\int\frac{\dd[3]{p}}{(2\pi)^32\omega(\vb{p})}
	\frac{\vb{p}}{\omega(\vb{p})}
	\abs{f(\vb{p})}^2
	\label{qkg:group_velocity}.
\end{equation}
Indeed in the non-relativistic regime $\omega(\vb{p})^2\sim m^2$
and the group velocity equals $\vb{p}/(2m^2)$ weighted by $\abs{f(\vb{p})}^2$.
The mean center-of-mass can be expressed in terms of the velocity
\begin{equation}
	\overline{\vb{x}}
	=
	\int\dd[3]{x}\vb{x}\rho(t,\vb{x})
	=
	\overline{\vb{v}}t
\end{equation}
The time-dependent variance of the position
\begin{equation}
	\sigma_x^2(t)
	=
	\sigma_x^2(0)
	+
	\sigma_v^2t^2
\end{equation}
can be interpreted as dispersion.

\subsubsection{Massless Gaussian wave packets}

Naumov~\cite{Naumov2013} discusses a massive covariant Gaussian wave packet.
We will discuss the massless case.

\subsection{Interactions and coherent states}

See also \cite{Itzykson2012}

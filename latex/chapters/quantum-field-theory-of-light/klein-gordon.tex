\section{Klein-Gordon field}

\subsection{Relativistic field theory}

\begin{definition}[Klein-Gordon Lagrangian]
	The Lorentz-invariant Lagrangian density
	\begin{equation}
		\mathcal{L}
		=
		\frac{1}{2}
		\left(\partial_\mu\phi\right)
		\left(\partial^\mu\phi\right)
		-
		\frac{1}{2}
		m^2\phi^2
		\label{eq:kg_lagrangian}
	\end{equation}
	describes a real scalar field $\phi(t,\vb{x})$ with mass $m>0$, the (massive) Klein-Gordon field.
\end{definition}
\begin{theorem}[Relativistic energy-momentum relation]\label{th:relativistic_energy_momentum}
	Excitations of the Klein-Gordon field satisfy the relativistic energy-momentum relation
	\begin{equation}
		\omega(\vb{p})
		=
		\sqrt{\vb{p}^2+m^2}
		=
		E(\vb{p})
		\label{eq:energy_momentum_relation}
	\end{equation}
	and the physical momentum space is constrained to the momentum lightcone
	\begin{equation}
		V
		=
		\left\{
			(p_0,\vb{p})\in\mathbb{R}^4
			\mid
			p_0^2=\omega(\vb{p})^2
		\right\}
		\label{eq:momentum_lightcone}
		.
	\end{equation}
\end{theorem}
\begin{proof}
	According to the action principle, the dynamics of the field are determined by the equations of motion which can be found by the relativistic Euler-Lagrange equations
	\begin{equation*}
		0
		=
		\partial_\mu\pdv{\mathcal{L}}{(\partial_\mu\phi)}
		-
		\pdv{\mathcal{L}}{\phi}
		=
		\left(
			\partial_\mu\partial^\mu
			+
			m^2
		\right)
		\phi(t,\vb{x})
		.
	\end{equation*}
	Assuming the existence of the Klein-Gordon field's Fourier representation~\cite[p.~341]{Cohen2019}
	\begin{equation}
		\phi(t,\vb{x})
		=
		\int_{\mathbb{R}^3}\frac{\dd[3]{p}}{(2\pi)^3}
		\phi(t,\vb{p})
		e^{-i\vb{p}\vdot\vb{x}}
		=
		\int_{\mathbb{R}^4}\frac{\dd[4]{p}}{(2\pi)^4}
		\phi(p_0,\vb{p})
		e^{+ip_\mu x^\mu}
		,
	\end{equation}
	the equation of motion in momentum space reduces to
	\begin{equation}
		0
		=
		\left(
			ip_\mu ip^\mu
			+
			m^2
		\right)
		\phi(p_0,\vb{p})
		=
		-
		\left(
			p_0^2
			-
			\omega(\vb{p})^2
		\right)
		\phi(p_0,\vb{p})
	\end{equation}
	which is satisfied if $p_0=\pm\omega(\vb{p})$.
\end{proof}

\begin{theorem}[Fourier expansion of the Klein-Gordon field]\label{thm:kg_fourier_expansion}
	The Fourier expansion of the Klein-Gordon field
	\begin{equation}
		\phi(t,\vb{x})
		=
		\int_{\mathbb{R}^3}\frac{\dd[3]{p}}{(2\pi)^3\sqrt{2\omega(\vb{p})}}
		\biggl\{
			a(\vb{p})^*
			e^{-ip_\mu x^\mu}
			+
			a(\vb{p})
			e^{+ip_\mu x^\mu}
		\biggr\}_{p_0=\omega(\vb{p})}
	\end{equation}
	satisfies the equations of motion for any choice of $a(\vb{p})=\phi(\omega(\vb{p}),\vb{p})^*$.
\end{theorem}
To get an intuition, the \proofref{proof in the appendix}{thm:kg_fourier_expansion} might be helpful.
\begin{remark}
	From now on assume $p_0=\omega(\vb{p})$ if not stated otherwise.
\end{remark}

\begin{corollary}[Conjugate momentum density of the Klein-Gordon field]
	The conjugate momentum density of the Klein-Gordon field is
	\begin{equation}
		\pi(t,\vb{x})
		=
		-i
		\int\frac{\dd[3]{p}}{(2\pi)^3}
		\sqrt{2\omega(\vb{p})}
		\left\{
			a(\vb{p})
			e^{-ip_\mu x^\mu}
			-
			a(\vb{p})^*
			e^{+ip_\mu x^\mu}
		\right\}
	\end{equation}
	which follows directly from the Legendre transformation
	\begin{equation}
		\pi(t,\vb{x})
		=
		\partial_t\pdv{\mathcal{L}}{(\partial_t\phi)}
		=
		\partial_t\phi(t,\vb{x})
	\end{equation}
	fundamental to Hamiltonian mechanics.
\end{corollary}

\begin{definition}[Energy-momentum tensor]
	We define the energy-momentum tensor
	\begin{equation}
		T^{\mu\nu}
		=
		\pdv{\mathcal{L}}{(\partial_\mu\phi)}\partial^\nu\phi
		-
		g^{\mu\nu}\mathcal{L}
		\label{eq:energy_momentum_tensor}
	\end{equation}
	which encodes the field's observables in its components.
	For instance, the energy density is encoded in the
	\begin{equation*}
		T^{00}
		=
		\frac{1}{2}
		\left(\partial_t\phi\right)^2
		+
		\frac{1}{2}
		\left(\grad\phi\right)^2
		+
		\frac{1}{2}
		\left(m\phi\right)^2		
	\end{equation*}
	component and the momentum density in the
	\begin{equation*}
		T^{0i}
		=
		-\pi\partial_i\phi		
	\end{equation*}
	component, see Ref.~\cite{Peskin1995}.
\end{definition}

\begin{lemma}[Energy and momentum of the Klein-Gordon field]
	The total energy and momentum observables of the Klein-Gordon field are
	\begin{align}
		H
		=
		\int\frac{\dd[3]{p}}{(2\pi)^3}
		\omega(\vb{p})\abs{a(\vb{p})}^2
		&&
		\vb{P}
		=
		\int\frac{\dd[3]{p}}{(2\pi)^3}
		\vb{p}\abs{a(\vb{p})}^2
		\label{eq:kg_energy_momentum}
	\end{align}
\end{lemma}
\begin{proof}
	Perform a spatial integration over the energy and momentum densities, see Ref.~\cite{Peskin1995}.
\end{proof}

\subsection{Canonical quantization}

\begin{definition}[Klein-Gordon field operators]
	In the canonical quantization procedure, we promote the dynamical field variables to operators
	\begin{align}
		\hat\phi(t,\vb{x})
		&=
		\int\frac{\dd[3]{p}}{(2\pi)^3}
		\frac{1}{\sqrt{2\omega(\vb{p})}}
		\left\{
			\hat{a}(\vb{p})
			e^{-ip_\mu x^\mu}
			+
			\hat{a}^\dagger(\vb{p})
			e^{+ip_\mu x^\mu}
		\right\}
		\label{eq:qkg_pos}
		\\
		\hat\pi(t,\vb{x})
		&=
		\int\frac{\dd[3]{p}}{(2\pi)^3}
		\left(-i\sqrt{2\omega(\vb{p})}\right)
		\left\{
			\hat{a}(\vb{p})
			e^{-ip_\mu x^\mu}
			-
			\hat{a}^\dagger(\vb{p})
			e^{+ip_\mu x^\mu}
		\right\}
		\label{eq:qkg_mom}
	\end{align}
	which satisfy the canonical commutation relations
	\begin{align}
		\comm{\hat\phi(\vb{x})}{\hat\pi(\vb{y})}
		=
		i\delta^{(3)}(\vb{x}-\vb{y})
		&&
		\comm{\hat\phi(\vb{x})}{\hat\phi(\vb{y})}
		=
		0
		=
		\comm{\hat\pi(\vb{x})}{\hat\pi(\vb{y})}
		\label{eq:qkg_comm_pm}
		.
	\end{align}
	We refer to $\hat{a}(\vb{p})$ as the annihilation operator and $\hat{a}^\dagger(\vb{p})$ as the creation operator of the Klein-Gordon field with respect to momentum $\vb{p}$.
\end{definition}

\begin{theorem}[Commutation relations for the annihilation and creation operator]
	The annihilation and creation operator of the Klein-Gordon field obey the commutation relations
	\begin{align}
		\comm{\hat{a}(\vb{p})}{\hat{a}^\dagger(\vb{q})}
		=
		(2\pi)^3\delta^{(3)}(\vb{p}-\vb{q})
		&&
		\comm{\hat{a}(\vb{p})}{\hat{a}(\vb{q})}
		=
		0
		=
		\comm{\hat{a}^\dagger(\vb{p})}{\hat{a}^\dagger(\vb{q})}
		\label{eq:kg_comm_ac}
		.
	\end{align}	
\end{theorem}
\begin{proof}
	Insert the Klein-Gordon field operators in terms of the annihilation and creation operators, \cref{eq:qkg_pos} and \cref{eq:qkg_mom}, into the field commutation relations \cref{eq:qkg_comm_pm}.
\end{proof}

\begin{definition}[Normal-ordered product]
	Let $\hat{A}$ be an operator that can be expressed as a product of annihilation and creation operators of the Klein-Gordon field, then the normal-ordered product of $\hat{A}$ is
	\begin{equation}
		\hat{A}
		=
		\hat{a}^\dagger(\vb{p}_1)
		\dots
		\hat{a}^\dagger(\vb{p}_n)
		\hat{a}(\vb{q}_1)
		\dots
		\hat{a}(\vb{q}_n)
		,
	\end{equation}
	i.e., the creation operators are on the left and the annihilation operators are on the right side.
	As the annihilation operators and creation operators commute among another, the order of the operators on the left or right is not relevant.
\end{definition}

\begin{definition}[Correspondence principle]
	The correspondence principle relates classical with quantum observables.
	Given a classical observable, we replace the dynamical variables with the corresponding quantum operators in normal order.
\end{definition}
\begin{remark}
	The requirement of the quantum observable being in normal order is essential as in contrast to the classical dynamical variables, quantum operators do not commute.
	See Ref.~\cite[p.~20]{Mukhanov2007} for more details on the problem of operator ordering.
\end{remark}

\begin{lemma}[Energy and momentum operators of the Klein-Gordon field]
	Invoking the correspondence principle to the total energy and momentum we derived for the classical Klein-Gordon field, \cref{eq:kg_energy_momentum}, we find
	\begin{align}
		\hat{H}
		=
		\int\frac{\dd[3]{p}}{(2\pi)^3}
		\omega(\vb{p})\hat{a}^\dagger(\vb{p})\hat{a}(\vb{p})
		&&
		\hat{\vb{P}}
		=
		\int\frac{\dd[3]{p}}{(2\pi)^3}
		\vb{p}\hat{a}^\dagger(\vb{p})\hat{a}(\vb{p})
		\label{eq:qkg_energy_momentum}
	\end{align}
	to be the energy and momentum operators of the quantum Klein-Gordon field.
\end{lemma}
\begin{corollary}[Number operator of the Klein-Gordon field]
	In analogy with the quantum harmonic oscillator, the number operator is the unweighted part of the energy operator, i.e.,
	\begin{equation}
		\hat{N}
		=
		\int\frac{\dd[3]{p}}{(2\pi)^3}
		\hat{a}^\dagger(\vb{p})
		\hat{a}(\vb{p})
		\label{eq:qkg_number}
		.
	\end{equation}
	The squared number operator which is required for calculating the variance is the squared number operator in normal order~\cite{Barnett2002}
	\begin{equation}
		\hat{N}^2
		=
		\int\frac{\dd[3]{p_1}}{(2\pi)^3}
		\int\frac{\dd[3]{p_2}}{(2\pi)^3}
		\hat{a}^\dagger(\vb{p}_1)
		\hat{a}^\dagger(\vb{p}_2)
		\hat{a}(\vb{p}_1)
		\hat{a}(\vb{p}_2)
		.
	\end{equation}
\end{corollary}

\begin{definition}[Positive and negative frequency Klein-Gordon field operator]
	It will be useful to decompose the Klein-Gordon field into a positive and negative frequency part
	\begin{equation}
		\hat\phi(t,\vb{x})
		=
		\hat\phi^+(t,\vb{x})
		+
		\hat\phi^+(t,\vb{x})
	\end{equation}
	where the positive and negative frequency parts are defined as~\cite[p.~26]{Peskin1995}
	\begin{equation}
		\begin{split}
			\hat\phi^+(t,\vb{x})
			&=
			\int\frac{\dd[3]{p}}{(2\pi)^3\sqrt{2\omega(\vb{p})}}
			e^{-ip_\mu x^\mu}
			\hat{a}(\vb{p})
			\\
			\hat\phi^-(t,\vb{x})
			&=
			\int\frac{\dd[3]{p}}{(2\pi)^3\sqrt{2\omega(\vb{p})}}
			e^{+ip_\mu x^\mu}
			\hat{a}^\dagger(\vb{p})
		\end{split}
		\label{eq:qkg_positive_negative_frequency}
	\end{equation}
	and both parts are related by the hermitian conjugate $\hat\phi^-(t,\vb{x})=\hat\phi^+(t,\vb{x})^\dagger$.
\end{definition}

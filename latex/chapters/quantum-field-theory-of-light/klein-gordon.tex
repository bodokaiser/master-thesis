\section{Klein-Gordon field}

\subsection{Relativistic field theory}

The most simple non-trivial, free relativistic Lagrangian density of a real-valued scalar field $\phi$ to be constructed is
\begin{equation}
	\mathcal{L}
	=
	\frac{1}{2}
	\left(\partial_\mu\phi\right)
	\left(\partial^\mu\phi\right)
	-
	\frac{1}{2}
	m^2\phi^2
	\label{eq:kg_lagrangian}.
\end{equation}
The field equation of motion are found by the relativistic Euler-Lagrange equations
\begin{equation}
	0
	=
	\partial_\mu\pdv{\mathcal{L}}{(\partial_\mu\phi)}
	-
	\pdv{\mathcal{L}}{\phi}
	=
	\left(\partial_\mu\partial^\mu+m^2\right)
	\phi(t,\vb{x})
	\label{eq:kg_eom}
\end{equation}
which minimize the action.
Assuming the existence of the Fourier transform of the field
\begin{equation}
	\phi(t,\vb{x})
	=
	\int_{\mathbb{R}^3}\frac{\dd[3]{p}}{(2\pi)^3}
	\phi(t,\vb{p})e^{+i\vb{p}\vdot\vb{x}}
	=
	\int_{\mathbb{R}^4}\frac{\dd[4]{p}}{(2\pi)^4}
	\phi(p_0,\vb{p})e^{-ip_\mu x^\mu}
	\label{eq:kg_ft_naive},
\end{equation}
and inserting the Fourier transform into the equation of motion reduces \cref{eq:kg_eom} to
\begin{equation}
	\begin{split}
		0
		&=
		(ip_\mu ip^\mu+m^2)\phi(p_0,\vb{p})
		\\
		&=
		-(p_\mu p^\mu-m^2)\phi(p_0,\vb{p})
		\\
		&=
		-(p_0^2-\vb{p}^2-m^2)\phi(p_0,\vb{p})
		\\
		&=
		-\left(p_0^2-\omega(\vb{p})^2\right)\phi(p_0,\vb{p})
	\end{split}
	\label{eq:kg_eom_ft}
\end{equation}
where we defined the relativistic energy-momentum (or dispersion) relation
\begin{equation}
	E(\vb{p})
	=
	\omega(\vb{p})
	=
	\sqrt{\vb{p}^2+m^2}
	\label{eq:energy_momentum_relation}.
\end{equation}
By restricting the integration domain to the momentum lightcone
\begin{equation}
	V
	=
	\left\{
		(p_0,\vb{p})\in\mathbb{R}^4
		\mid
		p_0^2=\omega(\vb{p})^2
	\right\}
	\label{eq:momentum_lightcone},
\end{equation}
we ensure that the Fourier modes satisfy the equation of motion.
It is sensible to rewrite the restricted Fourier mode expansion
\begin{equation}
	\begin{split}
		\phi(t,\vb{x})
		&=
		\int_V\frac{\dd[4]{p}}{(2\pi)^4}
		\phi(p_0,\vb{p})e^{-ip_\mu x^\mu}
		\\
		&=
		\int_{\mathbb{R}^4}\frac{\dd[4]{p}}{(2\pi)^3}
		\delta^{(1)}\left(p_0^2-\omega(\vb{p})^2\right)
		\phi(p_0,\vb{p})e^{-ip_\mu x^\mu}
		\\
		&=
		\int_{\mathbb{R}^4}\frac{\dd[4]{p}}{(2\pi)^3\sqrt{2\omega(\vb{p})}}
		\left\{
			a(\vb{p})
			e^{-i\omega(\vb{p})t}
			+
			a(\vb{p})
			e^{+i\omega(\vb{p})t}
		\right\}
		e^{i\vb{p}\vdot\vb{x}}
		\\
		&=
		\int_{\mathbb{R}^4}\frac{\dd[4]{p}}{(2\pi)^3\sqrt{2\omega(\vb{p})}}
		\left\{
			a(\vb{p})
			e^{-i\omega(\vb{p})t}
			e^{+i\vb{p}\vdot\vb{x}}
			+
			a(-\vb{p})
			e^{+i\omega(\vb{p})t}
			e^{-i\vb{p}\vdot\vb{x}}
		\right\}
		\\
		&=
		\int_{\mathbb{R}^4}\frac{\dd[4]{p}}{(2\pi)^3\sqrt{2\omega(\vb{p})}}
		\eval{\left\{
			a(\vb{p})
			e^{-ip_\mu x^\mu}
			+
			a(-\vb{p})
			e^{+ip_\mu x^\mu}
		\right\}}_{p_0=\omega(\vb{p})}
	\end{split}
	\label{eq:kg_ft_final}
\end{equation}
where we used the composition property of the delta distribution
\begin{equation}
	\delta^{(1)}\left(p_0^2-\omega(\vb{p})^2\right)
	=
	\frac{
		\delta^{(1)}\left(p_0-\omega(\vb{p})\right)
		+
		\delta^{(1)}\left(p_0+\omega(\vb{p})\right)
	}{\sqrt{2\omega(\vb{p})}}
	\label{eq:kg_em_delta}
\end{equation}
and defined $a(\vb{p})=\phi(\omega(\vb{p}),\vb{p})$ which has conjugate symmetry, $a(-\vb{p})=a(\vb{p})^*$ as $\phi(t,\vb{x})$ is real-valued.

We close the chapter by transitioning from Lagrangian to Hamiltonian field mechanics.
The canonical momentum density is given by
\begin{equation}
	\pi(t,\vb{x})
	=
	\partial_t\pdv{\mathcal{L}}{(\partial_t\phi)}
	=
	\partial_t\phi(t,\vb{x})
	\label{eq:canonical_momentum}
\end{equation}
and we find the conjugate variables to be
\begin{align}
	\hat\phi(t,\vb{x})
	&=
	\int\frac{\dd[3]{p}}{(2\pi)^3}
	\frac{1}{\sqrt{2\omega(\vb{p})}}
	\left\{
		a(\vb{p})
		e^{+ip_\mu x^\mu}
		+
		a(\vb{p})^*
		e^{-ip_\mu x^\mu}
	\right\}
	\label{eq:kg_pos}
	\\
	\hat\pi(t,\vb{p})
	&=
	\int\frac{\dd[3]{p}}{(2\pi)^3}
	\left(-i\sqrt{2\omega(\vb{p})}\right)
	\left\{
		a(\vb{p})
		e^{+ip_\mu x^\mu}
		-
		a(\vb{p})^*
		e^{-ip_\mu x^\mu}
	\right\}
	\label{eq:kg_mom}.
\end{align}
Field observables are encoded in the canonical energy-momentum, sometimes called (Hilbert) stress-energy, tensor
\begin{equation}
	T^{\mu\nu}
	=
	\pdv{\mathcal{L}}{(\partial_\mu\phi)}\partial^\nu\phi
	-
	g^{\mu\nu}\mathcal{L}
	\label{eq:energy_momentum_tensor}.
\end{equation}
For instance, the time-time component
\begin{equation}
	T^{00}
	=
	\frac{1}{2}
	\left\{
		\left(\partial_t\phi\right)^2
		+
		\left(\grad\phi\right)^2
		+
		\left(m\phi\right)^2
	\right\}
	=
	\mathcal{H}
	\label{eq:kg_energy_density}
\end{equation}
yields the energy density which integrated becomes the total energy
\begin{equation}
	H
	=
	\int\dd[3]{x}\mathcal{H}
	=
	\int\frac{\dd[3]{p}}{(2\pi)^3}
	\omega(\vb{p})\abs{a(\vb{p})}^2
	\label{eq:kg_total_energy},
\end{equation}
and the time-spatial components evaluate to the momentum density
\begin{equation}
	T^{0i}
	=
	-\pi(t,\vb{x})\partial_i\phi(t,\vb{x})
	=
	p_i
	\label{eq:kg_momentum_density}
\end{equation}
which integrated yields the total momentum
\begin{equation}
	\vb{P}
	=
	\int\dd[3]{x}T^{0i}
	=
	\int\frac{\dd[3]{p}}{(2\pi)^3}
	\vb{p}\abs{a(\vb{p})}^2
	\label{eq:kg_total_momentum}.
\end{equation}

\subsection{Canonical quantization}

In the canonical quantization procedure, we promote the dynamical field variables to operators
\begin{align}
	\hat\phi(t,\vb{x})
	&=
	\int\frac{\dd[3]{p}}{(2\pi)^3}
	\frac{1}{\sqrt{2\omega(\vb{p})}}
	\left\{
		\hat{a}(\vb{p})
		e^{+ip_\mu x^\mu}
		+
		\hat{a}^\dagger(\vb{p})
		e^{-ip_\mu x^\mu}
	\right\}
	\label{eq:qkg_pos}
	\\
	\hat\pi(t,\vb{p})
	&=
	\int\frac{\dd[3]{p}}{(2\pi)^3}
	\left(-i\sqrt{2\omega(\vb{p})}\right)
	\left\{
		\hat{a}(\vb{p})
		e^{+ip_\mu x^\mu}
		-
		\hat{a}^\dagger(\vb{p})
		e^{-ip_\mu x^\mu}
	\right\}
	\label{eq:qkg_mom}
\end{align}
satisfying the canonical commutation relations
\begin{align}
	\comm{\hat\phi(\vb{x})}{\hat\pi(\vb{y})}
	=
	i\delta^{(3)}(\vb{x}-\vb{y})
	&&
	\comm{\hat\phi(\vb{x})}{\hat\phi(\vb{y})}
	=
	0
	=
	\comm{\hat\pi(\vb{x})}{\hat\pi(\vb{y})}
	\label{eq:qkg_comm_pm}	
\end{align}
which encode the uncertainty relation.
Inserting \cref{eq:qkg_pos} and \cref{eq:qkg_mom} into \cref{eq:qkg_comm_pm} reveals the commutation relations for the Fourier mode operators
\begin{align}
	\comm{\hat{a}(\vb{p})}{\hat{a}^\dagger(\vb{q})}
	=
	(2\pi)^3\delta^{(3)}(\vb{p}-\vb{q})
	&&
	\comm{\hat{a}(\vb{p})}{\hat{a}(\vb{q})}
	=
	0
	=
	\comm{\hat{a}^\dagger(\vb{p})}{\hat{a}^\dagger(\vb{q})}
	\label{eq:kg_comm_ac}.
\end{align}
Extending the corresponding principle by normal-ordering and applying it to the classical energy and momentum observables, \cref{eq:kg_total_energy} and \cref{eq:kg_total_momentum}, we find
\begin{align}
	\hat{H}
	=
	\int\frac{\dd[3]{p}}{(2\pi)^3}
	\omega(\vb{p})\hat{a}^\dagger(\vb{p})\hat{a}(\vb{p})
	&&
	\hat{\vb{P}}
	=
	\int\frac{\dd[3]{p}}{(2\pi)^3}
	\vb{p}\hat{a}^\dagger(\vb{p})\hat{a}(\vb{p})
	\label{eq:qkg_total_energy_momentum}.
\end{align}
In analogy with the quantum harmonic oscillator, we define the number operator to be
\begin{equation}
	\hat{N}
	=
	\int\frac{\dd[3]{p}}{(2\pi)^3}
	\hat{a}^\dagger(\vb{p})
	\hat{a}(\vb{p})
	\label{eq:qkg_number}.
\end{equation}

\subsubsection{Vacuum and momentum eigenstates}
Existence of a unique (?) vacuum state $\ket{0}$ for which we find
\begin{equation}
	\expval{\hat{H}}{0}
	=
	0
	=
	\expval{\hat{\vb{p}}}{0}
	\label{eq:qkg_expval_energy_momentum_vacuum}.
\end{equation}
The first and second moment of the momentum operator
\begin{align}
	\expval{\hat{a}(\vb{p})\hat{\vb{P}}\hat{a}^\dagger(\vb{p})}{0}
	\propto
	\vb{p}
	&&
	\expval{\hat{a}(\vb{p})\hat{\vb{P}}^2\hat{a}^\dagger(\vb{p})}{0}
	\propto
	\vb{p}^2
\end{align}
suggest that $\hat{a}^\dagger(\vb{p})\ket{0}=\ket{\vb{p}}$ is a momentum eigenstate, and
\begin{equation}
	\expval{\hat{\vb{P}}}{\vb{p}_1,\vb{p}_2}
	\propto
	\vb{p}_1+\vb{p}_2
\end{equation}
where we defined
\begin{equation}
	\ket{\vb{p}_1,\vb{p}_2}
	=
	\hat{a}^\dagger(\vb{p}_2)\ket{\vb{p}_1}
	=
	\hat{a}^\dagger(\vb{p}_2)\hat{a}^\dagger(\vb{p}_1)\ket{0}
	=
	\hat{a}^\dagger(\vb{p}_1)\hat{a}^\dagger(\vb{p}_2)\ket{0}
	=
	\ket{\vb{p}_2,\vb{p}_1}
\end{equation}
suggests that $\hat{a}^\dagger(\vb{p})$ acting on a state to the right, adds an excitation with momentum $\vb{p}$ to the field.

The momentum eigenstates are not normalizable
\begin{equation}
	\bra{\vb{p}}\ket{\vb{p}}
	=
	\expval{\hat{a}(\vb{p})\hat{a}^\dagger(\vb{p})}{0}
	=
	(2\pi)^3\delta^{(3)}(0)
\end{equation}
because $\delta^{(3)}(0)$ has no consistent mathematical definition but $\bra{\vb{p}}\ket{\vb{p}}$ appears. in all observables associated with the momentum eigenstates.
Actually, the wave function
\begin{equation}
	\psi(0,\vb{x})
	=
	\bra{0}\hat\phi(0,\vb{x})\ket{\vb{p}}
	=
	\int\frac{\dd[3]{q}}{(2\pi)^3}
	(2\pi)^3\delta^{(3)}(\vb{q}-\vb{p})
	e^{-i\vb{q}\vdot\vb{x}}
	=
	e^{-i\vb{p}\vdot\vb{x}}
\end{equation}
is a plane-wave which is not square-integrable and therefore not an element of the Hilbert space.

Decomposing the Klein-Gordon field into a positive and negative frequency part
\begin{equation}
	\hat\phi(t,\vb{x})
	=
	\hat\phi^+(t,\vb{x})
	+
	\hat\phi^+(t,\vb{x})
\end{equation}
where the positive frequency part is defined to be
\begin{equation}
	\hat\phi(t,\vb{x})
	=
	\int\frac{\dd[3]{p}}{(2\pi)^3\sqrt{2\omega(\vb{p})}}
	e^{-ip_\mu x^\mu}
	\hat{a}^\dagger(\vb{p})
\end{equation}
and the negative frequency part $\hat\phi^-(t,\vb{x})=\hat\phi^+(t,\vb{x})^\dagger$.
The field operator acting on the vacuum
\begin{equation}
	\hat\phi(t,\vb{x})\ket{0}
	=
	\hat\phi^+(t,\vb{x})\ket{0}
	=
	\int\frac{\dd[3]{p}}{(2\pi)^3\sqrt{2\omega(\vb{p})}}
	e^{-ip_\mu x^\mu}
	\hat{a}^\dagger(\vb{p})
	\ket{0}
\end{equation}
appears to resemble the relativistic Fourier transform of an momentum eigenstate.
In that sense $\hat\phi(t,\vb{x})\ket{0}$ corresponds to a particle localized exactly at $t,\vb{x}$

\subsection{Smearing and single-particle states}

\subsubsection{Wightman axioms and smearing}

According to Wightman quantum field theory, a quantum field $\hat\Phi$ is an operator-valued tempered distribution which needs be combined with some Lorentz-invariant Schwartz function $f\in\mathcal{S}(\mathbb{R}^4,\mathbb{K})$
\begin{equation}
	\hat\Phi[f]
	=
	\int\dd[4]{x}f(t,\vb{x})\hat\Phi(t,\vb{x})	
\end{equation}
to avoid ambiguities arising from the idealized non-physical ultra-localized momentum states.

\subsubsection{Definition smeared single-particle state}

Smearing the positive frequency operator $\hat\phi^+$  with $f$ and defining $f(\vb{p})=f(\omega(\vb{p}),\vb{p})$
\begin{align}
	\hat\phi^+[f]
	&=
	\int\dd[4]{x}
	f(t,\vb{x})
	\hat\phi^+(t,\vb{x})
	\\
	&=
	\int\frac{\dd[3]{p}}{(2\pi)^3\sqrt{2\omega(\vb{p})}}
	\left(
		\int\dd[4]{x}
		f(t,\vb{x})
		e^{-ip_\mu x^\mu}
	\right)
	\hat{a}^\dagger(\vb{p})
	\\
	&=
	\int\frac{\dd[3]{p}}{(2\pi)^3\sqrt{2\omega(\vb{p})}}
	f(\vb{p})
	\hat{a}^\dagger(\vb{p})
\end{align}
and extrapolating from the known action of $\hat\phi^+$ on the vacuum, suggests
\begin{equation}
	\ket{f}
	=
	\hat\phi[f]\ket{0}
	=
	\hat\phi^+[f]\ket{0}
	=
	\int\frac{\dd[3]{p}}{(2\pi)^3\sqrt{2\omega(\vb{p})}}
	f(\vb{p})\hat{a}^\dagger(\vb{p})\ket{0}
\end{equation}
to represent a smeared particle state with localization in spacetime $f(t,\vb{x})$ respective momentum space $f(\vb{p})$.
The probability amplitude to measure $\ket{g}$ given $\ket{f}$ was prepared is
\begin{equation}
	\begin{split}
		\bra{g}\ket{f}
		&=
		\int\frac{\dd[3]{p}}{(2\pi)^3\sqrt{2\omega(\vb{p})}}
		\int\frac{\dd[3]{q}}{(2\pi)^3\sqrt{2\omega(\vb{q})}}
		f(\vb{p})g(\vb{q})^*
		\expval{\hat{a}(\vb{q})\hat{a}^\dagger(\vb{p})}{0}
		\\
		&=
		\int\frac{\dd[3]{p}}{(2\pi)^3\sqrt{2\omega(\vb{p})}}
		\int\frac{\dd[3]{q}}{(2\pi)^3\sqrt{2\omega(\vb{q})}}
		f(\vb{p})g(\vb{q})^*
		(2\pi)^3\delta^{(3)}(\vb{q}-\vb{p})
		\\
		&=
		\int\frac{\dd[3]{p}}{(2\pi)^32\omega(\vb{p})}
		f(\vb{p})g(\vb{p})^*
	\end{split}
\end{equation}
and suggests the normalization to be
\begin{equation}
	\bra{f}\ket{f}
	=
	\int\frac{\dd[3]{p}}{(2\pi)^32\omega(\vb{p})}
	\abs{f(\vb{p})}^2
	=
	1
\end{equation}
Alternatively, in coordinate space
\begin{equation}
	\begin{split}
		\bra{g}\ket{f}
		=
		\expval{\hat\phi[g]\hat\phi[f]}{0}
		&=
		\int\dd[4]{x}
		\int\dd[4]{y}
		f(x^0,\vb{x})g(y^0,\vb{y})^*
		\expval{\hat\phi(x^0,\vb{x})\hat\phi(y^0,\vb{y})}{0}
		\\
		&=
		\int\dd[4]{x}
		\int\dd[4]{y}
		f(x^0,\vb{x})g(y^0,\vb{y})^*
		D(x^0-y^0,\vb{x}-\vb{y})
	\end{split}
\end{equation}
where $D(x^0-y^0,\vb{x}-\vb{y})$ is the Feynman propagator~\cite[p.~27]{Peskin1995}.

\subsubsection{Expectation values of field observables}

To get a better "feel" for the smeared particle state, we evaluate some expectation values.
First we derive the auxiliary result
\begin{equation}
	\begin{split}
		\expval{
			\hat{a}(\vb{q})
			\hat{N}
			\hat{a}^\dagger(\vb{p})
		}{0}
		&=
		\int\frac{\dd[3]{k}}{(2\pi)^3}
		\expval{
			\hat{a}(\vb{q})
			\hat{a}^\dagger(\vb{k})
			\hat{a}(\vb{k})
			\hat{a}^\dagger(\vb{p})
		}{0}
		\\
		&=
		\int\frac{\dd[3]{k}}{(2\pi)^3}
		(2\pi)^3\delta^{(3)}(\vb{k}-\vb{q})
		\expval{
			\hat{a}(\vb{k})
			\hat{a}^\dagger(\vb{p})
		}{0}
		\\
		&=
		\expval{
			\hat{a}(\vb{q})
			\hat{a}^\dagger(\vb{p})
		}{0}
		=
		(2\pi)^3\delta^{(3)}(\vb{q}-\vb{p})
	\end{split}
\end{equation}
and
\begin{equation}
	\begin{split}
		\expval{\hat{a}(\vb{q})\hat{N}^2\hat{a}^\dagger(\vb{p})}{0}
		&=
		\int\frac{\dd[3]{k_1}}{(2\pi)^3}
		\int\frac{\dd[3]{k_2}}{(2\pi)^3}
		\\
		&\times
		\expval{
			\hat{a}(\vb{q})
			\hat{a}^\dagger(\vb{k}_1)
			\hat{a}(\vb{k}_1)
			\hat{a}^\dagger(\vb{k}_2)
			\hat{a}(\vb{k}_2)
			\hat{a}^\dagger(\vb{p})
		}{0}
		\\
		&=
		\int\frac{\dd[3]{k_1}}{(2\pi)^3}
		\int\frac{\dd[3]{k_2}}{(2\pi)^3}
		\\
		&\times
		(2\pi)^3\delta^{(3)}(\vb{k}_1-\vb{q})
		\expval{
			\hat{a}(\vb{k}_1)
			\hat{a}^\dagger(\vb{k}_2)
		}{0}
		(2\pi)^3\delta^{(3)}(\vb{k}_2-\vb{p})
		\\
		&=
		\expval{
			\hat{a}(\vb{q})
			\hat{a}^\dagger(\vb{p})
		}{0}
		=
		(2\pi)^3\delta^{(3)}(\vb{q}-\vb{p})
	\end{split}
\end{equation}
Most trivially, we find the first moments of the number operator to equal
\begin{equation}
	\begin{split}
		\expval{\hat{N}}{f}
		&=
		\int\frac{\dd[3]{p}}{(2\pi)^3\sqrt{2\omega(\vb{p})}}
		\int\frac{\dd[3]{q}}{(2\pi)^3\sqrt{2\omega(\vb{q})}}
		f(\vb{p})f(\vb{q})^*
		\expval{\hat{a}(\vb{q})\hat{N}\hat{a}^\dagger(\vb{p})}{0}
		\\
		&=
		\int\frac{\dd[3]{p}}{(2\pi)^3\sqrt{2\omega(\vb{p})}}
		\int\frac{\dd[3]{q}}{(2\pi)^3\sqrt{2\omega(\vb{q})}}
		f(\vb{p})f(\vb{q})^*
		(2\pi)^3\delta^{(3)}(\vb{q}-\vb{p})
		\\
		&=
		\int\frac{\dd[3]{p}}{(2\pi)^32\omega(\vb{p})}
		\abs{f(\vb{p})}^2
		=
		1
	\end{split}
\end{equation}
and the second moment to equal
\begin{equation}
	\begin{split}
		\expval{\hat{N}^2}{f}
		&=
		\int\frac{\dd[3]{p}}{(2\pi)^3\sqrt{2\omega(\vb{p})}}
		\int\frac{\dd[3]{q}}{(2\pi)^3\sqrt{2\omega(\vb{q})}}
		f(\vb{p})f(\vb{q})^*
		\expval{\hat{a}(\vb{q})\hat{N}^2\hat{a}^\dagger(\vb{p})}{0}
		\\
		&=
		\int\frac{\dd[3]{p}}{(2\pi)^3\sqrt{2\omega(\vb{p})}}
		\int\frac{\dd[3]{q}}{(2\pi)^3\sqrt{2\omega(\vb{q})}}
		f(\vb{p})f(\vb{q})^*
		(2\pi)^3\delta^{(3)}(\vb{q}-\vb{p})
		\\
		&=
		\int\frac{\dd[3]{p}}{(2\pi)^32\omega(\vb{p})}
		\abs{f(\vb{p})}^2
		=
		1
	\end{split}
\end{equation}
such that the variance vanishes
\begin{equation}
	\expval{(\Delta\hat{N})^2}{f}
	=
	\expval{\hat{N}^2}{f}
	-
	\expval{\hat{N}}{f}^2
	=
	0
\end{equation}
suggesting $\ket{f}$ to be a single-particle number state.
Similar the first moment of the total energy is
\begin{equation}
	\begin{split}
		\expval{\hat{H}}{f}
		&=
		\int\frac{\dd[3]{p}}{(2\pi)^3\sqrt{2\omega(\vb{p})}}
		\int\frac{\dd[3]{q}}{(2\pi)^3\sqrt{2\omega(\vb{q})}}
		f(\vb{p})f(\vb{q})^*
		\expval{\hat{a}(\vb{q})\hat{H}\hat{a}^\dagger(\vb{p})}{0}
		\\
		&=
		\int\frac{\dd[3]{p}}{(2\pi)^3\sqrt{2\omega(\vb{p})}}
		\int\frac{\dd[3]{q}}{(2\pi)^3\sqrt{2\omega(\vb{q})}}
		f(\vb{p})f(\vb{q})^*
		\omega(\vb{p})
		(2\pi)^3\delta^{(3)}(\vb{q}-\vb{p})
		\\
		&=
		\int\frac{\dd[3]{p}}{(2\pi)^32\omega(\vb{p})}
		\omega(\vb{p})
		\abs{f(\vb{p})}^2
	\end{split}
\end{equation}
where we used
\begin{equation}
	\begin{split}
		\expval{\hat{a}(\vb{q})\hat{H}\hat{a}^\dagger(\vb{p})}{0}
		&=
		\int\frac{\dd[3]{k}}{(2\pi)^3}
		\omega(\vb{k})
		\expval{\hat{a}(\vb{q})\hat{a}^\dagger(\vb{k})\hat{a}(\vb{k})\hat{a}^\dagger(\vb{p})}{0}
		\\
		&=
		\int\frac{\dd[3]{k}}{(2\pi)^3}
		\omega(\vb{k})
		\expval{\hat{a}(\vb{q})\hat{a}^\dagger(\vb{k})}{0}
		(2\pi)^3\delta^{(3)}(\vb{k}-\vb{p})
		\\
		&=
		\omega(\vb{p})
		\expval{\hat{a}(\vb{q})\hat{a}^\dagger(\vb{p})}{0}
		=
		\omega(\vb{p})
		(2\pi)^3\delta^{(3)}(\vb{q}-\vb{p})
	\end{split}	
\end{equation}
and the second moment is
\begin{equation}
	\begin{split}
		\expval{\hat{H}^2}{f}
		&=
		\int\frac{\dd[3]{p}}{(2\pi)^3\sqrt{2\omega(\vb{p})}}
		\int\frac{\dd[3]{q}}{(2\pi)^3\sqrt{2\omega(\vb{q})}}
		f(\vb{p})f(\vb{q})^*
		\expval{\hat{a}(\vb{q})\hat{H}^2\hat{a}^\dagger(\vb{p})}{0}
		\\
		&=
		\int\frac{\dd[3]{p}}{(2\pi)^3\sqrt{2\omega(\vb{p})}}
		\int\frac{\dd[3]{q}}{(2\pi)^3\sqrt{2\omega(\vb{q})}}
		f(\vb{p})f(\vb{q})^*
		\omega(\vb{p})^2
		(2\pi)^3\delta^{(3)}(\vb{q}-\vb{p})
		\\
		&=
		\int\frac{\dd[3]{p}}{(2\pi)^32\omega(\vb{p})}
		\omega(\vb{p})^2
		\abs{f(\vb{p})}^2
	\end{split}
\end{equation}
where we used
\begin{equation}
	\begin{split}
		\expval{\hat{a}(\vb{q})\hat{H}^2\hat{a}^\dagger(\vb{p})}{0}
		&=
		\int\frac{\dd[3]{k_1}}{(2\pi)^3}
		\int\frac{\dd[3]{k_2}}{(2\pi)^3}
		\omega(\vb{k}_1)\omega(\vb{k}_2)
		\\
		&\times
		\expval{
			\hat{a}(\vb{q})
			\hat{a}^\dagger(\vb{k}_1)
			\hat{a}(\vb{k}_1)
			\hat{a}^\dagger(\vb{k}_2)
			\hat{a}(\vb{k}_2)
			\hat{a}^\dagger(\vb{p})
		}{0}
		\\
		&=
		\int\frac{\dd[3]{k_1}}{(2\pi)^3}
		\int\frac{\dd[3]{k_2}}{(2\pi)^3}
		\omega(\vb{k}_1)\omega(\vb{k}_2)
		\\
		&\times
		(2\pi)^3\delta^{(3)}(\vb{k}_1-\vb{q})
		\expval{
			\hat{a}(\vb{k}_1)
			\hat{a}^\dagger(\vb{k}_2)
		}{0}
		(2\pi)^3\delta^{(3)}(\vb{k}_2-\vb{p})
		\\
		&=
		\omega(\vb{q})\omega(\vb{p})
		\expval{
			\hat{a}(\vb{q})
			\hat{a}^\dagger(\vb{p})
		}{0}
		\\
		&=
		\omega(\vb{p})^2
		(2\pi)^3\delta^{(3)}(\vb{q}-\vb{p})
	\end{split}
\end{equation}
Before we calculate the energy fluctuations, we need
\begin{equation}
	\begin{split}
		\expval{\hat{H}}{f}^2
		&=
		\int\frac{\dd[3]{p}}{(2\pi)^32\omega(\vb{p})}
		\omega(\vb{p})
		\abs{f(\vb{p})}^2
		\int\frac{\dd[3]{q}}{(2\pi)^32\omega(\vb{q})}
		\omega(\vb{q})
		\abs{f(\vb{q})}^2
	\end{split}	
\end{equation}
The energy fluctuations turn out to be non-zero (?)
\begin{equation}
	\begin{split}
		\expval{\left(\Delta\hat{H}\right)^2}{f}
		&=
		\expval{\hat{H}^2}{f}
		-
		\expval{\hat{H}}{f}^2
		>
		0
	\end{split}
\end{equation}

By an completely analog calculation, we find for the total momentum
\begin{align}
	\expval{\hat{\vb{P}}}{f}
	=
	\int\frac{\dd[3]{p}}{(2\pi)^32\omega(\vb{p})}
	\vb{p}
	\abs{f(\vb{p})}^2
	&&
	\expval{\hat{\vb{P}}^2}{f}
	=
	\int\frac{\dd[3]{p}}{(2\pi)^32\omega(\vb{p})}
	\vb{p}^2
	\abs{f(\vb{p})}^2
\end{align}
in agreement with results reported in~\cite[eqs.~10 and 11]{Naumov2013}.

\subsubsection{Expectation values of wave packet observables}

From the spacetime representation of the wave function
\begin{equation}
	\psi(t,\vb{x})
	=
	\bra{0}\hat\phi(t,\vb{x})\ket{f}
\end{equation}
we can calculate the probability current
\begin{equation}
	j^\mu
	=
	i\left(
		\psi^*\partial_\mu\psi
		-
		\psi\partial_\mu\psi^*
	\right)
	\label{eq:qkg_probability_current}
\end{equation}
from which we can calculate many interesting properties~\cite{Naumov2013}.
The probability density $\rho=j_0$ has proper normalization
\begin{equation}
	\int\dd[3]{x}\rho(t,\vb{x})
	=
	1
\end{equation}
and the probability current yields the (group) velocity
\begin{equation}
	\int\dd[3]{x}\vb{j}(t,\vb{x})
	=
	\overline{\vb{v}}
	\label{qkg:group_velocity}.
\end{equation}
The mean center-of-mass can be expressed in terms of the velocity
\begin{equation}
	\overline{\vb{x}}
	=
	\int\dd[3]{x}\vb{x}\rho(t,\vb{x})
	=
	\overline{\vb{v}}t
	\label{qkg:mean_position}
\end{equation}
which gives the smeared single-particle state a sensible interpretation in terms of localization and speed.

The time-dependent variance of the position
\begin{equation}
	\sigma_x^2(t)
	=
	\sigma_x^2(0)
	+
	\sigma_v^2t^2
\end{equation}
can be interpreted as dispersion.

\subsubsection{Covariant Gaussian wave packets}

Naumov~\cite{Naumov2013} discusses a covariant Gaussian wave packet. In particular
\begin{equation}
	f(\vb{p})
	\propto
	\exp\left\{\frac{(p_\mu-k_\mu)(p^\mu-k^\mu)}{4\sigma_p^2}\right\}
\end{equation}
describes a covariant Gaussian wave packet localized at $\vb{k}$ in momentum space.
Assuming $\sigma_P$ to be small, we can perform a series expansion of the dispersion relation and obtain
\begin{equation}
	f(\vb{p})
	=
	\sqrt{2m}
	\left(\frac{2\pi}{\sigma_p^2}\right)^{3/4}
	\exp\left\{
		-\frac{(\vb{p}-\vb{k}_\parallel)^2}{4\sigma_{p\parallel}^2}
	\right\}
\end{equation}

\subsection{Interactions and coherent states}

See also \cite{Itzykson2012}

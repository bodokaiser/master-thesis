\section{Klein-Gordon field}

Choosing a reference frame in which the polarization basis is independent of the momentum, we find that
\begin{equation}
	\hat{\vb{A}}(t,\vb{x})
	=
	\sum_{\lambda=1,2}
	c_\lambda
	\hat\phi(t,\vb{x})
	\boldsymbol{\varepsilon}_\lambda
\end{equation}
with $c_1,c_2\in\mathbb{C}$ and $\abs{c_1}^2+\abs{c_2}^2=1$.
Alternatively, we can perform the replacement
\begin{equation}
	\hat{a}(\vb{p})
	\to
	\sum_{\lambda=1,2}
	\hat{a}_\lambda(\vb{p})
	\boldsymbol{\varepsilon}_\lambda(\vb{p})
\end{equation}
in our results obtained with the Klein-Gordon field to transfer the results to the Maxwell field in the Coulomb gauge.

\subsection{Relativistic field theory}

\begin{definition}[Klein-Gordon Lagrangian]
	The Lorentz-invariant Lagrangian density
	\begin{equation}
		\mathcal{L}
		=
		\frac{1}{2}
		\left(\partial_\mu\phi\right)
		\left(\partial^\mu\phi\right)
		-
		\frac{1}{2}
		m^2\phi^2
		\label{eq:kg_lagrangian}
	\end{equation}
	describes a real scalar field $\phi(t,\vb{x})$ with mass $m>0$, the (massive) Klein-Gordon field.
\end{definition}
\begin{theorem}[Relativistic energy-momentum relation]\label{th:relativistic_energy_momentum}
	Excitations of the Klein-Gordon field satisfy the relativistic energy-momentum relation
	\begin{equation}
		\omega(\vb{p})
		=
		\sqrt{\vb{p}^2+m^2}
		=
		E(\vb{p})
		\label{eq:energy_momentum_relation}
	\end{equation}
	and the physical momentum space is constrained to the momentum lightcone
	\begin{equation}
		V
		=
		\left\{
			(p_0,\vb{p})\in\mathbb{R}^4
			\mid
			p_0^2=\omega(\vb{p})^2
		\right\}
		\label{eq:momentum_lightcone}
		.
	\end{equation}
\end{theorem}
\begin{theorem}[Mode expansion of the Klein-Gordon field]\label{thm:kg_fourier_expansion}
	The Fourier expansion of the Klein-Gordon field
	\begin{equation}
		\phi(t,\vb{x})
		=
		\int_{\mathbb{R}^3}\frac{\dd[3]{p}}{(2\pi)^3\sqrt{2\omega(\vb{p})}}
		\biggl\{
			a(\vb{p})^*
			e^{-ip_\mu x^\mu}
			+
			a(\vb{p})
			e^{+ip_\mu x^\mu}
		\biggr\}_{p_0=\omega(\vb{p})}
	\end{equation}
	satisfies the equations of motion for any choice of $a(\vb{p})=\phi(\omega(\vb{p}),\vb{p})^*$.
\end{theorem}
From now on assume $p_0=\omega(\vb{p})$ if not stated otherwise.
\begin{corollary}[Conjugate momentum density of the Klein-Gordon field]
	The conjugate momentum density of the Klein-Gordon field is
	\begin{equation}
		\pi(t,\vb{x})
		=
		-i
		\int\frac{\dd[3]{p}}{(2\pi)^3}
		\sqrt{2\omega(\vb{p})}
		\left\{
			a(\vb{p})
			e^{-ip_\mu x^\mu}
			-
			a(\vb{p})^*
			e^{+ip_\mu x^\mu}
		\right\}
	\end{equation}
	which follows directly from the Legendre transformation
	\begin{equation}
		\pi(t,\vb{x})
		=
		\partial_t\pdv{\mathcal{L}}{(\partial_t\phi)}
		=
		\partial_t\phi(t,\vb{x})
	\end{equation}
	fundamental to Hamiltonian mechanics.
\end{corollary}
\begin{definition}[Energy-momentum tensor]
	We define the energy-momentum tensor
	\begin{equation}
		T^{\mu\nu}
		=
		\pdv{\mathcal{L}}{(\partial_\mu\phi)}\partial^\nu\phi
		-
		g^{\mu\nu}\mathcal{L}
		\label{eq:energy_momentum_tensor}
	\end{equation}
	which encodes the field's observables in its components.
	For instance, the energy density is encoded in the
	\begin{equation*}
		T^{00}
		=
		\frac{1}{2}
		\left(\partial_t\phi\right)^2
		+
		\frac{1}{2}
		\left(\grad\phi\right)^2
		+
		\frac{1}{2}
		\left(m\phi\right)^2		
	\end{equation*}
	component and the momentum density in the
	\begin{equation*}
		T^{0i}
		=
		-\pi\partial_i\phi		
	\end{equation*}
	component, see Ref.~\cite{Peskin1995}.
\end{definition}
\begin{lemma}\label{thm:kg_total_energy_momentum}
	The total energy and momentum of the Klein-Gordon field,
	\begin{align}
		H
		=
		\int\frac{\dd[3]{p}}{(2\pi)^3}
		\omega(\vb{p})\abs{a(\vb{p})}^2
		&&
		\vb{P}
		=
		\int\frac{\dd[3]{p}}{(2\pi)^3}
		\vb{p}\abs{a(\vb{p})}^2
		\label{eq:kg_energy_momentum}
		,
	\end{align}
	are the energy and momentum weighted by the mode amplitudes.
\end{lemma}

\subsection{Canonical quantization}

\begin{definition}[Canonical quantization]
	In the canonical quantization procedure, the dynamical variables are promoted to operators satisfying the equal-time canonical commutation relations
	\begin{align}
		\comm{\hat\phi(\vb{x})}{\hat\pi(\vb{y})}
		&=
		i\delta^{(3)}(\vb{x}-\vb{y})
		\\
		\comm{\hat\phi(\vb{x})}{\hat\phi(\vb{y})}
		&=
		\comm{\hat\pi(\vb{x})}{\hat\pi(\vb{y})}
		=
		0
		\label{eq:qkg_comm_pm}
		.
	\end{align}
\end{definition}
\begin{theorem}[Klein-Gordon field operators]
	The Klein-Gordon field operators are
	\begin{align}
		\hat\phi(t,\vb{x})
		&=
		\int\frac{\dd[3]{p}}{(2\pi)^3}
		\frac{1}{\sqrt{2\omega(\vb{p})}}
		\left\{
			\hat{a}(\vb{p})
			e^{-ip_\mu x^\mu}
			+
			\hat{a}^\dagger(\vb{p})
			e^{+ip_\mu x^\mu}
		\right\}
		\label{eq:qkg_pos}
		\\
		\hat\pi(t,\vb{x})
		&=
		\int\frac{\dd[3]{p}}{(2\pi)^3}
		\left(-i\sqrt{2\omega(\vb{p})}\right)
		\left\{
			\hat{a}(\vb{p})
			e^{-ip_\mu x^\mu}
			-
			\hat{a}^\dagger(\vb{p})
			e^{+ip_\mu x^\mu}
		\right\}
		\label{eq:qkg_mom}
		,
	\end{align}
	where $\hat\phi(t,\vb{x})$ and $\hat\pi(t,\vb{x})$ satisfy the equal-time canonical commutation relations.
\end{theorem}
\begin{theorem}\label{thm:kg_comm_ac}
	The annihilation and creation operator of the Klein-Gordon field obey the commutation relations
	\begin{align}
		\comm{\hat{a}(\vb{p})}{\hat{a}^\dagger(\vb{q})}
		&=
		(2\pi)^3
		\delta^{(3)}(\vb{p}-\vb{q})
		\\
		\comm{\hat{a}^\dagger(\vb{p})}{\hat{a}^\dagger(\vb{q})}
		&=
		\comm{\hat{a}(\vb{p})}{\hat{a}(\vb{q})}
		=
		0
		\label{eq:kg_comm_ac}
		.
	\end{align}	
\end{theorem}
\begin{definition}[Normal-ordered product]
	The normal-ordered product of annihilation and creation operators
	\begin{equation}
		N\left\{
			\left(
				\hat{a}^\dagger
				\hat{a}
			\right)^l
		\right\}
		=
		\left(\hat{a}^\dagger\right)^l
		\hat{a}^l
	\end{equation}
	places all creation operators to the left and all annihilation operators to the right.
\end{definition}
\begin{definition}[Correspondence principle]
	The correspondence principle is a prescription to find the quantum from the classical observables by promoting the dynamical variables to operators in normal-order.\footnote{See Ref.~\cite[p.~20]{Mukhanov2007} for details on the problem of operator ordering.}
\end{definition}
\begin{corollary}
	The total energy and momentum operators of the Klein-Gordon field are
	\begin{align}
		\hat{H}
		=
		\int\frac{\dd[3]{p}}{(2\pi)^3}
		\omega(\vb{p})\hat{a}^\dagger(\vb{p})\hat{a}(\vb{p})
		&&
		\hat{\vb{P}}
		=
		\int\frac{\dd[3]{p}}{(2\pi)^3}
		\vb{p}\hat{a}^\dagger(\vb{p})\hat{a}(\vb{p})
		\label{eq:qkg_energy_momentum}
		.
	\end{align}
\end{corollary}
\begin{corollary}
	The number operator is the unweighted part of the energy and momentum operators
	\begin{equation}
		\hat{N}
		=
		\int\frac{\dd[3]{p}}{(2\pi)^3}
		\hat{a}^\dagger(\vb{p})
		\hat{a}(\vb{p})
		\label{eq:qkg_number}
		.
	\end{equation}
\end{corollary}
\begin{definition}[Positive and negative frequency Klein-Gordon field operator]
	The positive and negative frequency decomposition of the Klein-Gordon field operator
	\begin{equation}
		\hat\phi(t,\vb{x})
		=
		\hat\phi^+(t,\vb{x})
		+
		\hat\phi^-(t,\vb{x})
	\end{equation}
	wherein the positive and negative frequency Klein-Gordon field operators
	\begin{equation}
		\begin{split}
			\hat\phi^+(t,\vb{x})
			&=
			\int\frac{\dd[3]{p}}{(2\pi)^3\sqrt{2\omega(\vb{p})}}
			e^{-ip_\mu x^\mu}
			\hat{a}(\vb{p})
			\\
			\hat\phi^-(t,\vb{x})
			&=
			\int\frac{\dd[3]{p}}{(2\pi)^3\sqrt{2\omega(\vb{p})}}
			e^{+ip_\mu x^\mu}
			\hat{a}^\dagger(\vb{p})
		\end{split}
		\label{eq:qkg_positive_negative_frequency}
	\end{equation}
	are related by the hermitian conjugate $\hat\phi^-(t,\vb{x})=\hat\phi^+(t,\vb{x})^\dagger$, see Ref.~\cite[p.~26]{Peskin1995} for details.
\end{definition}

\subsection{Coordinate wave function properties}

\begin{definition}[Coordinate wave function]
	The equivalent of a coordinate wave function given a state $\ket{\psi}$ and a field operator $\hat\phi$ is
	\begin{equation}
		\psi(t,\vb{x})
		=
		\bra{0}\hat\phi(t,\vb{x})\ket{\psi}
		.
	\end{equation}
\end{definition}
\begin{definition}[Probability current]
	The manifest Lorentz-covariant probability current\footnote{The probability current is the conserved Noether's current of the Klein-Gordon field, see Ref.~\cite[p.~18]{Peskin1995}.} is
	\begin{equation}
		j_\mu(t,\vb{x})
		=
		2
		\Im\left\{
			\psi(t,\vb{x})^*
			\partial_\mu
			\psi(t,\vb{x})
		\right\}
		\label{eq:qkg_probability_current}
	\end{equation}
	where $\psi(t,\vb{x})$ is a coordinate wave function.
	The time component $j^0(t,\vb{x})$ is equal to the probability density $\rho(t,\vb{x})$ and the spatial components $j^i(t,\vb{x})$ are equal to the probability current.
\end{definition}
\begin{definition}[Localization]
	The center-of-mass position of the probability density,
	\begin{equation}
		\expval{\vb{x}(t)}
		=
		\int\dd[3]{x}
		\vb{x}
		\rho(t,\vb{x})
		,
	\end{equation}
	allows us to make statements about the localization of a field excitation.\footnote{There exists no position operator free of contradictions in quantum field theory!}
\end{definition}
\begin{definition}[Group velocity]
	The total probability current
	\begin{equation}
		\expval{\vb{v}(t)}
		=
		\int\dd[3]{x}
		\vb{j}(t,\vb{x})
		\label{eq:group_velocity}
	\end{equation}
	equals the group velocity of a field excitation.
\end{definition}
\begin{definition}[Spatial dispersion]
	The spatial dispersion is equal to the variance of the position weighted by the probability density
	\begin{equation}
		\sigma_x(t)^2
		=
		\expval{\vb{x}(t)^2}
		-
		\expval{\vb{x}(t)}^2
	\end{equation}
	and quantifies the spatial spread of a wave packet.
\end{definition}
\addsec{Summary}

\subsection*{Time-dependent coherent states}

The most important result of the present chapter is a self-consistent derivation of how to represent a time-continuous continuous-mode coherent state, i.e.,
\begin{equation}
	\ket{\alpha(t)}
	=
	\exp\left\{
		\sum_{\lambda=1,2}
		\int\frac{\dd[3]{p}}{(2\pi)^3\sqrt{2\omega(\vb{p})}}
		\left\{
			\alpha(t,\vb{p})
			\hat{a}_\lambda(\vb{p})
			-
			\alpha(t,\vb{p})^*
			\hat{a}_\lambda^\dagger(\vb{p})
		\right\}
	\right\}
	.
\end{equation}

\subsection*{Time-dependent interactions}

\subsection*{Assumptions and approximations}

By selecting a not manifest Lorentz-covariant gauge, the Coulomb gauge, we restrict the validity of our results to a fixed inertial frame.
Whenever we want to perform Lorentz boosts, relevant for, e.g., the Unruh effect, our description starts to breaks down.

When considering wave packets with a sharply peaked momentum distribution, i.e., traveling in one direction, we can approximate the Maxwell field as one-dimensional
\begin{equation}
	A(t,\vb{x})
	=
	\int_{-\infty}^{+\infty}
	\frac{\dd{p}}{(2\pi)\sqrt{2p}}
	\left\{
		\hat{a}(\vb{p})
		e^{-ip(t-x)}
		+
		\text{h.c.}
	\right\}
	.
\end{equation}
The integration domain can be further limited if one considers only forward propagating wave packets to
\begin{equation}
	A(t,\vb{x})
	=
	\int_0^{+\infty}
	\frac{\dd{p}}{(2\pi)\sqrt{2p}}
	\left\{
		\hat{a}(\vb{p})
		e^{-ip(t-x)}
		+
		\text{h.c.}
	\right\}
	.
\end{equation}
In the one-dimensional approximation, the Maxwell field is effectively described as a Klein-Gordon field.

The most commonly employed approximation concerns the polarization degrees of freedom of the field.
The polarization becomes important whenever the spin of the photon needs to be considered.
Neglecting polarization, the Maxwell field takes the scalar form

An additional approximation which is usually made is to drop the relativistic correction if the integration measure
\begin{equation*}
	\frac{\dd[3]{p}}{(2\pi)^3\sqrt{2\omega(\vb{p})}}
	\to
	\frac{\dd[3]{p}}{(2\pi)^3}
	.
\end{equation*}
There are two arguments: first one can argue that any experimentally measured quantity includes the relativistic factor and therefore we can absorb $1/\sqrt{2\omega(\vb{p})}$ into our wave packet.
Second, any practical measurement has a finite bandwidth over which the $1/\sqrt{2\omega(\vb{p})}$ should have negligible effect and can be removed by the mean value theorem for definite integrals.
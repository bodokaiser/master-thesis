\section{Approximations and summary}

In the previous sections we derived the number state which is an eigenstate of the number operator $\hat{N}$
\begin{align}
	\ket{n}
	&=
	\frac{1}{n!}
	\left(
		\int\frac{\dd[3]{p}}{(2\pi)^3\sqrt{2\omega(\vb{p})}}
		f(\vb{p})
		\hat{a}^\dagger(\vb{p})
	\right)^n
	\ket{0}
	\\
	\ket{\alpha}
	&=
	e^{-\overline{n}/2}
	\exp\left\{
		\int\frac{\dd[3]{p}}{(2\pi)^3\sqrt{2\omega(\vb{p})}}
		\alpha(\vb{p})
		\hat{a}^\dagger(\vb{p})
	\right\}
	\ket{0}
\end{align}
and the coherent state which is an eigenstate of the annihilation operator.
For the number observable of these two states, we found
\begin{align}
	\expval{\hat{N}}{n}
	=
	n
	&&
	\expval{(\Delta\hat{N})^2}{n}
	=
	0
	\\
	\expval{\hat{N}}{\alpha}
	=
	\overline{n}
	&&
	\expval{(\Delta\hat{N})^2}{n}
	=
	\overline{n}
\end{align}
while for the energy observable, we found.

Our definitions of the number and coherent state are manifest Lorentz-covariant as the integral measure and the integrand are both Lorentz-invariant.
However, our measurements are not Lorentz-invariant, i.e., our spectrum analyzer measures $\beta(\vb{p})=\alpha(\vb{p})/\sqrt{2\omega(\vb{p})}$ and not $\alpha(\vb{p})$ for the spectrum.
Therefore, we can redefine the spectrum and use
\begin{align}
	\ket{n}
	&=
	\frac{1}{n!}
	\left(
		\int\frac{\dd[3]{p}}{(2\pi)^3}
		g(\vb{p})
		\hat{a}^\dagger(\vb{p})
	\right)^n
	\ket{0}
	\\
	\ket{\beta}
	&=
	e^{-\overline{n}/2}
	\exp\left\{
		\int\frac{\dd[3]{p}}{(2\pi)^3}
		\beta(\vb{p})
		\hat{a}^\dagger(\vb{p})
	\right\}
	\ket{0}
\end{align}
without loss of generality.\footnote{We kept the Lorentz-invariant measure and spectrum throughout the chapter to compare the results directly with the literature.}
\section{Time-dependent interactions}

\subsection{Time-evolution operator}

Let $\ket{\psi(t_0)}$ be a state at time $t_0$, then the time-evolution relates the state $\ket{\psi(t)}$ at some later time $t>t_0$ to $\ket{\psi(t_0)}$ via
\begin{equation}
	\ket{\psi(t)}
	=
	\hat{U}(t,t_0)
	\ket{\psi(t_0)}
	.
\end{equation}
Inserting $\ket{\psi(t)}$ into the Schrödinger equation leads to
\begin{equation}
	i\dv{t}
	\hat{U}(t,t_0)
	=
	\hat{H}(t)
	\hat{U}(t,t_0)
\end{equation}
which formal solution is the time-ordered exponential, see Ref.~\cite[p.~380]{Bartelmann2018},
\begin{equation}
	\hat{U}(t,t_0)
	=
	T\exp\left\{
		-i
		\int_{t_0}^t\dd{t^\prime}
		\hat{H}(t^\prime)
	\right\}
\end{equation}
where $T$ denotes the time-ordering symbol.
Only for simple time-dependent systems an exact time-evolution operator exists.
In contrast to the Dyson expansion, the Magnus expansion yields a unitary time-evolution operator even for finite order, in particular,
\begin{equation}
	\hat{U}(t,t_0)
	=
	\exp\left\{
		\sum_{n=1}
		\hat{\Omega}^{(n)}(t,t_0)
	\right\}
\end{equation}
where the first two expansion terms are given by
\begin{align}
	\hat{\Omega}^{(1)}(t,t_0)
	&=
	\frac{(-i)}{1!}
	\int_{t_0}^t\dd{t^\prime}
	\hat{H}(t^\prime)
	\\
	\hat{\Omega}^{(2)}(t,t_0)
	&=
	\frac{(-i)^2}{2!}
	\int_{t_0}^t\dd{t^\prime}
	\int_{t_0}^{t^\prime}\dd{t^{\prime\prime}}
	\comm{\hat{H}(t^\prime)}{\hat{H}(t^{\prime\prime})}
\end{align}
and represent time-ordering corrections, see Ref.~\cite{QuesadaMejia2015}.

\subsection{Interaction with a classical current field}

The Schrödinger-picture Hamiltonian describing the interaction of the Maxwell field $\hat{\vb{A}}$ with a classical current $\vb{j}$ is
\begin{equation}
	\hat{H}_\text{int}(t)
	=
	-
	\int_{\mathbb{R}^3}\dd[3]{x}
	\vb{j}(t,\vb{x})
	\vdot
	\hat{\vb{A}}(t,\vb{x})
	.
\end{equation}
Taking the spatial Fourier transform of the current
\begin{equation}
	\vb{j}(t,\vb{x})
	=
	\int_{\mathbb{R}^3}\frac{\dd[3]{q}}{(2\pi)^3}
	\vb{j}(t,\vb{q})
	e^{+i\vb{q}\vdot\vb{x}}
\end{equation}
and inserting the mode expansion, we find the interaction Hamiltonian to be
\begin{equation}
	\hat{H}_\text{int}(t)
	=
	-
	\sum_{\lambda=1,2}
	\int_{\mathbb{R}^3}\frac{\dd[3]{p}}{(2\pi)^3\sqrt{2\omega(\vb{p})}}
	\left\{
		j_\lambda(t,\vb{p})
		\hat{a}_\lambda(\vb{p})
		e^{-i\omega(\vb{p})t}
		+
		\text{h.c.}
	\right\}
\end{equation}
where we have the transverse current
\begin{equation}
	j_\lambda(q_0,\vb{p})
	=
	\vb{j}(q_0,\vb{p})
	\vdot
	\vu{e}_\lambda(\vb{p})
	.
\end{equation}
The first term in the Magnus expansion turns out to be
\begin{equation}
	\hat{\Omega}^{(1)}(t,t_0)
	=
	i
	\sum_{\lambda=1,2}
	\int_{\mathbb{R}^3}\frac{\dd[3]{p}}{(2\pi)^3\sqrt{2\omega(\vb{p})}}
	\left\{
		J_\lambda(t,t_0;\vb{p})
		\hat{a}_\lambda(\vb{p})
		+
		\text{h.c.}
	\right\}
\end{equation}
where we defined
\begin{equation}
	J_\lambda(t,t_0;\vb{p})
	=
	\int_{t_0}^t\dd{t^\prime}
	j_\lambda(t^\prime,\vb{p})
	e^{-i\omega(\vb{p})t^\prime}
	.
\end{equation}
For the second term in the Magnus expansion, we first evaluate the commutator
\begin{equation}
	\comm{\hat{H}(t^\prime)}{\hat{H}(t^{\prime\prime})}
	=
	i\sum_{\lambda=1,2}
	\int_{\mathbb{R}^3}\frac{\dd[3]{p}}{(2\pi)^3\omega(\vb{p})}
	\Im\left\{
		j_\lambda(t^\prime,\vb{p})
		j_\lambda(t^{\prime\prime},\vb{p})^*
		e^{-i\omega(\vb{p})(t^\prime-t^{\prime\prime})}
	\right\}
\end{equation}
and notice that it is complex valued, hence, all higher commutators vanish and the Magnus expansion with the first two terms is exact.
In summary, the second term of the Magnus expansion turns out to be
\begin{equation}
	\hat{\Omega}^{(2)}(t,t_0)
	=
	i\sum_{\lambda=1,2}
	\int_{\mathbb{R}^3}\frac{\dd[3]{p}}{(2\pi)^3\omega(\vb{p})}
	\Im\left\{
		J_\lambda(t_0,t^\prime;\vb{p})
		J_\lambda(t_0,t^{\prime\prime};\vb{p})^*
	\right\}
\end{equation}
The second term contributes a phase to the time-evolution operator.
As long as we consider a single current source, no interference of phases can occur and we can ignore the phase factor.
The exact time-evolution operator of the Maxwell field interacting with a classical source current therefore is
\begin{equation}
	\hat{U}(t,t_0)
	=
	\exp\left\{
		i\sum_{\lambda=1,2}
		\int_{\mathbb{R}^3}\frac{\dd[3]{p}}{(2\pi)^3\sqrt{2\omega(\vb{p})}}
		\left\{
			J_\lambda(t,t_0;\vb{p})
			\hat{a}_\lambda(\vb{p})
			+
			\text{h.c.}
		\right\}
	\right\}
\end{equation}
which equals the displacement operator for a time-dependent spectrum $\hat{D}[\alpha(t,t_0)]$.

\subsection{Interaction with a charged particle in a potential}

Based on Refs.~\cite[p.~687]{Mandel1995},~\cite[p.~128]{Cohen1992}

The Hamiltonian of the particle with charge $q$ and mass $m$
\begin{equation}
	\hat{H}_q
	=
	\frac{1}{2m}
	\hat{\vb{p}}^2
	+
	qV(\hat{\vb{x}})
\end{equation}
wherein $V$ is a classical external potential and the position and momentum operator satisfy the canonical commutation relations
\begin{align}
	\comm{\hat{x}_i}{\hat{p}_j}
	&=
	i\delta_{ij}
	&
	\comm{\hat{x}_i}{\hat{x}_j}
	&=
	0
	=
	\comm{\hat{p}_i}{\hat{p}_j}
	.
\end{align}
Interaction of the charged particle with an electromagnetic potential $\hat{\vb{A}}$ due to minimal coupling os obtained by the replacement~\cite{Itzykson2012}
\begin{equation}
	\hat{\vb{p}}
	\to
	\hat{\vb{p}}
	-
	q\vb{\hat{A}}
\end{equation}
and expanding the kinetic term of the particle's Hamiltonian
\begin{equation}
	\begin{split}
		\frac{1}{2m}
		\left(\hat{\vb{p}}-q\hat{\vb{A}}\right)^2
		&=
		\frac{1}{2m}
		\hat{\vb{p}}^2
		-
		\frac{q}{2m}
		\left(
			\hat{\vb{p}}
			\vdot
			\hat{\vb{A}}
			+
			\hat{\vb{A}}
			\vdot
			\hat{\vb{p}}
		\right)
		+
		\frac{q^2}{2m}
		\hat{\vb{A}}^2
		\\
		&=
		\frac{1}{2m}
		\hat{\vb{p}}^2
		-
		\frac{q}{m}
		\hat{\vb{p}}
		\vdot
		\hat{\vb{A}}
		+
		\frac{q^2}{2m}
		\hat{\vb{A}}^2
	\end{split}
\end{equation}
where we used that the particle's momentum and the Maxwell field operator commute in the Coulomb gauge, see Ref.~\cite[p.~687]{Mandel1995}.
The interaction term quadratic in the Maxwell field becomes relevant for very high intensities, see Ref.~\cite[p.~198]{Cohen1989} and Ref.~\cite[p.~689]{Mandel1995} for a detailed discussion, which we are not relevant for out use case.
We conclude the interaction Hamiltonian to be
\begin{equation}
	\hat{H}_\text{int}(t,\vb{x})
	=
	-\frac{q}{m}
	\hat{\vb{p}}(t)
	\vdot
	\hat{\vb{A}}(t,\vb{x})
	.
\end{equation}
In the dipole approximation, we consider the Maxwell field constant over the support of the particle wave function~\cite[p.~688]{Mandel1995} and thus
\begin{equation}
	\hat{\vb{A}}(t,\vb{x})
	\ket{\Psi}
	\approx
	\hat{\vb{A}}(t,\vb{x}_0)
	\ket{\Psi}
\end{equation}
with $\vb{x}_0$ being the particle's \gls{com}.

For an alternative approach on how the minimal coupling leads to the dipole interaction, see Ref.~\cite[p.~635]{Cohen1992}.

\subsection{Photodetection}

\subsubsection{Gerry and Knight}

Based on Ref.~\cite[p.~120]{Gerry2005}.

We have the dipole interaction
\begin{equation}
	\hat{H}_\text{int}(t,\vb{x})
	=
	-
	\hat{\vb{d}}
	\vdot
	\hat{\vb{E}}(\vb{x},t)
	=
	-
	\hat{\vb{d}}
	\vdot
	\left(
		\hat{\vb{E}}^{(+)}(\vb{x},t)
		+
		\hat{\vb{E}}^{(-)}(\vb{x},t)
	\right)
\end{equation}
wherein
\begin{equation}
	\hat{\vb{E}}^{(+)}(\vb{x},t)
	=
	i\sum_\lambda
	\int\frac{\dd[3]{p}}{(2\pi)^3\sqrt{2\omega}}
	\vu{e}_\lambda(\vb{p})
	\hat{a}_\lambda(\vb{p})
	e^{+i\vb{p}\vdot\vb{x}}
	\approx
	i\sum_\lambda
	\int\frac{\dd[3]{p}}{(2\pi)^3\sqrt{2\omega}}
	\vu{e}_\lambda(\vb{p})
	\hat{a}_\lambda(\vb{p})
\end{equation}
where the dipole approximation $\norm{\vb{p}\vdot\vb{x}}\ll1$ has been implemented using $e^{i\vb{p}\vdot\vb{x}}\approx1$.

The matrix element for the photoemission is
\begin{equation}
	\bra{e,f}\hat{H}_\text{int}\ket{g,i}
	=
	-
	\bra{e}\hat{\vb{d}}\ket{g}
	\bra{f}\hat{\vb{E}}^{(+)}(\vb{x},t)\ket{i}
	.
\end{equation}
We do not care about the final states of the field, hence we marginalize the probability of an absorption
\begin{equation}
	\sum_f\abs{\bra{f}\hat{\vb{E}}^{(+)}(\vb{x},t)\ket{i}}^2
	=
	\sum_f
	\bra{i}
	\hat{\vb{E}}^{(-)}
	\ketbra{f}
	\hat{\vb{E}}^{(+)}
	\ket{i}
	=
	\expval{\hat{\vb{E}}^{(-)}\vdot\hat{\vb{E}}^{(+)}}{i}
	.
\end{equation}
For a more general initial radiation state $\hat\rho_i=\sum_jp_j\ketbra{j}$, we can rewrite the previous equation as
\begin{equation}
	\tr\left\{
		\hat\rho_f
		\hat{\vb{E}}^{(-)}
		\vdot
		\hat{\vb{E}}^{(+)}
	\right\}
	.
\end{equation}

\subsubsection{Cohen-Tannoudji}

Based on Ref.~\cite[p.~128]{Cohen1992}.

\begin{equation}
	\hat{H}
	=
	\hat{H}_a
	+
	\hat{H}_f
	+
	\hat{H}_i
\end{equation}
wherein \textcolor{red}{need to show this!}
\begin{equation}
	\hat{H}_i
	=
	-
	\hat{\vb{d}}
	\vdot
	\hat{\vb{E}}
\end{equation}
Assuming the dipole moment and the electric field to be parallel, we find in the interaction picture
\begin{align}
	\hat{\vb{d}}(t)
	&=
	e^{+i\hat{H}_at}
	\hat{\vb{d}}(0)
	e^{-i\hat{H}_at}
	\\
	\hat{\vb{E}}(t)
	&=
	e^{+i\hat{H}_ft}
	\hat{\vb{E}}(0)
	e^{-i\hat{H}_ft}
	.
\end{align}
We have a photoelectron emission in time interval $\Delta t$ whenever an electron is excited from the ground state $\ket{g}$, i.e.,
\begin{equation}
	p_{\Delta t}
	=
	\sum_{f,e}
	\abs{\bra{f,e}\hat{U}(\Delta t)\ket{i,g}}^2
\end{equation}
where we marginalized the final states as we do not care about them, and the interaction time-evolution operator is
\begin{equation}
	\hat{U}(\Delta t)
	=
	\mathcal{T}_+
	\exp\left\{
		-i\int_0^{\Delta t}\dd{t^\prime}
		\hat{H}_\text{int}(t^\prime)
	\right\}
	.
\end{equation}
Expanding the time-evolution operator, we find
\begin{equation}
	p_{\Delta t}
	=
	\sum_{f\neq i,e\neq g}
	\int_0^{\Delta t}\dd{t^\prime}
	\int_0^{\Delta t}\dd{t^{\prime\prime}}
	\abs{\bra{f,e}\hat{H}_\text{int}(t^\prime)\hat{H}_\text{int}(t^{\prime\prime})\ket{i,g}}^2
\end{equation}
where the first term (zeroth order in $\hat{H}_\text{int}$ vanishes because of the orthogonality of the final and initial states, and the second term vanishes because the dipole moment operator is asymmetric~\cite[p.~131]{Cohen1992}.
Writing out the dipole and electric field operators, we find
\begin{equation}
	\begin{split}
		p_{\Delta t}
		&=
		\sum_{f\neq i,e\neq g}
		\int_0^{\Delta t}\dd{t^\prime}
		\int_0^{\Delta t}\dd{t^{\prime\prime}}
		\bra{i}
		\hat{\vb{E}}(t^\prime)
		\ketbra{f}
		\hat{\vb{E}}(t^{\prime\prime})
		\ket{i}
		\bra{g}
		\hat{\vb{d}}(t^\prime)
		\ketbra{e}
		\hat{\vb{d}}(t^{\prime\prime})
		\ket{g}
		\\
		&=
		\int_0^{\Delta t}\dd{t^\prime}
		\int_0^{\Delta t}\dd{t^{\prime\prime}}
		\bra{i}
		\hat{\vb{E}}(t^\prime)
		\left(
			\sum_f
			\ketbra{f}
		\right)
		\hat{\vb{E}}(t^{\prime\prime})
		\ket{i}
		\bra{g}
		\hat{\vb{d}}(t^\prime)
		\left(
			\sum_e
			\ketbra{e}
		\right)
		\hat{\vb{d}}(t^{\prime\prime})
		\ket{g}
		\\
		&=
		\int_0^{\Delta t}\dd{t^\prime}
		\int_0^{\Delta t}\dd{t^{\prime\prime}}
		\bra{i}
		\hat{\vb{E}}(t^\prime)
		\hat{\vb{E}}(t^{\prime\prime})
		\ket{i}
		\bra{g}
		\hat{\vb{d}}(t^\prime)
		\hat{\vb{d}}(t^{\prime\prime})
		\ket{g}
		\\
		&=
		\int_0^{\Delta t}\dd{t^\prime}
		\int_0^{\Delta t}\dd{t^{\prime\prime}}
		G_i(t^\prime,t^{\prime\prime})^*
		G_g(t^\prime,t^{\prime\prime})
	\end{split}
\end{equation}
where $G_i$ and $G_g$ are the two-time correlation functions of the atomic detector system and radiation field.

\subsubsection{Mandel and Wolf}

In the model presented in Ref.~\cite[p.~685]{Mandel1995}, the interaction Hamiltonian is~\cite[p.~689]{Mandel1995}
\begin{equation}
	\hat{H}_\text{int}(t)
	=
	-
	\hat{\vb{p}}(t)
	\vdot
	\hat{\vb{A}}(\vb{x}_0,t)
\end{equation}
wherein $\vb{x}_0$ is the detector atom's \gls{com}.
The interaction term can be transformed into the dipole moment operator $\hat{\vb{d}}$ and the dielectric displacement field operator $\hat{\vb{D}}$~\cite[p.~689]{Mandel1995} giving a similar interaction term as discussed by Cohen-Tannoudji.

We then consider a bound electron in a potential well. When bound, the electron state $\ket{\Psi_0}$ satisfies
\begin{equation}
	\hat{H}_a
	\ket{g}
	=
	E_g
	\ket{g}
	=
	-\omega_g
	\ket{g}
	.
\end{equation}
Let $\hat\rho_f$ be the radiation field state in the Schrödinger picture
\begin{equation}
	\hat\rho^{(S)}(t_0)
	=
	\ketbra{g}
	\otimes
	\hat\rho_f(t_0)
\end{equation}
then in the interaction picture, the state is~\cite[p.~685]{Mandel1995}
\begin{equation}
	\hat\rho^{(I)}(t)
	=
	e^{+i\hat{H}_0(t-t_0)}
	\hat\rho^{(S)}(t)
	e^{-i\hat{H}_0(t-t_0)}
\end{equation}
and the electron's momentum operator takes the form
\begin{equation}
	\hat{\vb{p}}(t)
	=
	e^{+i\hat{H}_a(t-t_0)}
	\hat{\vb{p}}
	e^{-i\hat{H}_a(t-t_0)}
\end{equation}
and the Maxwell field is
\begin{equation}
	\begin{split}
		\hat{\vb{A}}(\vb{x}_0,t)
		&=
		\hat{\vb{A}}^{(+)}(\vb{x}_0,t)
		+
		\hat{\vb{A}}^{(-)}(\vb{x}_0,t)
		\\
		&=
		\sum_{\lambda=1,2}
		\int_{\mathbb{R}^3}
		\frac{\dd[3]{p}}{(2\pi)^3\sqrt{2\omega(\vb{p})}}
		\hat{a}_\lambda(\vb{p})
		\boldsymbol{\varepsilon}_\lambda(\vb{p})
		e^{+i\vb{p}\vdot\vb{x}_0-i\omega(\vb{p})(t-t_0)}
		+
		\text{h.c.}
		.
	\end{split}
\end{equation}
In the interaction picture, the quantum state evolves according to
\begin{equation}
	\dv{\hat\rho^{(I)}}{t}
	=
	i\comm{\hat\rho^{(I)}}{\hat{H}^{(I)}_\text{int}}
\end{equation}
which can be solved by Magnus expansion.
For instance,
\begin{equation}
	\hat\rho^{(I)}(t)
	=
	\hat\rho(t_0)
	+
	i\int_{t_0}^t\dd{t^\prime}
	\comm{\hat\rho(t_0)}{\hat{H}_\text{int}(t^\prime)}
	+
	i^2
	\int_{t_0}^t\dd{t^\prime}
	\int_{t_0}^{t^\prime}\dd{t^{\prime\prime}}
	\dots
\end{equation}

The probability amplitude for the transition
\begin{equation}
	\ket{g,i}
	\to
	\ket{e,f}
\end{equation}
is equal to~\cite[p.~686]{Mandel1995}
\begin{equation}
	\begin{split}
		p(t_0,\Delta t)
		=
		\tr\left\{
			\hat\rho_{e,f}
			\hat\rho_{g,i}(t_0+\Delta t)
		\right\}
		&=
		\tr\left\{
			\hat\rho_{e,f}
			\hat\rho_{g,i}(t_0)
		\right\}
		\\
		&+
		\frac{1}{i}
		\tr\left\{
			\hat\rho_{e,f}
			\int_{t_0}^{t_0+\Delta t}
			\dd{t^\prime}
			\comm{\hat{H}_\text{int}(t^\prime)}{\hat\rho_{g,i}(t_0)}
		\right\}
		\\
		&+
		\frac{1}{i^2}
		\tr\left\{
			\hat\rho_{e,f}
			\int_{t_0}^{t_0+\Delta t}\dd{t^\prime}
			\int_{t_0}^{t^\prime}\dd{t^{\prime\prime}}
			\comm{\hat{H}_\text{int}(t^\prime)}{\comm{\hat{H}_\text{int}(t^{\prime\prime})}{\hat\rho_{g,i}(t_0)}}
		\right\}
	\end{split}
\end{equation}
the first two terms vanish and we have
\begin{equation}
	p(t_0,\Delta t)
	=
	\int_{t_0}^{t_0+\Delta t}\dd{t^\prime}
	\int_{t_0}^{t^\prime}\dd{t^{\prime\prime}}
	\expval{\hat{H}_\text{int}(t^\prime)\hat\rho(t_0)\hat{H}_\text{int}(t^{\prime\prime})}{e,f}
	+
	\text{c.c.}
\end{equation}
We now take the interaction Hamiltonian
\begin{equation}
	\hat{H}_\text{int}(t)
	=
	e^{+i\hat{H}_a(t-t_0)}
	\hat{\vb{p}}
	e^{-i\hat{H}_a(t-t_0)}
	\hat{\vb{A}}(\vb{x}_0,t)
\end{equation}
which we evaluate with \textcolor{red}{check this!}
\begin{equation}
	\begin{split}
		\expval{\hat{H}_\text{int}(t^\prime)\hat\rho(t_0)\hat{H}_\text{int}(t^{\prime\prime})}{e,f}
		&=
		\bra{e}\hat{p}_i\ket{g}
		\bra{g}\hat{p}_j\ket{e}
		e^{i(E-E_0)(t^\prime-t^{\prime\prime})}
		\\
		&\times
		\bra{f}
		\hat{A}_i(\vb{x}_0,t^\prime)
		\braket{i}
		\hat{A}_j(\vb{x}_0,t^{\prime\prime})
		\ket{f}
		+
		\text{c.c.}
	\end{split}
\end{equation}
Expanding the initial state in the coherent state basis
\begin{equation}
	\ket{i}
	=
	\int\dd[2]{\alpha}
	p_i(\alpha,t_0)
	\ketbra{\alpha}
\end{equation}
we can use the eigenvalue equation of the coherent state and sum over all final states to remove the final state dependency, i.e.,
\begin{equation}
	\sum_i
	p(t_0,\Delta t)
	=
	\int_{t_0}^{t_0+\Delta t}\dd{t^\prime}
	\int_{t_0}^{t^\prime}\dd{t^{\prime\prime}}
	\bra{e}\hat{p}_i\ket{g}
	\bra{g}\hat{p}_j\ket{e}
	e^{i(E-E_0)(t^\prime-t^{\prime\prime})}
	\expval{\hat{A}_i(\vb{x}_0,t^\prime)\hat{A}_j(\vb{x}_0,t^{\prime\prime})}
	+
	\text{c.c.}
\end{equation}
We then sum of all final electron states weighted by the density of states times the the probability of being collected by the detector and find
\begin{equation}
	P(t_0,\Delta t)
	=
	\int_{t_0}^{t_0+\Delta t}\dd{t^\prime}
	\int_{t_0}^{t^\prime}\dd{t^{\prime\prime}}
	k_{ij}(t^\prime-t^{\prime\prime})
	\expval{\hat{A}_i(\vb{x}_0,t^\prime)\hat{A}_j(\vb{x}_0,t^{\prime\prime})}
	+
	\text{c.c.}
\end{equation}
see Ref.~\cite[p.~694]{Mandel1995} for an explicit representation of the response function $k_{ij}$.
Employing normal-ordering of the Maxwell field operators, we find that the second term, the vacuum contribution, becomes zero Ref.~\cite[p.~694]{Mandel1995} and we can write
\begin{equation}
	P(t_0,\Delta t)
	=
	\int_{t_0}^{t_0+\Delta t}\dd{t^\prime}
	\int_{t_0}^{t^\prime}\dd{t^{\prime\prime}}
	k_{ij}(t^\prime-t^{\prime\prime})
	\expval{\colon\hat{A}_i(\vb{x}_0,t^\prime)\hat{A}_j(\vb{x}_0,t^{\prime\prime})\colon}
	+
	\text{c.c.}
\end{equation}
\textcolor{red}{Can we rewrite this in terms of electric field operators?}

\subsubsection{Vogel}

Based on Ref.~\cite[p.~48]{Vogel2006}

The minimal-coupling Hamiltonian
\begin{equation}
	\hat{H}
	=
	\int\dd[3]{x}
	\int_0^\omega\dd{\omega}
	\hat{\vb{f}}^\dagger(\vb{x},\omega)
	\hat{\vb{f}}(\vb{x},\omega)
	+
	\sum_j\frac{\left(\hat{\vb{p}}_j-q_j\hat{\vb{A}}(\hat{\vb{x}}_j,t)\right)^2}{2m_j}
	+
	\hat{W}_\text{Coul}
\end{equation}
wherein $\hat{W}_\text{Coul}$ is the Coulomb interaction between the different charges.
For bound atomic states, we can expand the Maxwell field around the atom's \gls{com} and the interaction Hamiltonian takes the form
\begin{equation}
	\hat{H}_\text{int}
	=
	-
	\sum_j\frac{q_j}{m_j}
	\hat{\vb{p}}_j
	\vdot
	\hat{\vb{A}}(\vb{x}_j)
	+
	\sum_j\frac{q_j^2}{2m_j}
	\hat{\vb{A}}(\vb{x}_j)^2
\end{equation}
In the electric-dipole approximation we have
\begin{equation}
	\hat{H}_\text{int}
	\approx
	-
	\sum_j\frac{q_j}{m_j}
	\hat{\vb{p}}_j
	\vdot
	\hat{\vb{A}}(\vb{x}_j)	
\end{equation}
\textcolor{red}{fancy reasoning why the former interaction Hamiltonian is equivalent to} (maybe \cite[p.~691]{Mandel1995} or Cohen-Tanodji?)
\begin{equation}
	\hat{H}_\text{int}(t)
	\approx
	-
	\sum_j
	\hat{\vb{d}}_j
	\vdot
	\hat{\vb{E}}(\vb{x}_j,t)
	.
\end{equation}

Based on Ref.~\cite[p.~173]{Vogel2006}

The photoemission probability is equal to
\begin{equation}
	\begin{split}
		\abs{\bra{e,f}\hat{U}(t_0,t_0+\Delta t)\ket{g,i}}^2
		&=
		\bra{e,f}
		\hat{U}(t_0,t_0+\Delta t)
		\ket{g,i}
		\\
		&\times
		\bra{g,i}
		\hat{U}^\dagger(t_0,t_0+\Delta t)
		\ket{e,f}
		\\
		&=
		\tr\biggl\{
			\bra{e,f}
			\hat{U}(t_0,t_0+\Delta t)
			\ket{g,i}
			\\
			&\times
			\bra{g,i}
			\hat{U}^\dagger(t_0,t_0+\Delta t)
			\ket{e,f}
		\biggr\}
		\\
		&=
		\tr\biggl\{
			\ketbra{e,f}
			\hat{U}(t_0,t_0+\Delta t)
			\ketbra{g,i}
			\hat{U}^\dagger(t_0,t_0+\Delta t)
		\biggr\}
		\\
		&=
		\tr\left\{
			\hat\rho_{e,f}
			\hat\rho_{g,i}(t_0+\Delta t)
		\right\}
	\end{split}
\end{equation}
wherein the time-evolution operator is
\begin{equation}
	\hat{U}(t_0,t_0+\Delta t)
	=
	\mathcal{T}_+
	\exp\left\{
		-i
		\int_{t_0}^{t_0+\Delta t}\dd{t^\prime}
		\hat{H}_\text{int}(t^\prime)
	\right\}
\end{equation}
then, evaluating the transition amplitude
\begin{equation}
	\bra{g,f}
	\hat{U}(t_0,t_0+\Delta t)
	\ket{e,i}
	=
	\mathcal{T}_+
	\sum_{n=0}^\infty
	\frac{1}{n!}
	\bra{g,f}
	\left[
		i
		\int_{t_0}^{t_0+\Delta t}\dd{t^\prime}
		\vb{d}_{fg}
		\vdot
		\hat{\vb{E}}^{(+)}(\vb{x}_0,t^\prime)
		e^{i\omega_{fg}(t^\prime-t)}
	\right]^n
	\ket{e,i}
\end{equation}
\textcolor{red}{why do we only have $E^+$ here? How to derive this exactly?}
\section{Maxwell field}

The Maxwell Lagrangian with an external source $J^\mu(t,\vb{x})$ is~\cite[p.~339]{Srednicki2007}
\begin{equation}
	\begin{split}
		\mathcal{L}
		&=
		-
		\frac{1}{4}
		F_{\mu\nu}
		F^{\mu\nu}
		+
		A_\mu J^\mu
		\\
		&=
		-
		\frac{1}{2}
		(\partial_\mu A_\nu)
		(\partial^\mu A^\nu)
		+
		\frac{1}{2}
		(\partial_\mu A_\nu)
		(\partial^\nu A^\mu)
		+
		A_\mu J^\mu
	\end{split}
	\label{eq:mw_lagrangian}
\end{equation}
with the relativistic Euler-Lagrange equation yielding
\begin{equation}
	0
	=
	\partial_\mu\pdv{\mathcal{L}}{(\partial_\mu A_\nu)}
	-
	\pdv{\mathcal{L}}{A_\nu}
	=
	\partial_\mu\partial^\mu A^\nu
	-
	\partial^\nu\partial_\mu A^\mu
	-
	J^\nu	
	\label{eq:mw_eom}	
\end{equation}

\subsubsection{Coulomb gauge}

The Maxwell Lagrangian is invariant under local gauge transformations
\begin{equation}
	A_\mu(t,\vb{x})
	\to
	A_\mu^\prime(t,\vb{x})
	=
	A_\mu(t,\vb{x})
	+
	\partial_\mu\Lambda(t,\vb{x})
	\label{eq:mw_local_gauge_transform}
\end{equation}
where $\Lambda(t,\vb{x})$ is a local gauge field.
For instance, the physical field-strength tensor transforms under \cref{eq:mw_local_gauge_transform} as
\begin{equation}
	\begin{split}
		F_{\mu\nu}
		\to
		F_{\mu\nu}^\prime
		&=
		\partial_\mu\left(A_\nu+\partial_\nu\Lambda\right)
		-
		\partial_\nu\left(A_\mu+\partial_\mu\Lambda\right)
		\\
		&=
		F_{\mu\nu}
		+
		\partial_\mu\partial_\nu\Lambda
		-
		\partial_\nu\partial_\mu\Lambda
		=
		F_{\mu\nu}
	\end{split}
	\label{eq:mw_field_strength_gauge_transform}.
\end{equation}
The invariance of the Maxwell field under local gauge transformation is used to remove a degree of freedom from the field using a gauge condition.
The most popular gauge conditions are the Lorentz gauge $\partial_\mu A^\mu=0$ and the Coulomb gauge $\div\vb{A}=0$.
While the Lorentz gauge is manifest Lorentz invariant it suffers from unphysical scalar and longitudinal polarization states destroying unitarity.
The Coulomb gauge is manifest unitarity but has to be imposed in every reference frame.
As Lorentz boosts are not of interest for us, we will adapt the Coulomb gauge condition.
The Coulomb gauge leaves a residual gauge freedom which allows us to choose $A_0=0$.\footnote{If external static sources are present, we need to be more careful with the residual gauge fixing.}

Applying the Coulomb gauge
\begin{align}
	\div\vb{A}(t,\vb{x})
	=
	0
	&&
	A_0
	=
	0
	\label{eq:mw_coulomb_gauge}
\end{align}
to the free equation of motion \cref{eq:mw_eom} yields the relativistic wave equation
\begin{equation}
	0
	=
	\partial_\mu\partial^\mu
	\vb{A}(t,\vb{x})
	=
	\partial_t^2
	\vb{A}(t,\vb{x})
	-
	\grad^2
	\vb{A}(t,\vb{x})
	\label{eq:mw_relatistic_wave}
\end{equation}
which is solved by plane-waves satisfying the massless dispersion relation
\begin{equation}
	\omega(\vb{p})
	=
	\norm{\vb{p}}
\end{equation}

\subsubsection{Mode decomposition}

As with the Klein-Gordon field, we start with the four-dimensional Fourier transform of $A^\mu(t,\vb{x})$, insert it into the free equations of motion ($J^\mu=0$), and perform the mode decomposition
\begin{equation}
	\vb{A}(t,\vb{x})
	=
	\sum_{\lambda=1,2}
	\int_{\mathbb{R}^3}\frac{\dd[3]{p}}{(2\pi)^3\sqrt{2\omega(\vb{p})}}
	\left\{
		a_\lambda(\vb{p})
		\vb{\epsilon}_\lambda(\vb{p})
		e^{-ip_\mu x^\mu}
		+
		\text{c.c.}
	\right\}
	\label{eq:mw_ft}
\end{equation}
where we defined $a_\lambda(\vb{p})\vb{\epsilon}_\lambda(\vb{p})=\vb{A}(\omega(\vb{p}),\vb{p})$ and $\vb{\epsilon}_\lambda(\vb{p})$ denotes the polarization vectors.
In the Coulomb gauge, the polarization vectors satisfy~\cite[p.~341]{Srednicki2007}
\begin{align}
	\vb{p}
	\vdot
	\boldsymbol{\varepsilon}_\lambda(\vb{p})
	&=
	0
	\\
	\boldsymbol{\varepsilon}_\alpha(\vb{p})
	\vdot
	\boldsymbol{\varepsilon}_\beta(\vb{p})^*
	&=
	\delta_{\alpha\beta}
	\\
	\sum_{\lambda=1,2}
	\boldsymbol{\varepsilon}_\alpha^i(\vb{p})
	\boldsymbol{\varepsilon}_\alpha^j(\vb{p})^*
	&=
	\delta^{ij}
	-
	\frac{p^ip^j}{\vb{p}^2}
\end{align}

\subsection{Maxwell equations}

The covariant inhomogeneous Maxwell equations are obtained from the equations of motion
\begin{equation}
	J^\nu
	=
	\partial_\mu F^{\mu\nu}
	=
	\partial_\mu\partial^\mu A^\nu
	-
	\partial^\nu\partial_\mu A^\mu
	\label{eq:mw_inhomo},
\end{equation}
and the covariant homogeneous Maxwell equations are a consequence of the Bianchi identity
\begin{equation}
	0
	=
	\partial_\mu\tilde{F}^{\mu\nu}
	\label{eq:mw_homo}
\end{equation}
where we defined $\tilde{F}^{\mu\nu}=\frac{1}{2}\varepsilon^{\mu\nu\alpha\beta}F_{\alpha\beta}$.

The components of the antisymmetric field-strength tensor $F^{\mu\nu}=-F^{\nu\mu}$ relate to the electromagnetic field components via~\cite[p.~336]{Srednicki2007}
\begin{align}
	F^{0i}
	=
	E^i
	&&
	F^{ij}
	=
	\varepsilon^{ijk}B_k
	\label{eq:mw_field_strength_components}.
\end{align}
Evaluating the time component of \cref{eq:mw_homo} yields the Gauss' law for magnetism
\begin{equation}
	\begin{split}
		0
		=
		\varepsilon_{0\lambda\mu\nu}\partial^\lambda F^{\mu\nu}
		&=
		\varepsilon_{0ijk}\partial^iF^{jk}
		\\
		&=
		-
		\varepsilon_{ijk}\varepsilon_{ljk}
		\partial^i B_l
		=
		2\partial_iB^i
	\end{split}
	\label{eq:mw_gauss_law_mag}
\end{equation}
and the spatial component yields Ampere's circuit law
\begin{equation}
	\begin{split}
		0
		=
		\varepsilon_{i\lambda\mu\nu}
		\partial^\lambda
		F^{\mu\nu}
		&=
		-
		\varepsilon_{ijk}
		\varepsilon^{ljk}
		\partial_t B_l
		-
		2\varepsilon_{ijk}
		\partial^jE^k
		\\
		&=
		\partial_tB_i
		+
		\varepsilon_{ijk}
		\partial^jE_k
	\end{split}
	\label{eq:mw_ampere_law}.
\end{equation}
The time component of the inhomogeneous covariant Maxwell equation \cref{eq:mw_inhomo} yields Gauss' law
\begin{equation}
	J^0
	=
	\rho
	=
	\partial_\mu F^{\mu\nu}
	=
	\partial_i E^i
	\label{eq:mw_gauss_law},
\end{equation}
and the spatial component yields Faraday's law of induction
\begin{equation}
	J^i
	=
	\partial_\mu F^{\mu i}
	=
	-\partial_t E^i
	+\varepsilon^{ijk}\partial_j B_k
	\label{eq:mw_faraday_law}.
\end{equation}
and we derived the vector Maxwell equations from first principles
\begin{align}
	\div\vb{E}
	=
	\rho
	&&
	\div\vb{B}
	=
	0
	\label{eq:mw_homo_vec}
	\\
	\curl\vb{E}
	=
	-
	\partial_t\vb{B}
	&&
	\curl\vb{B}
	=
	\vb{J}
	+
	\partial_t\vb{E}
	\label{eq:mw_inhomo_vec}
\end{align}

\subsubsection{Field observables}

From \cref{eq:mw_field_strength_components} in the Coulomb gauge, we find
\begin{align}
	\vb{E}
	=
	-\partial_t\vb{A}
	&&
	\vb{B}
	=
	\curl\vb{A}
\end{align}
and inserting the Fourier mode decomposition \cref{eq:mw_ft}
\begin{align}
	\vb{E}(t,\vb{x})
	&=
	\sum_{\lambda=1,2}
	\int\frac{\dd[3]{p}}{(2\pi)^3\sqrt{2\omega(\vb{p})}}
	\left(
		i\omega(\vb{p})
		\boldsymbol{\varepsilon}_\lambda(\vb{p})
	\right)
	\left\{
		a_\lambda(\vb{p})
		e^{-ip_\mu x^\mu}
		-
		\text{c.c.}
	\right\}
	\\
	\vb{B}(t,\vb{x})
	&=
	\sum_{\lambda=1,2}
	\int\frac{\dd[3]{p}}{(2\pi)^3\sqrt{2\omega(\vb{p})}}
	\left(
		i\vb{p}\cross\omega(\vb{p})
		\boldsymbol{\varepsilon}_\lambda(\vb{p})
	\right)
	\left\{
		a_\lambda(\vb{p})
		e^{-ip_\mu x^\mu}
		-
		\text{c.c.}
	\right\}
\end{align}
see Greiner~\cite[p.~198]{Greiner2013} who uses a symmetric definition for the Fourier transform.

The canonical momentum density turns out to be the electric field
\begin{equation}
	\pi(t,\vb{x})
	=
	\partial_t\pdv{\mathcal{L}}{(\partial_t\vb{A})}
	=
	\vb{E}(t,\vb{x})
	\label{eq:mw_mom}.
\end{equation}
The energy density is
\begin{equation}
	\mathcal{H}
	=
	T^{00}
	=
	\pdv{\mathcal{L}}{(\partial_t\vb{A})}
	\partial_t\vb{A}
	-
	\mathcal{L}
\end{equation}
while the total energy becomes
\begin{equation}
	H
	=
	\int\dd[3]{x}
\end{equation}

\subsection{Canonical quantization}

According to Greiner~\cite[p.~196]{Greiner2013}
\begin{align}
	\vb{A}(t,\vb{x})
	&=
	\sum_{\lambda=1,2}
	\int\frac{\dd[3]{p}}{(2\pi)^3\sqrt{2\omega(\vb{p})}}
	\boldsymbol{\varepsilon}_\lambda(\vb{p})
	\left\{
		a_\lambda(\vb{p})
		e^{-ip_\mu x^\mu}
		+
		\text{c.c.}
	\right\}
	\label{eq:qmw_pos}
	\\
	\vb{E}(t,\vb{x})
	&=
	\sum_{\lambda=1,2}
	\int\frac{\dd[3]{p}}{(2\pi)^3\sqrt{2\omega(\vb{p})}}
	\left(
		i\omega(\vb{p})
		\boldsymbol{\varepsilon}_\lambda(\vb{p})
	\right)
	\left\{
		a_\lambda(\vb{p})
		e^{-ip_\mu x^\mu}
		-
		\text{c.c.}
	\right\}
	\label{eq:qmw_mom}
	\\
	\vb{B}(t,\vb{x})
	&=
	\sum_{\lambda=1,2}
	\int\frac{\dd[3]{p}}{(2\pi)^3\sqrt{2\omega(\vb{p})}}
	\left(
		i\vb{p}\cross\omega(\vb{p})
		\boldsymbol{\varepsilon}_\lambda(\vb{p})
	\right)
	\left\{
		a_\lambda(\vb{p})
		e^{-ip_\mu x^\mu}
		-
		\text{c.c.}
	\right\}
	\label{eq:qmw_mag}
\end{align}
and the commutation relations are
\begin{align}
	\comm{\hat{A}_i(t,\vb{x})}{\hat{E}_j(t,\vb{y})}
	=
	-i\delta^{(3)}_{\perp ij}(\vb{x}-\vb{y})
	&&
	\comm{\hat{A}_i(t,\vb{x})}{\hat{A}_j(t,\vb{y})}
	=
	0
	=
	\comm{\hat{E}_i(t,\vb{x})}{\hat{E}_j(t,\vb{y})}
\end{align}
with the transverse delta function
\begin{equation}
	\delta^{(3)}_{\perp ij}(\vb{x})
	=
	\int\frac{\dd[3]{p}}{(2\pi)^3}
	\left(\delta_{ij}-\frac{p_ip_j}{\vb{p}^2}\right)
	e^{i\vb{p}\vdot\vb{x}}
	\label{eq:transverse_delta}
\end{equation}
which is required for the quantization in the Coulomb gauge.

The total energy of the field is
\begin{equation}
	\hat{H}
	=
	\sum_{\lambda=1,2}
	\int\frac{\dd[3]{p}}{(2\pi)^3}
	\omega(\vb{p})
	\hat{a}_\lambda^\dagger(\vb{p})
	\hat{a}_\lambda(\vb{p})
	\label{eq:qmw_total_energy}
\end{equation}
and the total momentum is
\begin{equation}
	\hat{\vb{P}}
	=
	\sum_{\lambda=1,2}
	\int\frac{\dd[3]{p}}{(2\pi)^3}
	\vb{p}
	\hat{a}_\lambda^\dagger(\vb{p})
	\hat{a}_\lambda(\vb{p})
	\label{eq:qmw_total_momentum}
\end{equation}

\subsection{Single-particle}

The (smeared) single-particle state is expected to generalize to
\begin{equation}
	\ket{\vb{f}}
	=
	\hat{\vb{A}}[\vb{f}]
	=
	\sum_{\lambda=1,2}
	\int\frac{\dd[3]{p}}{(2\pi)^3\sqrt{2\omega(\vb{p})}}
	f_\lambda(\vb{p})
	\boldsymbol{\varepsilon}_\lambda(\vb{p})
	\hat{a}_\lambda^\dagger(\vb{p})
	\ket{0}
\end{equation}
and we expect the results for the Klein-Gordon field to hold as we only get additional polarization degrees of freedom.

\subsection{Coherent state and interactions}

Zhang~\cite{Zhang1999} gives a beautiful discussion of coherent states for the Maxwell field in the Lorentz gauge.
In the Coulomb gauge, we find
\begin{equation}
	\hat{S}
	=
	\mathcal{T}
	\exp\left\{
		-i\int\dd[4]{x}
		\vb{A}(t,\vb{x})\vdot\vb{J}(t,\vb{x})
	\right\}
\end{equation}
where $\mathcal{T}$ denotes the time-ordering symbol.

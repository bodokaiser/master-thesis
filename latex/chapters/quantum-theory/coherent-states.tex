\section{Coherent states}

\subsection{Radiation from classical current}

The interaction of a classical current $\vb{j}(t,\vb{x})$ with the Maxwell field operator in the Coulomb gauge $\vu{A}(t,\vb{x})$ is given by the interaction Hamiltonian
\begin{equation}
	\hat{H}_\text{int}(t)
	=
	-
	\int\dd[3]{x}
	\vb{j}(t,\vb{x})
	\vdot
	\hat{\vb{A}}(t,\vb{x})
	.
\end{equation}
Inserting the spatial Fourier transform of the current $\vb{j}(t,\vb{p})$ and the quantum mode mode expansion, \cref{eq:maxwell_positive_operator,eq:maxwell_negative_operator}, the interaction Hamiltonian becomes
\begin{equation}
	\hat{H}_\text{int}(t)
	=
	-
	\sum_{\lambda=1,2}
	\int\frac{\dd[3]{p}}{(2\pi)^3\sqrt{2\omega(\vb{p})}}
	\left\{
		\left(
			\vb{j}(t,-\vb{p})
			\vdot
			\vu{e}_\lambda(\vb{p})
		\right)
		\hat{a}_\lambda(\vb{p})
		e^{-i\omega(\vb{p})t}
		+
		\text{h.c.}
	\right\}
	.
\end{equation}

The effect of an interaction acting on a quantum state from time $t_0$ to $t$ is encoded in the time-evolution operator\footnote{See Ref.~\cite[p.~215]{Greiner2013} for an introduction into the time-evolution operator and interactions.}
\begin{equation}
	\hat{U}(t,t_0)
	=
	\mathcal{T}_+
	\exp\left\{
		-i
		\int_{t_0}^t\dd{t^\prime}
		\hat{H}_\text{int}(t^\prime)
	\right\}
	\label{eq:time_evolution_operator}
\end{equation}
wherein $\mathcal{T}_+$ denotes the time-ordering symbol.
The Magnus expansion presents a systematic approach in finding an explicit form of the time-evolution operator\footnote{See Ref.~\cite[p.~42]{QuesadaMejia2015}, for an introduction to the Magnus expansion with application to nonlinear processes.}, it is given by
\begin{equation}
	\hat{U}(t,t_0)
	=
	\exp\left\{
		\sum_{n=1}\Omega^{(n)(t_0,t)}
	\right\}
\end{equation}
wherein the first two terms are given by
\begin{align}
	\hat{\Omega}^{(1)}(t,t_0)
	&=
	-i
	\int_{t_0}^t\dd{t^\prime}
	\hat{H}_\text{int}(t^\prime)
	\\
	\hat{\Omega}^{(2)}(t,t_0)
	&=
	\frac{(-i)^2}{2!}
	\int_{t_0}^t\dd{t^\prime}
	\int_{t_0}^{t^\prime}\dd{t^{\prime\prime}}
	\comm{\hat{H}_\text{int}(t^\prime)}{\hat{H}_\text{int}(t^{\prime\prime})}
	.
\end{align}
For some interactions there exists no exact solution and we can truncate the expansion up to some finite term.
Compared to other expansions, e.g. the Neumann expansion, the truncated Magnus expansion is still unitary.

The first term in the Magnus expansion turns out to be
\begin{equation}
	\hat{\Omega}^{(1)}(t,t_0)
	=
	i
	\sum_{\lambda=1,2}
	\int_{\mathbb{R}^3}\frac{\dd[3]{p}}{(2\pi)^3\sqrt{2\omega(\vb{p})}}
	\left\{
		J_\lambda(t,t_0;\vb{p})
		\hat{a}_\lambda(\vb{p})
		+
		\text{h.c.}
	\right\}
\end{equation}
where we defined
\begin{equation}
	J_\lambda(t,t_0;\vb{p})
	=
	\int_{t_0}^t\dd{t^\prime}
	j_\lambda(t^\prime,\vb{p})
	e^{-i\omega(\vb{p})t^\prime}
	.
\end{equation}
For the second term in the Magnus expansion, we first evaluate the commutator
\begin{equation}
	\comm{\hat{H}(t^\prime)}{\hat{H}(t^{\prime\prime})}
	=
	i\sum_{\lambda=1,2}
	\int_{\mathbb{R}^3}\frac{\dd[3]{p}}{(2\pi)^3\omega(\vb{p})}
	\Im\left\{
		j_\lambda(t^\prime,\vb{p})
		j_\lambda(t^{\prime\prime},\vb{p})^*
		e^{-i\omega(\vb{p})(t^\prime-t^{\prime\prime})}
	\right\}
\end{equation}
and notice that it is complex valued, hence, all higher commutators vanish and the Magnus expansion with the first two terms is exact.
In summary, the second term of the Magnus expansion turns out to be
\begin{equation}
	\hat{\Omega}^{(2)}(t,t_0)
	=
	i\sum_{\lambda=1,2}
	\int_{\mathbb{R}^3}\frac{\dd[3]{p}}{(2\pi)^3\omega(\vb{p})}
	\Im\left\{
		J_\lambda(t_0,t^\prime;\vb{p})
		J_\lambda(t_0,t^{\prime\prime};\vb{p})^*
	\right\}
\end{equation}
The second term contributes a phase to the time-evolution operator.
As long as we consider a single current source, no interference of phases can occur and we can ignore the phase factor.
The exact time-evolution operator of the Maxwell field interacting with a classical source current therefore is
\begin{equation}
	\hat{U}(t,t_0)
	=
	\exp\left\{
		i\sum_{\lambda=1,2}
		\int_{\mathbb{R}^3}\frac{\dd[3]{p}}{(2\pi)^3\sqrt{2\omega(\vb{p})}}
		\left\{
			J_\lambda(t,t_0;\vb{p})
			\hat{a}_\lambda(\vb{p})
			+
			\text{h.c.}
		\right\}
	\right\}
\end{equation}
which equals the displacement operator for a time-dependent spectrum $\hat{D}[\alpha(t,t_0)]$.

\subsection{Displacement operator}
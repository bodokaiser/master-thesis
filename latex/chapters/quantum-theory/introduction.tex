Quantum optical communication suffers from a complete theoretical framework describing the quantum and signal aspects of light.

The closest domain to quantum optical communication having a sufficiently developed theoretical framework is quantum optics.
Quantum optics typically models~\cite{Fox2006,Gerry2005,Haroche2006,Meystre2007} light as an extension to the non-relativistic one-dimensional quantum harmonic oscillator.
The quantum harmonic oscillator is a standard model in undergraduate physics and sufficient to describe the key features of quantum optics experiments.
That said, quantum optics suffers from the severe limitation of having no notion of a (frequency) spectrum - essential to communication.
We are only aware of few resources~\cite{Barnett2002,Shapiro2009,Loudon2000} attempting to incorporate a spectrum into quantum optics.
Unfortunately, these few resources are sparse on the justification of their results, and it would be desirable to have a more established theory.

The standard model, the most complete theory at our disposal, is built upon relativistic field theory.
Relativistic field theories naturally include a momentum spectrum compatible with relativistic particles like photons.
As the mediator of the electromagnetic force, the Maxwell field reduces to classical electrodynamics and allows for the physical insights missing from quantum optics.
However, the quantized version of relativistic field theory, quantum field theory, is usually applied to scattering experiments and misses the diverse set of quantum states we know from quantum optics.

In the following chapter, we propose a complete theoretical framework suited for quantum optical communication.
We use the well established framework of quantum field theory to motivate and interpret the relevant field operators, which naturally include a three-dimensional momentum spectrum.
We then axiomatically construct generalized quantum states from these field operators, derive and analyze their properties, and compare them to their quantum optical equivalent.
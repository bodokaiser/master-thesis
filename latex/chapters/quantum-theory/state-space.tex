\section{State space construction}

% explain that we neglect polarization in the following (Klein-Gordon approximation)
% 
We attempt a similar construction as in Ref.~\cite[p.~506]{Cohen2019} was done for the quantum harmonic oscillator.

\subsection{Vacuum state}

We assume the existence of a unique (up to a constant phase factor) vacuum state $\ket{0}$ which is invariant under a unitary Poincaré transformation~\cite[p.~97]{Streater2016}
\begin{equation}
	\hat{U}(a,\Lambda)
	\ket{0}
	=
	\ket{0}
	\label{eq:vacuum_invariance}
\end{equation}
where $\Lambda$ denotes a Lorentz transformation and $a$ a spacetime translation.
The vacuum state is an element of the one-dimensional complex Hilbert space $\mathcal{H}^{(0)}=\mathcal{H}(\mathbb{C})$, the zero-particle state space.
The generator of the unitary spacetime translation is the four-momentum~\cite[p.~28]{Haag2012}
\begin{equation}
	\hat{U}(a)
	=
	\hat{U}(a,\mathbb{1})
	=
	e^{iP_\mu a^\mu}
	\label{eq:spacetime_translation}
	.
\end{equation}
The invariance of the vacuum, \cref{eq:vacuum_invariance}, implies for spacetime translations, \cref{eq:spacetime_translation}, that the vacuum is an eigenstate of the Hamilton and momentum operator to eigenvalue zero, i.e.,
\begin{align}
	\hat{H}
	\ket{0}
	&=
	0
	\label{eq:vacuum_energy}
	\\
	\vu{P}
	\ket{0}
	&=
	\vb{0}
	\label{eq:vacuum_momentum}
	.	
\end{align}
Inserting the Hamilton operator, \cref{eq:maxwell_hamilton_operator}, into \cref{eq:vacuum_energy}, implies the vacuum state being a zero eigenstate of the annihilation operator
\begin{equation}
	\hat{a}(\vb{p})
	\ket{0}
	=
	0
	\label{eq:vacuum_annihilation}
	.
\end{equation}
The annihilation operator "destroying" the vacuum state is an important for the vacuum state to be the lowest energy state and have no negative energies or particle numbers.

\subsection{Particle states}

The commutators of the annihilation and creation operator with the Hamilton operator, \cref{eq:maxwell_hamilton_operator}, are
\begin{align}
	\comm{\hat{H}}{\hat{a}^\dagger(\vb{p})}
	&=
	+
	\omega(\vb{p})
	\hat{a}^\dagger(\vb{p})
	\\
	\comm{\hat{H}}{\hat{a}(\vb{p})}
	&=
	-
	\omega(\vb{p})
	\hat{a}(\vb{p})
	.
\end{align}
Applying the commutators to the left of the vacuum state and using the energy eigenvalue, \cref{eq:vacuum_energy}, we find the eigenvalue equation
\begin{equation}
	\hat{H}
	\hat{a}^\dagger(\vb{p})
	\ket{0}
	=
	\omega(\vb{p})
	\hat{a}^\dagger(\vb{p})
	\ket{0}
	\label{eq:creation_energy_eigenstate}
	,
\end{equation}
i.e., $\hat{a}^\dagger(\vb{p})\ket{0}$ is an eigenstate to the Hamilton operator $\hat{H}$ with eigenvalue $\omega(\vb{p})$ which suggests
\begin{equation}
	\ket{\vb{p}}
	\propto
	\hat{a}^\dagger(\vb{p})
	\ket{0}
	\label{eq:momentum_state}
\end{equation}
to be a momentum state with momentum $\vb{p}$ and energy $\omega(\vb{p})$.
A particular problem of the momentum state is that it is not normalizable which is not surprising considering the fact the momentum state corresponds to an ideal plane-wave mode.
From a mathematical perspective, the annihilation and creation operators are not functions but distributions acting on some square-integrable smearing function $f$, i.e.,
\begin{equation}
	\ket{1_f}
	=
	\int\frac{\dd[3]{p}}{(2\pi)^3}
	f(\vb{p})
	\ket{\vb{p}}
	\label{eq:single_particle_state}
\end{equation}
is a mathematical well-defined physical single-particle state if the spectrum is normalized
\begin{equation}
	\braket{1_f}{1_f}
	=
	\int\frac{\dd[3]{p}}{(2\pi)^3}
	\abs*{f(\vb{p})}^2
	=
	1
	.
\end{equation}
The mathematical smearing function encodes the physical spectral properties of the wave packet.
See, for instance, Ref.~\cite{Naumov2013} and Ref.~\cite{Naumov2009}, for a discussion of the wave packet properties like center-of-mass position, group velocity and dispersion.

\subsection{Symmetric Fock space}

The single-particle state defined in \cref{eq:single_particle_state} is an element of the one-particle Hilbert space of square-integrable functions defined on three-dimensional space $\mathcal{H}^{(1)}=\mathcal{H}\left(L^2(\mathbb{R}^3)\right)$.
The generalization of the one-particle Hilbert space $\mathcal{H}^{(1)}$ to an $n$-particle Hilbert space $\mathcal{H}^{(n)}$ is the tensor product of one-particle Hilbert spaces
\begin{equation}
	\mathcal{H}^{(n)}
	=
	\bigotimes^n_{i=1}
	\mathcal{H}^{(1)}
	.
\end{equation}
Now, it is possible to have a superposition of, e.g., the vacuum state and a particle state
\begin{equation}
	\ket{\psi}
	=
	c_1
	\ket{0}
	+
	c_2
	\ket{1_f}
\end{equation}
with $c_1,c_2\in\mathbb{C}$ which means that we need to combine orthonormal $n$-particle states.
We first construct a tensor algebra over the Hilbert space $\mathcal{H}^{(1)}$ as the direct sum~\cite[p.~290]{Bogolubov1989}
\begin{equation}
	\bigoplus^\infty_{n=0}
	S_+
	\mathcal{H}^{(n)}
\end{equation}
wherein $S_+$ symmetrizes the Hilbert space for bosons.
Equipping the tensor algebra with an inner product and using the completeness of the $n$-particle Hilbert spaces, we obtain again a Hilbert space, named the symmetric Fock space $\mathcal{F}_+$~\cite[p.~35]{Haag2012}.
\section*{Summary}

In \cite[p.~53]{Cohen2019} there is an extra section about the relationship between one- and three-dimensional wave packets.

Typically, we 
Often, the Lorentz-invariant measure is approximated\footnote{We need to keep in mind that the approximation still requires the energy-momentum relation to hold!}
\begin{equation}
	\dd{\mu}(\vb{p})
	=
	\frac{\dd[3]{p}}{(2\pi)^3\sqrt{2\omega(\vb{p})}}
	\approx
	\frac{\dd[3]{p}}{(2\pi)^3}
	.
\end{equation}
By imposing the Coulomb gauge, we already selected a particular reference frame.
Additionally, the typical detector bandwidth $B$ is much smaller than the center frequency $\omega_0$ and we can do by the mean value theorem
\begin{equation}
	\int
	\frac{\dd[3]{p}}{(2\pi)^3\sqrt{2\omega(\vb{p})}}
	\approx
	\frac{1}{\sqrt{2\omega_0}}
	\int
	\frac{\dd[3]{p}}{(2\pi)^3}
\end{equation}
However, we know that the field dynamics are not affected by multiplication with an overall constant.

Neglecting polarization, we effectively reduce the transverse Maxwell field $A$ to a massless Klein-Gordon (scalar) field which has the Lagrangian density
\begin{equation}
	\mathcal{L}
	=
	\frac{1}{2}
	\left(
		\partial_\mu A
	\right)
	\left(
		\partial^\mu A
	\right)
\end{equation}
which leads to the equations of motion
\begin{equation}
	\partial_t^2 A
	=
	\laplace A
\end{equation}
resembling a relativistic wave equation.

Another often made approximation is neglecting the transverse profile and only considering momentum in the direction of propagation.
Let us take the propagation direction to be $z$.
The advantage is that the energy equal one momentum component $\omega(\vb{p})=p_z$.

% electric vs vector potential (vector potenial more fundamental, Abramov-Bohm effect, photon number, most cases equivalent, see pA vs dE)

% one-dimensional approximation
% forward momentum

% we switch between single- and continuous-mode
\section*{Summary}

We motivated the Maxwell and electric quantum field operators, \cref{eq:maxwell_operator} and \cref{eq:electric_operator}, in the Coulomb gauge from first principles.
The Coulomb gauge restricts our results to a stationary reference frame appropriate for quantum communication where transmitter and receiver are stationary or moving slowly.

% what have we done
% - derived the Maxwell and electric field operators + Coulomb gauge
% - we constructed the number and coherent states -> generalized creation and annihilation operators
% - number state are not anymore eigenenergy and momentum states!
% - coherent states from the interaction with a classical source, behave exactly like their single-mode coutnerparts except having a spectrum


% approximations
% Lorentz-invariant measure approximately constant
% reduction from three to one dimension + forward propagation
% give field operators and stats in the one-dimensional approximation and compare with Barnett

% TODO: quadrature operator! -> can we motivate the quadrature from the electric field (phase-rotated unitless electric field operator?)?

We furthermore define the generalized quadrature operator~\cite[p.~79]{Barnett2002} to be
\begin{equation}
	\hat{X}(\vartheta)
	=
	\int\frac{\dd[3]{p}}{(2\pi)^3}
	\frac{1}{\sqrt{2}}
	\left\{
		\hat{a}(\vb{p})
		e^{-i\vartheta}
		+
		\hat{a}^\dagger(\vb{p})
		e^{+i\vartheta}
	\right\}
	\label{eq:quadrature_operator}
	.
\end{equation}

In \cite[p.~53]{Cohen2019} there is an extra section about the relationship between one- and three-dimensional wave packets.

By imposing the Coulomb gauge, we already selected a particular reference frame.
Additionally, the typical detector bandwidth $B$ is much smaller than the center frequency $\omega_0$ and we can do by the mean value theorem
\begin{equation}
	\int
	\frac{\dd[3]{p}}{(2\pi)^3\sqrt{2\omega(\vb{p})}}
	\approx
	\frac{1}{\sqrt{2\omega_0}}
	\int
	\frac{\dd[3]{p}}{(2\pi)^3}
\end{equation}
However, we know that the field dynamics are not affected by multiplication with an overall constant.
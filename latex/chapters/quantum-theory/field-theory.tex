\section{Canonical quantization of Maxwell field}

What do we do here?
- Section structure
- Expected results

Hamiltonian vs Lagrangian mechanics?
Why we start with Lagrangian mechanics?

From the Lagrangian we can infer everything else.

\subsection{Maxwell Lagrangian from first principles}

From a theoretical physics viewpoint, Maxwell theory is a relativistic vector field theory with local gauge symmetry.

As a relativistic vector field, we expect the Maxwell field $A^\mu$ to have four components, one time and three spatial components, and to transform as a Lorentz vector~\cite[p.~37]{Peskin1995}
\begin{equation}
	A^\mu(x)
	\to
	A^{\prime\mu}(x^\prime)
	=
	\Lambda^\mu_\nu
	A^\nu(\Lambda^{-1}x)
	.
\end{equation}
Contracting Lorentz vectors, or in general Lorentz tensors, to a scalar yields a Lorentz-invariant quantity.
Exclusively using Lorentz invariant quantities when constructing our theory ensures the compatibility of our theory with special relativity.

As a starting point for our theory's the Lagrangian (density), we propose
\begin{equation}
	\mathcal{L}
	=
	c_1
	\left(
		\partial_\mu
		A^\nu
	\right)
	\left(
		\partial^\mu
		A_\nu
	\right)
	+
	c_2
	\left(
		\partial_\mu
		A^\mu
	\right)
	\left(
		\partial_\nu
		A^\nu
	\right)
	+
	c_3
	\left(
		\partial_\mu
		A^\nu
	\right)
	\left(
		\partial_\nu
		A^\mu
	\right)
	+
	c_4
	m^2
	A_\mu A^\mu
	\label{eq:proposed_maxwell_lagrangian_proposal},
\end{equation}
wherein we restricted ourselves to terms which allow for dimensionless coefficients $c1,c2,c3,c4\in\mathbb{R}$, a mass term, and no (self-)interactions.\footnote{The restriction to dimensionless coefficients can be further motivated by renormalization arguments.}
For the action integral
\begin{equation}
	S
	=
	\int\dd{t}
	L
	=
	\int\dd[4]{x}
	\mathcal{L}
\end{equation}
to exist, we need $A^\mu$ to be square-integrable, i.e., vanish at the integration boundaries.
Using partial integration, we can show the redundancy of two of the first three terms in \cref{eq:proposed_maxwell_lagrangian}, see Ref.~\cite{deRham2014}.
We remove the redundancy by setting the third coefficient to zero, $c_3=0$.
Demanding invariance under local gauge transformations requires removing the mass term and determining the Lagrangian up to an overall constant.
The overall constant does not affect the dynamics, and its absolute value is equal to $1/2$ by convention.
We finally arrive at the well-known Lagrangian density of the Maxwell field,
\begin{equation}
	\mathcal{L}
	=
	-
	\frac{1}{2}
	\left(
		\partial_\mu
		A^\nu
	\right)
	\left(
		\partial^\mu
		A_\nu
	\right)
	+
	\frac{1}{2}
	\left(
		\partial_\mu
		A^\mu
	\right)
	\left(
		\partial_\nu
		A^\nu
	\right)
	.
\end{equation}
While we can experimentally verify the vector nature of the Maxwell field by polarization experiments, the requirement of the Maxwell field having local gauge symmetry appears artificial.
If we accept the Dirac Lagrangian, describing charged fermions, then Noether's theorem links local gauge symmetry to charge conservation.
However, the Dirac Lagrangian itself cannot be made gauge-invariant without coupling to a gauge-invariant vector field, which turns out to be the Maxwell field.
The complete gauge-invariant Lagrangian describing charged fermions coupled to the Maxwell field lays down the foundation of \gls{qed}.

\subsection{Lorentz force, field-strength tensor, electromagnetic components and Maxwell equations}

% How do we get from the vector field to the field-strength tensor?
We consider the action of a particle with mass $m$ and charge $q$ coupled to the vector field~\cite[p.~244]{Zee2013}
\begin{equation}
	S
	=
	-
	m\int\dd{\tau}
	\sqrt{-g_{\mu\nu}\odv{x^\mu}{\tau}\odv{x^\nu}{\tau}}
	+
	q\int\dd{\tau}
	A_\mu\left(x(\tau)\right)
	\odv{x^\mu}{\tau}
\end{equation}
for which the action principle yields the equation of motion
\begin{equation}
	m
	\odv[order={2}]{x^\mu}{\tau}
	=
	qF^\mu_\nu(x)\odv{x^\nu}{\tau}
	\label{eq:cov_lorentz_force}
\end{equation}
corresponding to a covariant formulation of Lorentz force law, wherein the field-strength tensor
\begin{equation}
	F_{\mu\nu}
	=
	\partial_\mu
	A_\nu
	-
	\partial_\nu
	A_\mu
\end{equation}
is the covariant generalization of the electromagnetic field.

Comparison of the covariant Lorentz force law, \cref{eq:cov_lorentz_force}, with the vector Lorentz force law
\begin{equation}
	m
	\odv[order={2}]{\vb{x}}{t}
	=
	q\left(
		\vb{E}
		+
		\odv{\vb{x}}{t}
		\vcross
		\vb{B}
	\right)
\end{equation}
we can identify the components of the field-strength tensor with the electromagnetic field~\cite[p.~245]{Zee2013}
\begin{align}
	E^1
	&=
	F^{01}
	=
	-
	F_{01}
	&
	E^2
	&=
	F^{02}
	=
	-
	F_{02}
	&
	E^3
	&=
	F^{03}
	=
	-
	F_{03}
	\\
	B^1
	&=
	F^{23}
	=
	-
	F_{23}
	&
	B^2
	&=
	F^{31}
	=
	-
	F_{31}
	&
	B^3
	&=
	F^{12}
	=
	-
	F_{12}
\end{align}
or in index notation~\cite[p.~336]{Srednicki2007}
\begin{align}
	F^{0i}
	&=
	E^i
	&
	F^{ij}
	&=
	\varepsilon^{ijk}
	B_k
	.
\end{align}
\textcolor{red}{identify charged moving particle with current}

\begin{equation}
	\mathcal{L}
	=
	-
	\frac{1}{4}
	F_{\mu\nu}
	F^{\mu\nu}
	-
	j_\mu A^\mu
\end{equation}

\begin{align}
	\div\vb{E}
	&=
	j_0
	&
	\div\vb{B}
	&=
	0
	\\
	\curl\vb{E}
	&=
	-\partial_t\vb{B}
	&
	\curl\vb{B}
	&=
	\vb{j}
	+
	\partial_t\vb{E}
\end{align}

\subsection{Coulomb gauge and plane-wave expansion}

The physical observable of the Maxwell field is the field-strength tensor $F_{\mu\nu}$.
The field-strength tensor $F_{\mu\nu}$ is invariant under local gauge transformations.
We perform a local gauge transformation 
We choose a gauge field $\chi$ satisfying
\begin{align}
	\laplace\chi
	&=
	\partial_i A^i
	=
	-
	\div\vb{A}
	&
	\partial_t
	\chi
	&=
	-
	A_0
\end{align}
and perform a gauge transformation.
After the gauge transformation, the Maxwell field satisfies
\begin{align}
	\div\vb{A}
	&=
	0
	&
	A_0
	&=0
\end{align}
corresponding to the Coulomb and temporal gauge condition~\cite[p.~144]{Greiner2013},
The temporal gauge condition $A_0=0$ implicitly assumes that there are no free charges.

% Relation between Coulomb and temporal gauge and transverse and longitudinal components
% Coulomb interaction energy and A_0 component

% Lorentz-invariant measure

% plane-wave expansion
% energy-momentum relation

\subsection{Canonical quantization and quantum operators}

The canonical energy-momentum tensor~\cite[p.~174]{Greiner2013}
\begin{equation}
	\begin{split}
		T^{\mu\nu}
		&=
		\pdv{\mathcal{L}}{(\partial_\mu A_\sigma)}
		\partial^\nu A_\sigma
		-
		g^{\mu\nu}\mathcal{L}
		\\
		&=
		-
		\left(
			\partial^\mu
			A^\sigma
		\right)
		\left(
			\partial^\nu
			A_\sigma
		\right)
		+
		\frac{1}{2}
		g^{\mu\nu}
		\left(
			\partial^\rho
			A^\sigma
		\right)
		\left(
			\partial_\rho
			A_\sigma
		\right)
	\end{split}
\end{equation}

The energy density is the time-time component of the energy-momentum tensor
\begin{equation}
	\begin{split}
		T^{00}
		&=
		-
		\left(
			\partial^0
			A^\sigma
		\right)
		\left(
			\partial^0
			A_\sigma
		\right)
		+
		\frac{1}{2}
		g^{00}
		\left(
			\partial^\rho
			A^\sigma
		\right)
		\left(
			\partial_\rho
			A_\sigma
		\right)
		\\
		&=
		-
		\left(
			\partial^0
			A^j
		\right)
		\left(
			\partial^0
			A_j
		\right)
		+
		\frac{1}{2}
		\left(
			\partial^0
			A^j
		\right)
		\left(
			\partial_0
			A_j
		\right)
		+
		\frac{1}{2}
		\left(
			\partial^i
			A^j
		\right)
		\left(
			\partial_i
			A_j
		\right)
		\\
		&=
		-
		\frac{1}{2}
		\left(
			\partial_t
			A^j
		\right)
		\left(
			\partial_t
			A_j
		\right)
		+
	\end{split}
\end{equation}

% normal order to remove (infinite) vacuum energy
\begin{equation}
	H
	=
\end{equation}

\begin{equation}
	P_j
	=
\end{equation}

We define the transverse delta distribution as the transverse projection of the three-dimensional delta distribution
\begin{equation}
	\delta_{\perp,ij}^{(3)}
	\left(\vb{x}\right)
	=
	P_{\perp,ij}
	\delta^{(3)}
	\left(\vb{x}\right)
	=
	\int\frac{\dd[3]{p}}{(2\pi)^3}
	\left(
		\delta_{ij}
		-
		\frac{p_ip_j}{\vb{p}^2}
	\right)
	e^{i\vb{p}\vdot\vb{x}}
\end{equation}

\begin{align}
	\comm{\hat{A}_i(t,\vb{x})}{\hat{E}_j(t,\vb{y})}
	&=
	-i
	\delta_{\perp,ij}^{(3)}
	\left(\vb{x}-\vb{y}\right)
	\\
	\comm{\hat{A}_i(t,\vb{x})}{\hat{A}_j(t,\vb{y})}
	&=
	0
	=
	\comm{\hat{E}_i(t,\vb{x})}{\hat{E}_j(t,\vb{y})}
\end{align}
where the negative sign for the non-vanishing commutator arises from the relation between the canonical momentum density and the electric field, $\hat\pi_i=-\hat{E}_i$ (?).

In the Coulomb gauge, the Maxwell field operator is
\begin{equation}
	\vu{A}(t,\vb{x})
	=
	\vu{A}^{(+)}(t,\vb{x})
	+
	\vu{A}^{(-)}(t,\vb{x})
\end{equation}
\begin{align}
	\vu{A}^{(-)}(t,\vb{x})
	&=
	\sum_{\lambda=1,2}
	\int\dd{\mu(\vb{p})}
	\hat{a}_\lambda(\vb{p})
	\vu{e}_\lambda(\vb{p})
	e^{-i\omega(\vb{p})t+i\vb{p}\vdot\vb{x}}
	\\
	\vu{A}^{(+)}(t,\vb{x})
	&=
	\sum_{\lambda=1,2}
	\int\dd{\mu(\vb{p})}
	\hat{a}_\lambda^\dagger(\vb{p})
	\vu{e}_\lambda(\vb{p})^*
	e^{+i\omega(\vb{p})t-i\vb{p}\vdot\vb{x}}
\end{align}
Disagreement between ~\cite[p.~341]{Srednicki2007}, ~\cite[p.~198]{Greiner2013}, and \cite[p.~123]{Peskin1995}

\subsection{Quantum observable operators}

\begin{equation}
	\hat{H}
	=
\end{equation}
\begin{equation}
	\hat{P}
	=
\end{equation}

The electric field operator is given by
\begin{equation}
	\vu{E}(t,\vb{x})
	=
	\vu{E}^{(+)}(t,\vb{x})
	+
	\vu{E}^{(-)}(t,\vb{x})
	.
\end{equation}
\begin{align}
	\vu{E}^{(-)}(t,\vb{x})
	&=
	-i
	\sum_{\lambda=1,2}
	\int\dd{\mu(\vb{p})}
	\omega(\vb{p})
	\hat{a}_\lambda(\vb{p})
	\vu{e}_\lambda(\vb{p})
	e^{-i\omega(\vb{p})t+i\vb{p}\vdot\vb{x}}
	\\
	\vu{E}^{(+)}(t,\vb{x})
	&=
	+i
	\sum_{\lambda=1,2}
	\int\dd{\mu(\vb{p})}
	\omega(\vb{p})
	\hat{a}_\lambda^\dagger(\vb{p})
	\vu{e}_\lambda(\vb{p})^*
	e^{+i\omega(\vb{p})t-i\vb{p}\vdot\vb{x}}
\end{align}
\section{Canonical quantization of Maxwell field}

What do we do here?
- Section structure
- Expected results

Hamiltonian vs Lagrangian mechanics?
Why we start with Lagrangian mechanics?

% TODO: Photon field does not get attention and has non-trivial quantization procedure

\subsection{Maxwell Lagrangian from first principles}

% TODO: Check Greiner p. 149 for additional motivation on how to find the Maxwell Lagrangian

From a theoretical physics viewpoint, Maxwell theory is a relativistic vector field theory with local gauge symmetry.
That alone is sufficient to "guess" the Maxwell Lagrangian from first principles.
Having established the Maxwell Lagrangian, we can fully infer the physics of electromagnetism inside the field theoretical framework.

As a relativistic vector field, we expect the Maxwell field $A^\mu$ to have four components, one time and three spatial components, and to transform as a Lorentz vector~\cite[p.~37]{Peskin1995}
\begin{equation}
	A^\mu(x)
	\to
	A^{\prime\mu}(x^\prime)
	=
	\Lambda^\mu_\nu
	A^\nu(\Lambda^{-1}x)
	.
\end{equation}
Contracting Lorentz vectors, or in general Lorentz tensors, to a scalar yields a Lorentz-invariant quantity.
Exclusively using Lorentz invariant quantities when constructing our theory ensures the compatibility of our theory with special relativity.

As a starting point for our theory's the Lagrangian (density), we propose
\begin{equation}
	\mathcal{L}
	=
	c_1
	\left(
		\partial_\mu
		A^\nu
	\right)
	\left(
		\partial^\mu
		A_\nu
	\right)
	+
	c_2
	\left(
		\partial_\mu
		A^\mu
	\right)
	\left(
		\partial_\nu
		A^\nu
	\right)
	+
	c_3
	\left(
		\partial_\mu
		A^\nu
	\right)
	\left(
		\partial_\nu
		A^\mu
	\right)
	+
	c_4
	m^2
	A_\mu A^\mu
	\label{eq:proposed_maxwell_lagrangian_proposal},
\end{equation}
wherein we restricted ourselves to terms which allow for dimensionless coefficients $c1,c2,c3,c4\in\mathbb{R}$, a mass term, and no (self-)interactions.\footnote{The restriction to dimensionless coefficients can be further motivated by renormalization arguments.}
For the action integral
\begin{equation}
	S
	=
	\int\dd{t}
	L
	=
	\int\dd[4]{x}
	\mathcal{L}
\end{equation}
to exist, we need $A^\mu$ to be square-integrable, i.e., vanish at the integration boundaries.
Using partial integration, we can show the redundancy of two of the first three terms in \cref{eq:proposed_maxwell_lagrangian}, see Ref.~\cite{deRham2014}.
We remove the redundancy by setting the third coefficient to zero, $c_3=0$.
Demanding invariance under local gauge transformations requires removing the mass term and determining the Lagrangian up to an overall constant.
The overall constant does not affect the dynamics, and its absolute value is equal to $1/2$ by convention.
We finally arrive at the well-known Lagrangian density of the Maxwell field,
\begin{equation}
	\mathcal{L}
	=
	-
	\frac{1}{2}
	\left(
		\partial_\mu
		A^\nu
	\right)
	\left(
		\partial^\mu
		A_\nu
	\right)
	+
	\frac{1}{2}
	\left(
		\partial_\mu
		A^\mu
	\right)
	\left(
		\partial_\nu
		A^\nu
	\right)
	\label{eq:maxwell_lagrangian_field}
	.
\end{equation}
While we can experimentally verify the vector nature of the Maxwell field by polarization experiments, the requirement of the Maxwell field having local gauge symmetry appears artificial.
If we accept the Dirac Lagrangian, describing charged fermions, then Noether's theorem links local gauge symmetry to charge conservation.
However, the Dirac Lagrangian itself cannot be made gauge-invariant without coupling to a gauge-invariant vector field, which turns out to be the Maxwell field.
The complete gauge-invariant Lagrangian describing charged fermions coupled to the Maxwell field lays down the foundation of \gls{qed}.

\subsection{Lorentz force law, field-strength tensor and electromagnetic fields}

We have introduced the Maxwell field as the fundamental mediator of the electromagnetic force.
However, so far, it is unclear how the electromagnetic field components relate to the Maxwell field.
The (manifest) covariant formulation makes it particularly difficult to identify the non-covariant electromagnetic vector fields.
To shed some light, we first derive the covariant Lorentz force law and then compare it to the non-covariant vector formulation to identify the electromagnetic field components.

To derive the covariant Lorentz force law, let us consider the action of a point particle with mass $m$ and charge $q$ coupled to the Maxwell field $A^\mu$~\cite[p.~244]{Zee2013}
\begin{equation}
	S
	=
	-
	m\int\dd{\tau}
	\sqrt{-g_{\mu\nu}\odv{x^\mu}{\tau}\odv{x^\nu}{\tau}}
	+
	q\int\dd{\tau}
	A_\mu\left(x(\tau)\right)
	\odv{x^\mu}{\tau}
	\label{eq:lorentz_force_action}
\end{equation}
wherein $g_{\mu\nu}$ is the Minkowski metric.
The first term is the generalization of a line integral, parametrized by tau, to Minkowski spacetime.
The second term describes the coordinate-dependent coupling of the Maxwell field with the charge of the particle.
Invoking the variational calculus on the action, we recover the covariant formulation of Lorentz force law
\begin{equation}
	m
	\odv[order={2}]{x^\mu}{\tau}
	=
	qF^\mu_\nu(x)\odv{x^\nu}{\tau}
	\label{eq:covariant_lorentz_force}
\end{equation}
where we introduced the asymmetric field-strength tensor
\begin{equation}
	F_{\mu\nu}
	=
	\partial_\mu
	A_\nu
	-
	\partial_\nu
	A_\mu
	\label{eq:field_strength_tensor}
	.
\end{equation}
The field-strength tensor covariantly encodes the electromagnetic field components.

Using the field-strength tensor $F_{\mu\nu}$ and including the interaction of the Maxwell field with an external classical current $j^\mu$, we can rewrite the Maxwell Lagrangian of \cref{eq:maxwell_lagrangian_field} as
\begin{equation}
	\mathcal{L}
	=
	-
	\frac{1}{4}
	F_{\mu\nu}
	F^{\mu\nu}
	+
	A_\mu j^\mu
	\label{eq:maxwell_lagrangian_field_strength}
	.
\end{equation}
We note that the action of the interaction term in \cref{eq:maxwell_lagrangian_field_strength} reduces to the second action term in \cref{eq:lorentz_force_action} when using the current of a point particle with charge $q$~\cite[p.~177]{Peskin1995}
\begin{equation}
	j^\mu(x)
	=
	q
	\int\dd{\tau}
	\odv{y^\mu(\tau)}{\tau}
	\delta^{(4)}\left(x-y(\tau)\right)
	.
\end{equation}

Comparison of the covariant Lorentz force law, \cref{eq:covariant_lorentz_force}, with the non-covariant version
\begin{equation}
	m
	\odv[order={2}]{\vb{x}}{t}
	=
	q\left(
		\vb{E}
		+
		\odv{\vb{x}}{t}
		\vcross
		\vb{B}
	\right)
	\label{eq:lorentz_force}
	,
\end{equation}
we can relate the components of the field-strength tensor and the electromagnetic field~\cite[p.~245]{Zee2013}
\begin{align}
	E^1
	&=
	F^{01}
	=
	-
	F_{01}
	&
	E^2
	&=
	F^{02}
	=
	-
	F_{02}
	&
	E^3
	&=
	F^{03}
	=
	-
	F_{03}
	\\
	B^1
	&=
	F^{23}
	=
	-
	F_{23}
	&
	B^2
	&=
	F^{31}
	=
	-
	F_{31}
	&
	B^3
	&=
	F^{12}
	=
	-
	F_{12}
	,
\end{align}
which we can compactly summarize.~\cite[p.~336]{Srednicki2007}
\begin{align}
	F^{0i}
	&=
	E^i
	&
	F^{ij}
	&=
	\varepsilon^{ijk}
	B_k
	\label{eq:field_strength_components}
	.
\end{align}
We have successfully related the rather abstract Maxwell field with the observable electromagnetic field components using the covariant field-strength tensor.

In a final step, we would like to express the Maxwell Lagrangian in terms of the electromagnetic fields to complete the bridge to classical electrodynamics.
Using \cref{eq:field_strength_components}, we can derive the identity~\cite[p.~142]{Greiner2013}
\begin{equation}
	F_{\mu\nu}
	F^{\mu\nu}
	=
	-2
	\left(
		\vb{E}^2
		-
		\vb{B}^2
	\right)
	\label{eq:field_strength_scalar}
	.
\end{equation}
Inserting \cref{eq:field_strength_scalar} into \cref{eq:maxwell_lagrangian_field_strength} we find the Maxwell Lagrangian as reported in many books on classical electrodynamics
\begin{equation}
	\mathcal{L}
	=
	\frac{1}{2}
	\left(
		\vb{E}^2
		-
		\vb{B}^2
	\right)
	+
	A_0j^0
	-
	\vb{A}
	\vdot
	\vb{j}
\end{equation}
where we expanded the Minkowski inner product $A_\mu j^\mu$ in terms of the time and spatial components.
We identify $j^0$ and $\vb{j}$ as the charge and $\vb{j}$ as the current density as well as $A_0$ as the scalar and $\vb{A}$ as the vector potential of electromagnetism.

\subsection{Covariant and non-covariant Maxwell equations}

The manifest Lorentz-covariant Maxwell equations are~\cite[p.~336]{Srednicki2007}
\begin{align}
	\partial_\mu
	\tilde{F}^{\mu\nu}
	&=
	0
	\label{eq:covariant_homogeneous_maxwell_equations}
	\\
	\partial_\mu
	F^{\mu\nu}
	&=
	j^\mu
	\label{eq:covariant_inhomogeneous_maxwell_equations}
\end{align}
where we defined the dual field-strength tensor~\cite[p.~142]{Greiner2013}
\begin{equation}
	\tilde{F}^{\mu\nu}
	=
	\frac{1}{2}
	\varepsilon^{\mu\nu\rho\sigma}
	F_{\rho\sigma}
	\label{eq:dual_field_strength}
\end{equation}
with $\varepsilon^{\mu\nu\rho\sigma}$ being the completely antisymmetric unit tensor.
\Cref{eq:covariant_homogeneous_maxwell_equations} summarizes the homogeneous Maxwell equations and can be derived from the Bianchi identity.
\Cref{eq:covariant_inhomogeneous_maxwell_equations} summarizes the inhomogeneous Maxwell equations and follows from the variational calculus of the Maxwell Lagrangian, i.e., represents the Maxwell field's \gls{eom}s.
Evaluating the non-zero components of the Lorentz tensor equations and inserting the relation of the field-strength to the electromagnetic field, \cref{eq:field_strength_components}, we arrive at the microscopic Maxwell equations
\begin{align}
	\div\vb{E}
	&=
	j_0
	&
	\div\vb{B}
	&=
	0
	\\
	\curl\vb{E}
	&=
	-\partial_t\vb{B}
	&
	\curl\vb{B}
	&=
	\vb{j}
	+
	\partial_t\vb{E}
	.
\end{align}
We derived Maxwell equations by guessing the Maxwell Lagrangian from fundamental principles and symmetries, whereas historically, Maxwell equations summarized decades of experiments studying the electromagnetic field.

\subsection{Coulomb gauge and plane-wave expansion}

As a four-dimensional vector field, we would expect the Maxwell field $A^\mu$ to have four \gls{dof}s, one temporal $A^0$, one longitudinal $A_\parallel$, and two transverse $\vb{A}_\perp$.
At the same time, freely propagating electromagnetic waves have only two \gls{dof}s, the polarization.\footnote{Alternatively, we can take the particle picture and say that the photon has two helicity $\pm1$.}
The zero mass of the photon requires the photon to travel at light speed, restricting the photon's temporal and longitudinal \gls{dof}.
Local gauge symmetry reflects the non-physicality, more precisely, the mathematical redundancy of two of the four \gls{dof}s.
To remove the unphysical \gls{dof}s, we impose a gauge condition, i.e., we choose a specific gauge field $\chi$ and perform a local gauge transformation
\begin{equation}
	A_\mu
	\to
	A_\mu^\prime
	=
	A_\mu
	+
	\partial_\mu\chi
\end{equation}
to remove some components of the Maxwell field.
Different gauge conditions exist, see Ref.~\cite[p.~144]{Greiner2013} and Ref.~\cite[p.~339]{Srednicki2007}, and some gauges may be more convenient than others depending on the problem under consideration.
For instance, the Lorenz gauge
\begin{equation}
	\partial_\mu
	A^\mu
	=
	0	
\end{equation}
has the advantage of being manifestly Lorentz covariant at the cost of having the different \gls{dof}s intertwined to be independent of a particular reference frame.
The Coulomb gauge
\begin{equation}
	\partial_i A^i
	=
	\div\vb{A}
	=
	0
\end{equation}
corresponds to selecting a stationary reference frame in which the electromagnetic radiation is purely transverse.
Imposing the Coulomb gauge only removes one \gls{dof} from the Maxwell field.
We use the remaining residual gauge freedom to impose the temporal gauge
\begin{equation}
	A_0
	=
	0
	.
\end{equation}
The temporal gauge is valid when there are no static charge distributions.
Static charge distributions add a Coulomb interaction that does not involve the exchange of physical photons and is not subject to quantization.
However, static charges add constant energy to the system and a longitudinal component to the electric field.
In most cases, it is sufficient to discuss the effects of the Coulomb interaction separately from the radiation, justifying the temporal gauge.
See Ref.~\cite[p.~145,187,200]{Greiner2013} for more detail on the Coulomb interaction and the Maxwell field's longitudinal \gls{dof}.

% Lorentz-invariant measure

% plane-wave expansion
% energy-momentum relation

\subsection{Canonical quantization in the Coulomb gauge}

The canonical energy-momentum tensor~\cite[p.~174]{Greiner2013}
\begin{equation}
	\begin{split}
		T^{\mu\nu}
		&=
		\pdv{\mathcal{L}}{(\partial_\mu A_\sigma)}
		\partial^\nu A_\sigma
		-
		g^{\mu\nu}\mathcal{L}
		\\
		&=
		-
		\left(
			\partial^\mu
			A^\sigma
		\right)
		\left(
			\partial^\nu
			A_\sigma
		\right)
		+
		\frac{1}{2}
		g^{\mu\nu}
		\left(
			\partial^\rho
			A^\sigma
		\right)
		\left(
			\partial_\rho
			A_\sigma
		\right)
	\end{split}
\end{equation}

The energy density is the time-time component of the energy-momentum tensor
\begin{equation}
	\begin{split}
		T^{00}
		&=
		-
		\left(
			\partial^0
			A^\sigma
		\right)
		\left(
			\partial^0
			A_\sigma
		\right)
		+
		\frac{1}{2}
		g^{00}
		\left(
			\partial^\rho
			A^\sigma
		\right)
		\left(
			\partial_\rho
			A_\sigma
		\right)
		\\
		&=
		-
		\left(
			\partial^0
			A^j
		\right)
		\left(
			\partial^0
			A_j
		\right)
		+
		\frac{1}{2}
		\left(
			\partial^0
			A^j
		\right)
		\left(
			\partial_0
			A_j
		\right)
		+
		\frac{1}{2}
		\left(
			\partial^i
			A^j
		\right)
		\left(
			\partial_i
			A_j
		\right)
		\\
		&=
		-
		\frac{1}{2}
		\left(
			\partial_t
			A^j
		\right)
		\left(
			\partial_t
			A_j
		\right)
		+
	\end{split}
\end{equation}

% normal order to remove (infinite) vacuum energy
\begin{equation}
	H
	=
\end{equation}

\begin{equation}
	P_j
	=
\end{equation}

We define the transverse delta distribution as the transverse projection of the three-dimensional delta distribution
\begin{equation}
	\delta_{\perp,ij}^{(3)}
	\left(\vb{x}\right)
	=
	P_{\perp,ij}
	\delta^{(3)}
	\left(\vb{x}\right)
	=
	\int\frac{\dd[3]{p}}{(2\pi)^3}
	\left(
		\delta_{ij}
		-
		\frac{p_ip_j}{\vb{p}^2}
	\right)
	e^{i\vb{p}\vdot\vb{x}}
\end{equation}

\begin{align}
	\comm{\hat{A}_i(t,\vb{x})}{\hat{E}_j(t,\vb{y})}
	&=
	-i
	\delta_{\perp,ij}^{(3)}
	\left(\vb{x}-\vb{y}\right)
	\\
	\comm{\hat{A}_i(t,\vb{x})}{\hat{A}_j(t,\vb{y})}
	&=
	0
	=
	\comm{\hat{E}_i(t,\vb{x})}{\hat{E}_j(t,\vb{y})}
\end{align}
where the negative sign for the non-vanishing commutator arises from the relation between the canonical momentum density and the electric field, $\hat\pi_i=-\hat{E}_i$ (?).

In the Coulomb gauge, the Maxwell field operator is
\begin{equation}
	\vu{A}(t,\vb{x})
	=
	\vu{A}^{(+)}(t,\vb{x})
	+
	\vu{A}^{(-)}(t,\vb{x})
\end{equation}
\begin{align}
	\vu{A}^{(-)}(t,\vb{x})
	&=
	\sum_{\lambda=1,2}
	\int\dd{\mu(\vb{p})}
	\hat{a}_\lambda(\vb{p})
	\vu{e}_\lambda(\vb{p})
	e^{-i\omega(\vb{p})t+i\vb{p}\vdot\vb{x}}
	\\
	\vu{A}^{(+)}(t,\vb{x})
	&=
	\sum_{\lambda=1,2}
	\int\dd{\mu(\vb{p})}
	\hat{a}_\lambda^\dagger(\vb{p})
	\vu{e}_\lambda(\vb{p})^*
	e^{+i\omega(\vb{p})t-i\vb{p}\vdot\vb{x}}
\end{align}
Disagreement between ~\cite[p.~341]{Srednicki2007}, ~\cite[p.~198]{Greiner2013}, and \cite[p.~123]{Peskin1995}

\subsection{Quantum observable operators}

\begin{equation}
	\hat{H}
	=
\end{equation}
\begin{equation}
	\hat{P}
	=
\end{equation}

The electric field operator is given by
\begin{equation}
	\vu{E}(t,\vb{x})
	=
	\vu{E}^{(+)}(t,\vb{x})
	+
	\vu{E}^{(-)}(t,\vb{x})
	.
\end{equation}
\begin{align}
	\vu{E}^{(-)}(t,\vb{x})
	&=
	-i
	\sum_{\lambda=1,2}
	\int\dd{\mu(\vb{p})}
	\omega(\vb{p})
	\hat{a}_\lambda(\vb{p})
	\vu{e}_\lambda(\vb{p})
	e^{-i\omega(\vb{p})t+i\vb{p}\vdot\vb{x}}
	\\
	\vu{E}^{(+)}(t,\vb{x})
	&=
	+i
	\sum_{\lambda=1,2}
	\int\dd{\mu(\vb{p})}
	\omega(\vb{p})
	\hat{a}_\lambda^\dagger(\vb{p})
	\vu{e}_\lambda(\vb{p})^*
	e^{+i\omega(\vb{p})t-i\vb{p}\vdot\vb{x}}
\end{align}
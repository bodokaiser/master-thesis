\section{Canonical quantization of Maxwell field}

What do we do here?
- Section structure
- Expected results

Hamiltonian vs Lagrangian mechanics?
Why we start with Lagrangian mechanics?

From the Lagrangian we can infer everything else.

\subsection{Maxwell Lagrangian from first principles}

Maxwell theory is the relativistic field theory of spin-1 particles (photons) with gauge symmetry.

Photons are the excitations of a Lorentz invariant vector field $A^\mu$.
A vector field $A^\mu(x)$ is Lorentz invariant, if it changes under a Lorentz transformation $\Lambda$ according to~\cite[p.~37]{Peskin1995}
\begin{equation}
	A^\mu(x)
	\to
	A^{\prime\mu}(x^\prime)
	=
	\Lambda^\mu_\nu
	A^\nu(\Lambda^{-1}x)
	.
\end{equation}
Lorentz transformations extends Galilean transformations with Lorentz boosts, i.e., transformations in moving reference systems, but also include rotations.

For the action to be a Lorentz scalar, we need to construct the Lagrangian density from contracted Lorentz tensors.
The most general Lagrangian density without (self-)interactions is~\cite{deRham2014}
\textcolor{red}{this is not the most general density!}
\begin{equation}
	\mathcal{L}
	=
	c_1
	\left(
		\partial_\mu
		A^\nu
	\right)
	\left(
		\partial^\mu
		A_\nu
	\right)
	+
	c_2
	\left(
		\partial_\mu
		A^\mu
	\right)
	\left(
		\partial_\nu
		A^\nu
	\right)
	+
	c_3
	\left(
		\partial_\mu
		A^\nu
	\right)
	\left(
		\partial_\nu
		A^\mu
	\right)
	+
	c_4
	A_\mu A^\mu
	\label{eq:vector_field_lagrangian1}
	.
\end{equation}
The Maxwell field $A^\mu$ has to be square-integrable for the action
\begin{equation}
	S
	=
	\int\odif{t}
	L
	=
	\int\odif[order={4}]{x}
	\mathcal{L}
\end{equation}
to exist, implying that $A^\mu$ vanishes at the integration boundaries.
Using partial integration with vanishing boundary term of the third term in \cref{eq:vector_field_lagrangian1} twice, we find
\begin{equation}
	\mathcal{L}
	=
	c_1
	\left(
		\partial_\mu
		A^\nu
	\right)
	\left(
		\partial^\mu
		A_\nu
	\right)
	+
	(c_2+c_3)
	\left(
		\partial_\mu
		A^\mu
	\right)
	\left(
		\partial_\nu
		A^\nu
	\right)
	+
	c_4
	A_\mu A^\mu
	\label{eq:vector_field_lagrangian2}
	.
\end{equation}
Demanding the Lagrangian density in \Cref{eq:vector_field_lagrangian2} to be invariant under the gauge transformation
\begin{equation}
	A^\mu
	\to
	A^{\prime\mu}
	=
	A^\mu
	-
	\partial^\mu
	\chi
\end{equation}
with gauge field $\chi$, constraints the coefficients to
\begin{align*}
	c_1
	&=
	-
	(c_2+c_3)
	&
	c_4
	&=
	0
\end{align*}
and further simplifying the Lagrangian density to
\begin{equation}
	\mathcal{L}
	=
	c_1
	\left(
		\partial_\mu
		A^\nu
	\right)
	\left(
		\partial^\mu
		A_\nu
	\right)
	-
	c_1
	\left(
		\partial_\mu
		A^\mu
	\right)
	\left(
		\partial_\nu
		A^\nu
	\right)
	\label{eq:vector_field_lagrangian3}
	.
\end{equation}
The remaining overall constant $c_1$ does not affect the field dynamics and can be arbitrarily chosen.
Adapting the standard convention $c_1=-1/2$, we successfully derived the Maxwell Lagrangian
\begin{equation}
	\mathcal{L}
	=
	-
	\frac{1}{2}
	\left(
		\partial_\mu
		A^\nu
	\right)
	\left(
		\partial^\mu
		A_\nu
	\right)
	+
	\frac{1}{2}
	\left(
		\partial_\mu
		A^\mu
	\right)
	\left(
		\partial_\nu
		A^\nu
	\right)
	.
\end{equation}

We showed that imposing gauge symmetry on the Lagrangian of a relativistic vector field theory leads to the Maxwell Lagrangian.
While the spin of a field can be determined by polarization experiments, it appears somewhat ad-hoc to impose gauge theory.
If we start from the Dirac Lagrangian describing charged spin-1/2 particles, local gauge invariance of the Dirac field is required for charge conservation by Noether's theorem.
However, the Dirac Lagrangian itself cannot be made gauge invariant but requires an additional coupling term with a gauge invariant vector field.
In that sense we can say that charge conservation of spin-1/2 particles implies gauge invariance of the vector field.

\subsection{Lorentz force, electromagnetic field components and Maxwell equations}

% How do we get from the vector field to the field-strength tensor?
We consider the action of a particle with mass $m$ and charge $q$ coupled to the vector field~\cite[p.~244]{Zee2013}
\begin{equation}
	S
	=
	-
	m\int\odif{\tau}
	\sqrt{-g_{\mu\nu}\odv{x^\mu}{\tau}\odv{x^\nu}{\tau}}
	+
	q\int\odif{\tau}
	A_\mu\left(x(\tau)\right)
	\odv{x^\mu}{\tau}
\end{equation}
for which the action principle yields the equation of motion
\begin{equation}
	m
	\odv[order={2}]{x^\mu}{\tau}
	=
	qF^\mu_\nu(x)\odv{x^\nu}{\tau}
	\label{eq:cov_lorentz_force}
\end{equation}
corresponding to a covariant formulation of Lorentz force law, wherein the field-strength tensor
\begin{equation}
	F_{\mu\nu}
	=
	\partial_\mu
	A_\nu
	-
	\partial_\nu
	A_\mu
\end{equation}
is the covariant generalization of the electromagnetic field.

Comparison of the covariant Lorentz force law, \cref{eq:cov_lorentz_force}, with the vector Lorentz force law
\begin{equation}
	m
	\odv[order={2}]{\vb{x}}{t}
	=
	q\left(
		\vb{E}
		+
		\odv{\vb{x}}{t}
		\vcross
		\vb{B}
	\right)
\end{equation}
we can identify the components of the field-strength tensor with the electromagnetic field~\cite[p.~245]{Zee2013}
\begin{align}
	E^1
	&=
	F^{01}
	=
	-
	F_{01}
	&
	E^2
	&=
	F^{02}
	=
	-
	F_{02}
	&
	E^3
	&=
	F^{03}
	=
	-
	F_{03}
	\\
	B^1
	&=
	F^{23}
	=
	-
	F_{23}
	&
	B^2
	&=
	F^{31}
	=
	-
	F_{31}
	&
	B^3
	&=
	F^{12}
	=
	-
	F_{12}
\end{align}
or in index notation~\cite[p.~336]{Srednicki2007}
\begin{align}
	F^{0i}
	&=
	E^i
	&
	F^{ij}
	&=
	\varepsilon^{ijk}
	B_k
	.
\end{align}
\textcolor{red}{identify charged moving particle with current}

\begin{equation}
	\mathcal{L}
	=
	-
	\frac{1}{4}
	F_{\mu\nu}
	F^{\mu\nu}
	-
	j_\mu A^\mu
\end{equation}

\begin{align}
	\div\vb{E}
	&=
	j_0
	&
	\div\vb{B}
	&=
	0
	\\
	\curl\vb{E}
	&=
	-\partial_t\vb{B}
	&
	\curl\vb{B}
	&=
	\vb{j}
	+
	\partial_t\vb{E}
\end{align}

\subsection{Coulomb gauge and plane-wave expansion}

% implications of Coulomb gauge
% relation to Klein-Gordon field
% one-dimensional approximation
% Lorentz-invariant measure

\subsection{Energy-momentum tensor and Hamiltonian}

\subsection{Canonical quantization}

% Canonical commutation relation
% Quantization of gauge fields, Coulomb gauge requires transverse delta distribution
% Hamiltonian operator
% electromagnetic field operator
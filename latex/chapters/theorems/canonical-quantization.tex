\section{Canonical quantization (Klein-Gordon)}

\begin{definition}[Canonical quantization]
	In the canonical quantization procedure, the dynamical variables are promoted to operators satisfying the equal-time canonical commutation relations
	\begin{align}
		\comm{\hat\phi(t,\vb{x})}{\hat\pi(t,\vb{y})}
		&=
		i\delta^{(3)}(\vb{x}-\vb{y})
		\\
		\comm{\hat\phi(t,\vb{x})}{\hat\phi(t,\vb{y})}
		&=
		\comm{\hat\pi(t,\vb{x})}{\hat\pi(t,\vb{y})}
		=
		0
		\label{eq:qkg_comm_pm}
		.
	\end{align}
\end{definition}
\begin{corollary}[Klein-Gordon field operators]
	The Klein-Gordon field operators are
	\begin{align}
		\hat\phi(x^\mu)
		&=
		\int\frac{\dd[3]{p}}{(2\pi)^3}
		\frac{1}{\sqrt{2\omega(\vb{p})}}
		\left\{
			\hat{a}(\vb{p})
			e^{-ip_\mu x^\mu}
			+
			\hat{a}^\dagger(\vb{p})
			e^{+ip_\mu x^\mu}
		\right\}_{p_0=\omega(\vb{p})}
		\label{eq:qkg_pos}
		\\
		\hat\pi(x^\mu)
		&=
		\int\frac{\dd[3]{p}}{(2\pi)^3}
		\left(-i\sqrt{\frac{\omega(\vb{p})}{2}}\right)
		\left\{
			\hat{a}(\vb{p})
			e^{-ip_\mu x^\mu}
			-
			\hat{a}^\dagger(\vb{p})
			e^{+ip_\mu x^\mu}
		\right\}_{p_0=\omega(\vb{p})}
		\label{eq:qkg_mom}
	\end{align}
	where $\hat\phi(x^\mu)$ and $\hat\pi(x^\mu)$ satisfy the equal-time canonical commutation relations.
\end{corollary}

\begin{theorem}\label{thm:qkg_comm_ac}%{qkgcommac}
	The operators $\hat{a}(\vb{p}),\hat{a}^\dagger(\vb{p})$ obey the commutation relations
	\begin{align}
		\comm{\hat{a}(\vb{p})}{\hat{a}^\dagger(\vb{q})}
		&=
		(2\pi)^3
		\delta^{(3)}(\vb{p}-\vb{q})
		\\
		\comm{\hat{a}^\dagger(\vb{p})}{\hat{a}^\dagger(\vb{q})}
		&=
		\comm{\hat{a}(\vb{p})}{\hat{a}(\vb{q})}
		=
		0
		\label{eq:kg_comm_ac}
		.
	\end{align}	
\end{theorem}
\begin{proof}
	The equal-time commutation relations are valid for any time $t$ so also for $t=0$, inserting \cref{eq:qkg_pos} into \cref{eq:qkg_comm_pm}, we find
	\begin{equation*}
		\begin{split}
			0
			=
			\comm{\hat\phi(0,\vb{x})}{\hat\phi(0,\vb{y})}
			&=
			\int\frac{\dd[3]{p}}{(2\pi)^3}
			\frac{1}{\sqrt{2\omega(\vb{p})}}
			\int\frac{\dd[3]{q}}{(2\pi)^3}
			\frac{1}{\sqrt{2\omega(\vb{q})}}
			\\
			&\times
			\comm{
				\hat{a}(\vb{p})
				e^{+i\vb{p}\vdot\vb{x}}
				+
				\hat{a}^\dagger(\vb{p})
				e^{-i\vb{p}\vdot\vb{x}}
			}{
				\hat{a}(\vb{q})
				e^{+i\vb{q}\vdot\vb{y}}
				+
				\hat{a}^\dagger(\vb{q})
				e^{-i\vb{q}\vdot\vb{y}}
			}
			,
		\end{split}
	\end{equation*}
		and for \cref{eq:qkg_mom}, we find
	\begin{equation*}
		\begin{split}
			0
			=
			\comm{\hat\phi(0,\vb{x})}{\hat\phi(0,\vb{y})}
			&=
			\int\frac{\dd[3]{p}}{(2\pi)^3}
			\left(
				-i
				\sqrt{\frac{\omega(\vb{p})}{2}}
			\right)
			\int\frac{\dd[3]{q}}{(2\pi)^3}
			\left(
				-i
				\sqrt{\frac{\omega(\vb{q})}{2}}
			\right)
			\\
			&\times
			\comm{
				\hat{a}(\vb{p})
				e^{+i\vb{p}\vdot\vb{x}}
				-
				\hat{a}^\dagger(\vb{p})
				e^{-i\vb{p}\vdot\vb{x}}
			}{
				\hat{a}(\vb{q})
				e^{+i\vb{q}\vdot\vb{y}}
				-
				\hat{a}^\dagger(\vb{q})
				e^{-i\vb{q}\vdot\vb{y}}
			}
			.
		\end{split}
	\end{equation*}
	The double integral is zero iff the integrand is zero, therefore
	\begin{align*}
		\comm{
			\hat{a}(\vb{p})
			e^{+i\vb{p}\vdot\vb{x}}
			+
			\hat{a}^\dagger(\vb{p})
			e^{-i\vb{p}\vdot\vb{x}}
		}{
			\hat{a}(\vb{q})
			e^{+i\vb{q}\vdot\vb{y}}
			+
			\hat{a}^\dagger(\vb{q})
			e^{-i\vb{q}\vdot\vb{y}}
		}
		&=
		0
		\\
		\comm{
			\hat{a}(\vb{p})
			e^{+i\vb{p}\vdot\vb{x}}
			-
			\hat{a}^\dagger(\vb{p})
			e^{-i\vb{p}\vdot\vb{x}}
		}{
			\hat{a}(\vb{q})
			e^{+i\vb{q}\vdot\vb{y}}
			-
			\hat{a}^\dagger(\vb{q})
			e^{-i\vb{q}\vdot\vb{y}}
		}
		&=
		0
		.
	\end{align*}
	Adding and subtracting these equations from another implies
	\begin{equation*}
		\comm{\hat{a}(\vb{p})}{\hat{a}(\vb{q})}
		=
		0
		=
		\comm{\hat{a}^\dagger(\vb{p})}{\hat{a}^\dagger(\vb{q})}
		.
	\end{equation*}
	Finally, inserting \cref{eq:qkg_pos} and \cref{eq:qkg_mom} into \cref{eq:qkg_comm_pm}, reveals
	\begin{equation*}
		\begin{split}
			i\delta^{(3)}(\vb{x}-\vb{y})
			=
			\comm{\hat\phi(0,\vb{x})}{\hat\pi(0,\vb{y})}
			&=
			\int\frac{\dd[3]{p}}{(2\pi)^3}
			\frac{1}{\sqrt{2\omega(\vb{p})}}
			\int\frac{\dd[3]{q}}{(2\pi)^3}
			\left(
				-i
				\sqrt{\frac{\omega(\vb{q})}{2}}
			\right)
			\\
			&\times
			\comm{
				\hat{a}(\vb{p})
				e^{+i\vb{p}\vdot\vb{x}}
				+
				\hat{a}^\dagger(\vb{p})
				e^{-i\vb{p}\vdot\vb{x}}
			}{
				\hat{a}(\vb{q})
				e^{+i\vb{q}\vdot\vb{y}}
				-
				\hat{a}^\dagger(\vb{q})
				e^{-i\vb{p}\vdot\vb{y}}
			}
			\\
			&=
			\int\frac{\dd[3]{p}}{(2\pi)^3}
			\int\frac{\dd[3]{q}}{(2\pi)^3}
			\left(
				-
				\frac{i}{2}
				\sqrt{\frac{\omega(\vb{q})}{\omega(\vb{p})}}
			\right)
			\\
			&\times
			\left\{
				-
				\comm{\hat{a}(\vb{p})}{\hat{a}^\dagger(\vb{q})}
				e^{+i\vb{p}\vdot\vb{x}}
				e^{-i\vb{q}\vdot\vb{y}}
				+
				\comm{\hat{a}^\dagger(\vb{p})}{\hat{a}(\vb{q})}
				e^{-i\vb{p}\vdot\vb{x}}
				e^{+i\vb{q}\vdot\vb{y}}
			\right\}
		\end{split}
	\end{equation*}
	which is satisfied for
	\begin{equation*}
		\comm{\hat{a}(\vb{p})}{\hat{a}^\dagger(\vb{q})}
		=
		(2\pi)^3
		\delta^{(3)}(\vb{q}-\vb{p})
		.
	\end{equation*}
\end{proof}

\begin{definition}[Energy and momentum operator]\label{def:qkg_energy_momentum}
	The Klein-Gordon's total energy and total momentum operators are
	\begin{align}
		\hat{H}
		=
		\int\frac{\dd[3]{p}}{(2\pi)^3}
		\omega(\vb{p})\hat{a}^\dagger(\vb{p})\hat{a}(\vb{p})
		&&
		\hat{\vb{P}}
		=
		\int\frac{\dd[3]{p}}{(2\pi)^3}
		\vb{p}\hat{a}^\dagger(\vb{p})\hat{a}(\vb{p})
		\label{eq:qkg_energy_momentum}
		.
	\end{align}
\end{definition}
\begin{definition}\label{def:vacuum_state}
	The vacuum state $\ket{0}$ is the unique eigenstate of the total energy and total momentum operators with eigenvalue zero, i.e.,
	\begin{align}
		\hat{H}
		\ket{0}
		=
		0
		&&
		\hat{\vb{P}}
		\ket{0}
		=
		0
		.
	\end{align}
\end{definition}
\begin{corollary}\label{thm:vacuum_state_ac}
	The definition of the vacuum state and the energy operator implies
	\begin{align}
		\hat{a}(\vb{p})
		\ket{0}
		=
		0
		&&
		\bra{0}
		\hat{a}^\dagger(\vb{p})
		=
		0
		.
	\end{align}
\end{corollary}
\begin{corollary}
	The commutators of the operators $\hat{a}(\vb{p}),\hat{a}^\dagger(\vb{p})$ with the Hamiltonian $\hat{H}$ yield
	\begin{align}
		\comm{\hat{H}}{\hat{a}^\dagger(\vb{p})}
		=
		\omega(\vb{p})
		\hat{a}^\dagger(\vb{p})
		&&
		\comm{\hat{H}}{\hat{a}(\vb{p})}
		=
		-
		\omega(\vb{p})
		\hat{a}(\vb{p})
		.
	\end{align}
\end{corollary}
\begin{definition}
	The number density and total number operators are
	\begin{align}
		\hat{n}(\vb{p})
		=
		\hat{a}^\dagger(\vb{p})
		\hat{a}(\vb{p})
		&&
		\hat{N}
		=
		\int\frac{\dd[3]{p}}{(2\pi)^3}
		\hat{n}(\vb{p})
		=
		\int\frac{\dd[3]{p}}{(2\pi)^3}
		\hat{a}^\dagger(\vb{p})
		\hat{a}(\vb{p})
		.
	\end{align}
\end{definition}
\begin{corollary}
	The total momentum and number observables are conserved
	\begin{equation}
		\comm{\hat{H}}{\hat{\vb{P}}}
		=
		0
		=
		\comm{\hat{H}}{\hat{N}}
		.
	\end{equation}
\end{corollary}

\begin{definition}[Positive and negative frequency Klein-Gordon operator]
	The Klein-Gordon operators can be decomposed into the sum
	\begin{equation}
		\hat\phi(x^\mu)
		=
		\hat\phi^+(x^\mu)
		+
		\hat\phi^-(x^\mu)
	\end{equation}
	wherein the positive and negative frequency Klein-Gordon operators are
	\begin{equation}
		\begin{split}
			\hat\phi^+(x^\mu)
			&=
			\int\frac{\dd[3]{p}}{(2\pi)^3\sqrt{2\omega(\vb{p})}}
			\eval{
				e^{-ip_\mu x^\mu}
				\hat{a}(\vb{p})
			}_{p_0=\omega(\vb{p})}
			\\
			\hat\phi^-(x^\mu)
			&=
			\int\frac{\dd[3]{p}}{(2\pi)^3\sqrt{2\omega(\vb{p})}}
			\eval{
				e^{+ip_\mu x^\mu}
				\hat{a}^\dagger(\vb{p})
			}_{p_0=\omega(\vb{p})}
		\end{split}
		\label{eq:qkg_positive_negative_frequency}
		.
	\end{equation}
	The positive and negative frequency Klein-Gordon operators are related by the hermitian conjugate
	\begin{equation}
		\hat\phi^-(x^\mu)
		=
		\hat\phi^+(x^\mu)^\dagger
		.
	\end{equation}
\end{definition}
\begin{corollary}\label{thm:vacuum_state_pn}
	The positive and negative frequency Klein-Gordon operators satisfy
	\begin{align}
		\hat\phi^+(x^\mu)
		\ket{0}
		=
		0
		&&
		\bra{0}
		\hat\phi^-(x^\mu)
		=
		0
		\label{eq:vacuum_state_pn}
		.
	\end{align}
\end{corollary}
\begin{definition}[Klein-Gordon Propagator]
	The Klein-Gordon propagator is
	\begin{equation}
		D(x^\mu-y^\mu)
		=
		\int\frac{\dd[3]{p}}{(2\pi)^32\omega(\vb{p})}
		\eval{
			e^{-ip_\mu (x^\mu-y^\mu)}
		}_{p_0=\omega(\vb{p})}
		.
	\end{equation}
\end{definition}

\begin{lemma}\label{thm:qkg_comm_pn}
%{qkgcommpn}	
	The positive and negative frequency Klein-Gordon operators satisfy the commutation relations
	\begin{align}
		\comm{\hat\phi^+(x^\mu)}{\hat\phi^-(y^\mu)}
		&=
		D(x^\mu-y^\mu)
		\\
		\comm{\hat\phi^+(x^\mu)}{\hat\phi^+(y^\mu)}
		&=
		\comm{\hat\phi^-(x^\mu)}{\hat\phi^-(y^\mu)}
		=
		0
	\end{align}
\end{lemma}
\begin{proof}
	That positive respective negative frequency Klein-Gordon operators commute follows from \Cref{thm:qkg_comm_ac}.
	Using \Cref{thm:qkg_comm_ac} for the remaining commutator
	\begin{equation*}
		\begin{split}
			\comm{\hat\phi^+(x^\mu)}{\hat\phi^-(y^\mu)}
			&=
			\int\frac{\dd[3]{p}}{(2\pi)^3\sqrt{2\omega(\vb{p})}}
			\int\frac{\dd[3]{q}}{(2\pi)^3\sqrt{2\omega(\vb{q})}}
			\comm{\hat{a}(\vb{p})}{\hat{a}^\dagger(\vb{p})}
			\eval{
				e^{-ip_\mu x^\mu}
				e^{+iq_\mu y^\mu}
			}_{\substack{p_0=\omega(\vb{p})\\q_0=\omega(\vb{q})}}
			\\
			&=
			\int\frac{\dd[3]{p}}{(2\pi)^32\omega(\vb{p})}
			\eval{
				e^{-ip_\mu(x^\mu-y^\mu)}
			}_{p_0=\omega(\vb{p})}
			=
			D(x^\mu-y^\mu)
		\end{split}
	\end{equation*}
	recovers the initial claim after identification of the Klein-Gordon propagator.
\end{proof}

\begin{lemma}\label{thm:qkg_prop_corr}
%{qkgpropcorr}
	The propagator is equal to the expectation value
	\begin{equation}
		D(x^\mu-y^\mu)
		=
		\expval{\hat\phi(x^\mu)\hat\phi(y^\mu)}{0}
		.
	\end{equation}
\end{lemma}
\begin{proof}
	Decomposing the Klein-Gordon operator into positive and negative frequency and using \Cref{thm:vacuum_state_pn}, we find
	\begin{equation*}
		\begin{split}
			\expval{\hat\phi(x^\mu)\hat\phi(y^\mu)}{0}
			&=
			\expval{\hat\phi^+(x^\mu)\hat\phi^-(y^\mu)}{0}
			\\
			&=
			\expval{\comm{\hat\phi^+(x^\mu)}{\hat\phi^-(y^\mu)}}{0}
			\\
			&=
			\expval{D(x^\mu-y^\mu)}{0}
			\\
			&=
			D(x^\mu-y^\mu)
		\end{split}
	\end{equation*}
	where we used \Cref{thm:qkg_comm_pn}.
\end{proof}

\begin{definition}
	The generalized quadrature operator is
	\begin{equation}
		\hat\chi(\theta)
		=
		\frac{1}{2}
		\int\frac{\dd[3]{p}}{(2\pi)^3}
		\left\{
			\hat{a}^\dagger(\vb{p})
			e^{+i\theta}
			+
			\hat{a}(\vb{p})
			e^{-i\theta}
		\right\}
	\end{equation}
	and reduces to the quadrature and in-phase operators
	\begin{align}
		\hat{X}
		&=
		\frac{1}{2}
		\int\frac{\dd[3]{p}}{(2\pi)^3}
		\left\{
			\hat{a}^\dagger(\vb{p})
			+
			\hat{a}(\vb{p})
		\right\}
		=
		\hat\chi(0)
		\\
		\hat{P}
		&=
		\frac{i}{2}
		\int\frac{\dd[3]{p}}{(2\pi)^3}
		\left\{
			\hat{a}^\dagger(\vb{p})
			-
			\hat{a}(\vb{p})
		\right\}
		=
		\hat\chi\left(\frac{\pi}{2}\right)
		.
	\end{align}
\end{definition}
\begin{lemma}
	\begin{equation}
		\comm{\hat\chi(\theta)}{\hat\chi(\theta^\prime)}
		\propto
		i\sin(\theta-\theta^\prime)
	\end{equation}
\end{lemma}
\begin{proof}
	\begin{equation*}
		\begin{split}
			\comm{\hat\chi(\theta)}{\hat\chi(\theta-\Delta\theta)}
			&=
			\frac{1}{4}
			\int\frac{\dd[3]{p}}{(2\pi)^3}
			\int\frac{\dd[3]{q}}{(2\pi)^3}
			\comm{
				\hat{a}^\dagger(\vb{p})
				e^{+i\theta}
				+
				\hat{a}(\vb{p})
				e^{-i\theta}
			}{
				\hat{a}^\dagger(\vb{q})
				e^{+i(\theta-\Delta\theta)}
				+
				\hat{a}(\vb{q})
				e^{-i(\theta-\Delta\theta)}
			}
			\\
			&=
			\frac{1}{4}
			\int\frac{\dd[3]{p}}{(2\pi)^3}
			\int\frac{\dd[3]{q}}{(2\pi)^3}
			\left\{
				\comm{\hat{a}^\dagger(\vb{p})}{\hat{a}(\vb{q})}
				e^{+i\Delta\theta}
				+
				\comm{\hat{a}(\vb{p})}{\hat{a}^\dagger(\vb{q})}
				e^{-i\Delta\theta}
			\right\}
			\\
			&=
			\frac{1}{4}
			\int\frac{\dd[3]{p}}{(2\pi)^3}
			\left\{
				-
				e^{+i\Delta\theta}
				+
				e^{-i\Delta\theta}
			\right\}
			\\
			&=
			\frac{i}{2}
			\int\frac{\dd[3]{p}}{(2\pi)^3}
			\Im e^{+i\Delta\theta}
			\\
			&=
			\frac{i}{2}
			\sin\Delta\theta
			\frac{4\pi}{(2\pi)^3}
			\int_0^\Lambda \dd{p}p^2
			\\
			&=
			i
			\sin\Delta\theta
			\frac{\Lambda^3}{3(2\pi)^2}
		\end{split}
	\end{equation*}
\end{proof}

\section{Canonical quantization (Maxwell)}

\begin{definition}[Transverse delta distribution]\label{def:delta_distribution_transverse}
	The transverse delta distribution
	\begin{equation}
		\label{eq:delta_distribution_transverse}
		\delta_{\perp ij}^{(3)}(\vb{x}-\vb{y})
		=
		P^{ij}_\perp
		\delta^{(3)}(\vb{x}-\vb{y})
		=
		\int\frac{\dd[3]{p}}{(2\pi)^3}
		\left(
			\delta_{ij}
			-
			\frac{p_ip_j}{\vb{p}^2}
		\right)
		e^{i\vb{p}\vdot(\vb{x}-\vb{y})}
		.
	\end{equation}
\end{definition}
\begin{corollary}[Transverse Maxwell operator]\label{thm:qmw_transverse}
	Promoting the dynamical field variables of the transverse Maxwell field to operators
	\begin{align}
		\label{eq:qmw_transverse}
		\hat{\vb{A}}_\perp(x^\mu)
		&=
		\sum_{\lambda=1,2}
		\int_{\mathbb{R}^3}\frac{\dd[3]{p}}{(2\pi)^3}
		\frac{1}{\sqrt{2\omega(\vb{p})}}
		\left\{
			\hat{a}_\lambda(\vb{p})
			\boldsymbol{\epsilon}_\lambda(\vb{p})
			e^{-ip_\mu x^\mu}
			+
			\text{c.c.}
		\right\}_{p_0=\omega(\vb{p})}
		\\
		\hat{\vb{E}}_\perp(x^\mu)
		&=
		\sum_{\lambda=1,2}
		\int_{\mathbb{R}^3}\frac{\dd[3]{p}}{(2\pi)^3}
		\left(
			-i
			\sqrt{\frac{\omega(\vb{p})}{2}}
		\right)
		\left\{
			\hat{a}_\lambda(\vb{p})
			\boldsymbol{\epsilon}_\lambda(\vb{p})
			e^{-ip_\mu x^\mu}
			-
			\text{c.c.}
		\right\}_{p_0=\omega(\vb{p})}
	\end{align}
	satisfying the equal-time commutation relations~\cite[p.~197]{Greiner2013}
	\begin{equation}
		\label{eq:qmw_transverse_comm}
		\comm{\hat{A}_\perp^i(t,\vb{x})}{\hat{E}_\perp^j(t,\vb{y})}
		=
		-i
		\delta_{\perp ij}^{(3)}(\vb{x}-\vb{y})
		=
		\comm{\hat{A}_\perp^i(t,\vb{x})}{-\hat\pi_\perp^j(t,\vb{y})}
	\end{equation}
	where $\hat\pi_\perp$ is the canonical momentum density operator, classically defined in \Cref{thm:mw_coulomb_canonical_momentum}.
\end{corollary}
\begin{definition}\label{thm:qmw_hamilton}
	The Hamilton operator of the transverse Maxwell field is the normal-ordered classical energy
	\begin{equation}
		\label{eq:qmw_hamilton}
		\hat{H}
		=
		\sum_{\lambda=1,2}
		\int\frac{\dd[3]{p}}{(2\pi)^3}
		\omega(\vb{p})
		\hat{a}_\lambda^\dagger(\vb{p})
		\hat{a}_\lambda(\vb{p})
	\end{equation}
	where $\omega(\vb{p})=\norm{\vb{p}}$.
\end{definition}
\begin{definition}\label{thm:qmw_number}
	The (particle) number operator of the transverse Maxwell field is the unweighted part of the Hamilton operator
	\begin{equation}
		\label{eq:qmw_number}
		\hat{N}
		=
		\sum_{\lambda=1,2}
		\int\frac{\dd[3]{p}}{(2\pi)^3}
		\hat{a}_\lambda^\dagger(\vb{p})
		\hat{a}_\lambda(\vb{p})
		.
	\end{equation}
\end{definition}
\begin{lemma}\label{thm:qmw_comm_ac}
	The annihilation and creation operators of the transverse Maxwell field satisfy
	\begin{align}
		\label{eq:qmw_comm_ac}
		\comm{\hat{a}_\lambda(\vb{p})}{\hat{a}_{\lambda^\prime}^\dagger(\vb{q})}
		&=
		(2\pi)^3
		\delta^{(3)}(\vb{q}-\vb{p})
		\delta_{\lambda\lambda^\prime}
		\\
		\comm{\hat{a}_\lambda^\dagger(\vb{p})}{\hat{a}_{\lambda^\prime}^\dagger(\vb{q})}
		&=
		\comm{\hat{a}_\lambda(\vb{p})}{\hat{a}_{\lambda^\prime}(\vb{q})}
		=
		0
		.
	\end{align}
\end{lemma}
\begin{proof}
	foobar
\end{proof}

\begin{definition}[Smearing vector function]
	Let $f_1,f_2,f_3\in\mathcal{S}(\mathbb{R},\mathbb{R}^3)$ be smearing functions, then
	\begin{equation}
		\begin{split}
			\vb{f}
			\colon
			\mathbb{R}\times \mathbb{R}^3
			&\to
			\mathbb{R}
			\\
			x^\mu
			&\mapsto
			\vb{f}(x^\mu)
			=
			\begin{pmatrix}
				f_1(x^\mu) \\
				f_2(x^\mu) \\
				f_3(x^\mu)				
			\end{pmatrix}
		\end{split}
	\end{equation}
	defines the smearing vector functioin.
\end{definition}
\begin{definition}[Transverse smearing function]
	Let $\vb{f}$ be a smearing vector, then
	\begin{equation}
		\label{eq:qmw_transverse_smearing_function}
		f_\lambda\left(\omega(\vb{p}),\vb{p}\right)
		=
		\vb{f}\left(\omega(\vb{p}),\vb{p}\right)
		\vdot
		\boldsymbol{\varepsilon}_\lambda(\vb{p})
	\end{equation}
	defines the transverse $\lambda$ component of the smearing vector function.
\end{definition}

\begin{definition}\label{thm:qmw_transverse_pn}
	The positive and negative transverse Maxwell operator are
	\begin{align}
		\label{eq:qmw_transverse_pn}
		\hat{\vb{A}}^+_\perp(x^\mu)
		&=
		\sum_{\lambda=1,2}
		\int_{\mathbb{R}^3}\frac{\dd[3]{p}}{(2\pi)^3\sqrt{2\omega(\vb{p})}}
		\hat{a}_\lambda^\dagger(\vb{p})
		\boldsymbol{\epsilon}_\lambda(\vb{p})^*
		\eval{e^{+ip_\mu x^\mu}}_{p_0=\omega(\vb{p})}
		\\
		\hat{\vb{A}}^-_\perp(x^\mu)
		&=
		\sum_{\lambda=1,2}
		\int_{\mathbb{R}^3}\frac{\dd[3]{p}}{(2\pi)^3\sqrt{2\omega(\vb{p})}}
		\hat{a}_\lambda(\vb{p})
		\boldsymbol{\epsilon}_\lambda(\vb{p})
		\eval{e^{-ip_\mu x^\mu}}_{p_0=\omega(\vb{p})}
		.
	\end{align}
\end{definition}
\begin{lemma}\label{thm:qmw_comm_pn}
	\begin{align}
		\label{eq:qmw_comm_ac}
		\comm{\hat{\vb{A}}_\perp^-(x^\mu)}{\hat{\vb{A}}_\perp^+(y^\mu)}
		&=
		\\
		\comm{\hat{\vb{A}}_\perp^-(x^\mu)}{\hat{\vb{A}}_\perp^-(y^\mu)}
		&=
		\comm{\hat{\vb{A}}_\perp^+(x^\mu)}{\hat{\vb{A}}_\perp^+(y^\mu)}
		=
		0
		.
	\end{align}
\end{lemma}

\begin{definition}\label{def:qmw_transverse_pn_smeared}
	Let $\vb{f}$ be a smearing vector function, then the smeared positive and negative frequency transverse Maxwell operators are
	\begin{align}
		\label{eq:qmw_transverse_pn_smeared}
		\hat{\vb{A}}_\perp^+[\vb{f}]
		&=
		\int\dd[4]{x}\
		\vb{f}(x^\mu)
		\vdot
		\hat{\vb{A}}_\perp^+(x^\mu)
		\\
		\hat{\vb{A}}_\perp^-[\vb{f}]
		&=
		\int\dd[4]{x}\
		\vb{f}(x^\mu)
		\vdot
		\hat{\vb{A}}_\perp^-(x^\mu)
		.
	\end{align}
\end{definition}
\begin{lemma}\label{thm:qmw_transverse_pn_smeared}
	The mode expanded smeared positive and negative frequency transverse Maxwell operators are
	\begin{align}
		\label{eq:qmw_transverse_pn}
		\hat{\vb{A}}^+_\perp(x^\mu)
		&=
		\sum_{\lambda=1,2}
		\int_{\mathbb{R}^3}\frac{\dd[3]{p}}{(2\pi)^3\sqrt{2\omega(\vb{p})}}
		f_\lambda\left(\omega(\vb{p}),\vb{p}\right)^*
		\hat{a}_\lambda^\dagger(\vb{p})
		\\
		\hat{\vb{A}}^-_\perp(x^\mu)
		&=
		\sum_{\lambda=1,2}
		\int_{\mathbb{R}^3}\frac{\dd[3]{p}}{(2\pi)^3\sqrt{2\omega(\vb{p})}}
		f_\lambda\left(\omega(\vb{p}),\vb{p}\right)
		\hat{a}_\lambda(\vb{p})
		.
	\end{align}	
\end{lemma}
\begin{proof}
	Inserting the definition of the positive and negative transverse Maxwell field and the Fourier transform, we find
	\begin{equation*}
		\begin{split}
			\int\dd[4]{x}\
			\vb{f}(x^\mu)
			\vdot
			\hat{\vb{A}}_\perp^+(x^\mu)
			&=
			\int\dd[4]{x}\
			\vb{f}(x^\mu)
			\vdot
			\sum_{\lambda=1,2}
			\int\frac{\dd[3]{p}}{(2\pi)^3\sqrt{2\omega(\vb{p})}}
			\hat{a}_\lambda^\dagger(\vb{p})
			\boldsymbol{\varepsilon}_\lambda(\vb{p})^*
			\eval{e^{+ip_\mu x^\mu}}_{p_0=\omega(\vb{p})}
			\\
			&=
			\sum_{\lambda=1,2}
			\int\frac{\dd[3]{p}}{(2\pi)^3\sqrt{2\omega(\vb{p})}}
			\hat{a}_\lambda^\dagger(\vb{p})
			\boldsymbol{\varepsilon}_\lambda(\vb{p})^*
			\vdot
			\left(
				\int\dd[4]{x}\
				\vb{f}(x^\mu)^*
				e^{-ip_\mu x^\mu}
			\right)_{p_0=\omega(\vb{p})}^*
			\\
			&=
			\sum_{\lambda=1,2}
			\int\frac{\dd[3]{p}}{(2\pi)^3\sqrt{2\omega(\vb{p})}}
			\hat{a}_\lambda^\dagger(\vb{p})
			\left[
				\vb{f}\left(\omega(\vb{p}),\vb{p}\right)
				\vdot
				\boldsymbol{\varepsilon}_\lambda(\vb{p})
			\right]^*
			.
		\end{split}
	\end{equation*}
	Taking the Hermitian conjugate, we find the smeared negative frequency transverse Maxwell operator.
\end{proof}

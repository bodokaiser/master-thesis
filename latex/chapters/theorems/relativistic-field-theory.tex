\section{Relativistic field theory (Klein-Gordon)}

\begin{definition}[Klein-Gordon Lagrangian]
	The Klein-Gordon Lagrangian density
	\begin{equation}
		\mathcal{L}
		=
		\frac{1}{2}
		\left(\partial_\mu\phi\right)
		\left(\partial^\mu\phi\right)
		-
		\frac{1}{2}
		m^2\phi^2
		\label{eq:kg_lagrangian}
	\end{equation}
	describes a real-valued massive scalar field $\phi(t,\vb{x})$.
\end{definition}

The Klein-Gordon Lagrangian is manifest Lorentz-invariant, i.e., as a Lorentz scalar, the Klein-Gordon Lagrangian is invariant under Lorentz transformations and thereby valid in any reference frame.
\begin{theorem}[Relativistic energy-momentum relation]{theorem}\label{th:relativistic_energy_momentum}
% rem
	Excitations of the Klein-Gordon field satisfy the relativistic energy-momentum relation
	\begin{equation}
		\omega(\vb{p})
		=
		\sqrt{\vb{p}^2+m^2}
		=
		E(\vb{p})
		\label{eq:energy_momentum_relation}
		.
	\end{equation}
\end{theorem}
\begin{proof}
	According to the action principle, the dynamics of the field are determined by the equations of motion which can be found by the relativistic Euler-Lagrange equations
	\begin{equation*}
		0
		=
		\partial_\mu\pdv{\mathcal{L}}{(\partial_\mu\phi)}
		-
		\pdv{\mathcal{L}}{\phi}
		=
		\left(
			\partial_\mu\partial^\mu
			+
			m^2
		\right)
		\phi(t,\vb{x})
		.
	\end{equation*}
	Assuming the existence of the Klein-Gordon field's Fourier representation\footnote{See Ref.~\cite[p.~341]{Cohen2019} for a definition of a Minkowski Fourier transform.}
	\begin{equation*}
		\phi(t,\vb{x})
		=
		\int_{\mathbb{R}^3}\frac{\dd[3]{p}}{(2\pi)^3}
		\phi(t,\vb{p})
		e^{-i\vb{p}\vdot\vb{x}}
		=
		\int_{\mathbb{R}^4}\frac{\dd[4]{p}}{(2\pi)^4}
		\phi(p_0,\vb{p})
		e^{+ip_\mu x^\mu}
		,
	\end{equation*}
	the equation of motion in momentum space reduces to
	\begin{equation*}
		0
		=
		\left(
			ip_\mu ip^\mu
			+
			m^2
		\right)
		\phi(p_0,\vb{p})
		=
		-
		\left(
			p_0^2
			-
			\omega(\vb{p})^2
		\right)
		\phi(p_0,\vb{p})
	\end{equation*}
	which is satisfied if $p_0=\pm\omega(\vb{p})$.
\end{proof}

\begin{theorem}[Mode expansion of the Klein-Gordon field]\label{thm:kg_mode_expansion}
%{kgmodeexp}
	The mode expansion of the Klein-Gordon field
	\begin{equation}
		\phi(x^\mu)
		=
		\int_{\mathbb{R}^3}\frac{\dd[3]{p}}{(2\pi)^3}
		\frac{1}{\sqrt{2\omega(\vb{p})}}
		\biggl\{
			a(\vb{p})
			e^{-ip_\mu x^\mu}
			+
			a(\vb{p})^*
			e^{+ip_\mu x^\mu}
		\biggr\}_{p_0=\omega(\vb{p})}
		\label{eq:kg_mode_expansion}
	\end{equation}
	satisfies the equations of motion for any choice of $a(\vb{p})=\phi(\omega(\vb{p}),\vb{p})$ where $\phi(p_0,\vb{p})=\phi(p^\mu)$ is the Fourier amplitude of the Klein-Gordon field $\phi(x^\mu)=\phi(t,\vb{x})$.
\end{theorem}
\begin{proof}
	There are two approaches to prove \cref{thm:kg_mode_expansion}.
	In a first approach, we calculate with the proposed Fourier expansion
	\begin{equation*}
		\begin{split}
			\partial_\mu
			\partial^\mu
			\phi(t,\vb{x})
			&=
			\int\frac{\dd[3]{p}}{(2\pi)^3\sqrt{2\omega(\vb{p})}}
			\left\{
				a^*(\vb{p})
				\partial_\mu
				\partial^\mu
				e^{-ip_\mu x^\mu}
				+
				\text{c.c.}
			\right\}_{p_0=\omega(\vb{p})}
			\\
			&=
			\int\frac{\dd[3]{p}}{(2\pi)^3\sqrt{2\omega(\vb{p})}}
			\left\{
				a^*(\vb{p})
				\left(
					-
					p_\mu
					p^\mu
				\right)
				e^{-ip_\mu x^\mu}
				+
				\text{c.c.}
			\right\}_{p_0=\omega(\vb{p})}
		\end{split}
	\end{equation*}
	where $p_\mu p^\mu=\omega(\vb{p})^2-\vb{p}=m^2$ and therefore the equation of motion
	\begin{equation*}
			\partial_\mu
			\partial^\mu
			\phi(t,\vb{x})
			=
			-
			m^2
			\phi(t,\vb{x})
	\end{equation*}
	is satisfied.
	While the first approach successfully shows why the theorem is true, it is not obvious how to arrive at the Fourier expansion.
	Therefor, a second approach starts with the Fourier transform of the Klein-Gordon field
	\begin{equation*}
		\phi(t,\vb{x})
		=
		\int_V\frac{\dd[4]{p}}{(2\pi)^4}
		\phi(p_0,\vb{p})
		e^{+ip_\mu x^\mu}
	\end{equation*}
	where the integration domain is constrained to the momentum lightcone $V$ and therefore $\omega(\vb{p})^2=p_0^2$ is automatically satisfied.
	We are left to rewrite the constrained integration domain
	\begin{equation*}
		\phi(t,\vb{x})
		=
		\int_{\mathbb{R}^4}\frac{\dd[4]{p}}{(2\pi)^3}
		\delta^{(1)}\left(p_0^2-\omega(\vb{p})^2\right)
		\phi(p_0,\vb{p})
		e^{+ip_\mu x^\mu}
	\end{equation*}
	and using the composition property of the delta distribution
	\begin{equation*}
		\delta^{(1)}\left(p_0^2-\omega(\vb{p})^2\right)
		=
		\frac{
			\delta^{(1)}\left(p_0-\omega(\vb{p})\right)
			+
			\delta^{(1)}\left(p_0+\omega(\vb{p})\right)
		}{\sqrt{2\omega(\vb{p})}}
	\end{equation*}
	which leaves us with
	\begin{equation*}
		\phi(t,\vb{x})
		=
		\int_{\mathbb{R}^3}\frac{\dd[3]{p}}{(2\pi)^3\sqrt{2\omega(\vb{p})}}
		\biggl\{
			\phi(\omega(\vb{p}),\vb{p})
			e^{+i\omega(\vb{p})t}
			+
			\phi(-\omega(\vb{p}),\vb{p})
			e^{-i\omega(\vb{p})t}
		\biggr\}
		e^{-i\vb{p}\vdot\vb{x}}
		.
	\end{equation*}
	We now only need to perform the substitution $\vb{p}\to-\vb{p}$ on the second term
	\begin{equation*}
		\begin{split}
			\phi(t,\vb{x})
			&=
			\int_{\mathbb{R}^3}\frac{\dd[3]{p}}{(2\pi)^3\sqrt{2\omega(\vb{p})}}
			\biggl\{
				\phi(\omega(\vb{p}),\vb{p})
				e^{+i\omega(\vb{p})t}
				e^{-i\vb{p}\vdot\vb{x}}
				+
				\phi(-\omega(\vb{p}),\vb{p})
				e^{-i\omega(\vb{p})t}
				e^{-i\vb{p}\vdot\vb{x}}
			\biggr\}
			\\
			&=
			\int_{\mathbb{R}^3}\frac{\dd[3]{p}}{(2\pi)^3\sqrt{2\omega(\vb{p})}}
			\biggl\{
				\phi(\omega(\vb{p}),\vb{p})
				e^{+i\omega(\vb{p})t}
				e^{-i\vb{p}\vdot\vb{x}}
				+
				\phi(-\omega(\vb{p}),-\vb{p})
				e^{-i\omega(\vb{p})t}
				e^{+i\vb{p}\vdot\vb{x}}
			\biggr\}
			\\
			&=
			\int_{\mathbb{R}^3}\frac{\dd[3]{p}}{(2\pi)^3\sqrt{2\omega(\vb{p})}}
			\biggl\{
				\phi(\omega(\vb{p}),\vb{p})
				e^{+ip_\mu x^\mu}
				+
				\phi(\omega(\vb{p}),\vb{p})^*
				e^{-ip_\mu x^\mu}
			\biggr\}_{p_0=\omega(\vb{p})}
		\end{split}
	\end{equation*}
	and use the conjugate symmetry of the Fourier amplitudes
	\begin{equation*}
		\phi(p_0,\vb{p})^*
		=
		\phi(-p_0,-\vb{p})
	\end{equation*}
	which is present because $\phi(t,\vb{x})$ is real-valued.
\end{proof}

\begin{definition}[Energy-momentum tensor]
	The energy-momentum tensor of a scalar field is
	\begin{equation}
		T^{\mu\nu}
		=
		\pdv{\mathcal{L}}{(\partial_\mu\phi)}\partial^\nu\phi
		-
		\eta^{\mu\nu}\mathcal{L}
		\label{eq:energy_momentum_tensor}
	\end{equation}
	where $\eta^{\mu\nu}$ is the Minkowski metric.
	The energy-momentum tensor's components encode the energy density $T^{00}$, the momentum density $T^{0i}$, and the stress densities $T^{ij}$.
\end{definition}
\begin{lemma}\label{thm:kg_energy_density}
	The energy density of the Klein-Gordon's energy-momentum tensor
	\begin{equation}
		T^{00}
		=
		\frac{1}{2}
		\left(\partial_t\phi\right)^2
		+
		\frac{1}{2}
		\left(\grad\phi\right)^2
		+
		\frac{1}{2}
		\left(m\phi\right)^2
		=
		\mathcal{H}
		\label{thm:kg_hamiltonian}
	\end{equation}
	is equal to its Hamiltonian density $\mathcal{H}$.
\end{lemma}
\begin{lemma}\label{thm:kg_energy_momentum}
	The energy and the momentum of the Klein-Gordon field are
	\begin{align}
		H
		=
		\int\frac{\dd[3]{p}}{(2\pi)^3}
		\omega(\vb{p})\abs{a(\vb{p})}^2
		&&
		\vb{P}
		=
		\int\frac{\dd[3]{p}}{(2\pi)^3}
		\vb{p}\abs{a(\vb{p})}^2
		\label{eq:kg_energy_momentum}
		.
	\end{align}
\end{lemma}
\begin{lemma}\label{thm:kg_conjugate_momentum}
	The Klein-Gordon field $\phi(x^\mu)$ with mode expansion given in \cref{eq:kg_mode_expansion}, and the conjugate momentum field
	\begin{equation}
		\pi(x^\mu)
		=
		\int_{\mathbb{R}^3}\frac{\dd[3]{p}}{(2\pi)^3}
		\left(
			-i
			\sqrt{\frac{\omega(\vb{p})}{2}}
		\right)
		\biggl\{
			a(\vb{p})
			e^{-ip_\mu x^\mu}
			-
			a(\vb{p})^*
			e^{+ip_\mu x^\mu}
		\biggr\}_{p_0=\omega(\vb{p})}
		\label{eq:kg_conjugate_momentum}
	\end{equation}
	satisfy the Hamiltonian equations of motion.
\end{lemma}

\section{Relativistic field theory (Maxwell)}

\begin{definition}[Maxwell Lagrangian]
	The Lagrangian of the Maxwell field $A^\mu(x^\mu)$ interacting with an external source $J^\mu(x^\mu)$ is~\cite[p.~339]{Srednicki2007}.
	\begin{equation}
		\mathcal{L}
		=
		\frac{1}{2}
		(\partial_\mu A_\nu)
		\left(
			\partial^\nu A^\mu
			-
			\partial^\mu A^\nu
		\right)
		-
		J_\mu A^\mu
		\label{eq:mw_lagrangian}
		.
	\end{equation}
\end{definition}
\begin{definition}[Field-strength tensor]
	The Maxwell's field covariant field-strength tensor and its dual are
	\begin{align}
		F^{\mu\nu}
		=
		\partial^\mu A^\nu
		-
		\partial^\nu A^\mu
		&&
		\tilde{F}^{\mu\nu}
		=
		\frac{1}{2}
		\varepsilon^{\mu\nu\alpha\beta}
		F_{\alpha\beta}
		\label{eq:mw_field_strength_tensors}
	\end{align}
	where $\varepsilon^{\mu\nu\alpha\beta}$ is the generalized antisymmetric tensor.
	The components of the field-strength tensor are
	\begin{align}
		F^{0i}
		=
		E^i
		&&
		F^{ij}
		=
		\varepsilon^{ijk}B_k
		\label{eq:mw_field_strength_components}
	\end{align}
	where $E^i$ and $B_k$ are electric respective magnetic field components~\cite[p.~20]{Carroll1997}.
\end{definition}
\begin{corollary}
	The field-strength tensors are antisymmetric
	\begin{align}
		F^{\mu\nu}
		=
		-
		F^{\nu\mu}
		&&
		\tilde{F}^{\mu\nu}
		=
		-
		\tilde{F}^{\nu\mu}
	\end{align}
	implying vanishing diagonal components $F^{\mu\mu}=0=\tilde{F}^{\mu\mu}$.
\end{corollary}

\begin{lemma}
	Using the field-strength tensor $F^{\mu\nu}$, the Maxwell Lagrangian becomes
	\begin{equation}
		\mathcal{L}
		=
		-
		\frac{1}{4}
		F_{\mu\nu}
		F^{\mu\nu}
		-
		J_\mu A^\mu
		\label{eq:mw_field_strength_lagrangian}
	\end{equation}
\end{lemma}
\begin{proof}
	Inserting the definition of the field-strength tensor \cref{eq:mw_field_strength_tensors} into the free term of the Lagrangian \cref{eq:mw_field_strength_lagrangian}
	\begin{equation*}
		\begin{split}
			-
			\frac{1}{4}
			F_{\mu\nu}
			F^{\mu\nu}
			&=
			-
			\frac{1}{4}
			\left(
				\partial_\mu A_\nu
				-
				\partial_\nu A_\mu
			\right)
			\left(
				\partial^\mu A^\nu
				-
				\partial^\nu A^\mu
			\right)
			\\
			&=
			-
			\frac{1}{4}
			\left[
				(\partial_\mu A_\nu)
				(\partial^\mu A^\nu)
				-
				(\partial_\mu A_\nu)
				(\partial^\nu A^\mu)
				-
				(\partial_\nu A_\mu)
				(\partial^\mu A^\nu)
				+
				(\partial_\nu A_\mu)
				(\partial^\nu A^\mu)
			\right]
			\\
			&=
			-
			\frac{1}{4}
			\left[
				(\partial_\mu A_\nu)
				(\partial^\mu A^\nu)
				-
				(\partial_\mu A_\nu)
				(\partial^\nu A^\mu)
				-
				(\partial_\mu A_\nu)
				(\partial^\nu A^\mu)
				+
				(\partial_\mu A_\nu)
				(\partial^\mu A^\nu)
			\right]
			\\
			&=
			-
			\frac{1}{2}
			\left[
				(\partial_\mu A_\nu)
				(\partial^\mu A^\nu)
				-
				(\partial_\mu A_\nu)
				(\partial^\nu A^\mu)
			\right]
			\\
			&=
			\frac{1}{2}
			(\partial_\mu A_\nu)
			\left(
				\partial_\mu A_\nu
				-
				\partial^\mu A^\nu
			\right)
		\end{split}
	\end{equation*}
	where we relabeled the indices in the two last terms at the second equal.
\end{proof}

\begin{lemma}
	The dynamics of the Maxwell field are governed by the equations of motion
	\begin{equation}
		J^\mu
		=
		\partial_\mu
		F^{\mu\nu}
		\label{eq:mw_eom}
		.
	\end{equation}
\end{lemma}
\begin{proof}
	Using the Lagrangian in terms of the field-strength tensor, we first calculate
	\begin{equation*}
		\pdv{\mathcal{L}}{(\partial_\mu A_\nu)}
		=
		-
		\frac{1}{2}
		F^{\alpha\beta}
		\pdv{F^{\alpha\beta}}{(\partial_\mu A_\nu)}
		=
		-
		\frac{1}{2}
		F^{\alpha\beta}
		\left(
			\delta^\alpha_\mu
			\delta^\beta_\nu
			-
			\delta^\alpha_\nu
			\delta^\beta_\mu
		\right)
		=
		-
		\frac{1}{2}
		F^{\mu\nu}
		+
		\frac{1}{2}
		F^{\nu\mu}
		=
		-
		F^{\mu\nu}
	\end{equation*}
	and use the result for the Euler-Lagrange equations
	\begin{equation*}
		0
		=
		\partial_\mu
		\pdv{\mathcal{L}}{(\partial_\mu A_\nu)}
		-
		\pdv{\mathcal{L}}{A_\nu}
		=
		-
		\partial_\mu
		F^{\mu\nu}
		+
		J^\nu
	\end{equation*}
	which produces \cref{eq:mw_eom}.
\end{proof}

\begin{lemma}
	Contracting the field-strength tensor and its dual yields
		\begin{align}
		F_{\mu\nu}
		F^{\mu\nu}
		=
		-2
		\left(
			\vb{E}^2
			-
			\vb{B}^2
		\right)
		&&
		\tilde{F}_{\mu\nu}
		F^{\mu\nu}
		=
		-
		4
		\vb{E}
		\vdot
		\vb{B}
		.
	\end{align}
\end{lemma}
\begin{proof}
	For the first contraction, we find
	\begin{equation*}
		\begin{split}
			F_{\mu\nu}
			F^{\mu\nu}
			&=
			F_{0\nu}
			F^{0\nu}
			+
			F_{i\nu}
			F^{i\nu}
			=
			F_{0i}
			F^{0i}
			+
			F_{i0}
			F^{i0}
			+
			F_{ij}
			F^{ij}
			\\
			&=
			2
			F_{0i}
			F^{0i}
			+
			F_{ij}
			F^{ij}
			=
			-2
			E_i
			E^i
			+
			\varepsilon^{ijk}B_k
			\varepsilon_{ijl}B^l
			\\
			&=
			-
			2
			E_i
			E^i
			+
			2
			B_k
			B^k
			=
			-
			2
			\left(
				\vb{E}^2
				-
				\vb{B}^2
			\right)
		\end{split}
	\end{equation*}
	and for the second contraction
	\begin{equation*}
		\begin{split}
			\tilde{F}_{\mu\nu}
			F^{\mu\nu}
			&=
			\tilde{F}_{0\nu}
			F^{0\nu}
			+
			\tilde{F}_{i\nu}
			F^{i\nu}
			=
			\tilde{F}_{0i}
			F^{0i}
			+
			\tilde{F}_{i0}
			F^{i0}
			+
			\tilde{F}_{ij}
			F^{ij}
			\\
			&=
			2
			\tilde{F}_{0i}
			F^{0i}
			+
			\tilde{F}_{ij}
			F^{ij}
			=
			-
			2B_iE^i
			-
			\varepsilon_{ijk}E^k
			\varepsilon^{ijkl}B_l
			\\
			&=
			-
			2B_iE^i
			-
			2
			E^k
			\delta_k^l
			B_l
			=
			-
			4
			\vb{E}\vdot\vb{B}
			.
		\end{split}
	\end{equation*}
\end{proof}

\begin{corollary}
	Using \Cref{thm:mw_field_strength_contracted}, we can write the Maxwell Lagrangian in terms of the electromagnetic fields
	\begin{equation}
		\mathcal{L}
		=
		\frac{1}{2}
		\left(
			\vb{E}^2
			-
			\vb{B}^2
		\right)
		-
		\rho A^0
		+
		\vb{J}\vdot\vb{A}
	\end{equation}
	where $\rho$ is a charge and $\vb{J}$ a current density which together form the relativistic current $J^\mu(\rho,\vb{J})$.
\end{corollary}

\begin{theorem}
	The manifest Lorentz-covariant Maxwell equations are
	\begin{align}
		0
		=
		\partial_\mu
		\tilde{F}^{\mu\nu}
		&&
		J^\nu
		=
		\partial_\mu
		F^{\mu\nu}
		\label{eq:tensor_maxwell}.
	\end{align}
\end{theorem}
\begin{proof}
	The inhomogeneous Maxwell equations are the equations of motion which are found from the Euler-Lagrange equation
	\begin{equation*}
		0
		=
		\partial_\mu
		\pdv{\mathcal{L}}{(\partial_\mu A_\nu)}
		-
		\pdv{\mathcal{L}}{A_\nu}
		=
		-
		\partial_\mu
		F^{\mu\nu}
		+
		J^\nu
	\end{equation*}
	where we used
	\begin{equation*}
		\begin{split}
			\pdv{\mathcal{L}}{(\partial_\mu A_\nu)}
			&=
			\pdv{\mathcal{L}}{F_{\alpha\beta}}
			\pdv{F_{\alpha\beta}}{(\partial_\mu A_\nu)}
			\\
			&=
			-
			\frac{1}{4}
			\pdv{(F_{\sigma\rho}F^{\sigma\rho})}{F_{\alpha\beta}}
			\pdv{(\partial_\alpha A_\beta-\partial_\beta A_\alpha)}{(\partial_\mu A_\nu)}
			\\
			&=
			-
			\frac{1}{2}
			F^{\alpha\beta}
			\left(
				\delta_\alpha^\mu
				\delta_\beta^\nu
				-
				\delta_\alpha^\nu
				\delta_\beta^\mu
			\right)
			\\
			&=
			-
			\frac{1}{2}
			F^{\mu\nu}
			+
			\frac{1}{2}
			F^{\nu\mu}
			=
			-
			F^{\mu\nu}			
		\end{split}
	\end{equation*}
	The homogeneous Maxwell equations are a consequence of the Bianchi identity and antisymmetry of $F^{\mu\nu}$.
\end{proof}

\begin{theorem}\label{thm:vector_maxwell}
	The non-covariant Maxwell equations are
	\begin{align}
		\div\vb{E}
		=
		\rho
		&&
		\div\vb{B}
		=
		0
		\label{eq:vector_maxwell_homo}
		\\
		\curl\vb{E}
		=
		-
		\partial_t\vb{B}
		&&
		\curl\vb{B}
		=
		\vb{J}
		+
		\partial_t\vb{E}
		\label{eq:vector_maxwell_inhomo}
	\end{align}
\end{theorem}
\begin{proof}
	Evaluating the time component of \cref{eq:tensor_maxwell} yields the Gauss' law for magnetism
	\begin{equation*}
		0
		=
		\varepsilon_{0\lambda\mu\nu}\partial^\lambda F^{\mu\nu}
		=
		\varepsilon_{0ijk}\partial^iF^{jk}
		=
		-
		\varepsilon_{ijk}\varepsilon_{ljk}
		\partial^i B_l
		=
		2\partial_iB^i
	\end{equation*}
	and the spatial component yields Ampere's circuit law
	\begin{equation*}
		0
		=
		\varepsilon_{i\lambda\mu\nu}
		\partial^\lambda
		F^{\mu\nu}
		=
		-
		\varepsilon_{ijk}
		\varepsilon^{ljk}
		\partial_t B_l
		-
		2\varepsilon_{ijk}
		\partial^jE^k
		=
		\partial_tB_i
		+
		\varepsilon_{ijk}
		\partial^jE_k
		.
	\end{equation*}
	The time component of the inhomogeneous covariant Maxwell equation \cref{eq:mw_eq} yields Gauss' law
	\begin{equation*}
		J^0
		=
		\rho
		=
		\partial_\mu F^{\mu\nu}
		=
		\partial_i E^i
		,
	\end{equation*}
	and the spatial component yields Faraday's law of induction
	\begin{equation*}
		J^i
		=
		\partial_\mu F^{\mu i}
		=
		-
		\partial_t E^i
		+
		\varepsilon^{ijk}
		\partial_j
		B_k
		.
	\end{equation*}
	and we derived the vector Maxwell equations from first principles.
\end{proof}

\begin{definition}[Local gauge transformation]
	A local gauge transformation
	\begin{equation}
		A_\mu(t,\vb{x})
		\to
		A_\mu^\prime(t,\vb{x})
		=
		A_\mu(t,\vb{x})
		+
		\partial_\mu\Lambda(t,\vb{x})
		\label{eq:mw_local_gauge_transform}
	\end{equation}
	adds the gradient of a scalar field to a (vector) field.\footnote{In a strict sense, a local gauge transformation is defined with respect to the Dirac field and \cref{eq:mw_local_gauge_transform} is a consequence that the Dirac combined with the Maxwell Lagrangian should be invariant under such transformations.}
\end{definition}


\begin{theorem}\label{thm:mw_local_gauge_invariance}
	The physical observables of the Maxwell field are invariant under local gauge transformations.
\end{theorem}
\begin{proof}
	The physical field-strength tensor transforms under \cref{eq:mw_local_gauge_transform} as
	\begin{equation*}
		\begin{split}
			F_{\mu\nu}
			\to
			F_{\mu\nu}^\prime
			&=
			\partial_\mu\left(A_\nu+\partial_\nu\Lambda\right)
			-
			\partial_\nu\left(A_\mu+\partial_\mu\Lambda\right)
			\\
			&=
			F_{\mu\nu}
			+
			\partial_\mu\partial_\nu\Lambda
			-
			\partial_\nu\partial_\mu\Lambda
			=
			F_{\mu\nu}
		\end{split}
	\end{equation*}	
\end{proof}


Unphysical degrees of freedom of the Maxwell field can be removed by choosing a specific scalar field $\Lambda(t,\vb{x})$ and performing a local gauge transformation.
The procedure is known as gauge fixing or imposing a gauge and the resulting restriction on the field is known as gauge condition.
Typical gauge conditions are:
\begin{definition}[Lorenz gauge]
	The Lorentz gauge,
	\begin{equation}
		\partial_\mu
		A^\mu
		=
		0
		,
	\end{equation}
	where the four-gradient of the Maxwell field is zero.
\end{definition}
The Lorenz gauge is manifestly Lorentz-covariant and calculations in the Lorenz gauge are valid in any reference frame.\footnote{See, for instance, Ref.~\cite[p.~144]{Greiner2013}.}
However, a drawback of the Lorentz gauge is that the quantization procedure becomes more elaborate giving rise to scalar and longitudinal polarization vectors which have to be removed by projecting out unphysical states of the Hilbert space, see Gupta-Beuler quantization in Ref.~\cite[p.~180]{Greiner2013}.
\begin{definition}[Coulomb gauge]
	In the Coulomb gauge, the three-gradient of the Maxwell field is zero, $\partial_iA^i=\div\vb{A}=0$.
\end{definition}
In contrast to the Lorentz gauge, the Coulomb gauge removes unphysical longitudinal and scalar degrees of freedom from the start by making the Maxwell field transverse at the cost of manifest Lorentz-covariance.
\begin{definition}[Temporal gauge]
	In the temporal gauge, the temporal component of the Maxwell field is zero, $A_0=0$.
\end{definition}
The Coulomb gauge leaves a residual gauge freedom which is further removed by the temporal gauge.
The temporal removes Coulomb interactions, i.e., interactions with a static charge distribution~\cite[p.~200]{Greiner2013}.

Quantum optical communication is described in the rest frame of the communication system and it appears natural to renounce manifest Lorentz-covariance for a simplified quantization procedure.
\begin{lemma}
	Choosing the local gauge field $\Lambda$ such that
	\begin{align}
		\laplacian
		\Lambda
		=
		-
		\div\vb{A}
		&&
		\partial_t
		\Lambda
		=
		-
		A_0
	\end{align}
	implements the Coulomb and temporal gauge.
\end{lemma}
\begin{proof}
	The time component of the Maxwell field transforms under the suggested local gauge transformation as
	\begin{equation*}
		A^0
		\to
		A^0
		+
		\partial^0\Lambda
		=
		A^0
		+
		\partial_t\Lambda
		=
		0
	\end{equation*}
	and the spatial components transform as
	\begin{equation*}
		A^i
		\to
		A^i
		+
		\partial^i\Lambda
		.
	\end{equation*}
	Taking the spatial derivative of the former, we see that the Coulomb gauge is satisfied
	\begin{equation*}
		\partial_i A^i
		\to
		\partial_i A^i
		+
		\partial_i \partial^i\Lambda
		=
		\partial_i A^i
		-
		\partial_i A^i
		=
		0
		.
	\end{equation*}
	It is important to note that it is incorrect to deduce
	\begin{equation*}
		\partial_i
		\Lambda
		=
		-
		\partial_i A^i
	\end{equation*}
	as then $A^i=0$.
\end{proof}

\begin{lemma}
	In the Coulomb gauge, the Maxwell field becomes transverse
	\begin{equation}
		\vb{A}
		\to
		P_\perp
		\vb{A}
		=
		\vb{A}_\perp
		.
	\end{equation}
\end{lemma}
\begin{proof}
	We first show that the transverse Maxwell field $\vb{A}_\perp$ satisfies the Coulomb gauge
	\begin{equation*}
		\div\vb{A}
		=
		\partial_i
		\left(
			\delta^{ij}
			-
			\frac{\partial^i\partial^j}{\laplacian}
		\right)
		A_j
		=
		\partial_iA^i
		-
		\partial_jA^j
		=
		0
		.
	\end{equation*}
\end{proof}

\begin{lemma}\label{thm:mw_coulomb_eom}
	The Maxwell field in the Coulomb gauge has the equation of motion becomes
	\begin{equation}
		\vb{J}_\perp
		=
		\partial_\mu \partial^\mu
		\vb{A}_\perp
		=
		\left(
			\partial_t^2
			-
			\laplacian
		\right)
		\vb{A}_\perp
		\label{eq:mw_coulomb_eom}
		.
	\end{equation}
\end{lemma}
\begin{proof}
	Starting from \cref{eq:mw_eom}, we expand the field-strength tensor
	\begin{equation*}
		\begin{split}
			J^\nu
			=
			\partial_\mu
			F^{\mu\nu}
			&=
			\partial_\mu
			\left(
				\partial^\mu
				A^\nu
				-
				\partial^\nu
				A^\mu
			\right)
			\\
			&=
			\partial_\mu
			\partial^\mu
			A^\nu
			-
			\partial^\nu
			\partial_\mu
			A^\mu
			\\
			&=
			\partial_\mu
			\partial^\mu
			A^\nu
			-
			\partial^\nu
			\left(
				\partial_t A^0
				+
				\partial_i A^i
			\right)
		\end{split}
	\end{equation*}
	and apply the Coulomb and temporal gauge conditions while assuming dynamical sources
	\begin{equation*}
		\vb{J}
		=
		\partial_\mu
		\partial^\mu
		\vb{A}
		.
	\end{equation*}
	Because the Maxwell field in the Coulomb and temporal gauge is transverse, the previous equation reduces further to
	\begin{equation*}
		\vb{J}_\perp
		=
		\partial_\mu
		\partial^\mu
		\vb{A}_\perp
		.		
	\end{equation*}
\end{proof}

\begin{theorem}\label{thm:mw_coulomb_mode_expansion}
	The mode expansion of the Maxwell field in the Coulomb gauge is
	\begin{equation}
		\vb{A}_\perp(x^\mu)
		=
		\sum_{\lambda=1,2}
		\int_{\mathbb{R}^3}\frac{\dd[3]{p}}{(2\pi)^3\sqrt{2\omega(\vb{p})}}
		\left\{
			a_\lambda(\vb{p})
			\boldsymbol{\epsilon}_\lambda(\vb{p})
			e^{-ip_\mu x^\mu}
			+
			\text{c.c.}
		\right\}_{p_0=\omega(\vb{p})}
		\label{eq:mw_coulomb_mode_expansion}
	\end{equation}
	with $\omega(\vb{p})=\norm{\vb{p}}$ where the polarization basis vectors satisfy
	\begin{align}
		\vb{p}
		\vdot
		\boldsymbol{\varepsilon}_\lambda(\vb{p})
		=
		0
		&&
		\boldsymbol{\varepsilon}_\alpha(\vb{p})
		\vdot
		\boldsymbol{\varepsilon}_\beta(\vb{p})^*
		=
		\delta_{\alpha\beta}
		\label{eq:mw_coulomb_polarization_basis}
	\end{align}
	and are complete~\cite[p.~341]{Srednicki2007}
	\begin{equation}
		\sum_{\lambda=1,2}
		\boldsymbol{\varepsilon}_\lambda^i(\vb{p})
		\boldsymbol{\varepsilon}_\lambda^j(\vb{p})^*
		=
		\delta^{ij}
		-
		\frac{p^ip^j}{\vb{p}^2}
		=
		P_\perp
		\label{eq:mw_coulomb_polarization_complete}
		.
	\end{equation}
\end{theorem}
\begin{proof}
	Inserting \cref{eq:mw_coulomb_mode_expansion} into the free equations of motion \cref{eq:mw_coulomb_eom}, we find
	\begin{equation*}
		0
		=
		p_\mu p^\mu
		=
		p_0^2
		-
		\vb{p}^2
	\end{equation*}
	which is satisfied if $p_0=\omega(\vb{p})$.
	$\vb{p}\vdot\boldsymbol{\varepsilon}_\lambda(\vb{p})=0$ follows from the Coulomb gauge $\div\vb{A}=0$.
	Orthogonality of the polarization vectors $\boldsymbol{\varepsilon}_1(\vb{p})\perp\boldsymbol{\varepsilon}_2(\vb{p})$ is required to have a complete solution and choosing the polarization vectors to be orthonormal is a convenient choice.
\end{proof}

\begin{theorem}\label{thm:mw_coulomb_mode_expansion_em_field}
	The mode expansion of the electric and magnetic fields reads
	\begin{align}
		\vb{E}(x^\mu)
		&=
		\sum_{\lambda=1,2}
		\int\frac{\dd[3]{p}}{(2\pi)^3\sqrt{2\omega(\vb{p})}}
		\left(
			-i
			\omega(\vb{p})
		\right)
		\left\{
			a_\lambda(\vb{p})
			\boldsymbol{\varepsilon}_\lambda(\vb{p})
			e^{-ip_\mu x^\mu}
			-
			\text{c.c.}
		\right\}_{p_0=\omega(\vb{p})}
		\\
		\vb{B}(x^\mu)
		&=
		\sum_{\lambda=1,2}
		\int\frac{\dd[3]{p}}{(2\pi)^3\sqrt{2\omega(\vb{p})}}
		\left(
			i\vb{p}
		\right)
		\cross
		\left\{
			a_\lambda(\vb{p})
			\boldsymbol{\varepsilon}_\lambda(\vb{p})
			e^{-ip_\mu x^\mu}
			-
			\text{c.c.}
		\right\}_{p_0=\omega(\vb{p})}
		\label{eq:mw_coulomb_mode_expansion_em_field}
		.
	\end{align}
\end{theorem}
\begin{proof}
	The electric field components follow directly from inserting the mode expansion of the field into the definition
	\begin{equation*}
		\begin{split}
			E^i
			&=
			F^{0i}
			=
			\partial_t
			A^i
			\\
			&=
			\sum_{\lambda=1,2}
			\int_{\mathbb{R}^3}\frac{\dd[3]{p}}{(2\pi)^3\sqrt{2\omega(\vb{p})}}
			\partial_t
			\left\{
				a_\lambda(\vb{p})
				\boldsymbol{\epsilon}_\lambda(\vb{p})^i
				e^{-ip_\mu x^\mu}
				+
				\text{c.c.}
			\right\}_{p_0=\omega(\vb{p})}
			\\
			&=
			\sum_{\lambda=1,2}
			\int_{\mathbb{R}^3}\frac{\dd[3]{p}}{(2\pi)^3\sqrt{2\omega(\vb{p})}}
			\left\{
				a_\lambda(\vb{p})
				\boldsymbol{\epsilon}_\lambda(\vb{p})^i
				\partial_t
				e^{-ip_0t}
				e^{+i\vb{p}\vdot\vb{x}}
				+
				\text{c.c.}
			\right\}_{p_0=\omega(\vb{p})}
			\\
			&=
			\sum_{\lambda=1,2}
			\int_{\mathbb{R}^3}\frac{\dd[3]{p}}{(2\pi)^3\sqrt{2\omega(\vb{p})}}
			\left\{
				a_\lambda(\vb{p})
				\boldsymbol{\epsilon}_\lambda(\vb{p})^i
				(-ip_0)
				e^{-ip_0t}
				e^{+i\vb{p}\vdot\vb{x}}
				+
				\text{c.c.}
			\right\}_{p_0=\omega(\vb{p})}
			\\
			&=
			\sum_{\lambda=1,2}
			\int_{\mathbb{R}^3}\frac{\dd[3]{p}}{(2\pi)^3\sqrt{2\omega(\vb{p})}}
			\left(-i\omega(\vb{p})\right)
			\left\{
				a_\lambda(\vb{p})
				\boldsymbol{\epsilon}_\lambda(\vb{p})^i
				e^{-ip_\mu x^\mu}
				-
				\text{c.c.}
			\right\}_{p_0=\omega(\vb{p})}
			.
		\end{split}
	\end{equation*}
	For the magnetic field components, we first need to express the magnetic field components in terms of the field-strength tensor
	\begin{equation*}
		\varepsilon_{ijk}
		F^{jk}
		=
		\varepsilon_{ijk}
		\varepsilon^{jkl}
		B_l
		=
		\varepsilon_{jki}
		\varepsilon^{jkl}
		B_l
		=
		2
		\delta_i^l
		B_l
		=
		2B_i
	\end{equation*}
	and insert the mode expansion of the field
	\begin{equation*}
		\begin{split}
			B_i
			&=
			\frac{1}{2}
			\varepsilon_{ijk}
			F^{jk}
			=
			\frac{1}{2}
			\varepsilon_{ijk}
			\left(
				\partial^j
				A^k
				-
				\partial^k
				A^j
			\right)
			\\
			&=
			\frac{1}{2}
			\varepsilon_{ijk}
			\partial^j
			A^k
			+
			\frac{1}{2}
			\varepsilon_{jik}
			\partial^k
			A^j
			=
			\varepsilon_{ijk}
			\partial^j
			A^k
			\\
			&=
			\sum_{\lambda=1,2}
			\int_{\mathbb{R}^3}\frac{\dd[3]{p}}{(2\pi)^3\sqrt{2\omega(\vb{p})}}
			\varepsilon_{ijk}
			\partial^j
			\left\{
				a_\lambda(\vb{p})
				\boldsymbol{\epsilon}_\lambda(\vb{p})^i
				e^{-ip_\mu x^\mu}
				+
				\text{c.c.}
			\right\}_{p_0=\omega(\vb{p})}
			\\
			&=
			\sum_{\lambda=1,2}
			\int_{\mathbb{R}^3}\frac{\dd[3]{p}}{(2\pi)^3\sqrt{2\omega(\vb{p})}}
			\varepsilon_{ijk}
			\left\{
				a_\lambda(\vb{p})
				\boldsymbol{\epsilon}_\lambda(\vb{p})^i
				\partial^j
				e^{-ip_0t}
				e^{+i\vb{p}\vdot\vb{x}}
				+
				\text{c.c.}
			\right\}_{p_0=\omega(\vb{p})}
			\\
			&=
			\sum_{\lambda=1,2}
			\int_{\mathbb{R}^3}\frac{\dd[3]{p}}{(2\pi)^3\sqrt{2\omega(\vb{p})}}
			\varepsilon_{ijk}
			\left\{
				a_\lambda(\vb{p})
				\boldsymbol{\epsilon}_\lambda(\vb{p})^i
				(+ip^j)
				e^{-ip_0t}
				e^{+i\vb{p}\vdot\vb{x}}
				+
				\text{c.c.}
			\right\}_{p_0=\omega(\vb{p})}
			\\
			&=
			\sum_{\lambda=1,2}
			\int_{\mathbb{R}^3}\frac{\dd[3]{p}}{(2\pi)^3\sqrt{2\omega(\vb{p})}}
			\left(
				i
				\varepsilon_{ijk}
				p^j
			\right)
			\left\{
				a_\lambda(\vb{p})
				\boldsymbol{\epsilon}_\lambda(\vb{p})^i
				e^{-ip_\mu x^\mu}
				-
				\text{c.c.}
			\right\}_{p_0=\omega(\vb{p})}
			\\
			&=
			\sum_{\lambda=1,2}
			\int_{\mathbb{R}^3}\frac{\dd[3]{p}}{(2\pi)^3\sqrt{2\omega(\vb{p})}}
			\left(
				-i
				\varepsilon_{ijk}
				p^i
			\right)
			\left\{
				a_\lambda(\vb{p})
				\boldsymbol{\epsilon}_\lambda(\vb{p})^j
				e^{-ip_\mu x^\mu}
				-
				\text{c.c.}
			\right\}_{p_0=\omega(\vb{p})}
			.
		\end{split}
	\end{equation*}
	Raising the index of the magnetic field with the metric yields
	\begin{equation*}
		B^i
		=
		\eta^{ij}
		B_j
		=
		-
		\delta^{ij}
		B_j
		=
		-
		B_i
	\end{equation*}
	and in vector notation we recover \cref{eq:mw_coulomb_mode_expansion_em_field} which up to a sign is in agreement with Ref.~\cite[p.~198]{Greiner2013}.
\end{proof}

\begin{lemma}\label{thm:mw_coulomb_canonical_momentum}
	The canonical momentum density
	\begin{equation}
		\pi^i(x^\mu)
		=
		\partial_t
		\pdv{\mathcal{L}}{(\partial_tA_i)}
		=
		-
		E^i(x^\mu)
		\label{eq:mw_coulomb_canonical_momentum}
	\end{equation}
	is equal to the electric field.
\end{lemma}
\begin{proof}
	In \Cref{thm:mw_coulomb_eom} we found
	\begin{equation*}
		\pdv{\mathcal{L}}{(\partial_\mu A_\nu)}
		=
		-
		F^{\mu\nu}
	\end{equation*}
	thus
	\begin{equation*}
		\partial_t
		\pdv{\mathcal{L}}{(\partial_tA_i)}
		=
		-
		\partial_t
		F^{0i}
		=
		-
		\partial_t
		E^i
		.
	\end{equation*}
\end{proof}

\begin{corollary}\label{thm:mw_coulomb_hamiltonian}
	The Hamiltonian density is
	\begin{equation}
		\mathcal{H}
		=
		\frac{1}{2}
		\boldsymbol{\pi}^2
		-
		\mathcal{L}
		=
		\frac{1}{2}
		\left(
			\vb{E}^2
			+
			\vb{B}^2
		\right)
		\label{thm:mw_hamiltonian}
		.
	\end{equation}
\end{corollary}

\begin{lemma}\label{thm:mw_coulomb_energy}
	The total energy is
	\begin{equation}
		H
		=
		\sum_{\lambda=1,2}
		\int\frac{\dd[3]{p}}{(2\pi)^3}
		\omega(\vb{p})
		\abs{a_\lambda(\vb{p})}^2
		\label{thm:mw_coulomb_energy}
		.
	\end{equation}
\end{lemma}
\begin{proof}
	Integration of the Hamiltonian density given in \Cref{thm:mw_hamiltonian} yields the total energy in terms of the electric and magnetic field amplitudes
	\begin{equation*}
		H
		=
		\int\dd[3]{x}
		\mathcal{H}
		=
		\frac{1}{2}
		\int\dd[3]{x}
		\left\{
			\vb{E}(x^\mu)^2
			+
			\vb{B}(x^\mu)^2
		\right\}
		.
	\end{equation*}
	Inserting the mode decomposition of \Cref{thm:mw_coulomb_mode_expansion_em_field} for the magnetic field, we evaluate
	\begin{equation*}
		\begin{split}
			\int\dd[3]{x}
			\vb{E}(x^\mu)^2
			&=
			\sum_{\lambda,\lambda^\prime=1,2}
			\int\frac{\dd[3]{p}}{(2\pi)^3\sqrt{2\omega(\vb{p})}}
			\int\frac{\dd[3]{q}}{(2\pi)^3\sqrt{2\omega(\vb{q})}}
			\left(
				-
				\omega(\vb{p})
				\omega(\vb{q})
			\right)
			\\
			&\times
			\biggl\{
				a_\lambda(\vb{p})
				a_{\lambda^\prime}(\vb{q})
				\boldsymbol{\varepsilon}_\lambda(\vb{p})
				\vdot
				\boldsymbol{\varepsilon}_{\lambda^\prime}(\vb{q})
				\int\dd[3]{x}
				e^{-i(p_\mu+q_\mu)x^\mu}
				\\
				& \ \ \
				-
				a_\lambda(\vb{p})
				a_{\lambda^\prime}(\vb{q})^*
				\boldsymbol{\varepsilon}_\lambda(\vb{p})
				\vdot
				\boldsymbol{\varepsilon}_{\lambda^\prime}(\vb{q})^*
				\int\dd[3]{x}
				e^{-i(p_\mu-q_\mu)x^\mu}
				+
				\text{c.c.}
			\biggr\}_{\substack{p_0=\omega(\vb{p})\\q_0=\omega(\vb{q})}}
		\end{split}
	\end{equation*}
	where we find
	\begin{equation*}
		\begin{split}
			\int\dd[3]{x}
			\eval{e^{-i(p_\mu\mp q_\mu)x^\mu}}_{\substack{p_0=\omega(\vb{p})\\q_0=\omega(\vb{q})}}
			&=
			e^{-i\omega(\vb{p})t}
			e^{\pm i\omega(\vb{q})t}
			\int\dd[3]{x}
			e^{+i(\vb{p}\mp\vb{q})\vdot\vb{x}}
			\\
			&=
			e^{-i\omega(\vb{p})t}
			e^{\pm i\omega(\vb{q})t}
			(2\pi)^3\delta^{(3)}(\vb{q}\mp\vb{p})
		\end{split}
	\end{equation*}
	which simplifies the first term in the curly brackets to
	\begin{equation*}
		\begin{split}
			&\
			a_\lambda(\vb{p})
			a_{\lambda^\prime}(\vb{q})
			\boldsymbol{\varepsilon}_\lambda(\vb{p})
			\vdot
			\boldsymbol{\varepsilon}_{\lambda^\prime}(\vb{q})
			e^{-i\omega(\vb{p})t}
			e^{-i\omega(\vb{q})t}
			(2\pi)^3\delta^{(3)}(\vb{q}+\vb{p})
			\\
			=&\
			a_\lambda(\vb{p})
			a_{\lambda^\prime}(-\vb{p})
			\boldsymbol{\varepsilon}_\lambda(\vb{p})
			\vdot
			\boldsymbol{\varepsilon}_{\lambda^\prime}(-\vb{p})
			e^{-i\omega(\vb{p})t}
			e^{-i\omega(-\vb{p})t}
			(2\pi)^3\delta^{(3)}(\vb{q}+\vb{p})
			\\
			=&\
			a_\lambda(\vb{p})
			a_{\lambda^\prime}(\vb{p})^*
			\boldsymbol{\varepsilon}_\lambda(\vb{p})
			\vdot
			\boldsymbol{\varepsilon}_{\lambda^\prime}(\vb{p})^*
			e^{-2i\omega(\vb{p})t}
			(2\pi)^3\delta^{(3)}(\vb{q}+\vb{p})
			\\
			=&\
			a_\lambda(\vb{p})
			a_{\lambda^\prime}(\vb{p})^*
			\delta_{\lambda\lambda^\prime}
			e^{-2i\omega(\vb{p})t}
			(2\pi)^3\delta^{(3)}(\vb{q}+\vb{p})
			\\
			=&\
			\abs{a_\lambda(\vb{p})}^2
			\delta_{\lambda\lambda^\prime}
			e^{-2i\omega(\vb{p})t}
			(2\pi)^3\delta^{(3)}(\vb{q}+\vb{p})
		\end{split}
	\end{equation*}
	and the second term to
	\begin{equation*}
		\begin{split}
			&\
			a_\lambda(\vb{p})
			a_{\lambda^\prime}(\vb{q})^*
			\boldsymbol{\varepsilon}_\lambda(\vb{p})
			\vdot
			\boldsymbol{\varepsilon}_{\lambda^\prime}(\vb{q})^*
			e^{-i\omega(\vb{p})t}
			e^{+i\omega(\vb{q})t}
			(2\pi)^3\delta^{(3)}(\vb{q}-\vb{p})
			\\
			=&\
			a_\lambda(\vb{p})
			a_{\lambda^\prime}(\vb{p})^*
			\boldsymbol{\varepsilon}_\lambda(\vb{p})
			\vdot
			\boldsymbol{\varepsilon}_{\lambda^\prime}(\vb{p})^*
			e^{-i\omega(\vb{p})t}
			e^{+i\omega(\vb{p})t}
			(2\pi)^3\delta^{(3)}(\vb{q}-\vb{p})
			\\
			=&\
			a_\lambda(\vb{p})
			a_{\lambda^\prime}(\vb{p})^*
			\delta_{\lambda\lambda^\prime}
			(2\pi)^3\delta^{(3)}(\vb{q}-\vb{p})
			\\
			=&\
			\abs{a_\lambda(\vb{p})}^2
			\delta_{\lambda\lambda^\prime}
			(2\pi)^3\delta^{(3)}(\vb{q}-\vb{p})
			.
		\end{split}
	\end{equation*}
	Inserting both terms back, we find
	\begin{equation*}
		\begin{split}
			\int\dd[3]{x}
			\vb{E}(x^\mu)^2
			&=
			\sum_{\lambda,\lambda^\prime=1,2}
			\int\frac{\dd[3]{p}}{(2\pi)^3\sqrt{2\omega(\vb{p})}}
			\int\frac{\dd[3]{q}}{(2\pi)^3\sqrt{2\omega(\vb{q})}}
			\left(
				-
				\omega(\vb{p})
				\omega(\vb{q})
			\right)
			\\
			&\times
			\left\{
				\abs{a_\lambda(\vb{p})}^2
				\delta_{\lambda\lambda^\prime}
				e^{-2i\omega(\vb{p})t}
				(2\pi)^3\delta^{(3)}(\vb{q}+\vb{p})
				-
				\abs{a_\lambda(\vb{p})}^2
				\delta_{\lambda\lambda^\prime}
				(2\pi)^3\delta^{(3)}(\vb{q}-\vb{p})
				+
				\text{c.c.}
			\right\}
			\\
			&=
			\sum_{\lambda=1,2}
			\int\frac{\dd[3]{p}}{(2\pi)^32\omega(\vb{p})}
			\left(
				-
				\omega(\vb{p})^2
			\right)
			\abs{a_\lambda(\vb{p})}^2
			\left\{
				e^{-2i\omega(\vb{p})t}
				-
				1
				+
				e^{+2i\omega(\vb{p})t}
				-
				1
			\right\}
			\\
			&=
			2
			\sum_{\lambda=1,2}
			\int\frac{\dd[3]{p}}{(2\pi)^32\omega(\vb{p})}
			\omega(\vb{p})^2
			\abs{a_\lambda(\vb{p})}^2
			\Re\left\{
				1
				-
				e^{-2i\omega(\vb{p})t}
			\right\}
			.
		\end{split}
	\end{equation*}
	For the magnetic field, we need to account for an additional sign flip of the momentum when evaluating the delta distribution, otherwise the steps are analog and we find
	\begin{equation*}
		\begin{split}
			\int\dd[3]{x}
			\vb{B}(x^\mu)^2
			&=
			2
			\sum_{\lambda,\lambda^\prime=1,2}
			\int\frac{\dd[3]{p}}{(2\pi)^3\sqrt{2\omega(\vb{p})}}
			\int\frac{\dd[3]{q}}{(2\pi)^3\sqrt{2\omega(\vb{q})}}
			(-1)
			\int\dd[3]{x}
			\\
			&\times
			\left\{
				a_\lambda(\vb{p})
				\left(
					\vb{p}
					\cp
					\boldsymbol{\varepsilon}_\lambda(\vb{p})
				\right)
				e^{-ip_\mu x^\mu}
				-
				\text{c.c.}
			\right\}_{p_0=\omega(\vb{p})}
			\\
			&\vdot
			\left\{
				a_{\lambda^\prime}(\vb{q})
				\left(
					\vb{q}
					\cp
					\boldsymbol{\varepsilon}_{\lambda^\prime}(\vb{q})
				\right)
				e^{-iq_\mu x^\mu}
				-
				\text{c.c.}
			\right\}_{q_0=\omega(\vb{q})}
		\end{split}
	\end{equation*}
	and the inner product evaluates to
	\begin{equation*}
		\begin{split}
			&\
			a_\lambda(\vb{p})
			a_{\lambda^\prime}(\vb{q})
			\left(
				\vb{p}
				\cp
				\boldsymbol{\varepsilon}_\lambda(\vb{p})
			\right)
			\vdot
			\left(
				\vb{q}
				\cp
				\boldsymbol{\varepsilon}_{\lambda^\prime}(\vb{q})
			\right)
			\int\dd[3]{x}
			\eval{e^{-i(p_\mu+q_\mu)x^\mu}}_{\substack{p_0=\omega(\vb{p})\\q_0=\omega(\vb{q})}}
			\\
			-&\
			a_\lambda(\vb{p})
			a_{\lambda^\prime}(\vb{q})^*
			\left(
				\vb{p}
				\cp
				\boldsymbol{\varepsilon}_\lambda(\vb{p})
			\right)
			\vdot
			\left(
				\vb{q}
				\cp
				\boldsymbol{\varepsilon}_{\lambda^\prime}(\vb{q})^*
			\right)
			\int\dd[3]{x}
			\eval{e^{-i(p_\mu-q_\mu)x^\mu}}_{\substack{p_0=\omega(\vb{p})\\q_0=\omega(\vb{q})}}
			+
			\text{c.c.}
			\\
			=&\
			a_\lambda(\vb{p})
			a_{\lambda^\prime}(\vb{q})
			\left(
				\vb{p}
				\cp
				\boldsymbol{\varepsilon}_\lambda(\vb{p})
			\right)
			\vdot
			\left(
				\vb{q}
				\cp
				\boldsymbol{\varepsilon}_{\lambda^\prime}(\vb{q})
			\right)
			e^{-i\omega(\vb{p})t}
			e^{-i\omega(\vb{q})t}
			(2\pi)^3\delta^{(3)}(\vb{q}+\vb{p})
			\\
			-&\
			a_\lambda(\vb{p})
			a_{\lambda^\prime}(\vb{q})^*
			\left(
				\vb{p}
				\cp
				\boldsymbol{\varepsilon}_\lambda(\vb{p})
			\right)
			\vdot
			\left(
				\vb{q}
				\cp
				\boldsymbol{\varepsilon}_{\lambda^\prime}(\vb{q})^*
			\right)
			e^{-i\omega(\vb{p})}
			e^{+i\omega(\vb{q})}
			(2\pi)^3\delta^{(3)}(\vb{q}-\vb{p})
			+
			\text{c.c.}
			\\
			=&\
			a_\lambda(\vb{p})
			a_{\lambda^\prime}(\vb{p})^*
			\left(
				\vb{p}
				\cp
				\boldsymbol{\varepsilon}_\lambda(\vb{p})
			\right)
			\vdot
			\left(
				-
				\vb{p}
				\cp
				\boldsymbol{\varepsilon}_{\lambda^\prime}(\vb{p})^*
			\right)
			e^{-2i\omega(\vb{p})t}
			(2\pi)^3\delta^{(3)}(\vb{q}+\vb{p})
			\\
			-&\
			a_\lambda(\vb{p})
			a_{\lambda^\prime}(\vb{q})^*
			\left(
				\vb{p}
				\cp
				\boldsymbol{\varepsilon}_\lambda(\vb{p})
			\right)
			\vdot
			\left(
				\vb{p}
				\cp
				\boldsymbol{\varepsilon}_{\lambda^\prime}(\vb{p})^*
			\right)
			(2\pi)^3\delta^{(3)}(\vb{q}-\vb{p})
			+
			\text{c.c.}
		\end{split}
	\end{equation*}
	and noting that
	\begin{equation*}
		\left(
			\vb{p}
			\cp
			\boldsymbol{\varepsilon}_\lambda(\vb{p})
		\right)
		\vdot
		\left(
			\vb{p}
			\cp
			\boldsymbol{\varepsilon}_{\lambda^\prime}(\vb{p})^*
		\right)
		=
		\left(
			\vb{p}^2
		\right)
		\left(
			\boldsymbol{\varepsilon}_\lambda(\vb{p})
			\vdot
			\boldsymbol{\varepsilon}_{\lambda^\prime}(\vb{p})^*
		\right)
		-
		\left(
			\vb{p}
			\vdot
			\boldsymbol{\varepsilon}_\lambda(\vb{p})
		\right)
		\left(
			\vb{p}
			\vdot
			\boldsymbol{\varepsilon}_{\lambda^\prime}(\vb{p})^*
		\right)
		=
		\vb{p}^2
		\delta_{\lambda\lambda^\prime}
	\end{equation*}
	we find
	\begin{equation*}
		\begin{split}
			\int\dd[3]{x}
			\vb{B}(x^\mu)^2
			&=
			\sum_{\lambda,\lambda^\prime=1,2}
			\int\frac{\dd[3]{p}}{(2\pi)^3\sqrt{2\omega(\vb{p})}}
			\int\frac{\dd[3]{q}}{(2\pi)^3\sqrt{2\omega(\vb{q})}}
			(-1)
			\\
			&\times
			\left\{
				-
				\abs{a_\lambda(\vb{p})}^2
				\vb{p}^2
				\delta_{\lambda\lambda^\prime}
				e^{-2i\omega(\vb{p})t}
				(2\pi)^3\delta^{(3)}(\vb{q}+\vb{p})
				-
				\abs{a_\lambda(\vb{p})}^2
				\vb{p}^2
				\delta_{\lambda\lambda^\prime}
				(2\pi)^3\delta^{(3)}(\vb{q}-\vb{p})
				+
				\text{c.c.}
			\right\}
			\\
			&=
			\sum_{\lambda=1,2}
			\int\frac{\dd[3]{p}}{(2\pi)^32\omega(\vb{p})}
			\abs{a_\lambda(\vb{p})}^2
			\vb{p}^2
			\left\{
				e^{-2i\omega(\vb{p})t}
				+
				1
				+
				\text{c.c.}
			\right\}
			\\
			&=
			2
			\sum_{\lambda=1,2}
			\int\frac{\dd[3]{p}}{(2\pi)^32\omega(\vb{p})}
			\abs{a_\lambda(\vb{p})}^2
			\vb{p}^2
			\Re\left\{
				1
				+
				e^{-2i\omega(\vb{p})t}
			\right\}
		\end{split}
		.
	\end{equation*}
	Adding the integral over the squared field amplitudes together and dividing by two gives the total energy
	\begin{equation*}
		\begin{split}
			H
			&=
			\frac{1}{2}
			\int\dd[3]{x}
			\vb{E}(x^\mu)^2
			+
			\frac{1}{2}
			\int\dd[3]{x}
			\vb{B}(x^\mu)^2
			\\
			&=
			2\sum_{\lambda=1,2}
			\int\frac{\dd[3]{p}}{(2\pi)^32\omega(\vb{p})}
			\omega(\vb{p})^2
			\abs{a_\lambda(\vb{p})}^2
			\\
			&=
			\int\frac{\dd[3]{p}}{(2\pi)^3}
			\omega(\vb{p})
			\abs{a_\lambda(\vb{p})}^2
		\end{split}
	\end{equation*}
	where we used $\vb{p}^2=\omega(\vb{p})^2$.
\end{proof}

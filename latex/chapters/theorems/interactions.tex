\section{Interactions}

\subsection{Dynamical pictures}

\begin{definition}
	In the Heisenberg picture, the operators satisfy the Heisenberg equation of motion
	\begin{equation}
		i\partial_t
		\hat{O}^H(t)
		=
		\comm{\hat{O}^H(t)}{\hat{H}}
	\end{equation}
	and the states are constant, i.e.,
	\begin{equation}
		\ket{\psi(t)}^H
		=
		\ket{\psi(0)}^H
		=
		\ket{\psi}^H
		.
	\end{equation}
\end{definition}

\begin{lemma}\label{thm:heisenberg_eom_sol}
	The Heisenberg equation of motion is solved by
	\begin{equation}
		\hat{O}^H(t)
		=
		e^{+i\hat{H}t}
		\hat{O}^H(0)
		e^{-i\hat{H}t}
		.
	\end{equation}
\end{lemma}
\begin{definition}
	In the Schrödinger picture, the operators are time-independent, and the states satisfy the Schrödinger equation
	\begin{equation}
		i\partial_t
		\ket{\psi(t)}^S
		=
		\hat{H}
		\ket{\psi(t)}^S
	\end{equation}
\end{definition}
\begin{lemma}\label{thm:schroedinger_eom_sol}
	For a time-independent Hamiltonian, the state solving the Schrödinger equation is
	\begin{equation}
		\ket{\psi(t)}^S
		=
		e^{-i\hat{H}t}
		\ket{\psi(0)}^S
		.
	\end{equation}
\end{lemma}
\begin{theorem}\label{thm:heisenberg_schroedinger_equivalence}
	Operators and states in the Heisenberg and Schrödinger picture are related by
	\begin{align}
		\hat{O}^S
		&=
		\hat{O}^H(0)
		\\
		\ket{\psi(0)}^S
		&=
		\ket{\psi}^H
		,
	\end{align}
	and are equivalent with respect to their physical predictions.
\end{theorem}
\begin{proof}
	Heisenberg and Schrödinger picture are unitarily equivalent by the Stone-von Neumann theorem.
	Let $\hat{O}^H(t)$ be an observable and $\ket{\psi}^H$ a state in the Heisenberg picture, then the expectation values agree
	\begin{equation*}
		\bra{\psi}^H
		\hat{O}^H(t)
		\ket{\psi}^H
		=
		\bra{\psi(0)}^H
		e^{+i\hat{H}t}
		\hat{O}^H(0)
		e^{-i\hat{H}t}
		\ket{\psi(0)}^H
		=
		\bra{\psi(t)}^S
		\hat{O}^S
		\ket{\psi(t)}^S
	\end{equation*}
	where we used \Cref{thm:heisenberg_eom_sol} and \Cref{thm:schroedinger_eom_sol}.
\end{proof}

\begin{definition}
	A canonical transformation leaves the commutation relations unchanged.	
\end{definition}
\begin{lemma}\label{thm:heisenberg_schroedinger_canonical_transformation}
	Heisenberg and Schrödinger picture are related by a canonical transformation.
\end{lemma}
\begin{proof}
	Let $\hat{A}^H,\hat{B}^H,\hat{C}^H$ be operators in the Heisenberg picture satisfying the commutation relation
	\begin{equation*}
		\comm{\hat{A}^H}{\hat{B}^H}
		=
		\hat{C}^H
	\end{equation*}
	then in the Schrödinger picture, we find
	\begin{equation*}
		\begin{split}
			\comm{\hat{A}^S}{\hat{B}^S}
			&=
			e^{-i\hat{H}t}
			\left\{
				\hat{A}^H
				e^{+i\hat{H}t}
				e^{-i\hat{H}t}
				\hat{B}^H
				-	
				\hat{B}^H
				e^{-i\hat{H}t}
				e^{+i\hat{H}t}
				\hat{A}^H
			\right\}
			e^{+i\hat{H}t}
			\\
			&=
			e^{-i\hat{H}t}
			\comm{\hat{A}^H}{\hat{B}^H}
			e^{+i\hat{H}t}
			\\
			&=
			e^{-i\hat{H}t}
			\hat{C}^H
			e^{+i\hat{H}t}
			=
			\hat{C}^S
		\end{split}
	\end{equation*}
	from Ref.~\cite[p.~213]{Greiner2013}.
\end{proof}

\begin{definition}
	If we can write the Hamiltonian $\hat{H}$ as the sum of a free $\hat{H}_0$ and an interacting part $\hat{H}_\text{int}$,
	\begin{equation}
		\hat{H}
		=
		\hat{H}_0
		+
		\hat{H}_\text{int}
		,
	\end{equation}
	the Dirac (or interaction) picture is defined as
	\begin{align}
		\hat{O}^D(t)
		&=
		e^{+i\hat{H}_0^St}
		\hat{O}^S
		e^{-i\hat{H}_0^St}
		\\
		\ket{\psi(t)}^D
		&=
		e^{+i\hat{H}_0^St}
		\ket{\psi(t)}^S
	\end{align}
\end{definition}
\begin{remark}
	For $\comm{\hat{H}_0^S}{\hat{H}_\text{int}^S}=0$, we have
	\begin{align}
		\hat{O}^D(t)
		&=
		e^{+i\hat{H}_0^St}
		\hat{O}^S
		e^{-i\hat{H}_0^St}
		\\
		\ket{\psi(t)}^D
		&=
		e^{-i\hat{H}_\text{int}^St}
		\ket{\psi(0)}^S
		,
	\end{align}
	i.e., the state evolves according to the Schrödinger picture with the interaction Hamiltonian $\hat{H}_\text{int}$ while the operators evolve according to the Heisenberg picture with the free Hamiltonian $\hat{H}_0$.
\end{remark}
\begin{corollary}
	For $\hat{H}_\text{int}=0$, the Dirac picture reduces to the Heisenberg picture.
\end{corollary}
\begin{corollary}
	At $t=0$, operators and states in Heisenberg, Schrödinger, and Dirac picture coincide.
\end{corollary}
\begin{lemma}
	The Dirac picture relates to the Heisenberg picture by
	\begin{align}
		\hat{O}^D(t)
		&=
		e^{+i\hat{H}_0^St}
		e^{-i\hat{H}t}
		\hat{O}^H(t)
		e^{+i\hat{H}t}
		e^{-i\hat{H}_0^St}
		\\
		\ket{\psi(t)}^D
		&=
		e^{+i\hat{H}_0^St}
		e^{-i\hat{H}t}
		\ket{\psi(t)}^S
		.
	\end{align}
\end{lemma}

%\begin{delayedproof}{thm:dirac_schroedinger_eom}
%	See Ref.~\cite[p.~214]{Greiner2013}.
%\end{delayedproof}

\subsection{Time-evolution operator}

\begin{proof}
%{thm:time_evolution_int}
	We integrate \cref{eq:time_evolution_diff} over the interval $[t_0,t]$
	\begin{equation*}
		\int_t^{t_0}\dd{t_1}
		\hat{H}_\text{int}^D(t_1,t_0)
		\hat{U}_D(t_1,t_0)
		=
		i\int_t^{t_0}\dd{t_1}
		\partial_{t_1}
		\hat{U}_D(t_1,t_0)
		=
		i\mathbb{I}
		-
		i\hat{U}_D(t,t_0)
	\end{equation*}
	where we used the fundamental theorem of calculus and the boundary condition $\hat{U}^D(t_0,t_0)=\mathbb{I}$ in the third line.
	We are left to rearrange the terms to find \cref{eq:time_evolution_int}.
\end{proof}
\begin{proof}
%{thm:time_evolution_exp_sol}
	Use \Cref{thm:time_ordered_integral} with $\hat{A}=\hat{H}_\text{int}^D$ to rewrite the iterative integral solution \Cref{thm:time_evolution_iter_sol}.
	A proof that \cref{eq:time_evolution_sol} satisfies \cref{eq:time_evolution_diff} can be found in Ref.~\cite[p.~219]{Greiner2013}.
\end{proof}

\begin{lemma}\label{thm:dirac_schroedinger_eom}
	The state in the Dirac picture evolves according to the Schrödinger equation
	\begin{equation}
		i\partial_t
		\ket{\psi(t)}^D
		=
		\hat{H}_\text{int}^D
		\ket{\psi(t)}^D
		\label{eq:dirac_schroedinger_eom}
	\end{equation}
	with the interaction Hamiltonian $\hat{H}_\text{int}^D=e^{+i\hat{H}_0^St}\hat{H}_\text{int}^Se^{-i\hat{H}_0^St}$ in the Dirac picture.
\end{lemma}
\begin{theorem}
	The solution to the Schrödinger equation in the Dirac picture, \cref{eq:dirac_schroedinger_eom}, is
	\begin{equation}
		\ket{\psi(t_1)}^D
		=
		e^{+i\hat{H}_0^St_1}
		e^{-i\hat{H}^S(t_1-t_0)}
		e^{-i\hat{H}_0^St_0}
		\ket{\psi(t_0)}^D
	\end{equation}
	.
\end{theorem}
\begin{definition}
	\begin{equation}
		\hat{U}^D(t_1,t_0)
		=
		e^{+i\hat{H}_0^St_1}
		e^{-i\hat{H}^S(t_1-t_0)}
		e^{-i\hat{H}_0^St_0}		
	\end{equation}
	is the time-evolution operator.
\end{definition}
\begin{lemma}
	The time-evolution operator $\hat{U}^D(t_1,t_0)$ satisfies
	\begin{enumerate}
		\item $\hat{U}^D(t_0,t_0)=\mathbb{I}$
		\item $\hat{U}^D(t_2,t_1)\hat{U}^D(t_1,t_0)=\hat{U}^D(t_2,t_0)$
		\item $\hat{U}^D(t_0,t_1)^{-1}=\hat{U}^D(t_1,t_0)$
		\item $\hat{U}^D(t_0,t_1)^\dagger=\hat{U}^D(t_1,t_0)^{-1}$
	\end{enumerate}
\end{lemma}
\begin{corollary}
	The time-evolution operator in the Dirac picture $\hat{U}_D$ satisfies
	\begin{equation}
		i\partial_t
		\hat{U}_D(t,t_0)
		=
		\hat{H}^D_\text{int}
		\hat{U}_D(t,t_0)
		\label{eq:time_evolution_diff}
	\end{equation}
	with the boundary condition $\hat{U}^D(t_0,t_0)=\mathbb{I}$.
\end{corollary}
\begin{lemma}\label{thm:time_evolution_int}
	The integral equation equivalent to \cref{eq:time_evolution_diff} is
	\begin{equation}
		\hat{U}_D(t,t_0)
		=
		\mathbb{I}
		+
		(-i)
		\int_{t_0}^t\dd{t^\prime}
		\hat{H}_\text{int}^D
		\hat{U}_D(t^\prime,t_0)
		\label{eq:time_evolution_int}
	\end{equation}
\end{lemma}
\begin{lemma}\label{thm:time_evolution_iter_sol}
	The iterative solution to the integral equation \cref{eq:time_evolution_int} is
	\begin{equation}
		\begin{split}
			\hat{U}_D(t,t_0)
			=
			\mathbb{I}
			&+
			(-i)
			\int_{t_0}^t\dd{t_1}
			\hat{H}_\text{int}^D(t_1)
			\\
			&+
			(-i)^2
			\int_{t_0}^t\dd{t_1}
			\int_{t_0}^{t_1}\dd{t_2}
			\hat{H}_\text{int}^D(t_1)
			\hat{H}_\text{int}^D(t_2)
			+\dots
			\\
			&+
			(-i)^n
			\int_{t_0}^t\dd{t_1}
			\dots
			\int_{t_0}^{t_{n-1}}\dd{t_n}
			\hat{H}_\text{int}^D(t_1)
			\hat{H}_\text{int}^D(t_2)
			\dots
			\hat{H}_\text{int}^D(t_n)
			+\dots
		\end{split}
	\end{equation}
	known as the Neumann series.
\end{lemma}
\begin{definition}
	Let $\hat{A}_1(t_1),\dots,\hat{A}_n(t_n)$ be time-dependent operators, then their time-ordered product is defined to be
	\begin{equation}
		T\left\{
			\hat{A}_1(t_1)
			\dots
			\hat{A}_n(t_n)
		\right\}
		=
		\hat{A}_1(t_{i_1})
		\dots
		\hat{A}_n(t_{i_n})
	\end{equation}
	where $t_{i_1}\geq t_{i_2}\geq\dots\geq t_{i_n}$.
\end{definition}
\begin{lemma}\label{thm:time_ordered_integral}
	\begin{equation}
		n!
		\int_{t_0}^t\dd{t_1}
		\dots
		\int_{t_0}^{t_{n-1}}\dd{t_n}
		\hat{A}(t_1)
		\dots
		\hat{A}(t_n)
		=
		\int_{t_0}^t\dd{t_1}
		\dots
		\int_{t_0}^t\dd{t_n}
		T\left\{
			\hat{A}_1(t_1)
			\dots
			\hat{A}_n(t_n)
		\right\}
	\end{equation}
\end{lemma}
\begin{proof}
	For a graphical proof, see Ref.~\cite[p.~218]{Greiner2013} and Ref.~\cite[p.~85]{Peskin1995}.
\end{proof}
\begin{theorem}\label{thm:time_evolution_exp_sol}
	The time-evolution operator in the Dyson picture solving \cref{eq:time_evolution_diff} is given by
	\begin{equation}
		\begin{split}
			\hat{U}(t,t_0)
			&=
			\sum_{n=0}^\infty
			\frac{(-i)^n}{n!}
			\int_{t_0}^t\dd{t_1}
			\dots
			\int_{t_0}^t\dd{t_n}
			T\left\{
				\hat{H}^D_\text{int}(t_1)
				\dots
				\hat{H}^D_\text{int}(t_n)
			\right\}
			\\
			&=
			T\exp\left\{
				-i
				\int_{t_0}^t
				\dd{t^\prime}
				\hat{H}^D_\text{int}(t^\prime)
			\right\}
			\\
			&=
			T\exp\left\{
				i
				\int_{t_0}^t
				\dd[4]{x}
				\mathcal{L}_\text{int}(x)
			\right\}			
		\end{split}
		\label{eq:time_evolution_sol}
	\end{equation}
	which is known as the Dyson series and time-ordered exponential.
\end{theorem}

\subsection{Scattering operator}

\begin{definition}
	Let $\hat{H}_0$ be the free and $\hat{H}_\text{int}$ the interaction Hamiltonian, then adiabatic switching refers to a Hamiltonian $\hat{H}_\varepsilon(t)$ which in the adiabatic limit $t\to\pm\infty$ equals the free Hamiltonian but at $t=0$ resembles the full Hamiltonian, i.e.,
	\begin{align}
		\hat{H}_\varepsilon(0)
		=
		\hat{H}
		&&
		\lim_{t\to\pm\infty}
		\hat{H}_\varepsilon(t)
		=
		\hat{H}_0
		.
	\end{align}
\end{definition}
\begin{definition}\label{thm:scattering_operator}
	The scattering operator is defined as the asymptotic limit of the time-evolution operator in the Dirac picture, i.e.,
	\begin{equation}
		\hat{S}
		=
		\lim_{t\to\infty}
		\hat{U}^D(t,-t)
		=
		T\exp\left\{
			i\int\dd[4]{x}
			\mathcal{L}_\text{int}
		\right\}
	\end{equation}
	where $\mathcal{L}_\text{int}$ is the Lagrangian density describing the interaction.
\end{definition}

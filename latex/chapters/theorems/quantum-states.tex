\section{Number states}

\begin{definition}[Number state]\label{def:qkg_number_state}
	The $n$-particle number state with spectrum $f(\vb{p})$ is
	\begin{equation}
		\ket{n_f}
		=
		\frac{1}{\sqrt{n!}}
		\left(
			\int\frac{\dd[3]{p}}{(2\pi)^3\sqrt{2\omega(\vb{p})}}
			f(\vb{p})
			\hat{a}^\dagger(\vb{p})
		\right)^n
		\ket{0}
		\label{eq:qkg_number_state}
	\end{equation}
	and the spectrum is required to satisfy
	\begin{equation}
		\int\frac{\dd[3]{p}}{(2\pi)^32\omega(\vb{p})}
		\abs{f(\vb{p})}^2
		=
		1
		\label{eq:number_state_spectrum_constraint}
		.
	\end{equation}
\end{definition}

\begin{lemma}\label{thm:qkg_number_state_inner_single}
%{qkgnumbersingleinner}
	Let $\ket{1_f},\ket{1_g}$ be two single-particle number states with spectrum $f(\vb{p}),g(\vb{p})$, then their inner product
	\begin{equation}
		\braket{1_f}{1_g}
		=
		\int\frac{\dd[3]{p}}{(2\pi)^32\omega(\vb{p})}
		f(\vb{p})^*
		g(\vb{p})
	\end{equation}
	is equal to the spectral overlap.
\end{lemma}
\begin{proof}
	Inserting the definition of the number state for $n=1$
	\begin{equation*}
		\begin{split}
			\braket{1_f}{1_g}
			&=
			\expval{
				\left(
					\int\frac{\dd[3]{p}}{(2\pi)^3\sqrt{2\omega(\vb{p})}}
					f(\vb{p})
					\hat{a}^\dagger(\vb{p})
				\right)^\dagger
				\left(
					\int\frac{\dd[3]{q}}{(2\pi)^3\sqrt{2\omega(\vb{q})}}
					g(\vb{q})
					\hat{a}^\dagger(\vb{q})
				\right)
			}{0}
			\\
			&=
			\int\frac{\dd[3]{p}}{(2\pi)^3\sqrt{2\omega(\vb{p})}}
			f(\vb{p})^*
			\int\frac{\dd[3]{q}}{(2\pi)^3\sqrt{2\omega(\vb{q})}}
			g(\vb{q})
			\expval{\hat{a}(\vb{p})\hat{a}^\dagger(\vb{q})}{0}
			\\
			&=
			\int\frac{\dd[3]{p}}{(2\pi)^3\sqrt{2\omega(\vb{p})}}
			f(\vb{p})^*
			\int\frac{\dd[3]{q}}{(2\pi)^3\sqrt{2\omega(\vb{q})}}
			g(\vb{q})
			\comm{\hat{a}(\vb{p})}{\hat{a}^\dagger(\vb{q})}
			\braket{0}
			\\
			&=
			\int\frac{\dd[3]{p}}{(2\pi)^3\sqrt{2\omega(\vb{p})}}
			f(\vb{p})^*
			\int\frac{\dd[3]{q}}{(2\pi)^3\sqrt{2\omega(\vb{q})}}
			g(\vb{q})
			(2\pi)^3\delta^{(3)}(\vb{q}-\vb{p})
			\\
			&=
			\int\frac{\dd[3]{p}}{(2\pi)^32\omega(\vb{p})}
			f(\vb{p})^*
			g(\vb{p})
		\end{split}
	\end{equation*}
	where we used $\hat{a}(\vb{p})\ket{0}=0$ and $\braket{0}=1$ to rewrite the expectation value as commutator.
\end{proof}

\begin{corollary}
	The single-particle number state is normalized
	\begin{equation}
		\braket{1_f}{1_f}
		=
		\int\frac{\dd[3]{p}}{(2\pi)^3\sqrt{2\omega(\vb{p})}}
		\abs{f(\vb{p})}^2
		=
		1
		.
	\end{equation}
\end{corollary}
In Wightman quantum field theory, we define quantum field operators as operator-valued distributions acting on the Schwartz function space, see Ref.~\cite{Bogolubov1989} and Ref.~\cite{Streater2016}.
It turns out that there is a deep connection between the spectrum $f(\vb{p})$ and the smearing function $f(x^\mu)\in\mathcal{S}(\mathbb{R}^4,\mathbb{R})$ of the negative Klein-Gordon operator.

\begin{lemma}\label{thm:qkg_number_state_smearing}
%{qkgnumbersmearing}
	Let $f(x^\mu)$ be a smearing function with Fourier transform $f(p^\mu)=f(p_0,\vb{p})$ satisfying the spectral constraint \cref{eq:number_state_spectrum_constraint} with $f(\vb{p})=f(\omega(\vb{p}),\vb{p})$, and $\hat\phi^-(x^\mu)$ the negative frequency Klein-Gordon operator as defined in Wightman quantum field theory, then
	\begin{equation}
		\ket{n_f}
		=
		\frac{1}{\sqrt{n!}}
		\hat\phi^-[f]^n
		\ket{0}
		=
		\frac{1}{\sqrt{n!}}
		\left(
			\int\dd[4]{x}
			f(x^\mu)
			\hat\phi^-(x^\mu)
		\right)^n
		\ket{0}
		\label{eq:qkg_number_state_smearing}
		,
	\end{equation}
	i.e., the smeared negative frequency Klein-Gordon operator creates a superposition of particles smeared with $f(x^\mu)$.
\end{lemma}
\begin{proof}
	Comparing \cref{eq:qkg_number_state_smearing} and \cref{eq:qkg_number_state} leaves us with showing
	\begin{equation*}
		\int\dd[4]{x}
		f(x^\mu)
		\hat\phi^-(x^\mu)
		=
		\int\frac{\dd[3]{p}}{(2\pi)^3\sqrt{2\omega(\vb{p})}}
		f(\vb{p})
		\hat{a}^\dagger(\vb{p})
		.
	\end{equation*}
	Inserting the negative frequency Klein-Gordon operator, \cref{eq:qkg_positive_negative_frequency}, into the left-hand side yields
	\begin{equation*}
		\begin{split}
			\int\dd[4]{x}
			f(x^\mu)
			\hat\phi^-(x^\mu)
			&=
			\int\dd[4]{x}
			f(x^\mu)
			\int\frac{\dd[3]{p}}{(2\pi)^3\sqrt{2\omega(\vb{p})}}
			\eval{
				e^{+ip_\mu x^\mu}
				\hat{a}^\dagger(\vb{p})
			}_{p_0=\omega(\vb{p})}
			\\
			&=
			\int\frac{\dd[3]{p}}{(2\pi)^3\sqrt{2\omega(\vb{p})}}
			\left(
				\int\dd[4]{x}
				f(x^\mu)
				e^{+ip_\mu x^\mu}
			\right)_{p_0=\omega(\vb{p})}
			\hat{a}^\dagger(\vb{p})
			\\
			&=
			\int\frac{\dd[3]{p}}{(2\pi)^3\sqrt{2\omega(\vb{p})}}
			\eval{f(p^\mu)}_{p_0=\omega(\vb{p})}
			\hat{a}^\dagger(\vb{p})
			\\
			&=
			\int\frac{\dd[3]{p}}{(2\pi)^3\sqrt{2\omega(\vb{p})}}
			f\left(\omega(\vb{p}),\vb{p}\right)
			\hat{a}^\dagger(\vb{p})
		\end{split}
	\end{equation*}
	which equals the right-hand side when we take $f(\vb{p})=f\left(\omega(\vb{p}),\vb{p}\right)$.
\end{proof}

\begin{theorem}\label{thm:qkg_comm_smeared_pn}
%{qkgcommsmearedpn}
	Let $n\in\mathbb{N}$ and $\hat\phi^+[f],\hat\phi^-[g]$ be the smeared positive and negative frequency Klein-Gordon operator, then
	\begin{equation}
		\comm{\hat\phi^+[f]}{\hat\phi^-[g]^n}
		=
		n
		\braket{1_f}{1_g}
		\hat\phi^-[g]^{n-1}
		.
	\end{equation}
\end{theorem}
\begin{proof}
	Proof by induction for $n\in\mathbb{N}$.
	\begin{enumerate}
		\item Induction start ($n=1$):
		\begin{equation*}
			\begin{split}
				\comm{\hat\phi^+[f]}{\hat\phi^-[g]}
				&=
				\int\dd[4]{x}f(x^\mu)
				\int\dd[4]{y}g(y^\mu)
				\comm{\hat\phi^+(x^\mu)}{\hat\phi^-(y^\mu)}
				\\
				&=
				\int\dd[4]{x}f(x^\mu)
				\int\dd[4]{y}g(y^\mu)
				D(x^\mu-y^\mu)
				\\
				&=
				\int\dd[4]{x}f(x^\mu)
				\int\dd[4]{y}g(y^\mu)
				\int\frac{\dd[3]{p}}{(2\pi)^32\omega(\vb{p})}
				\eval{e^{-ip_\mu(x^\mu-y^\mu)}}_{p_0=\omega(\vb{p})}
				\\
				&=
				\int\frac{\dd[3]{p}}{(2\pi)^32\omega(\vb{p})}
				\left(
					\int\dd[4]{x}
					f(x^\mu)
					e^{+ip_\mu x^\mu}
				\right)^*_{p_0=\omega(\vb{p})}
				\left(
					\int\dd[4]{y}
					g(y^\mu)
					e^{+ip_\mu y^\mu}
				\right)_{p_0=\omega(\vb{p})}
				\\
				&=
				\int\frac{\dd[3]{p}}{(2\pi)^32\omega(\vb{p})}
				f\left(\omega(\vb{p}),\vb{p}\right)^*
				g\left(\omega(\vb{p}),\vb{p}\right)
				\\
				&=
				\braket{1_f}{1_g}
			\end{split}
		\end{equation*}
		where we used \Cref{thm:qkg_comm_pn}, the four-Fourier transform and the definition of the propagator, alternatively, using the commutation relation of the annihilation and creation operator
		\begin{equation*}
			\begin{split}
				\comm{\hat\phi^+[f]}{\hat\phi^-[g]}
				&=
				\int\frac{\dd[3]{p}}{(2\pi)^3\sqrt{2\omega(\vb{p})}}
				f(\vb{p})^*
				\int\frac{\dd[3]{q}}{(2\pi)^3\sqrt{2\omega(\vb{q})}}
				g(\vb{q})
				\comm{\hat{a}(\vb{p})}{\hat{a}^\dagger(\vb{q})}
				\\
				&=
				\int\frac{\dd[3]{p}}{(2\pi)^3\sqrt{2\omega(\vb{p})}}
				f(\vb{p})^*
				\int\frac{\dd[3]{q}}{(2\pi)^3\sqrt{2\omega(\vb{q})}}
				g(\vb{q})
				(2\pi)^3\delta^{(3)}(\vb{q}-\vb{p})
				\\
				&=
				\int\frac{\dd[3]{p}}{(2\pi)^32\omega(\vb{p})}
				f(\vb{p})^*
				g(\vb{p})
				\\
				&=
				\braket{1_f}{1_g}
			\end{split}
		\end{equation*}
		which also equals $\braket{1_f}{1_g}$.
		\item Induction step ($n\to n+1$):
		\begin{equation*}
			\begin{split}
				\comm{\hat\phi^+[f]}{\hat\phi^-[g]^{n+1}}
				&=
				\comm{\hat\phi^+[f]}{\hat\phi^-[g]\hat\phi^-[g]^n}
				\\
				&=
				\hat\phi^-[g]
				\comm{\hat\phi^+[f]}{\hat\phi^-[g]^n}
				+
				\comm{\hat\phi^+[f]}{\hat\phi^-[g]}
				\hat\phi^-[g]^n
				\\
				&=
				\hat\phi^-[g]
				n\braket{1_f}{1_g}
				\hat\phi^-[g]^{n-1}
				+
				\braket{1_f}{1_g}
				\hat\phi^-[g]^n
				\\
				&=
				(n+1)
				\braket{1_f}{1_g}
				\hat\phi^-[g]^n
				.
			\end{split}
		\end{equation*}
	\end{enumerate}
\end{proof}

\begin{lemma}\label{thm:qkg_smeared_pos}
%{qkgsmearedpos}
	Let $\ket{n_f}$ be a number state and $\hat\phi^+[g]$ be the positive frequency Klein-Gordon operator, then
	\begin{equation}
		\hat\phi^+[g]
		\ket{n_f}
		=
		\sqrt{n}
		\braket{1_g}{1_f}
		\ket{n-1_f}
		.
	\end{equation}
\end{lemma}
\begin{proof}
	Inserting the definitions and using \Cref{thm:qkg_comm_smeared_pn}, we find
	\begin{equation*}
		\begin{split}
			\hat\phi^+[g]
			\ket{n_f}
			&=
			\frac{1}{\sqrt{n!}}
			\hat\phi^+[g]
			\hat\phi^-[f]^n
			\ket{0}
			\\
			&=
			\frac{1}{\sqrt{n!}}
			\comm{\hat\phi^+[g]}{\hat\phi^-[f]^n}
			\ket{0}
			\\
			&=
			\frac{1}{\sqrt{n!}}
			n
			\braket{1_g}{1_f}
			\hat\phi^-[f]^{n-1}
			\ket{0}
			\\
			&=
			\sqrt{n}
			\braket{1_g}{1_f}
			\ket{n-1_f}
			.
		\end{split}
	\end{equation*}
\end{proof}

\begin{lemma}\label{thm:qkg_smeared_pn}
%{qkgsmearedpn}
	Let $\ket{n_f}$ be a number state and $\hat\phi^+[f],\hat\phi^-[f]$ be the positive and negative frequency Klein-Gordon operator, then
	\begin{align}
		\hat\phi^+[f]
		\ket{n_f}
		=
		\sqrt{n}
		\ket{n-1_f}
		&&
		\hat\phi^-[f]
		\ket{n_f}
		=
		\sqrt{n+1}
		\ket{n+1_f}
		,
	\end{align}
	i.e., the smeared positive frequency operator $\hat\phi^+[f]$ acting on the number state removes a particle and $\hat\phi^-[f]$ adds a particle with spectrum $f(\vb{p})$.
\end{lemma}
\begin{proof}
	The left equation follows from \Cref{thm:qkg_smeared_pos} for $f=g$, the right equation follows from
	\begin{equation*}
		\begin{split}
			\hat\phi^-[f]
			\ket{n_f}
			&=
			\frac{1}{\sqrt{n!}}
			\hat\phi^-[f]
			\hat\phi^-[f]^n
			\ket{0}
			\\
			&=
			\sqrt{n+1}
			\frac{1}{\sqrt{(n+1)!}}
			\hat\phi^-[f]^{n+1}
			\ket{0}
			\\
			&=
			\sqrt{n+1}
			\ket{n+1_f}
			.
		\end{split}
	\end{equation*}
\end{proof}

\begin{theorem}\label{thm:qkg_number_state_eigenstate}%{qkgnumbereigenstate}
	The number state $\ket{n_f}$ is an eigenstate of the number operator $\hat{N}$ to eigenvalue $n$
	\begin{equation}
		\hat{N}
		\ket{n_f}
		=
		n
		\ket{n_f}
		.
	\end{equation}
\end{theorem}
\begin{proof}
	Using \Cref{thm:qkg_comm_annihilation_smeared_pos_kg} we can easily show the eigenvalue equation
	\begin{equation*}
		\begin{split}
			\hat{N}
			\ket{n_f}
			&=
			\int\frac{\dd[3]{p}}{(2\pi)^3}
			\hat{a}^\dagger(\vb{p})
			\hat{a}(\vb{p})
			\frac{1}{\sqrt{n!}}
			\hat\phi^-[f]^n
			\ket{0}
			\\
			&=
			\frac{1}{\sqrt{n!}}
			\int\frac{\dd[3]{p}}{(2\pi)^3}
			\hat{a}^\dagger(\vb{p})
			\comm{\hat{a}(\vb{p})}{\hat\phi^-[f]^n}
			\ket{0}
			\\
			&=
			\frac{1}{\sqrt{n!}}
			\int\frac{\dd[3]{p}}{(2\pi)^3}
			\hat{a}^\dagger(\vb{p})
			n
			\frac{f(\vb{p})}{\sqrt{2\omega(\vb{p})}}
			\hat\phi^-[f]^{n-1}
			\ket{0}
			\\
			&=
			\sqrt{n}
			\int\frac{\dd[3]{p}}{(2\pi)^3\sqrt{2\omega(\vb{p})}}
				f(\vb{p})
			\hat{a}^\dagger(\vb{p})
			\frac{1}{\sqrt{(n-1)!}}
			\hat\phi^-[f]^{n-1}
			\ket{0}
			\\
			&=
			\sqrt{n}
			\hat\phi^-[f]
			\ket{n-1_f}
			=
			n
			\ket{n_f}
		\end{split}
	\end{equation*}
	where we used \Cref{thm:qkg_smeared_pn} in the last step.
\end{proof}

\begin{theorem}\label{thm:qkg_number_state_inner_product}
%{qkgnumberinnerproduct}
	Let $\ket{n_f},\ket{m_f}$ be an $n$-particle and $m$-particle number state with spectrum $f(\vb{p})$, then their inner product is
	\begin{equation}
		\braket{n_f}{m_f}
		=
		\delta_{nm}
		\left(
			\int\frac{\dd[3]{p}}{(2\pi)^32\omega(\vb{p})}
			f(\vb{p})^*
			g(\vb{p})
		\right)^n
		.
	\end{equation}
\end{theorem}
\begin{proof}
	First, we note that the inner product $\braket{n_f}{m_g}$ is only non-zero iff $n=m$ and we are left to simplify
	\begin{equation*}
		\begin{split}
			\braket{n_f}{n_g}
			&=
			\expval{
				\frac{1}{\sqrt{n!}}
				\left(
					\int\frac{\dd[3]{p}}{(2\pi)^3\sqrt{2\omega(\vb{p})}}
					f(\vb{p})^*
					\hat{a}(\vb{p})
				\right)^n
				\left(
					\int\frac{\dd[3]{q}}{(2\pi)^3\sqrt{2\omega(\vb{q})}}
					g(\vb{q})
					\hat{a}^\dagger(\vb{q})
				\right)^n
			}{0}
			\\
			&=
			\frac{1}{n!}
			\int\frac{\dd[3]{p_1}}{(2\pi)^3\sqrt{2\omega(\vb{p}_1)}}
			f(\vb{p}_1)^*
			\dots
			\int\frac{\dd[3]{q_n}}{(2\pi)^3\sqrt{2\omega(\vb{q}_n)}}
			g(\vb{q}_n)
			\expval{
				\hat{a}(\vb{p}_1)
				\dots
				\hat{a}^\dagger(\vb{q}_n)
			}{0}
		\end{split}
		.
	\end{equation*}
	Using induction it is possible to show that
	\begin{equation*}
		\expval{
				\hat{a}(\vb{p}_1)
				\dots
				\hat{a}(\vb{p}_1)
				\hat{a}^\dagger(\vb{q}_1)
				\dots
				\hat{a}^\dagger(\vb{q}_n)
		}{0}
		=
		\sum_{\pi\in\textrm{perm}}
		\prod^n_{i=1}
		(2\pi)^3
		\delta^{(3)}(\vb{p}_i-\vb{q}_{\pi(i)})
		.
	\end{equation*}
	Inserting the former into the latter and relabeling the integration variables accordingly, the sum over all permutations account to a factor of $n!$ and thus,
	\begin{equation*}
		\begin{split}
			\braket{n_f}{n_g}
			&=
			\int\frac{\dd[3]{p_1}}{(2\pi)^3\sqrt{2\omega(\vb{p}_1)}}
			f(\vb{p}_1)^*
			\dots
			\int\frac{\dd[3]{q_n}}{(2\pi)^3\sqrt{2\omega(\vb{q}_n)}}
			g(\vb{q}_n)
			\\
			&\times
			(2\pi)^3
			\delta^{(3)}(\vb{p}_1-\vb{q}_1)
			\dots
			(2\pi)^3
			\delta^{(3)}(\vb{p}_n-\vb{q}_n)
			\\
			&=
			\left(
				\int\frac{\dd[3]{p}}{(2\pi)^3\sqrt{2\omega(\vb{p})}}
				\int\frac{\dd[3]{q}}{(2\pi)^3\sqrt{2\omega(\vb{q})}}
				f(\vb{p})^*
				g(\vb{q})
				(2\pi)^3
				\delta^{(3)}(\vb{p}-\vb{q})
			\right)^n
			\\
			&=
			\left(
				\int\frac{\dd[3]{p}}{(2\pi)^32\omega(\vb{p})}
				f(\vb{p})^*
				g(\vb{p})
			\right)^n
		\end{split}
		.
	\end{equation*}
	Alternatively, we show
	\begin{equation*}
		\braket{n_f}{n_f}
		=
		\frac{1}{n!}
		\expval{
			\hat\phi^+[f]^n
			\hat\phi^-[g]^n
		}{0}
		=
		\braket{1_f}{1_g}^n
	\end{equation*}
	 by induction
	\begin{itemize}
		\item Induction start ($n=2$):
		\begin{equation*}
			\begin{split}
				\braket{2_f}
				&=
				\frac{1}{2}
				\expval{
					\hat\phi^+[f]
					\hat\phi^+[f]
					\hat\phi^-[g]
					\hat\phi^-[g]
				}{0}
				\\
				&=
				\frac{1}{2}
				\expval{
					\hat\phi^+[f]
					\braket{1_f}{1_g}
					\hat\phi^-[g]
				}{0}
				+
				\frac{1}{2}
				\expval{
					\hat\phi^+[f]
					\hat\phi^-[g]
					\hat\phi^+[f]
					\hat\phi^-[g]
				}{0}
				\\
				&=
				\frac{1}{2}
				\braket{1_f}{1_g}
				\expval{
					\hat\phi^+[f]
					\hat\phi^-[g]
				}{0}
				+
				\frac{1}{2}
				\expval{
					\braket{1_f}{1_g}
					\braket{1_f}{1_g}
				}{0}
				\\
				&=
				\braket{1_f}{1_g}^2
			\end{split}
		\end{equation*}
		where we used
		\begin{equation*}
			\braket{1_f}{1_g}
			=
			\comm{\hat\phi^+[f]}{\hat\phi^-[g]}
			=
			\hat\phi^+[f]
			\hat\phi^-[g]
			-
			\hat\phi^-[g]
			\hat\phi^+[f]
		\end{equation*}
		which follows from \Cref{thm:qkg_comm_smeared_pn} for $n=1$.
		\item Induction start ($n\to n+1$):
		\begin{equation*}
			\begin{split}
				\braket{n+1_f}{n+1_g}
				&=
				\frac{1}{(n+1)!}
				\expval{
					\hat\phi^+[f]^n
					\hat\phi^+[f]
					\hat\phi^-[g]
					\hat\phi^-[g]^n
				}{0}
				\\
				&=
				\frac{1}{n+1}
				\braket{1_f}{1_g}
				\braket{n_f}{n_g}
				+
				\frac{1}{(n+1)!}
				\expval{
					\hat\phi^+[f]^n
					\hat\phi^-[g]
					\hat\phi^+[f]
					\hat\phi^-[g]^n
				}{0}
				\\
				&=
				\frac{1}{n+1}
				\braket{1_f}{1_g}^{n+1}
				+
				\frac{1}{(n+1)!}
				\expval{
					\hat\phi^+[f]^n
					\hat\phi^-[g]
					n
					\braket{n_f}{n_g}
					\hat\phi^-[g]^{n-1}
				}{0}
				\\
				&=
				\frac{1}{n+1}
				\braket{1_f}{1_g}^{n+1}
				+
				\frac{n}{(n+1)!}
				\braket{1_f}{1_g}
				\expval{
					\hat\phi^+[f]^n
					\hat\phi^-[g]^n
				}{0}
				\\
				&=
				\frac{1}{n+1}
				\braket{1_f}{1_g}^{n+1}
				+
				\frac{n}{n+1}
				\braket{1_f}{1_g}
				\braket{n_f}{n_g}
				\\
				&=
				\braket{1_f}{1_g}^{n+1}
			\end{split}
		\end{equation*}
		where we used
		\begin{equation*}
			\hat\phi^+[f]
			\hat\phi^-[g]^n
			\ket{0}
			=
			\comm{\hat\phi^+[f]}{\hat\phi^-[g]^n}
			\ket{0}
			=
			n
			\braket{1_f}{1_g}
			\hat\phi^-[g]^{n-1}
			\ket{0}
		\end{equation*}
		which follows from \Cref{thm:qkg_comm_smeared_pn}.
	\end{itemize}
\end{proof}

\begin{lemma}\label{thm:qkg_number_state_single_wave_function}
%{qkgsinglewavefunction}
	The coordinate wave function of a single-particle number state is
	\begin{equation}
		\psi(x^\mu)
		=
		\bra{0}
		\hat\phi(x^\mu)
		\ket{1_f}
		=
		\int\frac{\dd[3]{p}}{(2\pi)^32\omega(\vb{p})}
		\eval{
			f(\vb{p})
			e^{-ip_\mu x^\mu}
		}_{p_0=\omega(\vb{p})}
		.
	\end{equation}
\end{lemma}
\begin{proof}
	of the coordinate wave function
	\begin{equation*}
		\psi(t,\vb{x})
		=
		\bra{0}
		\hat\phi(t,\vb{x})
		\ket{1_f}
		=
		\bra{0}
		\hat\phi^+(t,\vb{x})
		\ket{1_f}
		.
	\end{equation*}
	Inserting the definition of the single-particle number state yields
	\begin{equation*}
		\begin{split}
			\psi(t,\vb{x})
			&=
			\int\frac{\dd[3]{p}}{(2\pi)^3\sqrt{2\omega(\vb{p})}}
			e^{-ip_\mu x^\mu}
			\expval{
				\hat{a}(\vb{p})
				\int\frac{\dd[3]{q}}{(2\pi)^3\sqrt{2\omega(\vb{q})}}
				f(\vb{q})
				\hat{a}^\dagger(\vb{p})
			}{0}
			\\
			&=
			\int\frac{\dd[3]{p}}{(2\pi)^3\sqrt{2\omega(\vb{p})}}
			\int\frac{\dd[3]{q}}{(2\pi)^3\sqrt{2\omega(\vb{q})}}
			e^{-ip_\mu x^\mu}
			f(\vb{q})
			\expval{
				\hat{a}(\vb{p})
				\hat{a}^\dagger(\vb{p})
			}{0}
			\\
			&=
			\int\frac{\dd[3]{p}}{(2\pi)^3\sqrt{2\omega(\vb{p})}}
			\int\frac{\dd[3]{q}}{(2\pi)^3\sqrt{2\omega(\vb{q})}}
			e^{-ip_\mu x^\mu}
			f(\vb{q})
			(2\pi)^3
			\delta^{(3)}(\vb{q}-\vb{p})
			\\
			&=
			\int\frac{\dd[3]{p}}{(2\pi)^32\omega(\vb{p})}
			e^{-ip_\mu x^\mu}
			f(\vb{p})
		\end{split}
	\end{equation*}
	in agreement with Ref.~\cite[eq.~4]{Naumov2013}.
\end{proof}

\begin{lemma}\label{thm:number_state_single_group_velocity}
%{qkgsinglegroupvelocity}
	The group velocity of a single-particle number state $\ket{1_f}$ is
	\begin{equation}
		\expval{\vb{v}}
		=
		\int\frac{\dd[3]{p}}{(2\pi)^32\omega(\vb{p})}
		\abs{f(\vb{p})}^2
		\frac{\vb{p}}{\omega(\vb{p})}
		\label{eq:single_particle_number_state_group_velocity}
		.
	\end{equation}
\end{lemma}
\begin{proof}
	The probability current is\footnote{See Ref.~\cite[p.~18]{Peskin1995} for a derivation from Noether's theorem}
	\begin{equation*}
		j^\mu(t,\vb{x})
		=
		i
		\bigl\{
			\psi(t,\vb{x})
			\partial^\mu
			\psi(t,\vb{x})^*
			-
			\psi(t,\vb{x})^*
			\partial^\mu
			\psi(t,\vb{x})
		\bigr\}
	\end{equation*}
	and can be written
	\begin{equation*}
		j^\mu(t,\vb{x})
		=
		2\frac{\psi(t,\vb{x})\partial^\mu\psi(t,\vb{x})^*-\text{c.c.}}{2i}
		=
		2\Im\left\{
			\psi(t,\vb{x})
			\partial^\mu
			\psi(t,\vb{x})^*
		\right\}
		.
	\end{equation*}
	We proceed with the argument of the imaginary part
	\begin{equation*}
		\begin{split}
			\psi(t,\vb{x})^*
			\partial^\mu
			\psi(t,\vb{x})
			&=
			\int\frac{\dd[3]{p}}{(2\pi)^32\omega(\vb{p})}
			e^{-ip_\mu x^\mu}
			f(\vb{p})
			\partial^\mu
			\int\frac{\dd[3]{q}}{(2\pi)^32\omega(\vb{q})}
			e^{+iq_\mu x^\mu}
			f(\vb{q})^*
			\\
			&=
			\int\frac{\dd[3]{p}}{(2\pi)^32\omega(\vb{p})}
			\int\frac{\dd[3]{q}}{(2\pi)^32\omega(\vb{q})}
			iq^\mu
			f(\vb{p})
			f(\vb{q})^*
			e^{-i(p_\mu-q_\mu)x^\mu}
		\end{split}
		.
	\end{equation*}
	Inserting the argument back into the probability current and using $\Im{iz}=\Re{z}$, we find
	\begin{equation*}
		j^\mu(t,\vb{x})
		=
		\int\frac{\dd[3]{p}}{(2\pi)^32\omega(\vb{p})}
		\int\frac{\dd[3]{q}}{(2\pi)^32\omega(\vb{q})}
		2\Re\left\{
			q^\mu
			f(\vb{p})
			f(\vb{q})^*
			e^{-i(p_\mu-q_\mu)x^\mu}
		\right\}
		.
	\end{equation*}
	Writing out the real part and relabeling the integration variables in the second term yields
	\begin{equation*}
		j^\mu(t,\vb{x})
		=
		\int\frac{\dd[3]{p}}{(2\pi)^32\omega(\vb{p})}
		\int\frac{\dd[3]{q}}{(2\pi)^32\omega(\vb{q})}
		\left\{
			q^\mu
			+
			p^\mu
		\right\}
		f(\vb{p})
		f(\vb{q})^*
		e^{-i(p_\mu-q_\mu)x^\mu}
		.
	\end{equation*}	
	The last result is in agreement with Ref.~\cite[eqs.~36,37]{Naumov2013} if one further assumes $f(\vb{p})$ to be real.
\end{proof}

\begin{lemma}\label{thm:number_state_single_localization}
%{qkgsinglelocalization}
	The single-particle number state is localized on a trajectory
	\begin{equation}
		\expval{\vb{x}(t)}
		=
		\expval{\vb{v}}t
	\end{equation}
	moving with the group velocity $\expval{\vb{v}}$.
\end{lemma}
\begin{proof}
	No explicit proof, result claimed in Ref.~\cite[eq.~38]{Naumov2013}.
\end{proof}

\begin{lemma}\label{thm:number_state_momentum_density_mean}
%{qkgnumbermomentumdensitymean}
	\begin{equation}
		\expval{\hat\pi(x^\mu)}{n_f}
		=
		0
	\end{equation}
\end{lemma}
\begin{proof}
	Expanding the momentum density operator into positive and negative frequency terms, the positive frequency term has one creation operator more than annihilation operators, and the negative frequency term has one annihilation operator more than than creation operators, and thus, at some point a creation operator acting on the left vacuum or an annihilation operator acting on the right vacuum will be zero and the expectation value vanishes.
\end{proof}

\begin{lemma}\label{thm:number_state_momentum_density_corr}
%{qkgnumbermomentumdensitycorr}
	\begin{equation}
		\begin{split}
			\expval{\hat\pi(x^\mu)\hat\pi(y^\mu)}{n_f}
			&=
			\Re
			\int\frac{\dd[3]{p}}{(2\pi)^3}
			\omega(\vb{p})
			\eval{e^{+ip_\mu(x^\mu-y^\mu)}}_{p_0=\omega(\vb{p})}
			\\
			&+
			\Re
			\frac{n}{2}
			\int\frac{\dd[3]{p}}{(2\pi)^3}
			\eval{
				f(p_0,\vb{p})
				e^{+ip_\mu x^\mu}
			}_{p_0=\omega(\vb{p})}
			\int\frac{\dd[3]{q}}{(2\pi)^3}
			\eval{
				f(q_0,\vb{q})^*
				e^{-iq_\mu y^\mu}
			}_{q_0=\omega(\vb{q})}
		\end{split}
	\end{equation}
\end{lemma}
\begin{proof}
	\begin{equation*}
		\begin{split}
			\expval{\hat\pi(x^\mu)\hat\pi(y^\mu)}{n_f}
			&=
			\expval{\hat\pi^-(x^\mu)\hat\pi^-(y^\mu)}{n_f}
			+
			\expval{\hat\pi^+(x^\mu)\hat\pi^-(y^\mu)}{n_f}
			+
			\text{h.c.}
			\\
			&=
			\expval{\hat\pi^+(x^\mu)\hat\pi^-(y^\mu)}{n_f}
			+
			\text{h.c.}
		\end{split}
	\end{equation*}
	\begin{equation*}
		\expval{\hat\pi^+(x^\mu)\hat\pi^-(y^\mu)}{n_f}
		=
		\int\frac{\dd[3]{p}}{(2\pi)^3}
		\left(+i\sqrt{\frac{\omega(\vb{p})}{2}}\right)
		e^{+ip_\mu x^\mu}
		\int\frac{\dd[3]{q}}{(2\pi)^3}
		\left(-i\sqrt{\frac{\omega(\vb{q})}{2}}\right)
		e^{-iq_\mu x^\mu}
		\expval{\hat{a}(\vb{p})\hat{a}^\dagger(\vb{q})}{n_f}
	\end{equation*}
	\begin{equation*}
		\begin{split}
			\expval{\hat{a}(\vb{p})\hat{a}^\dagger(\vb{q})}{n_f}
			&=
			(2\pi)^3\delta^{(3)}(\vb{q}-\vb{p})
			\braket{n_f}
			+
			\expval{\hat{a}^\dagger(\vb{q})\hat{a}(\vb{p})}{n_f}
			\\
			&=
			(2\pi)^3\delta^{(3)}(\vb{q}-\vb{p})
			+
			\frac{1}{n!}
			\expval{
				\hat\phi^-[f]^n
				\hat{a}^\dagger(\vb{q})
				\hat{a}(\vb{p})
				\hat\phi^+[f]^n
			}{0}
			\\
			&=
			(2\pi)^3\delta^{(3)}(\vb{q}-\vb{p})
			+
			\frac{n^2}{n}
			\frac{f(\vb{p})}{\sqrt{2\omega(\vb{p})}}
			\frac{f(\vb{q})^*}{\sqrt{2\omega(\vb{q})}}
			\braket{n-1_f}
			\\
			&=
			(2\pi)^3\delta^{(3)}(\vb{q}-\vb{p})
			+
			n
			\frac{f(\vb{p})}{\sqrt{2\omega(\vb{p})}}
			\frac{f(\vb{q})^*}{\sqrt{2\omega(\vb{q})}}
		\end{split}
	\end{equation*}
	\begin{equation*}
		\begin{split}
			\expval{\hat\pi(x^\mu)\hat\pi(y^\mu)}{n_f}
			&=
			\Re
			\int\frac{\dd[3]{p}}{(2\pi)^3}
			\omega(\vb{p})
			\eval{e^{+ip_\mu(x^\mu-y^\mu)}}_{p_0=\omega(\vb{p})}
			\\
			&+
			\Re
			\frac{n}{2}
			\int\frac{\dd[3]{p}}{(2\pi)^3}
			\eval{
				f(p_0,\vb{p})
				e^{+ip_\mu x^\mu}
			}_{p_0=\omega(\vb{p})}
			\int\frac{\dd[3]{q}}{(2\pi)^3}
			\eval{
				f(q_0,\vb{q})^*
				e^{-iq_\mu y^\mu}
			}_{q_0=\omega(\vb{q})}
		\end{split}
	\end{equation*}
\end{proof}

\begin{corollary}
	The variance of the momentum density operator is
	\begin{equation}
		\begin{split}
			\expval{\left(\Delta\hat\pi(x^\mu)\right)^2}{n_f}
			&=
			\int\frac{\dd[3]{p}}{(2\pi)^3}
			\omega(\vb{p})
			+
			\frac{n}{2}
			\abs{
				\int\frac{\dd[3]{p}}{(2\pi)^3}
				\eval{
					f(p_0,\vb{p})
					e^{+ip_\mu x^\mu}
				}_{p_0=\omega(\vb{p})}
			}^2
		\end{split}
	\end{equation}
\end{corollary}

\section{Coherent states}

\begin{definition}[Displacement operator]
	The displacement operator with spectrum $\alpha(\vb{p})$ is
	\begin{equation}
		\hat{D}[\alpha]
		=
		\exp\left\{
			\int\frac{\dd[3]{p}}{(2\pi)^3\sqrt{2\omega(\vb{p})}}
			\left\{
				\alpha(\vb{p})
				\hat{a}^\dagger(\vb{p})
				-
				\alpha(\vb{p})^*
				\hat{a}(\vb{p})
			\right\}
		\right\}
		\label{eq:qkg_displacement_operator}
		.
	\end{equation}
\end{definition}
\begin{corollary}\label{thm:qkg_displacement_smeared}
	The displacement operators can be expressed in terms of the smeared positive and negative frequency Klein-Gordon operators
	\begin{equation}
		\hat{D}[\alpha]
		=
		\exp\left\{
			\hat\phi^-[\alpha]
			-
			\hat\phi^+[\alpha]
		\right\}
		.
	\end{equation}
\end{corollary}
\begin{lemma}\label{thm:qkg_displacement_ordered}
%{qkgdisplacementordered}
	The displacement operator can be put in normal-
	\begin{equation}\label{eq:qkg_displacement_normal}
		\hat{D}[\alpha]
		=
		\exp\left\{
			+\hat\phi^-[\alpha]
		\right\}
		\exp\left\{
			-\hat\phi^+[\alpha]
		\right\}
		\exp\left\{
			-
			\frac{1}{2}
			\comm{\hat\phi^+[\alpha]}{\hat\phi^-[\alpha]}
		\right\}
	\end{equation}
	and antinormal-order
	\begin{equation}\label{eq:qkg_displacement_antinormal}
		\hat{D}[\alpha]
		=
		\exp\left\{
			-\hat\phi^+[\alpha]
		\right\}
		\exp\left\{
			+\hat\phi^-[\alpha]
		\right\}
		\exp\left\{
			+
			\frac{1}{2}
			\comm{\hat\phi^+[\alpha]}{\hat\phi^-[\alpha]}
		\right\}
	\end{equation}
	wherein the commutator evaluates to
	\begin{equation}
		\comm{\hat\phi^+[\alpha]}{\hat\phi^-[\alpha]}
		=
		\int\frac{\dd[3]{p}}{(2\pi)^32\omega(\vb{p})}
		\abs{\alpha(\vb{p})}^2
	\end{equation}
	in momentum space.
\end{lemma}
\begin{proof}
	From \Cref{thm:qkg_comm_pn} we know that
	\begin{equation*}
		\comm{\hat\phi^+[\alpha]}{\hat\phi^-[\alpha]}
		=
		\int\dd[4]{x}
		\int\dd[4]{y}
		\alpha(x^\mu)
		D(x^\mu-y^\mu)
		\alpha(y^\mu)
	\end{equation*}
	is complex-valued and we can use \cref{eq:bch_formula_reduced2} with $\hat{X}=\hat\phi^-[\alpha]$ and $\hat{Y}=-\hat\phi^+[\alpha]$ to obtain the normal-ordered displacement operator
	\begin{equation*}
		\hat{D}[\alpha]
		=
		\exp\left\{
			+
			\hat\phi^-[\alpha]
		\right\}
		\exp\left\{
			-
			\hat\phi^+[\alpha]
		\right\}
		\exp\left\{
			-
			\frac{1}{2}
			\comm{\hat\phi^+[\alpha]}{\hat\phi^-[\alpha]}
		\right\}
		.
	\end{equation*}
	The sum on the right-hand side of \cref{eq:bch_formula_reduced2} commutes and we can take $\hat{X}=-\hat\phi^+[\alpha]$ and $\hat{Y}=-\hat\phi^-[\alpha]$ to obtain the antinormal-ordered displacement operator.
	The commutator in momentum space is analog to the proof of \Cref{thm:qkg_number_state_inner_product}.
\end{proof}

\begin{lemma}\label{thm:qkg_displacement_product}
%{qkgdisplacementproduct}
	Let $\hat{D}[\alpha],\hat{D}[\beta]$ be two displacement operators with spectrum $\alpha(\vb{p}),\beta(\vb{p})$, then their product equals
	\begin{equation}\label{eq:qkg_displacement_product}
		\begin{split}
			\hat{D}[\alpha]
			\hat{D}[\beta]
			&=
			\hat{D}[\alpha+\beta]
			\exp\left\{
				-
				\frac{1}{2}
				\comm{\hat\phi^+[\alpha]}{\hat\phi^-[\beta]}
				+
				\frac{1}{2}
				\comm{\hat\phi^+[\beta]}{\hat\phi^-[\alpha]}
			\right\}
			\\
			&=
			\hat{D}[\alpha+\beta]
			\exp\left\{
				-
				\frac{1}{2}
				\iint\frac{\dd[3]{p}}{(2\pi)^32\omega(\vb{p})}
				\left\{
					\alpha(\vb{p})^*
					\beta(\vb{p})
					-
					\alpha(\vb{p})
					\beta(\vb{p})^*
				\right\}
			\right\}
			.
		\end{split}
	\end{equation}
\end{lemma}
\begin{proof}
	Inserting the definitions of the normal-ordered displacement operator \cref{eq:qkg_displacement_normal}, we find
	\begin{equation*}
		\begin{split}
			\hat{D}[\alpha]
			\hat{D}[\beta]
			&=
			\left(
				\exp\left\{
					-
					\frac{1}{2}
					\comm{\hat\phi^+[\alpha]}{\hat\phi^-[\alpha]}
				\right\}
				\exp\left\{
					+\hat\phi^-[\alpha]
				\right\}
				\exp\left\{
					-\hat\phi^+[\alpha]
				\right\}
			\right)
			\\
			&\times
			\left(
				\exp\left\{
					-
					\frac{1}{2}
					\comm{\hat\phi^+[\beta]}{\hat\phi^-[\beta]}
				\right\}
				\exp\left\{
					+\hat\phi^-[\beta]
				\right\}
				\exp\left\{
					-\hat\phi^+[\beta]
				\right\}
			\right)
			\\
			&=
			\exp\left\{
				-
				\frac{1}{2}
				\comm{\hat\phi^+[\alpha]}{\hat\phi^-[\alpha]}
				-
				\frac{1}{2}
				\comm{\hat\phi^+[\beta]}{\hat\phi^-[\beta]}
			\right\}
			\\
			&\times
			\exp\left\{
				\hat\phi^-[\alpha]
			\right\}
			\left(
				\exp\left\{
					\hat\phi^+[-\alpha]
				\right\}
				\exp\left\{
					\hat\phi^-[\beta]
				\right\}
			\right)
			\exp\left\{
				\hat\phi^+[-\beta]
			\right\}
			.
		\end{split}
	\end{equation*}
	As with the previous proof, we use the reduced Baker-Campbell-Hausdorff formula, \cref{eq:bch_formula_reduced2} and find
	\begin{equation*}
		\begin{split}
			\exp\hat{X}
			\exp\hat{Y}
			&=
			\exp\left\{
				\hat{X}
				+
				\hat{Y}
				+
				\frac{1}{2}
				\comm{\hat{X}}{\hat{Y}}
			\right\}
			\\
			&=
			\exp\left\{
				\hat{Y}
				+
				\hat{X}
			\right\}
			\exp\left\{
				\frac{1}{2}
				\comm{\hat{X}}{\hat{Y}}
			\right\}
			\\
			&=
			\exp\hat{Y}
			\exp\hat{X}
			\exp\left\{
				-
				\frac{1}{2}
				\comm{\hat{Y}}{\hat{X}}
			\right\}
			\exp\left\{
				+
				\frac{1}{2}
				\comm{\hat{X}}{\hat{Y}}
			\right\}
			\\
			&=
			\exp\hat{Y}
			\exp\hat{X}
			\exp\left\{
				\comm{\hat{X}}{\hat{Y}}
			\right\}
		\end{split}
	\end{equation*}
	which we insert back into the first equation
	\begin{equation*}
		\begin{split}
			\hat{D}[\alpha]
			\hat{D}[\beta]
			&=
			\exp\left\{
				-
				\frac{1}{2}
				\comm{\hat\phi^+[\alpha]}{\hat\phi^-[\alpha]}
				-
				\frac{1}{2}
				\comm{\hat\phi^+[\beta]}{\hat\phi^-[\beta]}
			\right\}
			\exp\left\{
				\hat\phi^-[\alpha]
			\right\}
			\\
			&\times
			\left(
				\exp\left\{
					\hat\phi^-[\beta]
				\right\}
				\exp\left\{
					\hat\phi^+[-\alpha]
				\right\}
				\exp\left\{
					\comm{\hat\phi^+[-\alpha]}{\hat\phi^-[\beta]}
				\right\}
			\right)
			\exp\left\{
				\hat\phi^+[-\beta]
			\right\}
			\\
			&=
			\exp\left\{
				-
				\frac{1}{2}
				\comm{\hat\phi^+[\alpha]}{\hat\phi^-[\alpha]}
				-
				\frac{1}{2}
				\comm{\hat\phi^+[\beta]}{\hat\phi^-[\beta]}
				+
				\comm{\hat\phi^+[-\alpha]}{\hat\phi^-[\beta]}
			\right\}
			\\
			&\times
			\exp\left\{
				+
				\hat\phi^-[\alpha+\beta]
			\right\}
			\exp\left\{
				-
				\hat\phi^+[\alpha+\beta]
			\right\}
			.
		\end{split}
	\end{equation*}
	To identify the normal-ordered displacement operator again, we are left to summarize the commutators
	\begin{equation*}
		\begin{split}
			&\
			-
			\frac{1}{2}
			\comm{\hat\phi^+[\alpha]}{\hat\phi^-[\alpha]}
			-
			\frac{1}{2}
			\comm{\hat\phi^+[\beta]}{\hat\phi^-[\beta]}
			+
			\comm{\hat\phi^+[-\alpha]}{\hat\phi^-[\beta]}
			\\
			=&\
			-
			\frac{1}{2}
			\comm{\hat\phi^+[\alpha]}{\hat\phi^-[\alpha]}
			-
			\frac{1}{2}
			\comm{\hat\phi^+[\beta]}{\hat\phi^-[\beta]}
			-
			\comm{\hat\phi^+[\alpha]}{\hat\phi^-[\beta]}
			\\
			=&\
			-
			\frac{1}{2}
			\comm{\hat\phi^+[\alpha]}{\hat\phi^-[\alpha]}
			-
			\frac{1}{2}
			\comm{\hat\phi^+[\alpha]}{\hat\phi^-[\beta]}
			-
			\frac{1}{2}
			\comm{\hat\phi^+[\beta]}{\hat\phi^-[\beta]}
			-
			\frac{1}{2}
			\comm{\hat\phi^+[\alpha]}{\hat\phi^-[\beta]}
			\\
			=&\
			-
			\frac{1}{2}
			\comm{\hat\phi^+[\alpha]}{\hat\phi^-[\alpha+\beta]}
			-
			\frac{1}{2}
			\comm{\hat\phi^+[\beta]}{\hat\phi^-[\beta]}
			-
			\frac{1}{2}
			\comm{\hat\phi^+[\alpha]}{\hat\phi^-[\beta]}
			\\
			=&\
			-
			\frac{1}{2}
			\comm{\hat\phi^+[\alpha]}{\hat\phi^-[\alpha+\beta]}
			-
			\frac{1}{2}
			\comm{\hat\phi^+[\beta]}{\hat\phi^-[\alpha+\beta]}
			\\
			&\
			+
			\frac{1}{2}
			\comm{\hat\phi^+[\beta]}{\hat\phi^-[\alpha+\beta]}
			-
			\frac{1}{2}
			\comm{\hat\phi^+[\beta]}{\hat\phi^-[\beta]}
			-
			\frac{1}{2}
			\comm{\hat\phi^+[\alpha]}{\hat\phi^-[\beta]}
			\\
			=&\
			-
			\frac{1}{2}
			\comm{\hat\phi^+[\alpha+\beta]}{\hat\phi^-[\alpha+\beta]}
			+
			\frac{1}{2}
			\comm{\hat\phi^+[\beta]}{\hat\phi^-[\alpha]}
			\\
			&\
			+
			\frac{1}{2}
			\comm{\hat\phi^+[\beta]}{\hat\phi^-[\beta]}
			-
			\frac{1}{2}
			\comm{\hat\phi^+[\beta]}{\hat\phi^-[\beta]}
			-
			\frac{1}{2}
			\comm{\hat\phi^+[\alpha]}{\hat\phi^-[\beta]}
			\\
			=&\
			-
			\frac{1}{2}
			\comm{\hat\phi^+[\alpha+\beta]}{\hat\phi^-[\alpha+\beta]}
			+
			\frac{1}{2}
			\comm{\hat\phi^+[\beta]}{\hat\phi^-[\alpha]}
			-
			\frac{1}{2}
			\comm{\hat\phi^+[\alpha]}{\hat\phi^-[\beta]}
		\end{split}
	\end{equation*}
	where we extensively made use of the (multi-)linearity.
	\begin{equation*}
		\begin{split}
			\hat{D}[\alpha]
			\hat{D}[\beta]
			&=
			\exp\left\{
				-
				\frac{1}{2}
				\comm{\hat\phi^+[\alpha+\beta]}{\hat\phi^-[\alpha+\beta]}
			\right\}
			\exp\left\{
				+
				\hat\phi^-[\alpha+\beta]
			\right\}
			\exp\left\{
				-
				\hat\phi^+[\alpha+\beta]
			\right\}
			\\
			&\times
			\exp\left\{
				-
				\frac{1}{2}
				\comm{\hat\phi^+[\alpha]}{\hat\phi^-[\beta]}
				+
				\frac{1}{2}
				\comm{\hat\phi^+[\beta]}{\hat\phi^-[\alpha]}
			\right\}
			\\
			&=
			\hat{D}[\alpha+\beta]
			\exp\left\{
				-
				\frac{1}{2}
				\comm{\hat\phi^+[\alpha]}{\hat\phi^-[\beta]}
				+
				\frac{1}{2}
				\comm{\hat\phi^+[\beta]}{\hat\phi^-[\alpha]}
			\right\}
		\end{split}
	\end{equation*}
\end{proof}

\begin{lemma}\label{thm:qkg_displacement_unitary}
%{qkgdisplacementunitary}	
	The displacement operator is unitary
	\begin{equation}
		\hat{D}[\alpha]^{-1}
		=
		\hat{D}[\alpha]^\dagger
		=
		\hat{D}[-\alpha]
		.
	\end{equation}
\end{lemma}
\begin{proof}
	Using the normal-ordered displacement operator, we show $\hat{D}[\alpha]^\dagger=\hat{D}[-\alpha]$
	\begin{equation*}
		\begin{split}
			\hat{D}[\alpha]^\dagger
			&=
			\left(
				\exp\left\{
					-
					\frac{1}{2}
					\comm{\hat\phi^+[\alpha]}{\hat\phi^-[\alpha]}
				\right\}
				\exp\left\{
					+
					\hat\phi^-[\alpha]
				\right\}
				\exp\left\{
					-
					\hat\phi^+[\alpha]
				\right\}
			\right)^\dagger
			\\
			&=
			\exp\left\{
				-
				\frac{1}{2}
				\comm{\hat\phi^+[\alpha]}{\hat\phi^-[\alpha]}
			\right\}
			\exp\left\{
				+
				\hat\phi^-[\alpha]
			\right\}^\dagger
			\exp\left\{
				-
				\hat\phi^+[\alpha]
			\right\}^\dagger
			\\
			&=
			\exp\left\{
				-
				\frac{1}{2}
				\comm{\hat\phi^+[\alpha]}{\hat\phi^-[\alpha]}
			\right\}
			\exp\left\{
				+
				\hat\phi^+[\alpha]
			\right\}
			\exp\left\{
				-
				\hat\phi^-[\alpha]
			\right\}
			\\
			&=
			\exp\left\{
				-
				\frac{1}{2}
				\comm{\hat\phi^+[-\alpha]}{\hat\phi^-[-\alpha]}
			\right\}
			\exp\left\{
				-
				\hat\phi^+[-\alpha]
			\right\}
			\exp\left\{
				+
				\hat\phi^-[-\alpha]
			\right\}
			\\
			&=
			\hat{D}[-\alpha]
			.
		\end{split}
	\end{equation*}
	\Cref{thm:qkg_displacement_product} lets us evaluate the product of two displacement operators
	\begin{equation*}
		\hat{D}[\alpha]^\dagger
		\hat{D}[\alpha]
		=
		\hat{D}[-\alpha]
		\hat{D}[\alpha]
		=
		\hat{D}[-\alpha+\alpha]
		=
		\mathbb{I}
		.
	\end{equation*}
	We conclude that $\hat{D}[\alpha]^\dagger$ is the inverse of the displacement operator and therefore the displacement operator is unitary.
\end{proof}

\begin{definition}[Coherent state]
	A coherent state $\ket{\alpha}$ with spectrum $\alpha(\vb{p})$
	\begin{equation}
		\begin{split}
			\ket{\alpha}
			&=
			\exp\left\{
				-
				\frac{1}{2}
				\comm{\hat\phi^+[\alpha]}{\hat\phi^-[\alpha]}
			\right\}
			\exp\left\{
				+\hat\phi^-[\alpha]
			\right\}
			\ket{0}
			\\
			&=
			\exp\left\{
				-
				\frac{1}{2}
				\int\frac{\dd[3]{p}}{(2\pi)^32\omega(\vb{p})}
				\abs{\alpha(\vb{p})}^2
			\right\}
			\exp\left\{
				\int\frac{\dd[3]{p}}{(2\pi)^3\sqrt{2\omega(\vb{p})}}
				\alpha(\vb{p})
				\hat{a}^\dagger(\vb{p})
			\right\}
			\ket{0}
		\end{split}
	\end{equation}
	is a coherent superposition of number states without constrained spectrum.
\end{definition}

\begin{lemma}\label{thm:qkg_displacement_vacuum}
%{qkgdisplacementvacuum}	
	The displacement operator creates a coherent state from the vacuum
	\begin{equation}\label{eq:qkg_displacement_vacuum}
		\hat{D}[\alpha]
		\ket{0}
		=
		\ket{\alpha}
		.
	\end{equation}
\end{lemma}
\begin{proof}
	Applying the normal-ordered displacement operator \cref{eq:qkg_displacement_normal} to the vacuum
	\begin{equation*}
		\hat{D}[\alpha]
		\ket{0}
		=
		\exp\left\{
			-
			\frac{1}{2}
			\comm{\hat\phi^+[\alpha]}{\hat\phi^-[\alpha]}
		\right\}
		\exp\left\{
			+
			\hat\phi^-[\alpha]
		\right\}
		\exp\left\{
			-
			\hat\phi^+[\alpha]
		\right\}
		\ket{0}
	\end{equation*}
	and noting that
	\begin{equation*}
		\exp\left\{
			-
			\hat\phi^+[\alpha]
		\right\}
		\ket{0}
		=
		\sum_{n=0}^\infty
		\frac{1}{n!}
		\left(
			-
			\hat\phi^+[\alpha]
		\right)^n
		\ket{0}
		=
		\ket{0}
	\end{equation*}
	because $\hat\phi^+\ket{0}=0$ leads to \cref{eq:qkg_displacement_vacuum}.
\end{proof}

\begin{corollary}
	The coherent state is normalized\footnote{In contrast to the number state, there is no constraint on the spectrum of the coherent state $\alpha(\vb{p})$.}
	\begin{equation}
		\braket{\alpha}
		=
		1
		.
	\end{equation}
\end{corollary}

\begin{theorem}\label{thm:qkg_coherent_state_eigenstate}
%{qkgcoherenteigenstate}	
	The coherent state is an eigenstate of the annihilation operator to eigenvalue $\alpha(\vb{p})/\sqrt{2\omega(\vb{p})}$, i.e.,
	\begin{equation}\label{eq:qkg_coherent_state_eigenstate}
		\hat{a}(\vb{p})
		\ket{\alpha}
		=
		\frac{\alpha(\vb{p})}{\sqrt{2\omega(\vb{p})}}
		\ket{\alpha}
		.
	\end{equation}
\end{theorem}
\begin{proof}
	Using \Cref{thm:qkg_comm_annihilation_smeared_pos_kg}, we find
	\begin{equation*}
		\begin{split}
			\hat{a}(\vb{p})
			\exp\left\{
				+
				\hat\phi^-[\alpha]
			\right\}
			\ket{0}
			&=
			\hat{a}(\vb{p})
			\sum_{n=0}^\infty
			\frac{1}{n!}
			\hat\phi^-[\alpha]^n
			\ket{0}
			\\
			&=
			\sum_{n=0}^\infty
			\frac{1}{n!}
			\hat{a}(\vb{p})
			\hat\phi^-[\alpha]^n
			\ket{0}
			\\
			&=
			\sum_{n=0}^\infty
			\frac{1}{n!}
			\comm{\hat{a}(\vb{p})}{\hat\phi^-[\alpha]^n}
			\ket{0}
			\\
			&=
			\sum_{n=0}^\infty
			\frac{1}{n!}
			n
			\frac{\alpha(\vb{p})}{\sqrt{2\omega(\vb{p})}}
			\hat\phi^-[\alpha]^{n-1}
			\ket{0}
			\\
			&=
			\frac{\alpha(\vb{p})}{\sqrt{2\omega(\vb{p})}}
			\sum_{n=1}^\infty
			\frac{1}{(n-1)!}
			\hat\phi^-[\alpha]^{n-1}
			\ket{0}
			\\
			&=
			\frac{\alpha(\vb{p})}{\sqrt{2\omega(\vb{p})}}
			\sum_{n=0}^\infty
			\frac{1}{n!}
			\hat\phi^-[\alpha]^{n}
			\ket{0}
			\\
			&=
			\frac{\alpha(\vb{p})}{\sqrt{2\omega(\vb{p})}}
			\exp\left\{
				+
				\hat\phi^-[\alpha]
			\right\}
			\ket{0}
		\end{split}
	\end{equation*}
	which multiplied by $\exp\left\{-\frac{1}{2}\comm{\hat\phi^+[\alpha]}{\hat\phi^-[\alpha]}\right\}$ reproduces \cref{eq:qkg_coherent_state_eigenstate}.
\end{proof}

\begin{lemma}\label{thm:qkg_coherent_state_energy}
%{qkgcoherentenergy}	
	The mean energy of the coherent state is
	\begin{equation}
		\expval{\hat{H}}{\alpha}
		=
		\int\frac{\dd[3]{p}}{(2\pi)^32\omega(\vb{p})}
		\omega(\vb{p})
		\abs{\alpha(\vb{p})}^2
	\end{equation}
	and the variance is
	\begin{equation}
		\expval{\left(\Delta\hat{H}\right)^2}{\alpha}
		=
		\int\frac{\dd[3]{p}}{(2\pi)^32\omega(\vb{p})}
		\omega(\vb{p})^2
		\abs{\alpha(\vb{p})}^2
		.
	\end{equation}
\end{lemma}
\begin{proof}
	For the first moment of the energy observable, we insert the definitions
	\begin{equation*}
		\begin{split}
			\expval{\hat{H}}{\alpha}
			&=
			\int\frac{\dd[3]{p}}{(2\pi)^3}
			\omega(\vb{p})
			\expval{\hat{a}^\dagger(\vb{p})\hat{a}(\vb{p})}{\alpha}
			\\
			&=
			\int\frac{\dd[3]{p}}{(2\pi)^3}
			\omega(\vb{p})
			\expval{\frac{\alpha(\vb{p})^*}{2\omega(\vb{p})}\frac{\alpha(\vb{p})}{2\omega(\vb{p})}}{\alpha}
			\\
			&=
			\int\frac{\dd[3]{p}}{(2\pi)^32\omega(\vb{p})}
			\omega(\vb{p})
			\abs{\alpha(\vb{p})}^2
		\end{split}
	\end{equation*}
	and use the eigenvalue equation. For the second moment, we again use the definitions and the eigenvalue equation
	\begin{equation*}
		\begin{split}
			\expval{\hat{H}^2}{\alpha}
			&=
			\int\frac{\dd[3]{p}_1}{(2\pi)^3}
			\int\frac{\dd[3]{p}_2}{(2\pi)^3}
			\omega(\vb{p}_1)
			\omega(\vb{p}_2)
			\expval{
				\hat{a}^\dagger(\vb{p}_1)
				\hat{a}(\vb{p}_1)
				\hat{a}^\dagger(\vb{p}_2)
				\hat{a}(\vb{p}_2)
			}{\alpha}
			\\
			&=
			\int\frac{\dd[3]{p}_1}{(2\pi)^3\sqrt{2\omega(\vb{p}_1)}}
			\int\frac{\dd[3]{p}_2}{(2\pi)^3\sqrt{2\omega(\vb{p}_2)}}
			\omega(\vb{p}_1)
			\omega(\vb{p}_2)
			\alpha(\vb{p}_1)^*
			\alpha(\vb{p}_2)
			\expval{
				\hat{a}(\vb{p}_1)
				\hat{a}^\dagger(\vb{p}_2)
			}{\alpha}
			\\
			&=
			\int\frac{\dd[3]{p}_1}{(2\pi)^3\sqrt{2\omega(\vb{p}_1)}}
			\int\frac{\dd[3]{p}_2}{(2\pi)^3\sqrt{2\omega(\vb{p}_2)}}
			\omega(\vb{p}_1)
			\omega(\vb{p}_2)			
			\alpha(\vb{p}_1)^*
			\alpha(\vb{p}_2)
			\\
			&\times
			\expval{
				(2\pi)^3
				\delta^{(3)}(\vb{p}_2-\vb{p}_1)
				+
				\hat{a}^\dagger(\vb{p}_2)
				\hat{a}(\vb{p}_1)
			}{\alpha}
			\\
			&=
			\int\frac{\dd[3]{p}}{(2\pi)^32\omega(\vb{p}_1)}
			\omega(\vb{p})^2
			\abs{\alpha(\vb{p})}^2
			+
			\left(
				\int\frac{\dd[3]{p}_1}{(2\pi)^32\omega(\vb{p})}
				\omega(\vb{p})
				\abs{\alpha(\vb{p})}^2
			\right)^2
		\end{split}
	\end{equation*}
\end{proof}

Although, the mean energy of the coherent state is the same as the mean energy of the single-particle number state, the spectrum $\alpha(\vb{p})$ of the coherent state is not bound.
\begin{lemma}\label{thm:qkg_coherent_state_number}
%{qkgcoherentnumber}	
	The mean number of particles is
	\begin{equation}
		\overline{n}
		=
		\expval{\hat{N}}{\alpha}
		=
		\int\frac{\dd[3]{p}}{(2\pi)^32\omega(\vb{p})}
		\abs{\alpha(\vb{p})}^2
	\end{equation}
	and the variance is
	\begin{equation}
		\expval{\left(\Delta\hat{N}\right)^2}{\alpha}
		=
		\int\frac{\dd[3]{p}}{(2\pi)^32\omega(\vb{p})}
		\abs{\alpha(\vb{p})}^2
		=
		\overline{n}
		,
	\end{equation}
	i.e., the particle number is Poisson distributed.
\end{lemma}
\begin{proof}
	The number observable is a special case of \Cref{thm:qkg_coherent_state_energy} for $\omega(\vb{p})=1$.
\end{proof}

\begin{lemma}\label{thm:qkg_coherent_state_number_state_inner_product}
%{qkgcoherentnumberinnerproduct}
	The inner product between an $n$-particle number state with spectrum $f(\vb{p})$ and a coherent state with spectrum $\alpha(\vb{p})$ is
	\begin{equation}
		\braket{n_f}{\alpha}
		=
		\frac{1}{\sqrt{n!}}
		\left(
			\int\frac{\dd[3]{p}}{(2\pi)^32\omega(\vb{p})}
			f(\vb{p})^*
			\alpha(\vb{p})
		\right)^n
		e^{-\overline{n}/2}
	\end{equation}
	where $\overline{n}$ is the mean particle number of the coherent state.
\end{lemma}
\begin{proof}
	\begin{equation*}
		\begin{split}
			\braket{n_f}{\alpha}
			&=
			\expval{
				\frac{1}{\sqrt{n!}}
				\hat\phi^+[f]^n
				e^{-\overline{n}/2}
				e^{\hat\phi^-[\alpha]}
			}{0}
			\\
			&=
			\frac{1}{\sqrt{n!}}
			e^{-\overline{n}/2}
			\sum_{m=0}^\infty
			\frac{1}{m!}
			\expval{
				\hat\phi^+[f]^n
				\hat\phi^-[\alpha]^m
			}{0}
			\\
			&=
			e^{-\overline{n}/2}
			\sum_{m=0}^\infty
			\frac{1}{\sqrt{m!}}
			\braket{n_f}{m_\alpha}
			\\
			&=
			e^{-\overline{n}/2}
			\sum_{m=0}^\infty
			\frac{1}{\sqrt{m!}}
			\delta_{nm}
			\braket{1_f}{1_\alpha}^m
			\\
			&=
			\frac{1}{\sqrt{n!}}
			e^{-\overline{n}/2}
			\braket{1_f}{1_\alpha}^n
		\end{split}
	\end{equation*}
	where we used \Cref{thm:qkg_number_state_inner_product}.\footnote{Technically, $\ket{1_\alpha}$ is not a true number state because $\alpha(\vb{p})$ is not constrained.}
\end{proof}

\begin{corollary}
	If $\alpha(\vb{p})=f(\vb{p})$ the former inner product reduces to
	\begin{equation}
		\braket{n_f}{\alpha}
		=
		\frac{1}{\sqrt{n!}}
		e^{-\overline{n}/2}
	\end{equation}
	which absolute squared is a Poisson distribution with unit variance.
\end{corollary}
\begin{corollary}
	If the product of the spectrums is equal to
	\begin{equation}
		f(\vb{p})
		\alpha(\vb{p})
		=
		2\omega(\vb{p})
		(2\pi)^3
		\delta(\vb{p}-\vb{k})
	\end{equation}
	the inner product reduces to
	\begin{equation}
		\braket{n_f}{\alpha}
		=
		\frac{\alpha^n}{\sqrt{n!}}
		e^{-\overline{n}/2}
	\end{equation}
	and we recover the Poisson distribution known from single-mode quantum optics
	\begin{equation}
		p_n
		=
		\abs{\braket{n_f}{\alpha}}^2
		=
		\frac{\overline{n}^n}{\sqrt{n!}}
		e^{-\overline{n}}
	\end{equation}
	where $\overline{n}=\abs{\alpha}^2$.
\end{corollary}
\begin{lemma}\label{thm:coherent_state_inner_product}
%{qkgcoherentinnerproduct}	
	Let $\ket{\alpha},\ket{\beta}$ be two coherent states, then their inner product is
	\begin{equation}
		\begin{split}
			\braket{\alpha}{\beta}
			&=
			\exp\left\{
				-
				\frac{1}{2}
				\comm{\hat\phi^+[\alpha]}{\hat\phi^-[\alpha]}
				+
				\comm{\hat\phi^+[\alpha]}{\hat\phi^-[\beta]}
				-
				\frac{1}{2}
				\comm{\hat\phi^+[\beta]}{\hat\phi^-[\beta]}
			\right\}
			\\
			&=
			\exp\left\{
				-
				\frac{1}{2}
				\int\frac{\dd[3]{p}}{(2\pi)^32\omega(\vb{p})}
				\left\{
					\abs{\alpha(\vb{p})}^2
					+
					\abs{\beta(\vb{p})}^2
					-
					2\alpha(\vb{p})\beta(\vb{p})^*
				\right\}
			\right\}
		\end{split}
	\end{equation}
\end{lemma}
\begin{proof}
	\begin{equation*}
		\begin{split}
			\braket{\alpha}{\beta}
			&=
			\expval{
				\hat{D}[\alpha]^\dagger
				\hat{D}[\beta]
			}{0}
			\\
			&=
			\expval{
				\hat{D}[-\alpha]
				\hat{D}[\beta]
			}{0}
			\\
			&=
			\expval{
				\hat{D}[\beta-\alpha]
			}{0}
			\exp\left\{
				-
				\frac{1}{2}
				\comm{\hat\phi^+[-\alpha]}{\hat\phi^-[\beta]}
				+
				\frac{1}{2}
				\comm{\hat\phi^+[\beta]}{\hat\phi^-[-\alpha]}
			\right\}
		\end{split}
	\end{equation*}
	where we used \Cref{thm:qkg_displacement_product}.
	Using \Cref{thm:qkg_coherent_state_number_state_inner_product} with $n=0$ reveals
	\begin{equation*}
		\expval{\hat{D}[\beta-\alpha]}{0}
		=
		\braket{0}{-\alpha+\beta}
		=
		\exp\left\{
			-
			\frac{1}{2}
			\comm{\hat\phi^+[-\alpha+\beta]}{\hat\phi^-[-\alpha+\beta]}
		\right\}
	\end{equation*}
	and therefore
	\begin{equation*}
		\begin{split}
			\braket{\alpha}{\beta}
			&=
			\exp\left\{
				-
				\frac{1}{2}
				\comm{\hat\phi^+[-\alpha+\beta]}{\hat\phi^-[-\alpha+\beta]}
				-
				\frac{1}{2}
				\comm{\hat\phi^+[-\alpha]}{\hat\phi^-[\beta]}
				+
				\frac{1}{2}
				\comm{\hat\phi^+[\beta]}{\hat\phi^-[-\alpha]}
			\right\}
			\\
			&=
			\exp\biggl\{
				-
				\frac{1}{2}
				\comm{\hat\phi^+[\alpha]}{\hat\phi^-[\alpha]}
				+
				\frac{1}{2}
				\comm{\hat\phi^+[\alpha]}{\hat\phi^-[\beta]}
				+
				\frac{1}{2}
				\comm{\hat\phi^+[\beta]}{\hat\phi^-[\alpha]}
				\\
				&\ \ \ \ \ \ \ \ \ \
				-
				\frac{1}{2}
				\comm{\hat\phi^+[\beta]}{\hat\phi^-[\beta]}
				+
				\frac{1}{2}
				\comm{\hat\phi^+[\alpha]}{\hat\phi^-[\beta]}
				-
				\frac{1}{2}
				\comm{\hat\phi^+[\beta]}{\hat\phi^-[\alpha]}
			\biggr\}
			\\
			&=
			\exp\left\{
				-
				\frac{1}{2}
				\comm{\hat\phi^+[\alpha]}{\hat\phi^-[\alpha]}
				+
				\comm{\hat\phi^+[\alpha]}{\hat\phi^-[\beta]}
				-
				\frac{1}{2}
				\comm{\hat\phi^+[\beta]}{\hat\phi^-[\beta]}
			\right\}
			\\
			&=
			\exp\left\{
				-
				\frac{1}{2}
				\int\frac{\dd[3]{p}}{(2\pi)^32\omega(\vb{p})}
				\left\{
					\abs{\alpha(\vb{p})}^2
					+
					\abs{\beta(\vb{p})}^2
					-
					2\alpha(\vb{p})\beta(\vb{p})^*
				\right\}
			\right\}
			.
		\end{split}
	\end{equation*}
\end{proof}

\begin{lemma}\label{thm:coherent_state_wave_function}
%{qkgcoherentwavefunction}	
	Let $\ket{\alpha}$ be a coherent state, then its coordinate wave function is
	\begin{equation}
		\psi(x^\mu)
		=
		\bra{0}
		\hat\phi(x^\mu)
		\ket{\alpha}
		=
		e^{-\overline{n}/2}
		\int\frac{\dd[3]{p}}{(2\pi)^32\omega(\vb{p})}
		\alpha(\vb{p})
		\eval{e^{-ip_\mu x^\mu}}_{p_0=\omega(\vb{p})}
		.
	\end{equation}
	The coherent state's coordinate wave function is equal to the single-particle number state's coordinate wave function with the difference that the wave function of the coherent state is suppressed by $e^{-\overline{n}/2}$ and the spectrum $\alpha(\vb{p})$ is not constrained.
\end{lemma}
\begin{proof}
	Writing $\hat\phi^+(y^\mu)$ as a smeared Klein-Gordon operator
	\begin{equation*}
		\begin{split}
			\hat\phi^+[\delta_x]
			&=
			\int\dd[4]{y}
			\delta(x^\mu-y^\mu)
			\hat\phi^+(y^\mu)
			\\
			&=
			\int\frac{\dd[3]{p}}{(2\pi)^3\sqrt{2\omega(\vb{p})}}
			\hat{a}(\vb{p})
			\left(
				\int\dd[4]{y}
				\delta^{(4)}(x^\mu-y^\mu)
				e^{+ip_\mu y^\mu}
			\right)_{p_0=\omega(\vb{p})}^*
			\\
			&=
			\int\frac{\dd[3]{p}}{(2\pi)^3\sqrt{2\omega(\vb{p})}}
			\hat{a}(\vb{p})
			\eval{e^{-ip_\mu x^\mu}}_{p_0=\omega(\vb{p})}
		\end{split}
	\end{equation*}
	with spectrum $f(\vb{p})=\eval{e^{-ip_\mu x^\mu}}_{p_0=\omega(\vb{p})}$, we can use \Cref{thm:qkg_coherent_state_number_state_inner_product} by writing
	\begin{equation*}
		\begin{split}
			\bra{0}
			\hat\phi^+(x^\mu)
			\ket{\alpha}
			=
			\bra{0}
			\hat\phi^+[\delta_x]
			\ket{\alpha}
			=
			\bra{1_{\delta_x}}
			\ket{\alpha}
			=
			e^{-\overline{n}/2}
			\int\frac{\dd[3]{p}}{(2\pi)^32\omega(\vb{p})}
			\alpha(\vb{p})
			\eval{e^{-ip_\mu x^\mu}}_{p_0=\omega(\vb{p})}
			.
		\end{split}
	\end{equation*}
\end{proof}

\begin{corollary}
	Results obtained for the single-particle number state's coordinate wave function, e.g., the group velocity (\Cref{thm:number_state_single_group_velocity}) and the localization (\Cref{thm:number_state_single_localization}), carry over to the coherent state after performing the replacement $f(\vb{p})=e^{-\overline{n}/2}\alpha(\vb{p})$.
\end{corollary}

\begin{definition}
	The positive and negative frequency momentum density operator are
	\begin{align}
		\hat\pi^+(x^\mu)
		&=
		\int\frac{\dd[3]{p}}{(2\pi)^3}
		\left(
			+
			i\sqrt{\frac{p_0}{2}}
		\right)
		\hat{a}^\dagger(\vb{p})
		\eval{e^{+ip_\mu x^\mu}}_{p_0=\omega(\vb{p})}
		\\
		\hat\pi^-(x^\mu)
		&=
		\int\frac{\dd[3]{p}}{(2\pi)^3}
		\left(
			-
			i\sqrt{\frac{p_0}{2}}
		\right)
		\hat{a}(\vb{p})
		\eval{e^{-ip_\mu x^\mu}}_{p_0=\omega(\vb{p})}
		.
	\end{align}
\end{definition}
\begin{corollary}
	The momentum density operator can be written as a sum of positive and negative frequency momentum density operators
	\begin{equation}
		\hat\pi(x^\mu)
		=
		\hat\pi^+(x^\mu)
		+
		\hat\pi^-(x^\mu)
		.
	\end{equation}
\end{corollary}
\begin{corollary}
	The positive and negative frequency momentum density operator are related by the Hermitian conjugate
	\begin{equation}
		\hat\pi^-(x^\mu)
		=
		\hat\pi^+(x^\mu)^\dagger
		.
	\end{equation}	
\end{corollary}

\begin{lemma}\label{thm:qkg_coherent_momentum_density_mean}
%{qkgcoherentmomentumdensitymean}
	The expected mean of the momentum density operator w.r.t. the coherent state is
	\begin{equation}
		\begin{split}
			\expval{\hat\pi(x^\mu)}{\alpha}
			&=
			\int\frac{\dd[3]{p}}{(2\pi)^3\sqrt{2\omega(\vb{p})}}
			\Im\left\{
				\alpha\left(p_0,\vb{p}\right)
				e^{-ip_\mu x^\mu}	
			\right\}_{p_0=\omega(\vb{p})}
			\\
			&=
			\int\frac{\dd[3]{p}}{(2\pi)^3\sqrt{2\omega(\vb{p})}}
			\Im\alpha\left(\omega(\vb{p}),\vb{p}\right)
			\cos\left(\omega(\vb{p})t-\vb{p}\vdot\vb{x}\right)
			\\
			&-
			\int\frac{\dd[3]{p}}{(2\pi)^3\sqrt{2\omega(\vb{p})}}
			\Re\alpha\left(\omega(\vb{p}),\vb{p}\right)
			\sin\left(\omega(\vb{p})t-\vb{p}\vdot\vb{x}\right)
			.
		\end{split}
	\end{equation}
\end{lemma}
\begin{proof}
	We decompose the momentum density operator
	\begin{equation*}
		\expval{\hat\pi(x^\mu)}{\alpha}
		=
		\expval{\hat\pi^-(x^\mu)}{\alpha}
		+
		\expval{\hat\pi^+(x^\mu)}{\alpha}
	\end{equation*}
	and evaluate the first term
	\begin{equation*}
		\begin{split}
			\expval{\hat\pi^-(x^\mu)}{\alpha}
			&=
			\int\frac{\dd[3]{p}}{(2\pi)^3}
			\left(
				-i
				\sqrt{\frac{\omega(\vb{p})}{2}}
			\right)
			\expval{\hat{a}(\vb{p})}{\alpha}
			\eval{e^{-ip_\mu x^\mu}}_{p_0=\omega(\vb{p})}
			\\
			&=
			\int\frac{\dd[3]{p}}{(2\pi)^3}
			\left(
				-i
				\sqrt{\frac{\omega(\vb{p})}{2}}
			\right)
			\frac{\alpha\left(\omega(\vb{p}),\vb{p}\right)}{\sqrt{2\omega(\vb{p})}}
			\eval{e^{-ip_\mu x^\mu}}_{p_0=\omega(\vb{p})}
			\\
			&=
			-
			\frac{i}{2}
			\int\frac{\dd[3]{p}}{(2\pi)^3}
			\alpha\left(\omega(\vb{p}),\vb{p}\right)
			\eval{e^{-ip_\mu x^\mu}}_{p_0=\omega(\vb{p})}
		\end{split}
	\end{equation*}
	and the second term is equal to the Hermitian conjugate, thus
	\begin{equation*}
		\begin{split}
			\expval{\hat\pi(x^\mu)}{\alpha}
			&=
			-
			\frac{i}{2}
			\int\frac{\dd[3]{p}}{(2\pi)^3}
			\alpha\left(\omega(\vb{p}),\vb{p}\right)
			\eval{e^{-ip_\mu x^\mu}}_{p_0=\omega(\vb{p})}
			+
			\frac{i}{2}
			\int\frac{\dd[3]{p}}{(2\pi)^3}
			\alpha\left(\omega(\vb{p}),\vb{p}\right)^*
			\eval{e^{+ip_\mu x^\mu}}_{p_0=\omega(\vb{p})}
			\\
			&=
			\frac{1}{2i}
			\int\frac{\dd[3]{p}}{(2\pi)^3}
			\left\{
				\alpha\left(p_0,\vb{p}\right)
				e^{-ip_\mu x^\mu}
				-
				\alpha\left(p_0,\vb{p}\right)^*
				e^{+ip_\mu x^\mu}
			\right\}_{p_0=\omega(\vb{p})}
			\\
			&=
			\int\frac{\dd[3]{p}}{(2\pi)^3}
			\Im\left\{
				\alpha\left(p_0,\vb{p}\right)
				e^{-ip_\mu x^\mu}			
			\right\}_{p_0=\omega(\vb{p})}
			.
		\end{split}
	\end{equation*}
\end{proof}

\begin{lemma}\label{thm:qkg_coherent_momentum_density_corr}
%{qkgcoherentmomentumdensitycorr}	
	\begin{equation}
		\begin{split}
			\expval{\hat\pi(x^\mu)\hat\pi(y^\mu)}{\alpha}
			&=
			\frac{1}{2}
			\Re\left\{
				\int\frac{\dd[3]{p}}{(2\pi)^3\sqrt{2\omega(\vb{p})}}
				\left(
					\alpha(p_0,\vb{p})
					e^{-ip_\mu x^\mu}
				\right)^*_{p_0=\omega(\vb{p})}
				\int\frac{\dd[3]{q}}{(2\pi)^3\sqrt{2\omega(\vb{q})}}
				\left(
					\alpha(q_0,\vb{q})
					e^{-iq_\mu y^\mu}
				\right)_{q_0=\omega(\vb{q})}
			\right\}
			\\
			&-
			\frac{1}{2}
			\Re\left\{
				\int\frac{\dd[3]{p}}{(2\pi)^3\sqrt{2\omega(\vb{p})}}
				\left(
					\alpha(p_0,\vb{p})
					e^{-ip_\mu x^\mu}
				\right)_{p_0=\omega(\vb{p})}
				\int\frac{\dd[3]{q}}{(2\pi)^3\sqrt{2\omega(\vb{q})}}
				\left(
					\alpha(q_0,\vb{q})
					e^{-iq_\mu y^\mu}
				\right)_{q_0=\omega(\vb{q})}
			\right\}
		\end{split}
	\end{equation}
	Expanding $\alpha$ into real and imaginary parts, the first term can be written
	\begin{equation*}
		\begin{split}
			&\
			\Re\left\{
				\int\frac{\dd[3]{p}}{(2\pi)^3\sqrt{2\omega(\vb{p})}}
				\left(
					\alpha(p_0,\vb{p})
					e^{-ip_\mu x^\mu}
				\right)^*_{p_0=\omega(\vb{p})}
				\int\frac{\dd[3]{q}}{(2\pi)^3\sqrt{2\omega(\vb{q})}}
				\left(
					\alpha(q_0,\vb{q})
					e^{-iq_\mu y^\mu}
				\right)_{q_0=\omega(\vb{q})}			
			\right\}
			\\
			=&\
			\int\frac{\dd[3]{p}}{(2\pi)^3\sqrt{2\omega(\vb{p})}}
			\int\frac{\dd[3]{q}}{(2\pi)^3\sqrt{2\omega(\vb{q})}}
			\left\{
				\Re\alpha(p_0,\vb{p})
				\Re\alpha(q_0,\vb{q})
				+
				\Im\alpha(p_0,\vb{p})
				\Im\alpha(q_0,\vb{q})
			\right\}
			\cos\left[
				p_\mu x^\mu
				-
				q_\mu y^\mu
			\right]_{\substack{p_0=\omega(\vb{p})\\q_0=\omega(\vb{q})}}
			\\
			-&\
			\int\frac{\dd[3]{p}}{(2\pi)^3\sqrt{2\omega(\vb{p})}}
			\int\frac{\dd[3]{q}}{(2\pi)^3\sqrt{2\omega(\vb{q})}}
			\left\{
				\Re\alpha(p_0,\vb{p})
				\Im\alpha(q_0,\vb{q})
				-
				\Im\alpha(p_0,\vb{p})
				\Re\alpha(q_0,\vb{q})
			\right\}
			\sin\left[
				p_\mu x^\mu
				-
				q_\mu y^\mu
			\right]_{\substack{p_0=\omega(\vb{p})\\q_0=\omega(\vb{q})}}
		\end{split}
	\end{equation*}
	and the second term
	\begin{equation*}
		\begin{split}
			&\
			\Re\left\{
				\int\frac{\dd[3]{p}}{(2\pi)^3\sqrt{2\omega(\vb{p})}}
				\left(
					\alpha(p_0,\vb{p})
					e^{-ip_\mu x^\mu}
				\right)_{p_0=\omega(\vb{p})}
				\int\frac{\dd[3]{q}}{(2\pi)^3\sqrt{2\omega(\vb{q})}}
				\left(
					\alpha(q_0,\vb{q})
					e^{-iq_\mu y^\mu}
				\right)_{q_0=\omega(\vb{q})}			
			\right\}
			\\
			=&\
			\int\frac{\dd[3]{p}}{(2\pi)^3\sqrt{2\omega(\vb{p})}}
			\int\frac{\dd[3]{q}}{(2\pi)^3\sqrt{2\omega(\vb{q})}}
			\left\{
				\Re\alpha(p_0,\vb{p})
				\Re\alpha(q_0,\vb{q})
				-
				\Im\alpha(p_0,\vb{p})
				\Im\alpha(q_0,\vb{q})
			\right\}
			\cos\left[
				p_\mu x^\mu
				+
				q_\mu y^\mu
			\right]_{\substack{p_0=\omega(\vb{p})\\q_0=\omega(\vb{q})}}
			\\
			-&\
			\int\frac{\dd[3]{p}}{(2\pi)^3\sqrt{2\omega(\vb{p})}}
			\int\frac{\dd[3]{q}}{(2\pi)^3\sqrt{2\omega(\vb{q})}}
			\left\{
				\Re\alpha(p_0,\vb{p})
				\Im\alpha(q_0,\vb{q})
				+
				\Im\alpha(p_0,\vb{p})
				\Re\alpha(q_0,\vb{q})
			\right\}
			\sin\left[
				p_\mu x^\mu
				+
				q_\mu y^\mu
			\right]_{\substack{p_0=\omega(\vb{p})\\q_0=\omega(\vb{q})}}
		\end{split}
	\end{equation*}
\end{lemma}
\begin{proof}
	We write the expectation value in terms of positive and negative frequency momentum density operators
	\begin{equation*}
		\expval{\hat\pi(x^\mu)\hat\pi(y^\mu)}{\alpha}
		=
		\expval{\hat\pi^+(x^\mu)\hat\pi^+(y^\mu)}{\alpha}
		+
		\expval{\hat\pi^+(x^\mu)\hat\pi^-(y^\mu)}{\alpha}
		+
		\text{h.c.}
	\end{equation*}
	and evaluate the first term
	\begin{equation*}
		\begin{split}
			\expval{\hat\pi^+(x^\mu)\hat\pi^+(y^\mu)}{\alpha}
			&=
			\int\frac{\dd[3]{p}}{(2\pi)^3}
			\left(
				+i
				\sqrt{\frac{\omega(\vb{p})}{2}}
			\right)
			\int\frac{\dd[3]{q}}{(2\pi)^3}
			\left(
				+i
				\sqrt{\frac{\omega(\vb{q})}{2}}
			\right)
			\\
			&\times
			\expval{\hat{a}^\dagger(\vb{p})\hat{a}^\dagger(\vb{q})}{\alpha}
			\eval{e^{+ip_\mu x^\mu}}_{p_0=\omega(\vb{p})}
			\eval{e^{+iq_\mu y^\mu}}_{q_0=\omega(\vb{q})}
			\\
			&=
			\int\frac{\dd[3]{p}}{(2\pi)^3}
			\left(
				+i
				\sqrt{\frac{\omega(\vb{p})}{2}}
			\right)
			\int\frac{\dd[3]{q}}{(2\pi)^3}
			\left(
				+i
				\sqrt{\frac{\omega(\vb{q})}{2}}
			\right)
			\\
			&\times
			\frac{\alpha\left(\omega(\vb{p}),\vb{p}\right)^*}{\sqrt{2\omega(\vb{p})}}
			\frac{\alpha\left(\omega(\vb{q}),\vb{q}\right)^*}{\sqrt{2\omega(\vb{q})}}
			\eval{e^{+ip_\mu x^\mu}}_{p_0=\omega(\vb{p})}
			\eval{e^{+iq_\mu y^\mu}}_{q_0=\omega(\vb{q})}
			\\
			&=
			\frac{i}{2}
			\int\frac{\dd[3]{p}}{(2\pi)^3}
			\left(
				\alpha(p_0,\vb{p})
				e^{-ip_\mu x^\mu}
			\right)^*_{p_0=\omega(\vb{p})}
			\frac{i}{2}
			\int\frac{\dd[3]{q}}{(2\pi)^3}
			\left(
				\alpha(q_0,\vb{q})
				e^{-iq_\mu y^\mu}
			\right)^*_{q_0=\omega(\vb{q})}
			\\
			&=
			-
			\frac{1}{4}
			\int\frac{\dd[3]{p}}{(2\pi)^3}
			\left(
				\alpha(p_0,\vb{p})
				e^{-ip_\mu x^\mu}
			\right)^*_{p_0=\omega(\vb{p})}
			\int\frac{\dd[3]{q}}{(2\pi)^3}
			\left(
				\alpha(q_0,\vb{q})
				e^{-iq_\mu y^\mu}
			\right)^*_{q_0=\omega(\vb{q})}
		\end{split}
	\end{equation*}
	by using the Hermitian conjugate of the eigenvalue equation \Cref{thm:qkg_coherent_state_eigenstate} to the left.
	For the second term, we find
	\begin{equation*}
		\begin{split}
			\expval{\hat\pi^+(x^\mu)\hat\pi^-(y^\mu)}{\alpha}
			&=
			\int\frac{\dd[3]{p}}{(2\pi)^3}
			\left(
				+i
				\sqrt{\frac{\omega(\vb{p})}{2}}
			\right)
			\int\frac{\dd[3]{q}}{(2\pi)^3}
			\left(
				-i
				\sqrt{\frac{\omega(\vb{q})}{2}}
			\right)
			\\
			&\times
			\expval{\hat{a}^\dagger(\vb{p})\hat{a}(\vb{q})}{\alpha}
			\eval{e^{+ip_\mu x^\mu}}_{p_0=\omega(\vb{p})}
			\eval{e^{-iq_\mu y^\mu}}_{q_0=\omega(\vb{q})}
			\\
			&=
			\int\frac{\dd[3]{p}}{(2\pi)^3}
			\left(
				+i
				\sqrt{\frac{\omega(\vb{p})}{2}}
			\right)
			\int\frac{\dd[3]{q}}{(2\pi)^3}
			\left(
				-i
				\sqrt{\frac{\omega(\vb{q})}{2}}
			\right)
			\\
			&\times
			\frac{\alpha\left(\omega(\vb{p}),\vb{p}\right)}{\sqrt{2\omega(\vb{p})}}
			\frac{\alpha\left(\omega(\vb{q}),\vb{q}\right)^*}{\sqrt{2\omega(\vb{q})}}
			\eval{e^{+ip_\mu x^\mu}}_{p_0=\omega(\vb{p})}
			\eval{e^{-iq_\mu y^\mu}}_{q_0=\omega(\vb{q})}
			\\
			&=
			\frac{i}{2}
			\int\frac{\dd[3]{p}}{(2\pi)^3}
			\left(
				\alpha(p_0,\vb{p})
				e^{-ip_\mu x^\mu}
			\right)^*_{p_0=\omega(\vb{p})}
			\left(
				-
				\frac{i}{2}
			\right)
			\int\frac{\dd[3]{q}}{(2\pi)^3}
			\left(
				\alpha(q_0,\vb{q})
				e^{-iq_\mu y^\mu}
			\right)_{q_0=\omega(\vb{q})}
			\\
			&=
			+
			\frac{1}{4}
			\int\frac{\dd[3]{p}}{(2\pi)^3}
			\left(
				\alpha(p_0,\vb{p})
				e^{-ip_\mu x^\mu}
			\right)^*_{p_0=\omega(\vb{p})}
			\int\frac{\dd[3]{q}}{(2\pi)^3}
			\left(
				\alpha(q_0,\vb{q})
				e^{-iq_\mu y^\mu}
			\right)_{q_0=\omega(\vb{q})}
			,
		\end{split}
	\end{equation*}
	thus in sum
	\begin{equation*}
		\begin{split}
			\expval{\hat\pi(x^\mu)\hat\pi(y^\mu)}{\alpha}
			&=
			\frac{1}{2}
			\Re\left\{
				\int\frac{\dd[3]{p}}{(2\pi)^3}
				\left(
					\alpha(p_0,\vb{p})
					e^{-ip_\mu x^\mu}
				\right)^*_{p_0=\omega(\vb{p})}
				\int\frac{\dd[3]{q}}{(2\pi)^3}
				\left(
					\alpha(q_0,\vb{q})
					e^{-iq_\mu y^\mu}
				\right)_{q_0=\omega(\vb{q})}			
			\right\}
			\\
			&-
			\frac{1}{2}
			\Re\left\{
				\int\frac{\dd[3]{p}}{(2\pi)^3}
				\left(
					\alpha(p_0,\vb{p})
					e^{-ip_\mu x^\mu}
				\right)^*_{p_0=\omega(\vb{p})}
				\int\frac{\dd[3]{q}}{(2\pi)^3}
				\left(
					\alpha(q_0,\vb{q})
					e^{-iq_\mu y^\mu}
				\right)^*_{q_0=\omega(\vb{q})}
			\right\}
		\end{split}
	\end{equation*}
\end{proof}

\begin{lemma}\label{thm:qkg_coherent_momentum_density_var}
%{qkgcoherentmomentumdensityvar}	
	The variance of the momentum density is
	\begin{equation}
		\begin{split}
			\expval{\left(\Delta\hat\pi(x^\mu)\right)^2}{\alpha}
			&=
			\frac{1}{2}
			\abs{
				\int\frac{\dd[3]{p}}{(2\pi)^3\sqrt{2\omega(\vb{p})}}
				\eval{
					\alpha(p_0,\vb{p})
					e^{-ip_\mu x^\mu}
				}_{p_0=\omega(\vb{p})}
			}^2
			\\
			&-
			\frac{1}{2}
			\Re\left\{
				\int\frac{\dd[3]{p}}{(2\pi)^3\sqrt{2\omega(\vb{p})}}
				\eval{
					\alpha(p_0,\vb{p})
					e^{-ip_\mu x^\mu}
				}_{p_0=\omega(\vb{p})}^2
			\right\}
		\end{split}
	\end{equation}
	or equivalently
	\begin{equation*}
		\begin{split}
			\expval{\left(\Delta\hat\pi(x^\mu)\right)^2}{\alpha}
		\end{split}
	\end{equation*}
\end{lemma}
\begin{proof}
	The second moment follows from the previous lemma
	\begin{equation*}
		\begin{split}
			\expval{\hat\pi(x^\mu)^2}{\alpha}
			&=
			\frac{1}{2}
			\Re\left\{
				\int\frac{\dd[3]{p}}{(2\pi)^3}
				\left(
					\alpha(p_0,\vb{p})
					e^{-ip_\mu x^\mu}
				\right)^*_{p_0=\omega(\vb{p})}
				\int\frac{\dd[3]{q}}{(2\pi)^3}
				\left(
					\alpha(q_0,\vb{q})
					e^{-iq_\mu x^\mu}
				\right)_{q_0=\omega(\vb{q})}			
			\right\}
			\\
			&-
			\frac{1}{2}
			\Re\left\{
				\int\frac{\dd[3]{p}}{(2\pi)^3}
				\left(
					\alpha(p_0,\vb{p})
					e^{-ip_\mu x^\mu}
				\right)_{p_0=\omega(\vb{p})}
				\int\frac{\dd[3]{q}}{(2\pi)^3}
				\left(
					\alpha(q_0,\vb{q})
					e^{-iq_\mu x^\mu}
				\right)_{q_0=\omega(\vb{q})}
			\right\}
			\\
			&=
			\frac{1}{2}
			\abs{
				\int\frac{\dd[3]{p}}{(2\pi)^3}
				\eval{
					\alpha(p_0,\vb{p})
					e^{-ip_\mu x^\mu}
				}_{p_0=\omega(\vb{p})}
			}^2
			-
			\frac{1}{2}
			\Re\left\{
				\int\frac{\dd[3]{p}}{(2\pi)^3}
				\eval{
					\alpha(p_0,\vb{p})
					e^{-ip_\mu x^\mu}
				}_{p_0=\omega(\vb{p})}
			\right\}^2
		\end{split}
	\end{equation*}
\end{proof}

\begin{lemma}
	\begin{equation}
		\begin{split}
			\expval{\hat\chi(\theta)}{\alpha}
			&=
			\int\frac{\dd[3]{p}}{(2\pi)^3\sqrt{2\omega(\vb{p})}}
			\Re\left\{
				\alpha(p_0,\vb{p})
				e^{-i\theta}
			\right\}_{p_0=\omega(\vb{p})}
			\\
			&=
			\int\frac{\dd[3]{p}}{(2\pi)^3\sqrt{2\omega(\vb{p})}}
			\Re\alpha\left(\omega(\vb{p}),\vb{p}\right)
			\cos\theta
			\\
			&+
			\int\frac{\dd[3]{p}}{(2\pi)^3\sqrt{2\omega(\vb{p})}}
			\Im\alpha\left(\omega(\vb{p}),\vb{p}\right)
			\sin\theta
		\end{split}
	\end{equation}
\end{lemma}
\begin{corollary}
	\begin{align}
		\expval{\hat{X}}{\alpha}
		&=
		\int\frac{\dd[3]{p}}{(2\pi)^3\sqrt{2\omega(\vb{p})}}
		\Re\alpha\left(\omega(\vb{p}),\vb{p}\right)
		\\
		\expval{\hat{P}}{\alpha}
		&=
		\int\frac{\dd[3]{p}}{(2\pi)^3\sqrt{2\omega(\vb{p})}}
		\Im\alpha\left(\omega(\vb{p}),\vb{p}\right)
	\end{align}
\end{corollary}
\begin{lemma}
	\begin{equation}
		\expval{\hat\chi(\theta)\hat\chi(\theta-\Delta\theta)}{\alpha}
	\end{equation}
\end{lemma}
\begin{proof}
	\begin{equation*}
		\begin{split}
			&\
			\expval{\hat\chi(\theta)\hat\chi(\theta-2\Delta\theta)}{\alpha}
			\\
			=&\
			\frac{1}{4}
			\int\frac{\dd[3]{p}}{(2\pi)^3}
			\int\frac{\dd[3]{q}}{(2\pi)^3}
			\expval{
				\left[
					\hat{a}^\dagger(\vb{p})
					e^{+i\theta}
					+
					\hat{a}(\vb{p})
					e^{-i\theta}
				\right]
				\left[
					\hat{a}^\dagger(\vb{q})
					e^{+i(\theta-2\Delta\theta)}
					+
					\hat{a}(\vb{q})
					e^{-i(\theta-2\Delta\theta)}
				\right]
			}{\alpha}
			\\
			=&\
			\frac{1}{4}
			\int\frac{\dd[3]{p}}{(2\pi)^3}
			\int\frac{\dd[3]{q}}{(2\pi)^3}
			\expval{
				\hat{a}^\dagger(\vb{p})
				\hat{a}^\dagger(\vb{q})
				e^{+2i(\theta-\Delta\theta)}
				+
				\hat{a}(\vb{p})
				\hat{a}(\vb{q})
				e^{-2i(\theta-\Delta\theta)}
				+
				\hat{a}(\vb{p})
				\hat{a}^\dagger(\vb{q})
				e^{-i\Delta\theta}
				+
				\hat{a}^\dagger(\vb{p})
				\hat{a}(\vb{q})
				e^{+i\Delta\theta}
			}{\alpha}
			\\
			=&\
			\frac{1}{2}
			\int\frac{\dd[3]{p}}{(2\pi)^3\sqrt{2\omega(\vb{p})}}
			\int\frac{\dd[3]{q}}{(2\pi)^3\sqrt{2\omega(\vb{q})}}
			\Re\left\{
				\alpha\left(\omega(\vb{p}),\vb{p}\right)
				\alpha\left(\omega(\vb{q}),\vb{q}\right)
				e^{-2i(\theta-\Delta\theta)}
			\right\}
			\\
			+&\
			\frac{1}{2}
			\int\frac{\dd[3]{p}}{(2\pi)^3\sqrt{2\omega(\vb{p})}}
			\int\frac{\dd[3]{q}}{(2\pi)^3\sqrt{2\omega(\vb{q})}}
			\Re\left[
				\alpha\left(\omega(\vb{p}),\vb{p}\right)
				\alpha\left(\omega(\vb{q}),\vb{q}\right)^*
			\right]
		\end{split}
	\end{equation*}
	We rewrite the integrand of the first term as
	\begin{equation*}
		\begin{split}
			\Re\left\{
				\alpha\left(\omega(\vb{p}),\vb{p}\right)
				\alpha\left(\omega(\vb{q}),\vb{q}\right)
				e^{-2i(\theta-\Delta\theta)}
			\right\}
			&=
			\Re\left\{
				\alpha\left(\omega(\vb{p}),\vb{p}\right)
				e^{-i(\theta-\Delta\theta)}
			\right\}
			\Re\left\{
				\alpha\left(\omega(\vb{q}),\vb{q}\right)
				e^{-i(\theta-\Delta\theta)}
			\right\}
			\\
			&-
			\Im\left\{
				\alpha\left(\omega(\vb{p}),\vb{p}\right)
				e^{-i(\theta-\Delta\theta)}
			\right\}
			\Im\left\{
				\alpha\left(\omega(\vb{q}),\vb{q}\right)
				e^{-i(\theta-\Delta\theta)}
			\right\}
		\end{split}
	\end{equation*}
	and noting that
	\begin{equation*}
		\Re\left\{
			\alpha\left(\omega(\vb{p}),\vb{p}\right)
			e^{-i(\theta+\pi/2)}
		\right\}
		=
		\Re\left\{
			(-i)
			\alpha\left(\omega(\vb{p}),\vb{p}\right)
			e^{-i\theta}
		\right\}
		=
		\Im\left\{
			\alpha\left(\omega(\vb{p}),\vb{p}\right)
			e^{-i\theta}
		\right\}		
	\end{equation*}
	we identify the first term with the expectation values
	\begin{equation*}
		\begin{split}
			&\
			\int\frac{\dd[3]{p}}{(2\pi)^3\sqrt{2\omega(\vb{p})}}
			\int\frac{\dd[3]{q}}{(2\pi)^3\sqrt{2\omega(\vb{q})}}
			\Re\left\{
				\alpha\left(\omega(\vb{p}),\vb{p}\right)
				\alpha\left(\omega(\vb{q}),\vb{q}\right)
				e^{-2i(\theta-\Delta\theta)}
			\right\}
			\\
			=&\
			\left(
				\int\frac{\dd[3]{p}}{(2\pi)^3\sqrt{2\omega(\vb{p})}}
				\Re\left\{
					\alpha\left(\omega(\vb{p}),\vb{p}\right)
					e^{-i(\theta-\Delta\theta)}
				\right\}
			\right)^2
			\\
			-&\
			\left(
				\int\frac{\dd[3]{p}}{(2\pi)^3\sqrt{2\omega(\vb{p})}}
				\Re\left\{
					\alpha\left(\omega(\vb{p}),\vb{p}\right)
					e^{-i(\theta-\Delta\theta+\pi/2)}
				\right\}
			\right)^2
			\\
			=&\
			\expval{\hat\chi\left(\theta-\Delta\theta\right)}{\alpha}^2
			-
			\expval{\hat\chi\left(\theta-\Delta\theta+\pi/2\right)}{\alpha}^2
		\end{split}
	\end{equation*}
	Similar, we find for the second term
	\begin{equation*}
		\begin{split}
			\frac{1}{2}
			\int\frac{\dd[3]{p}}{(2\pi)^3\sqrt{2\omega(\vb{p})}}
			\int\frac{\dd[3]{q}}{(2\pi)^3\sqrt{2\omega(\vb{q})}}
			\Re\left[
				\alpha\left(\omega(\vb{p}),\vb{p}\right)
				\alpha\left(\omega(\vb{q}),\vb{q}\right)^*
			\right]
			=
			\expval{\hat\chi(\theta)}{\alpha}^2
			+
			\expval{\hat\chi\left(\theta+\frac{\pi}{2}\right)}{\alpha}^2
		\end{split}
	\end{equation*}
	and in total
	\begin{equation*}
		\expval{\hat\chi(\theta)\hat\chi(\theta-2\Delta\theta)}{\alpha}
		=
		\frac{1}{2}
		\left[
			\expval{\hat\chi\left(\theta-\Delta\theta\right)}{\alpha}^2
			-
			\expval{\hat\chi\left(\theta-\Delta\theta+\pi/2\right)}{\alpha}^2
			-
			\expval{\hat\chi(\theta)}{\alpha}^2
			-
			\expval{\hat\chi\left(\theta+\frac{\pi}{2}\right)}{\alpha}^2
		\right]
	\end{equation*}
\end{proof}
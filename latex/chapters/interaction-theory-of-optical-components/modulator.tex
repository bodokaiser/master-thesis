\FloatBarrier
\section{Modulator}

Modulators are devices that change the properties of a periodic waveform, the carrier signal.
We are concerned with electro-optical modulators, i.e., modulators that change an optical carrier's waveform using the electro-optical effect.
In contrast to electro-optical modulators, acousto-optical modulators first transform an electric signal to an acoustic (mechanical) signal using a transducer.

Our setup uses the Mach-Zehnder modulator exclusively, but to understand the Mach-Zehnder modulator, we first need to understand the operating principle of an optical phase-shifter, in particular, the Pockels cell phase-shifter.

\FloatBarrier
\subsection{Pockels phase modulator}

The phase-shifter or phase modulator adds a phase of $\phi_2+\varphi+2(t)$ to a coherent state compared to a coherent state following the unmodulated path, see \Cref{fig:phase_shifter}.
The phase difference between the two coherent states is
\begin{align}
	\Delta\varphi(t)
	=
	\varphi_2(t)
	+
	\Delta\phi
	&&
	\Delta\phi
	=
	\phi_2
	-
	\phi_1
	.
\end{align}
It is common practice to only refer to the phase difference $\Delta\varphi(t)$ by defining an appropriate global phase reference or equivalently applying a global gauge transformation to remove the constant phase difference $\Delta\phi$.
If multiple phase modulators are present in parallel the constant phase difference $\Delta\phi$ becomes relevant but can be compensated electronically through the time-dependent part $\varphi_2(t)$.
\begin{figure}[htb]
    \centering
    \includegraphics{figures/circuitikz/phase-shifter}
    \caption{Block diagram of a time-dependent phase-shifter: An initial coherent state $\ket{\alpha(t)}$ is split with equal power into an upper and lower branch. The lower branch corresponds to the coherent state $\ket{\frac{1}{\sqrt{2}}\alpha(t)e^{i\phi_1}}$ which accumulated a constant phase of $\phi_1$ due to the splitting and the lower optical path length. The upper branch passes a variable phase-shifter driven by the electric signal $x(t)$ which adds a phase $\phi_2+\varphi_2(t)$ compared to the initial coherent state.}\label{fig:phase_shifter}
\end{figure}

In the following, we justify the postulated the phase relations between the in- and output states for a phase-shifter utilizing the Pockels effect in three steps:
First, we discuss the Pockels effect in a simplified setting where the Pockels effect is static, i.e., time-independent and the analysis is based on classical wave mechanics.
Second, we analyze the time-dependency which arises in continuous modulation but still in a classical setting.
Third, we use the previous result to justify why we can treat the modulation to be effectively static and summarize a quantum approach from the literature to derive the unitary transformation relating in- and output states.

From a purely phenomenological point of view, the Pockels effect - also known as the linear electro-optical effect - describes a linear change of the refractive index of a dielectric material in the presence of an external electric field $E$, i.e.,
\begin{equation}
	n(E)
	=
	n^{(0)}
	+
	n^{(1)}E
\end{equation}
wherein $n^{(0)}=n(0)$ is the refractive index without an external field and $n^{(1)}$ is a proportionality constant.
One way to create the external electric field $E$ is to place the Pockels dielectric inside a plate capacitor and apply a voltage $V$ to the plates.
Neglecting boundary effects, we find a homogeneous static electric field between the plates of $E=V/d$ where $d$ is the plate distance.
Such a configuration is known as Pockels cell and depicted in \Cref{fig:pockels_cell} where the field inside the Pockels cell is labeled $E_x(t)$.
\begin{figure}[htb]
    \centering
    \includegraphics{figures/tikz/pockels-cell}
    \caption{Pockels cell of length $l$ and thickness $d$ embedded in a waveguide with constant dielectric permittivity $\varepsilon_1$: The dielectric permittivity of the Pockels cell $\varepsilon_2(t)$ depends on the electric field $E_x(t)$ across the Pockels cell plates. The optical field $E_z(t)$ enters the Pockels cell from the left and leaves it to the right as $E^\prime_z(t)$.}\label{fig:pockels_cell}
\end{figure}
Let us first consider a monochromatic electromagnetic wave of frequency $\omega_0$ propagating through a dielectric of length $l$ with refractive index $n_1^{(0)}=\sqrt{\varepsilon_1}$.
Inside the dielectric, the wave propagates with phase velocity $c/n_1^{(0)}$ and takes $T_1=ln_1^{(0)}/c$ to pass through the dielectric.
Directly after the transit, the wave accumulated a total phase shift of
\begin{equation*}
	\phi_1
	=
	\omega_0T_1
	=
	2\pi\frac{n_1^{(0)}l}{\lambda_0}
\end{equation*}
where $\lambda_0=c/f$ is the vacuum wave length.
If we now consider the same monochromatic electromagnetic wave propagating through a Pockels cell of the same length $l$ but without applied voltage, the wave would accumulate a phase shift of
\begin{equation}
	\phi_2
	=
	\omega_0T_2
	=
	2\pi\frac{n_2^{(0)}l}{\lambda_0}
	.
\end{equation}
So even when there is no electric field inside the Pockels cell, we find a phase difference of
\begin{equation}
	\Delta\phi
	=
	\phi_2
	-
	\phi_1
	=
	2\pi\frac{n_2^{(0)}-n_1^{(0)}}{\lambda_0}l
\end{equation}
because of the different dielectric constants of the materials.
If we apply a voltage to the Pockels cell, the refractive index of the Pockels dielectric contributes a second phase shift of
\begin{equation}
	\varphi_2
	=
	2\pi\frac{n_2^{(1)}E}{\lambda_0}l
\end{equation}
and the total phase difference is $\varphi_2+\Delta\phi$.
Extending our discussion to wave packets in dispersive media, we need to replace the phase velocity $v_p$ with the group velocity $v_g$.
The group velocity can be expressed in terms of the refractive index via~\cite[p.~211]{Jackson2007}
\begin{equation}
	v_g(\omega_0)
	=
	\left[
		n
		+
		\omega
		\pdv{n}{\omega}
	\right]^{-1}_{\omega=\omega_0}
\end{equation}
where $\omega_0$ is the center frequency of the wave packet.
The concept of group velocity can be extended to quantum states, see, for instance, Ref.~\cite[p.~3]{Naumov2013}.
We refrain from a further discussion which would require specific knowledge of the wave packet's pulse shape and frequency-dependency of the refractive index and is best simulated using finite element methods.

Thus far, we have treated a static electric field in the Pockels cell.
We are now going to extend our model to allow for time-dependent electric field inside the Pockels cell and justify why we can treat the time-dependency of the modulation field to be effectively static.
An external electric field induces a dipole moment in the constituents of a dielectric.
The average dipole moment per volume is referred to the macroscopic polarization $\vb{P}$.
Assuming the dielectric to be a time-invariant system, i.e., to be memoryless, we can expand each component of the macroscopic polarization $P^i$ up to second-order in terms of electric susceptibility tensors $\chi$~\cite[p.~17]{Murti2014}
\begin{equation}
	P_i(t)
	=
	\int_{\mathbb{R}}\dd{t^\prime}
	\chi^{(1)}_{ij}(t-t^\prime)
	E^j(t^\prime)
	+
	\iint_{\mathbb{R}^2}\dd{t^\prime}\dd{t^{\prime\prime}}
	\chi^{(2)}_{ijk}(t-t^\prime,t-t^{\prime\prime})
	E^j(t^\prime)E^k(t^{\prime\prime})
\end{equation}
where causality demands $\chi^{(n)}(t-t^\prime)=0$ for $t^\prime>t$.
Inserting the Fourier representation, we find
\begin{align}
	P_i^{(1)}(t)
	&=
	\int_{\mathbb{R}}
	\frac{\dd{\omega_1}}{2\pi}
	\chi^{(1)}_{ij}(\omega_1)
	E^j(\omega_1)
	e^{+i\omega_1t}
	\\
	P_i^{(2)}(t)
	&=
	\iint_{\mathbb{R}^2}
	\frac{\dd{\omega_1}}{2\pi}
	\frac{\dd{\omega_2}}{2\pi}
	\chi^{(2)}_{ijk}(\omega_1,\omega_2)
	E^j(\omega_1)E^k(\omega_2)
	e^{+i(\omega_1+\omega_2)t}	
\end{align}
for the first two terms of the expansion.
The inverse Fourier transform yields the macroscopic polarization in the frequency domain~\cite[p.~1070]{Mandel1995} yields for the first term
\begin{equation}
	P_i^{(1)}(\omega)
	=
	\int\dd{\omega^\prime}
	\chi^{(1)}_{ij}(\omega^\prime)
	E^j(\omega^\prime)
	\delta^{(1)}(\omega-\omega^\prime)
	=
	\chi^{(1)}_{ij}(\omega)
	E^j(\omega)
\end{equation}
which we will later identify as part of the constant refractive index with $E^j$ being the optical field passing through the Pockels cell
For the second term we find
\begin{equation}
	\begin{split}
		P_i^{(2)}(\omega)
		&=
		\iint\frac{\dd{\omega^\prime}\dd{\omega^{\prime\prime}}}{2\pi}
		\chi^{(2)}_{ijk}(\omega^\prime,\omega^{\prime\prime})
		E^j(\omega^\prime)
		E^k(\omega^{\prime\prime})
		(2\pi)	
		\delta^{(1)}(\omega-\omega^\prime-\omega^{\prime\prime})
		\\
		&=
		\int\frac{\dd{\omega^\prime}}{2\pi}
		\chi^{(2)}_{ijk}(\omega-\omega^\prime,\omega^\prime)
		E^j(\omega-\omega^\prime)
		E^k(\omega^\prime)
	\end{split}
\end{equation}
where we will later identify $E^k$ to be the radio frequency field driving the Pockels cell.
The specific form indicates the presence of frequency sidebands by modulation.
Let's assume radio frequency field $E^k$ to have non-zero support, i.e., bandwidth, $\Delta\Omega$, then according to the mean value theorem
\begin{equation}
	\begin{split}
		P_i^{(2)}(\omega)
		&=
		\int_{\Delta\Omega}\frac{\dd{\omega^\prime}}{2\pi}
		\chi^{(2)}_{ijk}(\omega-\omega^\prime,\omega^\prime)
		E^j(\omega-\omega^\prime)
		E^k(\omega^\prime)
		\\
		&=
		\frac{\Delta\Omega}{2\pi}
		\chi^{(2)}_{ijk}(\omega-\Omega_0,\Omega_0)
		E^j(\omega-\Omega_0)
		E^k(\Omega_0)
	\end{split}
\end{equation}
where $\Omega_0$ is the mean frequency of $E^k$.
Restricting $P_i(\omega)$ to the optical domain, we have $\omega\gg\Omega_0$ which let's us Taylor expand $E^j(\omega-\Omega_0)$ around $\omega$, i.e.,
\begin{equation}
	P_i^{(2)}(\omega)
	\approx
	\frac{\Delta\Omega}{2\pi}
	\chi^{(2)}_{ijk}(\omega,\Omega_0)
	\left(
		E^j(\omega)
		+
		\pdv{E^j}{\omega}(\omega)\Omega_0
	\right)
	E^k(\Omega_0)
\end{equation}
where we assumed a flat response $\pdv{\chi_{ijk}^{(2)}}{\Omega_0}\approx0$.
Finally, we redefine the electric susceptibility tensor such that
\begin{equation}
	P_i(\omega)
	\approx
	\left(
		\tilde{\chi}^{(1)}_{ij}(\omega)
		+
		\tilde{\chi}^{(2)}_{ijk}(\omega)
		E^k(\Omega_0)
	\right)
	E^j(\omega)
\end{equation}
which is valid if $\omega$ is an optical frequency.
The dielectric permittivity tensor $\varepsilon_{ij}$ is defined implicitly through the displacement field
\begin{equation}
	D_i(\omega)
	=
	E_i(\omega)
	+
	P_i(\omega)
	=
	\varepsilon_{ij}(\omega)
	E^j(\omega)
\end{equation}
and we identify by comparison the dielectric permittivity tensor to relate to the electric susceptibility by
\begin{equation}
	\varepsilon_{ij}(\omega)
	=
	1
	+
	\tilde{\chi}^{(1)}_{ij}(\omega)
	+
	\tilde{\chi}^{(2)}_{ijk}(\omega)
	E^k(\Omega_0)
	.
\end{equation}
The refractive index tensor for a dielectric, non-magnetic and non-chiral medium is given after a series expansion by~\cite{Rerat2020}
\begin{equation}
	n_{ij}(\omega)
	=
	\sqrt{\varepsilon_{ij}(\omega)}
	\approx
	n^{(0)}_{ij}(\omega)
	+
	n^{(1)}_{ijk}(\omega)
	E^k(\Omega_0)
\end{equation}
where we defined the refractive index tensors
\begin{align}
	n^{(0)}_{ij}(\omega)
	=
	\sqrt{1+\tilde{\chi}^{(1)}_{ij}(\omega)}
	&&
	n^{(1)}_{ijk}(\omega)
	=
	\frac{\tilde{\chi}^{(2)}_{ijk}(\omega)
	E^k(\Omega_0)}{2\sqrt{1+\tilde{\chi}^{(1)}_{ij}(\omega)}}
\end{align}
which recovers the linear electro-optical effect we first discussed in the phenomenological approach.
For the Pockels cell depicted in \Cref{fig:pockels_cell}, the refractive index relevant for the optical field takes the explicit form
\begin{equation}
	n_{zz}(\omega)
	\approx
	n^{(0)}_{zz}(\omega)
	+
	n^{(1)}_{zzx}(\omega)
	E^x(\Omega_0)
	.
\end{equation}
For practical Pockels media, e.g., Lithium-Niobate, the tensorial \gls{dof} can be substantially reduced by considering crystal symmetries, see, for instance, Ref.~\cite[p.~237]{Yariv1984}.

We derived the relation between the macroscopic polarization, response and refractive index and justified why we can treat the modulation field to be effectively static compared to the fast oscillating optical field.
For the last step where we summarize the unitary transformation associated with a Pockels phase modulator, we assume the dielectric to be constant on the time scale of the electrical field.

\FloatBarrier
\subsection{Mach-Zehnder (amplitude)}

\Cref{fig:mach_zehnder_modulator} shows an embodiment of a \gls{mzm} using free-ray components comprising two beam splitters BS1 and BS2, two mirrors M1 and M2, and two variable phases shifters with phase shift $+\varphi/2$ respective $-\varphi/2$ (push-pull configuration).
Beam splitter BS1 combines and distributes the input fields $E_1,E_2$ equally among the interferometer arms.
In the upper arm, the beam is phase-shifted by $+\varphi/2$ before being directed with mirror M1 onto BS2.
In the lower arm, the beam is directed with mirror M2 into a phase-shifter with phase-shift $-\varphi/2$ and then directed into the other input of BS2.
Beam splitter BS2 recombines both inputs into outputs $E_1^\prime,E_2^\prime$.
\begin{figure}[htb]
    \centering
    \includegraphics{figures/pstricks/mach-zehnder-interferometer}
    \caption{Embodiment of the \gls{mzm} using free-ray components. The optical inputs $E_1,E_2$ are split into an upper and lower beam path by beam splitter BS1. The phase of each arm is modulated by $\varphi_1$ respective $\varphi_2$ and then recombined by beam splitter BS2 which outputs $E_1^\prime,E_2^\prime$.}\label{fig:mach_zehnder_modulator}
\end{figure}
We found in \Cref{sec:beam_splitter} that a real beam splitter is described by the scattering parameters of a complex $4\times 4$ matrix.
Assuming, an ideal balanced beam splitter, we still have three \gls{dof} per beam splitter.
To make any progress, we assume an ideal symmetric beam splitter described by \cref{eq:beam_splitter_unitary_transform_balanced_qo}, and that there is no phase difference between the interferometer arms.
In this case, the complex output fields $\mathcal{E}_1^\prime,\mathcal{E}_2^\prime$ relate to the complex input fields $\mathcal{E}_1,\mathcal{E}_2$ via
\begin{equation}
    \begin{pmatrix}
        \mathcal{E}_1^\prime
        \\
        \mathcal{E}_2^\prime
    \end{pmatrix}
    =
    \frac{1}{\sqrt{2}}
    \begin{pmatrix}
        i & 1
        \\
        1 & i
    \end{pmatrix}
    \begin{pmatrix}
        e^{i\pi} & 0
        \\
        0 & e^{-i\varphi/2}
    \end{pmatrix}
        \begin{pmatrix}
        e^{+i\varphi/2} & 0
        \\
        0 & e^{i\pi}
    \end{pmatrix}
    \frac{1}{\sqrt{2}}
    \begin{pmatrix}
        1 & i
        \\
        i & 1
    \end{pmatrix}
    \begin{pmatrix}
        \mathcal{E}_1
        \\
        \mathcal{E}_2
    \end{pmatrix}
    \label{eq:mzm_transform}
\end{equation}
which can be understood as applying the linear transformations corresponding to the optical instruments from right to left.\footnote{The rows of the BS2 matrix are exchanged to keep consistent with the labels.}
Performing the matrix multiplication, the transformation linking complex input and output fields simplifies to
\begin{align}
    U_\text{MZM}(\varphi)
    &=
    \frac{1}{2}
    \begin{pmatrix}
        i & 1
        \\
        1 & i
    \end{pmatrix}
    \begin{pmatrix}
        ie^{+i\varphi/2} & 0
        \\
        0 & ie^{-i\varphi/2}
    \end{pmatrix}
    \begin{pmatrix}
        1 & i
        \\
        i & 1
    \end{pmatrix}
    \\
    &=
    \frac{i}{2}
    \begin{pmatrix}
        i & 1
        \\
        1 & i
    \end{pmatrix}
    \begin{pmatrix}
        e^{+i\varphi/2} & ie^{+i\varphi/2}
        \\
        ie^{-i\varphi/2} & e^{-i\varphi/2}
    \end{pmatrix}
    \\
    &=
    \frac{i}{2}
    \begin{pmatrix}
        ie^{+i\varphi/2}+ie^{-i\varphi/2} & -e^{+i\varphi/2}+e^{-i\varphi/2}
        \\
        e^{+i\varphi/2}-e^{-i\varphi/2} & ie^{+i\varphi/2}+ie^{-i\varphi/2}
    \end{pmatrix}
    \\
    &=
    -
    \begin{pmatrix}
        \cos(\varphi/2) & -\sin(\varphi/2)
        \\
        \sin(\varphi/2) & \cos(\varphi/2)
    \end{pmatrix}
    \label{eq:mzm_rotation},
\end{align}
resembling a rotation, i.e., the \gls{mzm} effectively mixes the input fields by $\varphi$.

In a more general treatment only assuming the \gls{mzm} to be lossless, every component is represented by a unitary transformation.
As the unitary transformations represent a group $U(2)$, the multiplication is closed in the sense that the result of a multiplication is also in the group $U(2)$.
From our discussion of the beam splitter, \cref{eq:beam_splitter_unitary_transform}, we know that an element of $U(2)$ can be parametrized by
\begin{equation}
    e^{i\Lambda}
    \begin{pmatrix}
        e^{+i\Phi} & 0
        \\
        0 & e^{-i\Phi}
    \end{pmatrix}
    \begin{pmatrix}
        \cos\Theta & \sin\Theta
        \\
        -\sin\Theta & \cos\Theta
    \end{pmatrix}
    \begin{pmatrix}
        e^{+i\Psi} & 0
        \\
        0 & e^{-i\Psi}
    \end{pmatrix}
    \in U(2)
\end{equation}
which compared to \cref{eq:mzm_rotation} is the same for $\Lambda=\pi,\Theta=\varphi/2,\Psi=0=\Phi$.
The parameter choice $\Psi=\Phi$ highlights that our model makes very strongs assumptions regarding the phase properties of the \gls{mzm} which are not realistic.
In fact, finding an operating point of the \gls{mzm} where the phase properties are well-controlled involves quite some engineering.

As our simplified transformation of \cref{eq:mzm_rotation} is not able to explain the sensible phase dependence of the output fields, we are going to neglect the global phase of $-1=e^{i\pi}$ as it does not contribute to the following insights.

\subsubsection{Single-mode and classic}

So far, we assumed non-zero input fields which will become important for the following quantum treatment.
Yet, the purpose of the \gls{mzm} is to perform amplitude modulation of a single input.
Setting $E_2=0,E=E_1,E^\prime=E_2^\prime$, the complex output fields become
\begin{equation}
    \begin{pmatrix}
        \mathcal{E}_1^\prime
        \\
        \mathcal{E}_2^\prime
    \end{pmatrix}
    =
    \mathcal{E}
    \begin{pmatrix}
        \cos(\varphi/2)
        \\
        \sin(\varphi/2)
    \end{pmatrix}
    \label{eq:mzm_classic_sm}.
\end{equation}
One of the outputs, e.g., $E_1^\prime$, is usually monitored with a photodiode as part of a feedback loop to find an operating point of the phases.
The other field is further processed or transmitted.

One further note needs to be done on the modulation: \cref{eq:mzm_classic_sm} says that we have a non-linear modulation with respect to $\varphi$.
Therefore, we perform a small-angle approximation near the operating point of our \gls{mzm} which gives us
\begin{equation}
    E^\prime(t)
    =
    E(t)\sin(\varphi)
    =
    E(t)\varphi+\mathcal{O}(\varphi^2)
\end{equation}
i.e., an amplitude modulation of the initial electric field input $E(t)$.

\subsubsection{Single-mode and quantum}

Our discussion of the \gls{mzm} so far was limited to the classical case.
For the generalization to the quantum case, we need to replace the complex electrical field amplitudes $\mathcal{E}$ with the single-mode annihilation operator $\hat{a}$ in \cref{eq:mzm_transform}
\begin{equation}
    \begin{pmatrix}
        \hat{a}_1^\prime
        \\
        \hat{a}_2^\prime
    \end{pmatrix}
    =
    \begin{pmatrix}
        \cos(\varphi/2) & -\sin(\varphi/2)
        \\
        \sin(\varphi/2) & \cos(\varphi/2)
    \end{pmatrix}
    \begin{pmatrix}
        \hat{a}_1
        \\
        \hat{a}_2
    \end{pmatrix}
    \label{eq:mzm_operator_transform_sm}
\end{equation}
where it is strictly necessary to include both in- and outputs to further satisfy the commutation relation of the transformed operators $\hat{a}^\prime$.\footnote{If we drop one of the operators, we would find $\comm{\hat{a}^\prime}{\left(\hat{a}^\prime\right)^\dagger}=\pm\sin(\varphi/2)\cos(\varphi/2)\comm{\hat{a}}{\hat{a}^\dagger}<1$.}

The quantum equivalent to a non-zero field with a zero field input is the tensor product of a coherent and the vacuum state $\ket{\alpha,0}$.
To obtain the corresponding output state, we need to invert \cref{eq:mzm_operator_transform_sm} by noting that $U_\text{MZM}\in O(2)$ is an orthogonal transform for which the inverse is simply the transpose
\begin{equation}
    \begin{pmatrix}
        \hat{a}_1
        \\
        \hat{a}_2
    \end{pmatrix}
    =
    \begin{pmatrix}
        \cos(\varphi/2) & \sin(\varphi/2)
        \\
        -\sin(\varphi/2) & \cos(\varphi/2)
    \end{pmatrix}
    \begin{pmatrix}
        \hat{a}_1^\prime
        \\
        \hat{a}_2^\prime
    \end{pmatrix}
    =
    \begin{pmatrix}
        \hat{a}_1^\prime\cos(\varphi/2) + \hat{a}_2^\prime\sin(\varphi/2)
        \\
       -\hat{a}_1^\prime\sin(\varphi/2) + \hat{a}_2^\prime\cos(\varphi/2)
    \end{pmatrix}
\label{eq:mzm_operator_transform_inverse_sm}.
\end{equation}
Now, we can find the output state by replacing the annihilation and creation operators in the displacement operator describing the input state~\cite[p.~134]{Haroche2006}
\begin{align}
    \hat{D}_1(\alpha)
    =&
    \exp\left\{\alpha\hat{a}_1^\dagger-\alpha^*\hat{a}_1\right\}
    \\
    =&
    \exp\biggl\{
        \alpha\left[(\hat{a}_1^\prime)^\dagger\cos(\varphi/2) + (\hat{a}_2^\prime)^\dagger\sin(\varphi/2)\right]
        \\
        -&
        \alpha^*\left[\hat{a}_1^\prime\cos(\varphi/2) + \hat{a}_2^\prime\sin(\varphi/2)\right]
    \biggr\}
    \\
    =&
    \exp\left\{
        \left(\alpha\cos(\varphi/2)\right)(\hat{a}_1^\prime)^\dagger
        -
        \left(\alpha\cos(\varphi/2)\right)^*\hat{a}_1^\prime
    \right\}
    \\
    &\exp\left\{
        \left(\alpha\sin(\varphi/2)\right)(\hat{a}_2^\prime)^\dagger
        -
        \left(\alpha\sin(\varphi/2)\right)^*\hat{a}_2^\prime
    \right\}
    \\
    =&
    \hat{D}^\prime_1\left(\alpha\sin(\varphi/2)\right)
    \hat{D}^\prime_2\left(\alpha\cos(\varphi/2)\right)
\end{align}
and thereby express the output state via
\begin{align}
    \ket{\alpha,0}
    &=
    \hat{D}_1(\alpha)
    \ket{0,0}
    \\
    &=
    \hat{D}^\prime_1\left(\alpha\sin(\varphi/2)\right)
    \hat{D}^\prime_2\left(\alpha\cos(\varphi/2)\right)
    \ket{0,0}
    \\
    &=
    \ket{\alpha\sin(\varphi/2),\alpha\cos(\varphi/2)}
    .
\end{align}
We recognize the output state as the tensor product of two coherent states.

In \cref{eq:operator_number_coherent_mom1_sm} we gave the expectation value of a general coherent state $\ket{\alpha}$, substituting the output states we find
\begin{align}
    \expval{\alpha\sin(\varphi/2)}{\hat{E}(t)}
    &=
    E(0)\sin(\varphi/2)
    \\
    \expval{\alpha\cos(\varphi/2)}{\hat{E}(t)}
    &=
    E(0)\cos(\varphi/2)
\end{align}
in agreement with the classical prediction.
In distinction to the classical prediction, the coherent output states have an intrinsic uncertainty of the quantum vacuum noise $\mathcal{E}_0$.
Classically, we would expect the beam splitter to also divide the uncertainty but this is in contradiction to the observation that a coherent state always has uncertainty $\mathcal{E}_0$.
Shapiro~\cite{Shapiro2009} offers a resolution to this puzzle: the missing noise is collected by the noise of the second vacuum input which is non-zero.

\subsubsection{Continuous-mode and classic}

For the continuous-mode description of the \gls{mzm}, we need to consider the spectrum of the input fields
\begin{equation}
    \mathcal{E}(t)
    =
    \int\dd{\omega}\mathcal{E}(\omega)e^{i\omega t}
\end{equation}
In general, the parameters of the unitary transformation describing the beam splitter are frequency-dependent
\begin{equation}
    \begin{pmatrix}
        \mathcal{E}_1^\prime(\omega)
        \\
        \mathcal{E}_2^\prime(\omega)
    \end{pmatrix}
    =
    e^{i\Lambda(\omega)}
    \begin{pmatrix}
        e^{+i\Phi(\omega)} & 0
        \\
        0 & e^{-i\Phi(\omega)}
    \end{pmatrix}
    \begin{pmatrix}
        \cos\Theta(\omega) & \sin\Theta(\omega)
        \\
        -\sin\Theta(\omega) & \cos\Theta(\omega)
    \end{pmatrix}
    \begin{pmatrix}
        e^{+i\Psi(\omega)} & 0
        \\
        0 & e^{-i\Psi(\omega)}
    \end{pmatrix}
    \begin{pmatrix}
        \mathcal{E}_1(\omega)
        \\
        \mathcal{E}_2(\omega)
    \end{pmatrix}
\end{equation}
Neglecting phases this simplifies to
\begin{equation}
    \begin{pmatrix}
        \mathcal{E}_1^\prime(\omega)
        \\
        \mathcal{E}_2^\prime(\omega)
    \end{pmatrix}
    =
    \begin{pmatrix}
        \cos\Theta(\omega) & \sin\Theta(\omega)
        \\
        -\sin\Theta(\omega) & \cos\Theta(\omega)
    \end{pmatrix}
    \begin{pmatrix}
        \mathcal{E}_1(\omega)
        \\
        \mathcal{E}_2(\omega)
    \end{pmatrix}
\end{equation}
and in the time-domain
\begin{equation}
    \begin{pmatrix}
        \mathcal{E}_1^\prime(t)
        \\
        \mathcal{E}_2^\prime(t)
    \end{pmatrix}
    =
    \int\dd{\omega}e^{i\omega t}
    \begin{pmatrix}
        \cos\Theta(\omega) & \sin\Theta(\omega)
        \\
        -\sin\Theta(\omega) & \cos\Theta(\omega)
    \end{pmatrix}
    \begin{pmatrix}
        \mathcal{E}_1(\omega)
        \\
        \mathcal{E}_2(\omega)
    \end{pmatrix}
\end{equation}

\subsubsection{Continuous-mode and quantum}

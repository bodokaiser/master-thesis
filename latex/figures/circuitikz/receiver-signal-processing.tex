\documentclass[tikz]{standalone}

\usepackage{amsmath}
\usepackage{unicode-math}
\usepackage{mathtools}
\usepackage{derivative}

\setmainfont{Stix Two Text}
\setmathfont{Stix Two Math}

\usetikzlibrary{arrows.meta,fit,positioning}

\renewcommand{\familydefault}{\sfdefault}

% prefix equation numbers with section number
\numberwithin{equation}{section}

\DeclarePairedDelimiter{\ceil}{\lceil}{\rceil}
\DeclarePairedDelimiter{\floor}{\lfloor}{\rfloor}
\DeclarePairedDelimiter{\abs}{\lvert}{\rvert}
\DeclarePairedDelimiter{\norm}{\lVert}{\rVert}
\DeclarePairedDelimiter{\bra}{\langle}{\rvert}
\DeclarePairedDelimiter{\ket}{\lvert}{\rangle}
\DeclarePairedDelimiter{\expval}{\langle}{\rangle}
\DeclarePairedDelimiter{\norder}{\mathcolon}{\mathcolon}
\DeclarePairedDelimiter{\anorder}{\typecolon}{\typecolon}
	
\newcommand{\laplace}{\mbfnabla^2}
\newcommand{\trans}{{\scriptscriptstyle\mathsf{T}}}

\newcommand{\vdot}{\cdot}
\newcommand{\vcross}{\vectimes}
\newcommand{\vb}[1]{\symbfup{#1}}
\newcommand{\vu}[1]{\hat{\vb{#1}}}
\newcommand*\dd[2][\relax]{\mathop{\ifx\relax#1\odif{#2}\else \odif[order={#1}]{#2}\fi\,}}

\newcommand{\vacuum}{\ket*{\vb{0}}}

\DeclareMathOperator{\trace}{Tr}
\DeclareMathOperator{\sinc}{sinc}

\AtBeginDocument{
	\let\Re\relax
	\let\Im\relax
	\DeclareMathOperator{\Re}{Re}
	\DeclareMathOperator{\Im}{Im}

	\renewcommand{\div}{\mathop{\mbfnabla\vdot}}
	\newcommand{\curl}{\mathop{\mbfnabla\vectimes}}
}

\DeclarePairedDelimiterX{\comm}[2]{[}{]}{#1,#2}

\DeclarePairedDelimiterX{\braket}[2]{\langle}{\rangle}{#1\delimsize\vert#2}
\DeclarePairedDelimiterX{\ketbra}[1]{\lvert}{\rvert}{#1\rangle\delimsize\langle#1}



\begin{document}
	\begin{tikzpicture}[
		line width=1pt,
		node distance=20pt,
	]
		\coordinate (in);
		\node (split) [twoportsplitshape, right=of in, circuitikz/t1={$.5$}, circuitikz/t2={$.5$}] {};
		\node (phase split) [twoportsplitshape, circuitikz/t1={\ \SI{90}{\degree}}, circuitikz/t2={\SI{0}{\degree}}, right=of split)] {};
		\node (osc) [oscillator, right=of phase split] {};
		\node (mixerx) [mixer, above=of phase split] {};
		\node (mixerp) [mixer, below=of phase split] {};
		\node (adder) [adder, right=of osc] {};
		\node (lp) [lowpassshape, right=of adder] {};
		\node (adc) [adcshape, right=of lp] {};
		\node (dc) [mixer, right=of adc] {};
		\node (ds1) [twoportshape, t={$\downarrow l$}, right=of dc] {};
		\node (rrc) [twoportshape, t={RRC}, right=of ds1] {};
		\node (ds2) [twoportshape, t={$\downarrow k$}, right=of rrc] {};
		\coordinate[right=of ds2] (out);
		
		\draw (in) ++(-0.5,0) node[above] {$\Re\{\alpha(t)e^{-i\omega_ct}\}$};
		\draw (osc.north) node[above, yshift=2pt] {$\sin(\omega_l t)$};

		\draw[-Latex] (in) -- (split);
		\draw[-Latex] (osc) -- (phase split);
		\draw[-Latex] (adder) -- (lp);
		\draw[-Latex] (lp) -- (adc);
		\draw[-Latex] (adc) -- (dc);
		\draw[-Latex] (dc) -- (ds1);
		\draw[-Latex] (ds1) -- (rrc);
		\draw[-Latex] (rrc) -- (ds2);
		\draw[-Latex] (ds2) -- (out);
		
		\draw[-Latex] (split.north) -- (split.north|-mixerx.west) -- (mixerx);
		\draw[-Latex] (phase split) -- (mixerx);
		\draw[-Latex] (mixerx.east) -- (mixerx.east-|adder.north)  -- (adder);
				
		\draw[-Latex] (split.south) -- (split.south|-mixerp.west) -- (mixerp);
		\draw[-Latex] (phase split) -- (mixerp);
		\draw[-Latex] (mixerp.east) -- (mixerp.east-|adder.south) -- (adder);
	\end{tikzpicture}
\end{document}

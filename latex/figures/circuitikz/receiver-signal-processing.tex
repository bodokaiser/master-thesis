\documentclass[tikz]{standalone}

\usepackage{amsmath}
\usepackage{unicode-math}
\usepackage{mathtools}
\usepackage{derivative}
\usepackage{circuitikz}
\usepackage{siunitx}

\setmainfont{Stix Two Text}
\setmathfont{Stix Two Math}

\renewcommand{\familydefault}{\sfdefault}

% approximately proportional to symbol
% https://tex.stackexchange.com/questions/33538/how-to-get-an-approximately-proportional-to-symbol
\def\app#1#2{%
    \mathrel{%
        \setbox0=\hbox{$#1\sim$}%
        \setbox2=\hbox{%
            \rlap{\hbox{$#1\propto$}}%
            \lower1.1\ht0\box0%
        }%
        \raise0.25\ht2\box2%
    }%
}
\def\approxprop{\mathpalette\app\relax}

% overwrite real and imaginary part operators
\let\Re\undefined
\let\Im\undefined
\DeclareMathOperator{\Re}{\operatorname{Re}}
\DeclareMathOperator{\Im}{\operatorname{Im}}
% other functions
\DeclareMathOperator{\sinc}{\operatorname{sinc}}

% transpose
% https://tex.stackexchange.com/questions/403104/small-caps-mathsf-font-for-writing-transpose-of-a-matrix
\newcommand{\trans}{\scriptscriptstyle\mathsf{T}}

% theorems
\newtheorem{theorem}{Theorem}[section]
\newtheorem{lemma}[theorem]{Lemma}
\newtheorem{corollary}[theorem]{Corollary}
\theoremstyle{definition}
\newtheorem{definition}{Definition}[section]
\newtheorem{conjecture}{Conjecture}[section]
\newtheorem{example}{Example}[section]
\theoremstyle{remark}
\newtheorem*{remark}{Remark}

\ctikzset{bipoles/thickness=1}

\usetikzlibrary{positioning}


\begin{document}
	\begin{tikzpicture}[
		line width=1pt,
		node distance=20pt,
	]
		\coordinate (in);
		\node (split) [twoportsplitshape, right=of in, circuitikz/t1={$.5$}, circuitikz/t2={$.5$}] {};
		\node (phase split) [twoportsplitshape, circuitikz/t1={\ \SI{90}{\degree}}, circuitikz/t2={\SI{0}{\degree}}, right=of split)] {};
		\node (osc) [oscillator, right=of phase split] {};
		\node (mixerx) [mixer, above=of phase split] {};
		\node (mixerp) [mixer, below=of phase split] {};
		\node (adder) [adder, right=of osc] {};
		\node (lp) [lowpassshape, right=of adder] {};
		\node (adc) [adcshape, right=of lp] {};
		\node (dc) [mixer, right=of adc] {};
		\node (ds1) [twoportshape, t={$\downarrow l$}, right=of dc] {};
		\node (rrc) [twoportshape, t={RRC}, right=of ds1] {};
		\node (ds2) [twoportshape, t={$\downarrow k$}, right=of rrc] {};
		\coordinate[right=of ds2] (out);
		
		\draw (in) ++(-0.5,0) node[above] {$\Re\{\alpha(t)e^{-i\omega_ct}\}$};
		\draw (osc.north) node[above, yshift=2pt] {$\sin(\omega_l t)$};

		\draw[-Latex] (in) -- (split);
		\draw[-Latex] (osc) -- (phase split);
		\draw[-Latex] (adder) -- (lp);
		\draw[-Latex] (lp) -- (adc);
		\draw[-Latex] (adc) -- (dc);
		\draw[-Latex] (dc) -- (ds1);
		\draw[-Latex] (ds1) -- (rrc);
		\draw[-Latex] (rrc) -- (ds2);
		\draw[-Latex] (ds2) -- (out);
		
		\draw[-Latex] (split.north) -- (split.north|-mixerx.west) -- (mixerx);
		\draw[-Latex] (phase split) -- (mixerx);
		\draw[-Latex] (mixerx.east) -- (mixerx.east-|adder.north)  -- (adder);
				
		\draw[-Latex] (split.south) -- (split.south|-mixerp.west) -- (mixerp);
		\draw[-Latex] (phase split) -- (mixerp);
		\draw[-Latex] (mixerp.east) -- (mixerp.east-|adder.south) -- (adder);
	\end{tikzpicture}
\end{document}

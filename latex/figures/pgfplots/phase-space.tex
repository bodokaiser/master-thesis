\documentclass[tikz]{standalone}
% https://tex.stackexchange.com/questions/31708/draw-a-bivariate-normal-distribution-in-tikz

\usepackage{physics}
\usepackage{pgfplots}

\begin{document}
	\pgfplotsset{compat=1.17}
	\pgfplotsset{colormap={whitered}{color(0cm)=(white); color(1cm)=(orange!75!red)}}

	\begin{tikzpicture}[
    	declare function={mu1=1;},
	    declare function={mu2=2;},
	    declare function={sigma1=0.5;},
		declare function={sigma2=0.5;},
	    declare function={normal(\m,\s)=1/(2*\s*sqrt(pi))*exp(-(x-\m)^2/(2*\s^2));},
	    declare function={bivar(\ma,\sa,\mb,\sb)=1/(2*pi*\sa*\sb) * exp(-((x-\ma)^2/\sa^2 + (y-\mb)^2/\sb^2))/2;}]
		\begin{axis}[
		    colormap name=whitered,
		    width=15cm,
		    view={45}{65},
		    enlargelimits=false,
		    grid=major,
		    domain=-4:4,
		    y domain=-4:4,
		    samples=52,
		    xlabel=$x$,
		    ylabel=$p$,
		    zlabel={$P(x,p)$},
		    colorbar,
		    colorbar style={
	        	at={(1,0)},
    		    anchor=south west,
		        height=0.25*\pgfkeysvalueof{/pgfplots/parent axis height},
		        title={$P(x,p)$}
		    }
	]
			\addplot3[surf] {bivar(mu1,sigma1,mu2,sigma2)};
			\addplot3[domain=-4:4,samples=31, samples y=0, thick, smooth] (x,4,{normal(mu1,sigma1)});
			\addplot3[domain=-4:4,samples=31, samples y=0, thick, smooth] (-4,x,{normal(mu2,sigma2)});
			\draw [black!50] (axis cs:-4,0,0) -- (axis cs:4,0,0);
			\draw [black!50] (axis cs:0,-4,0) -- (axis cs:0,4,0);
			\node at (axis cs:-4,1,0.18) [pin=165:$P(x)$] {};
			\node at (axis cs:1.5,4,0.32) [pin=-15:$P(p)$] {};
		\end{axis}
	\end{tikzpicture}
\end{document}